\section{SCSI Host Device}\label{sec:Device Types / SCSI Host Device}

The virtio SCSI host device groups together one or more virtual
logical units (such as disks), and allows communicating to them
using the SCSI protocol. An instance of the device represents a
SCSI host to which many targets and LUNs are attached.

The virtio SCSI device services two kinds of requests:
\begin{itemize}
\item command requests for a logical unit;

\item task management functions related to a logical unit, target or
  command.
\end{itemize}

The device is also able to send out notifications about added and
removed logical units. Together, these capabilities provide a
SCSI transport protocol that uses virtqueues as the transfer
medium. In the transport protocol, the virtio driver acts as the
initiator, while the virtio SCSI host provides one or more
targets that receive and process the requests.

This section relies on definitions from \hyperref[intro:SAM]{SAM}.

\subsection{Device ID}\label{sec:Device Types / SCSI Host Device / Device ID}
  8

\subsection{Virtqueues}\label{sec:Device Types / SCSI Host Device / Virtqueues}

\begin{description}
\item[0] controlq
\item[1] eventq
\item[2\ldots n] request queues
\end{description}

\subsection{Feature bits}\label{sec:Device Types / SCSI Host Device / Feature bits}

\begin{description}
\item[VIRTIO_SCSI_F_INOUT (0)] A single request can include both
    device-readable and device-writable data buffers.

\item[VIRTIO_SCSI_F_HOTPLUG (1)] The host SHOULD enable reporting of
    hot-plug and hot-unplug events for LUNs and targets on the SCSI bus.
    The guest SHOULD handle hot-plug and hot-unplug events.

\item[VIRTIO_SCSI_F_CHANGE (2)] The host will report changes to LUN
    parameters via a VIRTIO_SCSI_T_PARAM_CHANGE event; the guest
    SHOULD handle them.

\item[VIRTIO_SCSI_F_T10_PI (3)] The extended fields for T10 protection
    information (DIF/DIX) are included in the SCSI request header.
\end{description}

\subsection{Device configuration layout}\label{sec:Device Types / SCSI Host Device / Device configuration layout}

  All fields of this configuration are always available.

\begin{lstlisting}
struct virtio_scsi_config {
        le32 num_queues;
        le32 seg_max;
        le32 max_sectors;
        le32 cmd_per_lun;
        le32 event_info_size;
        le32 sense_size;
        le32 cdb_size;
        le16 max_channel;
        le16 max_target;
        le32 max_lun;
};
\end{lstlisting}

\begin{description}
\item[\field{num_queues}] is the total number of request virtqueues exposed by
    the device. The driver MAY use only one request queue,
    or it can use more to achieve better performance.

\item[\field{seg_max}] is the maximum number of segments that can be in a
    command. A bidirectional command can include \field{seg_max} input
    segments and \field{seg_max} output segments.

\item[\field{max_sectors}] is a hint to the driver about the maximum transfer
    size to use.

\item[\field{cmd_per_lun}] tells the driver the maximum number of
    linked commands it can send to one LUN.

\item[\field{event_info_size}] is the maximum size that the device will fill
    for buffers that the driver places in the eventq. It is
    written by the device depending on the set of negotiated
    features.

\item[\field{sense_size}] is the maximum size of the sense data that the
    device will write. The default value is written by the device
    and MUST be 96, but the driver can modify it. It is
    restored to the default when the device is reset.

\item[\field{cdb_size}] is the maximum size of the CDB that the driver will
    write. The default value is written by the device and MUST
    be 32, but the driver can likewise modify it. It is
    restored to the default when the device is reset.

\item[\field{max_channel}, \field{max_target} and \field{max_lun}] can be
    used by the driver as hints to constrain scanning the logical units
    on the host to channel/target/logical unit numbers that are less than
    or equal to the value of the fields.  \field{max_channel} SHOULD
    be zero.  \field{max_target} SHOULD be less than or equal to 255.
    \field{max_lun} SHOULD be less than or equal to 16383.
\end{description}

\drivernormative{\subsubsection}{Device configuration layout}{Device Types / SCSI Host Device / Device configuration layout}

The driver MUST NOT write to device configuration fields other than
\field{sense_size} and \field{cdb_size}.

The driver MUST NOT send more than \field{cmd_per_lun} linked commands
to one LUN, and MUST NOT send more than the virtqueue size number of
linked commands to one LUN.

\devicenormative{\subsubsection}{Device configuration layout}{Device Types / SCSI Host Device / Device configuration layout}

On reset, the device MUST set \field{sense_size} to 96 and
\field{cdb_size} to 32.

\subsubsection{Legacy Interface: Device configuration layout}\label{sec:Device Types / SCSI Host Device / Device configuration layout / Legacy Interface: Device configuration layout}
When using the legacy interface, transitional devices and drivers
MUST format the fields in struct virtio_scsi_config
according to the native endian of the guest rather than
(necessarily when not using the legacy interface) little-endian.

\devicenormative{\subsection}{Device Initialization}{Device Types / SCSI Host Device / Device Initialization}

On initialization the driver SHOULD first discover the
device's virtqueues.

If the driver uses the eventq, the driver SHOULD place at least one
buffer in the eventq.

The driver MAY immediately issue requests\footnote{For example, INQUIRY
or REPORT LUNS.} or task management functions\footnote{For example, I_T
RESET.}.

\subsection{Device Operation}\label{sec:Device Types / SCSI Host Device / Device Operation}

Device operation consists of operating request queues, the control
queue and the event queue.

\paragraph{Legacy Interface: Device Operation}\label{sec:Device
Types / SCSI Host Device / Device Operation / Legacy
Interface: Device Operation}
When using the legacy interface, the driver SHOULD ignore the
used length values.
\begin{note}
Historically, devices put the total descriptor length,
or the total length of device-writable buffers there,
even when only part of the buffers were actually written.
\end{note}

\subsubsection{Device Operation: Request Queues}\label{sec:Device Types / SCSI Host Device / Device Operation / Device Operation: Request Queues}

The driver enqueues requests to an arbitrary request queue, and
they are used by the device on that same queue. It is the
responsibility of the driver to ensure strict request ordering
for commands placed on different queues, because they will be
consumed with no order constraints.

Requests have the following format:

\begin{lstlisting}
struct virtio_scsi_req_cmd {
        // Device-readable part
        u8 lun[8];
        le64 id;
        u8 task_attr;
        u8 prio;
        u8 crn;
        u8 cdb[cdb_size];
        // The next three fields are only present if VIRTIO_SCSI_F_T10_PI
        // is negotiated.
        le32 pi_bytesout;
        le32 pi_bytesin;
        u8 pi_out[pi_bytesout];
        u8 dataout[];

        // Device-writable part
        le32 sense_len;
        le32 residual;
        le16 status_qualifier;
        u8 status;
        u8 response;
        u8 sense[sense_size];
        // The next field is only present if VIRTIO_SCSI_F_T10_PI
        // is negotiated
        u8 pi_in[pi_bytesin];
        u8 datain[];
};


/* command-specific response values */
#define VIRTIO_SCSI_S_OK                0
#define VIRTIO_SCSI_S_OVERRUN           1
#define VIRTIO_SCSI_S_ABORTED           2
#define VIRTIO_SCSI_S_BAD_TARGET        3
#define VIRTIO_SCSI_S_RESET             4
#define VIRTIO_SCSI_S_BUSY              5
#define VIRTIO_SCSI_S_TRANSPORT_FAILURE 6
#define VIRTIO_SCSI_S_TARGET_FAILURE    7
#define VIRTIO_SCSI_S_NEXUS_FAILURE     8
#define VIRTIO_SCSI_S_FAILURE           9

/* task_attr */
#define VIRTIO_SCSI_S_SIMPLE            0
#define VIRTIO_SCSI_S_ORDERED           1
#define VIRTIO_SCSI_S_HEAD              2
#define VIRTIO_SCSI_S_ACA               3
\end{lstlisting}

\field{lun} addresses the REPORT LUNS well-known logical unit, or
a target and logical unit in the virtio-scsi device's SCSI domain.
When used to address the REPORT LUNS logical unit, \field{lun} is 0xC1,
0x01 and six zero bytes.  The virtio-scsi device SHOULD implement the
REPORT LUNS well-known logical unit.

When used to address a target and logical unit, the only supported format
for \field{lun} is: first byte set to 1, second byte set to target,
third and fourth byte representing a single level LUN structure, followed
by four zero bytes. With this representation, a virtio-scsi device can
serve up to 256 targets and 16384 LUNs per target.  The device MAY also
support having a well-known logical units in the third and fourth byte.

\field{id} is the command identifier (``tag'').

\field{task_attr} defines the task attribute as in the table above, but
all task attributes MAY be mapped to SIMPLE by the device.  Some commands
are defined by SCSI standards as "implicit head of queue"; for such
commands, all task attributes MAY also be mapped to HEAD OF QUEUE.
Drivers and applications SHOULD NOT send a command with the ORDERED
task attribute if the command has an implicit HEAD OF QUEUE attribute,
because whether the ORDERED task attribute is honored is vendor-specific.

\field{crn} may also be provided by clients, but is generally expected
to be 0. The maximum CRN value defined by the protocol is 255, since
CRN is stored in an 8-bit integer.

The CDB is included in \field{cdb} and its size, \field{cdb_size},
is taken from the configuration space.

All of these fields are defined in \hyperref[intro:SAM]{SAM} and are
always device-readable.

\field{pi_bytesout} determines the size of the \field{pi_out} field
in bytes.  If it is nonzero, the \field{pi_out} field contains outgoing
protection information for write operations.  \field{pi_bytesin} determines
the size of the \field{pi_in} field in the device-writable section, in bytes.
All three fields are only present if VIRTIO_SCSI_F_T10_PI has been negotiated.

The remainder of the device-readable part is the data output buffer,
\field{dataout}.

\field{sense} and subsequent fields are always device-writable. \field{sense_len}
indicates the number of bytes actually written to the sense
buffer.

\field{residual} indicates the residual size,
calculated as ``data_length - number_of_transferred_bytes'', for
read or write operations. For bidirectional commands, the
number_of_transferred_bytes includes both read and written bytes.
A \field{residual} that is less than the size of \field{datain} means that
\field{dataout} was processed entirely. A \field{residual} that
exceeds the size of \field{datain} means that \field{dataout} was
processed partially and \field{datain} was not processed at
all.

If the \field{pi_bytesin} is nonzero, the \field{pi_in} field contains
incoming protection information for read operations.  \field{pi_in} is
only present if VIRTIO_SCSI_F_T10_PI has been negotiated\footnote{There
  is no separate residual size for \field{pi_bytesout} and
  \field{pi_bytesin}.  It can be computed from the \field{residual} field,
  the size of the data integrity information per sector, and the sizes
  of \field{pi_out}, \field{pi_in}, \field{dataout} and \field{datain}.}.

The remainder of the device-writable part is the data input buffer,
\field{datain}.


\devicenormative{\paragraph}{Device Operation: Request Queues}{Device Types / SCSI Host Device / Device Operation / Device Operation: Request Queues}

The device MUST write the \field{status} byte as the status code as
defined in \hyperref[intro:SAM]{SAM}.

The device MUST write the \field{response} byte as one of the following:

\begin{description}

\item[VIRTIO_SCSI_S_OK] when the request was completed and the \field{status}
  byte is filled with a SCSI status code (not necessarily
  ``GOOD'').

\item[VIRTIO_SCSI_S_OVERRUN] if the content of the CDB (such as the
  allocation length, parameter length or transfer size) requires
  more data than is available in the datain and dataout buffers.

\item[VIRTIO_SCSI_S_ABORTED] if the request was cancelled due to an
  ABORT TASK or ABORT TASK SET task management function.

\item[VIRTIO_SCSI_S_BAD_TARGET] if the request was never processed
  because the target indicated by \field{lun} does not exist.

\item[VIRTIO_SCSI_S_RESET] if the request was cancelled due to a bus
  or device reset (including a task management function).

\item[VIRTIO_SCSI_S_TRANSPORT_FAILURE] if the request failed due to a
  problem in the connection between the host and the target
  (severed link).

\item[VIRTIO_SCSI_S_TARGET_FAILURE] if the target is suffering a
  failure and to tell the driver not to retry on other paths.

\item[VIRTIO_SCSI_S_NEXUS_FAILURE] if the nexus is suffering a failure
  but retrying on other paths might yield a different result.

\item[VIRTIO_SCSI_S_BUSY] if the request failed but retrying on the
  same path is likely to work.

\item[VIRTIO_SCSI_S_FAILURE] for other host or driver error. In
  particular, if neither \field{dataout} nor \field{datain} is empty, and the
  VIRTIO_SCSI_F_INOUT feature has not been negotiated, the
  request will be immediately returned with a response equal to
  VIRTIO_SCSI_S_FAILURE.
\end{description}

All commands must be completed before the virtio-scsi device is
reset or unplugged.  The device MAY choose to abort them, or if
it does not do so MUST pick the VIRTIO_SCSI_S_FAILURE response.

\drivernormative{\paragraph}{Device Operation: Request Queues}{Device Types / SCSI Host Device / Device Operation / Device Operation: Request Queues}

\field{task_attr}, \field{prio} and \field{crn} SHOULD be zero.

Upon receiving a VIRTIO_SCSI_S_TARGET_FAILURE response, the driver
SHOULD NOT retry the request on other paths.

\paragraph{Legacy Interface: Device Operation: Request Queues}\label{sec:Device Types / SCSI Host Device / Device Operation / Device Operation: Request Queues / Legacy Interface: Device Operation: Request Queues}
When using the legacy interface, transitional devices and drivers
MUST format the fields in struct virtio_scsi_req_cmd
according to the native endian of the guest rather than
(necessarily when not using the legacy interface) little-endian.

\subsubsection{Device Operation: controlq}\label{sec:Device Types / SCSI Host Device / Device Operation / Device Operation: controlq}

The controlq is used for other SCSI transport operations.
Requests have the following format:

{
\lstset{escapechar=\$}
\begin{lstlisting}
struct virtio_scsi_ctrl {
        le32 type;
$\ldots$
        u8 response;
};

/* response values valid for all commands */
#define VIRTIO_SCSI_S_OK                       0
#define VIRTIO_SCSI_S_BAD_TARGET               3
#define VIRTIO_SCSI_S_BUSY                     5
#define VIRTIO_SCSI_S_TRANSPORT_FAILURE        6
#define VIRTIO_SCSI_S_TARGET_FAILURE           7
#define VIRTIO_SCSI_S_NEXUS_FAILURE            8
#define VIRTIO_SCSI_S_FAILURE                  9
#define VIRTIO_SCSI_S_INCORRECT_LUN            12
\end{lstlisting}
}

The \field{type} identifies the remaining fields.

The following commands are defined:

\begin{itemize}
\item Task management function.
\begin{lstlisting}
#define VIRTIO_SCSI_T_TMF                      0

#define VIRTIO_SCSI_T_TMF_ABORT_TASK           0
#define VIRTIO_SCSI_T_TMF_ABORT_TASK_SET       1
#define VIRTIO_SCSI_T_TMF_CLEAR_ACA            2
#define VIRTIO_SCSI_T_TMF_CLEAR_TASK_SET       3
#define VIRTIO_SCSI_T_TMF_I_T_NEXUS_RESET      4
#define VIRTIO_SCSI_T_TMF_LOGICAL_UNIT_RESET   5
#define VIRTIO_SCSI_T_TMF_QUERY_TASK           6
#define VIRTIO_SCSI_T_TMF_QUERY_TASK_SET       7

struct virtio_scsi_ctrl_tmf {
        // Device-readable part
        le32 type;
        le32 subtype;
        u8   lun[8];
        le64 id;
        // Device-writable part
        u8   response;
};

/* command-specific response values */
#define VIRTIO_SCSI_S_FUNCTION_COMPLETE        0
#define VIRTIO_SCSI_S_FUNCTION_SUCCEEDED       10
#define VIRTIO_SCSI_S_FUNCTION_REJECTED        11
\end{lstlisting}

  The \field{type} is VIRTIO_SCSI_T_TMF; \field{subtype} defines which
  task management function. All
  fields except \field{response} are filled by the driver.

  Other fields which are irrelevant for the requested TMF
  are ignored but they are still present. \field{lun}
  is in the same format specified for request queues; the
  single level LUN is ignored when the task management function
  addresses a whole I_T nexus. When relevant, the value of \field{id}
  is matched against the id values passed on the requestq.

  The outcome of the task management function is written by the
  device in \field{response}. The command-specific response
  values map 1-to-1 with those defined in \hyperref[intro:SAM]{SAM}.

  Task management function can affect the response value for commands that
  are in the request queue and have not been completed yet.  For example,
  the device MUST complete all active commands on a logical unit
  or target (possibly with a VIRTIO_SCSI_S_RESET response code)
  upon receiving a "logical unit reset" or "I_T nexus reset" TMF.
  Similarly, the device MUST complete the selected commands (possibly
  with a VIRTIO_SCSI_S_ABORTED response code) upon receiving an "abort
  task" or "abort task set" TMF.  Such effects MUST take place before
  the TMF itself is successfully completed, and the device MUST use
  memory barriers appropriately in order to ensure that the driver sees
  these writes in the correct order.

\item Asynchronous notification query.
\begin{lstlisting}
#define VIRTIO_SCSI_T_AN_QUERY                    1

struct virtio_scsi_ctrl_an {
    // Device-readable part
    le32 type;
    u8   lun[8];
    le32 event_requested;
    // Device-writable part
    le32 event_actual;
    u8   response;
};

#define VIRTIO_SCSI_EVT_ASYNC_OPERATIONAL_CHANGE  2
#define VIRTIO_SCSI_EVT_ASYNC_POWER_MGMT          4
#define VIRTIO_SCSI_EVT_ASYNC_EXTERNAL_REQUEST    8
#define VIRTIO_SCSI_EVT_ASYNC_MEDIA_CHANGE        16
#define VIRTIO_SCSI_EVT_ASYNC_MULTI_HOST          32
#define VIRTIO_SCSI_EVT_ASYNC_DEVICE_BUSY         64
\end{lstlisting}

  By sending this command, the driver asks the device which
  events the given LUN can report, as described in paragraphs 6.6
  and A.6 of \hyperref[intro:SCSI MMC]{SCSI MMC}. The driver writes the
  events it is interested in into \field{event_requested}; the device
  responds by writing the events that it supports into
  \field{event_actual}.

  The \field{type} is VIRTIO_SCSI_T_AN_QUERY. \field{lun} and \field{event_requested}
  are written by the driver. \field{event_actual} and \field{response}
  fields are written by the device.

  No command-specific values are defined for the \field{response} byte.

\item Asynchronous notification subscription.
\begin{lstlisting}
#define VIRTIO_SCSI_T_AN_SUBSCRIBE                2

struct virtio_scsi_ctrl_an {
        // Device-readable part
        le32 type;
        u8   lun[8];
        le32 event_requested;
        // Device-writable part
        le32 event_actual;
        u8   response;
};
\end{lstlisting}

  By sending this command, the driver asks the specified LUN to
  report events for its physical interface, again as described in
   \hyperref[intro:SCSI MMC]{SCSI MMC}. The driver writes the events it is
  interested in into \field{event_requested}; the device responds by
  writing the events that it supports into \field{event_actual}.

  Event types are the same as for the asynchronous notification
  query message.

  The \field{type} is VIRTIO_SCSI_T_AN_SUBSCRIBE. \field{lun} and
  \field{event_requested} are written by the driver.
  \field{event_actual} and \field{response} are written by the device.

  No command-specific values are defined for the response byte.
\end{itemize}

\paragraph{Legacy Interface: Device Operation: controlq}\label{sec:Device Types / SCSI Host Device / Device Operation / Device Operation: controlq / Legacy Interface: Device Operation: controlq}

When using the legacy interface, transitional devices and drivers
MUST format the fields in struct virtio_scsi_ctrl, struct
virtio_scsi_ctrl_tmf, struct virtio_scsi_ctrl_an and struct
virtio_scsi_ctrl_an
according to the native endian of the guest rather than
(necessarily when not using the legacy interface) little-endian.


\subsubsection{Device Operation: eventq}\label{sec:Device Types / SCSI Host Device / Device Operation / Device Operation: eventq}

The eventq is populated by the driver for the device to report information on logical
units that are attached to it. In general, the device will not
queue events to cope with an empty eventq, and will end up
dropping events if it finds no buffer ready. However, when
reporting events for many LUNs (e.g. when a whole target
disappears), the device can throttle events to avoid dropping
them. For this reason, placing 10-15 buffers on the event queue
is sufficient.

Buffers returned by the device on the eventq will be referred to
as ``events'' in the rest of this section. Events have the
following format:

\begin{lstlisting}
#define VIRTIO_SCSI_T_EVENTS_MISSED   0x80000000

struct virtio_scsi_event {
        // Device-writable part
        le32 event;
        u8  lun[8];
        le32 reason;
};
\end{lstlisting}

The devices sets bit 31 in \field{event} to report lost events
due to missing buffers.

The meaning of \field{reason} depends on the
contents of \field{event}. The following events are defined:

\begin{itemize}
\item No event.
\begin{lstlisting}
#define VIRTIO_SCSI_T_NO_EVENT         0
\end{lstlisting}

  This event is fired in the following cases:

\begin{itemize}
\item When the device detects in the eventq a buffer that is
    shorter than what is indicated in the configuration field, it
    MAY use it immediately and put this dummy value in \field{event}.
    A well-written driver will never observe this
    situation.

\item When events are dropped, the device MAY signal this event as
    soon as the drivers makes a buffer available, in order to
    request action from the driver. In this case, of course, this
    event will be reported with the VIRTIO_SCSI_T_EVENTS_MISSED
    flag.
\end{itemize}

\item Transport reset
\begin{lstlisting}
#define VIRTIO_SCSI_T_TRANSPORT_RESET  1

#define VIRTIO_SCSI_EVT_RESET_HARD         0
#define VIRTIO_SCSI_EVT_RESET_RESCAN       1
#define VIRTIO_SCSI_EVT_RESET_REMOVED      2
\end{lstlisting}

  By sending this event, the device signals that a logical unit
  on a target has been reset, including the case of a new device
  appearing or disappearing on the bus. The device fills in all
  fields. \field{event} is set to
  VIRTIO_SCSI_T_TRANSPORT_RESET. \field{lun} addresses a
  logical unit in the SCSI host.

  The \field{reason} value is one of the three \#define values appearing
  above:

  \begin{description}
  \item[VIRTIO_SCSI_EVT_RESET_REMOVED] (``LUN/target removed'') is used
    if the target or logical unit is no longer able to receive
    commands.

  \item[VIRTIO_SCSI_EVT_RESET_HARD] (``LUN hard reset'') is used if the
    logical unit has been reset, but is still present.

  \item[VIRTIO_SCSI_EVT_RESET_RESCAN] (``rescan LUN/target'') is used if
    a target or logical unit has just appeared on the device.
  \end{description}

  The ``removed'' and ``rescan'' events can happen when
  VIRTIO_SCSI_F_HOTPLUG feature was negotiated; when sent for LUN 0,
  they MAY apply to the entire target so the driver can ask the
  initiator to rescan the target to detect this.

  Events will also be reported via sense codes (this obviously
  does not apply to newly appeared buses or targets, since the
  application has never discovered them):

  \begin{itemize}
  \item ``LUN/target removed'' maps to sense key ILLEGAL REQUEST, asc
    0x25, ascq 0x00 (LOGICAL UNIT NOT SUPPORTED)

  \item ``LUN hard reset'' maps to sense key UNIT ATTENTION, asc 0x29
    (POWER ON, RESET OR BUS DEVICE RESET OCCURRED)

  \item ``rescan LUN/target'' maps to sense key UNIT ATTENTION, asc
    0x3f, ascq 0x0e (REPORTED LUNS DATA HAS CHANGED)
  \end{itemize}

  The preferred way to detect transport reset is always to use
  events, because sense codes are only seen by the driver when it
  sends a SCSI command to the logical unit or target. However, in
  case events are dropped, the initiator will still be able to
  synchronize with the actual state of the controller if the
  driver asks the initiator to rescan of the SCSI bus. During the
  rescan, the initiator will be able to observe the above sense
  codes, and it will process them as if it the driver had
  received the equivalent event.

  \item Asynchronous notification
\begin{lstlisting}
#define VIRTIO_SCSI_T_ASYNC_NOTIFY     2
\end{lstlisting}

  By sending this event, the device signals that an asynchronous
  event was fired from a physical interface.

  All fields are written by the device. \field{event} is set to
  VIRTIO_SCSI_T_ASYNC_NOTIFY. \field{lun} addresses a logical
  unit in the SCSI host. \field{reason} is a subset of the
  events that the driver has subscribed to via the ``Asynchronous
  notification subscription'' command.

  \item LUN parameter change
\begin{lstlisting}
#define VIRTIO_SCSI_T_PARAM_CHANGE  3
\end{lstlisting}

  By sending this event, the device signals a change in the configuration parameters
  of a logical unit, for example the capacity or cache mode.
  \field{event} is set to VIRTIO_SCSI_T_PARAM_CHANGE.
  \field{lun} addresses a logical unit in the SCSI host.

  The same event SHOULD also be reported as a unit attention condition.
  \field{reason} contains the additional sense code and additional sense code qualifier,
  respectively in bits 0\ldots 7 and 8\ldots 15.
  \begin{note}
  For example, a change in capacity will be reported as asc 0x2a, ascq 0x09
  (CAPACITY DATA HAS CHANGED).
  \end{note}

  For MMC devices (inquiry type 5) there would be some overlap between this
  event and the asynchronous notification event, so for simplicity the host never
  reports this event for MMC devices.
\end{itemize}

\drivernormative{\paragraph}{Device Operation: eventq}{Device Types / SCSI Host Device / Device Operation / Device Operation: eventq}

The driver SHOULD keep the eventq populated with buffers.  These
buffers MUST be device-writable, and SHOULD be at least
\field{event_info_size} bytes long, and MUST be at least the size of
struct virtio_scsi_event.

If \field{event} has bit 31 set, the driver SHOULD
poll the logical units for unit attention conditions, and/or do
whatever form of bus scan is appropriate for the guest operating
system and SHOULD poll for asynchronous events manually using SCSI commands.

When receiving a VIRTIO_SCSI_T_TRANSPORT_RESET message with
\field{reason} set to VIRTIO_SCSI_EVT_RESET_REMOVED or
VIRTIO_SCSI_EVT_RESET_RESCAN for LUN 0, the driver SHOULD ask the
initiator to rescan the target, in order to detect the case when an
entire target has appeared or disappeared.

\devicenormative{\paragraph}{Device Operation: eventq}{Device Types / SCSI Host Device / Device Operation / Device Operation: eventq}

The device MUST set bit 31 in \field{event} if events were lost due to
missing buffers, and it MAY use a VIRTIO_SCSI_T_NO_EVENT event to report
this.

The device MUST NOT send VIRTIO_SCSI_T_TRANSPORT_RESET messages
with \field{reason} set to VIRTIO_SCSI_EVT_RESET_REMOVED or
VIRTIO_SCSI_EVT_RESET_RESCAN unless VIRTIO_SCSI_F_HOTPLUG was negotiated.

The device MUST NOT report VIRTIO_SCSI_T_PARAM_CHANGE for MMC devices.

\paragraph{Legacy Interface: Device Operation: eventq}\label{sec:Device Types / SCSI Host Device / Device Operation / Device Operation: eventq / Legacy Interface: Device Operation: eventq}
When using the legacy interface, transitional devices and drivers
MUST format the fields in struct virtio_scsi_event
according to the native endian of the guest rather than
(necessarily when not using the legacy interface) little-endian.

\subsubsection{Legacy Interface: Framing Requirements}\label{sec:Device
Types / SCSI Host Device / Legacy Interface: Framing Requirements}

When using legacy interfaces, transitional drivers which have not
negotiated VIRTIO_F_ANY_LAYOUT MUST use a single descriptor for the
\field{lun}, \field{id}, \field{task_attr}, \field{prio},
\field{crn} and \field{cdb} fields, and MUST only use a single
descriptor for the \field{sense_len}, \field{residual},
\field{status_qualifier}, \field{status}, \field{response} and
\field{sense} fields.

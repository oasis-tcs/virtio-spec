\chapter{Basic Facilities of a Virtio Device}\label{sec:Basic Facilities of a Virtio Device}

A virtio device is discovered and identified by a bus-specific method
(see the bus specific sections: \ref{sec:Virtio Transport Options / Virtio Over PCI Bus}~\nameref{sec:Virtio Transport Options / Virtio Over PCI Bus},
\ref{sec:Virtio Transport Options / Virtio Over MMIO}~\nameref{sec:Virtio Transport Options / Virtio Over MMIO} and \ref{sec:Virtio Transport Options / Virtio Over Channel I/O}~\nameref{sec:Virtio Transport Options / Virtio Over Channel I/O}).  Each
device consists of the following parts:

\begin{itemize}
\item Device status field
\item Feature bits
\item Notifications
\item Device Configuration space
\item One or more virtqueues
\end{itemize}

\section{\field{Device Status} Field}\label{sec:Basic Facilities of a Virtio Device / Device Status Field}
During device initialization by a driver,
the driver follows the sequence of steps specified in
\ref{sec:General Initialization And Device Operation / Device
Initialization}.

The \field{device status} field provides a simple low-level
indication of the completed steps of this sequence.
It's most useful to imagine it hooked up to traffic
lights on the console indicating the status of each device.  The
following bits are defined (listed below in the order in which
they would be typically set):
\begin{description}
\item[ACKNOWLEDGE (1)] Indicates that the guest OS has found the
  device and recognized it as a valid virtio device.

\item[DRIVER (2)] Indicates that the guest OS knows how to drive the
  device.
  \begin{note}
    There could be a significant (or infinite) delay before setting
    this bit.  For example, under Linux, drivers can be loadable modules.
  \end{note}

\item[FAILED (128)] Indicates that something went wrong in the guest,
  and it has given up on the device. This could be an internal
  error, or the driver didn't like the device for some reason, or
  even a fatal error during device operation.

\item[FEATURES_OK (8)] Indicates that the driver has acknowledged all the
  features it understands, and feature negotiation is complete.

\item[DRIVER_OK (4)] Indicates that the driver is set up and ready to
  drive the device.

\item[DEVICE_NEEDS_RESET (64)] Indicates that the device has experienced
  an error from which it can't recover.
\end{description}

The \field{device status} field starts out as 0, and is reinitialized to 0 by
the device during reset.

\drivernormative{\subsection}{Device Status Field}{Basic Facilities of a Virtio Device / Device Status Field}
The driver MUST update \field{device status},
setting bits to indicate the completed steps of the driver
initialization sequence specified in
\ref{sec:General Initialization And Device Operation / Device
Initialization}.
The driver MUST NOT clear a
\field{device status} bit.  If the driver sets the FAILED bit,
the driver MUST later reset the device before attempting to re-initialize.

The driver SHOULD NOT rely on completion of operations of a
device if DEVICE_NEEDS_RESET is set.
\begin{note}
For example, the driver can't assume requests in flight will be
completed if DEVICE_NEEDS_RESET is set, nor can it assume that
they have not been completed.  A good implementation will try to
recover by issuing a reset.
\end{note}

\devicenormative{\subsection}{Device Status Field}{Basic Facilities of a Virtio Device / Device Status Field}

The device MUST NOT consume buffers or send any used buffer
notifications to the driver before DRIVER_OK.

\label{sec:Basic Facilities of a Virtio Device / Device Status Field / DEVICENEEDSRESET}The device SHOULD set DEVICE_NEEDS_RESET when it enters an error state
that a reset is needed.  If DRIVER_OK is set, after it sets DEVICE_NEEDS_RESET, the device
MUST send a device configuration change notification to the driver.

\section{Feature Bits}\label{sec:Basic Facilities of a Virtio Device / Feature Bits}

Each virtio device offers all the features it understands.  During
device initialization, the driver reads this and tells the device the
subset that it accepts.  The only way to renegotiate is to reset
the device.

This allows for forwards and backwards compatibility: if the device is
enhanced with a new feature bit, older drivers will not write that
feature bit back to the device.  Similarly, if a driver is enhanced with a feature
that the device doesn't support, it see the new feature is not offered.

Feature bits are allocated as follows:

\begin{description}
\item[0 to 23, and 50 to 127] Feature bits for the specific device type

\item[24 to 41] Feature bits reserved for extensions to the queue and
  feature negotiation mechanisms, see \ref{sec:Reserved Feature Bits}

\item[42 to 49, and 128 and above] Feature bits reserved for future extensions.
\end{description}

\begin{note}
For example, feature bit 0 for a network device (i.e.
Device ID 1) indicates that the device supports checksumming of
packets.
\end{note}

In particular, new fields in the device configuration space are
indicated by offering a new feature bit.

To keep the feature negotiation mechanism extensible, it is
important that devices \em{do not} offer any feature bits that
they would not be able to handle if the driver accepted them
(even though drivers are not supposed to accept any unspecified,
reserved, or unsupported features even if offered, according to
the specification.) Likewise, it is important that drivers \em{do
not} accept feature bits they do not know how to handle (even
though devices are not supposed to offer any unspecified,
reserved, or unsupported features in the first place,
according to the specification.) The preferred
way for handling reserved and unexpected features is that the
driver ignores them.

In particular, this is
especially important for features limited to specific transports,
as enabling these for more transports in future versions of the
specification is highly likely to require changing the behaviour
from drivers and devices.  Drivers and devices supporting
multiple transports need to carefully maintain per-transport
lists of allowed features.

\drivernormative{\subsection}{Feature Bits}{Basic Facilities of a Virtio Device / Feature Bits}
The driver MUST NOT accept a feature which the device did not offer,
and MUST NOT accept a feature which requires another feature which was
not accepted.

The driver MUST validate the feature bits offered by the device.
The driver MUST ignore and MUST NOT accept any feature bit that is
\begin{itemize}
\item not described in this specification,
\item marked as reserved,
\item not supported for the specific transport,
\item not defined for the device type.
\end{itemize}

The driver SHOULD go into backwards compatibility mode
if the device does not offer a feature it understands, otherwise MUST
set the FAILED \field{device status} bit and cease initialization.

By contrast, the driver MUST NOT fail solely because a feature
it does not understand has been offered by the device.

\devicenormative{\subsection}{Feature Bits}{Basic Facilities of a Virtio Device / Feature Bits}
The device MUST NOT offer a feature which requires another feature
which was not offered.  The device SHOULD accept any valid subset
of features the driver accepts, otherwise it MUST fail to set the
FEATURES_OK \field{device status} bit when the driver writes it.

The device MUST NOT offer feature bits corresponding to features
it would not support if accepted by the driver (even if the
driver is prohibited from accepting the feature bits by the
specification); for the sake of clarity, this refers to feature
bits not described in this specification, reserved feature bits
and feature bits reserved or not supported for the specific
transport or the specific device type, but this does not preclude
devices written to a future version of this specification from
offering such feature bits should such a specification have a
provision for devices to support the corresponding features.

If a device has successfully negotiated a set of features
at least once (by accepting the FEATURES_OK \field{device
status} bit during device initialization), then it SHOULD
NOT fail re-negotiation of the same set of features after
a device or system reset.  Failure to do so would interfere
with resuming from suspend and error recovery.

\subsection{Legacy Interface: A Note on Feature
Bits}\label{sec:Basic Facilities of a Virtio Device / Feature
Bits / Legacy Interface: A Note on Feature Bits}

Transitional Drivers MUST detect Legacy Devices by detecting that
the feature bit VIRTIO_F_VERSION_1 is not offered.
Transitional devices MUST detect Legacy drivers by detecting that
VIRTIO_F_VERSION_1 has not been acknowledged by the driver.

In this case device is used through the legacy interface.

Legacy interface support is OPTIONAL.
Thus, both transitional and non-transitional devices and
drivers are compliant with this specification.

Requirements pertaining to transitional devices and drivers
is contained in sections named 'Legacy Interface' like this one.

When device is used through the legacy interface, transitional
devices and transitional drivers MUST operate according to the
requirements documented within these legacy interface sections.
Specification text within these sections generally does not apply
to non-transitional devices.

\section{Notifications}\label{sec:Basic Facilities of a Virtio Device
/ Notifications}

The notion of sending a notification (driver to device or device
to driver) plays an important role in this specification. The
modus operandi of the notifications is transport specific.

There are three types of notifications: 
\begin{itemize}
\item configuration change notification
\item available buffer notification
\item used buffer notification. 
\end{itemize}

Configuration change notifications and used buffer notifications are sent
by the device, the recipient is the driver. A configuration change
notification indicates that the device configuration space has changed; a
used buffer notification indicates that a buffer may have been made used
on the virtqueue designated by the notification.

Available buffer notifications are sent by the driver, the recipient is
the device. This type of notification indicates that a buffer may have
been made available on the virtqueue designated by the notification.

The semantics, the transport-specific implementations, and other
important aspects of the different notifications are specified in detail
in the following chapters.

Most transports implement notifications sent by the device to the
driver using interrupts. Therefore, in previous versions of this
specification, these notifications were often called interrupts.
Some names defined in this specification still retain this interrupt
terminology. Occasionally, the term event is used to refer to
a notification or a receipt of a notification.

\section{Device Reset}\label{sec:Basic Facilities of a Virtio Device / Device Reset}

The driver may want to initiate a device reset at various times; notably,
it is required to do so during device initialization and device cleanup.

The mechanism used by the driver to initiate the reset is transport specific.

\devicenormative{\subsection}{Device Reset}{Basic Facilities of a Virtio Device / Device Reset}

A device MUST reinitialize \field{device status} to 0 after receiving a reset.

A device MUST NOT send notifications or interact with the queues after
indicating completion of the reset by reinitializing \field{device status}
to 0, until the driver re-initializes the device.

\drivernormative{\subsection}{Device Reset}{Basic Facilities of a Virtio Device / Device Reset}

The driver SHOULD consider a driver-initiated reset complete when it
reads \field{device status} as 0.

\section{Device Configuration Space}\label{sec:Basic Facilities of a Virtio Device / Device Configuration Space}

Device configuration space is generally used for rarely-changing or
initialization-time parameters.  Where configuration fields are
optional, their existence is indicated by feature bits: Future
versions of this specification will likely extend the device
configuration space by adding extra fields at the tail.

\begin{note}
The device configuration space uses the little-endian format
for multi-byte fields.
\end{note}

Each transport also provides a generation count for the device configuration
space, which will change whenever there is a possibility that two
accesses to the device configuration space can see different versions of that
space.

\drivernormative{\subsection}{Device Configuration Space}{Basic Facilities of a Virtio Device / Device Configuration Space}
Drivers MUST NOT assume reads from
fields greater than 32 bits wide are atomic, nor are reads from
multiple fields: drivers SHOULD read device configuration space fields like so:

\begin{lstlisting}
u32 before, after;
do {
        before = get_config_generation(device);
        // read config entry/entries.
        after = get_config_generation(device);
} while (after != before);
\end{lstlisting}

For optional configuration space fields, the driver MUST check that the
corresponding feature is offered before accessing that part of the configuration
space.
\begin{note}
See section \ref{sec:General Initialization And Device Operation / Device Initialization} for details on feature negotiation.
\end{note}

Drivers MUST
NOT limit structure size and device configuration space size.  Instead,
drivers SHOULD only check that device configuration space is {\em large enough} to
contain the fields necessary for device operation.

\begin{note}
For example, if the specification states that device configuration
space 'includes a single 8-bit field' drivers should understand this to mean that
the device configuration space might also include an arbitrary amount of
tail padding, and accept any device configuration space size equal to or
greater than the specified 8-bit size.
\end{note}

\devicenormative{\subsection}{Device Configuration Space}{Basic Facilities of a Virtio Device / Device Configuration Space}
The device MUST allow reading of any device-specific configuration
field before FEATURES_OK is set by the driver.  This includes fields which are
conditional on feature bits, as long as those feature bits are offered
by the device.

\subsection{Legacy Interface: A Note on Device Configuration Space endian-ness}\label{sec:Basic Facilities of a Virtio Device / Device Configuration Space / Legacy Interface: A Note on Configuration Space endian-ness}

Note that for legacy interfaces, device configuration space is generally the
guest's native endian, rather than PCI's little-endian.
The correct endian-ness is documented for each device.

\subsection{Legacy Interface: Device Configuration Space}\label{sec:Basic Facilities of a Virtio Device / Device Configuration Space / Legacy Interface: Device Configuration Space}

Legacy devices did not have a configuration generation field, thus are
susceptible to race conditions if configuration is updated.  This
affects the block \field{capacity} (see \ref{sec:Device Types /
Block Device / Device configuration layout}) and
network \field{mac} (see \ref{sec:Device Types / Network Device /
Device configuration layout}) fields;
when using the legacy interface, drivers SHOULD
read these fields multiple times until two reads generate a consistent
result.

\section{Virtqueues}\label{sec:Basic Facilities of a Virtio Device / Virtqueues}

The mechanism for bulk data transport on virtio devices is
pretentiously called a virtqueue. Each device can have zero or more
virtqueues\footnote{For example, the simplest network device has one virtqueue for
transmit and one for receive.}.

A virtio device can have maximum of 65536 virtqueues. Each virtqueue is
identified by a virtqueue index. A virtqueue index has a value in the
range of 0 to 65535.

Driver makes requests available to device by adding
an available buffer to the queue, i.e., adding a buffer
describing the request to a virtqueue, and optionally triggering
a driver event, i.e., sending an available buffer notification
to the device.

Device executes the requests and - when complete - adds
a used buffer to the queue, i.e., lets the driver
know by marking the buffer as used. Device can then trigger
a device event, i.e., send a used buffer notification to the driver.

Device reports the number of bytes it has written to memory for
each buffer it uses. This is referred to as ``used length''.

Device is not generally required to use buffers in
the same order in which they have been made available
by the driver.

Some devices always use descriptors in the same order in which
they have been made available. These devices can offer the
VIRTIO_F_IN_ORDER feature. If negotiated, this knowledge
might allow optimizations or simplify driver and/or device code.

Each virtqueue can consist of up to 3 parts:
\begin{itemize}
\item Descriptor Area - used for describing buffers
\item Driver Area - extra data supplied by driver to the device
\item Device Area - extra data supplied by device to driver
\end{itemize}

\begin{note}
Note that previous versions of this spec used different names for
these parts (following \ref{sec:Basic Facilities of a Virtio Device / Split Virtqueues}):
\begin{itemize}
\item Descriptor Table - for the Descriptor Area
\item Available Ring - for the Driver Area
\item Used Ring - for the Device Area
\end{itemize}

\end{note}

Two formats are supported: Split Virtqueues (see \ref{sec:Basic
Facilities of a Virtio Device / Split
Virtqueues}~\nameref{sec:Basic Facilities of a Virtio Device /
Split Virtqueues}) and Packed Virtqueues (see \ref{sec:Basic
Facilities of a Virtio Device / Packed
Virtqueues}~\nameref{sec:Basic Facilities of a Virtio Device /
Packed Virtqueues}).

Every driver and device supports either the Packed or the Split
Virtqueue format, or both.

\subsection{Virtqueue Reset}\label{sec:Basic Facilities of a Virtio Device / Virtqueues / Virtqueue Reset}

When VIRTIO_F_RING_RESET is negotiated, the driver can reset a virtqueue
individually. The way to reset the virtqueue is transport specific.

Virtqueue reset is divided into two parts. The driver first resets a queue and
can afterwards optionally re-enable it.

\subsubsection{Virtqueue Reset}\label{sec:Basic Facilities of a Virtio Device / Virtqueues / Virtqueue Reset / Virtqueue Reset}

\devicenormative{\paragraph}{Virtqueue Reset}{Basic Facilities of a Virtio Device / Virtqueues / Virtqueue Reset / Virtqueue Reset}

After a queue has been reset by the driver, the device MUST NOT execute
any requests from that virtqueue, or notify the driver for it.

The device MUST reset any state of a virtqueue to the default state,
including the available state and the used state.

\drivernormative{\paragraph}{Virtqueue Reset}{Basic Facilities of a Virtio Device / Virtqueues / Virtqueue Reset / Virtqueue Reset}

After the driver tells the device to reset a queue, the driver MUST verify that
the queue has actually been reset.

After the queue has been successfully reset, the driver MAY release any
resource associated with that virtqueue.

\subsubsection{Virtqueue Re-enable}\label{sec:Basic Facilities of a Virtio Device / Virtqueues / Virtqueue Reset / Virtqueue Re-enable}

This process is the same as the initialization process of a single queue during
the initialization of the entire device.

\devicenormative{\paragraph}{Virtqueue Re-enable}{Basic Facilities of a Virtio Device / Virtqueues / Virtqueue Reset / Virtqueue Re-enable}

The device MUST observe any queue configuration that may have been
changed by the driver, like the maximum queue size.

\drivernormative{\paragraph}{Virtqueue Re-enable}{Basic Facilities of a Virtio Device / Virtqueues / Virtqueue Reset / Virtqueue Re-enable}

When re-enabling a queue, the driver MUST configure the queue resources
as during initial virtqueue discovery, but optionally with different
parameters.

\input{split-ring.tex}

\input{packed-ring.tex}

\section{Driver Notifications} \label{sec:Basic Facilities of a Virtio Device / Driver notifications}
The driver is sometimes required to send an available buffer
notification to the device.

When VIRTIO_F_NOTIFICATION_DATA has not been negotiated,
this notification contains either a virtqueue index if
VIRTIO_F_NOTIF_CONFIG_DATA is not negotiated or device supplied virtqueue
notification config data if VIRTIO_F_NOTIF_CONFIG_DATA is negotiated.

The notification method and supplying any such virtqueue notification config data
is transport specific.

However, some devices benefit from the ability to find out the
amount of available data in the queue without accessing the virtqueue in memory:
for efficiency or as a debugging aid.

To help with these optimizations, when VIRTIO_F_NOTIFICATION_DATA
has been negotiated, driver notifications to the device include
the following information:

\begin{description}
\item [vq_index or vq_notif_config_data] Either virtqueue index or device
      supplied queue notification config data corresponding to a virtqueue.
\item [next_off] Offset
      within the ring where the next available ring entry
      will be written.
      When VIRTIO_F_RING_PACKED has not been negotiated this refers to the
      15 least significant bits of the available index.
      When VIRTIO_F_RING_PACKED has been negotiated this refers to the offset
      (in units of descriptor entries)
      within the descriptor ring where the next available
      descriptor will be written.
\item [next_wrap] Wrap Counter.
      With VIRTIO_F_RING_PACKED this is the wrap counter
      referring to the next available descriptor.
      Without VIRTIO_F_RING_PACKED this is the most significant bit
      (bit 15) of the available index.
\end{description}

Note that the driver can send multiple notifications even without
making any more buffers available. When VIRTIO_F_NOTIFICATION_DATA
has been negotiated, these notifications would then have
identical \field{next_off} and \field{next_wrap} values.

\input{shared-mem.tex}

\section{Exporting Objects}\label{sec:Basic Facilities of a Virtio Device / Exporting Objects}

When an object created by one virtio device needs to be
shared with a separate virtio device, the first device can
export the object by generating a UUID which can then
be passed to the second device to identify the object.

What constitutes an object, how to export objects, and
how to import objects are defined by the individual device
types. It is RECOMMENDED that devices generate version 4
UUIDs as specified by \hyperref[intro:rfc4122]{[RFC4122]}.

\input{admin.tex}
\section{Device parts}\label{sec:Basic Facilities of a Virtio Device / Device parts}

Device parts represent the device state, with parts for basic
device facilities such as driver features, as well as transport specific
and device type specific parts. In memory, each device part consists
of a header \field{struct virtio_dev_part_hdr} followed by
the device part data in \field{value}. The driver can get and set
these device parts using administration commands.

\begin{lstlisting}
struct virtio_dev_part_hdr {
        le16 part_type;
        u8 flags;
        u8 reserved;
        union {
                struct {
                        le32 offset;
                        le32 reserved;
                } pci_common_cfg;
                struct  {
                        le16 index;
                        u8 reserved[6];
                } vq_index;
                u8 device_type_raw[8];
        } selector;
        le32 length;
};

#define VIRTIO_DEV_PART_F_OPTIONAL 0

struct virtio_dev_part {
        struct virtio_dev_part_hdr hdr;
        u8 value[];
};

\end{lstlisting}

Each device part consists of a fixed size \field{hdr} followed by optional
part data in field \field{value}. The device parts are divided into
two categories and identified by \field{part_type}. The common device parts are
independent of the device type and, are in the range \field{0x0000 - 0x01FF}. Common
device parts are listed in
\ref{table:Basic Facilities of a Virtio Device / Device parts / Common device parts}
The device parts in the range \field{0x0200 - 0x05FF} are specific to a device type
such as a network or console device.
The device part is identified by the \field{part_type} field as listed:

\begin{description}
\item[0x0000 - 0x01FF] - common part - used to describe a part of the device that
                         is independent of the device type
\item[0x0200 - 0x05FF] - device type specific part - used to indicate parts
                         that are device type specific
\item[0x0600 - 0xFFFF] - reserved
\end{description}

Some device parts are optional, the device can function without them.
For example, such parts can help improve performance, with the device working
slower, yet still correctly, even without the parts. In another example,
optional parts can be used for validation, with the device being able to deduce
the part itself, the part being helpful to detect driver or user errors.
Such device parts are marked optional by setting bit 0
(VIRTIO_DEV_PART_F_OPTIONAL) in the \field{flags}.

\field{reserved} is reserved and set to zero.

\field{length} indicates the length of the \field{value} in bytes. The length
of the device part depends on the device part itself and is described separately.
The device part data is in \field{value} and is \field{part_type} specific.

\field{selector} further specifies the part. It is only used for some
\field{part_type} values.

\field{selector.pci_common_cfg.offset} is the offset of the
field in the \nameref{sec:Virtio Transport Options / Virtio Over PCI Bus / PCI Device Layout / Common configuration structure layout}. It is valid only when the \field{part_type} is set to VIRTIO_DEV_PART_PCI_COMMON_CFG,
otherwise it is reserved and set to 0.

\field{selector.vq_index.index} is the index of the virtqueue. It is valid
only when the \field{part_type} is VIRTIO_DEV_PART_VQ_CFG or
VIRTIO_DEV_PART_VQ_NOTIFY_CFG.

\field{selector.device_type_raw} is applicable only when the \field{part_type}
corresponds to a device-specific range. The format of
\field{selector.device_type_raw} is device type specific.

\subsection{Common device parts}\label{sec:Basic Facilities of a Virtio Device / Device parts / Common device parts}

Common parts are independent of the device type.
\field{part_type} and \field{value} for each part are documented as follows:

\begin{table}
\caption{Common device parts}
\label{table:Basic Facilities of a Virtio Device / Device parts / Common device parts}
\begin{tabularx}{\textwidth}{ |l||l|X| }
\hline
Type & Name & Description \\
\hline \hline
0x100 & VIRTIO_DEV_PART_DEV_FEATURES & Device features, see \ref{sec:Basic Facilities of a Virtio Device / Device parts / Common device parts / VIRTIO-DEV-PART-DEV-FEATURES} \\
\hline
0x101 & VIRTIO_DEV_PART_DRV_FEATURES & Driver features, \ref{sec:Basic Facilities of a Virtio Device / Device parts / Common device parts / VIRTIO-DEV-PART-DRV-FEATURES} \\
\hline
0x102 & VIRTIO_DEV_PART_PCI_COMMON_CFG & PCI common configuration, see \ref{sec:Basic Facilities of a Virtio Device / Device parts / Common device parts / VIRTIO-DEV-PART-PCI-COMMON-CFG} \\
\hline
0x103 & VIRTIO_DEV_PART_DEVICE_STATUS & Device status, see \ref{sec:Basic Facilities of a Virtio Device / Device parts / Common device parts / VIRTIO-DEV-PART-DEVICE-STATUS} \\
\hline
0x104 & VIRTIO_DEV_PART_VQ_CFG & Virtqueue configuration, see \ref{sec:Basic Facilities of a Virtio Device / Device parts / Common device parts / VIRTIO-DEV-PART-VQ-CFG} \\
\hline
0x105 & VIRTIO_DEV_PART_VQ_NOTIFY_CFG & Virtqueue notification configuration, see \ref{sec:Basic Facilities of a Virtio Device / Device parts / Common device parts / VIRTIO-DEV-PART-VQ-NOTIFY-CFG} \\
\hline
0x106 - 0x2FF & - & Common device parts range reserved for future \\
\hline
\hline
\end{tabularx}
\end{table}

\subsubsection{VIRTIO_DEV_PART_DEV_FEATURES}
\label{sec:Basic Facilities of a Virtio Device / Device parts / Common device parts / VIRTIO-DEV-PART-DEV-FEATURES}

For VIRTIO_DEV_PART_DEV_FEATURES, \field{part_type} is set to 0x100.
The VIRTIO_DEV_PART_DEV_FEATURES field indicates features offered by the device.
\field{value} is in the format of \field{struct virtio_dev_part_features}.
\field{feature_bits} is in the format listed in
\ref{sec:Basic Facilities of a Virtio Device / Feature Bits}.
\field{length} is the length of the \field{struct virtio_dev_part_features}.

If the VIRTIO_DEV_PART_DEV_FEATURES device part is present, there is exactly
one instance of it in the get or set commands.

The VIRTIO_DEV_PART_DEV_FEATURES part is optional for which
the VIRTIO_DEV_PART_F_OPTIONAL (bit 0) \field{flags} is set.

\begin{lstlisting}
struct virtio_dev_part_features {
        le64 feature_bits[];
};
\end{lstlisting}

\subsubsection{VIRTIO_DEV_PART_DRV_FEATURES}
\label{sec:Basic Facilities of a Virtio Device / Device parts / Common device parts / VIRTIO-DEV-PART-DRV-FEATURES}

For VIRTIO_DEV_PART_DRV_FEATURES, \field{part_type} is set to 0x101.
The VIRTIO_DEV_PART_DRV_FEATURES field indicates features set by the driver.
\field{value} is in the format of \field{struct virtio_dev_part_features}.
\field{feature_bits} is in the format listed in
\ref{sec:Basic Facilities of a Virtio Device / Feature Bits}.
\field{length} is the length of the \field{struct virtio_dev_part_features}.

If the VIRTIO_DEV_PART_DEV_FEATURES device part present, there is exactly
one instance of it in the get or set commands.

\subsubsection{VIRTIO_DEV_PART_PCI_COMMON_CFG}
\label{sec:Basic Facilities of a Virtio Device / Device parts / Common device parts / VIRTIO-DEV-PART-PCI-COMMON-CFG}

For VIRTIO_DEV_PART_PCI_COMMON_CFG, \field{part_type} is set to 0x102.
VIRTIO_DEV_PART_PCI_COMMON_CFG refers to the common device configuration
fields. \field{offset} refers to the
byte offset of single field in the common configuration layout described in
\field{struct virtio_pci_common_cfg}. \field{value} is in the format depending on
the \field{offset}, for example when \field{cfg_offset = 18}, \field{value}
is in the format of \field{num_queues}. \field{length} is the length of
\field{value} in bytes of a single structure field whose offset is \field{offset}.

One or multiple VIRTIO_DEV_PART_PCI_COMMON_CFG parts may exist in the
get or set commands; each such part corresponds to a unique \field{offset}.

\subsubsection{VIRTIO_DEV_PART_DEVICE_STATUS}
\label{sec:Basic Facilities of a Virtio Device / Device parts / Common device parts / VIRTIO-DEV-PART-DEVICE-STATUS}

For VIRTIO_DEV_PART_DEVICE_STATUS, \field{part_type} is set to 0x103.
The VIRTIO_DEV_PART_DEVICE_STATUS field indicates the device status as listed in
\ref{sec:Basic Facilities of a Virtio Device / Device Status Field}.
\field{value} is in the format \field{device_status} of
\field{struct virtio_pci_common_cfg}.

If the VIRTIO_DEV_PART_DEV_FEATURES device part is present, there is exactly
one instance of it in the get or set commands.

There is exactly one part may exist in the get or set
commands.

\subsubsection{VIRTIO_DEV_PART_VQ_CFG}
\label{sec:Basic Facilities of a Virtio Device / Device parts / Common device parts / VIRTIO-DEV-PART-VQ-CFG}

For VIRTIO_DEV_PART_VQ_CFG, \field{part_type} is set to 0x104.
\field{value} is in the format \field{struct virtio_dev_part_vq_cfg}.
\field{length} is the length of \field{struct virtio_dev_part_vq_cfg}.

\begin{lstlisting}
struct virtio_dev_part_vq_cfg {
        le16 queue_size;
        le16 vector;
        le16 enabled;
        le16 reserved;
        le64 queue_desc;
        le64 queue_driver;
        le64 queue_device;
};
\end{lstlisting}

\field{queue_size}, \field{vector}, \field{queue_desc},
\field{queue_driver} and \field{queue_device} correspond to the
fields of \field{struct virtio_pci_common_cfg} when used for PCI transport.

One or multiple instances of the device part VIRTIO_DEV_PART_VQ_CFG may exist in
the get and set commands. Each such device part corresponds to a unique virtqueue identified
by the \field{vq_index.index}.

\subsubsection{VIRTIO_DEV_PART_VQ_NOTIFY_CFG}
\label{sec:Basic Facilities of a Virtio Device / Device parts / Common device parts / VIRTIO-DEV-PART-VQ-NOTIFY-CFG}

For VIRTIO_DEV_PART_VQ_NOTIFY_CFG, \field{part_type} is set to 0x105.
\field{value} is in the format \field{struct virtio_dev_part_vq_notify_data}.
\field{length} is the length of \field{struct virtio_dev_part_vq_notify_data}.

\begin{lstlisting}
struct virtio_dev_part_vq_notify_data {
        le16 queue_notify_off;
        le16 queue_notif_config_data;
        u8 reserved[4];
};
\end{lstlisting}

\field{queue_notify_off} and \field{queue_notif_config_data} corresponds to the
fields in \field{struct virtio_pci_common_cfg} described in the
\nameref{sec:Virtio Transport Options / Virtio Over PCI Bus / PCI Device Layout / Common configuration structure layout}.

One or multiple instance of the device part VIRTIO_DEV_PART_VQ_NOTIFY_CFG may exist
in the get and set commands, each such device part corresponds to a unique
virtqueue identified by the \field{vq_index.index}.

\field{reserved} is reserved and set to 0.

\subsection{Assumptions}
For the SR-IOV group type, some hypervisors do not allow the driver to access
the PCI configuration space and the MSI-X Table space directly. Such hypervisors
query and save these fields without the need for this device parts.
Therefore, this version of the specification does not have it in the device parts. A future
extension of the device part may further include them as new device part.


\chapter{General Initialization And Device Operation}\label{sec:General Initialization And Device Operation}

We start with an overview of device initialization, then expand on the
details of the device and how each step is performed.  This section
is best read along with the bus-specific section which describes
how to communicate with the specific device.

\section{Device Initialization}\label{sec:General Initialization And Device Operation / Device Initialization}

\drivernormative{\subsection}{Device Initialization}{General Initialization And Device Operation / Device Initialization}
The driver MUST follow this sequence to initialize a device:

\begin{enumerate}
\item Reset the device.

\item Set the ACKNOWLEDGE status bit: the guest OS has noticed the device.

\item Set the DRIVER status bit: the guest OS knows how to drive the device.

\item\label{itm:General Initialization And Device Operation /
Device Initialization / Read feature bits} Read device feature bits, and write the subset of feature bits
   understood by the OS and driver to the device.  During this step the
   driver MAY read (but MUST NOT write) the device-specific configuration fields to check that it can support the device before accepting it.

\item\label{itm:General Initialization And Device Operation / Device Initialization / Set FEATURES-OK} Set the FEATURES_OK status bit.  The driver MUST NOT accept
   new feature bits after this step.

\item\label{itm:General Initialization And Device Operation / Device Initialization / Re-read FEATURES-OK} Re-read \field{device status} to ensure the FEATURES_OK bit is still
   set: otherwise, the device does not support our subset of features
   and the device is unusable.

\item\label{itm:General Initialization And Device Operation / Device Initialization / Device-specific Setup} Perform device-specific setup, including discovery of virtqueues for the
   device, optional per-bus setup, reading and possibly writing the
   device's virtio configuration space, and population of virtqueues.

\item\label{itm:General Initialization And Device Operation / Device Initialization / Set DRIVER-OK} Set the DRIVER_OK status bit.  At this point the device is
   ``live''.
\end{enumerate}

If any of these steps go irrecoverably wrong, the driver SHOULD
set the FAILED status bit to indicate that it has given up on the
device (it can reset the device later to restart if desired).  The
driver MUST NOT continue initialization in that case.

The driver MUST NOT send any buffer available notifications to
the device before setting DRIVER_OK.

\subsection{Legacy Interface: Device Initialization}\label{sec:General Initialization And Device Operation / Device Initialization / Legacy Interface: Device Initialization}
Legacy devices did not support the FEATURES_OK status bit, and thus did
not have a graceful way for the device to indicate unsupported feature
combinations.  They also did not provide a clear mechanism to end
feature negotiation, which meant that devices finalized features on
first-use, and no features could be introduced which radically changed
the initial operation of the device.

Legacy driver implementations often used the device before setting the
DRIVER_OK bit, and sometimes even before writing the feature bits
to the device.

The result was the steps \ref{itm:General Initialization And
Device Operation / Device Initialization / Set FEATURES-OK} and
\ref{itm:General Initialization And Device Operation / Device
Initialization / Re-read FEATURES-OK} were omitted, and steps
\ref{itm:General Initialization And Device Operation /
Device Initialization / Read feature bits},
\ref{itm:General Initialization And Device Operation / Device Initialization / Device-specific Setup} and \ref{itm:General Initialization And Device Operation / Device Initialization / Set DRIVER-OK}
were conflated.

Therefore, when using the legacy interface:
\begin{itemize}
\item
The transitional driver MUST execute the initialization
sequence as described in \ref{sec:General Initialization And Device
Operation / Device Initialization}
but omitting the steps \ref{itm:General Initialization And Device
Operation / Device Initialization / Set FEATURES-OK} and
\ref{itm:General Initialization And Device Operation / Device
Initialization / Re-read FEATURES-OK}.

\item
The transitional device MUST support the driver
writing device configuration fields
before the step \ref{itm:General Initialization And Device Operation /
Device Initialization / Read feature bits}.
\item
The transitional device MUST support the driver
using the device before the step \ref{itm:General Initialization
And Device Operation / Device Initialization / Set DRIVER-OK}.
\end{itemize}

\section{Device Operation}\label{sec:General Initialization And Device Operation / Device Operation}

When operating the device, each field in the device configuration
space can be changed by either the driver or the device.

Whenever such a configuration change is triggered by the device,
driver is notified. This makes it possible for drivers to
cache device configuration, avoiding expensive configuration
reads unless notified.


\subsection{Notification of Device Configuration Changes}\label{sec:General Initialization And Device Operation / Device Operation / Notification of Device Configuration Changes}

For devices where the device-specific configuration information can be
changed, a configuration change notification is sent when a
device-specific configuration change occurs.

In addition, this notification is triggered by the device setting
DEVICE_NEEDS_RESET (see \ref{sec:Basic Facilities of a Virtio Device / Device Status Field / DEVICENEEDSRESET}).

\section{Device Cleanup}\label{sec:General Initialization And Device Operation / Device Cleanup}

Once the driver has set the DRIVER_OK status bit, all the configured
virtqueue of the device are considered live.  None of the virtqueues
of a device are live once the device has been reset.

\drivernormative{\subsection}{Device Cleanup}{General Initialization And Device Operation / Device Cleanup}

A driver MUST NOT alter virtqueue entries for exposed buffers,
i.e., buffers which have been
made available to the device (and not been used by the device)
of a live virtqueue.

Thus a driver MUST ensure a virtqueue isn't live (by device reset) before removing exposed buffers.

\chapter{Virtio Transport Options}\label{sec:Virtio Transport Options}

Virtio can use various different buses, thus the standard is split
into virtio general and bus-specific sections.

\section{Virtio Over PCI Bus}\label{sec:Virtio Transport Options / Virtio Over PCI Bus}

Virtio devices are commonly implemented as PCI devices.

A Virtio device can be implemented as any kind of PCI device:
a Conventional PCI device or a PCI Express
device.  To assure designs meet the latest level
requirements, see
the PCI-SIG home page at \url{http://www.pcisig.com} for any
approved changes.

\devicenormative{\subsection}{Virtio Over PCI Bus}{Virtio Transport Options / Virtio Over PCI Bus}
A Virtio device using Virtio Over PCI Bus MUST expose to
guest an interface that meets the specification requirements of
the appropriate PCI specification: \hyperref[intro:PCI]{[PCI]}
and \hyperref[intro:PCIe]{[PCIe]}
respectively.

\subsection{PCI Device Discovery}\label{sec:Virtio Transport Options / Virtio Over PCI Bus / PCI Device Discovery}

Any PCI device with PCI Vendor ID 0x1AF4, and PCI Device ID 0x1000 through
0x107F inclusive is a virtio device. The actual value within this range
indicates which virtio device is supported by the device.
The PCI Device ID is calculated by adding 0x1040 to the Virtio Device ID,
as indicated in section \ref{sec:Device Types}.
Additionally, devices MAY utilize a Transitional PCI Device ID range,
0x1000 to 0x103F depending on the device type.

\devicenormative{\subsubsection}{PCI Device Discovery}{Virtio Transport Options / Virtio Over PCI Bus / PCI Device Discovery}

Devices MUST have the PCI Vendor ID 0x1AF4.
Devices MUST either have the PCI Device ID calculated by adding 0x1040
to the Virtio Device ID, as indicated in section \ref{sec:Device
Types} or have the Transitional PCI Device ID depending on the device type,
as follows:

\begin{tabular}{|l|c|}
\hline
Transitional PCI Device ID  &  Virtio Device    \\
\hline \hline
0x1000      &   network device     \\
\hline
0x1001     &   block device     \\
\hline
0x1002     & memory ballooning (traditional)  \\
\hline
0x1003     &      console       \\
\hline
0x1004     &     SCSI host      \\
\hline
0x1005     &  entropy source    \\
\hline
0x1009     &   9P transport     \\
\hline
\end{tabular}

For example, the network device with the Virtio Device ID 1
has the PCI Device ID 0x1041 or the Transitional PCI Device ID 0x1000.

The PCI Subsystem Vendor ID and the PCI Subsystem Device ID MAY reflect
the PCI Vendor and Device ID of the environment (for informational purposes by the driver).

Non-transitional devices SHOULD have a PCI Device ID in the range
0x1040 to 0x107f.
Non-transitional devices SHOULD have a PCI Revision ID of 1 or higher.
Non-transitional devices SHOULD have a PCI Subsystem Device ID of 0x40 or higher.

This is to reduce the chance of a legacy driver attempting
to drive the device.

\drivernormative{\subsubsection}{PCI Device Discovery}{Virtio Transport Options / Virtio Over PCI Bus / PCI Device Discovery}
Drivers MUST match devices with the PCI Vendor ID 0x1AF4 and
the PCI Device ID in the range 0x1040 to 0x107f,
calculated by adding 0x1040 to the Virtio Device ID,
as indicated in section \ref{sec:Device Types}.
Drivers for device types listed in section \ref{sec:Virtio
Transport Options / Virtio Over PCI Bus / PCI Device Discovery}
MUST match devices with the PCI Vendor ID 0x1AF4 and
the Transitional PCI Device ID indicated in section
 \ref{sec:Virtio
Transport Options / Virtio Over PCI Bus / PCI Device Discovery}.

Drivers MUST match any PCI Revision ID value.
Drivers MAY match any PCI Subsystem Vendor ID and any
PCI Subsystem Device ID value.

\subsubsection{Legacy Interfaces: A Note on PCI Device Discovery}\label{sec:Virtio Transport Options / Virtio Over PCI Bus / PCI Device Discovery / Legacy Interfaces: A Note on PCI Device Discovery}
Transitional devices MUST have a PCI Revision ID of 0.
Transitional devices MUST have the PCI Subsystem Device ID
matching the Virtio Device ID, as indicated in section \ref{sec:Device Types}.
Transitional devices MUST have the Transitional PCI Device ID in
the range 0x1000 to 0x103f.

This is to match legacy drivers.

\subsection{PCI Device Layout}\label{sec:Virtio Transport Options / Virtio Over PCI Bus / PCI Device Layout}

The device is configured via I/O and/or memory regions (though see
\ref{sec:Virtio Transport Options / Virtio Over PCI Bus / PCI Device Layout / PCI configuration access capability}
for access via the PCI configuration space), as specified by Virtio
Structure PCI Capabilities.

Fields of different sizes are present in the device
configuration regions.
All 64-bit, 32-bit and 16-bit fields are little-endian.
64-bit fields are to be treated as two 32-bit fields,
with low 32 bit part followed by the high 32 bit part.

\drivernormative{\subsubsection}{PCI Device Layout}{Virtio Transport Options / Virtio Over PCI Bus / PCI Device Layout}

For device configuration access, the driver MUST use 8-bit wide
accesses for 8-bit wide fields, 16-bit wide and aligned accesses
for 16-bit wide fields and 32-bit wide and aligned accesses for
32-bit and 64-bit wide fields. For 64-bit fields, the driver MAY
access each of the high and low 32-bit parts of the field
independently.

\devicenormative{\subsubsection}{PCI Device Layout}{Virtio Transport Options / Virtio Over PCI Bus / PCI Device Layout}

For 64-bit device configuration fields, the device MUST allow driver
independent access to high and low 32-bit parts of the field.

\subsection{Virtio Structure PCI Capabilities}\label{sec:Virtio Transport Options / Virtio Over PCI Bus / Virtio Structure PCI Capabilities}

The virtio device configuration layout includes several structures:
\begin{itemize}
\item Common configuration
\item Notifications
\item ISR Status
\item Device-specific configuration (optional)
\item PCI configuration access
\end{itemize}

Each structure can be mapped by a Base Address register (BAR) belonging to
the function, or accessed via the special VIRTIO_PCI_CAP_PCI_CFG field in the PCI configuration space.

The location of each structure is specified using a vendor-specific PCI capability located
on the capability list in PCI configuration space of the device.
This virtio structure capability uses little-endian format; all fields are
read-only for the driver unless stated otherwise:

\begin{lstlisting}
struct virtio_pci_cap {
        u8 cap_vndr;    /* Generic PCI field: PCI_CAP_ID_VNDR */
        u8 cap_next;    /* Generic PCI field: next ptr. */
        u8 cap_len;     /* Generic PCI field: capability length */
        u8 cfg_type;    /* Identifies the structure. */
        u8 bar;         /* Where to find it. */
        u8 id;          /* Multiple capabilities of the same type */
        u8 padding[2];  /* Pad to full dword. */
        le32 offset;    /* Offset within bar. */
        le32 length;    /* Length of the structure, in bytes. */
};
\end{lstlisting}

This structure can be followed by extra data, depending on
\field{cfg_type}, as documented below.

The fields are interpreted as follows:

\begin{description}
\item[\field{cap_vndr}]
        0x09; Identifies a vendor-specific capability.

\item[\field{cap_next}]
        Link to next capability in the capability list in the PCI configuration space.

\item[\field{cap_len}]
        Length of this capability structure, including the whole of
        struct virtio_pci_cap, and extra data if any.
        This length MAY include padding, or fields unused by the driver.

\item[\field{cfg_type}]
        identifies the structure, according to the following table:

\begin{lstlisting}
/* Common configuration */
#define VIRTIO_PCI_CAP_COMMON_CFG        1
/* Notifications */
#define VIRTIO_PCI_CAP_NOTIFY_CFG        2
/* ISR Status */
#define VIRTIO_PCI_CAP_ISR_CFG           3
/* Device specific configuration */
#define VIRTIO_PCI_CAP_DEVICE_CFG        4
/* PCI configuration access */
#define VIRTIO_PCI_CAP_PCI_CFG           5
/* Shared memory region */
#define VIRTIO_PCI_CAP_SHARED_MEMORY_CFG 8
/* Vendor-specific data */
#define VIRTIO_PCI_CAP_VENDOR_CFG        9
\end{lstlisting}

        Any other value is reserved for future use.

        Each structure is detailed individually below.

        The device MAY offer more than one structure of any type - this makes it
        possible for the device to expose multiple interfaces to drivers.  The order of
        the capabilities in the capability list specifies the order of preference
        suggested by the device.  A device may specify that this ordering mechanism be
        overridden by the use of the \field{id} field.
        \begin{note}
          For example, on some hypervisors, notifications using IO accesses are
        faster than memory accesses. In this case, the device would expose two
        capabilities with \field{cfg_type} set to VIRTIO_PCI_CAP_NOTIFY_CFG:
        the first one addressing an I/O BAR, the second one addressing a memory BAR.
        In this example, the driver would use the I/O BAR if I/O resources are available, and fall back on
        memory BAR when I/O resources are unavailable.
        \end{note}

\item[\field{bar}]
        values 0x0 to 0x5 specify a Base Address register (BAR) belonging to
        the function located beginning at 10h in PCI Configuration Space
        and used to map the structure into Memory or I/O Space.
        The BAR is permitted to be either 32-bit or 64-bit, it can map Memory Space
        or I/O Space.

        Any other value is reserved for future use.

\item[\field{id}]
        Used by some device types to uniquely identify multiple capabilities
        of a certain type. If the device type does not specify the meaning of
        this field, its contents are undefined.


\item[\field{offset}]
        indicates where the structure begins relative to the base address associated
        with the BAR.  The alignment requirements of \field{offset} are indicated
        in each structure-specific section below.

\item[\field{length}]
        indicates the length of the structure.

        \field{length} MAY include padding, or fields unused by the driver, or
        future extensions.

        \begin{note}
        For example, a future device might present a large structure size of several
        MBytes.
        As current devices never utilize structures larger than 4KBytes in size,
        driver MAY limit the mapped structure size to e.g.
        4KBytes (thus ignoring parts of structure after the first
        4KBytes) to allow forward compatibility with such devices without loss of
        functionality and without wasting resources.
        \end{note}
\end{description}

A variant of this type, struct virtio_pci_cap64, is defined for
those capabilities that require offsets or lengths larger than
4GiB:

\begin{lstlisting}
struct virtio_pci_cap64 {
        struct virtio_pci_cap cap;
        le32 offset_hi;
        le32 length_hi;
};
\end{lstlisting}

Given that the \field{cap.length} and \field{cap.offset} fields
are only 32 bit, the additional \field{offset_hi} and \field{length_hi}
fields provide the most significant 32 bits of a total 64 bit offset and
length within the BAR specified by \field{cap.bar}.

\drivernormative{\subsubsection}{Virtio Structure PCI Capabilities}{Virtio Transport Options / Virtio Over PCI Bus / Virtio Structure PCI Capabilities}

The driver MUST ignore any vendor-specific capability structure which has
a reserved \field{cfg_type} value.

The driver SHOULD use the first instance of each virtio structure type they can
support.

The driver MUST accept a \field{cap_len} value which is larger than specified here.

The driver MUST ignore any vendor-specific capability structure which has
a reserved \field{bar} value.

        The drivers SHOULD only map part of configuration structure
        large enough for device operation.  The drivers MUST handle
        an unexpectedly large \field{length}, but MAY check that \field{length}
        is large enough for device operation.

The driver MUST NOT write into any field of the capability structure,
with the exception of those with \field{cap_type} VIRTIO_PCI_CAP_PCI_CFG as
detailed in \ref{drivernormative:Virtio Transport Options / Virtio Over PCI Bus / PCI Device Layout / PCI configuration access capability}.

\devicenormative{\subsubsection}{Virtio Structure PCI Capabilities}{Virtio Transport Options / Virtio Over PCI Bus / Virtio Structure PCI Capabilities}

The device MUST include any extra data (from the beginning of the \field{cap_vndr} field
through end of the extra data fields if any) in \field{cap_len}.
The device MAY append extra data
or padding to any structure beyond that.

If the device presents multiple structures of the same type, it SHOULD order
them from optimal (first) to least-optimal (last).

\subsubsection{Common configuration structure layout}\label{sec:Virtio Transport Options / Virtio Over PCI Bus / PCI Device Layout / Common configuration structure layout}

The common configuration structure is found at the \field{bar} and \field{offset} within the VIRTIO_PCI_CAP_COMMON_CFG capability; its layout is below.

\begin{lstlisting}
struct virtio_pci_common_cfg {
        /* About the whole device. */
        le32 device_feature_select;     /* read-write */
        le32 device_feature;            /* read-only for driver */
        le32 driver_feature_select;     /* read-write */
        le32 driver_feature;            /* read-write */
        le16 config_msix_vector;        /* read-write */
        le16 num_queues;                /* read-only for driver */
        u8 device_status;               /* read-write */
        u8 config_generation;           /* read-only for driver */

        /* About a specific virtqueue. */
        le16 queue_select;              /* read-write */
        le16 queue_size;                /* read-write */
        le16 queue_msix_vector;         /* read-write */
        le16 queue_enable;              /* read-write */
        le16 queue_notify_off;          /* read-only for driver */
        le64 queue_desc;                /* read-write */
        le64 queue_driver;              /* read-write */
        le64 queue_device;              /* read-write */
        le16 queue_notif_config_data;   /* read-only for driver */
        le16 queue_reset;               /* read-write */

        /* About the administration virtqueue. */
        le16 admin_queue_index;         /* read-only for driver */
        le16 admin_queue_num;         /* read-only for driver */
};
\end{lstlisting}

\begin{description}
\item[\field{device_feature_select}]
        The driver uses this to select which feature bits \field{device_feature} shows.
        Value 0x0 selects Feature Bits 0 to 31, 0x1 selects Feature Bits 32 to 63, etc.

\item[\field{device_feature}]
        The device uses this to report which feature bits it is
        offering to the driver: the driver writes to
        \field{device_feature_select} to select which feature bits are presented.

\item[\field{driver_feature_select}]
        The driver uses this to select which feature bits \field{driver_feature} shows.
        Value 0x0 selects Feature Bits 0 to 31, 0x1 selects Feature Bits 32 to 63, etc.

\item[\field{driver_feature}]
        The driver writes this to accept feature bits offered by the device.
        Driver Feature Bits selected by \field{driver_feature_select}.

\item[\field{config_msix_vector}]
        Set by the driver to the MSI-X vector for configuration change notifications.

\item[\field{num_queues}]
        The device specifies the maximum number of virtqueues supported here.
        This excludes administration virtqueues if any are supported.

\item[\field{device_status}]
        The driver writes the device status here (see \ref{sec:Basic Facilities of a Virtio Device / Device Status Field}). Writing 0 into this
        field resets the device.

\item[\field{config_generation}]
        Configuration atomicity value.  The device changes this every time the
        configuration noticeably changes.

\item[\field{queue_select}]
        Queue Select. The driver selects which virtqueue the following
        fields refer to.

\item[\field{queue_size}]
        Queue Size.  On reset, specifies the maximum queue size supported by
        the device. This can be modified by the driver to reduce memory requirements.
        A 0 means the queue is unavailable.

\item[\field{queue_msix_vector}]
        Set by the driver to the MSI-X vector for virtqueue notifications.

\item[\field{queue_enable}]
        The driver uses this to selectively prevent the device from executing requests from this virtqueue.
        1 - enabled; 0 - disabled.

\item[\field{queue_notify_off}]
        The driver reads this to calculate the offset from start of Notification structure at
        which this virtqueue is located.
        \begin{note} this is \em{not} an offset in bytes.
        See \ref{sec:Virtio Transport Options / Virtio Over PCI Bus / PCI Device Layout / Notification capability} below.
        \end{note}

\item[\field{queue_desc}]
        The driver writes the physical address of Descriptor Area here.  See section \ref{sec:Basic Facilities of a Virtio Device / Virtqueues}.

\item[\field{queue_driver}]
        The driver writes the physical address of Driver Area here.  See section \ref{sec:Basic Facilities of a Virtio Device / Virtqueues}.

\item[\field{queue_device}]
        The driver writes the physical address of Device Area here.  See section \ref{sec:Basic Facilities of a Virtio Device / Virtqueues}.

\item[\field{queue_notif_config_data}]
        This field exists only if VIRTIO_F_NOTIF_CONFIG_DATA has been negotiated.
        The driver will use this value when driver sends available buffer
        notification to the device.
        See section \ref{sec:Virtio Transport Options / Virtio Over PCI Bus / PCI-specific Initialization And Device Operation / Available Buffer Notifications}.
        \begin{note}
        This field provides the device with flexibility to determine how virtqueues
        will be referred to in available buffer notifications.
        In a trivial case the device can set \field{queue_notif_config_data} to
        the virtqueue index. Some devices may benefit from providing another value,
        for example an internal virtqueue identifier, or an internal offset
        related to the virtqueue index.
        \end{note}
        \begin{note}
        This field was previously known as queue_notify_data.
        \end{note}

\item[\field{queue_reset}]
        The driver uses this to selectively reset the queue.
        This field exists only if VIRTIO_F_RING_RESET has been
        negotiated. (see \ref{sec:Basic Facilities of a Virtio Device / Virtqueues / Virtqueue Reset}).

\item[\field{admin_queue_index}]
        The device uses this to report the index of the first administration virtqueue.
        This field is valid only if VIRTIO_F_ADMIN_VQ has been negotiated.
\item[\field{admin_queue_num}]
	The device uses this to report the number of the
	supported administration virtqueues.
	Virtqueues with index
	between \field{admin_queue_index} and (\field{admin_queue_index} +
	\field{admin_queue_num} - 1) inclusive serve as administration
	virtqueues.
	The value 0 indicates no supported administration virtqueues.
	This field is valid only if VIRTIO_F_ADMIN_VQ has been
	negotiated.
\end{description}

\devicenormative{\paragraph}{Common configuration structure layout}{Virtio Transport Options / Virtio Over PCI Bus / PCI Device Layout / Common configuration structure layout}
\field{offset} MUST be 4-byte aligned.

The device MUST present at least one common configuration capability.

The device MUST present the feature bits it is offering in \field{device_feature}, starting at bit \field{device_feature_select} $*$ 32 for any \field{device_feature_select} written by the driver.
\begin{note}
  This means that it will present 0 for any \field{device_feature_select} other than 0 or 1, since no feature defined here exceeds 63.
\end{note}

The device MUST present any valid feature bits the driver has written in \field{driver_feature}, starting at bit \field{driver_feature_select} $*$ 32 for any \field{driver_feature_select} written by the driver.  Valid feature bits are those which are subset of the corresponding \field{device_feature} bits.  The device MAY present invalid bits written by the driver.

\begin{note}
  This means that a device can ignore writes for feature bits it never
  offers, and simply present 0 on reads.  Or it can just mirror what the driver wrote
  (but it will still have to check them when the driver sets FEATURES_OK).
\end{note}

\begin{note}
  A driver shouldn't write invalid bits anyway, as per \ref{drivernormative:General Initialization And Device Operation / Device Initialization}, but this attempts to handle it.
\end{note}

The device MUST present a changed \field{config_generation} after the
driver has read a device-specific configuration value which has
changed since any part of the device-specific configuration was last
read.
\begin{note}
As \field{config_generation} is an 8-bit value, simply incrementing it
on every configuration change could violate this requirement due to wrap.
Better would be to set an internal flag when it has changed,
and if that flag is set when the driver reads from the device-specific
configuration, increment \field{config_generation} and clear the flag.
\end{note}

The device MUST reset when 0 is written to \field{device_status}, and
present a 0 in \field{device_status} once that is done.

The device MUST present a 0 in \field{queue_enable} on reset.

If VIRTIO_F_RING_RESET has been negotiated, the device MUST present a 0 in
\field{queue_reset} on reset.

If VIRTIO_F_RING_RESET has been negotiated, the device MUST present a 0 in
\field{queue_reset} after the virtqueue is enabled with \field{queue_enable}.

The device MUST reset the queue when 1 is written to \field{queue_reset}. The
device MUST continue to present 1 in \field{queue_reset} as long as the queue reset
is ongoing. The device MUST present 0 in both \field{queue_reset} and \field{queue_enable}
when queue reset has completed.
(see \ref{sec:Basic Facilities of a Virtio Device / Virtqueues / Virtqueue Reset}).

The device MUST present a 0 in \field{queue_size} if the virtqueue
corresponding to the current \field{queue_select} is unavailable.

If VIRTIO_F_RING_PACKED has not been negotiated, the device MUST
present either a value of 0 or a power of 2 in
\field{queue_size}.

If VIRTIO_F_ADMIN_VQ has been negotiated, the value
\field{admin_queue_index} MUST be equal to, or bigger than
\field{num_queues}; also, \field{admin_queue_num} MUST be
smaller than, or equal to 0x10000 - \field{admin_queue_index},
to ensure that indices of valid admin queues fit into
a 16 bit range beyond all other virtqueues.

\drivernormative{\paragraph}{Common configuration structure layout}{Virtio Transport Options / Virtio Over PCI Bus / PCI Device Layout / Common configuration structure layout}

The driver MUST NOT write to \field{device_feature}, \field{num_queues},
\field{config_generation}, \field{queue_notify_off} or
\field{queue_notif_config_data}.

If VIRTIO_F_RING_PACKED has been negotiated,
the driver MUST NOT write the value 0 to \field{queue_size}.
If VIRTIO_F_RING_PACKED has not been negotiated,
the driver MUST NOT write a value which is not a power of 2 to \field{queue_size}.

The driver MUST configure the other virtqueue fields before enabling the virtqueue
with \field{queue_enable}.

After writing 0 to \field{device_status}, the driver MUST wait for a read of
\field{device_status} to return 0 before reinitializing the device.

The driver MUST NOT write a 0 to \field{queue_enable}.

If VIRTIO_F_RING_RESET has been negotiated, after the driver writes 1 to
\field{queue_reset} to reset the queue, the driver MUST NOT consider queue
reset to be complete until it reads back 0 in \field{queue_reset}. The driver
MAY re-enable the queue by writing 1 to \field{queue_enable} after ensuring
that other virtqueue fields have been set up correctly. The driver MAY set
driver-writeable queue configuration values to different values than those that
were used before the queue reset.
(see \ref{sec:Basic Facilities of a Virtio Device / Virtqueues / Virtqueue Reset}).

If VIRTIO_F_ADMIN_VQ has been negotiated, and if the driver
configures any administration virtqueues, the driver MUST
configure the administration virtqueues using the index
in the range \field{admin_queue_index} to
\field{admin_queue_index} + \field{admin_queue_num} - 1 inclusive.
The driver MAY configure fewer administration virtqueues than
supported by the device.

\subsubsection{Notification structure layout}\label{sec:Virtio Transport Options / Virtio Over PCI Bus / PCI Device Layout / Notification capability}

The notification location is found using the VIRTIO_PCI_CAP_NOTIFY_CFG
capability.  This capability is immediately followed by an additional
field, like so:

\begin{lstlisting}
struct virtio_pci_notify_cap {
        struct virtio_pci_cap cap;
        le32 notify_off_multiplier; /* Multiplier for queue_notify_off. */
};
\end{lstlisting}

\field{notify_off_multiplier} is combined with the \field{queue_notify_off} to
derive the Queue Notify address within a BAR for a virtqueue:

\begin{lstlisting}
        cap.offset + queue_notify_off * notify_off_multiplier
\end{lstlisting}

The \field{cap.offset} and \field{notify_off_multiplier} are taken from the
notification capability structure above, and the \field{queue_notify_off} is
taken from the common configuration structure.

\begin{note}
For example, if \field{notifier_off_multiplier} is 0, the device uses
the same Queue Notify address for all queues.
\end{note}

\devicenormative{\paragraph}{Notification capability}{Virtio Transport Options / Virtio Over PCI Bus / PCI Device Layout / Notification capability}
The device MUST present at least one notification capability.

For devices not offering VIRTIO_F_NOTIFICATION_DATA:

The \field{cap.offset} MUST be 2-byte aligned.

The device MUST either present \field{notify_off_multiplier} as an even power of 2,
or present \field{notify_off_multiplier} as 0.

The value \field{cap.length} presented by the device MUST be at least 2
and MUST be large enough to support queue notification offsets
for all supported queues in all possible configurations.

For all queues, the value \field{cap.length} presented by the device MUST satisfy:
\begin{lstlisting}
cap.length >= queue_notify_off * notify_off_multiplier + 2
\end{lstlisting}

For devices offering VIRTIO_F_NOTIFICATION_DATA:

The device MUST either present \field{notify_off_multiplier} as a
number that is a power of 2 that is also a multiple 4,
or present \field{notify_off_multiplier} as 0.

The \field{cap.offset} MUST be 4-byte aligned.

The value \field{cap.length} presented by the device MUST be at least 4
and MUST be large enough to support queue notification offsets
for all supported queues in all possible configurations.

For all queues, the value \field{cap.length} presented by the device MUST satisfy:
\begin{lstlisting}
cap.length >= queue_notify_off * notify_off_multiplier + 4
\end{lstlisting}

\subsubsection{ISR status capability}\label{sec:Virtio Transport Options / Virtio Over PCI Bus / PCI Device Layout / ISR status capability}

The VIRTIO_PCI_CAP_ISR_CFG capability
refers to at least a single byte, which contains the 8-bit ISR status field
to be used for INT\#x interrupt handling.

The \field{offset} for the \field{ISR status} has no alignment requirements.

The ISR bits allow the driver to distinguish between device-specific configuration
change interrupts and normal virtqueue interrupts:

\begin{tabular}{ |l||l|l|l| }
\hline
Bits       & 0                               & 1               &  2 to 31 \\
\hline
Purpose    & Queue Interrupt  & Device Configuration Interrupt & Reserved \\
\hline
\end{tabular}

To avoid an extra access, simply reading this register resets it to 0 and
causes the device to de-assert the interrupt.

In this way, driver read of ISR status causes the device to de-assert
an interrupt.

See sections \ref{sec:Virtio Transport Options / Virtio Over PCI Bus / PCI-specific Initialization And Device Operation / Used Buffer Notifications} and \ref{sec:Virtio Transport Options / Virtio Over PCI Bus / PCI-specific Initialization And Device Operation / Notification of Device Configuration Changes} for how this is used.

\devicenormative{\paragraph}{ISR status capability}{Virtio Transport Options / Virtio Over PCI Bus / PCI Device Layout / ISR status capability}

The device MUST present at least one VIRTIO_PCI_CAP_ISR_CFG capability.

The device MUST set the Device Configuration Interrupt bit
in \field{ISR status} before sending a device configuration
change notification to the driver.

If MSI-X capability is disabled, the device MUST set the Queue
Interrupt bit in \field{ISR status} before sending a virtqueue
notification to the driver.

If MSI-X capability is disabled, the device MUST set the Interrupt Status
bit in the PCI Status register in the PCI Configuration Header of
the device to the logical OR of all bits in \field{ISR status} of
the device.  The device then asserts/deasserts INT\#x interrupts unless masked
according to standard PCI rules \hyperref[intro:PCI]{[PCI]}.

The device MUST reset \field{ISR status} to 0 on driver read.

\drivernormative{\paragraph}{ISR status capability}{Virtio Transport Options / Virtio Over PCI Bus / PCI Device Layout / ISR status capability}

If MSI-X capability is enabled, the driver SHOULD NOT access
\field{ISR status} upon detecting a Queue Interrupt.

\subsubsection{Device-specific configuration}\label{sec:Virtio Transport Options / Virtio Over PCI Bus / PCI Device Layout / Device-specific configuration}

The device MUST present at least one VIRTIO_PCI_CAP_DEVICE_CFG capability for
any device type which has a device-specific configuration.

\devicenormative{\paragraph}{Device-specific configuration}{Virtio Transport Options / Virtio Over PCI Bus / PCI Device Layout / Device-specific configuration}

The \field{offset} for the device-specific configuration MUST be 4-byte aligned.

\subsubsection{Shared memory capability}\label{sec:Virtio Transport Options / Virtio Over PCI Bus / PCI Device Layout / Shared memory capability}

Shared memory regions \ref{sec:Basic Facilities of a Virtio
Device / Shared Memory Regions} are enumerated on the PCI transport
as a sequence of VIRTIO_PCI_CAP_SHARED_MEMORY_CFG capabilities, one per region.

The capability is defined by a struct virtio_pci_cap64 and
utilises the \field{cap.id} to allow multiple shared memory
regions per device.
The identifier in \field{cap.id} does not denote a certain order of
preference; it is only used to uniquely identify a region.

\devicenormative{\paragraph}{Shared memory capability}{Virtio Transport Options / Virtio Over PCI Bus / PCI Device Layout / Shared memory capability}

The region defined by the combination of the \field{cap.offset},
\field{offset_hi}, and \field{cap.length}, \field{length_hi}
fields MUST be contained within the BAR specified by
\field{cap.bar}.

The \field{cap.id} MUST be unique for any one device instance.

\subsubsection{Vendor data capability}\label{sec:Virtio
Transport Options / Virtio Over PCI Bus / PCI Device Layout /
Vendor data capability}

The optional Vendor data capability allows the device to present
vendor-specific data to the driver, without
conflicts, for debugging and/or reporting purposes,
and without conflicting with standard functionality.

This capability augments but does not replace the standard
subsystem ID and subsystem vendor ID fields
(offsets 0x2C and 0x2E in the PCI configuration space header)
as specified by \hyperref[intro:PCI]{[PCI]}.

Vendor data capability is enumerated on the PCI transport
as a VIRTIO_PCI_CAP_VENDOR_CFG capability.

The capability has the following structure:
\begin{lstlisting}
struct virtio_pci_vndr_data {
        u8 cap_vndr;    /* Generic PCI field: PCI_CAP_ID_VNDR */
        u8 cap_next;    /* Generic PCI field: next ptr. */
        u8 cap_len;     /* Generic PCI field: capability length */
        u8 cfg_type;    /* Identifies the structure. */
        u16 vendor_id;  /* Identifies the vendor-specific format. */
	/* For Vendor Definition */
	/* Pads structure to a multiple of 4 bytes */
	/* Reads must not have side effects */
};
\end{lstlisting}

Where \field{vendor_id} identifies the PCI-SIG assigned Vendor ID
as specified by \hyperref[intro:PCI]{[PCI]}.

Note that the capability size is required to be a multiple of 4.

To make it safe for a generic driver to access the capability,
reads from this capability MUST NOT have any side effects.

\devicenormative{\paragraph}{Vendor data capability}{Virtio
Transport Options / Virtio Over PCI Bus / PCI Device Layout /
Vendor data capability}

Devices CAN present \field{vendor_id} that does not match
either the PCI Vendor ID or the PCI Subsystem Vendor ID.

Devices CAN present multiple Vendor data capabilities with
either different or identical \field{vendor_id} values.

The value \field{vendor_id} MUST NOT equal 0x1AF4.

The size of the Vendor data capability MUST be a multiple of 4 bytes.

Reads of the Vendor data capability by the driver MUST NOT have any
side effects.

\drivernormative{\paragraph}{Vendor data capability}{Virtio
Transport Options / Virtio Over PCI Bus / PCI Device Layout /
Vendor data capability}

The driver SHOULD NOT use the Vendor data capability except
for debugging and reporting purposes.

The driver MUST qualify the \field{vendor_id} before
interpreting or writing into the Vendor data capability.

\subsubsection{PCI configuration access capability}\label{sec:Virtio Transport Options / Virtio Over PCI Bus / PCI Device Layout / PCI configuration access capability}

The VIRTIO_PCI_CAP_PCI_CFG capability
creates an alternative (and likely suboptimal) access method to the
common configuration, notification, ISR and device-specific configuration regions.

The capability is immediately followed by an additional field like so:

\begin{lstlisting}
struct virtio_pci_cfg_cap {
        struct virtio_pci_cap cap;
        u8 pci_cfg_data[4]; /* Data for BAR access. */
};
\end{lstlisting}

The fields \field{cap.bar}, \field{cap.length}, \field{cap.offset} and
\field{pci_cfg_data} are read-write (RW) for the driver.

To access a device region, the driver writes into the capability
structure (ie. within the PCI configuration space) as follows:

\begin{itemize}
\item The driver sets the BAR to access by writing to \field{cap.bar}.

\item The driver sets the size of the access by writing 1, 2 or 4 to
  \field{cap.length}.

\item The driver sets the offset within the BAR by writing to
  \field{cap.offset}.
\end{itemize}

At that point, \field{pci_cfg_data} will provide a window of size
\field{cap.length} into the given \field{cap.bar} at offset \field{cap.offset}.

\devicenormative{\paragraph}{PCI configuration access capability}{Virtio Transport Options / Virtio Over PCI Bus / PCI Device Layout / PCI configuration access capability}

The device MUST present at least one VIRTIO_PCI_CAP_PCI_CFG capability.

Upon detecting driver write access
to \field{pci_cfg_data}, the device MUST execute a write access
at offset \field{cap.offset} at BAR selected by \field{cap.bar} using the first \field{cap.length}
bytes from \field{pci_cfg_data}.

Upon detecting driver read access
to \field{pci_cfg_data}, the device MUST
execute a read access of length cap.length at offset \field{cap.offset}
at BAR selected by \field{cap.bar} and store the first \field{cap.length} bytes in
\field{pci_cfg_data}.

\drivernormative{\paragraph}{PCI configuration access capability}{Virtio Transport Options / Virtio Over PCI Bus / PCI Device Layout / PCI configuration access capability}

The driver MUST NOT write a \field{cap.offset} which is not
a multiple of \field{cap.length} (ie. all accesses MUST be aligned).

The driver MUST NOT read or write \field{pci_cfg_data}
unless \field{cap.bar}, \field{cap.length} and \field{cap.offset}
address \field{cap.length} bytes within a BAR range
specified by some other Virtio Structure PCI Capability
of type other than \field{VIRTIO_PCI_CAP_PCI_CFG}.

\subsubsection{Legacy Interfaces: A Note on PCI Device Layout}\label{sec:Virtio Transport Options / Virtio Over PCI Bus / PCI Device Layout / Legacy Interfaces: A Note on PCI Device Layout}

Transitional devices MUST present part of configuration
registers in a legacy configuration structure in BAR0 in the first I/O
region of the PCI device, as documented below.
When using the legacy interface, transitional drivers
MUST use the legacy configuration structure in BAR0 in the first
I/O region of the PCI device, as documented below.

When using the legacy interface the driver MAY access
the device-specific configuration region using any width accesses, and
a transitional device MUST present driver with the same results as
when accessed using the ``natural'' access method (i.e.
32-bit accesses for 32-bit fields, etc).

Note that this is possible because while the virtio common configuration structure is PCI
(i.e. little) endian, when using the legacy interface the device-specific
configuration region is encoded in the native endian of the guest (where such distinction is
applicable).

When used through the legacy interface, the virtio common configuration structure looks as follows:

\begin{tabularx}{\textwidth}{ |X||X|X|X|X|X|X|X|X| }
\hline
 Bits & 32 & 32 & 32 & 16 & 16 & 16 & 8 & 8 \\
\hline
 Read / Write & R & R+W & R+W & R & R+W & R+W & R+W & R \\
\hline
 Purpose & Device Features bits 0:31 & Driver Features bits 0:31 &
  Queue Address & \field{queue_size} & \field{queue_select} & Queue Notify &
  Device Status & ISR \newline Status \\
\hline
\end{tabularx}

If MSI-X is enabled for the device, two additional fields
immediately follow this header:

\begin{tabular}{ |l||l|l| }
\hline
Bits       & 16             & 16     \\
\hline
Read/Write & R+W            & R+W    \\
\hline
Purpose (MSI-X) & \field{config_msix_vector}  & \field{queue_msix_vector} \\
\hline
\end{tabular}

Note: When MSI-X capability is enabled, device-specific configuration starts at
byte offset 24 in virtio common configuration structure. When MSI-X capability is not
enabled, device-specific configuration starts at byte offset 20 in virtio
header.  ie. once you enable MSI-X on the device, the other fields move.
If you turn it off again, they move back!

Any device-specific configuration space immediately follows
these general headers:

\begin{tabular}{|l||l|l|}
\hline
Bits & Device Specific & \multirow{3}{*}{\ldots} \\
\cline{1-2}
Read / Write & Device Specific & \\
\cline{1-2}
Purpose & Device Specific & \\
\hline
\end{tabular}

When accessing the device-specific configuration space
using the legacy interface, transitional
drivers MUST access the device-specific configuration space
at an offset immediately following the general headers.

When using the legacy interface, transitional
devices MUST present the device-specific configuration space
if any at an offset immediately following the general headers.

Note that only Feature Bits 0 to 31 are accessible through the
Legacy Interface. When used through the Legacy Interface,
Transitional Devices MUST assume that Feature Bits 32 to 63
are not acknowledged by Driver.

As legacy devices had no \field{config_generation} field,
see \ref{sec:Basic Facilities of a Virtio Device / Device
Configuration Space / Legacy Interface: Device Configuration
Space}~\nameref{sec:Basic Facilities of a Virtio Device / Device Configuration Space / Legacy Interface: Device Configuration Space} for workarounds.

\subsubsection{Non-transitional Device With Legacy Driver: A Note
on PCI Device Layout}\label{sec:Virtio Transport Options / Virtio
Over PCI Bus / PCI Device Layout / Non-transitional Device With
Legacy Driver: A Note on PCI Device Layout}

All known legacy drivers check either the PCI Revision or the
Device and Vendor IDs, and thus won't attempt to drive a
non-transitional device.

A buggy legacy driver might mistakenly attempt to drive a
non-transitional device. If support for such drivers is required
(as opposed to fixing the bug), the following would be the
recommended way to detect and handle them.
\begin{note}
Such buggy drivers are not currently known to be used in
production.
\end{note}

\subparagraph{Device Requirements: Non-transitional Device With Legacy Driver}
\label{drivernormative:Virtio Transport Options / Virtio Over PCI
Bus / PCI-specific Initialization And Device Operation /
Device Initialization / Non-transitional Device With Legacy
Driver}
\label{devicenormative:Virtio Transport Options / Virtio Over PCI
Bus / PCI-specific Initialization And Device Operation /
Device Initialization / Non-transitional Device With Legacy
Driver}

Non-transitional devices, on a platform where a legacy driver for
a legacy device with the same ID (including PCI Revision, Device
and Vendor IDs) is known to have previously existed,
SHOULD take the following steps to cause the legacy driver to
fail gracefully when it attempts to drive them:

\begin{enumerate}
\item Present an I/O BAR in BAR0, and
\item Respond to a single-byte zero write to offset 18
   (corresponding to Device Status register in the legacy layout)
   of BAR0 by presenting zeroes on every BAR and ignoring writes.
\end{enumerate}

\subsection{PCI-specific Initialization And Device Operation}\label{sec:Virtio Transport Options / Virtio Over PCI Bus / PCI-specific Initialization And Device Operation}

\subsubsection{Device Initialization}\label{sec:Virtio Transport Options / Virtio Over PCI Bus / PCI-specific Initialization And Device Operation / Device Initialization}

This documents PCI-specific steps executed during Device Initialization.

\paragraph{Virtio Device Configuration Layout Detection}\label{sec:Virtio Transport Options / Virtio Over PCI Bus / PCI-specific Initialization And Device Operation / Device Initialization / Virtio Device Configuration Layout Detection}

As a prerequisite to device initialization, the driver scans the
PCI capability list, detecting virtio configuration layout using Virtio
Structure PCI capabilities as detailed in \ref{sec:Virtio Transport Options / Virtio Over PCI Bus / Virtio Structure PCI Capabilities}

\subparagraph{Legacy Interface: A Note on Device Layout Detection}\label{sec:Virtio Transport Options / Virtio Over PCI Bus / PCI-specific Initialization And Device Operation / Device Initialization / Virtio Device Configuration Layout Detection / Legacy Interface: A Note on Device Layout Detection}

Legacy drivers skipped the Device Layout Detection step, assuming legacy
device configuration space in BAR0 in I/O space unconditionally.

Legacy devices did not have the Virtio PCI Capability in their
capability list.

Therefore:

Transitional devices MUST expose the Legacy Interface in I/O
space in BAR0.

Transitional drivers MUST look for the Virtio PCI
Capabilities on the capability list.
If these are not present, driver MUST assume a legacy device,
and use it through the legacy interface.

Non-transitional drivers MUST look for the Virtio PCI
Capabilities on the capability list.
If these are not present, driver MUST assume a legacy device,
and fail gracefully.

\paragraph{MSI-X Vector Configuration}\label{sec:Virtio Transport Options / Virtio Over PCI Bus / PCI-specific Initialization And Device Operation / Device Initialization / MSI-X Vector Configuration}

When MSI-X capability is present and enabled in the device
(through standard PCI configuration space) \field{config_msix_vector} and \field{queue_msix_vector} are used to map configuration change and queue
interrupts to MSI-X vectors. In this case, the ISR Status is unused.

Writing a valid MSI-X Table entry number, 0 to 0x7FF, to
\field{config_msix_vector}/\field{queue_msix_vector} maps interrupts triggered
by the configuration change/selected queue events respectively to
the corresponding MSI-X vector. To disable interrupts for an
event type, the driver unmaps this event by writing a special NO_VECTOR
value:

\begin{lstlisting}
/* Vector value used to disable MSI for queue */
#define VIRTIO_MSI_NO_VECTOR            0xffff
\end{lstlisting}

Note that mapping an event to vector might require device to
allocate internal device resources, and thus could fail.

\devicenormative{\subparagraph}{MSI-X Vector Configuration}{Virtio Transport Options / Virtio Over PCI Bus / PCI-specific Initialization And Device Operation / Device Initialization / MSI-X Vector Configuration}

A device that has an MSI-X capability SHOULD support at least 2
and at most 0x800 MSI-X vectors.
Device MUST report the number of vectors supported in
\field{Table Size} in the MSI-X Capability as specified in
\hyperref[intro:PCI]{[PCI]}.
The device SHOULD restrict the reported MSI-X Table Size field
to a value that might benefit system performance.
\begin{note}
For example, a device which does not expect to send
interrupts at a high rate might only specify 2 MSI-X vectors.
\end{note}
Device MUST support mapping any event type to any valid
vector 0 to MSI-X \field{Table Size}.
Device MUST support unmapping any event type.

The device MUST return vector mapped to a given event,
(NO_VECTOR if unmapped) on read of \field{config_msix_vector}/\field{queue_msix_vector}.
The device MUST have all queue and configuration change
events are unmapped upon reset.

Devices SHOULD NOT cause mapping an event to vector to fail
unless it is impossible for the device to satisfy the mapping
request.  Devices MUST report mapping
failures by returning the NO_VECTOR value when the relevant
\field{config_msix_vector}/\field{queue_msix_vector} field is read.

\drivernormative{\subparagraph}{MSI-X Vector Configuration}{Virtio Transport Options / Virtio Over PCI Bus / PCI-specific Initialization And Device Operation / Device Initialization / MSI-X Vector Configuration}

Driver MUST support device with any MSI-X Table Size 0 to 0x7FF.
Driver MAY fall back on using INT\#x interrupts for a device
which only supports one MSI-X vector (MSI-X Table Size = 0).

Driver MAY interpret the Table Size as a hint from the device
for the suggested number of MSI-X vectors to use.

Driver MUST NOT attempt to map an event to a vector
outside the MSI-X Table supported by the device,
as reported by \field{Table Size} in the MSI-X Capability.

After mapping an event to vector, the
driver MUST verify success by reading the Vector field value: on
success, the previously written value is returned, and on
failure, NO_VECTOR is returned. If a mapping failure is detected,
the driver MAY retry mapping with fewer vectors, disable MSI-X
or report device failure.

\paragraph{Virtqueue Configuration}\label{sec:Virtio Transport Options / Virtio Over PCI Bus / PCI-specific Initialization And Device Operation / Device Initialization / Virtqueue Configuration}

As a device can have zero or more virtqueues for bulk data
transport\footnote{For example, the simplest network device has two virtqueues.}, the driver
needs to configure them as part of the device-specific
configuration.

The driver typically does this as follows, for each virtqueue a device has:

\begin{enumerate}
\item Write the virtqueue index to \field{queue_select}.

\item Read the virtqueue size from \field{queue_size}. This controls how big the virtqueue is
  (see \ref{sec:Basic Facilities of a Virtio Device / Virtqueues}~\nameref{sec:Basic Facilities of a Virtio Device / Virtqueues}). If this field is 0, the virtqueue does not exist.

\item Optionally, select a smaller virtqueue size and write it to \field{queue_size}.

\item Allocate and zero Descriptor Table, Available and Used rings for the
   virtqueue in contiguous physical memory.

\item Optionally, if MSI-X capability is present and enabled on the
  device, select a vector to use to request interrupts triggered
  by virtqueue events. Write the MSI-X Table entry number
  corresponding to this vector into \field{queue_msix_vector}. Read
  \field{queue_msix_vector}: on success, previously written value is
  returned; on failure, NO_VECTOR value is returned.
\end{enumerate}

\subparagraph{Legacy Interface: A Note on Virtqueue Configuration}\label{sec:Virtio Transport Options / Virtio Over PCI Bus / PCI-specific Initialization And Device Operation / Device Initialization / Virtqueue Configuration / Legacy Interface: A Note on Virtqueue Configuration}
When using the legacy interface, the queue layout follows \ref{sec:Basic Facilities of a Virtio Device / Virtqueues / Legacy Interfaces: A Note on Virtqueue Layout}~\nameref{sec:Basic Facilities of a Virtio Device / Virtqueues / Legacy Interfaces: A Note on Virtqueue Layout} with an alignment of 4096.
Driver writes the physical address, divided
by 4096 to the Queue Address field\footnote{The 4096 is based on the x86 page size, but it's also large
enough to ensure that the separate parts of the virtqueue are on
separate cache lines.
}.  There was no mechanism to negotiate the queue size.

\subsubsection{Available Buffer Notifications}\label{sec:Virtio Transport Options / Virtio Over PCI Bus / PCI-specific Initialization And Device Operation / Available Buffer Notifications}

When VIRTIO_F_NOTIFICATION_DATA has not been negotiated,
the driver sends an available buffer notification to the device by writing
only the 16-bit notification value to the Queue Notify address of the
virtqueue. A notification value depends on the negotiation of
VIRTIO_F_NOTIF_CONFIG_DATA.

If VIRTIO_F_NOTIFICATION_DATA has been negotiated, the driver sends an
available buffer notification to the device by writing the following 32-bit
value to the Queue Notify address:
\lstinputlisting{notifications-data-le.c}

\begin{itemize}
\item When VIRTIO_F_NOTIF_CONFIG_DATA is not negotiated \field{vq_index} is set
to the virtqueue index.

\item When VIRTIO_F_NOTIFICATION_DATA is negotiated,
\field{vq_notif_config_data} is set to \field{queue_notif_config_data}.
\end{itemize}

See \ref{sec:Basic Facilities of a Virtio Device / Driver notifications}~\nameref{sec:Basic Facilities of a Virtio Device / Driver notifications}
for the definition of the components.

See \ref{sec:Virtio Transport Options / Virtio Over PCI Bus / PCI Device Layout / Notification capability}
for how to calculate the Queue Notify address.

\drivernormative{\paragraph}{Available Buffer Notifications}{Virtio Transport Options / Virtio Over PCI Bus / PCI-specific Initialization And Device Operation / Available Buffer Notifications}

If VIRTIO_F_NOTIFICATION_DATA is not negotiated, the driver notification
MUST be a 16-bit notification.

If VIRTIO_F_NOTIFICATION_DATA is negotiated, the driver notification
MUST be a 32-bit notification.

If VIRTIO_F_NOTIF_CONFIG_DATA is not negotiated:
\begin{itemize}
\item If VIRTIO_F_NOTIFICATION_DATA is not negotiated, the driver MUST set the
notification value to the virtqueue index.

\item If VIRTIO_F_NOTIFICATION_DATA is negotiated, the driver MUST set the
\field{vq_index} to the virtqueue index.

\end{itemize}

If VIRTIO_F_NOTIF_CONFIG_DATA is negotiated:
\begin{itemize}
\item If VIRTIO_F_NOTIFICATION_DATA is not negotiated, the driver MUST set
the notification value to \field{queue_notif_config_data}.

\item If VIRTIO_F_NOTIFICATION_DATA is negotiated, the driver MUST set the
\field{vq_notify_config_data} to the \field{queue_notif_config_data} value.
\end{itemize}

\subsubsection{Used Buffer Notifications}\label{sec:Virtio Transport Options / Virtio Over PCI Bus / PCI-specific Initialization And Device Operation / Used Buffer Notifications}

If a used buffer notification is necessary for a virtqueue, the device would typically act as follows:

\begin{itemize}
  \item If MSI-X capability is disabled:
    \begin{enumerate}
    \item Set the lower bit of the ISR Status field for the device.

    \item Send the appropriate PCI interrupt for the device.
    \end{enumerate}

  \item If MSI-X capability is enabled:
    \begin{enumerate}
    \item If \field{queue_msix_vector} is not NO_VECTOR,
      request the appropriate MSI-X interrupt message for the
      device, \field{queue_msix_vector} sets the MSI-X Table entry
      number.
    \end{enumerate}
\end{itemize}

\devicenormative{\paragraph}{Used Buffer Notifications}{Virtio Transport Options / Virtio Over PCI Bus / PCI-specific Initialization And Device Operation / Used Buffer Notifications}

If MSI-X capability is enabled and \field{queue_msix_vector} is
NO_VECTOR for a virtqueue, the device MUST NOT deliver an interrupt
for that virtqueue.

\subsubsection{Notification of Device Configuration Changes}\label{sec:Virtio Transport Options / Virtio Over PCI Bus / PCI-specific Initialization And Device Operation / Notification of Device Configuration Changes}

Some virtio PCI devices can change the device configuration
state, as reflected in the device-specific configuration region of the device. In this case:

\begin{itemize}
  \item If MSI-X capability is disabled:
    \begin{enumerate}
    \item Set the second lower bit of the ISR Status field for the device.

    \item Send the appropriate PCI interrupt for the device.
    \end{enumerate}

  \item If MSI-X capability is enabled:
    \begin{enumerate}
    \item If \field{config_msix_vector} is not NO_VECTOR,
      request the appropriate MSI-X interrupt message for the
      device, \field{config_msix_vector} sets the MSI-X Table entry
      number.
    \end{enumerate}
\end{itemize}

A single interrupt MAY indicate both that one or more virtqueue has
been used and that the configuration space has changed.

\devicenormative{\paragraph}{Notification of Device Configuration Changes}{Virtio Transport Options / Virtio Over PCI Bus / PCI-specific Initialization And Device Operation / Notification of Device Configuration Changes}

If MSI-X capability is enabled and \field{config_msix_vector} is
NO_VECTOR, the device MUST NOT deliver an interrupt
for device configuration space changes.

\drivernormative{\paragraph}{Notification of Device Configuration Changes}{Virtio Transport Options / Virtio Over PCI Bus / PCI-specific Initialization And Device Operation / Notification of Device Configuration Changes}

A driver MUST handle the case where the same interrupt is used to indicate
both device configuration space change and one or more virtqueues being used.

\subsubsection{Driver Handling Interrupts}\label{sec:Virtio Transport Options / Virtio Over PCI Bus / PCI-specific Initialization And Device Operation / Driver Handling Interrupts}
The driver interrupt handler would typically:

\begin{itemize}
  \item If MSI-X capability is disabled:
    \begin{itemize}
      \item Read the ISR Status field, which will reset it to zero.
      \item If the lower bit is set:
        look through all virtqueues for the
        device, to see if any progress has been made by the device
        which requires servicing.
      \item If the second lower bit is set:
        re-examine the configuration space to see what changed.
    \end{itemize}
  \item If MSI-X capability is enabled:
    \begin{itemize}
      \item
        Look through all virtqueues mapped to that MSI-X vector for the
        device, to see if any progress has been made by the device
        which requires servicing.
      \item
        If the MSI-X vector is equal to \field{config_msix_vector},
        re-examine the configuration space to see what changed.
    \end{itemize}
\end{itemize}

\input{transport-mmio.tex}
\input{transport-ccw.tex}

\chapter{Device Types}\label{sec:Device Types}

On top of the queues, config space and feature negotiation facilities
built into virtio, several devices are defined.

The following device IDs are used to identify different types of virtio
devices.  Some device IDs are reserved for devices which are not currently
defined in this standard.

Discovering what devices are available and their type is bus-dependent.

\begin{longtable} { |l|c| }
\hline
Device ID  &  Virtio Device    \\
\hline \hline
0          & reserved (invalid) \\
\hline
1          &   network device     \\
\hline
2          &   block device     \\
\hline
3          &      console       \\
\hline
4          &  entropy source    \\
\hline
5          & memory ballooning (traditional)  \\
\hline
6          &     ioMemory       \\
\hline
7          &       rpmsg        \\
\hline
8          &     SCSI host      \\
\hline
9          &   9P transport     \\
\hline
10         &   mac80211 wlan    \\
\hline
11         &   rproc serial     \\
\hline
12         &   virtio CAIF      \\
\hline
13         &  memory balloon    \\
\hline
16         &   GPU device       \\
\hline
17         &   Timer/Clock device \\
\hline
18         &   Input device \\
\hline
19         &   Socket device \\
\hline
20         &   Crypto device \\
\hline
21         &   Signal Distribution Module \\
\hline
22         &   pstore device \\
\hline
23         &   IOMMU device \\
\hline
24         &   Memory device \\
\hline
25         &   Sound device \\
\hline
26         &   file system device \\
\hline
27         &   PMEM device \\
\hline
28         &   RPMB device \\
\hline
29         &   mac80211 hwsim wireless simulation device \\
\hline
30         &   Video encoder device \\
\hline
31         &   Video decoder device \\
\hline
32         &   SCMI device \\
\hline
33         &   NitroSecureModule \\
\hline
34         &   I2C adapter \\
\hline
35         &   Watchdog \\
\hline
36         &   CAN device \\
\hline
38         &   Parameter Server \\
\hline
39         &   Audio policy device \\
\hline
40         &   Bluetooth device \\
\hline
41         &   GPIO device \\
\hline
42         &   RDMA device \\
\hline
43         &   Camera device \\
\hline
44         &   ISM device \\
\hline
45         &   SPI controller \\
\hline
46         &   TEE device \\
\hline
47         &   CPU balloon device \\
\hline
48         &   Media device \\
\hline
49         &   USB controller \\
\hline
\end{longtable}

Some of the devices above are unspecified by this document,
because they are seen as immature or especially niche.  Be warned
that some are only specified by the sole existing implementation;
they could become part of a future specification, be abandoned
entirely, or live on outside this standard.  We shall speak of
them no further.

\section{Network Device}\label{sec:Device Types / Network Device}

The virtio network device is a virtual network interface controller.
It consists of a virtual Ethernet link which connects the device
to the Ethernet network. The device has transmit and receive
queues. The driver adds empty buffers to the receive virtqueue.
The device receives incoming packets from the link; the device
places these incoming packets in the receive virtqueue buffers.
The driver adds outgoing packets to the transmit virtqueue. The device
removes these packets from the transmit virtqueue and sends them to
the link. The device may have a control virtqueue. The driver
uses the control virtqueue to dynamically manipulate various
features of the initialized device.

\subsection{Device ID}\label{sec:Device Types / Network Device / Device ID}

 1

\subsection{Virtqueues}\label{sec:Device Types / Network Device / Virtqueues}

\begin{description}
\item[0] receiveq1
\item[1] transmitq1
\item[\ldots]
\item[2(N-1)] receiveqN
\item[2(N-1)+1] transmitqN
\item[2N] controlq
\end{description}

 N=1 if neither VIRTIO_NET_F_MQ nor VIRTIO_NET_F_RSS are negotiated, otherwise N is set by
 \field{max_virtqueue_pairs}.

controlq is optional; it only exists if VIRTIO_NET_F_CTRL_VQ is
negotiated.

\subsection{Feature bits}\label{sec:Device Types / Network Device / Feature bits}

\begin{description}
\item[VIRTIO_NET_F_CSUM (0)] Device handles packets with partial checksum offload.

\item[VIRTIO_NET_F_GUEST_CSUM (1)] Driver handles packets with partial checksum.

\item[VIRTIO_NET_F_CTRL_GUEST_OFFLOADS (2)] Control channel offloads
        reconfiguration support.

\item[VIRTIO_NET_F_MTU(3)] Device maximum MTU reporting is supported. If
    offered by the device, device advises driver about the value of
    its maximum MTU. If negotiated, the driver uses \field{mtu} as
    the maximum MTU value.

\item[VIRTIO_NET_F_MAC (5)] Device has given MAC address.

\item[VIRTIO_NET_F_GUEST_TSO4 (7)] Driver can receive TSOv4.

\item[VIRTIO_NET_F_GUEST_TSO6 (8)] Driver can receive TSOv6.

\item[VIRTIO_NET_F_GUEST_ECN (9)] Driver can receive TSO with ECN.

\item[VIRTIO_NET_F_GUEST_UFO (10)] Driver can receive UFO.

\item[VIRTIO_NET_F_HOST_TSO4 (11)] Device can receive TSOv4.

\item[VIRTIO_NET_F_HOST_TSO6 (12)] Device can receive TSOv6.

\item[VIRTIO_NET_F_HOST_ECN (13)] Device can receive TSO with ECN.

\item[VIRTIO_NET_F_HOST_UFO (14)] Device can receive UFO.

\item[VIRTIO_NET_F_MRG_RXBUF (15)] Driver can merge receive buffers.

\item[VIRTIO_NET_F_STATUS (16)] Configuration status field is
    available.

\item[VIRTIO_NET_F_CTRL_VQ (17)] Control channel is available.

\item[VIRTIO_NET_F_CTRL_RX (18)] Control channel RX mode support.

\item[VIRTIO_NET_F_CTRL_VLAN (19)] Control channel VLAN filtering.

\item[VIRTIO_NET_F_CTRL_RX_EXTRA (20)]	Control channel RX extra mode support.

\item[VIRTIO_NET_F_GUEST_ANNOUNCE(21)] Driver can send gratuitous
    packets.

\item[VIRTIO_NET_F_MQ(22)] Device supports multiqueue with automatic
    receive steering.

\item[VIRTIO_NET_F_CTRL_MAC_ADDR(23)] Set MAC address through control
    channel.

\item[VIRTIO_NET_F_DEVICE_STATS(50)] Device can provide device-level statistics
    to the driver through the control virtqueue.

\item[VIRTIO_NET_F_HASH_TUNNEL(51)] Device supports inner header hash for encapsulated packets.

\item[VIRTIO_NET_F_VQ_NOTF_COAL(52)] Device supports virtqueue notification coalescing.

\item[VIRTIO_NET_F_NOTF_COAL(53)] Device supports notifications coalescing.

\item[VIRTIO_NET_F_GUEST_USO4 (54)] Driver can receive USOv4 packets.

\item[VIRTIO_NET_F_GUEST_USO6 (55)] Driver can receive USOv6 packets.

\item[VIRTIO_NET_F_HOST_USO (56)] Device can receive USO packets. Unlike UFO
 (fragmenting the packet) the USO splits large UDP packet
 to several segments when each of these smaller packets has UDP header.

\item[VIRTIO_NET_F_HASH_REPORT(57)] Device can report per-packet hash
    value and a type of calculated hash.

\item[VIRTIO_NET_F_GUEST_HDRLEN(59)] Driver can provide the exact \field{hdr_len}
    value. Device benefits from knowing the exact header length.

\item[VIRTIO_NET_F_RSS(60)] Device supports RSS (receive-side scaling)
    with Toeplitz hash calculation and configurable hash
    parameters for receive steering.

\item[VIRTIO_NET_F_RSC_EXT(61)] Device can process duplicated ACKs
    and report number of coalesced segments and duplicated ACKs.

\item[VIRTIO_NET_F_STANDBY(62)] Device may act as a standby for a primary
    device with the same MAC address.

\item[VIRTIO_NET_F_SPEED_DUPLEX(63)] Device reports speed and duplex.

\item[VIRTIO_NET_F_RSS_CONTEXT(64)] Device supports multiple RSS contexts.

\item[VIRTIO_NET_F_GUEST_UDP_TUNNEL_GSO (65)] Driver can receive GSO packets
  carried by a UDP tunnel.

\item[VIRTIO_NET_F_GUEST_UDP_TUNNEL_GSO_CSUM (66)] Driver handles packets
  carried by a UDP tunnel with partial csum for the outer header.

\item[VIRTIO_NET_F_HOST_UDP_TUNNEL_GSO (67)] Device can receive GSO packets
  carried by a UDP tunnel.

\item[VIRTIO_NET_F_HOST_UDP_TUNNEL_GSO_CSUM (68)] Device handles packets
  carried by a UDP tunnel with partial csum for the outer header.
\end{description}

\subsubsection{Feature bit requirements}\label{sec:Device Types / Network Device / Feature bits / Feature bit requirements}

Some networking feature bits require other networking feature bits
(see \ref{drivernormative:Basic Facilities of a Virtio Device / Feature Bits}):

\begin{description}
\item[VIRTIO_NET_F_GUEST_TSO4] Requires VIRTIO_NET_F_GUEST_CSUM.
\item[VIRTIO_NET_F_GUEST_TSO6] Requires VIRTIO_NET_F_GUEST_CSUM.
\item[VIRTIO_NET_F_GUEST_ECN] Requires VIRTIO_NET_F_GUEST_TSO4 or VIRTIO_NET_F_GUEST_TSO6.
\item[VIRTIO_NET_F_GUEST_UFO] Requires VIRTIO_NET_F_GUEST_CSUM.
\item[VIRTIO_NET_F_GUEST_USO4] Requires VIRTIO_NET_F_GUEST_CSUM.
\item[VIRTIO_NET_F_GUEST_USO6] Requires VIRTIO_NET_F_GUEST_CSUM.
\item[VIRTIO_NET_F_GUEST_UDP_TUNNEL_GSO] Requires VIRTIO_NET_F_GUEST_TSO4, VIRTIO_NET_F_GUEST_TSO6,
   VIRTIO_NET_F_GUEST_USO4 and VIRTIO_NET_F_GUEST_USO6.
\item[VIRTIO_NET_F_GUEST_UDP_TUNNEL_GSO_CSUM] Requires VIRTIO_NET_F_GUEST_UDP_TUNNEL_GSO

\item[VIRTIO_NET_F_HOST_TSO4] Requires VIRTIO_NET_F_CSUM.
\item[VIRTIO_NET_F_HOST_TSO6] Requires VIRTIO_NET_F_CSUM.
\item[VIRTIO_NET_F_HOST_ECN] Requires VIRTIO_NET_F_HOST_TSO4 or VIRTIO_NET_F_HOST_TSO6.
\item[VIRTIO_NET_F_HOST_UFO] Requires VIRTIO_NET_F_CSUM.
\item[VIRTIO_NET_F_HOST_USO] Requires VIRTIO_NET_F_CSUM.
\item[VIRTIO_NET_F_HOST_UDP_TUNNEL_GSO] Requires VIRTIO_NET_F_HOST_TSO4, VIRTIO_NET_F_HOST_TSO6
   and VIRTIO_NET_F_HOST_USO.
\item[VIRTIO_NET_F_HOST_UDP_TUNNEL_GSO_CSUM] Requires VIRTIO_NET_F_HOST_UDP_TUNNEL_GSO

\item[VIRTIO_NET_F_CTRL_RX] Requires VIRTIO_NET_F_CTRL_VQ.
\item[VIRTIO_NET_F_CTRL_VLAN] Requires VIRTIO_NET_F_CTRL_VQ.
\item[VIRTIO_NET_F_GUEST_ANNOUNCE] Requires VIRTIO_NET_F_CTRL_VQ.
\item[VIRTIO_NET_F_MQ] Requires VIRTIO_NET_F_CTRL_VQ.
\item[VIRTIO_NET_F_CTRL_MAC_ADDR] Requires VIRTIO_NET_F_CTRL_VQ.
\item[VIRTIO_NET_F_NOTF_COAL] Requires VIRTIO_NET_F_CTRL_VQ.
\item[VIRTIO_NET_F_RSC_EXT] Requires VIRTIO_NET_F_HOST_TSO4 or VIRTIO_NET_F_HOST_TSO6.
\item[VIRTIO_NET_F_RSS] Requires VIRTIO_NET_F_CTRL_VQ.
\item[VIRTIO_NET_F_VQ_NOTF_COAL] Requires VIRTIO_NET_F_CTRL_VQ.
\item[VIRTIO_NET_F_HASH_TUNNEL] Requires VIRTIO_NET_F_CTRL_VQ along with VIRTIO_NET_F_RSS or VIRTIO_NET_F_HASH_REPORT.
\item[VIRTIO_NET_F_RSS_CONTEXT] Requires VIRTIO_NET_F_CTRL_VQ and VIRTIO_NET_F_RSS.
\end{description}

\begin{note}
The dependency between UDP_TUNNEL_GSO_CSUM and UDP_TUNNEL_GSO is intentionally
in the opposite direction with respect to the plain GSO features and the plain
checksum offload because UDP tunnel checksum offload gives very little gain
for non GSO packets and is quite complex to implement in H/W.
\end{note}

\subsubsection{Legacy Interface: Feature bits}\label{sec:Device Types / Network Device / Feature bits / Legacy Interface: Feature bits}
\begin{description}
\item[VIRTIO_NET_F_GSO (6)] Device handles packets with any GSO type. This was supposed to indicate segmentation offload support, but
upon further investigation it became clear that multiple bits were needed.
\item[VIRTIO_NET_F_GUEST_RSC4 (41)] Device coalesces TCPIP v4 packets. This was implemented by hypervisor patch for certification
purposes and current Windows driver depends on it. It will not function if virtio-net device reports this feature.
\item[VIRTIO_NET_F_GUEST_RSC6 (42)] Device coalesces TCPIP v6 packets. Similar to VIRTIO_NET_F_GUEST_RSC4.
\end{description}

\subsection{Device configuration layout}\label{sec:Device Types / Network Device / Device configuration layout}
\label{sec:Device Types / Block Device / Feature bits / Device configuration layout}

The network device has the following device configuration layout.
All of the device configuration fields are read-only for the driver.

\begin{lstlisting}
struct virtio_net_config {
        u8 mac[6];
        le16 status;
        le16 max_virtqueue_pairs;
        le16 mtu;
        le32 speed;
        u8 duplex;
        u8 rss_max_key_size;
        le16 rss_max_indirection_table_length;
        le32 supported_hash_types;
        le32 supported_tunnel_types;
};
\end{lstlisting}

The \field{mac} address field always exists (although it is only
valid if VIRTIO_NET_F_MAC is set).

The \field{status} only exists if VIRTIO_NET_F_STATUS is set.
Two bits are currently defined for the status field: VIRTIO_NET_S_LINK_UP
and VIRTIO_NET_S_ANNOUNCE.

\begin{lstlisting}
#define VIRTIO_NET_S_LINK_UP     1
#define VIRTIO_NET_S_ANNOUNCE    2
\end{lstlisting}

The following field, \field{max_virtqueue_pairs} only exists if
VIRTIO_NET_F_MQ or VIRTIO_NET_F_RSS is set. This field specifies the maximum number
of each of transmit and receive virtqueues (receiveq1\ldots receiveqN
and transmitq1\ldots transmitqN respectively) that can be configured once at least one of these features
is negotiated.

The following field, \field{mtu} only exists if VIRTIO_NET_F_MTU
is set. This field specifies the maximum MTU for the driver to
use.

The following two fields, \field{speed} and \field{duplex}, only
exist if VIRTIO_NET_F_SPEED_DUPLEX is set.

\field{speed} contains the device speed, in units of 1 MBit per
second, 0 to 0x7fffffff, or 0xffffffff for unknown speed.

\field{duplex} has the values of 0x01 for full duplex, 0x00 for
half duplex and 0xff for unknown duplex state.

Both \field{speed} and \field{duplex} can change, thus the driver
is expected to re-read these values after receiving a
configuration change notification.

The following field, \field{rss_max_key_size} only exists if VIRTIO_NET_F_RSS or VIRTIO_NET_F_HASH_REPORT is set.
It specifies the maximum supported length of RSS key in bytes.

The following field, \field{rss_max_indirection_table_length} only exists if VIRTIO_NET_F_RSS is set.
It specifies the maximum number of 16-bit entries in RSS indirection table.

The next field, \field{supported_hash_types} only exists if the device supports hash calculation,
i.e. if VIRTIO_NET_F_RSS or VIRTIO_NET_F_HASH_REPORT is set.

Field \field{supported_hash_types} contains the bitmask of supported hash types.
See \ref{sec:Device Types / Network Device / Device Operation / Processing of Incoming Packets / Hash calculation for incoming packets / Supported/enabled hash types} for details of supported hash types.

Field \field{supported_tunnel_types} only exists if the device supports inner header hash, i.e. if VIRTIO_NET_F_HASH_TUNNEL is set.

Field \field{supported_tunnel_types} contains the bitmask of encapsulation types supported by the device for inner header hash.
Encapsulation types are defined in \ref{sec:Device Types / Network Device / Device Operation / Processing of Incoming Packets /
Hash calculation for incoming packets / Encapsulation types supported/enabled for inner header hash}.

\devicenormative{\subsubsection}{Device configuration layout}{Device Types / Network Device / Device configuration layout}

The device MUST set \field{max_virtqueue_pairs} to between 1 and 0x8000 inclusive,
if it offers VIRTIO_NET_F_MQ.

The device MUST set \field{mtu} to between 68 and 65535 inclusive,
if it offers VIRTIO_NET_F_MTU.

The device SHOULD set \field{mtu} to at least 1280, if it offers
VIRTIO_NET_F_MTU.

The device MUST NOT modify \field{mtu} once it has been set.

The device MUST NOT pass received packets that exceed \field{mtu} (plus low
level ethernet header length) size with \field{gso_type} NONE or ECN
after VIRTIO_NET_F_MTU has been successfully negotiated.

The device MUST forward transmitted packets of up to \field{mtu} (plus low
level ethernet header length) size with \field{gso_type} NONE or ECN, and do
so without fragmentation, after VIRTIO_NET_F_MTU has been successfully
negotiated.

The device MUST set \field{rss_max_key_size} to at least 40, if it offers
VIRTIO_NET_F_RSS or VIRTIO_NET_F_HASH_REPORT.

The device MUST set \field{rss_max_indirection_table_length} to at least 128, if it offers
VIRTIO_NET_F_RSS.

If the driver negotiates the VIRTIO_NET_F_STANDBY feature, the device MAY act
as a standby device for a primary device with the same MAC address.

If VIRTIO_NET_F_SPEED_DUPLEX has been negotiated, \field{speed}
MUST contain the device speed, in units of 1 MBit per second, 0 to
0x7ffffffff, or 0xfffffffff for unknown.

If VIRTIO_NET_F_SPEED_DUPLEX has been negotiated, \field{duplex}
MUST have the values of 0x00 for full duplex, 0x01 for half
duplex, or 0xff for unknown.

If VIRTIO_NET_F_SPEED_DUPLEX and VIRTIO_NET_F_STATUS have both
been negotiated, the device SHOULD NOT change the \field{speed} and
\field{duplex} fields as long as VIRTIO_NET_S_LINK_UP is set in
the \field{status}.

The device SHOULD NOT offer VIRTIO_NET_F_HASH_REPORT if it
does not offer VIRTIO_NET_F_CTRL_VQ.

The device SHOULD NOT offer VIRTIO_NET_F_CTRL_RX_EXTRA if it
does not offer VIRTIO_NET_F_CTRL_VQ.

\drivernormative{\subsubsection}{Device configuration layout}{Device Types / Network Device / Device configuration layout}

The driver MUST NOT write to any of the device configuration fields.

A driver SHOULD negotiate VIRTIO_NET_F_MAC if the device offers it.
If the driver negotiates the VIRTIO_NET_F_MAC feature, the driver MUST set
the physical address of the NIC to \field{mac}.  Otherwise, it SHOULD
use a locally-administered MAC address (see \hyperref[intro:IEEE 802]{IEEE 802},
``9.2 48-bit universal LAN MAC addresses'').

If the driver does not negotiate the VIRTIO_NET_F_STATUS feature, it SHOULD
assume the link is active, otherwise it SHOULD read the link status from
the bottom bit of \field{status}.

A driver SHOULD negotiate VIRTIO_NET_F_MTU if the device offers it.

If the driver negotiates VIRTIO_NET_F_MTU, it MUST supply enough receive
buffers to receive at least one receive packet of size \field{mtu} (plus low
level ethernet header length) with \field{gso_type} NONE or ECN.

If the driver negotiates VIRTIO_NET_F_MTU, it MUST NOT transmit packets of
size exceeding the value of \field{mtu} (plus low level ethernet header length)
with \field{gso_type} NONE or ECN.

A driver SHOULD negotiate the VIRTIO_NET_F_STANDBY feature if the device offers it.

If VIRTIO_NET_F_SPEED_DUPLEX has been negotiated,
the driver MUST treat any value of \field{speed} above
0x7fffffff as well as any value of \field{duplex} not
matching 0x00 or 0x01 as an unknown value.

If VIRTIO_NET_F_SPEED_DUPLEX has been negotiated, the driver
SHOULD re-read \field{speed} and \field{duplex} after a
configuration change notification.

A driver SHOULD NOT negotiate VIRTIO_NET_F_HASH_REPORT if it
does not negotiate VIRTIO_NET_F_CTRL_VQ.

A driver SHOULD NOT negotiate VIRTIO_NET_F_CTRL_RX_EXTRA if it
does not negotiate VIRTIO_NET_F_CTRL_VQ.

\subsubsection{Legacy Interface: Device configuration layout}\label{sec:Device Types / Network Device / Device configuration layout / Legacy Interface: Device configuration layout}
\label{sec:Device Types / Block Device / Feature bits / Device configuration layout / Legacy Interface: Device configuration layout}
When using the legacy interface, transitional devices and drivers
MUST format \field{status} and
\field{max_virtqueue_pairs} in struct virtio_net_config
according to the native endian of the guest rather than
(necessarily when not using the legacy interface) little-endian.

When using the legacy interface, \field{mac} is driver-writable
which provided a way for drivers to update the MAC without
negotiating VIRTIO_NET_F_CTRL_MAC_ADDR.

\subsection{Device Initialization}\label{sec:Device Types / Network Device / Device Initialization}

A driver would perform a typical initialization routine like so:

\begin{enumerate}
\item Identify and initialize the receive and
  transmission virtqueues, up to N of each kind. If
  VIRTIO_NET_F_MQ feature bit is negotiated,
  N=\field{max_virtqueue_pairs}, otherwise identify N=1.

\item If the VIRTIO_NET_F_CTRL_VQ feature bit is negotiated,
  identify the control virtqueue.

\item Fill the receive queues with buffers: see \ref{sec:Device Types / Network Device / Device Operation / Setting Up Receive Buffers}.

\item Even with VIRTIO_NET_F_MQ, only receiveq1, transmitq1 and
  controlq are used by default.  The driver would send the
  VIRTIO_NET_CTRL_MQ_VQ_PAIRS_SET command specifying the
  number of the transmit and receive queues to use.

\item If the VIRTIO_NET_F_MAC feature bit is set, the configuration
  space \field{mac} entry indicates the ``physical'' address of the
  device, otherwise the driver would typically generate a random
  local MAC address.

\item If the VIRTIO_NET_F_STATUS feature bit is negotiated, the link
  status comes from the bottom bit of \field{status}.
  Otherwise, the driver assumes it's active.

\item A performant driver would indicate that it will generate checksumless
  packets by negotiating the VIRTIO_NET_F_CSUM feature.

\item If that feature is negotiated, a driver can use TCP segmentation or UDP
  segmentation/fragmentation offload by negotiating the VIRTIO_NET_F_HOST_TSO4 (IPv4
  TCP), VIRTIO_NET_F_HOST_TSO6 (IPv6 TCP), VIRTIO_NET_F_HOST_UFO
  (UDP fragmentation) and VIRTIO_NET_F_HOST_USO (UDP segmentation) features.

\item If the VIRTIO_NET_F_HOST_TSO6, VIRTIO_NET_F_HOST_TSO4 and VIRTIO_NET_F_HOST_USO
  segmentation features are negotiated, a driver can
  use TCP segmentation or UDP segmentation on top of UDP encapsulation
  offload, when the outer header does not require checksumming - e.g.
  the outer UDP checksum is zero - by negotiating the
  VIRTIO_NET_F_HOST_UDP_TUNNEL_GSO feature.
  GSO over UDP tunnels packets carry two sets of headers: the outer ones
  and the inner ones. The outer transport protocol is UDP, the inner
  could be either TCP or UDP. Only a single level of encapsulation
  offload is supported.

\item If VIRTIO_NET_F_HOST_UDP_TUNNEL_GSO is negotiated, a driver can
  additionally use TCP segmentation or UDP segmentation on top of UDP
  encapsulation with the outer header requiring checksum offload,
  negotiating the VIRTIO_NET_F_HOST_UDP_TUNNEL_GSO_CSUM feature.

\item The converse features are also available: a driver can save
  the virtual device some work by negotiating these features.\note{For example, a network packet transported between two guests on
the same system might not need checksumming at all, nor segmentation,
if both guests are amenable.}
   The VIRTIO_NET_F_GUEST_CSUM feature indicates that partially
  checksummed packets can be received, and if it can do that then
  the VIRTIO_NET_F_GUEST_TSO4, VIRTIO_NET_F_GUEST_TSO6,
  VIRTIO_NET_F_GUEST_UFO, VIRTIO_NET_F_GUEST_ECN, VIRTIO_NET_F_GUEST_USO4,
  VIRTIO_NET_F_GUEST_USO6 VIRTIO_NET_F_GUEST_UDP_TUNNEL_GSO and
  VIRTIO_NET_F_GUEST_UDP_TUNNEL_GSO_CSUM are the input equivalents of
  the features described above.
  See \ref{sec:Device Types / Network Device / Device Operation /
Setting Up Receive Buffers}~\nameref{sec:Device Types / Network
Device / Device Operation / Setting Up Receive Buffers} and
\ref{sec:Device Types / Network Device / Device Operation /
Processing of Incoming Packets}~\nameref{sec:Device Types /
Network Device / Device Operation / Processing of Incoming Packets} below.
\end{enumerate}

A truly minimal driver would only accept VIRTIO_NET_F_MAC and ignore
everything else.

\subsection{Device and driver capabilities}\label{sec:Device Types / Network Device / Device and driver capabilities}

The network device has the following capabilities.

\begin{tabularx}{\textwidth}{ |l||l|X| }
\hline
Identifier & Name & Description \\
\hline \hline
0x0800 & \hyperref[par:Device Types / Network Device / Device Operation / Flow filter / Device and driver capabilities / VIRTIO-NET-FF-RESOURCE-CAP]{VIRTIO_NET_FF_RESOURCE_CAP} & Flow filter resource capability \\
\hline
0x0801 & \hyperref[par:Device Types / Network Device / Device Operation / Flow filter / Device and driver capabilities / VIRTIO-NET-FF-SELECTOR-CAP]{VIRTIO_NET_FF_SELECTOR_CAP} & Flow filter classifier capability \\
\hline
0x0802 & \hyperref[par:Device Types / Network Device / Device Operation / Flow filter / Device and driver capabilities / VIRTIO-NET-FF-ACTION-CAP]{VIRTIO_NET_FF_ACTION_CAP} & Flow filter action capability \\
\hline
\end{tabularx}

\subsection{Device resource objects}\label{sec:Device Types / Network Device / Device resource objects}

The network device has the following resource objects.

\begin{tabularx}{\textwidth}{ |l||l|X| }
\hline
type & Name & Description \\
\hline \hline
0x0200 & \hyperref[par:Device Types / Network Device / Device Operation / Flow filter / Resource objects / VIRTIO-NET-RESOURCE-OBJ-FF-GROUP]{VIRTIO_NET_RESOURCE_OBJ_FF_GROUP} & Flow filter group resource object \\
\hline
0x0201 & \hyperref[par:Device Types / Network Device / Device Operation / Flow filter / Resource objects / VIRTIO-NET-RESOURCE-OBJ-FF-CLASSIFIER]{VIRTIO_NET_RESOURCE_OBJ_FF_CLASSIFIER} & Flow filter mask object \\
\hline
0x0202 & \hyperref[par:Device Types / Network Device / Device Operation / Flow filter / Resource objects / VIRTIO-NET-RESOURCE-OBJ-FF-RULE]{VIRTIO_NET_RESOURCE_OBJ_FF_RULE} & Flow filter rule object \\
\hline
\end{tabularx}

\subsection{Device parts}\label{sec:Device Types / Network Device / Device parts}

Network device parts represent the configuration done by the driver using control
virtqueue commands. Network device part is in the format of
\field{struct virtio_dev_part}.

\begin{tabularx}{\textwidth}{ |l||l|X| }
\hline
Type & Name & Description \\
\hline \hline
0x200 & VIRTIO_NET_DEV_PART_CVQ_CFG_PART & Represents device configuration done through a control virtqueue command, see \ref{sec:Device Types / Network Device / Device parts / VIRTIO-NET-DEV-PART-CVQ-CFG-PART} \\
\hline
0x201 - 0x5FF & - & reserved for future \\
\hline
\hline
\end{tabularx}

\subsubsection{VIRTIO_NET_DEV_PART_CVQ_CFG_PART}\label{sec:Device Types / Network Device / Device parts / VIRTIO-NET-DEV-PART-CVQ-CFG-PART}

For VIRTIO_NET_DEV_PART_CVQ_CFG_PART, \field{part_type} is set to 0x200. The
VIRTIO_NET_DEV_PART_CVQ_CFG_PART part indicates configuration performed by the
driver using a control virtqueue command.

\begin{lstlisting}
struct virtio_net_dev_part_cvq_selector {
        u8 class;
        u8 command;
        u8 reserved[6];
};
\end{lstlisting}

There is one device part of type VIRTIO_NET_DEV_PART_CVQ_CFG_PART for each
individual configuration. Each part is identified by a unique selector value.
The selector, \field{device_type_raw}, is in the format
\field{struct virtio_net_dev_part_cvq_selector}.

The selector consists of two fields: \field{class} and \field{command}. These
fields correspond to the \field{class} and \field{command} defined in
\field{struct virtio_net_ctrl}, as described in the relevant sections of
\ref{sec:Device Types / Network Device / Device Operation / Control Virtqueue}.

The value corresponding to each part’s selector follows the same format as the
respective \field{command-specific-data} described in the relevant sections of
\ref{sec:Device Types / Network Device / Device Operation / Control Virtqueue}.

For example, when the \field{class} is VIRTIO_NET_CTRL_MAC, the \field{command}
can be either VIRTIO_NET_CTRL_MAC_TABLE_SET or VIRTIO_NET_CTRL_MAC_ADDR_SET;
when \field{command} is set to VIRTIO_NET_CTRL_MAC_TABLE_SET, \field{value}
is in the format of \field{struct virtio_net_ctrl_mac}.

Supported selectors are listed in the table:

\begin{tabularx}{\textwidth}{ |l|X| }
\hline
Class selector & Command selector \\
\hline \hline
VIRTIO_NET_CTRL_RX & VIRTIO_NET_CTRL_RX_PROMISC \\
\hline
VIRTIO_NET_CTRL_RX & VIRTIO_NET_CTRL_RX_ALLMULTI \\
\hline
VIRTIO_NET_CTRL_RX & VIRTIO_NET_CTRL_RX_ALLUNI \\
\hline
VIRTIO_NET_CTRL_RX & VIRTIO_NET_CTRL_RX_NOMULTI \\
\hline
VIRTIO_NET_CTRL_RX & VIRTIO_NET_CTRL_RX_NOUNI \\
\hline
VIRTIO_NET_CTRL_RX & VIRTIO_NET_CTRL_RX_NOBCAST \\
\hline
VIRTIO_NET_CTRL_MAC & VIRTIO_NET_CTRL_MAC_TABLE_SET \\
\hline
VIRTIO_NET_CTRL_MAC & VIRTIO_NET_CTRL_MAC_ADDR_SET \\
\hline
VIRTIO_NET_CTRL_VLAN & VIRTIO_NET_CTRL_VLAN_ADD \\
\hline
VIRTIO_NET_CTRL_ANNOUNCE & VIRTIO_NET_CTRL_ANNOUNCE_ACK \\
\hline
VIRTIO_NET_CTRL_MQ & VIRTIO_NET_CTRL_MQ_VQ_PAIRS_SET \\
\hline
VIRTIO_NET_CTRL_MQ & VIRTIO_NET_CTRL_MQ_RSS_CONFIG \\
\hline
VIRTIO_NET_CTRL_MQ & VIRTIO_NET_CTRL_MQ_HASH_CONFIG \\
\hline
\hline
\end{tabularx}

For command selector VIRTIO_NET_CTRL_VLAN_ADD, device part consists of a whole
VLAN table.

\field{reserved} is reserved and set to zero.

\subsection{Device Operation}\label{sec:Device Types / Network Device / Device Operation}

Packets are transmitted by placing them in the
transmitq1\ldots transmitqN, and buffers for incoming packets are
placed in the receiveq1\ldots receiveqN. In each case, the packet
itself is preceded by a header:

\begin{lstlisting}
struct virtio_net_hdr {
#define VIRTIO_NET_HDR_F_NEEDS_CSUM    1
#define VIRTIO_NET_HDR_F_DATA_VALID    2
#define VIRTIO_NET_HDR_F_RSC_INFO      4
#define VIRTIO_NET_HDR_F_UDP_TUNNEL_CSUM 8
        u8 flags;
#define VIRTIO_NET_HDR_GSO_NONE        0
#define VIRTIO_NET_HDR_GSO_TCPV4       1
#define VIRTIO_NET_HDR_GSO_UDP         3
#define VIRTIO_NET_HDR_GSO_TCPV6       4
#define VIRTIO_NET_HDR_GSO_UDP_L4      5
#define VIRTIO_NET_HDR_GSO_UDP_TUNNEL_IPV4 0x20
#define VIRTIO_NET_HDR_GSO_UDP_TUNNEL_IPV6 0x40
#define VIRTIO_NET_HDR_GSO_ECN      0x80
        u8 gso_type;
        le16 hdr_len;
        le16 gso_size;
        le16 csum_start;
        le16 csum_offset;
        le16 num_buffers;
        le32 hash_value;        (Only if VIRTIO_NET_F_HASH_REPORT negotiated)
        le16 hash_report;       (Only if VIRTIO_NET_F_HASH_REPORT negotiated)
        le16 padding_reserved;  (Only if VIRTIO_NET_F_HASH_REPORT negotiated)
        le16 outer_th_offset    (Only if VIRTIO_NET_F_HOST_UDP_TUNNEL_GSO or VIRTIO_NET_F_GUEST_UDP_TUNNEL_GSO negotiated)
        le16 inner_nh_offset;   (Only if VIRTIO_NET_F_HOST_UDP_TUNNEL_GSO or VIRTIO_NET_F_GUEST_UDP_TUNNEL_GSO negotiated)
};
\end{lstlisting}

The controlq is used to control various device features described further in
section \ref{sec:Device Types / Network Device / Device Operation / Control Virtqueue}.

\subsubsection{Legacy Interface: Device Operation}\label{sec:Device Types / Network Device / Device Operation / Legacy Interface: Device Operation}
When using the legacy interface, transitional devices and drivers
MUST format the fields in \field{struct virtio_net_hdr}
according to the native endian of the guest rather than
(necessarily when not using the legacy interface) little-endian.

The legacy driver only presented \field{num_buffers} in the \field{struct virtio_net_hdr}
when VIRTIO_NET_F_MRG_RXBUF was negotiated; without that feature the
structure was 2 bytes shorter.

When using the legacy interface, the driver SHOULD ignore the
used length for the transmit queues
and the controlq queue.
\begin{note}
Historically, some devices put
the total descriptor length there, even though no data was
actually written.
\end{note}

\subsubsection{Packet Transmission}\label{sec:Device Types / Network Device / Device Operation / Packet Transmission}

Transmitting a single packet is simple, but varies depending on
the different features the driver negotiated.

\begin{enumerate}
\item The driver can send a completely checksummed packet.  In this case,
  \field{flags} will be zero, and \field{gso_type} will be VIRTIO_NET_HDR_GSO_NONE.

\item If the driver negotiated VIRTIO_NET_F_CSUM, it can skip
  checksumming the packet:
  \begin{itemize}
  \item \field{flags} has the VIRTIO_NET_HDR_F_NEEDS_CSUM set,

  \item \field{csum_start} is set to the offset within the packet to begin checksumming,
    and

  \item \field{csum_offset} indicates how many bytes after the csum_start the
    new (16 bit ones' complement) checksum is placed by the device.

  \item The TCP checksum field in the packet is set to the sum
    of the TCP pseudo header, so that replacing it by the ones'
    complement checksum of the TCP header and body will give the
    correct result.
  \end{itemize}

\begin{note}
For example, consider a partially checksummed TCP (IPv4) packet.
It will have a 14 byte ethernet header and 20 byte IP header
followed by the TCP header (with the TCP checksum field 16 bytes
into that header). \field{csum_start} will be 14+20 = 34 (the TCP
checksum includes the header), and \field{csum_offset} will be 16.
If the given packet has the VIRTIO_NET_HDR_GSO_UDP_TUNNEL_IPV4 bit or the
VIRTIO_NET_HDR_GSO_UDP_TUNNEL_IPV6 bit set,
the above checksum fields refer to the inner header checksum, see
the example below.
\end{note}

\item If the driver negotiated
  VIRTIO_NET_F_HOST_TSO4, TSO6, USO or UFO, and the packet requires
  TCP segmentation, UDP segmentation or fragmentation, then \field{gso_type}
  is set to VIRTIO_NET_HDR_GSO_TCPV4, TCPV6, UDP_L4 or UDP.
  (Otherwise, it is set to VIRTIO_NET_HDR_GSO_NONE). In this
  case, packets larger than 1514 bytes can be transmitted: the
  metadata indicates how to replicate the packet header to cut it
  into smaller packets. The other gso fields are set:

  \begin{itemize}
  \item If the VIRTIO_NET_F_GUEST_HDRLEN feature has been negotiated,
    \field{hdr_len} indicates the header length that needs to be replicated
    for each packet. It's the number of bytes from the beginning of the packet
    to the beginning of the transport payload.
    If the \field{gso_type} has the VIRTIO_NET_HDR_GSO_UDP_TUNNEL_IPV4 bit or
    VIRTIO_NET_HDR_GSO_UDP_TUNNEL_IPV6 bit set, \field{hdr_len} accounts for
    all the headers up to and including the inner transport.
    Otherwise, if the VIRTIO_NET_F_GUEST_HDRLEN feature has not been negotiated,
    \field{hdr_len} is a hint to the device as to how much of the header
    needs to be kept to copy into each packet, usually set to the
    length of the headers, including the transport header\footnote{Due to various bugs in implementations, this field is not useful
as a guarantee of the transport header size.
}.

  \begin{note}
  Some devices benefit from knowledge of the exact header length.
  \end{note}

  \item \field{gso_size} is the maximum size of each packet beyond that
    header (ie. MSS).

  \item If the driver negotiated the VIRTIO_NET_F_HOST_ECN feature,
    the VIRTIO_NET_HDR_GSO_ECN bit in \field{gso_type}
    indicates that the TCP packet has the ECN bit set\footnote{This case is not handled by some older hardware, so is called out
specifically in the protocol.}.
   \end{itemize}

\item If the driver negotiated the VIRTIO_NET_F_HOST_UDP_TUNNEL_GSO feature and the
  \field{gso_type} has the VIRTIO_NET_HDR_GSO_UDP_TUNNEL_IPV4 bit or
  VIRTIO_NET_HDR_GSO_UDP_TUNNEL_IPV6 bit set, the GSO protocol is encapsulated
  in a UDP tunnel.
  If the outer UDP header requires checksumming, the driver must have
  additionally negotiated the VIRTIO_NET_F_HOST_UDP_TUNNEL_GSO_CSUM feature
  and offloaded the outer checksum accordingly, otherwise
  the outer UDP header must not require checksum validation, i.e. the outer
  UDP checksum must be positive zero (0x0) as defined in UDP RFC 768.
  The other tunnel-related fields indicate how to replicate the packet
  headers to cut it into smaller packets:

  \begin{itemize}
  \item \field{outer_th_offset} field indicates the outer transport header within
      the packet. This field differs from \field{csum_start} as the latter
      points to the inner transport header within the packet.

  \item \field{inner_nh_offset} field indicates the inner network header within
      the packet.
  \end{itemize}

\begin{note}
For example, consider a partially checksummed TCP (IPv4) packet carried over a
Geneve UDP tunnel (again IPv4) with no tunnel options. The
only relevant variable related to the tunnel type is the tunnel header length.
The packet will have a 14 byte outer ethernet header, 20 byte outer IP header
followed by the 8 byte UDP header (with a 0 checksum value), 8 byte Geneve header,
14 byte inner ethernet header, 20 byte inner IP header
and the TCP header (with the TCP checksum field 16 bytes
into that header). \field{csum_start} will be 14+20+8+8+14+20 = 84 (the TCP
checksum includes the header), \field{csum_offset} will be 16.
\field{inner_nh_offset} will be 14+20+8+8+14 = 62, \field{outer_th_offset} will be
14+20+8 = 42 and \field{gso_type} will be
VIRTIO_NET_HDR_GSO_TCPV4 | VIRTIO_NET_HDR_GSO_UDP_TUNNEL_IPV4 = 0x21
\end{note}

\item If the driver negotiated the VIRTIO_NET_F_HOST_UDP_TUNNEL_GSO_CSUM feature,
  the transmitted packet is a GSO one encapsulated in a UDP tunnel, and
  the outer UDP header requires checksumming, the driver can skip checksumming
  the outer header:

  \begin{itemize}
  \item \field{flags} has the VIRTIO_NET_HDR_F_UDP_TUNNEL_CSUM set,

  \item The outer UDP checksum field in the packet is set to the sum
    of the UDP pseudo header, so that replacing it by the ones'
    complement checksum of the outer UDP header and payload will give the
    correct result.
  \end{itemize}

\item \field{num_buffers} is set to zero.  This field is unused on transmitted packets.

\item The header and packet are added as one output descriptor to the
  transmitq, and the device is notified of the new entry
  (see \ref{sec:Device Types / Network Device / Device Initialization}~\nameref{sec:Device Types / Network Device / Device Initialization}).
\end{enumerate}

\drivernormative{\paragraph}{Packet Transmission}{Device Types / Network Device / Device Operation / Packet Transmission}

For the transmit packet buffer, the driver MUST use the size of the
structure \field{struct virtio_net_hdr} same as the receive packet buffer.

The driver MUST set \field{num_buffers} to zero.

If VIRTIO_NET_F_CSUM is not negotiated, the driver MUST set
\field{flags} to zero and SHOULD supply a fully checksummed
packet to the device.

If VIRTIO_NET_F_HOST_TSO4 is negotiated, the driver MAY set
\field{gso_type} to VIRTIO_NET_HDR_GSO_TCPV4 to request TCPv4
segmentation, otherwise the driver MUST NOT set
\field{gso_type} to VIRTIO_NET_HDR_GSO_TCPV4.

If VIRTIO_NET_F_HOST_TSO6 is negotiated, the driver MAY set
\field{gso_type} to VIRTIO_NET_HDR_GSO_TCPV6 to request TCPv6
segmentation, otherwise the driver MUST NOT set
\field{gso_type} to VIRTIO_NET_HDR_GSO_TCPV6.

If VIRTIO_NET_F_HOST_UFO is negotiated, the driver MAY set
\field{gso_type} to VIRTIO_NET_HDR_GSO_UDP to request UDP
fragmentation, otherwise the driver MUST NOT set
\field{gso_type} to VIRTIO_NET_HDR_GSO_UDP.

If VIRTIO_NET_F_HOST_USO is negotiated, the driver MAY set
\field{gso_type} to VIRTIO_NET_HDR_GSO_UDP_L4 to request UDP
segmentation, otherwise the driver MUST NOT set
\field{gso_type} to VIRTIO_NET_HDR_GSO_UDP_L4.

The driver SHOULD NOT send to the device TCP packets requiring segmentation offload
which have the Explicit Congestion Notification bit set, unless the
VIRTIO_NET_F_HOST_ECN feature is negotiated, in which case the
driver MUST set the VIRTIO_NET_HDR_GSO_ECN bit in
\field{gso_type}.

If VIRTIO_NET_F_HOST_UDP_TUNNEL_GSO is negotiated, the driver MAY set
VIRTIO_NET_HDR_GSO_UDP_TUNNEL_IPV4 bit or the VIRTIO_NET_HDR_GSO_UDP_TUNNEL_IPV6 bit
in \field{gso_type} according to the inner network header protocol type
to request GSO packets over UDPv4 or UDPv6 tunnel segmentation,
otherwise the driver MUST NOT set either the
VIRTIO_NET_HDR_GSO_UDP_TUNNEL_IPV4 bit or the VIRTIO_NET_HDR_GSO_UDP_TUNNEL_IPV6 bit
in \field{gso_type}.

When requesting GSO segmentation over UDP tunnel, the driver MUST SET the
VIRTIO_NET_HDR_GSO_UDP_TUNNEL_IPV4 bit if the inner network header is IPv4, i.e. the
packet is a TCPv4 GSO one, otherwise, if the inner network header is IPv6, the driver
MUST SET the VIRTIO_NET_HDR_GSO_UDP_TUNNEL_IPV6 bit.

The driver MUST NOT send to the device GSO packets over UDP tunnel
requiring segmentation and outer UDP checksum offload, unless both the
VIRTIO_NET_F_HOST_UDP_TUNNEL_GSO and VIRTIO_NET_F_HOST_UDP_TUNNEL_GSO_CSUM features
are negotiated, in which case the driver MUST set either the
VIRTIO_NET_HDR_GSO_UDP_TUNNEL_IPV4 bit or the VIRTIO_NET_HDR_GSO_UDP_TUNNEL_IPV6
bit in the \field{gso_type} and the VIRTIO_NET_HDR_F_UDP_TUNNEL_CSUM bit in
the \field{flags}.

If VIRTIO_NET_F_HOST_UDP_TUNNEL_GSO_CSUM is not negotiated, the driver MUST not set
the VIRTIO_NET_HDR_F_UDP_TUNNEL_CSUM bit in the \field{flags} and
MUST NOT send to the device GSO packets over UDP tunnel
requiring segmentation and outer UDP checksum offload.

The driver MUST NOT set the VIRTIO_NET_HDR_GSO_UDP_TUNNEL_IPV4 bit or the
VIRTIO_NET_HDR_GSO_UDP_TUNNEL_IPV6 bit together with VIRTIO_NET_HDR_GSO_UDP, as the
latter is deprecated in favor of UDP_L4 and no new feature will support it.

The driver MUST NOT set the VIRTIO_NET_HDR_GSO_UDP_TUNNEL_IPV4 bit and the
VIRTIO_NET_HDR_GSO_UDP_TUNNEL_IPV6 bit together.

The driver MUST NOT set the VIRTIO_NET_HDR_F_UDP_TUNNEL_CSUM bit \field{flags}
without setting either the VIRTIO_NET_HDR_GSO_UDP_TUNNEL_IPV4 bit or
the VIRTIO_NET_HDR_GSO_UDP_TUNNEL_IPV6 bit in \field{gso_type}.

If the VIRTIO_NET_F_CSUM feature has been negotiated, the
driver MAY set the VIRTIO_NET_HDR_F_NEEDS_CSUM bit in
\field{flags}, if so:
\begin{enumerate}
\item the driver MUST validate the packet checksum at
	offset \field{csum_offset} from \field{csum_start} as well as all
	preceding offsets;
\begin{note}
If \field{gso_type} differs from VIRTIO_NET_HDR_GSO_NONE and the
VIRTIO_NET_HDR_GSO_UDP_TUNNEL_IPV4 bit or the VIRTIO_NET_HDR_GSO_UDP_TUNNEL_IPV6
bit are not set in \field{gso_type}, \field{csum_offset}
points to the only transport header present in the packet, and there are no
additional preceding checksums validated by VIRTIO_NET_HDR_F_NEEDS_CSUM.
\end{note}
\item the driver MUST set the packet checksum stored in the
	buffer to the TCP/UDP pseudo header;
\item the driver MUST set \field{csum_start} and
	\field{csum_offset} such that calculating a ones'
	complement checksum from \field{csum_start} up until the end of
	the packet and storing the result at offset \field{csum_offset}
	from  \field{csum_start} will result in a fully checksummed
	packet;
\end{enumerate}

If none of the VIRTIO_NET_F_HOST_TSO4, TSO6, USO or UFO options have
been negotiated, the driver MUST set \field{gso_type} to
VIRTIO_NET_HDR_GSO_NONE.

If \field{gso_type} differs from VIRTIO_NET_HDR_GSO_NONE, then
the driver MUST also set the VIRTIO_NET_HDR_F_NEEDS_CSUM bit in
\field{flags} and MUST set \field{gso_size} to indicate the
desired MSS.

If one of the VIRTIO_NET_F_HOST_TSO4, TSO6, USO or UFO options have
been negotiated:
\begin{itemize}
\item If the VIRTIO_NET_F_GUEST_HDRLEN feature has been negotiated,
	and \field{gso_type} differs from VIRTIO_NET_HDR_GSO_NONE,
	the driver MUST set \field{hdr_len} to a value equal to the length
	of the headers, including the transport header. If \field{gso_type}
	has the VIRTIO_NET_HDR_GSO_UDP_TUNNEL_IPV4 bit or the
	VIRTIO_NET_HDR_GSO_UDP_TUNNEL_IPV6 bit set, \field{hdr_len} includes
	the inner transport header.

\item If the VIRTIO_NET_F_GUEST_HDRLEN feature has not been negotiated,
	or \field{gso_type} is VIRTIO_NET_HDR_GSO_NONE,
	the driver SHOULD set \field{hdr_len} to a value
	not less than the length of the headers, including the transport
	header.
\end{itemize}

If the VIRTIO_NET_F_HOST_UDP_TUNNEL_GSO option has been negotiated, the
driver MAY set the VIRTIO_NET_HDR_GSO_UDP_TUNNEL_IPV4 bit or the
VIRTIO_NET_HDR_GSO_UDP_TUNNEL_IPV6 bit in \field{gso_type}, if so:
\begin{itemize}
\item the driver MUST set \field{outer_th_offset} to the outer UDP header
  offset and \field{inner_nh_offset} to the inner network header offset.
  The \field{csum_start} and \field{csum_offset} fields point respectively
  to the inner transport header and inner transport checksum field.
\end{itemize}

If the VIRTIO_NET_F_HOST_UDP_TUNNEL_GSO_CSUM feature has been negotiated,
and the VIRTIO_NET_HDR_GSO_UDP_TUNNEL_IPV4 bit or
VIRTIO_NET_HDR_GSO_UDP_TUNNEL_IPV6 bit in \field{gso_type} are set,
the driver MAY set the VIRTIO_NET_HDR_F_UDP_TUNNEL_CSUM bit in
\field{flags}, if so the driver MUST set the packet outer UDP header checksum
to the outer UDP pseudo header checksum.

\begin{note}
calculating a ones' complement checksum from \field{outer_th_offset}
up until the end of the packet and storing the result at offset 6
from \field{outer_th_offset} will result in a fully checksummed outer UDP packet;
\end{note}

If the VIRTIO_NET_HDR_GSO_UDP_TUNNEL_IPV4 bit or the
VIRTIO_NET_HDR_GSO_UDP_TUNNEL_IPV6 bit in \field{gso_type} are set
and the VIRTIO_NET_F_HOST_UDP_TUNNEL_GSO_CSUM feature has not
been negotiated, the
outer UDP header MUST NOT require checksum validation. That is, the
outer UDP checksum value MUST be 0 or the validated complete checksum
for such header.

\begin{note}
The valid complete checksum of the outer UDP header of individual segments
can be computed by the driver prior to segmentation only if the GSO packet
size is a multiple of \field{gso_size}, because then all segments
have the same size and thus all data included in the outer UDP
checksum is the same for every segment. These pre-computed segment
length and checksum fields are different from those of the GSO
packet.
In this scenario the outer UDP header of the GSO packet must carry the
segmented UDP packet length.
\end{note}

If the VIRTIO_NET_F_HOST_UDP_TUNNEL_GSO option has not
been negotiated, the driver MUST NOT set either the VIRTIO_NET_HDR_F_GSO_UDP_TUNNEL_IPV4
bit or the VIRTIO_NET_HDR_F_GSO_UDP_TUNNEL_IPV6 in \field{gso_type}.

If the VIRTIO_NET_F_HOST_UDP_TUNNEL_GSO_CSUM option has not been negotiated,
the driver MUST NOT set the VIRTIO_NET_HDR_F_UDP_TUNNEL_CSUM bit
in \field{flags}.

The driver SHOULD accept the VIRTIO_NET_F_GUEST_HDRLEN feature if it has
been offered, and if it's able to provide the exact header length.

The driver MUST NOT set the VIRTIO_NET_HDR_F_DATA_VALID and
VIRTIO_NET_HDR_F_RSC_INFO bits in \field{flags}.

The driver MUST NOT set the VIRTIO_NET_HDR_F_DATA_VALID bit in \field{flags}
together with the VIRTIO_NET_HDR_F_GSO_UDP_TUNNEL_IPV4 bit or the
VIRTIO_NET_HDR_F_GSO_UDP_TUNNEL_IPV6 bit in \field{gso_type}.

\devicenormative{\paragraph}{Packet Transmission}{Device Types / Network Device / Device Operation / Packet Transmission}
The device MUST ignore \field{flag} bits that it does not recognize.

If VIRTIO_NET_HDR_F_NEEDS_CSUM bit in \field{flags} is not set, the
device MUST NOT use the \field{csum_start} and \field{csum_offset}.

If one of the VIRTIO_NET_F_HOST_TSO4, TSO6, USO or UFO options have
been negotiated:
\begin{itemize}
\item If the VIRTIO_NET_F_GUEST_HDRLEN feature has been negotiated,
	and \field{gso_type} differs from VIRTIO_NET_HDR_GSO_NONE,
	the device MAY use \field{hdr_len} as the transport header size.

	\begin{note}
	Caution should be taken by the implementation so as to prevent
	a malicious driver from attacking the device by setting an incorrect hdr_len.
	\end{note}

\item If the VIRTIO_NET_F_GUEST_HDRLEN feature has not been negotiated,
	or \field{gso_type} is VIRTIO_NET_HDR_GSO_NONE,
	the device MAY use \field{hdr_len} only as a hint about the
	transport header size.
	The device MUST NOT rely on \field{hdr_len} to be correct.

	\begin{note}
	This is due to various bugs in implementations.
	\end{note}
\end{itemize}

If both the VIRTIO_NET_HDR_GSO_UDP_TUNNEL_IPV4 bit and
the VIRTIO_NET_HDR_GSO_UDP_TUNNEL_IPV6 bit in in \field{gso_type} are set,
the device MUST NOT accept the packet.

If the VIRTIO_NET_HDR_GSO_UDP_TUNNEL_IPV4 bit and the VIRTIO_NET_HDR_GSO_UDP_TUNNEL_IPV6
bit in \field{gso_type} are not set, the device MUST NOT use the
\field{outer_th_offset} and \field{inner_nh_offset}.

If either the VIRTIO_NET_HDR_GSO_UDP_TUNNEL_IPV4 bit or
the VIRTIO_NET_HDR_GSO_UDP_TUNNEL_IPV6 bit in \field{gso_type} are set, and any of
the following is true:
\begin{itemize}
\item the VIRTIO_NET_HDR_F_NEEDS_CSUM is not set in \field{flags}
\item the VIRTIO_NET_HDR_F_DATA_VALID is set in \field{flags}
\item the \field{gso_type} excluding the VIRTIO_NET_HDR_GSO_UDP_TUNNEL_IPV4
bit and the VIRTIO_NET_HDR_GSO_UDP_TUNNEL_IPV6 bit is VIRTIO_NET_HDR_GSO_NONE
\end{itemize}
the device MUST NOT accept the packet.

If the VIRTIO_NET_HDR_F_UDP_TUNNEL_CSUM bit in \field{flags} is set,
and both the bits VIRTIO_NET_HDR_GSO_UDP_TUNNEL_IPV4 and
VIRTIO_NET_HDR_GSO_UDP_TUNNEL_IPV6 in \field{gso_type} are not set,
the device MOST NOT accept the packet.

If VIRTIO_NET_HDR_F_NEEDS_CSUM is not set, the device MUST NOT
rely on the packet checksum being correct.
\paragraph{Packet Transmission Interrupt}\label{sec:Device Types / Network Device / Device Operation / Packet Transmission / Packet Transmission Interrupt}

Often a driver will suppress transmission virtqueue interrupts
and check for used packets in the transmit path of following
packets.

The normal behavior in this interrupt handler is to retrieve
used buffers from the virtqueue and free the corresponding
headers and packets.

\subsubsection{Setting Up Receive Buffers}\label{sec:Device Types / Network Device / Device Operation / Setting Up Receive Buffers}

It is generally a good idea to keep the receive virtqueue as
fully populated as possible: if it runs out, network performance
will suffer.

If the VIRTIO_NET_F_GUEST_TSO4, VIRTIO_NET_F_GUEST_TSO6,
VIRTIO_NET_F_GUEST_UFO, VIRTIO_NET_F_GUEST_USO4 or VIRTIO_NET_F_GUEST_USO6
features are used, the maximum incoming packet
will be 65589 bytes long (14 bytes of Ethernet header, plus 40 bytes of
the IPv6 header, plus 65535 bytes of maximum IPv6 payload including any
extension header), otherwise 1514 bytes.
When VIRTIO_NET_F_HASH_REPORT is not negotiated, the required receive buffer
size is either 65601 or 1526 bytes accounting for 20 bytes of
\field{struct virtio_net_hdr} followed by receive packet.
When VIRTIO_NET_F_HASH_REPORT is negotiated, the required receive buffer
size is either 65609 or 1534 bytes accounting for 12 bytes of
\field{struct virtio_net_hdr} followed by receive packet.

\drivernormative{\paragraph}{Setting Up Receive Buffers}{Device Types / Network Device / Device Operation / Setting Up Receive Buffers}

\begin{itemize}
\item If VIRTIO_NET_F_MRG_RXBUF is not negotiated:
  \begin{itemize}
    \item If VIRTIO_NET_F_GUEST_TSO4, VIRTIO_NET_F_GUEST_TSO6, VIRTIO_NET_F_GUEST_UFO,
	VIRTIO_NET_F_GUEST_USO4 or VIRTIO_NET_F_GUEST_USO6 are negotiated, the driver SHOULD populate
      the receive queue(s) with buffers of at least 65609 bytes if
      VIRTIO_NET_F_HASH_REPORT is negotiated, and of at least 65601 bytes if not.
    \item Otherwise, the driver SHOULD populate the receive queue(s)
      with buffers of at least 1534 bytes if VIRTIO_NET_F_HASH_REPORT
      is negotiated, and of at least 1526 bytes if not.
  \end{itemize}
\item If VIRTIO_NET_F_MRG_RXBUF is negotiated, each buffer MUST be at
least size of \field{struct virtio_net_hdr},
i.e. 20 bytes if VIRTIO_NET_F_HASH_REPORT is negotiated, and 12 bytes if not.
\end{itemize}

\begin{note}
Obviously each buffer can be split across multiple descriptor elements.
\end{note}

When calculating the size of \field{struct virtio_net_hdr}, the driver
MUST consider all the fields inclusive up to \field{padding_reserved},
i.e. 20 bytes if VIRTIO_NET_F_HASH_REPORT is negotiated, and 12 bytes if not.

If VIRTIO_NET_F_MQ is negotiated, each of receiveq1\ldots receiveqN
that will be used SHOULD be populated with receive buffers.

\devicenormative{\paragraph}{Setting Up Receive Buffers}{Device Types / Network Device / Device Operation / Setting Up Receive Buffers}

The device MUST set \field{num_buffers} to the number of descriptors used to
hold the incoming packet.

The device MUST use only a single descriptor if VIRTIO_NET_F_MRG_RXBUF
was not negotiated.
\begin{note}
{This means that \field{num_buffers} will always be 1
if VIRTIO_NET_F_MRG_RXBUF is not negotiated.}
\end{note}

\subsubsection{Processing of Incoming Packets}\label{sec:Device Types / Network Device / Device Operation / Processing of Incoming Packets}
\label{sec:Device Types / Network Device / Device Operation / Processing of Packets}%old label for latexdiff

When a packet is copied into a buffer in the receiveq, the
optimal path is to disable further used buffer notifications for the
receiveq and process packets until no more are found, then re-enable
them.

Processing incoming packets involves:

\begin{enumerate}
\item \field{num_buffers} indicates how many descriptors
  this packet is spread over (including this one): this will
  always be 1 if VIRTIO_NET_F_MRG_RXBUF was not negotiated.
  This allows receipt of large packets without having to allocate large
  buffers: a packet that does not fit in a single buffer can flow
  over to the next buffer, and so on. In this case, there will be
  at least \field{num_buffers} used buffers in the virtqueue, and the device
  chains them together to form a single packet in a way similar to
  how it would store it in a single buffer spread over multiple
  descriptors.
  The other buffers will not begin with a \field{struct virtio_net_hdr}.

\item If
  \field{num_buffers} is one, then the entire packet will be
  contained within this buffer, immediately following the struct
  virtio_net_hdr.
\item If the VIRTIO_NET_F_GUEST_CSUM feature was negotiated, the
  VIRTIO_NET_HDR_F_DATA_VALID bit in \field{flags} can be
  set: if so, device has validated the packet checksum.
  If the VIRTIO_NET_F_GUEST_UDP_TUNNEL_GSO_CSUM feature has been negotiated,
  and the VIRTIO_NET_HDR_F_UDP_TUNNEL_CSUM bit is set in \field{flags},
  both the outer UDP checksum and the inner transport checksum
  have been validated, otherwise only one level of checksums (the outer one
  in case of tunnels) has been validated.
\end{enumerate}

Additionally, VIRTIO_NET_F_GUEST_CSUM, TSO4, TSO6, UDP, UDP_TUNNEL
and ECN features enable receive checksum, large receive offload and ECN
support which are the input equivalents of the transmit checksum,
transmit segmentation offloading and ECN features, as described
in \ref{sec:Device Types / Network Device / Device Operation /
Packet Transmission}:
\begin{enumerate}
\item If the VIRTIO_NET_F_GUEST_TSO4, TSO6, UFO, USO4 or USO6 options were
  negotiated, then \field{gso_type} MAY be something other than
  VIRTIO_NET_HDR_GSO_NONE, and \field{gso_size} field indicates the
  desired MSS (see Packet Transmission point 2).
\item If the VIRTIO_NET_F_RSC_EXT option was negotiated (this
  implies one of VIRTIO_NET_F_GUEST_TSO4, TSO6), the
  device processes also duplicated ACK segments, reports
  number of coalesced TCP segments in \field{csum_start} field and
  number of duplicated ACK segments in \field{csum_offset} field
  and sets bit VIRTIO_NET_HDR_F_RSC_INFO in \field{flags}.
\item If the VIRTIO_NET_F_GUEST_CSUM feature was negotiated, the
  VIRTIO_NET_HDR_F_NEEDS_CSUM bit in \field{flags} can be
  set: if so, the packet checksum at offset \field{csum_offset}
  from \field{csum_start} and any preceding checksums
  have been validated.  The checksum on the packet is incomplete and
  if bit VIRTIO_NET_HDR_F_RSC_INFO is not set in \field{flags},
  then \field{csum_start} and \field{csum_offset} indicate how to calculate it
  (see Packet Transmission point 1).
\begin{note}
If \field{gso_type} differs from VIRTIO_NET_HDR_GSO_NONE and the
VIRTIO_NET_HDR_GSO_UDP_TUNNEL_IPV4 bit or the VIRTIO_NET_HDR_GSO_UDP_TUNNEL_IPV6
bit are not set, \field{csum_offset}
points to the only transport header present in the packet, and there are no
additional preceding checksums validated by VIRTIO_NET_HDR_F_NEEDS_CSUM.
\end{note}
\item If the VIRTIO_NET_F_GUEST_UDP_TUNNEL_GSO option was negotiated and
  \field{gso_type} is not VIRTIO_NET_HDR_GSO_NONE, the
  VIRTIO_NET_HDR_GSO_UDP_TUNNEL_IPV4 bit or the VIRTIO_NET_HDR_GSO_UDP_TUNNEL_IPV6
  bit MAY be set. In such case the \field{outer_th_offset} and
  \field{inner_nh_offset} fields indicate the corresponding
  headers information.
\item If the VIRTIO_NET_F_GUEST_UDP_TUNNEL_GSO_CSUM feature was
negotiated, and
  the VIRTIO_NET_HDR_GSO_UDP_TUNNEL_IPV4 bit or the VIRTIO_NET_HDR_GSO_UDP_TUNNEL_IPV6
  are set in \field{gso_type}, the VIRTIO_NET_HDR_F_UDP_TUNNEL_CSUM bit in the
  \field{flags} can be set: if so, the outer UDP checksum has been validated
  and the UDP header checksum at offset 6 from from \field{outer_th_offset}
  is set to the outer UDP pseudo header checksum.

\begin{note}
If the VIRTIO_NET_HDR_GSO_UDP_TUNNEL_IPV4 bit or VIRTIO_NET_HDR_GSO_UDP_TUNNEL_IPV6
bit are set in \field{gso_type}, the \field{csum_start} field refers to
the inner transport header offset (see Packet Transmission point 1).
If the VIRTIO_NET_HDR_F_UDP_TUNNEL_CSUM bit in \field{flags} is set both
the inner and the outer header checksums have been validated by
VIRTIO_NET_HDR_F_NEEDS_CSUM, otherwise only the inner transport header
checksum has been validated.
\end{note}
\end{enumerate}

If applicable, the device calculates per-packet hash for incoming packets as
defined in \ref{sec:Device Types / Network Device / Device Operation / Processing of Incoming Packets / Hash calculation for incoming packets}.

If applicable, the device reports hash information for incoming packets as
defined in \ref{sec:Device Types / Network Device / Device Operation / Processing of Incoming Packets / Hash reporting for incoming packets}.

\devicenormative{\paragraph}{Processing of Incoming Packets}{Device Types / Network Device / Device Operation / Processing of Incoming Packets}
\label{devicenormative:Device Types / Network Device / Device Operation / Processing of Packets}%old label for latexdiff

If VIRTIO_NET_F_MRG_RXBUF has not been negotiated, the device MUST set
\field{num_buffers} to 1.

If VIRTIO_NET_F_MRG_RXBUF has been negotiated, the device MUST set
\field{num_buffers} to indicate the number of buffers
the packet (including the header) is spread over.

If a receive packet is spread over multiple buffers, the device
MUST use all buffers but the last (i.e. the first \field{num_buffers} -
1 buffers) completely up to the full length of each buffer
supplied by the driver.

The device MUST use all buffers used by a single receive
packet together, such that at least \field{num_buffers} are
observed by driver as used.

If VIRTIO_NET_F_GUEST_CSUM is not negotiated, the device MUST set
\field{flags} to zero and SHOULD supply a fully checksummed
packet to the driver.

If VIRTIO_NET_F_GUEST_TSO4 is not negotiated, the device MUST NOT set
\field{gso_type} to VIRTIO_NET_HDR_GSO_TCPV4.

If VIRTIO_NET_F_GUEST_UDP is not negotiated, the device MUST NOT set
\field{gso_type} to VIRTIO_NET_HDR_GSO_UDP.

If VIRTIO_NET_F_GUEST_TSO6 is not negotiated, the device MUST NOT set
\field{gso_type} to VIRTIO_NET_HDR_GSO_TCPV6.

If none of VIRTIO_NET_F_GUEST_USO4 or VIRTIO_NET_F_GUEST_USO6 have been negotiated,
the device MUST NOT set \field{gso_type} to VIRTIO_NET_HDR_GSO_UDP_L4.

If VIRTIO_NET_F_GUEST_UDP_TUNNEL_GSO is not negotiated, the device MUST NOT set
either the VIRTIO_NET_HDR_GSO_UDP_TUNNEL_IPV4 bit or the
VIRTIO_NET_HDR_GSO_UDP_TUNNEL_IPV6 bit in \field{gso_type}.

If VIRTIO_NET_F_GUEST_UDP_TUNNEL_GSO_CSUM is not negotiated the device MUST NOT set
the VIRTIO_NET_HDR_F_UDP_TUNNEL_CSUM bit in \field{flags}.

The device SHOULD NOT send to the driver TCP packets requiring segmentation offload
which have the Explicit Congestion Notification bit set, unless the
VIRTIO_NET_F_GUEST_ECN feature is negotiated, in which case the
device MUST set the VIRTIO_NET_HDR_GSO_ECN bit in
\field{gso_type}.

If the VIRTIO_NET_F_GUEST_CSUM feature has been negotiated, the
device MAY set the VIRTIO_NET_HDR_F_NEEDS_CSUM bit in
\field{flags}, if so:
\begin{enumerate}
\item the device MUST validate the packet checksum at
	offset \field{csum_offset} from \field{csum_start} as well as all
	preceding offsets;
\item the device MUST set the packet checksum stored in the
	receive buffer to the TCP/UDP pseudo header;
\item the device MUST set \field{csum_start} and
	\field{csum_offset} such that calculating a ones'
	complement checksum from \field{csum_start} up until the
	end of the packet and storing the result at offset
	\field{csum_offset} from  \field{csum_start} will result in a
	fully checksummed packet;
\end{enumerate}

The device MUST NOT send to the driver GSO packets encapsulated in UDP
tunnel and requiring segmentation offload, unless the
VIRTIO_NET_F_GUEST_UDP_TUNNEL_GSO is negotiated, in which case the device MUST set
the VIRTIO_NET_HDR_GSO_UDP_TUNNEL_IPV4 bit or the VIRTIO_NET_HDR_GSO_UDP_TUNNEL_IPV6
bit in \field{gso_type} according to the inner network header protocol type,
MUST set the \field{outer_th_offset} and \field{inner_nh_offset} fields
to the corresponding header information, and the outer UDP header MUST NOT
require checksum offload.

If the VIRTIO_NET_F_GUEST_UDP_TUNNEL_GSO_CSUM feature has not been negotiated,
the device MUST NOT send the driver GSO packets encapsulated in UDP
tunnel and requiring segmentation and outer checksum offload.

If none of the VIRTIO_NET_F_GUEST_TSO4, TSO6, UFO, USO4 or USO6 options have
been negotiated, the device MUST set \field{gso_type} to
VIRTIO_NET_HDR_GSO_NONE.

If \field{gso_type} differs from VIRTIO_NET_HDR_GSO_NONE, then
the device MUST also set the VIRTIO_NET_HDR_F_NEEDS_CSUM bit in
\field{flags} MUST set \field{gso_size} to indicate the desired MSS.
If VIRTIO_NET_F_RSC_EXT was negotiated, the device MUST also
set VIRTIO_NET_HDR_F_RSC_INFO bit in \field{flags},
set \field{csum_start} to number of coalesced TCP segments and
set \field{csum_offset} to number of received duplicated ACK segments.

If VIRTIO_NET_F_RSC_EXT was not negotiated, the device MUST
not set VIRTIO_NET_HDR_F_RSC_INFO bit in \field{flags}.

If one of the VIRTIO_NET_F_GUEST_TSO4, TSO6, UFO, USO4 or USO6 options have
been negotiated, the device SHOULD set \field{hdr_len} to a value
not less than the length of the headers, including the transport
header. If \field{gso_type} has the VIRTIO_NET_HDR_GSO_UDP_TUNNEL_IPV4 bit
or the VIRTIO_NET_HDR_GSO_UDP_TUNNEL_IPV6 bit set, the referenced transport
header is the inner one.

If the VIRTIO_NET_F_GUEST_CSUM feature has been negotiated, the
device MAY set the VIRTIO_NET_HDR_F_DATA_VALID bit in
\field{flags}, if so, the device MUST validate the packet
checksum. If the VIRTIO_NET_F_GUEST_UDP_TUNNEL_GSO_CSUM feature has
been negotiated, and the VIRTIO_NET_HDR_F_UDP_TUNNEL_CSUM bit set in
\field{flags}, both the outer UDP checksum and the inner transport
checksum have been validated.
Otherwise level of checksum is validated: in case of multiple
encapsulated protocols the outermost one.

If either the VIRTIO_NET_HDR_GSO_UDP_TUNNEL_IPV4 bit or the
VIRTIO_NET_HDR_GSO_UDP_TUNNEL_IPV6 bit in \field{gso_type} are set,
the device MUST NOT set the VIRTIO_NET_HDR_F_DATA_VALID bit in
\field{flags}.

If the VIRTIO_NET_F_GUEST_UDP_TUNNEL_GSO_CSUM feature has been negotiated
and either the VIRTIO_NET_HDR_GSO_UDP_TUNNEL_IPV4 bit is set or the
VIRTIO_NET_HDR_GSO_UDP_TUNNEL_IPV6 bit is set in \field{gso_type}, the
device MAY set the VIRTIO_NET_HDR_F_UDP_TUNNEL_CSUM bit in
\field{flags}, if so the device MUST set the packet outer UDP checksum
stored in the receive buffer to the outer UDP pseudo header.

Otherwise, the VIRTIO_NET_F_GUEST_UDP_TUNNEL_GSO_CSUM feature has been
negotiated, either the VIRTIO_NET_HDR_GSO_UDP_TUNNEL_IPV4 bit is set or the
VIRTIO_NET_HDR_GSO_UDP_TUNNEL_IPV6 bit is set in \field{gso_type},
and the bit VIRTIO_NET_HDR_F_UDP_TUNNEL_CSUM is not set in
\field{flags}, the device MUST either provide a zero outer UDP header
checksum or a fully checksummed outer UDP header.

\drivernormative{\paragraph}{Processing of Incoming
Packets}{Device Types / Network Device / Device Operation /
Processing of Incoming Packets}

The driver MUST ignore \field{flag} bits that it does not recognize.

If VIRTIO_NET_HDR_F_NEEDS_CSUM bit in \field{flags} is not set or
if VIRTIO_NET_HDR_F_RSC_INFO bit \field{flags} is set, the
driver MUST NOT use the \field{csum_start} and \field{csum_offset}.

If one of the VIRTIO_NET_F_GUEST_TSO4, TSO6, UFO, USO4 or USO6 options have
been negotiated, the driver MAY use \field{hdr_len} only as a hint about the
transport header size.
The driver MUST NOT rely on \field{hdr_len} to be correct.
\begin{note}
This is due to various bugs in implementations.
\end{note}

If neither VIRTIO_NET_HDR_F_NEEDS_CSUM nor
VIRTIO_NET_HDR_F_DATA_VALID is set, the driver MUST NOT
rely on the packet checksum being correct.

If both the VIRTIO_NET_HDR_GSO_UDP_TUNNEL_IPV4 bit and
the VIRTIO_NET_HDR_GSO_UDP_TUNNEL_IPV6 bit in in \field{gso_type} are set,
the driver MUST NOT accept the packet.

If the VIRTIO_NET_HDR_GSO_UDP_TUNNEL_IPV4 bit or the VIRTIO_NET_HDR_GSO_UDP_TUNNEL_IPV6
bit in \field{gso_type} are not set, the driver MUST NOT use the
\field{outer_th_offset} and \field{inner_nh_offset}.

If either the VIRTIO_NET_HDR_GSO_UDP_TUNNEL_IPV4 bit or
the VIRTIO_NET_HDR_GSO_UDP_TUNNEL_IPV6 bit in \field{gso_type} are set, and any of
the following is true:
\begin{itemize}
\item the VIRTIO_NET_HDR_F_NEEDS_CSUM bit is not set in \field{flags}
\item the VIRTIO_NET_HDR_F_DATA_VALID bit is set in \field{flags}
\item the \field{gso_type} excluding the VIRTIO_NET_HDR_GSO_UDP_TUNNEL_IPV4
bit and the VIRTIO_NET_HDR_GSO_UDP_TUNNEL_IPV6 bit is VIRTIO_NET_HDR_GSO_NONE
\end{itemize}
the driver MUST NOT accept the packet.

If the VIRTIO_NET_HDR_F_UDP_TUNNEL_CSUM bit and the VIRTIO_NET_HDR_F_NEEDS_CSUM
bit in \field{flags} are set,
and both the bits VIRTIO_NET_HDR_GSO_UDP_TUNNEL_IPV4 and
VIRTIO_NET_HDR_GSO_UDP_TUNNEL_IPV6 in \field{gso_type} are not set,
the driver MOST NOT accept the packet.

\paragraph{Hash calculation for incoming packets}
\label{sec:Device Types / Network Device / Device Operation / Processing of Incoming Packets / Hash calculation for incoming packets}

A device attempts to calculate a per-packet hash in the following cases:
\begin{itemize}
\item The feature VIRTIO_NET_F_RSS was negotiated. The device uses the hash to determine the receive virtqueue to place incoming packets.
\item The feature VIRTIO_NET_F_HASH_REPORT was negotiated. The device reports the hash value and the hash type with the packet.
\end{itemize}

If the feature VIRTIO_NET_F_RSS was negotiated:
\begin{itemize}
\item The device uses \field{hash_types} of the virtio_net_rss_config structure as 'Enabled hash types' bitmask.
\item If additionally the feature VIRTIO_NET_F_HASH_TUNNEL was negotiated, the device uses \field{enabled_tunnel_types} of the
      virtnet_hash_tunnel structure as 'Encapsulation types enabled for inner header hash' bitmask.
\item The device uses a key as defined in \field{hash_key_data} and \field{hash_key_length} of the virtio_net_rss_config structure (see
\ref{sec:Device Types / Network Device / Device Operation / Control Virtqueue / Receive-side scaling (RSS) / Setting RSS parameters}).
\end{itemize}

If the feature VIRTIO_NET_F_RSS was not negotiated:
\begin{itemize}
\item The device uses \field{hash_types} of the virtio_net_hash_config structure as 'Enabled hash types' bitmask.
\item If additionally the feature VIRTIO_NET_F_HASH_TUNNEL was negotiated, the device uses \field{enabled_tunnel_types} of the
      virtnet_hash_tunnel structure as 'Encapsulation types enabled for inner header hash' bitmask.
\item The device uses a key as defined in \field{hash_key_data} and \field{hash_key_length} of the virtio_net_hash_config structure (see
\ref{sec:Device Types / Network Device / Device Operation / Control Virtqueue / Automatic receive steering in multiqueue mode / Hash calculation}).
\end{itemize}

Note that if the device offers VIRTIO_NET_F_HASH_REPORT, even if it supports only one pair of virtqueues, it MUST support
at least one of commands of VIRTIO_NET_CTRL_MQ class to configure reported hash parameters:
\begin{itemize}
\item If the device offers VIRTIO_NET_F_RSS, it MUST support VIRTIO_NET_CTRL_MQ_RSS_CONFIG command per
 \ref{sec:Device Types / Network Device / Device Operation / Control Virtqueue / Receive-side scaling (RSS) / Setting RSS parameters}.
\item Otherwise the device MUST support VIRTIO_NET_CTRL_MQ_HASH_CONFIG command per
 \ref{sec:Device Types / Network Device / Device Operation / Control Virtqueue / Automatic receive steering in multiqueue mode / Hash calculation}.
\end{itemize}

The per-packet hash calculation can depend on the IP packet type. See
\hyperref[intro:IP]{[IP]}, \hyperref[intro:UDP]{[UDP]} and \hyperref[intro:TCP]{[TCP]}.

\subparagraph{Supported/enabled hash types}
\label{sec:Device Types / Network Device / Device Operation / Processing of Incoming Packets / Hash calculation for incoming packets / Supported/enabled hash types}
Hash types applicable for IPv4 packets:
\begin{lstlisting}
#define VIRTIO_NET_HASH_TYPE_IPv4              (1 << 0)
#define VIRTIO_NET_HASH_TYPE_TCPv4             (1 << 1)
#define VIRTIO_NET_HASH_TYPE_UDPv4             (1 << 2)
\end{lstlisting}
Hash types applicable for IPv6 packets without extension headers
\begin{lstlisting}
#define VIRTIO_NET_HASH_TYPE_IPv6              (1 << 3)
#define VIRTIO_NET_HASH_TYPE_TCPv6             (1 << 4)
#define VIRTIO_NET_HASH_TYPE_UDPv6             (1 << 5)
\end{lstlisting}
Hash types applicable for IPv6 packets with extension headers
\begin{lstlisting}
#define VIRTIO_NET_HASH_TYPE_IP_EX             (1 << 6)
#define VIRTIO_NET_HASH_TYPE_TCP_EX            (1 << 7)
#define VIRTIO_NET_HASH_TYPE_UDP_EX            (1 << 8)
\end{lstlisting}

\subparagraph{IPv4 packets}
\label{sec:Device Types / Network Device / Device Operation / Processing of Incoming Packets / Hash calculation for incoming packets / IPv4 packets}
The device calculates the hash on IPv4 packets according to 'Enabled hash types' bitmask as follows:
\begin{itemize}
\item If VIRTIO_NET_HASH_TYPE_TCPv4 is set and the packet has
a TCP header, the hash is calculated over the following fields:
\begin{itemize}
\item Source IP address
\item Destination IP address
\item Source TCP port
\item Destination TCP port
\end{itemize}
\item Else if VIRTIO_NET_HASH_TYPE_UDPv4 is set and the
packet has a UDP header, the hash is calculated over the following fields:
\begin{itemize}
\item Source IP address
\item Destination IP address
\item Source UDP port
\item Destination UDP port
\end{itemize}
\item Else if VIRTIO_NET_HASH_TYPE_IPv4 is set, the hash is
calculated over the following fields:
\begin{itemize}
\item Source IP address
\item Destination IP address
\end{itemize}
\item Else the device does not calculate the hash
\end{itemize}

\subparagraph{IPv6 packets without extension header}
\label{sec:Device Types / Network Device / Device Operation / Processing of Incoming Packets / Hash calculation for incoming packets / IPv6 packets without extension header}
The device calculates the hash on IPv6 packets without extension
headers according to 'Enabled hash types' bitmask as follows:
\begin{itemize}
\item If VIRTIO_NET_HASH_TYPE_TCPv6 is set and the packet has
a TCPv6 header, the hash is calculated over the following fields:
\begin{itemize}
\item Source IPv6 address
\item Destination IPv6 address
\item Source TCP port
\item Destination TCP port
\end{itemize}
\item Else if VIRTIO_NET_HASH_TYPE_UDPv6 is set and the
packet has a UDPv6 header, the hash is calculated over the following fields:
\begin{itemize}
\item Source IPv6 address
\item Destination IPv6 address
\item Source UDP port
\item Destination UDP port
\end{itemize}
\item Else if VIRTIO_NET_HASH_TYPE_IPv6 is set, the hash is
calculated over the following fields:
\begin{itemize}
\item Source IPv6 address
\item Destination IPv6 address
\end{itemize}
\item Else the device does not calculate the hash
\end{itemize}

\subparagraph{IPv6 packets with extension header}
\label{sec:Device Types / Network Device / Device Operation / Processing of Incoming Packets / Hash calculation for incoming packets / IPv6 packets with extension header}
The device calculates the hash on IPv6 packets with extension
headers according to 'Enabled hash types' bitmask as follows:
\begin{itemize}
\item If VIRTIO_NET_HASH_TYPE_TCP_EX is set and the packet
has a TCPv6 header, the hash is calculated over the following fields:
\begin{itemize}
\item Home address from the home address option in the IPv6 destination options header. If the extension header is not present, use the Source IPv6 address.
\item IPv6 address that is contained in the Routing-Header-Type-2 from the associated extension header. If the extension header is not present, use the Destination IPv6 address.
\item Source TCP port
\item Destination TCP port
\end{itemize}
\item Else if VIRTIO_NET_HASH_TYPE_UDP_EX is set and the
packet has a UDPv6 header, the hash is calculated over the following fields:
\begin{itemize}
\item Home address from the home address option in the IPv6 destination options header. If the extension header is not present, use the Source IPv6 address.
\item IPv6 address that is contained in the Routing-Header-Type-2 from the associated extension header. If the extension header is not present, use the Destination IPv6 address.
\item Source UDP port
\item Destination UDP port
\end{itemize}
\item Else if VIRTIO_NET_HASH_TYPE_IP_EX is set, the hash is
calculated over the following fields:
\begin{itemize}
\item Home address from the home address option in the IPv6 destination options header. If the extension header is not present, use the Source IPv6 address.
\item IPv6 address that is contained in the Routing-Header-Type-2 from the associated extension header. If the extension header is not present, use the Destination IPv6 address.
\end{itemize}
\item Else skip IPv6 extension headers and calculate the hash as
defined for an IPv6 packet without extension headers
(see \ref{sec:Device Types / Network Device / Device Operation / Processing of Incoming Packets / Hash calculation for incoming packets / IPv6 packets without extension header}).
\end{itemize}

\paragraph{Inner Header Hash}
\label{sec:Device Types / Network Device / Device Operation / Processing of Incoming Packets / Inner Header Hash}

If VIRTIO_NET_F_HASH_TUNNEL has been negotiated, the driver can send the command
VIRTIO_NET_CTRL_HASH_TUNNEL_SET to configure the calculation of the inner header hash.

\begin{lstlisting}
struct virtnet_hash_tunnel {
    le32 enabled_tunnel_types;
};

#define VIRTIO_NET_CTRL_HASH_TUNNEL 7
 #define VIRTIO_NET_CTRL_HASH_TUNNEL_SET 0
\end{lstlisting}

Field \field{enabled_tunnel_types} contains the bitmask of encapsulation types enabled for inner header hash.
See \ref{sec:Device Types / Network Device / Device Operation / Processing of Incoming Packets /
Hash calculation for incoming packets / Encapsulation types supported/enabled for inner header hash}.

The class VIRTIO_NET_CTRL_HASH_TUNNEL has one command:
VIRTIO_NET_CTRL_HASH_TUNNEL_SET sets \field{enabled_tunnel_types} for the device using the
virtnet_hash_tunnel structure, which is read-only for the device.

Inner header hash is disabled by VIRTIO_NET_CTRL_HASH_TUNNEL_SET with \field{enabled_tunnel_types} set to 0.

Initially (before the driver sends any VIRTIO_NET_CTRL_HASH_TUNNEL_SET command) all
encapsulation types are disabled for inner header hash.

\subparagraph{Encapsulated packet}
\label{sec:Device Types / Network Device / Device Operation / Processing of Incoming Packets / Hash calculation for incoming packets / Encapsulated packet}

Multiple tunneling protocols allow encapsulating an inner, payload packet in an outer, encapsulated packet.
The encapsulated packet thus contains an outer header and an inner header, and the device calculates the
hash over either the inner header or the outer header.

If VIRTIO_NET_F_HASH_TUNNEL is negotiated and a received encapsulated packet's outer header matches one of the
encapsulation types enabled in \field{enabled_tunnel_types}, then the device uses the inner header for hash
calculations (only a single level of encapsulation is currently supported).

If VIRTIO_NET_F_HASH_TUNNEL is negotiated and a received packet's (outer) header does not match any encapsulation
types enabled in \field{enabled_tunnel_types}, then the device uses the outer header for hash calculations.

\subparagraph{Encapsulation types supported/enabled for inner header hash}
\label{sec:Device Types / Network Device / Device Operation / Processing of Incoming Packets /
Hash calculation for incoming packets / Encapsulation types supported/enabled for inner header hash}

Encapsulation types applicable for inner header hash:
\begin{lstlisting}[escapechar=|]
#define VIRTIO_NET_HASH_TUNNEL_TYPE_GRE_2784    (1 << 0) /* |\hyperref[intro:rfc2784]{[RFC2784]}| */
#define VIRTIO_NET_HASH_TUNNEL_TYPE_GRE_2890    (1 << 1) /* |\hyperref[intro:rfc2890]{[RFC2890]}| */
#define VIRTIO_NET_HASH_TUNNEL_TYPE_GRE_7676    (1 << 2) /* |\hyperref[intro:rfc7676]{[RFC7676]}| */
#define VIRTIO_NET_HASH_TUNNEL_TYPE_GRE_UDP     (1 << 3) /* |\hyperref[intro:rfc8086]{[GRE-in-UDP]}| */
#define VIRTIO_NET_HASH_TUNNEL_TYPE_VXLAN       (1 << 4) /* |\hyperref[intro:vxlan]{[VXLAN]}| */
#define VIRTIO_NET_HASH_TUNNEL_TYPE_VXLAN_GPE   (1 << 5) /* |\hyperref[intro:vxlan-gpe]{[VXLAN-GPE]}| */
#define VIRTIO_NET_HASH_TUNNEL_TYPE_GENEVE      (1 << 6) /* |\hyperref[intro:geneve]{[GENEVE]}| */
#define VIRTIO_NET_HASH_TUNNEL_TYPE_IPIP        (1 << 7) /* |\hyperref[intro:ipip]{[IPIP]}| */
#define VIRTIO_NET_HASH_TUNNEL_TYPE_NVGRE       (1 << 8) /* |\hyperref[intro:nvgre]{[NVGRE]}| */
\end{lstlisting}

\subparagraph{Advice}
Example uses of the inner header hash:
\begin{itemize}
\item Legacy tunneling protocols, lacking the outer header entropy, can use RSS with the inner header hash to
      distribute flows with identical outer but different inner headers across various queues, improving performance.
\item Identify an inner flow distributed across multiple outer tunnels.
\end{itemize}

As using the inner header hash completely discards the outer header entropy, care must be taken
if the inner header is controlled by an adversary, as the adversary can then intentionally create
configurations with insufficient entropy.

Besides disabling the inner header hash, mitigations would depend on how the hash is used. When the hash
use is limited to the RSS queue selection, the inner header hash may have quality of service (QoS) limitations.

\devicenormative{\subparagraph}{Inner Header Hash}{Device Types / Network Device / Device Operation / Control Virtqueue / Inner Header Hash}

If the (outer) header of the received packet does not match any encapsulation types enabled
in \field{enabled_tunnel_types}, the device MUST calculate the hash on the outer header.

If the device receives any bits in \field{enabled_tunnel_types} which are not set in \field{supported_tunnel_types},
it SHOULD respond to the VIRTIO_NET_CTRL_HASH_TUNNEL_SET command with VIRTIO_NET_ERR.

If the driver sets \field{enabled_tunnel_types} to 0 through VIRTIO_NET_CTRL_HASH_TUNNEL_SET or upon the device reset,
the device MUST disable the inner header hash for all encapsulation types.

\drivernormative{\subparagraph}{Inner Header Hash}{Device Types / Network Device / Device Operation / Control Virtqueue / Inner Header Hash}

The driver MUST have negotiated the VIRTIO_NET_F_HASH_TUNNEL feature when issuing the VIRTIO_NET_CTRL_HASH_TUNNEL_SET command.

The driver MUST NOT set any bits in \field{enabled_tunnel_types} which are not set in \field{supported_tunnel_types}.

The driver MUST ignore bits in \field{supported_tunnel_types} which are not documented in this specification.

\paragraph{Hash reporting for incoming packets}
\label{sec:Device Types / Network Device / Device Operation / Processing of Incoming Packets / Hash reporting for incoming packets}

If VIRTIO_NET_F_HASH_REPORT was negotiated and
 the device has calculated the hash for the packet, the device fills \field{hash_report} with the report type of calculated hash
and \field{hash_value} with the value of calculated hash.

If VIRTIO_NET_F_HASH_REPORT was negotiated but due to any reason the
hash was not calculated, the device sets \field{hash_report} to VIRTIO_NET_HASH_REPORT_NONE.

Possible values that the device can report in \field{hash_report} are defined below.
They correspond to supported hash types defined in
\ref{sec:Device Types / Network Device / Device Operation / Processing of Incoming Packets / Hash calculation for incoming packets / Supported/enabled hash types}
as follows:

VIRTIO_NET_HASH_TYPE_XXX = 1 << (VIRTIO_NET_HASH_REPORT_XXX - 1)

\begin{lstlisting}
#define VIRTIO_NET_HASH_REPORT_NONE            0
#define VIRTIO_NET_HASH_REPORT_IPv4            1
#define VIRTIO_NET_HASH_REPORT_TCPv4           2
#define VIRTIO_NET_HASH_REPORT_UDPv4           3
#define VIRTIO_NET_HASH_REPORT_IPv6            4
#define VIRTIO_NET_HASH_REPORT_TCPv6           5
#define VIRTIO_NET_HASH_REPORT_UDPv6           6
#define VIRTIO_NET_HASH_REPORT_IPv6_EX         7
#define VIRTIO_NET_HASH_REPORT_TCPv6_EX        8
#define VIRTIO_NET_HASH_REPORT_UDPv6_EX        9
\end{lstlisting}

\subsubsection{Control Virtqueue}\label{sec:Device Types / Network Device / Device Operation / Control Virtqueue}

The driver uses the control virtqueue (if VIRTIO_NET_F_CTRL_VQ is
negotiated) to send commands to manipulate various features of
the device which would not easily map into the configuration
space.

All commands are of the following form:

\begin{lstlisting}
struct virtio_net_ctrl {
        u8 class;
        u8 command;
        u8 command-specific-data[];
        u8 ack;
        u8 command-specific-result[];
};

/* ack values */
#define VIRTIO_NET_OK     0
#define VIRTIO_NET_ERR    1
\end{lstlisting}

The \field{class}, \field{command} and command-specific-data are set by the
driver, and the device sets the \field{ack} byte and optionally
\field{command-specific-result}. There is little the driver can
do except issue a diagnostic if \field{ack} is not VIRTIO_NET_OK.

The command VIRTIO_NET_CTRL_STATS_QUERY and VIRTIO_NET_CTRL_STATS_GET contain
\field{command-specific-result}.

\paragraph{Packet Receive Filtering}\label{sec:Device Types / Network Device / Device Operation / Control Virtqueue / Packet Receive Filtering}
\label{sec:Device Types / Network Device / Device Operation / Control Virtqueue / Setting Promiscuous Mode}%old label for latexdiff

If the VIRTIO_NET_F_CTRL_RX and VIRTIO_NET_F_CTRL_RX_EXTRA
features are negotiated, the driver can send control commands for
promiscuous mode, multicast, unicast and broadcast receiving.

\begin{note}
In general, these commands are best-effort: unwanted
packets could still arrive.
\end{note}

\begin{lstlisting}
#define VIRTIO_NET_CTRL_RX    0
 #define VIRTIO_NET_CTRL_RX_PROMISC      0
 #define VIRTIO_NET_CTRL_RX_ALLMULTI     1
 #define VIRTIO_NET_CTRL_RX_ALLUNI       2
 #define VIRTIO_NET_CTRL_RX_NOMULTI      3
 #define VIRTIO_NET_CTRL_RX_NOUNI        4
 #define VIRTIO_NET_CTRL_RX_NOBCAST      5
\end{lstlisting}


\devicenormative{\subparagraph}{Packet Receive Filtering}{Device Types / Network Device / Device Operation / Control Virtqueue / Packet Receive Filtering}

If the VIRTIO_NET_F_CTRL_RX feature has been negotiated,
the device MUST support the following VIRTIO_NET_CTRL_RX class
commands:
\begin{itemize}
\item VIRTIO_NET_CTRL_RX_PROMISC turns promiscuous mode on and
off. The command-specific-data is one byte containing 0 (off) or
1 (on). If promiscuous mode is on, the device SHOULD receive all
incoming packets.
This SHOULD take effect even if one of the other modes set by
a VIRTIO_NET_CTRL_RX class command is on.
\item VIRTIO_NET_CTRL_RX_ALLMULTI turns all-multicast receive on and
off. The command-specific-data is one byte containing 0 (off) or
1 (on). When all-multicast receive is on the device SHOULD allow
all incoming multicast packets.
\end{itemize}

If the VIRTIO_NET_F_CTRL_RX_EXTRA feature has been negotiated,
the device MUST support the following VIRTIO_NET_CTRL_RX class
commands:
\begin{itemize}
\item VIRTIO_NET_CTRL_RX_ALLUNI turns all-unicast receive on and
off. The command-specific-data is one byte containing 0 (off) or
1 (on). When all-unicast receive is on the device SHOULD allow
all incoming unicast packets.
\item VIRTIO_NET_CTRL_RX_NOMULTI suppresses multicast receive.
The command-specific-data is one byte containing 0 (multicast
receive allowed) or 1 (multicast receive suppressed).
When multicast receive is suppressed, the device SHOULD NOT
send multicast packets to the driver.
This SHOULD take effect even if VIRTIO_NET_CTRL_RX_ALLMULTI is on.
This filter SHOULD NOT apply to broadcast packets.
\item VIRTIO_NET_CTRL_RX_NOUNI suppresses unicast receive.
The command-specific-data is one byte containing 0 (unicast
receive allowed) or 1 (unicast receive suppressed).
When unicast receive is suppressed, the device SHOULD NOT
send unicast packets to the driver.
This SHOULD take effect even if VIRTIO_NET_CTRL_RX_ALLUNI is on.
\item VIRTIO_NET_CTRL_RX_NOBCAST suppresses broadcast receive.
The command-specific-data is one byte containing 0 (broadcast
receive allowed) or 1 (broadcast receive suppressed).
When broadcast receive is suppressed, the device SHOULD NOT
send broadcast packets to the driver.
This SHOULD take effect even if VIRTIO_NET_CTRL_RX_ALLMULTI is on.
\end{itemize}

\drivernormative{\subparagraph}{Packet Receive Filtering}{Device Types / Network Device / Device Operation / Control Virtqueue / Packet Receive Filtering}

If the VIRTIO_NET_F_CTRL_RX feature has not been negotiated,
the driver MUST NOT issue commands VIRTIO_NET_CTRL_RX_PROMISC or
VIRTIO_NET_CTRL_RX_ALLMULTI.

If the VIRTIO_NET_F_CTRL_RX_EXTRA feature has not been negotiated,
the driver MUST NOT issue commands
 VIRTIO_NET_CTRL_RX_ALLUNI,
 VIRTIO_NET_CTRL_RX_NOMULTI,
 VIRTIO_NET_CTRL_RX_NOUNI or
 VIRTIO_NET_CTRL_RX_NOBCAST.

\paragraph{Setting MAC Address Filtering}\label{sec:Device Types / Network Device / Device Operation / Control Virtqueue / Setting MAC Address Filtering}

If the VIRTIO_NET_F_CTRL_RX feature is negotiated, the driver can
send control commands for MAC address filtering.

\begin{lstlisting}
struct virtio_net_ctrl_mac {
        le32 entries;
        u8 macs[entries][6];
};

#define VIRTIO_NET_CTRL_MAC    1
 #define VIRTIO_NET_CTRL_MAC_TABLE_SET        0
 #define VIRTIO_NET_CTRL_MAC_ADDR_SET         1
\end{lstlisting}

The device can filter incoming packets by any number of destination
MAC addresses\footnote{Since there are no guarantees, it can use a hash filter or
silently switch to allmulti or promiscuous mode if it is given too
many addresses.
}. This table is set using the class
VIRTIO_NET_CTRL_MAC and the command VIRTIO_NET_CTRL_MAC_TABLE_SET. The
command-specific-data is two variable length tables of 6-byte MAC
addresses (as described in struct virtio_net_ctrl_mac). The first table contains unicast addresses, and the second
contains multicast addresses.

The VIRTIO_NET_CTRL_MAC_ADDR_SET command is used to set the
default MAC address which rx filtering
accepts (and if VIRTIO_NET_F_MAC has been negotiated,
this will be reflected in \field{mac} in config space).

The command-specific-data for VIRTIO_NET_CTRL_MAC_ADDR_SET is
the 6-byte MAC address.

\devicenormative{\subparagraph}{Setting MAC Address Filtering}{Device Types / Network Device / Device Operation / Control Virtqueue / Setting MAC Address Filtering}

The device MUST have an empty MAC filtering table on reset.

The device MUST update the MAC filtering table before it consumes
the VIRTIO_NET_CTRL_MAC_TABLE_SET command.

The device MUST update \field{mac} in config space before it consumes
the VIRTIO_NET_CTRL_MAC_ADDR_SET command, if VIRTIO_NET_F_MAC has
been negotiated.

The device SHOULD drop incoming packets which have a destination MAC which
matches neither the \field{mac} (or that set with VIRTIO_NET_CTRL_MAC_ADDR_SET)
nor the MAC filtering table.

\drivernormative{\subparagraph}{Setting MAC Address Filtering}{Device Types / Network Device / Device Operation / Control Virtqueue / Setting MAC Address Filtering}

If VIRTIO_NET_F_CTRL_RX has not been negotiated,
the driver MUST NOT issue VIRTIO_NET_CTRL_MAC class commands.

If VIRTIO_NET_F_CTRL_RX has been negotiated,
the driver SHOULD issue VIRTIO_NET_CTRL_MAC_ADDR_SET
to set the default mac if it is different from \field{mac}.

The driver MUST follow the VIRTIO_NET_CTRL_MAC_TABLE_SET command
by a le32 number, followed by that number of non-multicast
MAC addresses, followed by another le32 number, followed by
that number of multicast addresses.  Either number MAY be 0.

\subparagraph{Legacy Interface: Setting MAC Address Filtering}\label{sec:Device Types / Network Device / Device Operation / Control Virtqueue / Setting MAC Address Filtering / Legacy Interface: Setting MAC Address Filtering}
When using the legacy interface, transitional devices and drivers
MUST format \field{entries} in struct virtio_net_ctrl_mac
according to the native endian of the guest rather than
(necessarily when not using the legacy interface) little-endian.

Legacy drivers that didn't negotiate VIRTIO_NET_F_CTRL_MAC_ADDR
changed \field{mac} in config space when NIC is accepting
incoming packets. These drivers always wrote the mac value from
first to last byte, therefore after detecting such drivers,
a transitional device MAY defer MAC update, or MAY defer
processing incoming packets until driver writes the last byte
of \field{mac} in the config space.

\paragraph{VLAN Filtering}\label{sec:Device Types / Network Device / Device Operation / Control Virtqueue / VLAN Filtering}

If the driver negotiates the VIRTIO_NET_F_CTRL_VLAN feature, it
can control a VLAN filter table in the device. The VLAN filter
table applies only to VLAN tagged packets.

When VIRTIO_NET_F_CTRL_VLAN is negotiated, the device starts with
an empty VLAN filter table.

\begin{note}
Similar to the MAC address based filtering, the VLAN filtering
is also best-effort: unwanted packets could still arrive.
\end{note}

\begin{lstlisting}
#define VIRTIO_NET_CTRL_VLAN       2
 #define VIRTIO_NET_CTRL_VLAN_ADD             0
 #define VIRTIO_NET_CTRL_VLAN_DEL             1
\end{lstlisting}

Both the VIRTIO_NET_CTRL_VLAN_ADD and VIRTIO_NET_CTRL_VLAN_DEL
command take a little-endian 16-bit VLAN id as the command-specific-data.

VIRTIO_NET_CTRL_VLAN_ADD command adds the specified VLAN to the
VLAN filter table.

VIRTIO_NET_CTRL_VLAN_DEL command removes the specified VLAN from
the VLAN filter table.

\devicenormative{\subparagraph}{VLAN Filtering}{Device Types / Network Device / Device Operation / Control Virtqueue / VLAN Filtering}

When VIRTIO_NET_F_CTRL_VLAN is not negotiated, the device MUST
accept all VLAN tagged packets.

When VIRTIO_NET_F_CTRL_VLAN is negotiated, the device MUST
accept all VLAN tagged packets whose VLAN tag is present in
the VLAN filter table and SHOULD drop all VLAN tagged packets
whose VLAN tag is absent in the VLAN filter table.

\subparagraph{Legacy Interface: VLAN Filtering}\label{sec:Device Types / Network Device / Device Operation / Control Virtqueue / VLAN Filtering / Legacy Interface: VLAN Filtering}
When using the legacy interface, transitional devices and drivers
MUST format the VLAN id
according to the native endian of the guest rather than
(necessarily when not using the legacy interface) little-endian.

\paragraph{Gratuitous Packet Sending}\label{sec:Device Types / Network Device / Device Operation / Control Virtqueue / Gratuitous Packet Sending}

If the driver negotiates the VIRTIO_NET_F_GUEST_ANNOUNCE (depends
on VIRTIO_NET_F_CTRL_VQ), the device can ask the driver to send gratuitous
packets; this is usually done after the guest has been physically
migrated, and needs to announce its presence on the new network
links. (As hypervisor does not have the knowledge of guest
network configuration (eg. tagged vlan) it is simplest to prod
the guest in this way).

\begin{lstlisting}
#define VIRTIO_NET_CTRL_ANNOUNCE       3
 #define VIRTIO_NET_CTRL_ANNOUNCE_ACK             0
\end{lstlisting}

The driver checks VIRTIO_NET_S_ANNOUNCE bit in the device configuration \field{status} field
when it notices the changes of device configuration. The
command VIRTIO_NET_CTRL_ANNOUNCE_ACK is used to indicate that
driver has received the notification and device clears the
VIRTIO_NET_S_ANNOUNCE bit in \field{status}.

Processing this notification involves:

\begin{enumerate}
\item Sending the gratuitous packets (eg. ARP) or marking there are pending
  gratuitous packets to be sent and letting deferred routine to
  send them.

\item Sending VIRTIO_NET_CTRL_ANNOUNCE_ACK command through control
  vq.
\end{enumerate}

\drivernormative{\subparagraph}{Gratuitous Packet Sending}{Device Types / Network Device / Device Operation / Control Virtqueue / Gratuitous Packet Sending}

If the driver negotiates VIRTIO_NET_F_GUEST_ANNOUNCE, it SHOULD notify
network peers of its new location after it sees the VIRTIO_NET_S_ANNOUNCE bit
in \field{status}.  The driver MUST send a command on the command queue
with class VIRTIO_NET_CTRL_ANNOUNCE and command VIRTIO_NET_CTRL_ANNOUNCE_ACK.

\devicenormative{\subparagraph}{Gratuitous Packet Sending}{Device Types / Network Device / Device Operation / Control Virtqueue / Gratuitous Packet Sending}

If VIRTIO_NET_F_GUEST_ANNOUNCE is negotiated, the device MUST clear the
VIRTIO_NET_S_ANNOUNCE bit in \field{status} upon receipt of a command buffer
with class VIRTIO_NET_CTRL_ANNOUNCE and command VIRTIO_NET_CTRL_ANNOUNCE_ACK
before marking the buffer as used.

\paragraph{Device operation in multiqueue mode}\label{sec:Device Types / Network Device / Device Operation / Control Virtqueue / Device operation in multiqueue mode}

This specification defines the following modes that a device MAY implement for operation with multiple transmit/receive virtqueues:
\begin{itemize}
\item Automatic receive steering as defined in \ref{sec:Device Types / Network Device / Device Operation / Control Virtqueue / Automatic receive steering in multiqueue mode}.
 If a device supports this mode, it offers the VIRTIO_NET_F_MQ feature bit.
\item Receive-side scaling as defined in \ref{devicenormative:Device Types / Network Device / Device Operation / Control Virtqueue / Receive-side scaling (RSS) / RSS processing}.
 If a device supports this mode, it offers the VIRTIO_NET_F_RSS feature bit.
\end{itemize}

A device MAY support one of these features or both. The driver MAY negotiate any set of these features that the device supports.

Multiqueue is disabled by default.

The driver enables multiqueue by sending a command using \field{class} VIRTIO_NET_CTRL_MQ. The \field{command} selects the mode of multiqueue operation, as follows:
\begin{lstlisting}
#define VIRTIO_NET_CTRL_MQ    4
 #define VIRTIO_NET_CTRL_MQ_VQ_PAIRS_SET        0 (for automatic receive steering)
 #define VIRTIO_NET_CTRL_MQ_RSS_CONFIG          1 (for configurable receive steering)
 #define VIRTIO_NET_CTRL_MQ_HASH_CONFIG         2 (for configurable hash calculation)
\end{lstlisting}

If more than one multiqueue mode is negotiated, the resulting device configuration is defined by the last command sent by the driver.

\paragraph{Automatic receive steering in multiqueue mode}\label{sec:Device Types / Network Device / Device Operation / Control Virtqueue / Automatic receive steering in multiqueue mode}

If the driver negotiates the VIRTIO_NET_F_MQ feature bit (depends on VIRTIO_NET_F_CTRL_VQ), it MAY transmit outgoing packets on one
of the multiple transmitq1\ldots transmitqN and ask the device to
queue incoming packets into one of the multiple receiveq1\ldots receiveqN
depending on the packet flow.

The driver enables multiqueue by
sending the VIRTIO_NET_CTRL_MQ_VQ_PAIRS_SET command, specifying
the number of the transmit and receive queues to be used up to
\field{max_virtqueue_pairs}; subsequently,
transmitq1\ldots transmitqn and receiveq1\ldots receiveqn where
n=\field{virtqueue_pairs} MAY be used.
\begin{lstlisting}
struct virtio_net_ctrl_mq_pairs_set {
       le16 virtqueue_pairs;
};
#define VIRTIO_NET_CTRL_MQ_VQ_PAIRS_MIN        1
#define VIRTIO_NET_CTRL_MQ_VQ_PAIRS_MAX        0x8000

\end{lstlisting}

When multiqueue is enabled by VIRTIO_NET_CTRL_MQ_VQ_PAIRS_SET command, the device MUST use automatic receive steering
based on packet flow. Programming of the receive steering
classificator is implicit. After the driver transmitted a packet of a
flow on transmitqX, the device SHOULD cause incoming packets for that flow to
be steered to receiveqX. For uni-directional protocols, or where
no packets have been transmitted yet, the device MAY steer a packet
to a random queue out of the specified receiveq1\ldots receiveqn.

Multiqueue is disabled by VIRTIO_NET_CTRL_MQ_VQ_PAIRS_SET with \field{virtqueue_pairs} to 1 (this is
the default) and waiting for the device to use the command buffer.

\drivernormative{\subparagraph}{Automatic receive steering in multiqueue mode}{Device Types / Network Device / Device Operation / Control Virtqueue / Automatic receive steering in multiqueue mode}

The driver MUST configure the virtqueues before enabling them with the
VIRTIO_NET_CTRL_MQ_VQ_PAIRS_SET command.

The driver MUST NOT request a \field{virtqueue_pairs} of 0 or
greater than \field{max_virtqueue_pairs} in the device configuration space.

The driver MUST queue packets only on any transmitq1 before the
VIRTIO_NET_CTRL_MQ_VQ_PAIRS_SET command.

The driver MUST NOT queue packets on transmit queues greater than
\field{virtqueue_pairs} once it has placed the VIRTIO_NET_CTRL_MQ_VQ_PAIRS_SET command in the available ring.

\devicenormative{\subparagraph}{Automatic receive steering in multiqueue mode}{Device Types / Network Device / Device Operation / Control Virtqueue / Automatic receive steering in multiqueue mode}

After initialization of reset, the device MUST queue packets only on receiveq1.

The device MUST NOT queue packets on receive queues greater than
\field{virtqueue_pairs} once it has placed the
VIRTIO_NET_CTRL_MQ_VQ_PAIRS_SET command in a used buffer.

If the destination receive queue is being reset (See \ref{sec:Basic Facilities of a Virtio Device / Virtqueues / Virtqueue Reset}),
the device SHOULD re-select another random queue. If all receive queues are
being reset, the device MUST drop the packet.

\subparagraph{Legacy Interface: Automatic receive steering in multiqueue mode}\label{sec:Device Types / Network Device / Device Operation / Control Virtqueue / Automatic receive steering in multiqueue mode / Legacy Interface: Automatic receive steering in multiqueue mode}
When using the legacy interface, transitional devices and drivers
MUST format \field{virtqueue_pairs}
according to the native endian of the guest rather than
(necessarily when not using the legacy interface) little-endian.

\subparagraph{Hash calculation}\label{sec:Device Types / Network Device / Device Operation / Control Virtqueue / Automatic receive steering in multiqueue mode / Hash calculation}
If VIRTIO_NET_F_HASH_REPORT was negotiated and the device uses automatic receive steering,
the device MUST support a command to configure hash calculation parameters.

The driver provides parameters for hash calculation as follows:

\field{class} VIRTIO_NET_CTRL_MQ, \field{command} VIRTIO_NET_CTRL_MQ_HASH_CONFIG.

The \field{command-specific-data} has following format:
\begin{lstlisting}
struct virtio_net_hash_config {
    le32 hash_types;
    le16 reserved[4];
    u8 hash_key_length;
    u8 hash_key_data[hash_key_length];
};
\end{lstlisting}
Field \field{hash_types} contains a bitmask of allowed hash types as
defined in
\ref{sec:Device Types / Network Device / Device Operation / Processing of Incoming Packets / Hash calculation for incoming packets / Supported/enabled hash types}.
Initially the device has all hash types disabled and reports only VIRTIO_NET_HASH_REPORT_NONE.

Field \field{reserved} MUST contain zeroes. It is defined to make the structure to match the layout of virtio_net_rss_config structure,
defined in \ref{sec:Device Types / Network Device / Device Operation / Control Virtqueue / Receive-side scaling (RSS)}.

Fields \field{hash_key_length} and \field{hash_key_data} define the key to be used in hash calculation.

\paragraph{Receive-side scaling (RSS)}\label{sec:Device Types / Network Device / Device Operation / Control Virtqueue / Receive-side scaling (RSS)}
A device offers the feature VIRTIO_NET_F_RSS if it supports RSS receive steering with Toeplitz hash calculation and configurable parameters.

A driver queries RSS capabilities of the device by reading device configuration as defined in \ref{sec:Device Types / Network Device / Device configuration layout}

\subparagraph{Setting RSS parameters}\label{sec:Device Types / Network Device / Device Operation / Control Virtqueue / Receive-side scaling (RSS) / Setting RSS parameters}

Driver sends a VIRTIO_NET_CTRL_MQ_RSS_CONFIG command using the following format for \field{command-specific-data}:
\begin{lstlisting}
struct rss_rq_id {
   le16 vq_index_1_16: 15; /* Bits 1 to 16 of the virtqueue index */
   le16 reserved: 1; /* Set to zero */
};

struct virtio_net_rss_config {
    le32 hash_types;
    le16 indirection_table_mask;
    struct rss_rq_id unclassified_queue;
    struct rss_rq_id indirection_table[indirection_table_length];
    le16 max_tx_vq;
    u8 hash_key_length;
    u8 hash_key_data[hash_key_length];
};
\end{lstlisting}
Field \field{hash_types} contains a bitmask of allowed hash types as
defined in
\ref{sec:Device Types / Network Device / Device Operation / Processing of Incoming Packets / Hash calculation for incoming packets / Supported/enabled hash types}.

Field \field{indirection_table_mask} is a mask to be applied to
the calculated hash to produce an index in the
\field{indirection_table} array.
Number of entries in \field{indirection_table} is (\field{indirection_table_mask} + 1).

\field{rss_rq_id} is a receive virtqueue id. \field{vq_index_1_16}
consists of bits 1 to 16 of a virtqueue index. For example, a
\field{vq_index_1_16} value of 3 corresponds to virtqueue index 6,
which maps to receiveq4.

Field \field{unclassified_queue} specifies the receive virtqueue id in which to
place unclassified packets.

Field \field{indirection_table} is an array of receive virtqueues ids.

A driver sets \field{max_tx_vq} to inform a device how many transmit virtqueues it may use (transmitq1\ldots transmitq \field{max_tx_vq}).

Fields \field{hash_key_length} and \field{hash_key_data} define the key to be used in hash calculation.

\drivernormative{\subparagraph}{Setting RSS parameters}{Device Types / Network Device / Device Operation / Control Virtqueue / Receive-side scaling (RSS) }

A driver MUST NOT send the VIRTIO_NET_CTRL_MQ_RSS_CONFIG command if the feature VIRTIO_NET_F_RSS has not been negotiated.

A driver MUST fill the \field{indirection_table} array only with
enabled receive virtqueues ids.

The number of entries in \field{indirection_table} (\field{indirection_table_mask} + 1) MUST be a power of two.

A driver MUST use \field{indirection_table_mask} values that are less than \field{rss_max_indirection_table_length} reported by a device.

A driver MUST NOT set any VIRTIO_NET_HASH_TYPE_ flags that are not supported by a device.

\devicenormative{\subparagraph}{RSS processing}{Device Types / Network Device / Device Operation / Control Virtqueue / Receive-side scaling (RSS) / RSS processing}
The device MUST determine the destination queue for a network packet as follows:
\begin{itemize}
\item Calculate the hash of the packet as defined in \ref{sec:Device Types / Network Device / Device Operation / Processing of Incoming Packets / Hash calculation for incoming packets}.
\item If the device did not calculate the hash for the specific packet, the device directs the packet to the receiveq specified by \field{unclassified_queue} of virtio_net_rss_config structure.
\item Apply \field{indirection_table_mask} to the calculated hash
and use the result as the index in the indirection table to get
the destination receive virtqueue id.
\item If the destination receive queue is being reset (See \ref{sec:Basic Facilities of a Virtio Device / Virtqueues / Virtqueue Reset}), the device MUST drop the packet.
\end{itemize}

\paragraph{RSS Context}\label{sec:Device Types / Network Device / Device Operation / Control Virtqueue / RSS Context}

An RSS context consists of configurable parameters specified by \ref{sec:Device Types / Network Device
/ Device Operation / Control Virtqueue / Receive-side scaling (RSS)}.

The RSS configuration supported by VIRTIO_NET_F_RSS is considered the default RSS configuration.

The device offers the feature VIRTIO_NET_F_RSS_CONTEXT if it supports one or multiple RSS contexts
(excluding the default RSS configuration) and configurable parameters.

\subparagraph{Querying RSS Context Capability}\label{sec:Device Types / Network Device / Device Operation / Control Virtqueue / RSS Context / Querying RSS Context Capability}

\begin{lstlisting}
#define VIRTNET_RSS_CTX_CTRL 9
 #define VIRTNET_RSS_CTX_CTRL_CAP_GET  0
 #define VIRTNET_RSS_CTX_CTRL_ADD      1
 #define VIRTNET_RSS_CTX_CTRL_MOD      2
 #define VIRTNET_RSS_CTX_CTRL_DEL      3

struct virtnet_rss_ctx_cap {
    le16 max_rss_contexts;
}
\end{lstlisting}

Field \field{max_rss_contexts} specifies the maximum number of RSS contexts \ref{sec:Device Types / Network Device /
Device Operation / Control Virtqueue / RSS Context} supported by the device.

The driver queries the RSS context capability of the device by sending the command VIRTNET_RSS_CTX_CTRL_CAP_GET
with the structure virtnet_rss_ctx_cap.

For the command VIRTNET_RSS_CTX_CTRL_CAP_GET, the structure virtnet_rss_ctx_cap is write-only for the device.

\subparagraph{Setting RSS Context Parameters}\label{sec:Device Types / Network Device / Device Operation / Control Virtqueue / RSS Context / Setting RSS Context Parameters}

\begin{lstlisting}
struct virtnet_rss_ctx_add_modify {
    le16 rss_ctx_id;
    u8 reserved[6];
    struct virtio_net_rss_config rss;
};

struct virtnet_rss_ctx_del {
    le16 rss_ctx_id;
};
\end{lstlisting}

RSS context parameters:
\begin{itemize}
\item  \field{rss_ctx_id}: ID of the specific RSS context.
\item  \field{rss}: RSS context parameters of the specific RSS context whose id is \field{rss_ctx_id}.
\end{itemize}

\field{reserved} is reserved and it is ignored by the device.

If the feature VIRTIO_NET_F_RSS_CONTEXT has been negotiated, the driver can send the following
VIRTNET_RSS_CTX_CTRL class commands:
\begin{enumerate}
\item VIRTNET_RSS_CTX_CTRL_ADD: use the structure virtnet_rss_ctx_add_modify to
       add an RSS context configured as \field{rss} and id as \field{rss_ctx_id} for the device.
\item VIRTNET_RSS_CTX_CTRL_MOD: use the structure virtnet_rss_ctx_add_modify to
       configure parameters of the RSS context whose id is \field{rss_ctx_id} as \field{rss} for the device.
\item VIRTNET_RSS_CTX_CTRL_DEL: use the structure virtnet_rss_ctx_del to delete
       the RSS context whose id is \field{rss_ctx_id} for the device.
\end{enumerate}

For commands VIRTNET_RSS_CTX_CTRL_ADD and VIRTNET_RSS_CTX_CTRL_MOD, the structure virtnet_rss_ctx_add_modify is read-only for the device.
For the command VIRTNET_RSS_CTX_CTRL_DEL, the structure virtnet_rss_ctx_del is read-only for the device.

\devicenormative{\subparagraph}{RSS Context}{Device Types / Network Device / Device Operation / Control Virtqueue / RSS Context}

The device MUST set \field{max_rss_contexts} to at least 1 if it offers VIRTIO_NET_F_RSS_CONTEXT.

Upon reset, the device MUST clear all previously configured RSS contexts.

\drivernormative{\subparagraph}{RSS Context}{Device Types / Network Device / Device Operation / Control Virtqueue / RSS Context}

The driver MUST have negotiated the VIRTIO_NET_F_RSS_CONTEXT feature when issuing the VIRTNET_RSS_CTX_CTRL class commands.

The driver MUST set \field{rss_ctx_id} to between 1 and \field{max_rss_contexts} inclusive.

The driver MUST NOT send the command VIRTIO_NET_CTRL_MQ_VQ_PAIRS_SET when the device has successfully configured at least one RSS context.

\paragraph{Offloads State Configuration}\label{sec:Device Types / Network Device / Device Operation / Control Virtqueue / Offloads State Configuration}

If the VIRTIO_NET_F_CTRL_GUEST_OFFLOADS feature is negotiated, the driver can
send control commands for dynamic offloads state configuration.

\subparagraph{Setting Offloads State}\label{sec:Device Types / Network Device / Device Operation / Control Virtqueue / Offloads State Configuration / Setting Offloads State}

To configure the offloads, the following layout structure and
definitions are used:

\begin{lstlisting}
le64 offloads;

#define VIRTIO_NET_F_GUEST_CSUM       1
#define VIRTIO_NET_F_GUEST_TSO4       7
#define VIRTIO_NET_F_GUEST_TSO6       8
#define VIRTIO_NET_F_GUEST_ECN        9
#define VIRTIO_NET_F_GUEST_UFO        10
#define VIRTIO_NET_F_GUEST_UDP_TUNNEL_GSO  46
#define VIRTIO_NET_F_GUEST_UDP_TUNNEL_GSO_CSUM 47
#define VIRTIO_NET_F_GUEST_USO4       54
#define VIRTIO_NET_F_GUEST_USO6       55

#define VIRTIO_NET_CTRL_GUEST_OFFLOADS       5
 #define VIRTIO_NET_CTRL_GUEST_OFFLOADS_SET   0
\end{lstlisting}

The class VIRTIO_NET_CTRL_GUEST_OFFLOADS has one command:
VIRTIO_NET_CTRL_GUEST_OFFLOADS_SET applies the new offloads configuration.

le64 value passed as command data is a bitmask, bits set define
offloads to be enabled, bits cleared - offloads to be disabled.

There is a corresponding device feature for each offload. Upon feature
negotiation corresponding offload gets enabled to preserve backward
compatibility.

\drivernormative{\subparagraph}{Setting Offloads State}{Device Types / Network Device / Device Operation / Control Virtqueue / Offloads State Configuration / Setting Offloads State}

A driver MUST NOT enable an offload for which the appropriate feature
has not been negotiated.

\subparagraph{Legacy Interface: Setting Offloads State}\label{sec:Device Types / Network Device / Device Operation / Control Virtqueue / Offloads State Configuration / Setting Offloads State / Legacy Interface: Setting Offloads State}
When using the legacy interface, transitional devices and drivers
MUST format \field{offloads}
according to the native endian of the guest rather than
(necessarily when not using the legacy interface) little-endian.


\paragraph{Notifications Coalescing}\label{sec:Device Types / Network Device / Device Operation / Control Virtqueue / Notifications Coalescing}

If the VIRTIO_NET_F_NOTF_COAL feature is negotiated, the driver can
send commands VIRTIO_NET_CTRL_NOTF_COAL_TX_SET and VIRTIO_NET_CTRL_NOTF_COAL_RX_SET
for notification coalescing.

If the VIRTIO_NET_F_VQ_NOTF_COAL feature is negotiated, the driver can
send commands VIRTIO_NET_CTRL_NOTF_COAL_VQ_SET and VIRTIO_NET_CTRL_NOTF_COAL_VQ_GET
for virtqueue notification coalescing.

\begin{lstlisting}
struct virtio_net_ctrl_coal {
    le32 max_packets;
    le32 max_usecs;
};

struct virtio_net_ctrl_coal_vq {
    le16 vq_index;
    le16 reserved;
    struct virtio_net_ctrl_coal coal;
};

#define VIRTIO_NET_CTRL_NOTF_COAL 6
 #define VIRTIO_NET_CTRL_NOTF_COAL_TX_SET  0
 #define VIRTIO_NET_CTRL_NOTF_COAL_RX_SET 1
 #define VIRTIO_NET_CTRL_NOTF_COAL_VQ_SET 2
 #define VIRTIO_NET_CTRL_NOTF_COAL_VQ_GET 3
\end{lstlisting}

Coalescing parameters:
\begin{itemize}
\item \field{vq_index}: The virtqueue index of an enabled transmit or receive virtqueue.
\item \field{max_usecs} for RX: Maximum number of microseconds to delay a RX notification.
\item \field{max_usecs} for TX: Maximum number of microseconds to delay a TX notification.
\item \field{max_packets} for RX: Maximum number of packets to receive before a RX notification.
\item \field{max_packets} for TX: Maximum number of packets to send before a TX notification.
\end{itemize}

\field{reserved} is reserved and it is ignored by the device.

Read/Write attributes for coalescing parameters:
\begin{itemize}
\item For commands VIRTIO_NET_CTRL_NOTF_COAL_TX_SET and VIRTIO_NET_CTRL_NOTF_COAL_RX_SET, the structure virtio_net_ctrl_coal is write-only for the driver.
\item For the command VIRTIO_NET_CTRL_NOTF_COAL_VQ_SET, the structure virtio_net_ctrl_coal_vq is write-only for the driver.
\item For the command VIRTIO_NET_CTRL_NOTF_COAL_VQ_GET, \field{vq_index} and \field{reserved} are write-only
      for the driver, and the structure virtio_net_ctrl_coal is read-only for the driver.
\end{itemize}

The class VIRTIO_NET_CTRL_NOTF_COAL has the following commands:
\begin{enumerate}
\item VIRTIO_NET_CTRL_NOTF_COAL_TX_SET: use the structure virtio_net_ctrl_coal to set the \field{max_usecs} and \field{max_packets} parameters for all transmit virtqueues.
\item VIRTIO_NET_CTRL_NOTF_COAL_RX_SET: use the structure virtio_net_ctrl_coal to set the \field{max_usecs} and \field{max_packets} parameters for all receive virtqueues.
\item VIRTIO_NET_CTRL_NOTF_COAL_VQ_SET: use the structure virtio_net_ctrl_coal_vq to set the \field{max_usecs} and \field{max_packets} parameters
                                        for an enabled transmit/receive virtqueue whose index is \field{vq_index}.
\item VIRTIO_NET_CTRL_NOTF_COAL_VQ_GET: use the structure virtio_net_ctrl_coal_vq to get the \field{max_usecs} and \field{max_packets} parameters
                                        for an enabled transmit/receive virtqueue whose index is \field{vq_index}.
\end{enumerate}

The device may generate notifications more or less frequently than specified by set commands of the VIRTIO_NET_CTRL_NOTF_COAL class.

If coalescing parameters are being set, the device applies the last coalescing parameters set for a
virtqueue, regardless of the command used to set the parameters. Use the following command sequence
with two pairs of virtqueues as an example:
Each of the following commands sets \field{max_usecs} and \field{max_packets} parameters for virtqueues.
\begin{itemize}
\item Command1: VIRTIO_NET_CTRL_NOTF_COAL_RX_SET sets coalescing parameters for virtqueues having index 0 and index 2. Virtqueues having index 1 and index 3 retain their previous parameters.
\item Command2: VIRTIO_NET_CTRL_NOTF_COAL_VQ_SET with \field{vq_index} = 0 sets coalescing parameters for virtqueue having index 0. Virtqueue having index 2 retains the parameters from command1.
\item Command3: VIRTIO_NET_CTRL_NOTF_COAL_VQ_GET with \field{vq_index} = 0, the device responds with coalescing parameters of vq_index 0 set by command2.
\item Command4: VIRTIO_NET_CTRL_NOTF_COAL_VQ_SET with \field{vq_index} = 1 sets coalescing parameters for virtqueue having index 1. Virtqueue having index 3 retains its previous parameters.
\item Command5: VIRTIO_NET_CTRL_NOTF_COAL_TX_SET sets coalescing parameters for virtqueues having index 1 and index 3, and overrides the parameters set by command4.
\item Command6: VIRTIO_NET_CTRL_NOTF_COAL_VQ_GET with \field{vq_index} = 1, the device responds with coalescing parameters of index 1 set by command5.
\end{itemize}

\subparagraph{Operation}\label{sec:Device Types / Network Device / Device Operation / Control Virtqueue / Notifications Coalescing / Operation}

The device sends a used buffer notification once the notification conditions are met and if the notifications are not suppressed as explained in \ref{sec:Basic Facilities of a Virtio Device / Virtqueues / Used Buffer Notification Suppression}.

When the device has non-zero \field{max_usecs} and non-zero \field{max_packets}, it starts counting microseconds and packets upon receiving/sending a packet.
The device counts packets and microseconds for each receive virtqueue and transmit virtqueue separately.
In this case, the notification conditions are met when \field{max_usecs} microseconds elapse, or upon sending/receiving \field{max_packets} packets, whichever happens first.
Afterwards, the device waits for the next packet and starts counting packets and microseconds again.

When the device has \field{max_usecs} = 0 or \field{max_packets} = 0, the notification conditions are met after every packet received/sent.

\subparagraph{RX Example}\label{sec:Device Types / Network Device / Device Operation / Control Virtqueue / Notifications Coalescing / RX Example}

If, for example:
\begin{itemize}
\item \field{max_usecs} = 10.
\item \field{max_packets} = 15.
\end{itemize}
then each receive virtqueue of a device will operate as follows:
\begin{itemize}
\item The device will count packets received on each virtqueue until it accumulates 15, or until 10 microseconds elapsed since the first one was received.
\item If the notifications are not suppressed by the driver, the device will send an used buffer notification, otherwise, the device will not send an used buffer notification as long as the notifications are suppressed.
\end{itemize}

\subparagraph{TX Example}\label{sec:Device Types / Network Device / Device Operation / Control Virtqueue / Notifications Coalescing / TX Example}

If, for example:
\begin{itemize}
\item \field{max_usecs} = 10.
\item \field{max_packets} = 15.
\end{itemize}
then each transmit virtqueue of a device will operate as follows:
\begin{itemize}
\item The device will count packets sent on each virtqueue until it accumulates 15, or until 10 microseconds elapsed since the first one was sent.
\item If the notifications are not suppressed by the driver, the device will send an used buffer notification, otherwise, the device will not send an used buffer notification as long as the notifications are suppressed.
\end{itemize}

\subparagraph{Notifications When Coalescing Parameters Change}\label{sec:Device Types / Network Device / Device Operation / Control Virtqueue / Notifications Coalescing / Notifications When Coalescing Parameters Change}

When the coalescing parameters of a device change, the device needs to check if the new notification conditions are met and send a used buffer notification if so.

For example, \field{max_packets} = 15 for a device with a single transmit virtqueue: if the device sends 10 packets and afterwards receives a
VIRTIO_NET_CTRL_NOTF_COAL_TX_SET command with \field{max_packets} = 8, then the notification condition is immediately considered to be met;
the device needs to immediately send a used buffer notification, if the notifications are not suppressed by the driver.

\drivernormative{\subparagraph}{Notifications Coalescing}{Device Types / Network Device / Device Operation / Control Virtqueue / Notifications Coalescing}

The driver MUST set \field{vq_index} to the virtqueue index of an enabled transmit or receive virtqueue.

The driver MUST have negotiated the VIRTIO_NET_F_NOTF_COAL feature when issuing commands VIRTIO_NET_CTRL_NOTF_COAL_TX_SET and VIRTIO_NET_CTRL_NOTF_COAL_RX_SET.

The driver MUST have negotiated the VIRTIO_NET_F_VQ_NOTF_COAL feature when issuing commands VIRTIO_NET_CTRL_NOTF_COAL_VQ_SET and VIRTIO_NET_CTRL_NOTF_COAL_VQ_GET.

The driver MUST ignore the values of coalescing parameters received from the VIRTIO_NET_CTRL_NOTF_COAL_VQ_GET command if the device responds with VIRTIO_NET_ERR.

\devicenormative{\subparagraph}{Notifications Coalescing}{Device Types / Network Device / Device Operation / Control Virtqueue / Notifications Coalescing}

The device MUST ignore \field{reserved}.

The device SHOULD respond to VIRTIO_NET_CTRL_NOTF_COAL_TX_SET and VIRTIO_NET_CTRL_NOTF_COAL_RX_SET commands with VIRTIO_NET_ERR if it was not able to change the parameters.

The device MUST respond to the VIRTIO_NET_CTRL_NOTF_COAL_VQ_SET command with VIRTIO_NET_ERR if it was not able to change the parameters.

The device MUST respond to VIRTIO_NET_CTRL_NOTF_COAL_VQ_SET and VIRTIO_NET_CTRL_NOTF_COAL_VQ_GET commands with
VIRTIO_NET_ERR if the designated virtqueue is not an enabled transmit or receive virtqueue.

Upon disabling and re-enabling a transmit virtqueue, the device MUST set the coalescing parameters of the virtqueue
to those configured through the VIRTIO_NET_CTRL_NOTF_COAL_TX_SET command, or, if the driver did not set any TX coalescing parameters, to 0.

Upon disabling and re-enabling a receive virtqueue, the device MUST set the coalescing parameters of the virtqueue
to those configured through the VIRTIO_NET_CTRL_NOTF_COAL_RX_SET command, or, if the driver did not set any RX coalescing parameters, to 0.

The behavior of the device in response to set commands of the VIRTIO_NET_CTRL_NOTF_COAL class is best-effort:
the device MAY generate notifications more or less frequently than specified.

A device SHOULD NOT send used buffer notifications to the driver if the notifications are suppressed, even if the notification conditions are met.

Upon reset, a device MUST initialize all coalescing parameters to 0.

\paragraph{Device Statistics}\label{sec:Device Types / Network Device / Device Operation / Control Virtqueue / Device Statistics}

If the VIRTIO_NET_F_DEVICE_STATS feature is negotiated, the driver can obtain
device statistics from the device by using the following command.

Different types of virtqueues have different statistics. The statistics of the
receiveq are different from those of the transmitq.

The statistics of a certain type of virtqueue are also divided into multiple types
because different types require different features. This enables the expansion
of new statistics.

In one command, the driver can obtain the statistics of one or multiple virtqueues.
Additionally, the driver can obtain multiple type statistics of each virtqueue.

\subparagraph{Query Statistic Capabilities}\label{sec:Device Types / Network Device / Device Operation / Control Virtqueue / Device Statistics / Query Statistic Capabilities}

\begin{lstlisting}
#define VIRTIO_NET_CTRL_STATS         8
#define VIRTIO_NET_CTRL_STATS_QUERY   0
#define VIRTIO_NET_CTRL_STATS_GET     1

struct virtio_net_stats_capabilities {

#define VIRTIO_NET_STATS_TYPE_CVQ       (1 << 32)

#define VIRTIO_NET_STATS_TYPE_RX_BASIC  (1 << 0)
#define VIRTIO_NET_STATS_TYPE_RX_CSUM   (1 << 1)
#define VIRTIO_NET_STATS_TYPE_RX_GSO    (1 << 2)
#define VIRTIO_NET_STATS_TYPE_RX_SPEED  (1 << 3)

#define VIRTIO_NET_STATS_TYPE_TX_BASIC  (1 << 16)
#define VIRTIO_NET_STATS_TYPE_TX_CSUM   (1 << 17)
#define VIRTIO_NET_STATS_TYPE_TX_GSO    (1 << 18)
#define VIRTIO_NET_STATS_TYPE_TX_SPEED  (1 << 19)

    le64 supported_stats_types[1];
}
\end{lstlisting}

To obtain device statistic capability, use the VIRTIO_NET_CTRL_STATS_QUERY
command. When the command completes successfully, \field{command-specific-result}
is in the format of \field{struct virtio_net_stats_capabilities}.

\subparagraph{Get Statistics}\label{sec:Device Types / Network Device / Device Operation / Control Virtqueue / Device Statistics / Get Statistics}

\begin{lstlisting}
struct virtio_net_ctrl_queue_stats {
       struct {
           le16 vq_index;
           le16 reserved[3];
           le64 types_bitmap[1];
       } stats[];
};

struct virtio_net_stats_reply_hdr {
#define VIRTIO_NET_STATS_TYPE_REPLY_CVQ       32

#define VIRTIO_NET_STATS_TYPE_REPLY_RX_BASIC  0
#define VIRTIO_NET_STATS_TYPE_REPLY_RX_CSUM   1
#define VIRTIO_NET_STATS_TYPE_REPLY_RX_GSO    2
#define VIRTIO_NET_STATS_TYPE_REPLY_RX_SPEED  3

#define VIRTIO_NET_STATS_TYPE_REPLY_TX_BASIC  16
#define VIRTIO_NET_STATS_TYPE_REPLY_TX_CSUM   17
#define VIRTIO_NET_STATS_TYPE_REPLY_TX_GSO    18
#define VIRTIO_NET_STATS_TYPE_REPLY_TX_SPEED  19
    u8 type;
    u8 reserved;
    le16 vq_index;
    le16 reserved1;
    le16 size;
}
\end{lstlisting}

To obtain device statistics, use the VIRTIO_NET_CTRL_STATS_GET command with the
\field{command-specific-data} which is in the format of
\field{struct virtio_net_ctrl_queue_stats}. When the command completes
successfully, \field{command-specific-result} contains multiple statistic
results, each statistic result has the \field{struct virtio_net_stats_reply_hdr}
as the header.

The fields of the \field{struct virtio_net_ctrl_queue_stats}:
\begin{description}
    \item [vq_index]
        The index of the virtqueue to obtain the statistics.

    \item [types_bitmap]
        This is a bitmask of the types of statistics to be obtained. Therefore, a
        \field{stats} inside \field{struct virtio_net_ctrl_queue_stats} may
        indicate multiple statistic replies for the virtqueue.
\end{description}

The fields of the \field{struct virtio_net_stats_reply_hdr}:
\begin{description}
    \item [type]
        The type of the reply statistic.

    \item [vq_index]
        The virtqueue index of the reply statistic.

    \item [size]
        The number of bytes for the statistics entry including size of \field{struct virtio_net_stats_reply_hdr}.

\end{description}

\subparagraph{Controlq Statistics}\label{sec:Device Types / Network Device / Device Operation / Control Virtqueue / Device Statistics / Controlq Statistics}

The structure corresponding to the controlq statistics is
\field{struct virtio_net_stats_cvq}. The corresponding type is
VIRTIO_NET_STATS_TYPE_CVQ. This is for the controlq.

\begin{lstlisting}
struct virtio_net_stats_cvq {
    struct virtio_net_stats_reply_hdr hdr;

    le64 command_num;
    le64 ok_num;
};
\end{lstlisting}

\begin{description}
    \item [command_num]
        The number of commands received by the device including the current command.

    \item [ok_num]
        The number of commands completed successfully by the device including the current command.
\end{description}


\subparagraph{Receiveq Basic Statistics}\label{sec:Device Types / Network Device / Device Operation / Control Virtqueue / Device Statistics / Receiveq Basic Statistics}

The structure corresponding to the receiveq basic statistics is
\field{struct virtio_net_stats_rx_basic}. The corresponding type is
VIRTIO_NET_STATS_TYPE_RX_BASIC. This is for the receiveq.

Receiveq basic statistics do not require any feature. As long as the device supports
VIRTIO_NET_F_DEVICE_STATS, the following are the receiveq basic statistics.

\begin{lstlisting}
struct virtio_net_stats_rx_basic {
    struct virtio_net_stats_reply_hdr hdr;

    le64 rx_notifications;

    le64 rx_packets;
    le64 rx_bytes;

    le64 rx_interrupts;

    le64 rx_drops;
    le64 rx_drop_overruns;
};
\end{lstlisting}

The packets described below were all presented on the specified virtqueue.
\begin{description}
    \item [rx_notifications]
        The number of driver notifications received by the device for this
        receiveq.

    \item [rx_packets]
        This is the number of packets passed to the driver by the device.

    \item [rx_bytes]
        This is the bytes of packets passed to the driver by the device.

    \item [rx_interrupts]
        The number of interrupts generated by the device for this receiveq.

    \item [rx_drops]
        This is the number of packets dropped by the device. The count includes
        all types of packets dropped by the device.

    \item [rx_drop_overruns]
        This is the number of packets dropped by the device when no more
        descriptors were available.

\end{description}

\subparagraph{Transmitq Basic Statistics}\label{sec:Device Types / Network Device / Device Operation / Control Virtqueue / Device Statistics / Transmitq Basic Statistics}

The structure corresponding to the transmitq basic statistics is
\field{struct virtio_net_stats_tx_basic}. The corresponding type is
VIRTIO_NET_STATS_TYPE_TX_BASIC. This is for the transmitq.

Transmitq basic statistics do not require any feature. As long as the device supports
VIRTIO_NET_F_DEVICE_STATS, the following are the transmitq basic statistics.

\begin{lstlisting}
struct virtio_net_stats_tx_basic {
    struct virtio_net_stats_reply_hdr hdr;

    le64 tx_notifications;

    le64 tx_packets;
    le64 tx_bytes;

    le64 tx_interrupts;

    le64 tx_drops;
    le64 tx_drop_malformed;
};
\end{lstlisting}

The packets described below are all for a specific virtqueue.
\begin{description}
    \item [tx_notifications]
        The number of driver notifications received by the device for this
        transmitq.

    \item [tx_packets]
        This is the number of packets sent by the device (not the packets
        got from the driver).

    \item [tx_bytes]
        This is the number of bytes sent by the device for all the sent packets
        (not the bytes sent got from the driver).

    \item [tx_interrupts]
        The number of interrupts generated by the device for this transmitq.

    \item [tx_drops]
        The number of packets dropped by the device. The count includes all
        types of packets dropped by the device.

    \item [tx_drop_malformed]
        The number of packets dropped by the device, when the descriptors are
        malformed. For example, the buffer is too short.
\end{description}

\subparagraph{Receiveq CSUM Statistics}\label{sec:Device Types / Network Device / Device Operation / Control Virtqueue / Device Statistics / Receiveq CSUM Statistics}

The structure corresponding to the receiveq checksum statistics is
\field{struct virtio_net_stats_rx_csum}. The corresponding type is
VIRTIO_NET_STATS_TYPE_RX_CSUM. This is for the receiveq.

Only after the VIRTIO_NET_F_GUEST_CSUM is negotiated, the receiveq checksum
statistics can be obtained.

\begin{lstlisting}
struct virtio_net_stats_rx_csum {
    struct virtio_net_stats_reply_hdr hdr;

    le64 rx_csum_valid;
    le64 rx_needs_csum;
    le64 rx_csum_none;
    le64 rx_csum_bad;
};
\end{lstlisting}

The packets described below were all presented on the specified virtqueue.
\begin{description}
    \item [rx_csum_valid]
        The number of packets with VIRTIO_NET_HDR_F_DATA_VALID.

    \item [rx_needs_csum]
        The number of packets with VIRTIO_NET_HDR_F_NEEDS_CSUM.

    \item [rx_csum_none]
        The number of packets without hardware checksum. The packet here refers
        to the non-TCP/UDP packet that the device cannot recognize.

    \item [rx_csum_bad]
        The number of packets with checksum mismatch.

\end{description}

\subparagraph{Transmitq CSUM Statistics}\label{sec:Device Types / Network Device / Device Operation / Control Virtqueue / Device Statistics / Transmitq CSUM Statistics}

The structure corresponding to the transmitq checksum statistics is
\field{struct virtio_net_stats_tx_csum}. The corresponding type is
VIRTIO_NET_STATS_TYPE_TX_CSUM. This is for the transmitq.

Only after the VIRTIO_NET_F_CSUM is negotiated, the transmitq checksum
statistics can be obtained.

The following are the transmitq checksum statistics:

\begin{lstlisting}
struct virtio_net_stats_tx_csum {
    struct virtio_net_stats_reply_hdr hdr;

    le64 tx_csum_none;
    le64 tx_needs_csum;
};
\end{lstlisting}

The packets described below are all for a specific virtqueue.
\begin{description}
    \item [tx_csum_none]
        The number of packets which do not require hardware checksum.

    \item [tx_needs_csum]
        The number of packets which require checksum calculation by the device.

\end{description}

\subparagraph{Receiveq GSO Statistics}\label{sec:Device Types / Network Device / Device Operation / Control Virtqueue / Device Statistics / Receiveq GSO Statistics}

The structure corresponding to the receivq GSO statistics is
\field{struct virtio_net_stats_rx_gso}. The corresponding type is
VIRTIO_NET_STATS_TYPE_RX_GSO. This is for the receiveq.

If one or more of the VIRTIO_NET_F_GUEST_TSO4, VIRTIO_NET_F_GUEST_TSO6
have been negotiated, the receiveq GSO statistics can be obtained.

GSO packets refer to packets passed by the device to the driver where
\field{gso_type} is not VIRTIO_NET_HDR_GSO_NONE.

\begin{lstlisting}
struct virtio_net_stats_rx_gso {
    struct virtio_net_stats_reply_hdr hdr;

    le64 rx_gso_packets;
    le64 rx_gso_bytes;
    le64 rx_gso_packets_coalesced;
    le64 rx_gso_bytes_coalesced;
};
\end{lstlisting}

The packets described below were all presented on the specified virtqueue.
\begin{description}
    \item [rx_gso_packets]
        The number of the GSO packets received by the device.

    \item [rx_gso_bytes]
        The bytes of the GSO packets received by the device.
        This includes the header size of the GSO packet.

    \item [rx_gso_packets_coalesced]
        The number of the GSO packets coalesced by the device.

    \item [rx_gso_bytes_coalesced]
        The bytes of the GSO packets coalesced by the device.
        This includes the header size of the GSO packet.
\end{description}

\subparagraph{Transmitq GSO Statistics}\label{sec:Device Types / Network Device / Device Operation / Control Virtqueue / Device Statistics / Transmitq GSO Statistics}

The structure corresponding to the transmitq GSO statistics is
\field{struct virtio_net_stats_tx_gso}. The corresponding type is
VIRTIO_NET_STATS_TYPE_TX_GSO. This is for the transmitq.

If one or more of the VIRTIO_NET_F_HOST_TSO4, VIRTIO_NET_F_HOST_TSO6,
VIRTIO_NET_F_HOST_USO options have been negotiated, the transmitq GSO statistics
can be obtained.

GSO packets refer to packets passed by the driver to the device where
\field{gso_type} is not VIRTIO_NET_HDR_GSO_NONE.
See more \ref{sec:Device Types / Network Device / Device Operation / Packet
Transmission}.

\begin{lstlisting}
struct virtio_net_stats_tx_gso {
    struct virtio_net_stats_reply_hdr hdr;

    le64 tx_gso_packets;
    le64 tx_gso_bytes;
    le64 tx_gso_segments;
    le64 tx_gso_segments_bytes;
    le64 tx_gso_packets_noseg;
    le64 tx_gso_bytes_noseg;
};
\end{lstlisting}

The packets described below are all for a specific virtqueue.
\begin{description}
    \item [tx_gso_packets]
        The number of the GSO packets sent by the device.

    \item [tx_gso_bytes]
        The bytes of the GSO packets sent by the device.

    \item [tx_gso_segments]
        The number of segments prepared from GSO packets.

    \item [tx_gso_segments_bytes]
        The bytes of segments prepared from GSO packets.

    \item [tx_gso_packets_noseg]
        The number of the GSO packets without segmentation.

    \item [tx_gso_bytes_noseg]
        The bytes of the GSO packets without segmentation.

\end{description}

\subparagraph{Receiveq Speed Statistics}\label{sec:Device Types / Network Device / Device Operation / Control Virtqueue / Device Statistics / Receiveq Speed Statistics}

The structure corresponding to the receiveq speed statistics is
\field{struct virtio_net_stats_rx_speed}. The corresponding type is
VIRTIO_NET_STATS_TYPE_RX_SPEED. This is for the receiveq.

The device has the allowance for the speed. If VIRTIO_NET_F_SPEED_DUPLEX has
been negotiated, the driver can get this by \field{speed}. When the received
packets bitrate exceeds the \field{speed}, some packets may be dropped by the
device.

\begin{lstlisting}
struct virtio_net_stats_rx_speed {
    struct virtio_net_stats_reply_hdr hdr;

    le64 rx_packets_allowance_exceeded;
    le64 rx_bytes_allowance_exceeded;
};
\end{lstlisting}

The packets described below were all presented on the specified virtqueue.
\begin{description}
    \item [rx_packets_allowance_exceeded]
        The number of the packets dropped by the device due to the received
        packets bitrate exceeding the \field{speed}.

    \item [rx_bytes_allowance_exceeded]
        The bytes of the packets dropped by the device due to the received
        packets bitrate exceeding the \field{speed}.

\end{description}

\subparagraph{Transmitq Speed Statistics}\label{sec:Device Types / Network Device / Device Operation / Control Virtqueue / Device Statistics / Transmitq Speed Statistics}

The structure corresponding to the transmitq speed statistics is
\field{struct virtio_net_stats_tx_speed}. The corresponding type is
VIRTIO_NET_STATS_TYPE_TX_SPEED. This is for the transmitq.

The device has the allowance for the speed. If VIRTIO_NET_F_SPEED_DUPLEX has
been negotiated, the driver can get this by \field{speed}. When the transmit
packets bitrate exceeds the \field{speed}, some packets may be dropped by the
device.

\begin{lstlisting}
struct virtio_net_stats_tx_speed {
    struct virtio_net_stats_reply_hdr hdr;

    le64 tx_packets_allowance_exceeded;
    le64 tx_bytes_allowance_exceeded;
};
\end{lstlisting}

The packets described below were all presented on the specified virtqueue.
\begin{description}
    \item [tx_packets_allowance_exceeded]
        The number of the packets dropped by the device due to the transmit packets
        bitrate exceeding the \field{speed}.

    \item [tx_bytes_allowance_exceeded]
        The bytes of the packets dropped by the device due to the transmit packets
        bitrate exceeding the \field{speed}.

\end{description}

\devicenormative{\subparagraph}{Device Statistics}{Device Types / Network Device / Device Operation / Control Virtqueue / Device Statistics}

When the VIRTIO_NET_F_DEVICE_STATS feature is negotiated, the device MUST reply
to the command VIRTIO_NET_CTRL_STATS_QUERY with the
\field{struct virtio_net_stats_capabilities}. \field{supported_stats_types}
includes all the statistic types supported by the device.

If \field{struct virtio_net_ctrl_queue_stats} is incorrect (such as the
following), the device MUST set \field{ack} to VIRTIO_NET_ERR. Even if there is
only one error, the device MUST fail the entire command.
\begin{itemize}
    \item \field{vq_index} exceeds the queue range.
    \item \field{types_bitmap} contains unknown types.
    \item One or more of the bits present in \field{types_bitmap} is not valid
        for the specified virtqueue.
    \item The feature corresponding to the specified \field{types_bitmap} was
        not negotiated.
\end{itemize}

The device MUST set the actual size of the bytes occupied by the reply to the
\field{size} of the \field{hdr}. And the device MUST set the \field{type} and
the \field{vq_index} of the statistic header.

The \field{command-specific-result} buffer allocated by the driver may be
smaller or bigger than all the statistics specified by
\field{struct virtio_net_ctrl_queue_stats}. The device MUST fill up only upto
the valid bytes.

The statistics counter replied by the device MUST wrap around to zero by the
device on the overflow.

\drivernormative{\subparagraph}{Device Statistics}{Device Types / Network Device / Device Operation / Control Virtqueue / Device Statistics}

The types contained in the \field{types_bitmap} MUST be queried from the device
via command VIRTIO_NET_CTRL_STATS_QUERY.

\field{types_bitmap} in \field{struct virtio_net_ctrl_queue_stats} MUST be valid to the
vq specified by \field{vq_index}.

The \field{command-specific-result} buffer allocated by the driver MUST have
enough capacity to store all the statistics reply headers defined in
\field{struct virtio_net_ctrl_queue_stats}. If the
\field{command-specific-result} buffer is fully utilized by the device but some
replies are missed, it is possible that some statistics may exceed the capacity
of the driver's records. In such cases, the driver should allocate additional
space for the \field{command-specific-result} buffer.

\subsubsection{Flow filter}\label{sec:Device Types / Network Device / Device Operation / Flow filter}

A network device can support one or more flow filter rules. Each flow filter rule
is applied by matching a packet and then taking an action, such as directing the packet
to a specific receiveq or dropping the packet. An example of a match is
matching on specific source and destination IP addresses.

A flow filter rule is a device resource object that consists of a key,
a processing priority, and an action to either direct a packet to a
receive queue or drop the packet.

Each rule uses a classifier. The key is matched against the packet using
a classifier, defining which fields in the packet are matched.
A classifier resource object consists of one or more field selectors, each with
a type that specifies the header fields to be matched against, and a mask.
The mask can match whole fields or parts of a field in a header. Each
rule resource object depends on the classifier resource object.

When a packet is received, relevant fields are extracted
(in the same way) from both the packet and the key according to the
classifier. The resulting field contents are then compared -
if they are identical the rule action is taken, if they are not, the rule is ignored.

Multiple flow filter rules are part of a group. The rule resource object
depends on the group. Each rule within a
group has a rule priority, and each group also has a group priority. For a
packet, a group with the highest priority is selected first. Within a group,
rules are applied from highest to lowest priority, until one of the rules
matches the packet and an action is taken. If all the rules within a group
are ignored, the group with the next highest priority is selected, and so on.

The device and the driver indicates flow filter resource limits using the capability
\ref{par:Device Types / Network Device / Device Operation / Flow filter / Device and driver capabilities / VIRTIO-NET-FF-RESOURCE-CAP} specifying the limits on the number of flow filter rule,
group and classifier resource objects. The capability \ref{par:Device Types / Network Device / Device Operation / Flow filter / Device and driver capabilities / VIRTIO-NET-FF-SELECTOR-CAP} specifies which selectors the device supports.
The driver indicates the selectors it is using by setting the flow
filter selector capability, prior to adding any resource objects.

The capability \ref{par:Device Types / Network Device / Device Operation / Flow filter / Device and driver capabilities / VIRTIO-NET-FF-ACTION-CAP} specifies which actions the device supports.

The driver controls the flow filter rule, classifier and group resource objects using
administration commands described in
\ref{sec:Basic Facilities of a Virtio Device / Device groups / Group administration commands / Device resource objects}.

\paragraph{Packet processing order}\label{sec:sec:Device Types / Network Device / Device Operation / Flow filter / Packet processing order}

Note that flow filter rules are applied after MAC/VLAN filtering. Flow filter
rules take precedence over steering: if a flow filter rule results in an action,
the steering configuration does not apply. The steering configuration only applies
to packets for which no flow filter rule action was performed. For example,
incoming packets can be processed in the following order:

\begin{itemize}
\item apply steering configuration received using control virtqueue commands
      VIRTIO_NET_CTRL_RX, VIRTIO_NET_CTRL_MAC and VIRTIO_NET_CTRL_VLAN.
\item apply flow filter rules if any.
\item if no filter rule applied, apply steering configuration received using command
      VIRTIO_NET_CTRL_MQ_RSS_CONFIG or as per automatic receive steering.
\end{itemize}

Some incoming packet processing examples:
\begin{itemize}
\item If the packet is dropped by the flow filter rule, RSS
      steering is ignored for the packet.
\item If the packet is directed to a specific receiveq using flow filter rule,
      the RSS steering is ignored for the packet.
\item If a packet is dropped due to the VIRTIO_NET_CTRL_MAC configuration,
      both flow filter rules and the RSS steering are ignored for the packet.
\item If a packet does not match any flow filter rules,
      the RSS steering is used to select the receiveq for the packet (if enabled).
\item If there are two flow filter groups configured as group_A and group_B
      with respective group priorities as 4, and 5; flow filter rules of
      group_B are applied first having highest group priority, if there is a match,
      the flow filter rules of group_A are ignored; if there is no match for
      the flow filter rules in group_B, the flow filter rules of next level group_A are applied.
\end{itemize}

\paragraph{Device and driver capabilities}
\label{par:Device Types / Network Device / Device Operation / Flow filter / Device and driver capabilities}

\subparagraph{VIRTIO_NET_FF_RESOURCE_CAP}
\label{par:Device Types / Network Device / Device Operation / Flow filter / Device and driver capabilities / VIRTIO-NET-FF-RESOURCE-CAP}

The capability VIRTIO_NET_FF_RESOURCE_CAP indicates the flow filter resource limits.
\field{cap_specific_data} is in the format
\field{struct virtio_net_ff_cap_data}.

\begin{lstlisting}
struct virtio_net_ff_cap_data {
        le32 groups_limit;
        le32 selectors_limit;
        le32 rules_limit;
        le32 rules_per_group_limit;
        u8 last_rule_priority;
        u8 selectors_per_classifier_limit;
};
\end{lstlisting}

\field{groups_limit}, and \field{selectors_limit} represent the maximum
number of flow filter groups and selectors, respectively, that the driver can create.
 \field{rules_limit} is the maximum number of
flow fiilter rules that the driver can create across all the groups.
\field{rules_per_group_limit} is the maximum number of flow filter rules that the driver
can create for each flow filter group.

\field{last_rule_priority} is the highest priority that can be assigned to a
flow filter rule.

\field{selectors_per_classifier_limit} is the maximum number of selectors
that a classifier can have.

\subparagraph{VIRTIO_NET_FF_SELECTOR_CAP}
\label{par:Device Types / Network Device / Device Operation / Flow filter / Device and driver capabilities / VIRTIO-NET-FF-SELECTOR-CAP}

The capability VIRTIO_NET_FF_SELECTOR_CAP lists the supported selectors and the
supported packet header fields for each selector.
\field{cap_specific_data} is in the format \field{struct virtio_net_ff_cap_mask_data}.

\begin{lstlisting}[label={lst:Device Types / Network Device / Device Operation / Flow filter / Device and driver capabilities / VIRTIO-NET-FF-SELECTOR-CAP / virtio-net-ff-selector}]
struct virtio_net_ff_selector {
        u8 type;
        u8 flags;
        u8 reserved[2];
        u8 length;
        u8 reserved1[3];
        u8 mask[];
};

struct virtio_net_ff_cap_mask_data {
        u8 count;
        u8 reserved[7];
        struct virtio_net_ff_selector selectors[];
};

#define VIRTIO_NET_FF_MASK_F_PARTIAL_MASK (1 << 0)
\end{lstlisting}

\field{count} indicates number of valid entries in the \field{selectors} array.
\field{selectors[]} is an array of supported selectors. Within each array entry:
\field{type} specifies the type of the packet header, as defined in table
\ref{table:Device Types / Network Device / Device Operation / Flow filter / Device and driver capabilities / VIRTIO-NET-FF-SELECTOR-CAP / flow filter selector types}. \field{mask} specifies which fields of the
packet header can be matched in a flow filter rule.

Each \field{type} is also listed in table
\ref{table:Device Types / Network Device / Device Operation / Flow filter / Device and driver capabilities / VIRTIO-NET-FF-SELECTOR-CAP / flow filter selector types}. \field{mask} is a byte array
in network byte order. For example, when \field{type} is VIRTIO_NET_FF_MASK_TYPE_IPV6,
the \field{mask} is in the format \hyperref[intro:IPv6-Header-Format]{IPv6 Header Format}.

If partial masking is not set, then all bits in each field have to be either all 0s
to ignore this field or all 1s to match on this field. If partial masking is set,
then any combination of bits can bit set to match on these bits.
For example, when a selector \field{type} is VIRTIO_NET_FF_MASK_TYPE_ETH, if
\field{mask[0-12]} are zero and \field{mask[13-14]} are 0xff (all 1s), it
indicates that matching is only supported for \field{EtherType} of
\field{Ethernet MAC frame}, matching is not supported for
\field{Destination Address} and \field{Source Address}.

The entries in the array \field{selectors} are ordered by
\field{type}, with each \field{type} value only appearing once.

\field{length} is the length of a dynamic array \field{mask} in bytes.
\field{reserved} and \field{reserved1} are reserved and set to zero.

\begin{table}[H]
\caption{Flow filter selector types}
\label{table:Device Types / Network Device / Device Operation / Flow filter / Device and driver capabilities / VIRTIO-NET-FF-SELECTOR-CAP / flow filter selector types}
\begin{tabularx}{\textwidth}{ |l|X|X| }
\hline
Type & Name & Description \\
\hline \hline
0x0 & - & Reserved \\
\hline
0x1 & VIRTIO_NET_FF_MASK_TYPE_ETH & 14 bytes of frame header starting from destination address described in \hyperref[intro:IEEE 802.3-2022]{IEEE 802.3-2022} \\
\hline
0x2 & VIRTIO_NET_FF_MASK_TYPE_IPV4 & 20 bytes of \hyperref[intro:Internet-Header-Format]{IPv4: Internet Header Format} \\
\hline
0x3 & VIRTIO_NET_FF_MASK_TYPE_IPV6 & 40 bytes of \hyperref[intro:IPv6-Header-Format]{IPv6 Header Format} \\
\hline
0x4 & VIRTIO_NET_FF_MASK_TYPE_TCP & 20 bytes of \hyperref[intro:TCP-Header-Format]{TCP Header Format} \\
\hline
0x5 & VIRTIO_NET_FF_MASK_TYPE_UDP & 8 bytes of UDP header described in \hyperref[intro:UDP]{UDP} \\
\hline
0x6 - 0xFF & & Reserved for future \\
\hline
\end{tabularx}
\end{table}

When VIRTIO_NET_FF_MASK_F_PARTIAL_MASK (bit 0) is set, it indicates that
partial masking is supported for all the fields of the selector identified by \field{type}.

For the selector \field{type} VIRTIO_NET_FF_MASK_TYPE_IPV4, if a partial mask is unsupported,
then matching on an individual bit of \field{Flags} in the
\field{IPv4: Internet Header Format} is unsupported. \field{Flags} has to match as a whole
if it is supported.

For the selector \field{type} VIRTIO_NET_FF_MASK_TYPE_IPV4, \field{mask} includes fields
up to the \field{Destination Address}; that is, \field{Options} and
\field{Padding} are excluded.

For the selector \field{type} VIRTIO_NET_FF_MASK_TYPE_IPV6, the \field{Next Header} field
of the \field{mask} corresponds to the \field{Next Header} in the packet
when \field{IPv6 Extension Headers} are not present. When the packet includes
one or more \field{IPv6 Extension Headers}, the \field{Next Header} field of
the \field{mask} corresponds to the \field{Next Header} of the last
\field{IPv6 Extension Header} in the packet.

For the selector \field{type} VIRTIO_NET_FF_MASK_TYPE_TCP, \field{Control bits}
are treated as individual fields for matching; that is, matching individual
\field{Control bits} does not depend on the partial mask support.

\subparagraph{VIRTIO_NET_FF_ACTION_CAP}
\label{par:Device Types / Network Device / Device Operation / Flow filter / Device and driver capabilities / VIRTIO-NET-FF-ACTION-CAP}

The capability VIRTIO_NET_FF_ACTION_CAP lists the supported actions in a rule.
\field{cap_specific_data} is in the format \field{struct virtio_net_ff_cap_actions}.

\begin{lstlisting}
struct virtio_net_ff_actions {
        u8 count;
        u8 reserved[7];
        u8 actions[];
};
\end{lstlisting}

\field{actions} is an array listing all possible actions.
The entries in the array are ordered from the smallest to the largest,
with each supported value appearing exactly once. Each entry can have the
following values:

\begin{table}[H]
\caption{Flow filter rule actions}
\label{table:Device Types / Network Device / Device Operation / Flow filter / Device and driver capabilities / VIRTIO-NET-FF-ACTION-CAP / flow filter rule actions}
\begin{tabularx}{\textwidth}{ |l|X|X| }
\hline
Action & Name & Description \\
\hline \hline
0x0 & - & reserved \\
\hline
0x1 & VIRTIO_NET_FF_ACTION_DROP & Matching packet will be dropped by the device \\
\hline
0x2 & VIRTIO_NET_FF_ACTION_DIRECT_RX_VQ & Matching packet will be directed to a receive queue \\
\hline
0x3 - 0xFF & & Reserved for future \\
\hline
\end{tabularx}
\end{table}

\paragraph{Resource objects}
\label{par:Device Types / Network Device / Device Operation / Flow filter / Resource objects}

\subparagraph{VIRTIO_NET_RESOURCE_OBJ_FF_GROUP}\label{par:Device Types / Network Device / Device Operation / Flow filter / Resource objects / VIRTIO-NET-RESOURCE-OBJ-FF-GROUP}

A flow filter group contains between 0 and \field{rules_limit} rules, as specified by the
capability VIRTIO_NET_FF_RESOURCE_CAP. For the flow filter group object both
\field{resource_obj_specific_data} and
\field{resource_obj_specific_result} are in the format
\field{struct virtio_net_resource_obj_ff_group}.

\begin{lstlisting}
struct virtio_net_resource_obj_ff_group {
        le16 group_priority;
};
\end{lstlisting}

\field{group_priority} specifies the priority for the group. Each group has a
distinct priority. For each incoming packet, the device tries to apply rules
from groups from higher \field{group_priority} value to lower, until either a
rule matches the packet or all groups have been tried.

\subparagraph{VIRTIO_NET_RESOURCE_OBJ_FF_CLASSIFIER}\label{par:Device Types / Network Device / Device Operation / Flow filter / Resource objects / VIRTIO-NET-RESOURCE-OBJ-FF-CLASSIFIER}

A classifier is used to match a flow filter key against a packet. The
classifier defines the desired packet fields to match, and is represented by
the VIRTIO_NET_RESOURCE_OBJ_FF_CLASSIFIER device resource object.

For the flow filter classifier object both \field{resource_obj_specific_data} and
\field{resource_obj_specific_result} are in the format
\field{struct virtio_net_resource_obj_ff_classifier}.

\begin{lstlisting}
struct virtio_net_resource_obj_ff_classifier {
        u8 count;
        u8 reserved[7];
        struct virtio_net_ff_selector selectors[];
};
\end{lstlisting}

A classifier is an array of \field{selectors}. The number of selectors in the
array is indicated by \field{count}. The selector has a type that specifies
the header fields to be matched against, and a mask.
See \ref{lst:Device Types / Network Device / Device Operation / Flow filter / Device and driver capabilities / VIRTIO-NET-FF-SELECTOR-CAP / virtio-net-ff-selector}
for details about selectors.

The first selector is always VIRTIO_NET_FF_MASK_TYPE_ETH. When there are multiple
selectors, a second selector can be either VIRTIO_NET_FF_MASK_TYPE_IPV4
or VIRTIO_NET_FF_MASK_TYPE_IPV6. If the third selector exists, the third
selector can be either VIRTIO_NET_FF_MASK_TYPE_UDP or VIRTIO_NET_FF_MASK_TYPE_TCP.
For example, to match a Ethernet IPv6 UDP packet,
\field{selectors[0].type} is set to VIRTIO_NET_FF_MASK_TYPE_ETH, \field{selectors[1].type}
is set to VIRTIO_NET_FF_MASK_TYPE_IPV6 and \field{selectors[2].type} is
set to VIRTIO_NET_FF_MASK_TYPE_UDP; accordingly, \field{selectors[0].mask[0-13]} is
for Ethernet header fields, \field{selectors[1].mask[0-39]} is set for IPV6 header
and \field{selectors[2].mask[0-7]} is set for UDP header.

When there are multiple selectors, the type of the (N+1)\textsuperscript{th} selector
affects the mask of the (N)\textsuperscript{th} selector. If
\field{count} is 2 or more, all the mask bits within \field{selectors[0]}
corresponding to \field{EtherType} of an Ethernet header are set.

If \field{count} is more than 2:
\begin{itemize}
\item if \field{selector[1].type} is, VIRTIO_NET_FF_MASK_TYPE_IPV4, then, all the mask bits within
\field{selector[1]} for \field{Protocol} is set.
\item if \field{selector[1].type} is, VIRTIO_NET_FF_MASK_TYPE_IPV6, then, all the mask bits within
\field{selector[1]} for \field{Next Header} is set.
\end{itemize}

If for a given packet header field, a subset of bits of a field is to be matched,
and if the partial mask is supported, the flow filter
mask object can specify a mask which has fewer bits set than the packet header
field size. For example, a partial mask for the Ethernet header source mac
address can be of 1-bit for multicast detection instead of 48-bits.

\subparagraph{VIRTIO_NET_RESOURCE_OBJ_FF_RULE}\label{par:Device Types / Network Device / Device Operation / Flow filter / Resource objects / VIRTIO-NET-RESOURCE-OBJ-FF-RULE}

Each flow filter rule resource object comprises a key, a priority, and an action.
For the flow filter rule object,
\field{resource_obj_specific_data} and
\field{resource_obj_specific_result} are in the format
\field{struct virtio_net_resource_obj_ff_rule}.

\begin{lstlisting}
struct virtio_net_resource_obj_ff_rule {
        le32 group_id;
        le32 classifier_id;
        u8 rule_priority;
        u8 key_length; /* length of key in bytes */
        u8 action;
        u8 reserved;
        le16 vq_index;
        u8 reserved1[2];
        u8 keys[][];
};
\end{lstlisting}

\field{group_id} is the resource object ID of the flow filter group to which
this rule belongs. \field{classifier_id} is the resource object ID of the
classifier used to match a packet against the \field{key}.

\field{rule_priority} denotes the priority of the rule within the group
specified by the \field{group_id}.
Rules within the group are applied from the highest to the lowest priority
until a rule matches the packet and an
action is taken. Rules with the same priority can be applied in any order.

\field{reserved} and \field{reserved1} are reserved and set to 0.

\field{keys[][]} is an array of keys to match against packets, using
the classifier specified by \field{classifier_id}. Each entry (key) comprises
a byte array, and they are located one immediately after another.
The size (number of entries) of the array is exactly the same as that of
\field{selectors} in the classifier, or in other words, \field{count}
in the classifier.

\field{key_length} specifies the total length of \field{keys} in bytes.
In other words, it equals the sum total of \field{length} of all
selectors in \field{selectors} in the classifier specified by
\field{classifier_id}.

For example, if a classifier object's \field{selectors[0].type} is
VIRTIO_NET_FF_MASK_TYPE_ETH and \field{selectors[1].type} is
VIRTIO_NET_FF_MASK_TYPE_IPV6,
then selectors[0].length is 14 and selectors[1].length is 40.
Accordingly, the \field{key_length} is set to 54.
This setting indicates that the \field{key} array's length is 54 bytes
comprising a first byte array of 14 bytes for the
Ethernet MAC header in bytes 0-13, immediately followed by 40 bytes for the
IPv6 header in bytes 14-53.

When there are multiple selectors in the classifier object, the key bytes
for (N)\textsuperscript{th} selector are set so that
(N+1)\textsuperscript{th} selector can be matched.

If \field{count} is 2 or more, key bytes of \field{EtherType}
are set according to \hyperref[intro:IEEE 802 Ethertypes]{IEEE 802 Ethertypes}
for VIRTIO_NET_FF_MASK_TYPE_IPV4 or VIRTIO_NET_FF_MASK_TYPE_IPV6 respectively.

If \field{count} is more than 2, when \field{selector[1].type} is
VIRTIO_NET_FF_MASK_TYPE_IPV4 or VIRTIO_NET_FF_MASK_TYPE_IPV6, key
bytes of \field{Protocol} or \field{Next Header} is set as per
\field{Protocol Numbers} defined \hyperref[intro:IANA Protocol Numbers]{IANA Protocol Numbers}
respectively.

\field{action} is the action to take when a packet matches the
\field{key} using the \field{classifier_id}. Supported actions are described in
\ref{table:Device Types / Network Device / Device Operation / Flow filter / Device and driver capabilities / VIRTIO-NET-FF-ACTION-CAP / flow filter rule actions}.

\field{vq_index} specifies a receive virtqueue. When the \field{action} is set
to VIRTIO_NET_FF_ACTION_DIRECT_RX_VQ, and the packet matches the \field{key},
the matching packet is directed to this virtqueue.

Note that at most one action is ever taken for a given packet. If a rule is
applied and an action is taken, the action of other rules is not taken.

\devicenormative{\paragraph}{Flow filter}{Device Types / Network Device / Device Operation / Flow filter}

When the device supports flow filter operations,
\begin{itemize}
\item the device MUST set VIRTIO_NET_FF_RESOURCE_CAP, VIRTIO_NET_FF_SELECTOR_CAP
and VIRTIO_NET_FF_ACTION_CAP capability in the \field{supported_caps} in the
command VIRTIO_ADMIN_CMD_CAP_SUPPORT_QUERY.
\item the device MUST support the administration commands
VIRTIO_ADMIN_CMD_RESOURCE_OBJ_CREATE,
VIRTIO_ADMIN_CMD_RESOURCE_OBJ_MODIFY, VIRTIO_ADMIN_CMD_RESOURCE_OBJ_QUERY,
VIRTIO_ADMIN_CMD_RESOURCE_OBJ_DESTROY for the resource types
VIRTIO_NET_RESOURCE_OBJ_FF_GROUP, VIRTIO_NET_RESOURCE_OBJ_FF_CLASSIFIER and
VIRTIO_NET_RESOURCE_OBJ_FF_RULE.
\end{itemize}

When any of the VIRTIO_NET_FF_RESOURCE_CAP, VIRTIO_NET_FF_SELECTOR_CAP, or
VIRTIO_NET_FF_ACTION_CAP capability is disabled, the device SHOULD set
\field{status} to VIRTIO_ADMIN_STATUS_Q_INVALID_OPCODE for the commands
VIRTIO_ADMIN_CMD_RESOURCE_OBJ_CREATE,
VIRTIO_ADMIN_CMD_RESOURCE_OBJ_MODIFY, VIRTIO_ADMIN_CMD_RESOURCE_OBJ_QUERY,
and VIRTIO_ADMIN_CMD_RESOURCE_OBJ_DESTROY. These commands apply to the resource
\field{type} of VIRTIO_NET_RESOURCE_OBJ_FF_GROUP, VIRTIO_NET_RESOURCE_OBJ_FF_CLASSIFIER, and
VIRTIO_NET_RESOURCE_OBJ_FF_RULE.

The device SHOULD set \field{status} to VIRTIO_ADMIN_STATUS_EINVAL for the
command VIRTIO_ADMIN_CMD_RESOURCE_OBJ_CREATE when the resource \field{type}
is VIRTIO_NET_RESOURCE_OBJ_FF_GROUP, if a flow filter group already exists
with the supplied \field{group_priority}.

The device SHOULD set \field{status} to VIRTIO_ADMIN_STATUS_ENOSPC for the
command VIRTIO_ADMIN_CMD_RESOURCE_OBJ_CREATE when the resource \field{type}
is VIRTIO_NET_RESOURCE_OBJ_FF_GROUP, if the number of flow filter group
objects in the device exceeds the lower of the configured driver
capabilities \field{groups_limit} and \field{rules_per_group_limit}.

The device SHOULD set \field{status} to VIRTIO_ADMIN_STATUS_ENOSPC for the
command VIRTIO_ADMIN_CMD_RESOURCE_OBJ_CREATE when the resource \field{type} is
VIRTIO_NET_RESOURCE_OBJ_FF_CLASSIFIER, if the number of flow filter selector
objects in the device exceeds the configured driver capability
\field{selectors_limit}.

The device SHOULD set \field{status} to VIRTIO_ADMIN_STATUS_EBUSY for the
command VIRTIO_ADMIN_CMD_RESOURCE_OBJ_DESTROY for a flow filter group when
the flow filter group has one or more flow filter rules depending on it.

The device SHOULD set \field{status} to VIRTIO_ADMIN_STATUS_EBUSY for the
command VIRTIO_ADMIN_CMD_RESOURCE_OBJ_DESTROY for a flow filter classifier when
the flow filter classifier has one or more flow filter rules depending on it.

The device SHOULD fail the command VIRTIO_ADMIN_CMD_RESOURCE_OBJ_CREATE for the
flow filter rule resource object if,
\begin{itemize}
\item \field{vq_index} is not a valid receive virtqueue index for
the VIRTIO_NET_FF_ACTION_DIRECT_RX_VQ action,
\item \field{priority} is greater than or equal to
      \field{last_rule_priority},
\item \field{id} is greater than or equal to \field{rules_limit} or
      greater than or equal to \field{rules_per_group_limit}, whichever is lower,
\item the length of \field{keys} and the length of all the mask bytes of
      \field{selectors[].mask} as referred by \field{classifier_id} differs,
\item the supplied \field{action} is not supported in the capability VIRTIO_NET_FF_ACTION_CAP.
\end{itemize}

When the flow filter directs a packet to the virtqueue identified by
\field{vq_index} and if the receive virtqueue is reset, the device
MUST drop such packets.

Upon applying a flow filter rule to a packet, the device MUST STOP any further
application of rules and cease applying any other steering configurations.

For multiple flow filter groups, the device MUST apply the rules from
the group with the highest priority. If any rule from this group is applied,
the device MUST ignore the remaining groups. If none of the rules from the
highest priority group match, the device MUST apply the rules from
the group with the next highest priority, until either a rule matches or
all groups have been attempted.

The device MUST apply the rules within the group from the highest to the
lowest priority until a rule matches the packet, and the device MUST take
the action. If an action is taken, the device MUST not take any other
action for this packet.

The device MAY apply the rules with the same \field{rule_priority} in any
order within the group.

The device MUST process incoming packets in the following order:
\begin{itemize}
\item apply the steering configuration received using control virtqueue
      commands VIRTIO_NET_CTRL_RX, VIRTIO_NET_CTRL_MAC, and
      VIRTIO_NET_CTRL_VLAN.
\item apply flow filter rules if any.
\item if no filter rule is applied, apply the steering configuration
      received using the command VIRTIO_NET_CTRL_MQ_RSS_CONFIG
      or according to automatic receive steering.
\end{itemize}

When processing an incoming packet, if the packet is dropped at any stage, the device
MUST skip further processing.

When the device drops the packet due to the configuration done using the control
virtqueue commands VIRTIO_NET_CTRL_RX or VIRTIO_NET_CTRL_MAC or VIRTIO_NET_CTRL_VLAN,
the device MUST skip flow filter rules for this packet.

When the device performs flow filter match operations and if the operation
result did not have any match in all the groups, the receive packet processing
continues to next level, i.e. to apply configuration done using
VIRTIO_NET_CTRL_MQ_RSS_CONFIG command.

The device MUST support the creation of flow filter classifier objects
using the command VIRTIO_ADMIN_CMD_RESOURCE_OBJ_CREATE with \field{flags}
set to VIRTIO_NET_FF_MASK_F_PARTIAL_MASK;
this support is required even if all the bits of the masks are set for
a field in \field{selectors}, provided that partial masking is supported
for the selectors.

\drivernormative{\paragraph}{Flow filter}{Device Types / Network Device / Device Operation / Flow filter}

The driver MUST enable VIRTIO_NET_FF_RESOURCE_CAP, VIRTIO_NET_FF_SELECTOR_CAP,
and VIRTIO_NET_FF_ACTION_CAP capabilities to use flow filter.

The driver SHOULD NOT remove a flow filter group using the command
VIRTIO_ADMIN_CMD_RESOURCE_OBJ_DESTROY when one or more flow filter rules
depend on that group. The driver SHOULD only destroy the group after
all the associated rules have been destroyed.

The driver SHOULD NOT remove a flow filter classifier using the command
VIRTIO_ADMIN_CMD_RESOURCE_OBJ_DESTROY when one or more flow filter rules
depend on the classifier. The driver SHOULD only destroy the classifier
after all the associated rules have been destroyed.

The driver SHOULD NOT add multiple flow filter rules with the same
\field{rule_priority} within a flow filter group, as these rules MAY match
the same packet. The driver SHOULD assign different \field{rule_priority}
values to different flow filter rules if multiple rules may match a single
packet.

For the command VIRTIO_ADMIN_CMD_RESOURCE_OBJ_CREATE, when creating a resource
of \field{type} VIRTIO_NET_RESOURCE_OBJ_FF_CLASSIFIER, the driver MUST set:
\begin{itemize}
\item \field{selectors[0].type} to VIRTIO_NET_FF_MASK_TYPE_ETH.
\item \field{selectors[1].type} to VIRTIO_NET_FF_MASK_TYPE_IPV4 or
      VIRTIO_NET_FF_MASK_TYPE_IPV6 when \field{count} is more than 1,
\item \field{selectors[2].type} VIRTIO_NET_FF_MASK_TYPE_UDP or
      VIRTIO_NET_FF_MASK_TYPE_TCP when \field{count} is more than 2.
\end{itemize}

For the command VIRTIO_ADMIN_CMD_RESOURCE_OBJ_CREATE, when creating a resource
of \field{type} VIRTIO_NET_RESOURCE_OBJ_FF_CLASSIFIER, the driver MUST set:
\begin{itemize}
\item \field{selectors[0].mask} bytes to all 1s for the \field{EtherType}
       when \field{count} is 2 or more.
\item \field{selectors[1].mask} bytes to all 1s for \field{Protocol} or \field{Next Header}
       when \field{selector[1].type} is VIRTIO_NET_FF_MASK_TYPE_IPV4 or VIRTIO_NET_FF_MASK_TYPE_IPV6,
       and when \field{count} is more than 2.
\end{itemize}

For the command VIRTIO_ADMIN_CMD_RESOURCE_OBJ_CREATE, the resource \field{type}
VIRTIO_NET_RESOURCE_OBJ_FF_RULE, if the corresponding classifier object's
\field{count} is 2 or more, the driver MUST SET the \field{keys} bytes of
\field{EtherType} in accordance with
\hyperref[intro:IEEE 802 Ethertypes]{IEEE 802 Ethertypes}
for either VIRTIO_NET_FF_MASK_TYPE_IPV4 or VIRTIO_NET_FF_MASK_TYPE_IPV6.

For the command VIRTIO_ADMIN_CMD_RESOURCE_OBJ_CREATE, when creating a resource of
\field{type} VIRTIO_NET_RESOURCE_OBJ_FF_RULE, if the corresponding classifier
object's \field{count} is more than 2, and the \field{selector[1].type} is either
VIRTIO_NET_FF_MASK_TYPE_IPV4 or VIRTIO_NET_FF_MASK_TYPE_IPV6, the driver MUST
set the \field{keys} bytes for the \field{Protocol} or \field{Next Header}
according to \hyperref[intro:IANA Protocol Numbers]{IANA Protocol Numbers} respectively.

The driver SHOULD set all the bits for a field in the mask of a selector in both the
capability and the classifier object, unless the VIRTIO_NET_FF_MASK_F_PARTIAL_MASK
is enabled.

\subsubsection{Legacy Interface: Framing Requirements}\label{sec:Device
Types / Network Device / Legacy Interface: Framing Requirements}

When using legacy interfaces, transitional drivers which have not
negotiated VIRTIO_F_ANY_LAYOUT MUST use a single descriptor for the
\field{struct virtio_net_hdr} on both transmit and receive, with the
network data in the following descriptors.

Additionally, when using the control virtqueue (see \ref{sec:Device
Types / Network Device / Device Operation / Control Virtqueue})
, transitional drivers which have not
negotiated VIRTIO_F_ANY_LAYOUT MUST:
\begin{itemize}
\item for all commands, use a single 2-byte descriptor including the first two
fields: \field{class} and \field{command}
\item for all commands except VIRTIO_NET_CTRL_MAC_TABLE_SET
use a single descriptor including command-specific-data
with no padding.
\item for the VIRTIO_NET_CTRL_MAC_TABLE_SET command use exactly
two descriptors including command-specific-data with no padding:
the first of these descriptors MUST include the
virtio_net_ctrl_mac table structure for the unicast addresses with no padding,
the second of these descriptors MUST include the
virtio_net_ctrl_mac table structure for the multicast addresses
with no padding.
\item for all commands, use a single 1-byte descriptor for the
\field{ack} field
\end{itemize}

See \ref{sec:Basic
Facilities of a Virtio Device / Virtqueues / Message Framing}.

\section{Network Device}\label{sec:Device Types / Network Device}

The virtio network device is a virtual network interface controller.
It consists of a virtual Ethernet link which connects the device
to the Ethernet network. The device has transmit and receive
queues. The driver adds empty buffers to the receive virtqueue.
The device receives incoming packets from the link; the device
places these incoming packets in the receive virtqueue buffers.
The driver adds outgoing packets to the transmit virtqueue. The device
removes these packets from the transmit virtqueue and sends them to
the link. The device may have a control virtqueue. The driver
uses the control virtqueue to dynamically manipulate various
features of the initialized device.

\subsection{Device ID}\label{sec:Device Types / Network Device / Device ID}

 1

\subsection{Virtqueues}\label{sec:Device Types / Network Device / Virtqueues}

\begin{description}
\item[0] receiveq1
\item[1] transmitq1
\item[\ldots]
\item[2(N-1)] receiveqN
\item[2(N-1)+1] transmitqN
\item[2N] controlq
\end{description}

 N=1 if neither VIRTIO_NET_F_MQ nor VIRTIO_NET_F_RSS are negotiated, otherwise N is set by
 \field{max_virtqueue_pairs}.

controlq is optional; it only exists if VIRTIO_NET_F_CTRL_VQ is
negotiated.

\subsection{Feature bits}\label{sec:Device Types / Network Device / Feature bits}

\begin{description}
\item[VIRTIO_NET_F_CSUM (0)] Device handles packets with partial checksum offload.

\item[VIRTIO_NET_F_GUEST_CSUM (1)] Driver handles packets with partial checksum.

\item[VIRTIO_NET_F_CTRL_GUEST_OFFLOADS (2)] Control channel offloads
        reconfiguration support.

\item[VIRTIO_NET_F_MTU(3)] Device maximum MTU reporting is supported. If
    offered by the device, device advises driver about the value of
    its maximum MTU. If negotiated, the driver uses \field{mtu} as
    the maximum MTU value.

\item[VIRTIO_NET_F_MAC (5)] Device has given MAC address.

\item[VIRTIO_NET_F_GUEST_TSO4 (7)] Driver can receive TSOv4.

\item[VIRTIO_NET_F_GUEST_TSO6 (8)] Driver can receive TSOv6.

\item[VIRTIO_NET_F_GUEST_ECN (9)] Driver can receive TSO with ECN.

\item[VIRTIO_NET_F_GUEST_UFO (10)] Driver can receive UFO.

\item[VIRTIO_NET_F_HOST_TSO4 (11)] Device can receive TSOv4.

\item[VIRTIO_NET_F_HOST_TSO6 (12)] Device can receive TSOv6.

\item[VIRTIO_NET_F_HOST_ECN (13)] Device can receive TSO with ECN.

\item[VIRTIO_NET_F_HOST_UFO (14)] Device can receive UFO.

\item[VIRTIO_NET_F_MRG_RXBUF (15)] Driver can merge receive buffers.

\item[VIRTIO_NET_F_STATUS (16)] Configuration status field is
    available.

\item[VIRTIO_NET_F_CTRL_VQ (17)] Control channel is available.

\item[VIRTIO_NET_F_CTRL_RX (18)] Control channel RX mode support.

\item[VIRTIO_NET_F_CTRL_VLAN (19)] Control channel VLAN filtering.

\item[VIRTIO_NET_F_CTRL_RX_EXTRA (20)]	Control channel RX extra mode support.

\item[VIRTIO_NET_F_GUEST_ANNOUNCE(21)] Driver can send gratuitous
    packets.

\item[VIRTIO_NET_F_MQ(22)] Device supports multiqueue with automatic
    receive steering.

\item[VIRTIO_NET_F_CTRL_MAC_ADDR(23)] Set MAC address through control
    channel.

\item[VIRTIO_NET_F_DEVICE_STATS(50)] Device can provide device-level statistics
    to the driver through the control virtqueue.

\item[VIRTIO_NET_F_HASH_TUNNEL(51)] Device supports inner header hash for encapsulated packets.

\item[VIRTIO_NET_F_VQ_NOTF_COAL(52)] Device supports virtqueue notification coalescing.

\item[VIRTIO_NET_F_NOTF_COAL(53)] Device supports notifications coalescing.

\item[VIRTIO_NET_F_GUEST_USO4 (54)] Driver can receive USOv4 packets.

\item[VIRTIO_NET_F_GUEST_USO6 (55)] Driver can receive USOv6 packets.

\item[VIRTIO_NET_F_HOST_USO (56)] Device can receive USO packets. Unlike UFO
 (fragmenting the packet) the USO splits large UDP packet
 to several segments when each of these smaller packets has UDP header.

\item[VIRTIO_NET_F_HASH_REPORT(57)] Device can report per-packet hash
    value and a type of calculated hash.

\item[VIRTIO_NET_F_GUEST_HDRLEN(59)] Driver can provide the exact \field{hdr_len}
    value. Device benefits from knowing the exact header length.

\item[VIRTIO_NET_F_RSS(60)] Device supports RSS (receive-side scaling)
    with Toeplitz hash calculation and configurable hash
    parameters for receive steering.

\item[VIRTIO_NET_F_RSC_EXT(61)] Device can process duplicated ACKs
    and report number of coalesced segments and duplicated ACKs.

\item[VIRTIO_NET_F_STANDBY(62)] Device may act as a standby for a primary
    device with the same MAC address.

\item[VIRTIO_NET_F_SPEED_DUPLEX(63)] Device reports speed and duplex.

\item[VIRTIO_NET_F_RSS_CONTEXT(64)] Device supports multiple RSS contexts.

\item[VIRTIO_NET_F_GUEST_UDP_TUNNEL_GSO (65)] Driver can receive GSO packets
  carried by a UDP tunnel.

\item[VIRTIO_NET_F_GUEST_UDP_TUNNEL_GSO_CSUM (66)] Driver handles packets
  carried by a UDP tunnel with partial csum for the outer header.

\item[VIRTIO_NET_F_HOST_UDP_TUNNEL_GSO (67)] Device can receive GSO packets
  carried by a UDP tunnel.

\item[VIRTIO_NET_F_HOST_UDP_TUNNEL_GSO_CSUM (68)] Device handles packets
  carried by a UDP tunnel with partial csum for the outer header.
\end{description}

\subsubsection{Feature bit requirements}\label{sec:Device Types / Network Device / Feature bits / Feature bit requirements}

Some networking feature bits require other networking feature bits
(see \ref{drivernormative:Basic Facilities of a Virtio Device / Feature Bits}):

\begin{description}
\item[VIRTIO_NET_F_GUEST_TSO4] Requires VIRTIO_NET_F_GUEST_CSUM.
\item[VIRTIO_NET_F_GUEST_TSO6] Requires VIRTIO_NET_F_GUEST_CSUM.
\item[VIRTIO_NET_F_GUEST_ECN] Requires VIRTIO_NET_F_GUEST_TSO4 or VIRTIO_NET_F_GUEST_TSO6.
\item[VIRTIO_NET_F_GUEST_UFO] Requires VIRTIO_NET_F_GUEST_CSUM.
\item[VIRTIO_NET_F_GUEST_USO4] Requires VIRTIO_NET_F_GUEST_CSUM.
\item[VIRTIO_NET_F_GUEST_USO6] Requires VIRTIO_NET_F_GUEST_CSUM.
\item[VIRTIO_NET_F_GUEST_UDP_TUNNEL_GSO] Requires VIRTIO_NET_F_GUEST_TSO4, VIRTIO_NET_F_GUEST_TSO6,
   VIRTIO_NET_F_GUEST_USO4 and VIRTIO_NET_F_GUEST_USO6.
\item[VIRTIO_NET_F_GUEST_UDP_TUNNEL_GSO_CSUM] Requires VIRTIO_NET_F_GUEST_UDP_TUNNEL_GSO

\item[VIRTIO_NET_F_HOST_TSO4] Requires VIRTIO_NET_F_CSUM.
\item[VIRTIO_NET_F_HOST_TSO6] Requires VIRTIO_NET_F_CSUM.
\item[VIRTIO_NET_F_HOST_ECN] Requires VIRTIO_NET_F_HOST_TSO4 or VIRTIO_NET_F_HOST_TSO6.
\item[VIRTIO_NET_F_HOST_UFO] Requires VIRTIO_NET_F_CSUM.
\item[VIRTIO_NET_F_HOST_USO] Requires VIRTIO_NET_F_CSUM.
\item[VIRTIO_NET_F_HOST_UDP_TUNNEL_GSO] Requires VIRTIO_NET_F_HOST_TSO4, VIRTIO_NET_F_HOST_TSO6
   and VIRTIO_NET_F_HOST_USO.
\item[VIRTIO_NET_F_HOST_UDP_TUNNEL_GSO_CSUM] Requires VIRTIO_NET_F_HOST_UDP_TUNNEL_GSO

\item[VIRTIO_NET_F_CTRL_RX] Requires VIRTIO_NET_F_CTRL_VQ.
\item[VIRTIO_NET_F_CTRL_VLAN] Requires VIRTIO_NET_F_CTRL_VQ.
\item[VIRTIO_NET_F_GUEST_ANNOUNCE] Requires VIRTIO_NET_F_CTRL_VQ.
\item[VIRTIO_NET_F_MQ] Requires VIRTIO_NET_F_CTRL_VQ.
\item[VIRTIO_NET_F_CTRL_MAC_ADDR] Requires VIRTIO_NET_F_CTRL_VQ.
\item[VIRTIO_NET_F_NOTF_COAL] Requires VIRTIO_NET_F_CTRL_VQ.
\item[VIRTIO_NET_F_RSC_EXT] Requires VIRTIO_NET_F_HOST_TSO4 or VIRTIO_NET_F_HOST_TSO6.
\item[VIRTIO_NET_F_RSS] Requires VIRTIO_NET_F_CTRL_VQ.
\item[VIRTIO_NET_F_VQ_NOTF_COAL] Requires VIRTIO_NET_F_CTRL_VQ.
\item[VIRTIO_NET_F_HASH_TUNNEL] Requires VIRTIO_NET_F_CTRL_VQ along with VIRTIO_NET_F_RSS or VIRTIO_NET_F_HASH_REPORT.
\item[VIRTIO_NET_F_RSS_CONTEXT] Requires VIRTIO_NET_F_CTRL_VQ and VIRTIO_NET_F_RSS.
\end{description}

\begin{note}
The dependency between UDP_TUNNEL_GSO_CSUM and UDP_TUNNEL_GSO is intentionally
in the opposite direction with respect to the plain GSO features and the plain
checksum offload because UDP tunnel checksum offload gives very little gain
for non GSO packets and is quite complex to implement in H/W.
\end{note}

\subsubsection{Legacy Interface: Feature bits}\label{sec:Device Types / Network Device / Feature bits / Legacy Interface: Feature bits}
\begin{description}
\item[VIRTIO_NET_F_GSO (6)] Device handles packets with any GSO type. This was supposed to indicate segmentation offload support, but
upon further investigation it became clear that multiple bits were needed.
\item[VIRTIO_NET_F_GUEST_RSC4 (41)] Device coalesces TCPIP v4 packets. This was implemented by hypervisor patch for certification
purposes and current Windows driver depends on it. It will not function if virtio-net device reports this feature.
\item[VIRTIO_NET_F_GUEST_RSC6 (42)] Device coalesces TCPIP v6 packets. Similar to VIRTIO_NET_F_GUEST_RSC4.
\end{description}

\subsection{Device configuration layout}\label{sec:Device Types / Network Device / Device configuration layout}
\label{sec:Device Types / Block Device / Feature bits / Device configuration layout}

The network device has the following device configuration layout.
All of the device configuration fields are read-only for the driver.

\begin{lstlisting}
struct virtio_net_config {
        u8 mac[6];
        le16 status;
        le16 max_virtqueue_pairs;
        le16 mtu;
        le32 speed;
        u8 duplex;
        u8 rss_max_key_size;
        le16 rss_max_indirection_table_length;
        le32 supported_hash_types;
        le32 supported_tunnel_types;
};
\end{lstlisting}

The \field{mac} address field always exists (although it is only
valid if VIRTIO_NET_F_MAC is set).

The \field{status} only exists if VIRTIO_NET_F_STATUS is set.
Two bits are currently defined for the status field: VIRTIO_NET_S_LINK_UP
and VIRTIO_NET_S_ANNOUNCE.

\begin{lstlisting}
#define VIRTIO_NET_S_LINK_UP     1
#define VIRTIO_NET_S_ANNOUNCE    2
\end{lstlisting}

The following field, \field{max_virtqueue_pairs} only exists if
VIRTIO_NET_F_MQ or VIRTIO_NET_F_RSS is set. This field specifies the maximum number
of each of transmit and receive virtqueues (receiveq1\ldots receiveqN
and transmitq1\ldots transmitqN respectively) that can be configured once at least one of these features
is negotiated.

The following field, \field{mtu} only exists if VIRTIO_NET_F_MTU
is set. This field specifies the maximum MTU for the driver to
use.

The following two fields, \field{speed} and \field{duplex}, only
exist if VIRTIO_NET_F_SPEED_DUPLEX is set.

\field{speed} contains the device speed, in units of 1 MBit per
second, 0 to 0x7fffffff, or 0xffffffff for unknown speed.

\field{duplex} has the values of 0x01 for full duplex, 0x00 for
half duplex and 0xff for unknown duplex state.

Both \field{speed} and \field{duplex} can change, thus the driver
is expected to re-read these values after receiving a
configuration change notification.

The following field, \field{rss_max_key_size} only exists if VIRTIO_NET_F_RSS or VIRTIO_NET_F_HASH_REPORT is set.
It specifies the maximum supported length of RSS key in bytes.

The following field, \field{rss_max_indirection_table_length} only exists if VIRTIO_NET_F_RSS is set.
It specifies the maximum number of 16-bit entries in RSS indirection table.

The next field, \field{supported_hash_types} only exists if the device supports hash calculation,
i.e. if VIRTIO_NET_F_RSS or VIRTIO_NET_F_HASH_REPORT is set.

Field \field{supported_hash_types} contains the bitmask of supported hash types.
See \ref{sec:Device Types / Network Device / Device Operation / Processing of Incoming Packets / Hash calculation for incoming packets / Supported/enabled hash types} for details of supported hash types.

Field \field{supported_tunnel_types} only exists if the device supports inner header hash, i.e. if VIRTIO_NET_F_HASH_TUNNEL is set.

Field \field{supported_tunnel_types} contains the bitmask of encapsulation types supported by the device for inner header hash.
Encapsulation types are defined in \ref{sec:Device Types / Network Device / Device Operation / Processing of Incoming Packets /
Hash calculation for incoming packets / Encapsulation types supported/enabled for inner header hash}.

\devicenormative{\subsubsection}{Device configuration layout}{Device Types / Network Device / Device configuration layout}

The device MUST set \field{max_virtqueue_pairs} to between 1 and 0x8000 inclusive,
if it offers VIRTIO_NET_F_MQ.

The device MUST set \field{mtu} to between 68 and 65535 inclusive,
if it offers VIRTIO_NET_F_MTU.

The device SHOULD set \field{mtu} to at least 1280, if it offers
VIRTIO_NET_F_MTU.

The device MUST NOT modify \field{mtu} once it has been set.

The device MUST NOT pass received packets that exceed \field{mtu} (plus low
level ethernet header length) size with \field{gso_type} NONE or ECN
after VIRTIO_NET_F_MTU has been successfully negotiated.

The device MUST forward transmitted packets of up to \field{mtu} (plus low
level ethernet header length) size with \field{gso_type} NONE or ECN, and do
so without fragmentation, after VIRTIO_NET_F_MTU has been successfully
negotiated.

The device MUST set \field{rss_max_key_size} to at least 40, if it offers
VIRTIO_NET_F_RSS or VIRTIO_NET_F_HASH_REPORT.

The device MUST set \field{rss_max_indirection_table_length} to at least 128, if it offers
VIRTIO_NET_F_RSS.

If the driver negotiates the VIRTIO_NET_F_STANDBY feature, the device MAY act
as a standby device for a primary device with the same MAC address.

If VIRTIO_NET_F_SPEED_DUPLEX has been negotiated, \field{speed}
MUST contain the device speed, in units of 1 MBit per second, 0 to
0x7ffffffff, or 0xfffffffff for unknown.

If VIRTIO_NET_F_SPEED_DUPLEX has been negotiated, \field{duplex}
MUST have the values of 0x00 for full duplex, 0x01 for half
duplex, or 0xff for unknown.

If VIRTIO_NET_F_SPEED_DUPLEX and VIRTIO_NET_F_STATUS have both
been negotiated, the device SHOULD NOT change the \field{speed} and
\field{duplex} fields as long as VIRTIO_NET_S_LINK_UP is set in
the \field{status}.

The device SHOULD NOT offer VIRTIO_NET_F_HASH_REPORT if it
does not offer VIRTIO_NET_F_CTRL_VQ.

The device SHOULD NOT offer VIRTIO_NET_F_CTRL_RX_EXTRA if it
does not offer VIRTIO_NET_F_CTRL_VQ.

\drivernormative{\subsubsection}{Device configuration layout}{Device Types / Network Device / Device configuration layout}

The driver MUST NOT write to any of the device configuration fields.

A driver SHOULD negotiate VIRTIO_NET_F_MAC if the device offers it.
If the driver negotiates the VIRTIO_NET_F_MAC feature, the driver MUST set
the physical address of the NIC to \field{mac}.  Otherwise, it SHOULD
use a locally-administered MAC address (see \hyperref[intro:IEEE 802]{IEEE 802},
``9.2 48-bit universal LAN MAC addresses'').

If the driver does not negotiate the VIRTIO_NET_F_STATUS feature, it SHOULD
assume the link is active, otherwise it SHOULD read the link status from
the bottom bit of \field{status}.

A driver SHOULD negotiate VIRTIO_NET_F_MTU if the device offers it.

If the driver negotiates VIRTIO_NET_F_MTU, it MUST supply enough receive
buffers to receive at least one receive packet of size \field{mtu} (plus low
level ethernet header length) with \field{gso_type} NONE or ECN.

If the driver negotiates VIRTIO_NET_F_MTU, it MUST NOT transmit packets of
size exceeding the value of \field{mtu} (plus low level ethernet header length)
with \field{gso_type} NONE or ECN.

A driver SHOULD negotiate the VIRTIO_NET_F_STANDBY feature if the device offers it.

If VIRTIO_NET_F_SPEED_DUPLEX has been negotiated,
the driver MUST treat any value of \field{speed} above
0x7fffffff as well as any value of \field{duplex} not
matching 0x00 or 0x01 as an unknown value.

If VIRTIO_NET_F_SPEED_DUPLEX has been negotiated, the driver
SHOULD re-read \field{speed} and \field{duplex} after a
configuration change notification.

A driver SHOULD NOT negotiate VIRTIO_NET_F_HASH_REPORT if it
does not negotiate VIRTIO_NET_F_CTRL_VQ.

A driver SHOULD NOT negotiate VIRTIO_NET_F_CTRL_RX_EXTRA if it
does not negotiate VIRTIO_NET_F_CTRL_VQ.

\subsubsection{Legacy Interface: Device configuration layout}\label{sec:Device Types / Network Device / Device configuration layout / Legacy Interface: Device configuration layout}
\label{sec:Device Types / Block Device / Feature bits / Device configuration layout / Legacy Interface: Device configuration layout}
When using the legacy interface, transitional devices and drivers
MUST format \field{status} and
\field{max_virtqueue_pairs} in struct virtio_net_config
according to the native endian of the guest rather than
(necessarily when not using the legacy interface) little-endian.

When using the legacy interface, \field{mac} is driver-writable
which provided a way for drivers to update the MAC without
negotiating VIRTIO_NET_F_CTRL_MAC_ADDR.

\subsection{Device Initialization}\label{sec:Device Types / Network Device / Device Initialization}

A driver would perform a typical initialization routine like so:

\begin{enumerate}
\item Identify and initialize the receive and
  transmission virtqueues, up to N of each kind. If
  VIRTIO_NET_F_MQ feature bit is negotiated,
  N=\field{max_virtqueue_pairs}, otherwise identify N=1.

\item If the VIRTIO_NET_F_CTRL_VQ feature bit is negotiated,
  identify the control virtqueue.

\item Fill the receive queues with buffers: see \ref{sec:Device Types / Network Device / Device Operation / Setting Up Receive Buffers}.

\item Even with VIRTIO_NET_F_MQ, only receiveq1, transmitq1 and
  controlq are used by default.  The driver would send the
  VIRTIO_NET_CTRL_MQ_VQ_PAIRS_SET command specifying the
  number of the transmit and receive queues to use.

\item If the VIRTIO_NET_F_MAC feature bit is set, the configuration
  space \field{mac} entry indicates the ``physical'' address of the
  device, otherwise the driver would typically generate a random
  local MAC address.

\item If the VIRTIO_NET_F_STATUS feature bit is negotiated, the link
  status comes from the bottom bit of \field{status}.
  Otherwise, the driver assumes it's active.

\item A performant driver would indicate that it will generate checksumless
  packets by negotiating the VIRTIO_NET_F_CSUM feature.

\item If that feature is negotiated, a driver can use TCP segmentation or UDP
  segmentation/fragmentation offload by negotiating the VIRTIO_NET_F_HOST_TSO4 (IPv4
  TCP), VIRTIO_NET_F_HOST_TSO6 (IPv6 TCP), VIRTIO_NET_F_HOST_UFO
  (UDP fragmentation) and VIRTIO_NET_F_HOST_USO (UDP segmentation) features.

\item If the VIRTIO_NET_F_HOST_TSO6, VIRTIO_NET_F_HOST_TSO4 and VIRTIO_NET_F_HOST_USO
  segmentation features are negotiated, a driver can
  use TCP segmentation or UDP segmentation on top of UDP encapsulation
  offload, when the outer header does not require checksumming - e.g.
  the outer UDP checksum is zero - by negotiating the
  VIRTIO_NET_F_HOST_UDP_TUNNEL_GSO feature.
  GSO over UDP tunnels packets carry two sets of headers: the outer ones
  and the inner ones. The outer transport protocol is UDP, the inner
  could be either TCP or UDP. Only a single level of encapsulation
  offload is supported.

\item If VIRTIO_NET_F_HOST_UDP_TUNNEL_GSO is negotiated, a driver can
  additionally use TCP segmentation or UDP segmentation on top of UDP
  encapsulation with the outer header requiring checksum offload,
  negotiating the VIRTIO_NET_F_HOST_UDP_TUNNEL_GSO_CSUM feature.

\item The converse features are also available: a driver can save
  the virtual device some work by negotiating these features.\note{For example, a network packet transported between two guests on
the same system might not need checksumming at all, nor segmentation,
if both guests are amenable.}
   The VIRTIO_NET_F_GUEST_CSUM feature indicates that partially
  checksummed packets can be received, and if it can do that then
  the VIRTIO_NET_F_GUEST_TSO4, VIRTIO_NET_F_GUEST_TSO6,
  VIRTIO_NET_F_GUEST_UFO, VIRTIO_NET_F_GUEST_ECN, VIRTIO_NET_F_GUEST_USO4,
  VIRTIO_NET_F_GUEST_USO6 VIRTIO_NET_F_GUEST_UDP_TUNNEL_GSO and
  VIRTIO_NET_F_GUEST_UDP_TUNNEL_GSO_CSUM are the input equivalents of
  the features described above.
  See \ref{sec:Device Types / Network Device / Device Operation /
Setting Up Receive Buffers}~\nameref{sec:Device Types / Network
Device / Device Operation / Setting Up Receive Buffers} and
\ref{sec:Device Types / Network Device / Device Operation /
Processing of Incoming Packets}~\nameref{sec:Device Types /
Network Device / Device Operation / Processing of Incoming Packets} below.
\end{enumerate}

A truly minimal driver would only accept VIRTIO_NET_F_MAC and ignore
everything else.

\subsection{Device and driver capabilities}\label{sec:Device Types / Network Device / Device and driver capabilities}

The network device has the following capabilities.

\begin{tabularx}{\textwidth}{ |l||l|X| }
\hline
Identifier & Name & Description \\
\hline \hline
0x0800 & \hyperref[par:Device Types / Network Device / Device Operation / Flow filter / Device and driver capabilities / VIRTIO-NET-FF-RESOURCE-CAP]{VIRTIO_NET_FF_RESOURCE_CAP} & Flow filter resource capability \\
\hline
0x0801 & \hyperref[par:Device Types / Network Device / Device Operation / Flow filter / Device and driver capabilities / VIRTIO-NET-FF-SELECTOR-CAP]{VIRTIO_NET_FF_SELECTOR_CAP} & Flow filter classifier capability \\
\hline
0x0802 & \hyperref[par:Device Types / Network Device / Device Operation / Flow filter / Device and driver capabilities / VIRTIO-NET-FF-ACTION-CAP]{VIRTIO_NET_FF_ACTION_CAP} & Flow filter action capability \\
\hline
\end{tabularx}

\subsection{Device resource objects}\label{sec:Device Types / Network Device / Device resource objects}

The network device has the following resource objects.

\begin{tabularx}{\textwidth}{ |l||l|X| }
\hline
type & Name & Description \\
\hline \hline
0x0200 & \hyperref[par:Device Types / Network Device / Device Operation / Flow filter / Resource objects / VIRTIO-NET-RESOURCE-OBJ-FF-GROUP]{VIRTIO_NET_RESOURCE_OBJ_FF_GROUP} & Flow filter group resource object \\
\hline
0x0201 & \hyperref[par:Device Types / Network Device / Device Operation / Flow filter / Resource objects / VIRTIO-NET-RESOURCE-OBJ-FF-CLASSIFIER]{VIRTIO_NET_RESOURCE_OBJ_FF_CLASSIFIER} & Flow filter mask object \\
\hline
0x0202 & \hyperref[par:Device Types / Network Device / Device Operation / Flow filter / Resource objects / VIRTIO-NET-RESOURCE-OBJ-FF-RULE]{VIRTIO_NET_RESOURCE_OBJ_FF_RULE} & Flow filter rule object \\
\hline
\end{tabularx}

\subsection{Device parts}\label{sec:Device Types / Network Device / Device parts}

Network device parts represent the configuration done by the driver using control
virtqueue commands. Network device part is in the format of
\field{struct virtio_dev_part}.

\begin{tabularx}{\textwidth}{ |l||l|X| }
\hline
Type & Name & Description \\
\hline \hline
0x200 & VIRTIO_NET_DEV_PART_CVQ_CFG_PART & Represents device configuration done through a control virtqueue command, see \ref{sec:Device Types / Network Device / Device parts / VIRTIO-NET-DEV-PART-CVQ-CFG-PART} \\
\hline
0x201 - 0x5FF & - & reserved for future \\
\hline
\hline
\end{tabularx}

\subsubsection{VIRTIO_NET_DEV_PART_CVQ_CFG_PART}\label{sec:Device Types / Network Device / Device parts / VIRTIO-NET-DEV-PART-CVQ-CFG-PART}

For VIRTIO_NET_DEV_PART_CVQ_CFG_PART, \field{part_type} is set to 0x200. The
VIRTIO_NET_DEV_PART_CVQ_CFG_PART part indicates configuration performed by the
driver using a control virtqueue command.

\begin{lstlisting}
struct virtio_net_dev_part_cvq_selector {
        u8 class;
        u8 command;
        u8 reserved[6];
};
\end{lstlisting}

There is one device part of type VIRTIO_NET_DEV_PART_CVQ_CFG_PART for each
individual configuration. Each part is identified by a unique selector value.
The selector, \field{device_type_raw}, is in the format
\field{struct virtio_net_dev_part_cvq_selector}.

The selector consists of two fields: \field{class} and \field{command}. These
fields correspond to the \field{class} and \field{command} defined in
\field{struct virtio_net_ctrl}, as described in the relevant sections of
\ref{sec:Device Types / Network Device / Device Operation / Control Virtqueue}.

The value corresponding to each part’s selector follows the same format as the
respective \field{command-specific-data} described in the relevant sections of
\ref{sec:Device Types / Network Device / Device Operation / Control Virtqueue}.

For example, when the \field{class} is VIRTIO_NET_CTRL_MAC, the \field{command}
can be either VIRTIO_NET_CTRL_MAC_TABLE_SET or VIRTIO_NET_CTRL_MAC_ADDR_SET;
when \field{command} is set to VIRTIO_NET_CTRL_MAC_TABLE_SET, \field{value}
is in the format of \field{struct virtio_net_ctrl_mac}.

Supported selectors are listed in the table:

\begin{tabularx}{\textwidth}{ |l|X| }
\hline
Class selector & Command selector \\
\hline \hline
VIRTIO_NET_CTRL_RX & VIRTIO_NET_CTRL_RX_PROMISC \\
\hline
VIRTIO_NET_CTRL_RX & VIRTIO_NET_CTRL_RX_ALLMULTI \\
\hline
VIRTIO_NET_CTRL_RX & VIRTIO_NET_CTRL_RX_ALLUNI \\
\hline
VIRTIO_NET_CTRL_RX & VIRTIO_NET_CTRL_RX_NOMULTI \\
\hline
VIRTIO_NET_CTRL_RX & VIRTIO_NET_CTRL_RX_NOUNI \\
\hline
VIRTIO_NET_CTRL_RX & VIRTIO_NET_CTRL_RX_NOBCAST \\
\hline
VIRTIO_NET_CTRL_MAC & VIRTIO_NET_CTRL_MAC_TABLE_SET \\
\hline
VIRTIO_NET_CTRL_MAC & VIRTIO_NET_CTRL_MAC_ADDR_SET \\
\hline
VIRTIO_NET_CTRL_VLAN & VIRTIO_NET_CTRL_VLAN_ADD \\
\hline
VIRTIO_NET_CTRL_ANNOUNCE & VIRTIO_NET_CTRL_ANNOUNCE_ACK \\
\hline
VIRTIO_NET_CTRL_MQ & VIRTIO_NET_CTRL_MQ_VQ_PAIRS_SET \\
\hline
VIRTIO_NET_CTRL_MQ & VIRTIO_NET_CTRL_MQ_RSS_CONFIG \\
\hline
VIRTIO_NET_CTRL_MQ & VIRTIO_NET_CTRL_MQ_HASH_CONFIG \\
\hline
\hline
\end{tabularx}

For command selector VIRTIO_NET_CTRL_VLAN_ADD, device part consists of a whole
VLAN table.

\field{reserved} is reserved and set to zero.

\subsection{Device Operation}\label{sec:Device Types / Network Device / Device Operation}

Packets are transmitted by placing them in the
transmitq1\ldots transmitqN, and buffers for incoming packets are
placed in the receiveq1\ldots receiveqN. In each case, the packet
itself is preceded by a header:

\begin{lstlisting}
struct virtio_net_hdr {
#define VIRTIO_NET_HDR_F_NEEDS_CSUM    1
#define VIRTIO_NET_HDR_F_DATA_VALID    2
#define VIRTIO_NET_HDR_F_RSC_INFO      4
#define VIRTIO_NET_HDR_F_UDP_TUNNEL_CSUM 8
        u8 flags;
#define VIRTIO_NET_HDR_GSO_NONE        0
#define VIRTIO_NET_HDR_GSO_TCPV4       1
#define VIRTIO_NET_HDR_GSO_UDP         3
#define VIRTIO_NET_HDR_GSO_TCPV6       4
#define VIRTIO_NET_HDR_GSO_UDP_L4      5
#define VIRTIO_NET_HDR_GSO_UDP_TUNNEL_IPV4 0x20
#define VIRTIO_NET_HDR_GSO_UDP_TUNNEL_IPV6 0x40
#define VIRTIO_NET_HDR_GSO_ECN      0x80
        u8 gso_type;
        le16 hdr_len;
        le16 gso_size;
        le16 csum_start;
        le16 csum_offset;
        le16 num_buffers;
        le32 hash_value;        (Only if VIRTIO_NET_F_HASH_REPORT negotiated)
        le16 hash_report;       (Only if VIRTIO_NET_F_HASH_REPORT negotiated)
        le16 padding_reserved;  (Only if VIRTIO_NET_F_HASH_REPORT negotiated)
        le16 outer_th_offset    (Only if VIRTIO_NET_F_HOST_UDP_TUNNEL_GSO or VIRTIO_NET_F_GUEST_UDP_TUNNEL_GSO negotiated)
        le16 inner_nh_offset;   (Only if VIRTIO_NET_F_HOST_UDP_TUNNEL_GSO or VIRTIO_NET_F_GUEST_UDP_TUNNEL_GSO negotiated)
};
\end{lstlisting}

The controlq is used to control various device features described further in
section \ref{sec:Device Types / Network Device / Device Operation / Control Virtqueue}.

\subsubsection{Legacy Interface: Device Operation}\label{sec:Device Types / Network Device / Device Operation / Legacy Interface: Device Operation}
When using the legacy interface, transitional devices and drivers
MUST format the fields in \field{struct virtio_net_hdr}
according to the native endian of the guest rather than
(necessarily when not using the legacy interface) little-endian.

The legacy driver only presented \field{num_buffers} in the \field{struct virtio_net_hdr}
when VIRTIO_NET_F_MRG_RXBUF was negotiated; without that feature the
structure was 2 bytes shorter.

When using the legacy interface, the driver SHOULD ignore the
used length for the transmit queues
and the controlq queue.
\begin{note}
Historically, some devices put
the total descriptor length there, even though no data was
actually written.
\end{note}

\subsubsection{Packet Transmission}\label{sec:Device Types / Network Device / Device Operation / Packet Transmission}

Transmitting a single packet is simple, but varies depending on
the different features the driver negotiated.

\begin{enumerate}
\item The driver can send a completely checksummed packet.  In this case,
  \field{flags} will be zero, and \field{gso_type} will be VIRTIO_NET_HDR_GSO_NONE.

\item If the driver negotiated VIRTIO_NET_F_CSUM, it can skip
  checksumming the packet:
  \begin{itemize}
  \item \field{flags} has the VIRTIO_NET_HDR_F_NEEDS_CSUM set,

  \item \field{csum_start} is set to the offset within the packet to begin checksumming,
    and

  \item \field{csum_offset} indicates how many bytes after the csum_start the
    new (16 bit ones' complement) checksum is placed by the device.

  \item The TCP checksum field in the packet is set to the sum
    of the TCP pseudo header, so that replacing it by the ones'
    complement checksum of the TCP header and body will give the
    correct result.
  \end{itemize}

\begin{note}
For example, consider a partially checksummed TCP (IPv4) packet.
It will have a 14 byte ethernet header and 20 byte IP header
followed by the TCP header (with the TCP checksum field 16 bytes
into that header). \field{csum_start} will be 14+20 = 34 (the TCP
checksum includes the header), and \field{csum_offset} will be 16.
If the given packet has the VIRTIO_NET_HDR_GSO_UDP_TUNNEL_IPV4 bit or the
VIRTIO_NET_HDR_GSO_UDP_TUNNEL_IPV6 bit set,
the above checksum fields refer to the inner header checksum, see
the example below.
\end{note}

\item If the driver negotiated
  VIRTIO_NET_F_HOST_TSO4, TSO6, USO or UFO, and the packet requires
  TCP segmentation, UDP segmentation or fragmentation, then \field{gso_type}
  is set to VIRTIO_NET_HDR_GSO_TCPV4, TCPV6, UDP_L4 or UDP.
  (Otherwise, it is set to VIRTIO_NET_HDR_GSO_NONE). In this
  case, packets larger than 1514 bytes can be transmitted: the
  metadata indicates how to replicate the packet header to cut it
  into smaller packets. The other gso fields are set:

  \begin{itemize}
  \item If the VIRTIO_NET_F_GUEST_HDRLEN feature has been negotiated,
    \field{hdr_len} indicates the header length that needs to be replicated
    for each packet. It's the number of bytes from the beginning of the packet
    to the beginning of the transport payload.
    If the \field{gso_type} has the VIRTIO_NET_HDR_GSO_UDP_TUNNEL_IPV4 bit or
    VIRTIO_NET_HDR_GSO_UDP_TUNNEL_IPV6 bit set, \field{hdr_len} accounts for
    all the headers up to and including the inner transport.
    Otherwise, if the VIRTIO_NET_F_GUEST_HDRLEN feature has not been negotiated,
    \field{hdr_len} is a hint to the device as to how much of the header
    needs to be kept to copy into each packet, usually set to the
    length of the headers, including the transport header\footnote{Due to various bugs in implementations, this field is not useful
as a guarantee of the transport header size.
}.

  \begin{note}
  Some devices benefit from knowledge of the exact header length.
  \end{note}

  \item \field{gso_size} is the maximum size of each packet beyond that
    header (ie. MSS).

  \item If the driver negotiated the VIRTIO_NET_F_HOST_ECN feature,
    the VIRTIO_NET_HDR_GSO_ECN bit in \field{gso_type}
    indicates that the TCP packet has the ECN bit set\footnote{This case is not handled by some older hardware, so is called out
specifically in the protocol.}.
   \end{itemize}

\item If the driver negotiated the VIRTIO_NET_F_HOST_UDP_TUNNEL_GSO feature and the
  \field{gso_type} has the VIRTIO_NET_HDR_GSO_UDP_TUNNEL_IPV4 bit or
  VIRTIO_NET_HDR_GSO_UDP_TUNNEL_IPV6 bit set, the GSO protocol is encapsulated
  in a UDP tunnel.
  If the outer UDP header requires checksumming, the driver must have
  additionally negotiated the VIRTIO_NET_F_HOST_UDP_TUNNEL_GSO_CSUM feature
  and offloaded the outer checksum accordingly, otherwise
  the outer UDP header must not require checksum validation, i.e. the outer
  UDP checksum must be positive zero (0x0) as defined in UDP RFC 768.
  The other tunnel-related fields indicate how to replicate the packet
  headers to cut it into smaller packets:

  \begin{itemize}
  \item \field{outer_th_offset} field indicates the outer transport header within
      the packet. This field differs from \field{csum_start} as the latter
      points to the inner transport header within the packet.

  \item \field{inner_nh_offset} field indicates the inner network header within
      the packet.
  \end{itemize}

\begin{note}
For example, consider a partially checksummed TCP (IPv4) packet carried over a
Geneve UDP tunnel (again IPv4) with no tunnel options. The
only relevant variable related to the tunnel type is the tunnel header length.
The packet will have a 14 byte outer ethernet header, 20 byte outer IP header
followed by the 8 byte UDP header (with a 0 checksum value), 8 byte Geneve header,
14 byte inner ethernet header, 20 byte inner IP header
and the TCP header (with the TCP checksum field 16 bytes
into that header). \field{csum_start} will be 14+20+8+8+14+20 = 84 (the TCP
checksum includes the header), \field{csum_offset} will be 16.
\field{inner_nh_offset} will be 14+20+8+8+14 = 62, \field{outer_th_offset} will be
14+20+8 = 42 and \field{gso_type} will be
VIRTIO_NET_HDR_GSO_TCPV4 | VIRTIO_NET_HDR_GSO_UDP_TUNNEL_IPV4 = 0x21
\end{note}

\item If the driver negotiated the VIRTIO_NET_F_HOST_UDP_TUNNEL_GSO_CSUM feature,
  the transmitted packet is a GSO one encapsulated in a UDP tunnel, and
  the outer UDP header requires checksumming, the driver can skip checksumming
  the outer header:

  \begin{itemize}
  \item \field{flags} has the VIRTIO_NET_HDR_F_UDP_TUNNEL_CSUM set,

  \item The outer UDP checksum field in the packet is set to the sum
    of the UDP pseudo header, so that replacing it by the ones'
    complement checksum of the outer UDP header and payload will give the
    correct result.
  \end{itemize}

\item \field{num_buffers} is set to zero.  This field is unused on transmitted packets.

\item The header and packet are added as one output descriptor to the
  transmitq, and the device is notified of the new entry
  (see \ref{sec:Device Types / Network Device / Device Initialization}~\nameref{sec:Device Types / Network Device / Device Initialization}).
\end{enumerate}

\drivernormative{\paragraph}{Packet Transmission}{Device Types / Network Device / Device Operation / Packet Transmission}

For the transmit packet buffer, the driver MUST use the size of the
structure \field{struct virtio_net_hdr} same as the receive packet buffer.

The driver MUST set \field{num_buffers} to zero.

If VIRTIO_NET_F_CSUM is not negotiated, the driver MUST set
\field{flags} to zero and SHOULD supply a fully checksummed
packet to the device.

If VIRTIO_NET_F_HOST_TSO4 is negotiated, the driver MAY set
\field{gso_type} to VIRTIO_NET_HDR_GSO_TCPV4 to request TCPv4
segmentation, otherwise the driver MUST NOT set
\field{gso_type} to VIRTIO_NET_HDR_GSO_TCPV4.

If VIRTIO_NET_F_HOST_TSO6 is negotiated, the driver MAY set
\field{gso_type} to VIRTIO_NET_HDR_GSO_TCPV6 to request TCPv6
segmentation, otherwise the driver MUST NOT set
\field{gso_type} to VIRTIO_NET_HDR_GSO_TCPV6.

If VIRTIO_NET_F_HOST_UFO is negotiated, the driver MAY set
\field{gso_type} to VIRTIO_NET_HDR_GSO_UDP to request UDP
fragmentation, otherwise the driver MUST NOT set
\field{gso_type} to VIRTIO_NET_HDR_GSO_UDP.

If VIRTIO_NET_F_HOST_USO is negotiated, the driver MAY set
\field{gso_type} to VIRTIO_NET_HDR_GSO_UDP_L4 to request UDP
segmentation, otherwise the driver MUST NOT set
\field{gso_type} to VIRTIO_NET_HDR_GSO_UDP_L4.

The driver SHOULD NOT send to the device TCP packets requiring segmentation offload
which have the Explicit Congestion Notification bit set, unless the
VIRTIO_NET_F_HOST_ECN feature is negotiated, in which case the
driver MUST set the VIRTIO_NET_HDR_GSO_ECN bit in
\field{gso_type}.

If VIRTIO_NET_F_HOST_UDP_TUNNEL_GSO is negotiated, the driver MAY set
VIRTIO_NET_HDR_GSO_UDP_TUNNEL_IPV4 bit or the VIRTIO_NET_HDR_GSO_UDP_TUNNEL_IPV6 bit
in \field{gso_type} according to the inner network header protocol type
to request GSO packets over UDPv4 or UDPv6 tunnel segmentation,
otherwise the driver MUST NOT set either the
VIRTIO_NET_HDR_GSO_UDP_TUNNEL_IPV4 bit or the VIRTIO_NET_HDR_GSO_UDP_TUNNEL_IPV6 bit
in \field{gso_type}.

When requesting GSO segmentation over UDP tunnel, the driver MUST SET the
VIRTIO_NET_HDR_GSO_UDP_TUNNEL_IPV4 bit if the inner network header is IPv4, i.e. the
packet is a TCPv4 GSO one, otherwise, if the inner network header is IPv6, the driver
MUST SET the VIRTIO_NET_HDR_GSO_UDP_TUNNEL_IPV6 bit.

The driver MUST NOT send to the device GSO packets over UDP tunnel
requiring segmentation and outer UDP checksum offload, unless both the
VIRTIO_NET_F_HOST_UDP_TUNNEL_GSO and VIRTIO_NET_F_HOST_UDP_TUNNEL_GSO_CSUM features
are negotiated, in which case the driver MUST set either the
VIRTIO_NET_HDR_GSO_UDP_TUNNEL_IPV4 bit or the VIRTIO_NET_HDR_GSO_UDP_TUNNEL_IPV6
bit in the \field{gso_type} and the VIRTIO_NET_HDR_F_UDP_TUNNEL_CSUM bit in
the \field{flags}.

If VIRTIO_NET_F_HOST_UDP_TUNNEL_GSO_CSUM is not negotiated, the driver MUST not set
the VIRTIO_NET_HDR_F_UDP_TUNNEL_CSUM bit in the \field{flags} and
MUST NOT send to the device GSO packets over UDP tunnel
requiring segmentation and outer UDP checksum offload.

The driver MUST NOT set the VIRTIO_NET_HDR_GSO_UDP_TUNNEL_IPV4 bit or the
VIRTIO_NET_HDR_GSO_UDP_TUNNEL_IPV6 bit together with VIRTIO_NET_HDR_GSO_UDP, as the
latter is deprecated in favor of UDP_L4 and no new feature will support it.

The driver MUST NOT set the VIRTIO_NET_HDR_GSO_UDP_TUNNEL_IPV4 bit and the
VIRTIO_NET_HDR_GSO_UDP_TUNNEL_IPV6 bit together.

The driver MUST NOT set the VIRTIO_NET_HDR_F_UDP_TUNNEL_CSUM bit \field{flags}
without setting either the VIRTIO_NET_HDR_GSO_UDP_TUNNEL_IPV4 bit or
the VIRTIO_NET_HDR_GSO_UDP_TUNNEL_IPV6 bit in \field{gso_type}.

If the VIRTIO_NET_F_CSUM feature has been negotiated, the
driver MAY set the VIRTIO_NET_HDR_F_NEEDS_CSUM bit in
\field{flags}, if so:
\begin{enumerate}
\item the driver MUST validate the packet checksum at
	offset \field{csum_offset} from \field{csum_start} as well as all
	preceding offsets;
\begin{note}
If \field{gso_type} differs from VIRTIO_NET_HDR_GSO_NONE and the
VIRTIO_NET_HDR_GSO_UDP_TUNNEL_IPV4 bit or the VIRTIO_NET_HDR_GSO_UDP_TUNNEL_IPV6
bit are not set in \field{gso_type}, \field{csum_offset}
points to the only transport header present in the packet, and there are no
additional preceding checksums validated by VIRTIO_NET_HDR_F_NEEDS_CSUM.
\end{note}
\item the driver MUST set the packet checksum stored in the
	buffer to the TCP/UDP pseudo header;
\item the driver MUST set \field{csum_start} and
	\field{csum_offset} such that calculating a ones'
	complement checksum from \field{csum_start} up until the end of
	the packet and storing the result at offset \field{csum_offset}
	from  \field{csum_start} will result in a fully checksummed
	packet;
\end{enumerate}

If none of the VIRTIO_NET_F_HOST_TSO4, TSO6, USO or UFO options have
been negotiated, the driver MUST set \field{gso_type} to
VIRTIO_NET_HDR_GSO_NONE.

If \field{gso_type} differs from VIRTIO_NET_HDR_GSO_NONE, then
the driver MUST also set the VIRTIO_NET_HDR_F_NEEDS_CSUM bit in
\field{flags} and MUST set \field{gso_size} to indicate the
desired MSS.

If one of the VIRTIO_NET_F_HOST_TSO4, TSO6, USO or UFO options have
been negotiated:
\begin{itemize}
\item If the VIRTIO_NET_F_GUEST_HDRLEN feature has been negotiated,
	and \field{gso_type} differs from VIRTIO_NET_HDR_GSO_NONE,
	the driver MUST set \field{hdr_len} to a value equal to the length
	of the headers, including the transport header. If \field{gso_type}
	has the VIRTIO_NET_HDR_GSO_UDP_TUNNEL_IPV4 bit or the
	VIRTIO_NET_HDR_GSO_UDP_TUNNEL_IPV6 bit set, \field{hdr_len} includes
	the inner transport header.

\item If the VIRTIO_NET_F_GUEST_HDRLEN feature has not been negotiated,
	or \field{gso_type} is VIRTIO_NET_HDR_GSO_NONE,
	the driver SHOULD set \field{hdr_len} to a value
	not less than the length of the headers, including the transport
	header.
\end{itemize}

If the VIRTIO_NET_F_HOST_UDP_TUNNEL_GSO option has been negotiated, the
driver MAY set the VIRTIO_NET_HDR_GSO_UDP_TUNNEL_IPV4 bit or the
VIRTIO_NET_HDR_GSO_UDP_TUNNEL_IPV6 bit in \field{gso_type}, if so:
\begin{itemize}
\item the driver MUST set \field{outer_th_offset} to the outer UDP header
  offset and \field{inner_nh_offset} to the inner network header offset.
  The \field{csum_start} and \field{csum_offset} fields point respectively
  to the inner transport header and inner transport checksum field.
\end{itemize}

If the VIRTIO_NET_F_HOST_UDP_TUNNEL_GSO_CSUM feature has been negotiated,
and the VIRTIO_NET_HDR_GSO_UDP_TUNNEL_IPV4 bit or
VIRTIO_NET_HDR_GSO_UDP_TUNNEL_IPV6 bit in \field{gso_type} are set,
the driver MAY set the VIRTIO_NET_HDR_F_UDP_TUNNEL_CSUM bit in
\field{flags}, if so the driver MUST set the packet outer UDP header checksum
to the outer UDP pseudo header checksum.

\begin{note}
calculating a ones' complement checksum from \field{outer_th_offset}
up until the end of the packet and storing the result at offset 6
from \field{outer_th_offset} will result in a fully checksummed outer UDP packet;
\end{note}

If the VIRTIO_NET_HDR_GSO_UDP_TUNNEL_IPV4 bit or the
VIRTIO_NET_HDR_GSO_UDP_TUNNEL_IPV6 bit in \field{gso_type} are set
and the VIRTIO_NET_F_HOST_UDP_TUNNEL_GSO_CSUM feature has not
been negotiated, the
outer UDP header MUST NOT require checksum validation. That is, the
outer UDP checksum value MUST be 0 or the validated complete checksum
for such header.

\begin{note}
The valid complete checksum of the outer UDP header of individual segments
can be computed by the driver prior to segmentation only if the GSO packet
size is a multiple of \field{gso_size}, because then all segments
have the same size and thus all data included in the outer UDP
checksum is the same for every segment. These pre-computed segment
length and checksum fields are different from those of the GSO
packet.
In this scenario the outer UDP header of the GSO packet must carry the
segmented UDP packet length.
\end{note}

If the VIRTIO_NET_F_HOST_UDP_TUNNEL_GSO option has not
been negotiated, the driver MUST NOT set either the VIRTIO_NET_HDR_F_GSO_UDP_TUNNEL_IPV4
bit or the VIRTIO_NET_HDR_F_GSO_UDP_TUNNEL_IPV6 in \field{gso_type}.

If the VIRTIO_NET_F_HOST_UDP_TUNNEL_GSO_CSUM option has not been negotiated,
the driver MUST NOT set the VIRTIO_NET_HDR_F_UDP_TUNNEL_CSUM bit
in \field{flags}.

The driver SHOULD accept the VIRTIO_NET_F_GUEST_HDRLEN feature if it has
been offered, and if it's able to provide the exact header length.

The driver MUST NOT set the VIRTIO_NET_HDR_F_DATA_VALID and
VIRTIO_NET_HDR_F_RSC_INFO bits in \field{flags}.

The driver MUST NOT set the VIRTIO_NET_HDR_F_DATA_VALID bit in \field{flags}
together with the VIRTIO_NET_HDR_F_GSO_UDP_TUNNEL_IPV4 bit or the
VIRTIO_NET_HDR_F_GSO_UDP_TUNNEL_IPV6 bit in \field{gso_type}.

\devicenormative{\paragraph}{Packet Transmission}{Device Types / Network Device / Device Operation / Packet Transmission}
The device MUST ignore \field{flag} bits that it does not recognize.

If VIRTIO_NET_HDR_F_NEEDS_CSUM bit in \field{flags} is not set, the
device MUST NOT use the \field{csum_start} and \field{csum_offset}.

If one of the VIRTIO_NET_F_HOST_TSO4, TSO6, USO or UFO options have
been negotiated:
\begin{itemize}
\item If the VIRTIO_NET_F_GUEST_HDRLEN feature has been negotiated,
	and \field{gso_type} differs from VIRTIO_NET_HDR_GSO_NONE,
	the device MAY use \field{hdr_len} as the transport header size.

	\begin{note}
	Caution should be taken by the implementation so as to prevent
	a malicious driver from attacking the device by setting an incorrect hdr_len.
	\end{note}

\item If the VIRTIO_NET_F_GUEST_HDRLEN feature has not been negotiated,
	or \field{gso_type} is VIRTIO_NET_HDR_GSO_NONE,
	the device MAY use \field{hdr_len} only as a hint about the
	transport header size.
	The device MUST NOT rely on \field{hdr_len} to be correct.

	\begin{note}
	This is due to various bugs in implementations.
	\end{note}
\end{itemize}

If both the VIRTIO_NET_HDR_GSO_UDP_TUNNEL_IPV4 bit and
the VIRTIO_NET_HDR_GSO_UDP_TUNNEL_IPV6 bit in in \field{gso_type} are set,
the device MUST NOT accept the packet.

If the VIRTIO_NET_HDR_GSO_UDP_TUNNEL_IPV4 bit and the VIRTIO_NET_HDR_GSO_UDP_TUNNEL_IPV6
bit in \field{gso_type} are not set, the device MUST NOT use the
\field{outer_th_offset} and \field{inner_nh_offset}.

If either the VIRTIO_NET_HDR_GSO_UDP_TUNNEL_IPV4 bit or
the VIRTIO_NET_HDR_GSO_UDP_TUNNEL_IPV6 bit in \field{gso_type} are set, and any of
the following is true:
\begin{itemize}
\item the VIRTIO_NET_HDR_F_NEEDS_CSUM is not set in \field{flags}
\item the VIRTIO_NET_HDR_F_DATA_VALID is set in \field{flags}
\item the \field{gso_type} excluding the VIRTIO_NET_HDR_GSO_UDP_TUNNEL_IPV4
bit and the VIRTIO_NET_HDR_GSO_UDP_TUNNEL_IPV6 bit is VIRTIO_NET_HDR_GSO_NONE
\end{itemize}
the device MUST NOT accept the packet.

If the VIRTIO_NET_HDR_F_UDP_TUNNEL_CSUM bit in \field{flags} is set,
and both the bits VIRTIO_NET_HDR_GSO_UDP_TUNNEL_IPV4 and
VIRTIO_NET_HDR_GSO_UDP_TUNNEL_IPV6 in \field{gso_type} are not set,
the device MOST NOT accept the packet.

If VIRTIO_NET_HDR_F_NEEDS_CSUM is not set, the device MUST NOT
rely on the packet checksum being correct.
\paragraph{Packet Transmission Interrupt}\label{sec:Device Types / Network Device / Device Operation / Packet Transmission / Packet Transmission Interrupt}

Often a driver will suppress transmission virtqueue interrupts
and check for used packets in the transmit path of following
packets.

The normal behavior in this interrupt handler is to retrieve
used buffers from the virtqueue and free the corresponding
headers and packets.

\subsubsection{Setting Up Receive Buffers}\label{sec:Device Types / Network Device / Device Operation / Setting Up Receive Buffers}

It is generally a good idea to keep the receive virtqueue as
fully populated as possible: if it runs out, network performance
will suffer.

If the VIRTIO_NET_F_GUEST_TSO4, VIRTIO_NET_F_GUEST_TSO6,
VIRTIO_NET_F_GUEST_UFO, VIRTIO_NET_F_GUEST_USO4 or VIRTIO_NET_F_GUEST_USO6
features are used, the maximum incoming packet
will be 65589 bytes long (14 bytes of Ethernet header, plus 40 bytes of
the IPv6 header, plus 65535 bytes of maximum IPv6 payload including any
extension header), otherwise 1514 bytes.
When VIRTIO_NET_F_HASH_REPORT is not negotiated, the required receive buffer
size is either 65601 or 1526 bytes accounting for 20 bytes of
\field{struct virtio_net_hdr} followed by receive packet.
When VIRTIO_NET_F_HASH_REPORT is negotiated, the required receive buffer
size is either 65609 or 1534 bytes accounting for 12 bytes of
\field{struct virtio_net_hdr} followed by receive packet.

\drivernormative{\paragraph}{Setting Up Receive Buffers}{Device Types / Network Device / Device Operation / Setting Up Receive Buffers}

\begin{itemize}
\item If VIRTIO_NET_F_MRG_RXBUF is not negotiated:
  \begin{itemize}
    \item If VIRTIO_NET_F_GUEST_TSO4, VIRTIO_NET_F_GUEST_TSO6, VIRTIO_NET_F_GUEST_UFO,
	VIRTIO_NET_F_GUEST_USO4 or VIRTIO_NET_F_GUEST_USO6 are negotiated, the driver SHOULD populate
      the receive queue(s) with buffers of at least 65609 bytes if
      VIRTIO_NET_F_HASH_REPORT is negotiated, and of at least 65601 bytes if not.
    \item Otherwise, the driver SHOULD populate the receive queue(s)
      with buffers of at least 1534 bytes if VIRTIO_NET_F_HASH_REPORT
      is negotiated, and of at least 1526 bytes if not.
  \end{itemize}
\item If VIRTIO_NET_F_MRG_RXBUF is negotiated, each buffer MUST be at
least size of \field{struct virtio_net_hdr},
i.e. 20 bytes if VIRTIO_NET_F_HASH_REPORT is negotiated, and 12 bytes if not.
\end{itemize}

\begin{note}
Obviously each buffer can be split across multiple descriptor elements.
\end{note}

When calculating the size of \field{struct virtio_net_hdr}, the driver
MUST consider all the fields inclusive up to \field{padding_reserved},
i.e. 20 bytes if VIRTIO_NET_F_HASH_REPORT is negotiated, and 12 bytes if not.

If VIRTIO_NET_F_MQ is negotiated, each of receiveq1\ldots receiveqN
that will be used SHOULD be populated with receive buffers.

\devicenormative{\paragraph}{Setting Up Receive Buffers}{Device Types / Network Device / Device Operation / Setting Up Receive Buffers}

The device MUST set \field{num_buffers} to the number of descriptors used to
hold the incoming packet.

The device MUST use only a single descriptor if VIRTIO_NET_F_MRG_RXBUF
was not negotiated.
\begin{note}
{This means that \field{num_buffers} will always be 1
if VIRTIO_NET_F_MRG_RXBUF is not negotiated.}
\end{note}

\subsubsection{Processing of Incoming Packets}\label{sec:Device Types / Network Device / Device Operation / Processing of Incoming Packets}
\label{sec:Device Types / Network Device / Device Operation / Processing of Packets}%old label for latexdiff

When a packet is copied into a buffer in the receiveq, the
optimal path is to disable further used buffer notifications for the
receiveq and process packets until no more are found, then re-enable
them.

Processing incoming packets involves:

\begin{enumerate}
\item \field{num_buffers} indicates how many descriptors
  this packet is spread over (including this one): this will
  always be 1 if VIRTIO_NET_F_MRG_RXBUF was not negotiated.
  This allows receipt of large packets without having to allocate large
  buffers: a packet that does not fit in a single buffer can flow
  over to the next buffer, and so on. In this case, there will be
  at least \field{num_buffers} used buffers in the virtqueue, and the device
  chains them together to form a single packet in a way similar to
  how it would store it in a single buffer spread over multiple
  descriptors.
  The other buffers will not begin with a \field{struct virtio_net_hdr}.

\item If
  \field{num_buffers} is one, then the entire packet will be
  contained within this buffer, immediately following the struct
  virtio_net_hdr.
\item If the VIRTIO_NET_F_GUEST_CSUM feature was negotiated, the
  VIRTIO_NET_HDR_F_DATA_VALID bit in \field{flags} can be
  set: if so, device has validated the packet checksum.
  If the VIRTIO_NET_F_GUEST_UDP_TUNNEL_GSO_CSUM feature has been negotiated,
  and the VIRTIO_NET_HDR_F_UDP_TUNNEL_CSUM bit is set in \field{flags},
  both the outer UDP checksum and the inner transport checksum
  have been validated, otherwise only one level of checksums (the outer one
  in case of tunnels) has been validated.
\end{enumerate}

Additionally, VIRTIO_NET_F_GUEST_CSUM, TSO4, TSO6, UDP, UDP_TUNNEL
and ECN features enable receive checksum, large receive offload and ECN
support which are the input equivalents of the transmit checksum,
transmit segmentation offloading and ECN features, as described
in \ref{sec:Device Types / Network Device / Device Operation /
Packet Transmission}:
\begin{enumerate}
\item If the VIRTIO_NET_F_GUEST_TSO4, TSO6, UFO, USO4 or USO6 options were
  negotiated, then \field{gso_type} MAY be something other than
  VIRTIO_NET_HDR_GSO_NONE, and \field{gso_size} field indicates the
  desired MSS (see Packet Transmission point 2).
\item If the VIRTIO_NET_F_RSC_EXT option was negotiated (this
  implies one of VIRTIO_NET_F_GUEST_TSO4, TSO6), the
  device processes also duplicated ACK segments, reports
  number of coalesced TCP segments in \field{csum_start} field and
  number of duplicated ACK segments in \field{csum_offset} field
  and sets bit VIRTIO_NET_HDR_F_RSC_INFO in \field{flags}.
\item If the VIRTIO_NET_F_GUEST_CSUM feature was negotiated, the
  VIRTIO_NET_HDR_F_NEEDS_CSUM bit in \field{flags} can be
  set: if so, the packet checksum at offset \field{csum_offset}
  from \field{csum_start} and any preceding checksums
  have been validated.  The checksum on the packet is incomplete and
  if bit VIRTIO_NET_HDR_F_RSC_INFO is not set in \field{flags},
  then \field{csum_start} and \field{csum_offset} indicate how to calculate it
  (see Packet Transmission point 1).
\begin{note}
If \field{gso_type} differs from VIRTIO_NET_HDR_GSO_NONE and the
VIRTIO_NET_HDR_GSO_UDP_TUNNEL_IPV4 bit or the VIRTIO_NET_HDR_GSO_UDP_TUNNEL_IPV6
bit are not set, \field{csum_offset}
points to the only transport header present in the packet, and there are no
additional preceding checksums validated by VIRTIO_NET_HDR_F_NEEDS_CSUM.
\end{note}
\item If the VIRTIO_NET_F_GUEST_UDP_TUNNEL_GSO option was negotiated and
  \field{gso_type} is not VIRTIO_NET_HDR_GSO_NONE, the
  VIRTIO_NET_HDR_GSO_UDP_TUNNEL_IPV4 bit or the VIRTIO_NET_HDR_GSO_UDP_TUNNEL_IPV6
  bit MAY be set. In such case the \field{outer_th_offset} and
  \field{inner_nh_offset} fields indicate the corresponding
  headers information.
\item If the VIRTIO_NET_F_GUEST_UDP_TUNNEL_GSO_CSUM feature was
negotiated, and
  the VIRTIO_NET_HDR_GSO_UDP_TUNNEL_IPV4 bit or the VIRTIO_NET_HDR_GSO_UDP_TUNNEL_IPV6
  are set in \field{gso_type}, the VIRTIO_NET_HDR_F_UDP_TUNNEL_CSUM bit in the
  \field{flags} can be set: if so, the outer UDP checksum has been validated
  and the UDP header checksum at offset 6 from from \field{outer_th_offset}
  is set to the outer UDP pseudo header checksum.

\begin{note}
If the VIRTIO_NET_HDR_GSO_UDP_TUNNEL_IPV4 bit or VIRTIO_NET_HDR_GSO_UDP_TUNNEL_IPV6
bit are set in \field{gso_type}, the \field{csum_start} field refers to
the inner transport header offset (see Packet Transmission point 1).
If the VIRTIO_NET_HDR_F_UDP_TUNNEL_CSUM bit in \field{flags} is set both
the inner and the outer header checksums have been validated by
VIRTIO_NET_HDR_F_NEEDS_CSUM, otherwise only the inner transport header
checksum has been validated.
\end{note}
\end{enumerate}

If applicable, the device calculates per-packet hash for incoming packets as
defined in \ref{sec:Device Types / Network Device / Device Operation / Processing of Incoming Packets / Hash calculation for incoming packets}.

If applicable, the device reports hash information for incoming packets as
defined in \ref{sec:Device Types / Network Device / Device Operation / Processing of Incoming Packets / Hash reporting for incoming packets}.

\devicenormative{\paragraph}{Processing of Incoming Packets}{Device Types / Network Device / Device Operation / Processing of Incoming Packets}
\label{devicenormative:Device Types / Network Device / Device Operation / Processing of Packets}%old label for latexdiff

If VIRTIO_NET_F_MRG_RXBUF has not been negotiated, the device MUST set
\field{num_buffers} to 1.

If VIRTIO_NET_F_MRG_RXBUF has been negotiated, the device MUST set
\field{num_buffers} to indicate the number of buffers
the packet (including the header) is spread over.

If a receive packet is spread over multiple buffers, the device
MUST use all buffers but the last (i.e. the first \field{num_buffers} -
1 buffers) completely up to the full length of each buffer
supplied by the driver.

The device MUST use all buffers used by a single receive
packet together, such that at least \field{num_buffers} are
observed by driver as used.

If VIRTIO_NET_F_GUEST_CSUM is not negotiated, the device MUST set
\field{flags} to zero and SHOULD supply a fully checksummed
packet to the driver.

If VIRTIO_NET_F_GUEST_TSO4 is not negotiated, the device MUST NOT set
\field{gso_type} to VIRTIO_NET_HDR_GSO_TCPV4.

If VIRTIO_NET_F_GUEST_UDP is not negotiated, the device MUST NOT set
\field{gso_type} to VIRTIO_NET_HDR_GSO_UDP.

If VIRTIO_NET_F_GUEST_TSO6 is not negotiated, the device MUST NOT set
\field{gso_type} to VIRTIO_NET_HDR_GSO_TCPV6.

If none of VIRTIO_NET_F_GUEST_USO4 or VIRTIO_NET_F_GUEST_USO6 have been negotiated,
the device MUST NOT set \field{gso_type} to VIRTIO_NET_HDR_GSO_UDP_L4.

If VIRTIO_NET_F_GUEST_UDP_TUNNEL_GSO is not negotiated, the device MUST NOT set
either the VIRTIO_NET_HDR_GSO_UDP_TUNNEL_IPV4 bit or the
VIRTIO_NET_HDR_GSO_UDP_TUNNEL_IPV6 bit in \field{gso_type}.

If VIRTIO_NET_F_GUEST_UDP_TUNNEL_GSO_CSUM is not negotiated the device MUST NOT set
the VIRTIO_NET_HDR_F_UDP_TUNNEL_CSUM bit in \field{flags}.

The device SHOULD NOT send to the driver TCP packets requiring segmentation offload
which have the Explicit Congestion Notification bit set, unless the
VIRTIO_NET_F_GUEST_ECN feature is negotiated, in which case the
device MUST set the VIRTIO_NET_HDR_GSO_ECN bit in
\field{gso_type}.

If the VIRTIO_NET_F_GUEST_CSUM feature has been negotiated, the
device MAY set the VIRTIO_NET_HDR_F_NEEDS_CSUM bit in
\field{flags}, if so:
\begin{enumerate}
\item the device MUST validate the packet checksum at
	offset \field{csum_offset} from \field{csum_start} as well as all
	preceding offsets;
\item the device MUST set the packet checksum stored in the
	receive buffer to the TCP/UDP pseudo header;
\item the device MUST set \field{csum_start} and
	\field{csum_offset} such that calculating a ones'
	complement checksum from \field{csum_start} up until the
	end of the packet and storing the result at offset
	\field{csum_offset} from  \field{csum_start} will result in a
	fully checksummed packet;
\end{enumerate}

The device MUST NOT send to the driver GSO packets encapsulated in UDP
tunnel and requiring segmentation offload, unless the
VIRTIO_NET_F_GUEST_UDP_TUNNEL_GSO is negotiated, in which case the device MUST set
the VIRTIO_NET_HDR_GSO_UDP_TUNNEL_IPV4 bit or the VIRTIO_NET_HDR_GSO_UDP_TUNNEL_IPV6
bit in \field{gso_type} according to the inner network header protocol type,
MUST set the \field{outer_th_offset} and \field{inner_nh_offset} fields
to the corresponding header information, and the outer UDP header MUST NOT
require checksum offload.

If the VIRTIO_NET_F_GUEST_UDP_TUNNEL_GSO_CSUM feature has not been negotiated,
the device MUST NOT send the driver GSO packets encapsulated in UDP
tunnel and requiring segmentation and outer checksum offload.

If none of the VIRTIO_NET_F_GUEST_TSO4, TSO6, UFO, USO4 or USO6 options have
been negotiated, the device MUST set \field{gso_type} to
VIRTIO_NET_HDR_GSO_NONE.

If \field{gso_type} differs from VIRTIO_NET_HDR_GSO_NONE, then
the device MUST also set the VIRTIO_NET_HDR_F_NEEDS_CSUM bit in
\field{flags} MUST set \field{gso_size} to indicate the desired MSS.
If VIRTIO_NET_F_RSC_EXT was negotiated, the device MUST also
set VIRTIO_NET_HDR_F_RSC_INFO bit in \field{flags},
set \field{csum_start} to number of coalesced TCP segments and
set \field{csum_offset} to number of received duplicated ACK segments.

If VIRTIO_NET_F_RSC_EXT was not negotiated, the device MUST
not set VIRTIO_NET_HDR_F_RSC_INFO bit in \field{flags}.

If one of the VIRTIO_NET_F_GUEST_TSO4, TSO6, UFO, USO4 or USO6 options have
been negotiated, the device SHOULD set \field{hdr_len} to a value
not less than the length of the headers, including the transport
header. If \field{gso_type} has the VIRTIO_NET_HDR_GSO_UDP_TUNNEL_IPV4 bit
or the VIRTIO_NET_HDR_GSO_UDP_TUNNEL_IPV6 bit set, the referenced transport
header is the inner one.

If the VIRTIO_NET_F_GUEST_CSUM feature has been negotiated, the
device MAY set the VIRTIO_NET_HDR_F_DATA_VALID bit in
\field{flags}, if so, the device MUST validate the packet
checksum. If the VIRTIO_NET_F_GUEST_UDP_TUNNEL_GSO_CSUM feature has
been negotiated, and the VIRTIO_NET_HDR_F_UDP_TUNNEL_CSUM bit set in
\field{flags}, both the outer UDP checksum and the inner transport
checksum have been validated.
Otherwise level of checksum is validated: in case of multiple
encapsulated protocols the outermost one.

If either the VIRTIO_NET_HDR_GSO_UDP_TUNNEL_IPV4 bit or the
VIRTIO_NET_HDR_GSO_UDP_TUNNEL_IPV6 bit in \field{gso_type} are set,
the device MUST NOT set the VIRTIO_NET_HDR_F_DATA_VALID bit in
\field{flags}.

If the VIRTIO_NET_F_GUEST_UDP_TUNNEL_GSO_CSUM feature has been negotiated
and either the VIRTIO_NET_HDR_GSO_UDP_TUNNEL_IPV4 bit is set or the
VIRTIO_NET_HDR_GSO_UDP_TUNNEL_IPV6 bit is set in \field{gso_type}, the
device MAY set the VIRTIO_NET_HDR_F_UDP_TUNNEL_CSUM bit in
\field{flags}, if so the device MUST set the packet outer UDP checksum
stored in the receive buffer to the outer UDP pseudo header.

Otherwise, the VIRTIO_NET_F_GUEST_UDP_TUNNEL_GSO_CSUM feature has been
negotiated, either the VIRTIO_NET_HDR_GSO_UDP_TUNNEL_IPV4 bit is set or the
VIRTIO_NET_HDR_GSO_UDP_TUNNEL_IPV6 bit is set in \field{gso_type},
and the bit VIRTIO_NET_HDR_F_UDP_TUNNEL_CSUM is not set in
\field{flags}, the device MUST either provide a zero outer UDP header
checksum or a fully checksummed outer UDP header.

\drivernormative{\paragraph}{Processing of Incoming
Packets}{Device Types / Network Device / Device Operation /
Processing of Incoming Packets}

The driver MUST ignore \field{flag} bits that it does not recognize.

If VIRTIO_NET_HDR_F_NEEDS_CSUM bit in \field{flags} is not set or
if VIRTIO_NET_HDR_F_RSC_INFO bit \field{flags} is set, the
driver MUST NOT use the \field{csum_start} and \field{csum_offset}.

If one of the VIRTIO_NET_F_GUEST_TSO4, TSO6, UFO, USO4 or USO6 options have
been negotiated, the driver MAY use \field{hdr_len} only as a hint about the
transport header size.
The driver MUST NOT rely on \field{hdr_len} to be correct.
\begin{note}
This is due to various bugs in implementations.
\end{note}

If neither VIRTIO_NET_HDR_F_NEEDS_CSUM nor
VIRTIO_NET_HDR_F_DATA_VALID is set, the driver MUST NOT
rely on the packet checksum being correct.

If both the VIRTIO_NET_HDR_GSO_UDP_TUNNEL_IPV4 bit and
the VIRTIO_NET_HDR_GSO_UDP_TUNNEL_IPV6 bit in in \field{gso_type} are set,
the driver MUST NOT accept the packet.

If the VIRTIO_NET_HDR_GSO_UDP_TUNNEL_IPV4 bit or the VIRTIO_NET_HDR_GSO_UDP_TUNNEL_IPV6
bit in \field{gso_type} are not set, the driver MUST NOT use the
\field{outer_th_offset} and \field{inner_nh_offset}.

If either the VIRTIO_NET_HDR_GSO_UDP_TUNNEL_IPV4 bit or
the VIRTIO_NET_HDR_GSO_UDP_TUNNEL_IPV6 bit in \field{gso_type} are set, and any of
the following is true:
\begin{itemize}
\item the VIRTIO_NET_HDR_F_NEEDS_CSUM bit is not set in \field{flags}
\item the VIRTIO_NET_HDR_F_DATA_VALID bit is set in \field{flags}
\item the \field{gso_type} excluding the VIRTIO_NET_HDR_GSO_UDP_TUNNEL_IPV4
bit and the VIRTIO_NET_HDR_GSO_UDP_TUNNEL_IPV6 bit is VIRTIO_NET_HDR_GSO_NONE
\end{itemize}
the driver MUST NOT accept the packet.

If the VIRTIO_NET_HDR_F_UDP_TUNNEL_CSUM bit and the VIRTIO_NET_HDR_F_NEEDS_CSUM
bit in \field{flags} are set,
and both the bits VIRTIO_NET_HDR_GSO_UDP_TUNNEL_IPV4 and
VIRTIO_NET_HDR_GSO_UDP_TUNNEL_IPV6 in \field{gso_type} are not set,
the driver MOST NOT accept the packet.

\paragraph{Hash calculation for incoming packets}
\label{sec:Device Types / Network Device / Device Operation / Processing of Incoming Packets / Hash calculation for incoming packets}

A device attempts to calculate a per-packet hash in the following cases:
\begin{itemize}
\item The feature VIRTIO_NET_F_RSS was negotiated. The device uses the hash to determine the receive virtqueue to place incoming packets.
\item The feature VIRTIO_NET_F_HASH_REPORT was negotiated. The device reports the hash value and the hash type with the packet.
\end{itemize}

If the feature VIRTIO_NET_F_RSS was negotiated:
\begin{itemize}
\item The device uses \field{hash_types} of the virtio_net_rss_config structure as 'Enabled hash types' bitmask.
\item If additionally the feature VIRTIO_NET_F_HASH_TUNNEL was negotiated, the device uses \field{enabled_tunnel_types} of the
      virtnet_hash_tunnel structure as 'Encapsulation types enabled for inner header hash' bitmask.
\item The device uses a key as defined in \field{hash_key_data} and \field{hash_key_length} of the virtio_net_rss_config structure (see
\ref{sec:Device Types / Network Device / Device Operation / Control Virtqueue / Receive-side scaling (RSS) / Setting RSS parameters}).
\end{itemize}

If the feature VIRTIO_NET_F_RSS was not negotiated:
\begin{itemize}
\item The device uses \field{hash_types} of the virtio_net_hash_config structure as 'Enabled hash types' bitmask.
\item If additionally the feature VIRTIO_NET_F_HASH_TUNNEL was negotiated, the device uses \field{enabled_tunnel_types} of the
      virtnet_hash_tunnel structure as 'Encapsulation types enabled for inner header hash' bitmask.
\item The device uses a key as defined in \field{hash_key_data} and \field{hash_key_length} of the virtio_net_hash_config structure (see
\ref{sec:Device Types / Network Device / Device Operation / Control Virtqueue / Automatic receive steering in multiqueue mode / Hash calculation}).
\end{itemize}

Note that if the device offers VIRTIO_NET_F_HASH_REPORT, even if it supports only one pair of virtqueues, it MUST support
at least one of commands of VIRTIO_NET_CTRL_MQ class to configure reported hash parameters:
\begin{itemize}
\item If the device offers VIRTIO_NET_F_RSS, it MUST support VIRTIO_NET_CTRL_MQ_RSS_CONFIG command per
 \ref{sec:Device Types / Network Device / Device Operation / Control Virtqueue / Receive-side scaling (RSS) / Setting RSS parameters}.
\item Otherwise the device MUST support VIRTIO_NET_CTRL_MQ_HASH_CONFIG command per
 \ref{sec:Device Types / Network Device / Device Operation / Control Virtqueue / Automatic receive steering in multiqueue mode / Hash calculation}.
\end{itemize}

The per-packet hash calculation can depend on the IP packet type. See
\hyperref[intro:IP]{[IP]}, \hyperref[intro:UDP]{[UDP]} and \hyperref[intro:TCP]{[TCP]}.

\subparagraph{Supported/enabled hash types}
\label{sec:Device Types / Network Device / Device Operation / Processing of Incoming Packets / Hash calculation for incoming packets / Supported/enabled hash types}
Hash types applicable for IPv4 packets:
\begin{lstlisting}
#define VIRTIO_NET_HASH_TYPE_IPv4              (1 << 0)
#define VIRTIO_NET_HASH_TYPE_TCPv4             (1 << 1)
#define VIRTIO_NET_HASH_TYPE_UDPv4             (1 << 2)
\end{lstlisting}
Hash types applicable for IPv6 packets without extension headers
\begin{lstlisting}
#define VIRTIO_NET_HASH_TYPE_IPv6              (1 << 3)
#define VIRTIO_NET_HASH_TYPE_TCPv6             (1 << 4)
#define VIRTIO_NET_HASH_TYPE_UDPv6             (1 << 5)
\end{lstlisting}
Hash types applicable for IPv6 packets with extension headers
\begin{lstlisting}
#define VIRTIO_NET_HASH_TYPE_IP_EX             (1 << 6)
#define VIRTIO_NET_HASH_TYPE_TCP_EX            (1 << 7)
#define VIRTIO_NET_HASH_TYPE_UDP_EX            (1 << 8)
\end{lstlisting}

\subparagraph{IPv4 packets}
\label{sec:Device Types / Network Device / Device Operation / Processing of Incoming Packets / Hash calculation for incoming packets / IPv4 packets}
The device calculates the hash on IPv4 packets according to 'Enabled hash types' bitmask as follows:
\begin{itemize}
\item If VIRTIO_NET_HASH_TYPE_TCPv4 is set and the packet has
a TCP header, the hash is calculated over the following fields:
\begin{itemize}
\item Source IP address
\item Destination IP address
\item Source TCP port
\item Destination TCP port
\end{itemize}
\item Else if VIRTIO_NET_HASH_TYPE_UDPv4 is set and the
packet has a UDP header, the hash is calculated over the following fields:
\begin{itemize}
\item Source IP address
\item Destination IP address
\item Source UDP port
\item Destination UDP port
\end{itemize}
\item Else if VIRTIO_NET_HASH_TYPE_IPv4 is set, the hash is
calculated over the following fields:
\begin{itemize}
\item Source IP address
\item Destination IP address
\end{itemize}
\item Else the device does not calculate the hash
\end{itemize}

\subparagraph{IPv6 packets without extension header}
\label{sec:Device Types / Network Device / Device Operation / Processing of Incoming Packets / Hash calculation for incoming packets / IPv6 packets without extension header}
The device calculates the hash on IPv6 packets without extension
headers according to 'Enabled hash types' bitmask as follows:
\begin{itemize}
\item If VIRTIO_NET_HASH_TYPE_TCPv6 is set and the packet has
a TCPv6 header, the hash is calculated over the following fields:
\begin{itemize}
\item Source IPv6 address
\item Destination IPv6 address
\item Source TCP port
\item Destination TCP port
\end{itemize}
\item Else if VIRTIO_NET_HASH_TYPE_UDPv6 is set and the
packet has a UDPv6 header, the hash is calculated over the following fields:
\begin{itemize}
\item Source IPv6 address
\item Destination IPv6 address
\item Source UDP port
\item Destination UDP port
\end{itemize}
\item Else if VIRTIO_NET_HASH_TYPE_IPv6 is set, the hash is
calculated over the following fields:
\begin{itemize}
\item Source IPv6 address
\item Destination IPv6 address
\end{itemize}
\item Else the device does not calculate the hash
\end{itemize}

\subparagraph{IPv6 packets with extension header}
\label{sec:Device Types / Network Device / Device Operation / Processing of Incoming Packets / Hash calculation for incoming packets / IPv6 packets with extension header}
The device calculates the hash on IPv6 packets with extension
headers according to 'Enabled hash types' bitmask as follows:
\begin{itemize}
\item If VIRTIO_NET_HASH_TYPE_TCP_EX is set and the packet
has a TCPv6 header, the hash is calculated over the following fields:
\begin{itemize}
\item Home address from the home address option in the IPv6 destination options header. If the extension header is not present, use the Source IPv6 address.
\item IPv6 address that is contained in the Routing-Header-Type-2 from the associated extension header. If the extension header is not present, use the Destination IPv6 address.
\item Source TCP port
\item Destination TCP port
\end{itemize}
\item Else if VIRTIO_NET_HASH_TYPE_UDP_EX is set and the
packet has a UDPv6 header, the hash is calculated over the following fields:
\begin{itemize}
\item Home address from the home address option in the IPv6 destination options header. If the extension header is not present, use the Source IPv6 address.
\item IPv6 address that is contained in the Routing-Header-Type-2 from the associated extension header. If the extension header is not present, use the Destination IPv6 address.
\item Source UDP port
\item Destination UDP port
\end{itemize}
\item Else if VIRTIO_NET_HASH_TYPE_IP_EX is set, the hash is
calculated over the following fields:
\begin{itemize}
\item Home address from the home address option in the IPv6 destination options header. If the extension header is not present, use the Source IPv6 address.
\item IPv6 address that is contained in the Routing-Header-Type-2 from the associated extension header. If the extension header is not present, use the Destination IPv6 address.
\end{itemize}
\item Else skip IPv6 extension headers and calculate the hash as
defined for an IPv6 packet without extension headers
(see \ref{sec:Device Types / Network Device / Device Operation / Processing of Incoming Packets / Hash calculation for incoming packets / IPv6 packets without extension header}).
\end{itemize}

\paragraph{Inner Header Hash}
\label{sec:Device Types / Network Device / Device Operation / Processing of Incoming Packets / Inner Header Hash}

If VIRTIO_NET_F_HASH_TUNNEL has been negotiated, the driver can send the command
VIRTIO_NET_CTRL_HASH_TUNNEL_SET to configure the calculation of the inner header hash.

\begin{lstlisting}
struct virtnet_hash_tunnel {
    le32 enabled_tunnel_types;
};

#define VIRTIO_NET_CTRL_HASH_TUNNEL 7
 #define VIRTIO_NET_CTRL_HASH_TUNNEL_SET 0
\end{lstlisting}

Field \field{enabled_tunnel_types} contains the bitmask of encapsulation types enabled for inner header hash.
See \ref{sec:Device Types / Network Device / Device Operation / Processing of Incoming Packets /
Hash calculation for incoming packets / Encapsulation types supported/enabled for inner header hash}.

The class VIRTIO_NET_CTRL_HASH_TUNNEL has one command:
VIRTIO_NET_CTRL_HASH_TUNNEL_SET sets \field{enabled_tunnel_types} for the device using the
virtnet_hash_tunnel structure, which is read-only for the device.

Inner header hash is disabled by VIRTIO_NET_CTRL_HASH_TUNNEL_SET with \field{enabled_tunnel_types} set to 0.

Initially (before the driver sends any VIRTIO_NET_CTRL_HASH_TUNNEL_SET command) all
encapsulation types are disabled for inner header hash.

\subparagraph{Encapsulated packet}
\label{sec:Device Types / Network Device / Device Operation / Processing of Incoming Packets / Hash calculation for incoming packets / Encapsulated packet}

Multiple tunneling protocols allow encapsulating an inner, payload packet in an outer, encapsulated packet.
The encapsulated packet thus contains an outer header and an inner header, and the device calculates the
hash over either the inner header or the outer header.

If VIRTIO_NET_F_HASH_TUNNEL is negotiated and a received encapsulated packet's outer header matches one of the
encapsulation types enabled in \field{enabled_tunnel_types}, then the device uses the inner header for hash
calculations (only a single level of encapsulation is currently supported).

If VIRTIO_NET_F_HASH_TUNNEL is negotiated and a received packet's (outer) header does not match any encapsulation
types enabled in \field{enabled_tunnel_types}, then the device uses the outer header for hash calculations.

\subparagraph{Encapsulation types supported/enabled for inner header hash}
\label{sec:Device Types / Network Device / Device Operation / Processing of Incoming Packets /
Hash calculation for incoming packets / Encapsulation types supported/enabled for inner header hash}

Encapsulation types applicable for inner header hash:
\begin{lstlisting}[escapechar=|]
#define VIRTIO_NET_HASH_TUNNEL_TYPE_GRE_2784    (1 << 0) /* |\hyperref[intro:rfc2784]{[RFC2784]}| */
#define VIRTIO_NET_HASH_TUNNEL_TYPE_GRE_2890    (1 << 1) /* |\hyperref[intro:rfc2890]{[RFC2890]}| */
#define VIRTIO_NET_HASH_TUNNEL_TYPE_GRE_7676    (1 << 2) /* |\hyperref[intro:rfc7676]{[RFC7676]}| */
#define VIRTIO_NET_HASH_TUNNEL_TYPE_GRE_UDP     (1 << 3) /* |\hyperref[intro:rfc8086]{[GRE-in-UDP]}| */
#define VIRTIO_NET_HASH_TUNNEL_TYPE_VXLAN       (1 << 4) /* |\hyperref[intro:vxlan]{[VXLAN]}| */
#define VIRTIO_NET_HASH_TUNNEL_TYPE_VXLAN_GPE   (1 << 5) /* |\hyperref[intro:vxlan-gpe]{[VXLAN-GPE]}| */
#define VIRTIO_NET_HASH_TUNNEL_TYPE_GENEVE      (1 << 6) /* |\hyperref[intro:geneve]{[GENEVE]}| */
#define VIRTIO_NET_HASH_TUNNEL_TYPE_IPIP        (1 << 7) /* |\hyperref[intro:ipip]{[IPIP]}| */
#define VIRTIO_NET_HASH_TUNNEL_TYPE_NVGRE       (1 << 8) /* |\hyperref[intro:nvgre]{[NVGRE]}| */
\end{lstlisting}

\subparagraph{Advice}
Example uses of the inner header hash:
\begin{itemize}
\item Legacy tunneling protocols, lacking the outer header entropy, can use RSS with the inner header hash to
      distribute flows with identical outer but different inner headers across various queues, improving performance.
\item Identify an inner flow distributed across multiple outer tunnels.
\end{itemize}

As using the inner header hash completely discards the outer header entropy, care must be taken
if the inner header is controlled by an adversary, as the adversary can then intentionally create
configurations with insufficient entropy.

Besides disabling the inner header hash, mitigations would depend on how the hash is used. When the hash
use is limited to the RSS queue selection, the inner header hash may have quality of service (QoS) limitations.

\devicenormative{\subparagraph}{Inner Header Hash}{Device Types / Network Device / Device Operation / Control Virtqueue / Inner Header Hash}

If the (outer) header of the received packet does not match any encapsulation types enabled
in \field{enabled_tunnel_types}, the device MUST calculate the hash on the outer header.

If the device receives any bits in \field{enabled_tunnel_types} which are not set in \field{supported_tunnel_types},
it SHOULD respond to the VIRTIO_NET_CTRL_HASH_TUNNEL_SET command with VIRTIO_NET_ERR.

If the driver sets \field{enabled_tunnel_types} to 0 through VIRTIO_NET_CTRL_HASH_TUNNEL_SET or upon the device reset,
the device MUST disable the inner header hash for all encapsulation types.

\drivernormative{\subparagraph}{Inner Header Hash}{Device Types / Network Device / Device Operation / Control Virtqueue / Inner Header Hash}

The driver MUST have negotiated the VIRTIO_NET_F_HASH_TUNNEL feature when issuing the VIRTIO_NET_CTRL_HASH_TUNNEL_SET command.

The driver MUST NOT set any bits in \field{enabled_tunnel_types} which are not set in \field{supported_tunnel_types}.

The driver MUST ignore bits in \field{supported_tunnel_types} which are not documented in this specification.

\paragraph{Hash reporting for incoming packets}
\label{sec:Device Types / Network Device / Device Operation / Processing of Incoming Packets / Hash reporting for incoming packets}

If VIRTIO_NET_F_HASH_REPORT was negotiated and
 the device has calculated the hash for the packet, the device fills \field{hash_report} with the report type of calculated hash
and \field{hash_value} with the value of calculated hash.

If VIRTIO_NET_F_HASH_REPORT was negotiated but due to any reason the
hash was not calculated, the device sets \field{hash_report} to VIRTIO_NET_HASH_REPORT_NONE.

Possible values that the device can report in \field{hash_report} are defined below.
They correspond to supported hash types defined in
\ref{sec:Device Types / Network Device / Device Operation / Processing of Incoming Packets / Hash calculation for incoming packets / Supported/enabled hash types}
as follows:

VIRTIO_NET_HASH_TYPE_XXX = 1 << (VIRTIO_NET_HASH_REPORT_XXX - 1)

\begin{lstlisting}
#define VIRTIO_NET_HASH_REPORT_NONE            0
#define VIRTIO_NET_HASH_REPORT_IPv4            1
#define VIRTIO_NET_HASH_REPORT_TCPv4           2
#define VIRTIO_NET_HASH_REPORT_UDPv4           3
#define VIRTIO_NET_HASH_REPORT_IPv6            4
#define VIRTIO_NET_HASH_REPORT_TCPv6           5
#define VIRTIO_NET_HASH_REPORT_UDPv6           6
#define VIRTIO_NET_HASH_REPORT_IPv6_EX         7
#define VIRTIO_NET_HASH_REPORT_TCPv6_EX        8
#define VIRTIO_NET_HASH_REPORT_UDPv6_EX        9
\end{lstlisting}

\subsubsection{Control Virtqueue}\label{sec:Device Types / Network Device / Device Operation / Control Virtqueue}

The driver uses the control virtqueue (if VIRTIO_NET_F_CTRL_VQ is
negotiated) to send commands to manipulate various features of
the device which would not easily map into the configuration
space.

All commands are of the following form:

\begin{lstlisting}
struct virtio_net_ctrl {
        u8 class;
        u8 command;
        u8 command-specific-data[];
        u8 ack;
        u8 command-specific-result[];
};

/* ack values */
#define VIRTIO_NET_OK     0
#define VIRTIO_NET_ERR    1
\end{lstlisting}

The \field{class}, \field{command} and command-specific-data are set by the
driver, and the device sets the \field{ack} byte and optionally
\field{command-specific-result}. There is little the driver can
do except issue a diagnostic if \field{ack} is not VIRTIO_NET_OK.

The command VIRTIO_NET_CTRL_STATS_QUERY and VIRTIO_NET_CTRL_STATS_GET contain
\field{command-specific-result}.

\paragraph{Packet Receive Filtering}\label{sec:Device Types / Network Device / Device Operation / Control Virtqueue / Packet Receive Filtering}
\label{sec:Device Types / Network Device / Device Operation / Control Virtqueue / Setting Promiscuous Mode}%old label for latexdiff

If the VIRTIO_NET_F_CTRL_RX and VIRTIO_NET_F_CTRL_RX_EXTRA
features are negotiated, the driver can send control commands for
promiscuous mode, multicast, unicast and broadcast receiving.

\begin{note}
In general, these commands are best-effort: unwanted
packets could still arrive.
\end{note}

\begin{lstlisting}
#define VIRTIO_NET_CTRL_RX    0
 #define VIRTIO_NET_CTRL_RX_PROMISC      0
 #define VIRTIO_NET_CTRL_RX_ALLMULTI     1
 #define VIRTIO_NET_CTRL_RX_ALLUNI       2
 #define VIRTIO_NET_CTRL_RX_NOMULTI      3
 #define VIRTIO_NET_CTRL_RX_NOUNI        4
 #define VIRTIO_NET_CTRL_RX_NOBCAST      5
\end{lstlisting}


\devicenormative{\subparagraph}{Packet Receive Filtering}{Device Types / Network Device / Device Operation / Control Virtqueue / Packet Receive Filtering}

If the VIRTIO_NET_F_CTRL_RX feature has been negotiated,
the device MUST support the following VIRTIO_NET_CTRL_RX class
commands:
\begin{itemize}
\item VIRTIO_NET_CTRL_RX_PROMISC turns promiscuous mode on and
off. The command-specific-data is one byte containing 0 (off) or
1 (on). If promiscuous mode is on, the device SHOULD receive all
incoming packets.
This SHOULD take effect even if one of the other modes set by
a VIRTIO_NET_CTRL_RX class command is on.
\item VIRTIO_NET_CTRL_RX_ALLMULTI turns all-multicast receive on and
off. The command-specific-data is one byte containing 0 (off) or
1 (on). When all-multicast receive is on the device SHOULD allow
all incoming multicast packets.
\end{itemize}

If the VIRTIO_NET_F_CTRL_RX_EXTRA feature has been negotiated,
the device MUST support the following VIRTIO_NET_CTRL_RX class
commands:
\begin{itemize}
\item VIRTIO_NET_CTRL_RX_ALLUNI turns all-unicast receive on and
off. The command-specific-data is one byte containing 0 (off) or
1 (on). When all-unicast receive is on the device SHOULD allow
all incoming unicast packets.
\item VIRTIO_NET_CTRL_RX_NOMULTI suppresses multicast receive.
The command-specific-data is one byte containing 0 (multicast
receive allowed) or 1 (multicast receive suppressed).
When multicast receive is suppressed, the device SHOULD NOT
send multicast packets to the driver.
This SHOULD take effect even if VIRTIO_NET_CTRL_RX_ALLMULTI is on.
This filter SHOULD NOT apply to broadcast packets.
\item VIRTIO_NET_CTRL_RX_NOUNI suppresses unicast receive.
The command-specific-data is one byte containing 0 (unicast
receive allowed) or 1 (unicast receive suppressed).
When unicast receive is suppressed, the device SHOULD NOT
send unicast packets to the driver.
This SHOULD take effect even if VIRTIO_NET_CTRL_RX_ALLUNI is on.
\item VIRTIO_NET_CTRL_RX_NOBCAST suppresses broadcast receive.
The command-specific-data is one byte containing 0 (broadcast
receive allowed) or 1 (broadcast receive suppressed).
When broadcast receive is suppressed, the device SHOULD NOT
send broadcast packets to the driver.
This SHOULD take effect even if VIRTIO_NET_CTRL_RX_ALLMULTI is on.
\end{itemize}

\drivernormative{\subparagraph}{Packet Receive Filtering}{Device Types / Network Device / Device Operation / Control Virtqueue / Packet Receive Filtering}

If the VIRTIO_NET_F_CTRL_RX feature has not been negotiated,
the driver MUST NOT issue commands VIRTIO_NET_CTRL_RX_PROMISC or
VIRTIO_NET_CTRL_RX_ALLMULTI.

If the VIRTIO_NET_F_CTRL_RX_EXTRA feature has not been negotiated,
the driver MUST NOT issue commands
 VIRTIO_NET_CTRL_RX_ALLUNI,
 VIRTIO_NET_CTRL_RX_NOMULTI,
 VIRTIO_NET_CTRL_RX_NOUNI or
 VIRTIO_NET_CTRL_RX_NOBCAST.

\paragraph{Setting MAC Address Filtering}\label{sec:Device Types / Network Device / Device Operation / Control Virtqueue / Setting MAC Address Filtering}

If the VIRTIO_NET_F_CTRL_RX feature is negotiated, the driver can
send control commands for MAC address filtering.

\begin{lstlisting}
struct virtio_net_ctrl_mac {
        le32 entries;
        u8 macs[entries][6];
};

#define VIRTIO_NET_CTRL_MAC    1
 #define VIRTIO_NET_CTRL_MAC_TABLE_SET        0
 #define VIRTIO_NET_CTRL_MAC_ADDR_SET         1
\end{lstlisting}

The device can filter incoming packets by any number of destination
MAC addresses\footnote{Since there are no guarantees, it can use a hash filter or
silently switch to allmulti or promiscuous mode if it is given too
many addresses.
}. This table is set using the class
VIRTIO_NET_CTRL_MAC and the command VIRTIO_NET_CTRL_MAC_TABLE_SET. The
command-specific-data is two variable length tables of 6-byte MAC
addresses (as described in struct virtio_net_ctrl_mac). The first table contains unicast addresses, and the second
contains multicast addresses.

The VIRTIO_NET_CTRL_MAC_ADDR_SET command is used to set the
default MAC address which rx filtering
accepts (and if VIRTIO_NET_F_MAC has been negotiated,
this will be reflected in \field{mac} in config space).

The command-specific-data for VIRTIO_NET_CTRL_MAC_ADDR_SET is
the 6-byte MAC address.

\devicenormative{\subparagraph}{Setting MAC Address Filtering}{Device Types / Network Device / Device Operation / Control Virtqueue / Setting MAC Address Filtering}

The device MUST have an empty MAC filtering table on reset.

The device MUST update the MAC filtering table before it consumes
the VIRTIO_NET_CTRL_MAC_TABLE_SET command.

The device MUST update \field{mac} in config space before it consumes
the VIRTIO_NET_CTRL_MAC_ADDR_SET command, if VIRTIO_NET_F_MAC has
been negotiated.

The device SHOULD drop incoming packets which have a destination MAC which
matches neither the \field{mac} (or that set with VIRTIO_NET_CTRL_MAC_ADDR_SET)
nor the MAC filtering table.

\drivernormative{\subparagraph}{Setting MAC Address Filtering}{Device Types / Network Device / Device Operation / Control Virtqueue / Setting MAC Address Filtering}

If VIRTIO_NET_F_CTRL_RX has not been negotiated,
the driver MUST NOT issue VIRTIO_NET_CTRL_MAC class commands.

If VIRTIO_NET_F_CTRL_RX has been negotiated,
the driver SHOULD issue VIRTIO_NET_CTRL_MAC_ADDR_SET
to set the default mac if it is different from \field{mac}.

The driver MUST follow the VIRTIO_NET_CTRL_MAC_TABLE_SET command
by a le32 number, followed by that number of non-multicast
MAC addresses, followed by another le32 number, followed by
that number of multicast addresses.  Either number MAY be 0.

\subparagraph{Legacy Interface: Setting MAC Address Filtering}\label{sec:Device Types / Network Device / Device Operation / Control Virtqueue / Setting MAC Address Filtering / Legacy Interface: Setting MAC Address Filtering}
When using the legacy interface, transitional devices and drivers
MUST format \field{entries} in struct virtio_net_ctrl_mac
according to the native endian of the guest rather than
(necessarily when not using the legacy interface) little-endian.

Legacy drivers that didn't negotiate VIRTIO_NET_F_CTRL_MAC_ADDR
changed \field{mac} in config space when NIC is accepting
incoming packets. These drivers always wrote the mac value from
first to last byte, therefore after detecting such drivers,
a transitional device MAY defer MAC update, or MAY defer
processing incoming packets until driver writes the last byte
of \field{mac} in the config space.

\paragraph{VLAN Filtering}\label{sec:Device Types / Network Device / Device Operation / Control Virtqueue / VLAN Filtering}

If the driver negotiates the VIRTIO_NET_F_CTRL_VLAN feature, it
can control a VLAN filter table in the device. The VLAN filter
table applies only to VLAN tagged packets.

When VIRTIO_NET_F_CTRL_VLAN is negotiated, the device starts with
an empty VLAN filter table.

\begin{note}
Similar to the MAC address based filtering, the VLAN filtering
is also best-effort: unwanted packets could still arrive.
\end{note}

\begin{lstlisting}
#define VIRTIO_NET_CTRL_VLAN       2
 #define VIRTIO_NET_CTRL_VLAN_ADD             0
 #define VIRTIO_NET_CTRL_VLAN_DEL             1
\end{lstlisting}

Both the VIRTIO_NET_CTRL_VLAN_ADD and VIRTIO_NET_CTRL_VLAN_DEL
command take a little-endian 16-bit VLAN id as the command-specific-data.

VIRTIO_NET_CTRL_VLAN_ADD command adds the specified VLAN to the
VLAN filter table.

VIRTIO_NET_CTRL_VLAN_DEL command removes the specified VLAN from
the VLAN filter table.

\devicenormative{\subparagraph}{VLAN Filtering}{Device Types / Network Device / Device Operation / Control Virtqueue / VLAN Filtering}

When VIRTIO_NET_F_CTRL_VLAN is not negotiated, the device MUST
accept all VLAN tagged packets.

When VIRTIO_NET_F_CTRL_VLAN is negotiated, the device MUST
accept all VLAN tagged packets whose VLAN tag is present in
the VLAN filter table and SHOULD drop all VLAN tagged packets
whose VLAN tag is absent in the VLAN filter table.

\subparagraph{Legacy Interface: VLAN Filtering}\label{sec:Device Types / Network Device / Device Operation / Control Virtqueue / VLAN Filtering / Legacy Interface: VLAN Filtering}
When using the legacy interface, transitional devices and drivers
MUST format the VLAN id
according to the native endian of the guest rather than
(necessarily when not using the legacy interface) little-endian.

\paragraph{Gratuitous Packet Sending}\label{sec:Device Types / Network Device / Device Operation / Control Virtqueue / Gratuitous Packet Sending}

If the driver negotiates the VIRTIO_NET_F_GUEST_ANNOUNCE (depends
on VIRTIO_NET_F_CTRL_VQ), the device can ask the driver to send gratuitous
packets; this is usually done after the guest has been physically
migrated, and needs to announce its presence on the new network
links. (As hypervisor does not have the knowledge of guest
network configuration (eg. tagged vlan) it is simplest to prod
the guest in this way).

\begin{lstlisting}
#define VIRTIO_NET_CTRL_ANNOUNCE       3
 #define VIRTIO_NET_CTRL_ANNOUNCE_ACK             0
\end{lstlisting}

The driver checks VIRTIO_NET_S_ANNOUNCE bit in the device configuration \field{status} field
when it notices the changes of device configuration. The
command VIRTIO_NET_CTRL_ANNOUNCE_ACK is used to indicate that
driver has received the notification and device clears the
VIRTIO_NET_S_ANNOUNCE bit in \field{status}.

Processing this notification involves:

\begin{enumerate}
\item Sending the gratuitous packets (eg. ARP) or marking there are pending
  gratuitous packets to be sent and letting deferred routine to
  send them.

\item Sending VIRTIO_NET_CTRL_ANNOUNCE_ACK command through control
  vq.
\end{enumerate}

\drivernormative{\subparagraph}{Gratuitous Packet Sending}{Device Types / Network Device / Device Operation / Control Virtqueue / Gratuitous Packet Sending}

If the driver negotiates VIRTIO_NET_F_GUEST_ANNOUNCE, it SHOULD notify
network peers of its new location after it sees the VIRTIO_NET_S_ANNOUNCE bit
in \field{status}.  The driver MUST send a command on the command queue
with class VIRTIO_NET_CTRL_ANNOUNCE and command VIRTIO_NET_CTRL_ANNOUNCE_ACK.

\devicenormative{\subparagraph}{Gratuitous Packet Sending}{Device Types / Network Device / Device Operation / Control Virtqueue / Gratuitous Packet Sending}

If VIRTIO_NET_F_GUEST_ANNOUNCE is negotiated, the device MUST clear the
VIRTIO_NET_S_ANNOUNCE bit in \field{status} upon receipt of a command buffer
with class VIRTIO_NET_CTRL_ANNOUNCE and command VIRTIO_NET_CTRL_ANNOUNCE_ACK
before marking the buffer as used.

\paragraph{Device operation in multiqueue mode}\label{sec:Device Types / Network Device / Device Operation / Control Virtqueue / Device operation in multiqueue mode}

This specification defines the following modes that a device MAY implement for operation with multiple transmit/receive virtqueues:
\begin{itemize}
\item Automatic receive steering as defined in \ref{sec:Device Types / Network Device / Device Operation / Control Virtqueue / Automatic receive steering in multiqueue mode}.
 If a device supports this mode, it offers the VIRTIO_NET_F_MQ feature bit.
\item Receive-side scaling as defined in \ref{devicenormative:Device Types / Network Device / Device Operation / Control Virtqueue / Receive-side scaling (RSS) / RSS processing}.
 If a device supports this mode, it offers the VIRTIO_NET_F_RSS feature bit.
\end{itemize}

A device MAY support one of these features or both. The driver MAY negotiate any set of these features that the device supports.

Multiqueue is disabled by default.

The driver enables multiqueue by sending a command using \field{class} VIRTIO_NET_CTRL_MQ. The \field{command} selects the mode of multiqueue operation, as follows:
\begin{lstlisting}
#define VIRTIO_NET_CTRL_MQ    4
 #define VIRTIO_NET_CTRL_MQ_VQ_PAIRS_SET        0 (for automatic receive steering)
 #define VIRTIO_NET_CTRL_MQ_RSS_CONFIG          1 (for configurable receive steering)
 #define VIRTIO_NET_CTRL_MQ_HASH_CONFIG         2 (for configurable hash calculation)
\end{lstlisting}

If more than one multiqueue mode is negotiated, the resulting device configuration is defined by the last command sent by the driver.

\paragraph{Automatic receive steering in multiqueue mode}\label{sec:Device Types / Network Device / Device Operation / Control Virtqueue / Automatic receive steering in multiqueue mode}

If the driver negotiates the VIRTIO_NET_F_MQ feature bit (depends on VIRTIO_NET_F_CTRL_VQ), it MAY transmit outgoing packets on one
of the multiple transmitq1\ldots transmitqN and ask the device to
queue incoming packets into one of the multiple receiveq1\ldots receiveqN
depending on the packet flow.

The driver enables multiqueue by
sending the VIRTIO_NET_CTRL_MQ_VQ_PAIRS_SET command, specifying
the number of the transmit and receive queues to be used up to
\field{max_virtqueue_pairs}; subsequently,
transmitq1\ldots transmitqn and receiveq1\ldots receiveqn where
n=\field{virtqueue_pairs} MAY be used.
\begin{lstlisting}
struct virtio_net_ctrl_mq_pairs_set {
       le16 virtqueue_pairs;
};
#define VIRTIO_NET_CTRL_MQ_VQ_PAIRS_MIN        1
#define VIRTIO_NET_CTRL_MQ_VQ_PAIRS_MAX        0x8000

\end{lstlisting}

When multiqueue is enabled by VIRTIO_NET_CTRL_MQ_VQ_PAIRS_SET command, the device MUST use automatic receive steering
based on packet flow. Programming of the receive steering
classificator is implicit. After the driver transmitted a packet of a
flow on transmitqX, the device SHOULD cause incoming packets for that flow to
be steered to receiveqX. For uni-directional protocols, or where
no packets have been transmitted yet, the device MAY steer a packet
to a random queue out of the specified receiveq1\ldots receiveqn.

Multiqueue is disabled by VIRTIO_NET_CTRL_MQ_VQ_PAIRS_SET with \field{virtqueue_pairs} to 1 (this is
the default) and waiting for the device to use the command buffer.

\drivernormative{\subparagraph}{Automatic receive steering in multiqueue mode}{Device Types / Network Device / Device Operation / Control Virtqueue / Automatic receive steering in multiqueue mode}

The driver MUST configure the virtqueues before enabling them with the
VIRTIO_NET_CTRL_MQ_VQ_PAIRS_SET command.

The driver MUST NOT request a \field{virtqueue_pairs} of 0 or
greater than \field{max_virtqueue_pairs} in the device configuration space.

The driver MUST queue packets only on any transmitq1 before the
VIRTIO_NET_CTRL_MQ_VQ_PAIRS_SET command.

The driver MUST NOT queue packets on transmit queues greater than
\field{virtqueue_pairs} once it has placed the VIRTIO_NET_CTRL_MQ_VQ_PAIRS_SET command in the available ring.

\devicenormative{\subparagraph}{Automatic receive steering in multiqueue mode}{Device Types / Network Device / Device Operation / Control Virtqueue / Automatic receive steering in multiqueue mode}

After initialization of reset, the device MUST queue packets only on receiveq1.

The device MUST NOT queue packets on receive queues greater than
\field{virtqueue_pairs} once it has placed the
VIRTIO_NET_CTRL_MQ_VQ_PAIRS_SET command in a used buffer.

If the destination receive queue is being reset (See \ref{sec:Basic Facilities of a Virtio Device / Virtqueues / Virtqueue Reset}),
the device SHOULD re-select another random queue. If all receive queues are
being reset, the device MUST drop the packet.

\subparagraph{Legacy Interface: Automatic receive steering in multiqueue mode}\label{sec:Device Types / Network Device / Device Operation / Control Virtqueue / Automatic receive steering in multiqueue mode / Legacy Interface: Automatic receive steering in multiqueue mode}
When using the legacy interface, transitional devices and drivers
MUST format \field{virtqueue_pairs}
according to the native endian of the guest rather than
(necessarily when not using the legacy interface) little-endian.

\subparagraph{Hash calculation}\label{sec:Device Types / Network Device / Device Operation / Control Virtqueue / Automatic receive steering in multiqueue mode / Hash calculation}
If VIRTIO_NET_F_HASH_REPORT was negotiated and the device uses automatic receive steering,
the device MUST support a command to configure hash calculation parameters.

The driver provides parameters for hash calculation as follows:

\field{class} VIRTIO_NET_CTRL_MQ, \field{command} VIRTIO_NET_CTRL_MQ_HASH_CONFIG.

The \field{command-specific-data} has following format:
\begin{lstlisting}
struct virtio_net_hash_config {
    le32 hash_types;
    le16 reserved[4];
    u8 hash_key_length;
    u8 hash_key_data[hash_key_length];
};
\end{lstlisting}
Field \field{hash_types} contains a bitmask of allowed hash types as
defined in
\ref{sec:Device Types / Network Device / Device Operation / Processing of Incoming Packets / Hash calculation for incoming packets / Supported/enabled hash types}.
Initially the device has all hash types disabled and reports only VIRTIO_NET_HASH_REPORT_NONE.

Field \field{reserved} MUST contain zeroes. It is defined to make the structure to match the layout of virtio_net_rss_config structure,
defined in \ref{sec:Device Types / Network Device / Device Operation / Control Virtqueue / Receive-side scaling (RSS)}.

Fields \field{hash_key_length} and \field{hash_key_data} define the key to be used in hash calculation.

\paragraph{Receive-side scaling (RSS)}\label{sec:Device Types / Network Device / Device Operation / Control Virtqueue / Receive-side scaling (RSS)}
A device offers the feature VIRTIO_NET_F_RSS if it supports RSS receive steering with Toeplitz hash calculation and configurable parameters.

A driver queries RSS capabilities of the device by reading device configuration as defined in \ref{sec:Device Types / Network Device / Device configuration layout}

\subparagraph{Setting RSS parameters}\label{sec:Device Types / Network Device / Device Operation / Control Virtqueue / Receive-side scaling (RSS) / Setting RSS parameters}

Driver sends a VIRTIO_NET_CTRL_MQ_RSS_CONFIG command using the following format for \field{command-specific-data}:
\begin{lstlisting}
struct rss_rq_id {
   le16 vq_index_1_16: 15; /* Bits 1 to 16 of the virtqueue index */
   le16 reserved: 1; /* Set to zero */
};

struct virtio_net_rss_config {
    le32 hash_types;
    le16 indirection_table_mask;
    struct rss_rq_id unclassified_queue;
    struct rss_rq_id indirection_table[indirection_table_length];
    le16 max_tx_vq;
    u8 hash_key_length;
    u8 hash_key_data[hash_key_length];
};
\end{lstlisting}
Field \field{hash_types} contains a bitmask of allowed hash types as
defined in
\ref{sec:Device Types / Network Device / Device Operation / Processing of Incoming Packets / Hash calculation for incoming packets / Supported/enabled hash types}.

Field \field{indirection_table_mask} is a mask to be applied to
the calculated hash to produce an index in the
\field{indirection_table} array.
Number of entries in \field{indirection_table} is (\field{indirection_table_mask} + 1).

\field{rss_rq_id} is a receive virtqueue id. \field{vq_index_1_16}
consists of bits 1 to 16 of a virtqueue index. For example, a
\field{vq_index_1_16} value of 3 corresponds to virtqueue index 6,
which maps to receiveq4.

Field \field{unclassified_queue} specifies the receive virtqueue id in which to
place unclassified packets.

Field \field{indirection_table} is an array of receive virtqueues ids.

A driver sets \field{max_tx_vq} to inform a device how many transmit virtqueues it may use (transmitq1\ldots transmitq \field{max_tx_vq}).

Fields \field{hash_key_length} and \field{hash_key_data} define the key to be used in hash calculation.

\drivernormative{\subparagraph}{Setting RSS parameters}{Device Types / Network Device / Device Operation / Control Virtqueue / Receive-side scaling (RSS) }

A driver MUST NOT send the VIRTIO_NET_CTRL_MQ_RSS_CONFIG command if the feature VIRTIO_NET_F_RSS has not been negotiated.

A driver MUST fill the \field{indirection_table} array only with
enabled receive virtqueues ids.

The number of entries in \field{indirection_table} (\field{indirection_table_mask} + 1) MUST be a power of two.

A driver MUST use \field{indirection_table_mask} values that are less than \field{rss_max_indirection_table_length} reported by a device.

A driver MUST NOT set any VIRTIO_NET_HASH_TYPE_ flags that are not supported by a device.

\devicenormative{\subparagraph}{RSS processing}{Device Types / Network Device / Device Operation / Control Virtqueue / Receive-side scaling (RSS) / RSS processing}
The device MUST determine the destination queue for a network packet as follows:
\begin{itemize}
\item Calculate the hash of the packet as defined in \ref{sec:Device Types / Network Device / Device Operation / Processing of Incoming Packets / Hash calculation for incoming packets}.
\item If the device did not calculate the hash for the specific packet, the device directs the packet to the receiveq specified by \field{unclassified_queue} of virtio_net_rss_config structure.
\item Apply \field{indirection_table_mask} to the calculated hash
and use the result as the index in the indirection table to get
the destination receive virtqueue id.
\item If the destination receive queue is being reset (See \ref{sec:Basic Facilities of a Virtio Device / Virtqueues / Virtqueue Reset}), the device MUST drop the packet.
\end{itemize}

\paragraph{RSS Context}\label{sec:Device Types / Network Device / Device Operation / Control Virtqueue / RSS Context}

An RSS context consists of configurable parameters specified by \ref{sec:Device Types / Network Device
/ Device Operation / Control Virtqueue / Receive-side scaling (RSS)}.

The RSS configuration supported by VIRTIO_NET_F_RSS is considered the default RSS configuration.

The device offers the feature VIRTIO_NET_F_RSS_CONTEXT if it supports one or multiple RSS contexts
(excluding the default RSS configuration) and configurable parameters.

\subparagraph{Querying RSS Context Capability}\label{sec:Device Types / Network Device / Device Operation / Control Virtqueue / RSS Context / Querying RSS Context Capability}

\begin{lstlisting}
#define VIRTNET_RSS_CTX_CTRL 9
 #define VIRTNET_RSS_CTX_CTRL_CAP_GET  0
 #define VIRTNET_RSS_CTX_CTRL_ADD      1
 #define VIRTNET_RSS_CTX_CTRL_MOD      2
 #define VIRTNET_RSS_CTX_CTRL_DEL      3

struct virtnet_rss_ctx_cap {
    le16 max_rss_contexts;
}
\end{lstlisting}

Field \field{max_rss_contexts} specifies the maximum number of RSS contexts \ref{sec:Device Types / Network Device /
Device Operation / Control Virtqueue / RSS Context} supported by the device.

The driver queries the RSS context capability of the device by sending the command VIRTNET_RSS_CTX_CTRL_CAP_GET
with the structure virtnet_rss_ctx_cap.

For the command VIRTNET_RSS_CTX_CTRL_CAP_GET, the structure virtnet_rss_ctx_cap is write-only for the device.

\subparagraph{Setting RSS Context Parameters}\label{sec:Device Types / Network Device / Device Operation / Control Virtqueue / RSS Context / Setting RSS Context Parameters}

\begin{lstlisting}
struct virtnet_rss_ctx_add_modify {
    le16 rss_ctx_id;
    u8 reserved[6];
    struct virtio_net_rss_config rss;
};

struct virtnet_rss_ctx_del {
    le16 rss_ctx_id;
};
\end{lstlisting}

RSS context parameters:
\begin{itemize}
\item  \field{rss_ctx_id}: ID of the specific RSS context.
\item  \field{rss}: RSS context parameters of the specific RSS context whose id is \field{rss_ctx_id}.
\end{itemize}

\field{reserved} is reserved and it is ignored by the device.

If the feature VIRTIO_NET_F_RSS_CONTEXT has been negotiated, the driver can send the following
VIRTNET_RSS_CTX_CTRL class commands:
\begin{enumerate}
\item VIRTNET_RSS_CTX_CTRL_ADD: use the structure virtnet_rss_ctx_add_modify to
       add an RSS context configured as \field{rss} and id as \field{rss_ctx_id} for the device.
\item VIRTNET_RSS_CTX_CTRL_MOD: use the structure virtnet_rss_ctx_add_modify to
       configure parameters of the RSS context whose id is \field{rss_ctx_id} as \field{rss} for the device.
\item VIRTNET_RSS_CTX_CTRL_DEL: use the structure virtnet_rss_ctx_del to delete
       the RSS context whose id is \field{rss_ctx_id} for the device.
\end{enumerate}

For commands VIRTNET_RSS_CTX_CTRL_ADD and VIRTNET_RSS_CTX_CTRL_MOD, the structure virtnet_rss_ctx_add_modify is read-only for the device.
For the command VIRTNET_RSS_CTX_CTRL_DEL, the structure virtnet_rss_ctx_del is read-only for the device.

\devicenormative{\subparagraph}{RSS Context}{Device Types / Network Device / Device Operation / Control Virtqueue / RSS Context}

The device MUST set \field{max_rss_contexts} to at least 1 if it offers VIRTIO_NET_F_RSS_CONTEXT.

Upon reset, the device MUST clear all previously configured RSS contexts.

\drivernormative{\subparagraph}{RSS Context}{Device Types / Network Device / Device Operation / Control Virtqueue / RSS Context}

The driver MUST have negotiated the VIRTIO_NET_F_RSS_CONTEXT feature when issuing the VIRTNET_RSS_CTX_CTRL class commands.

The driver MUST set \field{rss_ctx_id} to between 1 and \field{max_rss_contexts} inclusive.

The driver MUST NOT send the command VIRTIO_NET_CTRL_MQ_VQ_PAIRS_SET when the device has successfully configured at least one RSS context.

\paragraph{Offloads State Configuration}\label{sec:Device Types / Network Device / Device Operation / Control Virtqueue / Offloads State Configuration}

If the VIRTIO_NET_F_CTRL_GUEST_OFFLOADS feature is negotiated, the driver can
send control commands for dynamic offloads state configuration.

\subparagraph{Setting Offloads State}\label{sec:Device Types / Network Device / Device Operation / Control Virtqueue / Offloads State Configuration / Setting Offloads State}

To configure the offloads, the following layout structure and
definitions are used:

\begin{lstlisting}
le64 offloads;

#define VIRTIO_NET_F_GUEST_CSUM       1
#define VIRTIO_NET_F_GUEST_TSO4       7
#define VIRTIO_NET_F_GUEST_TSO6       8
#define VIRTIO_NET_F_GUEST_ECN        9
#define VIRTIO_NET_F_GUEST_UFO        10
#define VIRTIO_NET_F_GUEST_UDP_TUNNEL_GSO  46
#define VIRTIO_NET_F_GUEST_UDP_TUNNEL_GSO_CSUM 47
#define VIRTIO_NET_F_GUEST_USO4       54
#define VIRTIO_NET_F_GUEST_USO6       55

#define VIRTIO_NET_CTRL_GUEST_OFFLOADS       5
 #define VIRTIO_NET_CTRL_GUEST_OFFLOADS_SET   0
\end{lstlisting}

The class VIRTIO_NET_CTRL_GUEST_OFFLOADS has one command:
VIRTIO_NET_CTRL_GUEST_OFFLOADS_SET applies the new offloads configuration.

le64 value passed as command data is a bitmask, bits set define
offloads to be enabled, bits cleared - offloads to be disabled.

There is a corresponding device feature for each offload. Upon feature
negotiation corresponding offload gets enabled to preserve backward
compatibility.

\drivernormative{\subparagraph}{Setting Offloads State}{Device Types / Network Device / Device Operation / Control Virtqueue / Offloads State Configuration / Setting Offloads State}

A driver MUST NOT enable an offload for which the appropriate feature
has not been negotiated.

\subparagraph{Legacy Interface: Setting Offloads State}\label{sec:Device Types / Network Device / Device Operation / Control Virtqueue / Offloads State Configuration / Setting Offloads State / Legacy Interface: Setting Offloads State}
When using the legacy interface, transitional devices and drivers
MUST format \field{offloads}
according to the native endian of the guest rather than
(necessarily when not using the legacy interface) little-endian.


\paragraph{Notifications Coalescing}\label{sec:Device Types / Network Device / Device Operation / Control Virtqueue / Notifications Coalescing}

If the VIRTIO_NET_F_NOTF_COAL feature is negotiated, the driver can
send commands VIRTIO_NET_CTRL_NOTF_COAL_TX_SET and VIRTIO_NET_CTRL_NOTF_COAL_RX_SET
for notification coalescing.

If the VIRTIO_NET_F_VQ_NOTF_COAL feature is negotiated, the driver can
send commands VIRTIO_NET_CTRL_NOTF_COAL_VQ_SET and VIRTIO_NET_CTRL_NOTF_COAL_VQ_GET
for virtqueue notification coalescing.

\begin{lstlisting}
struct virtio_net_ctrl_coal {
    le32 max_packets;
    le32 max_usecs;
};

struct virtio_net_ctrl_coal_vq {
    le16 vq_index;
    le16 reserved;
    struct virtio_net_ctrl_coal coal;
};

#define VIRTIO_NET_CTRL_NOTF_COAL 6
 #define VIRTIO_NET_CTRL_NOTF_COAL_TX_SET  0
 #define VIRTIO_NET_CTRL_NOTF_COAL_RX_SET 1
 #define VIRTIO_NET_CTRL_NOTF_COAL_VQ_SET 2
 #define VIRTIO_NET_CTRL_NOTF_COAL_VQ_GET 3
\end{lstlisting}

Coalescing parameters:
\begin{itemize}
\item \field{vq_index}: The virtqueue index of an enabled transmit or receive virtqueue.
\item \field{max_usecs} for RX: Maximum number of microseconds to delay a RX notification.
\item \field{max_usecs} for TX: Maximum number of microseconds to delay a TX notification.
\item \field{max_packets} for RX: Maximum number of packets to receive before a RX notification.
\item \field{max_packets} for TX: Maximum number of packets to send before a TX notification.
\end{itemize}

\field{reserved} is reserved and it is ignored by the device.

Read/Write attributes for coalescing parameters:
\begin{itemize}
\item For commands VIRTIO_NET_CTRL_NOTF_COAL_TX_SET and VIRTIO_NET_CTRL_NOTF_COAL_RX_SET, the structure virtio_net_ctrl_coal is write-only for the driver.
\item For the command VIRTIO_NET_CTRL_NOTF_COAL_VQ_SET, the structure virtio_net_ctrl_coal_vq is write-only for the driver.
\item For the command VIRTIO_NET_CTRL_NOTF_COAL_VQ_GET, \field{vq_index} and \field{reserved} are write-only
      for the driver, and the structure virtio_net_ctrl_coal is read-only for the driver.
\end{itemize}

The class VIRTIO_NET_CTRL_NOTF_COAL has the following commands:
\begin{enumerate}
\item VIRTIO_NET_CTRL_NOTF_COAL_TX_SET: use the structure virtio_net_ctrl_coal to set the \field{max_usecs} and \field{max_packets} parameters for all transmit virtqueues.
\item VIRTIO_NET_CTRL_NOTF_COAL_RX_SET: use the structure virtio_net_ctrl_coal to set the \field{max_usecs} and \field{max_packets} parameters for all receive virtqueues.
\item VIRTIO_NET_CTRL_NOTF_COAL_VQ_SET: use the structure virtio_net_ctrl_coal_vq to set the \field{max_usecs} and \field{max_packets} parameters
                                        for an enabled transmit/receive virtqueue whose index is \field{vq_index}.
\item VIRTIO_NET_CTRL_NOTF_COAL_VQ_GET: use the structure virtio_net_ctrl_coal_vq to get the \field{max_usecs} and \field{max_packets} parameters
                                        for an enabled transmit/receive virtqueue whose index is \field{vq_index}.
\end{enumerate}

The device may generate notifications more or less frequently than specified by set commands of the VIRTIO_NET_CTRL_NOTF_COAL class.

If coalescing parameters are being set, the device applies the last coalescing parameters set for a
virtqueue, regardless of the command used to set the parameters. Use the following command sequence
with two pairs of virtqueues as an example:
Each of the following commands sets \field{max_usecs} and \field{max_packets} parameters for virtqueues.
\begin{itemize}
\item Command1: VIRTIO_NET_CTRL_NOTF_COAL_RX_SET sets coalescing parameters for virtqueues having index 0 and index 2. Virtqueues having index 1 and index 3 retain their previous parameters.
\item Command2: VIRTIO_NET_CTRL_NOTF_COAL_VQ_SET with \field{vq_index} = 0 sets coalescing parameters for virtqueue having index 0. Virtqueue having index 2 retains the parameters from command1.
\item Command3: VIRTIO_NET_CTRL_NOTF_COAL_VQ_GET with \field{vq_index} = 0, the device responds with coalescing parameters of vq_index 0 set by command2.
\item Command4: VIRTIO_NET_CTRL_NOTF_COAL_VQ_SET with \field{vq_index} = 1 sets coalescing parameters for virtqueue having index 1. Virtqueue having index 3 retains its previous parameters.
\item Command5: VIRTIO_NET_CTRL_NOTF_COAL_TX_SET sets coalescing parameters for virtqueues having index 1 and index 3, and overrides the parameters set by command4.
\item Command6: VIRTIO_NET_CTRL_NOTF_COAL_VQ_GET with \field{vq_index} = 1, the device responds with coalescing parameters of index 1 set by command5.
\end{itemize}

\subparagraph{Operation}\label{sec:Device Types / Network Device / Device Operation / Control Virtqueue / Notifications Coalescing / Operation}

The device sends a used buffer notification once the notification conditions are met and if the notifications are not suppressed as explained in \ref{sec:Basic Facilities of a Virtio Device / Virtqueues / Used Buffer Notification Suppression}.

When the device has non-zero \field{max_usecs} and non-zero \field{max_packets}, it starts counting microseconds and packets upon receiving/sending a packet.
The device counts packets and microseconds for each receive virtqueue and transmit virtqueue separately.
In this case, the notification conditions are met when \field{max_usecs} microseconds elapse, or upon sending/receiving \field{max_packets} packets, whichever happens first.
Afterwards, the device waits for the next packet and starts counting packets and microseconds again.

When the device has \field{max_usecs} = 0 or \field{max_packets} = 0, the notification conditions are met after every packet received/sent.

\subparagraph{RX Example}\label{sec:Device Types / Network Device / Device Operation / Control Virtqueue / Notifications Coalescing / RX Example}

If, for example:
\begin{itemize}
\item \field{max_usecs} = 10.
\item \field{max_packets} = 15.
\end{itemize}
then each receive virtqueue of a device will operate as follows:
\begin{itemize}
\item The device will count packets received on each virtqueue until it accumulates 15, or until 10 microseconds elapsed since the first one was received.
\item If the notifications are not suppressed by the driver, the device will send an used buffer notification, otherwise, the device will not send an used buffer notification as long as the notifications are suppressed.
\end{itemize}

\subparagraph{TX Example}\label{sec:Device Types / Network Device / Device Operation / Control Virtqueue / Notifications Coalescing / TX Example}

If, for example:
\begin{itemize}
\item \field{max_usecs} = 10.
\item \field{max_packets} = 15.
\end{itemize}
then each transmit virtqueue of a device will operate as follows:
\begin{itemize}
\item The device will count packets sent on each virtqueue until it accumulates 15, or until 10 microseconds elapsed since the first one was sent.
\item If the notifications are not suppressed by the driver, the device will send an used buffer notification, otherwise, the device will not send an used buffer notification as long as the notifications are suppressed.
\end{itemize}

\subparagraph{Notifications When Coalescing Parameters Change}\label{sec:Device Types / Network Device / Device Operation / Control Virtqueue / Notifications Coalescing / Notifications When Coalescing Parameters Change}

When the coalescing parameters of a device change, the device needs to check if the new notification conditions are met and send a used buffer notification if so.

For example, \field{max_packets} = 15 for a device with a single transmit virtqueue: if the device sends 10 packets and afterwards receives a
VIRTIO_NET_CTRL_NOTF_COAL_TX_SET command with \field{max_packets} = 8, then the notification condition is immediately considered to be met;
the device needs to immediately send a used buffer notification, if the notifications are not suppressed by the driver.

\drivernormative{\subparagraph}{Notifications Coalescing}{Device Types / Network Device / Device Operation / Control Virtqueue / Notifications Coalescing}

The driver MUST set \field{vq_index} to the virtqueue index of an enabled transmit or receive virtqueue.

The driver MUST have negotiated the VIRTIO_NET_F_NOTF_COAL feature when issuing commands VIRTIO_NET_CTRL_NOTF_COAL_TX_SET and VIRTIO_NET_CTRL_NOTF_COAL_RX_SET.

The driver MUST have negotiated the VIRTIO_NET_F_VQ_NOTF_COAL feature when issuing commands VIRTIO_NET_CTRL_NOTF_COAL_VQ_SET and VIRTIO_NET_CTRL_NOTF_COAL_VQ_GET.

The driver MUST ignore the values of coalescing parameters received from the VIRTIO_NET_CTRL_NOTF_COAL_VQ_GET command if the device responds with VIRTIO_NET_ERR.

\devicenormative{\subparagraph}{Notifications Coalescing}{Device Types / Network Device / Device Operation / Control Virtqueue / Notifications Coalescing}

The device MUST ignore \field{reserved}.

The device SHOULD respond to VIRTIO_NET_CTRL_NOTF_COAL_TX_SET and VIRTIO_NET_CTRL_NOTF_COAL_RX_SET commands with VIRTIO_NET_ERR if it was not able to change the parameters.

The device MUST respond to the VIRTIO_NET_CTRL_NOTF_COAL_VQ_SET command with VIRTIO_NET_ERR if it was not able to change the parameters.

The device MUST respond to VIRTIO_NET_CTRL_NOTF_COAL_VQ_SET and VIRTIO_NET_CTRL_NOTF_COAL_VQ_GET commands with
VIRTIO_NET_ERR if the designated virtqueue is not an enabled transmit or receive virtqueue.

Upon disabling and re-enabling a transmit virtqueue, the device MUST set the coalescing parameters of the virtqueue
to those configured through the VIRTIO_NET_CTRL_NOTF_COAL_TX_SET command, or, if the driver did not set any TX coalescing parameters, to 0.

Upon disabling and re-enabling a receive virtqueue, the device MUST set the coalescing parameters of the virtqueue
to those configured through the VIRTIO_NET_CTRL_NOTF_COAL_RX_SET command, or, if the driver did not set any RX coalescing parameters, to 0.

The behavior of the device in response to set commands of the VIRTIO_NET_CTRL_NOTF_COAL class is best-effort:
the device MAY generate notifications more or less frequently than specified.

A device SHOULD NOT send used buffer notifications to the driver if the notifications are suppressed, even if the notification conditions are met.

Upon reset, a device MUST initialize all coalescing parameters to 0.

\paragraph{Device Statistics}\label{sec:Device Types / Network Device / Device Operation / Control Virtqueue / Device Statistics}

If the VIRTIO_NET_F_DEVICE_STATS feature is negotiated, the driver can obtain
device statistics from the device by using the following command.

Different types of virtqueues have different statistics. The statistics of the
receiveq are different from those of the transmitq.

The statistics of a certain type of virtqueue are also divided into multiple types
because different types require different features. This enables the expansion
of new statistics.

In one command, the driver can obtain the statistics of one or multiple virtqueues.
Additionally, the driver can obtain multiple type statistics of each virtqueue.

\subparagraph{Query Statistic Capabilities}\label{sec:Device Types / Network Device / Device Operation / Control Virtqueue / Device Statistics / Query Statistic Capabilities}

\begin{lstlisting}
#define VIRTIO_NET_CTRL_STATS         8
#define VIRTIO_NET_CTRL_STATS_QUERY   0
#define VIRTIO_NET_CTRL_STATS_GET     1

struct virtio_net_stats_capabilities {

#define VIRTIO_NET_STATS_TYPE_CVQ       (1 << 32)

#define VIRTIO_NET_STATS_TYPE_RX_BASIC  (1 << 0)
#define VIRTIO_NET_STATS_TYPE_RX_CSUM   (1 << 1)
#define VIRTIO_NET_STATS_TYPE_RX_GSO    (1 << 2)
#define VIRTIO_NET_STATS_TYPE_RX_SPEED  (1 << 3)

#define VIRTIO_NET_STATS_TYPE_TX_BASIC  (1 << 16)
#define VIRTIO_NET_STATS_TYPE_TX_CSUM   (1 << 17)
#define VIRTIO_NET_STATS_TYPE_TX_GSO    (1 << 18)
#define VIRTIO_NET_STATS_TYPE_TX_SPEED  (1 << 19)

    le64 supported_stats_types[1];
}
\end{lstlisting}

To obtain device statistic capability, use the VIRTIO_NET_CTRL_STATS_QUERY
command. When the command completes successfully, \field{command-specific-result}
is in the format of \field{struct virtio_net_stats_capabilities}.

\subparagraph{Get Statistics}\label{sec:Device Types / Network Device / Device Operation / Control Virtqueue / Device Statistics / Get Statistics}

\begin{lstlisting}
struct virtio_net_ctrl_queue_stats {
       struct {
           le16 vq_index;
           le16 reserved[3];
           le64 types_bitmap[1];
       } stats[];
};

struct virtio_net_stats_reply_hdr {
#define VIRTIO_NET_STATS_TYPE_REPLY_CVQ       32

#define VIRTIO_NET_STATS_TYPE_REPLY_RX_BASIC  0
#define VIRTIO_NET_STATS_TYPE_REPLY_RX_CSUM   1
#define VIRTIO_NET_STATS_TYPE_REPLY_RX_GSO    2
#define VIRTIO_NET_STATS_TYPE_REPLY_RX_SPEED  3

#define VIRTIO_NET_STATS_TYPE_REPLY_TX_BASIC  16
#define VIRTIO_NET_STATS_TYPE_REPLY_TX_CSUM   17
#define VIRTIO_NET_STATS_TYPE_REPLY_TX_GSO    18
#define VIRTIO_NET_STATS_TYPE_REPLY_TX_SPEED  19
    u8 type;
    u8 reserved;
    le16 vq_index;
    le16 reserved1;
    le16 size;
}
\end{lstlisting}

To obtain device statistics, use the VIRTIO_NET_CTRL_STATS_GET command with the
\field{command-specific-data} which is in the format of
\field{struct virtio_net_ctrl_queue_stats}. When the command completes
successfully, \field{command-specific-result} contains multiple statistic
results, each statistic result has the \field{struct virtio_net_stats_reply_hdr}
as the header.

The fields of the \field{struct virtio_net_ctrl_queue_stats}:
\begin{description}
    \item [vq_index]
        The index of the virtqueue to obtain the statistics.

    \item [types_bitmap]
        This is a bitmask of the types of statistics to be obtained. Therefore, a
        \field{stats} inside \field{struct virtio_net_ctrl_queue_stats} may
        indicate multiple statistic replies for the virtqueue.
\end{description}

The fields of the \field{struct virtio_net_stats_reply_hdr}:
\begin{description}
    \item [type]
        The type of the reply statistic.

    \item [vq_index]
        The virtqueue index of the reply statistic.

    \item [size]
        The number of bytes for the statistics entry including size of \field{struct virtio_net_stats_reply_hdr}.

\end{description}

\subparagraph{Controlq Statistics}\label{sec:Device Types / Network Device / Device Operation / Control Virtqueue / Device Statistics / Controlq Statistics}

The structure corresponding to the controlq statistics is
\field{struct virtio_net_stats_cvq}. The corresponding type is
VIRTIO_NET_STATS_TYPE_CVQ. This is for the controlq.

\begin{lstlisting}
struct virtio_net_stats_cvq {
    struct virtio_net_stats_reply_hdr hdr;

    le64 command_num;
    le64 ok_num;
};
\end{lstlisting}

\begin{description}
    \item [command_num]
        The number of commands received by the device including the current command.

    \item [ok_num]
        The number of commands completed successfully by the device including the current command.
\end{description}


\subparagraph{Receiveq Basic Statistics}\label{sec:Device Types / Network Device / Device Operation / Control Virtqueue / Device Statistics / Receiveq Basic Statistics}

The structure corresponding to the receiveq basic statistics is
\field{struct virtio_net_stats_rx_basic}. The corresponding type is
VIRTIO_NET_STATS_TYPE_RX_BASIC. This is for the receiveq.

Receiveq basic statistics do not require any feature. As long as the device supports
VIRTIO_NET_F_DEVICE_STATS, the following are the receiveq basic statistics.

\begin{lstlisting}
struct virtio_net_stats_rx_basic {
    struct virtio_net_stats_reply_hdr hdr;

    le64 rx_notifications;

    le64 rx_packets;
    le64 rx_bytes;

    le64 rx_interrupts;

    le64 rx_drops;
    le64 rx_drop_overruns;
};
\end{lstlisting}

The packets described below were all presented on the specified virtqueue.
\begin{description}
    \item [rx_notifications]
        The number of driver notifications received by the device for this
        receiveq.

    \item [rx_packets]
        This is the number of packets passed to the driver by the device.

    \item [rx_bytes]
        This is the bytes of packets passed to the driver by the device.

    \item [rx_interrupts]
        The number of interrupts generated by the device for this receiveq.

    \item [rx_drops]
        This is the number of packets dropped by the device. The count includes
        all types of packets dropped by the device.

    \item [rx_drop_overruns]
        This is the number of packets dropped by the device when no more
        descriptors were available.

\end{description}

\subparagraph{Transmitq Basic Statistics}\label{sec:Device Types / Network Device / Device Operation / Control Virtqueue / Device Statistics / Transmitq Basic Statistics}

The structure corresponding to the transmitq basic statistics is
\field{struct virtio_net_stats_tx_basic}. The corresponding type is
VIRTIO_NET_STATS_TYPE_TX_BASIC. This is for the transmitq.

Transmitq basic statistics do not require any feature. As long as the device supports
VIRTIO_NET_F_DEVICE_STATS, the following are the transmitq basic statistics.

\begin{lstlisting}
struct virtio_net_stats_tx_basic {
    struct virtio_net_stats_reply_hdr hdr;

    le64 tx_notifications;

    le64 tx_packets;
    le64 tx_bytes;

    le64 tx_interrupts;

    le64 tx_drops;
    le64 tx_drop_malformed;
};
\end{lstlisting}

The packets described below are all for a specific virtqueue.
\begin{description}
    \item [tx_notifications]
        The number of driver notifications received by the device for this
        transmitq.

    \item [tx_packets]
        This is the number of packets sent by the device (not the packets
        got from the driver).

    \item [tx_bytes]
        This is the number of bytes sent by the device for all the sent packets
        (not the bytes sent got from the driver).

    \item [tx_interrupts]
        The number of interrupts generated by the device for this transmitq.

    \item [tx_drops]
        The number of packets dropped by the device. The count includes all
        types of packets dropped by the device.

    \item [tx_drop_malformed]
        The number of packets dropped by the device, when the descriptors are
        malformed. For example, the buffer is too short.
\end{description}

\subparagraph{Receiveq CSUM Statistics}\label{sec:Device Types / Network Device / Device Operation / Control Virtqueue / Device Statistics / Receiveq CSUM Statistics}

The structure corresponding to the receiveq checksum statistics is
\field{struct virtio_net_stats_rx_csum}. The corresponding type is
VIRTIO_NET_STATS_TYPE_RX_CSUM. This is for the receiveq.

Only after the VIRTIO_NET_F_GUEST_CSUM is negotiated, the receiveq checksum
statistics can be obtained.

\begin{lstlisting}
struct virtio_net_stats_rx_csum {
    struct virtio_net_stats_reply_hdr hdr;

    le64 rx_csum_valid;
    le64 rx_needs_csum;
    le64 rx_csum_none;
    le64 rx_csum_bad;
};
\end{lstlisting}

The packets described below were all presented on the specified virtqueue.
\begin{description}
    \item [rx_csum_valid]
        The number of packets with VIRTIO_NET_HDR_F_DATA_VALID.

    \item [rx_needs_csum]
        The number of packets with VIRTIO_NET_HDR_F_NEEDS_CSUM.

    \item [rx_csum_none]
        The number of packets without hardware checksum. The packet here refers
        to the non-TCP/UDP packet that the device cannot recognize.

    \item [rx_csum_bad]
        The number of packets with checksum mismatch.

\end{description}

\subparagraph{Transmitq CSUM Statistics}\label{sec:Device Types / Network Device / Device Operation / Control Virtqueue / Device Statistics / Transmitq CSUM Statistics}

The structure corresponding to the transmitq checksum statistics is
\field{struct virtio_net_stats_tx_csum}. The corresponding type is
VIRTIO_NET_STATS_TYPE_TX_CSUM. This is for the transmitq.

Only after the VIRTIO_NET_F_CSUM is negotiated, the transmitq checksum
statistics can be obtained.

The following are the transmitq checksum statistics:

\begin{lstlisting}
struct virtio_net_stats_tx_csum {
    struct virtio_net_stats_reply_hdr hdr;

    le64 tx_csum_none;
    le64 tx_needs_csum;
};
\end{lstlisting}

The packets described below are all for a specific virtqueue.
\begin{description}
    \item [tx_csum_none]
        The number of packets which do not require hardware checksum.

    \item [tx_needs_csum]
        The number of packets which require checksum calculation by the device.

\end{description}

\subparagraph{Receiveq GSO Statistics}\label{sec:Device Types / Network Device / Device Operation / Control Virtqueue / Device Statistics / Receiveq GSO Statistics}

The structure corresponding to the receivq GSO statistics is
\field{struct virtio_net_stats_rx_gso}. The corresponding type is
VIRTIO_NET_STATS_TYPE_RX_GSO. This is for the receiveq.

If one or more of the VIRTIO_NET_F_GUEST_TSO4, VIRTIO_NET_F_GUEST_TSO6
have been negotiated, the receiveq GSO statistics can be obtained.

GSO packets refer to packets passed by the device to the driver where
\field{gso_type} is not VIRTIO_NET_HDR_GSO_NONE.

\begin{lstlisting}
struct virtio_net_stats_rx_gso {
    struct virtio_net_stats_reply_hdr hdr;

    le64 rx_gso_packets;
    le64 rx_gso_bytes;
    le64 rx_gso_packets_coalesced;
    le64 rx_gso_bytes_coalesced;
};
\end{lstlisting}

The packets described below were all presented on the specified virtqueue.
\begin{description}
    \item [rx_gso_packets]
        The number of the GSO packets received by the device.

    \item [rx_gso_bytes]
        The bytes of the GSO packets received by the device.
        This includes the header size of the GSO packet.

    \item [rx_gso_packets_coalesced]
        The number of the GSO packets coalesced by the device.

    \item [rx_gso_bytes_coalesced]
        The bytes of the GSO packets coalesced by the device.
        This includes the header size of the GSO packet.
\end{description}

\subparagraph{Transmitq GSO Statistics}\label{sec:Device Types / Network Device / Device Operation / Control Virtqueue / Device Statistics / Transmitq GSO Statistics}

The structure corresponding to the transmitq GSO statistics is
\field{struct virtio_net_stats_tx_gso}. The corresponding type is
VIRTIO_NET_STATS_TYPE_TX_GSO. This is for the transmitq.

If one or more of the VIRTIO_NET_F_HOST_TSO4, VIRTIO_NET_F_HOST_TSO6,
VIRTIO_NET_F_HOST_USO options have been negotiated, the transmitq GSO statistics
can be obtained.

GSO packets refer to packets passed by the driver to the device where
\field{gso_type} is not VIRTIO_NET_HDR_GSO_NONE.
See more \ref{sec:Device Types / Network Device / Device Operation / Packet
Transmission}.

\begin{lstlisting}
struct virtio_net_stats_tx_gso {
    struct virtio_net_stats_reply_hdr hdr;

    le64 tx_gso_packets;
    le64 tx_gso_bytes;
    le64 tx_gso_segments;
    le64 tx_gso_segments_bytes;
    le64 tx_gso_packets_noseg;
    le64 tx_gso_bytes_noseg;
};
\end{lstlisting}

The packets described below are all for a specific virtqueue.
\begin{description}
    \item [tx_gso_packets]
        The number of the GSO packets sent by the device.

    \item [tx_gso_bytes]
        The bytes of the GSO packets sent by the device.

    \item [tx_gso_segments]
        The number of segments prepared from GSO packets.

    \item [tx_gso_segments_bytes]
        The bytes of segments prepared from GSO packets.

    \item [tx_gso_packets_noseg]
        The number of the GSO packets without segmentation.

    \item [tx_gso_bytes_noseg]
        The bytes of the GSO packets without segmentation.

\end{description}

\subparagraph{Receiveq Speed Statistics}\label{sec:Device Types / Network Device / Device Operation / Control Virtqueue / Device Statistics / Receiveq Speed Statistics}

The structure corresponding to the receiveq speed statistics is
\field{struct virtio_net_stats_rx_speed}. The corresponding type is
VIRTIO_NET_STATS_TYPE_RX_SPEED. This is for the receiveq.

The device has the allowance for the speed. If VIRTIO_NET_F_SPEED_DUPLEX has
been negotiated, the driver can get this by \field{speed}. When the received
packets bitrate exceeds the \field{speed}, some packets may be dropped by the
device.

\begin{lstlisting}
struct virtio_net_stats_rx_speed {
    struct virtio_net_stats_reply_hdr hdr;

    le64 rx_packets_allowance_exceeded;
    le64 rx_bytes_allowance_exceeded;
};
\end{lstlisting}

The packets described below were all presented on the specified virtqueue.
\begin{description}
    \item [rx_packets_allowance_exceeded]
        The number of the packets dropped by the device due to the received
        packets bitrate exceeding the \field{speed}.

    \item [rx_bytes_allowance_exceeded]
        The bytes of the packets dropped by the device due to the received
        packets bitrate exceeding the \field{speed}.

\end{description}

\subparagraph{Transmitq Speed Statistics}\label{sec:Device Types / Network Device / Device Operation / Control Virtqueue / Device Statistics / Transmitq Speed Statistics}

The structure corresponding to the transmitq speed statistics is
\field{struct virtio_net_stats_tx_speed}. The corresponding type is
VIRTIO_NET_STATS_TYPE_TX_SPEED. This is for the transmitq.

The device has the allowance for the speed. If VIRTIO_NET_F_SPEED_DUPLEX has
been negotiated, the driver can get this by \field{speed}. When the transmit
packets bitrate exceeds the \field{speed}, some packets may be dropped by the
device.

\begin{lstlisting}
struct virtio_net_stats_tx_speed {
    struct virtio_net_stats_reply_hdr hdr;

    le64 tx_packets_allowance_exceeded;
    le64 tx_bytes_allowance_exceeded;
};
\end{lstlisting}

The packets described below were all presented on the specified virtqueue.
\begin{description}
    \item [tx_packets_allowance_exceeded]
        The number of the packets dropped by the device due to the transmit packets
        bitrate exceeding the \field{speed}.

    \item [tx_bytes_allowance_exceeded]
        The bytes of the packets dropped by the device due to the transmit packets
        bitrate exceeding the \field{speed}.

\end{description}

\devicenormative{\subparagraph}{Device Statistics}{Device Types / Network Device / Device Operation / Control Virtqueue / Device Statistics}

When the VIRTIO_NET_F_DEVICE_STATS feature is negotiated, the device MUST reply
to the command VIRTIO_NET_CTRL_STATS_QUERY with the
\field{struct virtio_net_stats_capabilities}. \field{supported_stats_types}
includes all the statistic types supported by the device.

If \field{struct virtio_net_ctrl_queue_stats} is incorrect (such as the
following), the device MUST set \field{ack} to VIRTIO_NET_ERR. Even if there is
only one error, the device MUST fail the entire command.
\begin{itemize}
    \item \field{vq_index} exceeds the queue range.
    \item \field{types_bitmap} contains unknown types.
    \item One or more of the bits present in \field{types_bitmap} is not valid
        for the specified virtqueue.
    \item The feature corresponding to the specified \field{types_bitmap} was
        not negotiated.
\end{itemize}

The device MUST set the actual size of the bytes occupied by the reply to the
\field{size} of the \field{hdr}. And the device MUST set the \field{type} and
the \field{vq_index} of the statistic header.

The \field{command-specific-result} buffer allocated by the driver may be
smaller or bigger than all the statistics specified by
\field{struct virtio_net_ctrl_queue_stats}. The device MUST fill up only upto
the valid bytes.

The statistics counter replied by the device MUST wrap around to zero by the
device on the overflow.

\drivernormative{\subparagraph}{Device Statistics}{Device Types / Network Device / Device Operation / Control Virtqueue / Device Statistics}

The types contained in the \field{types_bitmap} MUST be queried from the device
via command VIRTIO_NET_CTRL_STATS_QUERY.

\field{types_bitmap} in \field{struct virtio_net_ctrl_queue_stats} MUST be valid to the
vq specified by \field{vq_index}.

The \field{command-specific-result} buffer allocated by the driver MUST have
enough capacity to store all the statistics reply headers defined in
\field{struct virtio_net_ctrl_queue_stats}. If the
\field{command-specific-result} buffer is fully utilized by the device but some
replies are missed, it is possible that some statistics may exceed the capacity
of the driver's records. In such cases, the driver should allocate additional
space for the \field{command-specific-result} buffer.

\subsubsection{Flow filter}\label{sec:Device Types / Network Device / Device Operation / Flow filter}

A network device can support one or more flow filter rules. Each flow filter rule
is applied by matching a packet and then taking an action, such as directing the packet
to a specific receiveq or dropping the packet. An example of a match is
matching on specific source and destination IP addresses.

A flow filter rule is a device resource object that consists of a key,
a processing priority, and an action to either direct a packet to a
receive queue or drop the packet.

Each rule uses a classifier. The key is matched against the packet using
a classifier, defining which fields in the packet are matched.
A classifier resource object consists of one or more field selectors, each with
a type that specifies the header fields to be matched against, and a mask.
The mask can match whole fields or parts of a field in a header. Each
rule resource object depends on the classifier resource object.

When a packet is received, relevant fields are extracted
(in the same way) from both the packet and the key according to the
classifier. The resulting field contents are then compared -
if they are identical the rule action is taken, if they are not, the rule is ignored.

Multiple flow filter rules are part of a group. The rule resource object
depends on the group. Each rule within a
group has a rule priority, and each group also has a group priority. For a
packet, a group with the highest priority is selected first. Within a group,
rules are applied from highest to lowest priority, until one of the rules
matches the packet and an action is taken. If all the rules within a group
are ignored, the group with the next highest priority is selected, and so on.

The device and the driver indicates flow filter resource limits using the capability
\ref{par:Device Types / Network Device / Device Operation / Flow filter / Device and driver capabilities / VIRTIO-NET-FF-RESOURCE-CAP} specifying the limits on the number of flow filter rule,
group and classifier resource objects. The capability \ref{par:Device Types / Network Device / Device Operation / Flow filter / Device and driver capabilities / VIRTIO-NET-FF-SELECTOR-CAP} specifies which selectors the device supports.
The driver indicates the selectors it is using by setting the flow
filter selector capability, prior to adding any resource objects.

The capability \ref{par:Device Types / Network Device / Device Operation / Flow filter / Device and driver capabilities / VIRTIO-NET-FF-ACTION-CAP} specifies which actions the device supports.

The driver controls the flow filter rule, classifier and group resource objects using
administration commands described in
\ref{sec:Basic Facilities of a Virtio Device / Device groups / Group administration commands / Device resource objects}.

\paragraph{Packet processing order}\label{sec:sec:Device Types / Network Device / Device Operation / Flow filter / Packet processing order}

Note that flow filter rules are applied after MAC/VLAN filtering. Flow filter
rules take precedence over steering: if a flow filter rule results in an action,
the steering configuration does not apply. The steering configuration only applies
to packets for which no flow filter rule action was performed. For example,
incoming packets can be processed in the following order:

\begin{itemize}
\item apply steering configuration received using control virtqueue commands
      VIRTIO_NET_CTRL_RX, VIRTIO_NET_CTRL_MAC and VIRTIO_NET_CTRL_VLAN.
\item apply flow filter rules if any.
\item if no filter rule applied, apply steering configuration received using command
      VIRTIO_NET_CTRL_MQ_RSS_CONFIG or as per automatic receive steering.
\end{itemize}

Some incoming packet processing examples:
\begin{itemize}
\item If the packet is dropped by the flow filter rule, RSS
      steering is ignored for the packet.
\item If the packet is directed to a specific receiveq using flow filter rule,
      the RSS steering is ignored for the packet.
\item If a packet is dropped due to the VIRTIO_NET_CTRL_MAC configuration,
      both flow filter rules and the RSS steering are ignored for the packet.
\item If a packet does not match any flow filter rules,
      the RSS steering is used to select the receiveq for the packet (if enabled).
\item If there are two flow filter groups configured as group_A and group_B
      with respective group priorities as 4, and 5; flow filter rules of
      group_B are applied first having highest group priority, if there is a match,
      the flow filter rules of group_A are ignored; if there is no match for
      the flow filter rules in group_B, the flow filter rules of next level group_A are applied.
\end{itemize}

\paragraph{Device and driver capabilities}
\label{par:Device Types / Network Device / Device Operation / Flow filter / Device and driver capabilities}

\subparagraph{VIRTIO_NET_FF_RESOURCE_CAP}
\label{par:Device Types / Network Device / Device Operation / Flow filter / Device and driver capabilities / VIRTIO-NET-FF-RESOURCE-CAP}

The capability VIRTIO_NET_FF_RESOURCE_CAP indicates the flow filter resource limits.
\field{cap_specific_data} is in the format
\field{struct virtio_net_ff_cap_data}.

\begin{lstlisting}
struct virtio_net_ff_cap_data {
        le32 groups_limit;
        le32 selectors_limit;
        le32 rules_limit;
        le32 rules_per_group_limit;
        u8 last_rule_priority;
        u8 selectors_per_classifier_limit;
};
\end{lstlisting}

\field{groups_limit}, and \field{selectors_limit} represent the maximum
number of flow filter groups and selectors, respectively, that the driver can create.
 \field{rules_limit} is the maximum number of
flow fiilter rules that the driver can create across all the groups.
\field{rules_per_group_limit} is the maximum number of flow filter rules that the driver
can create for each flow filter group.

\field{last_rule_priority} is the highest priority that can be assigned to a
flow filter rule.

\field{selectors_per_classifier_limit} is the maximum number of selectors
that a classifier can have.

\subparagraph{VIRTIO_NET_FF_SELECTOR_CAP}
\label{par:Device Types / Network Device / Device Operation / Flow filter / Device and driver capabilities / VIRTIO-NET-FF-SELECTOR-CAP}

The capability VIRTIO_NET_FF_SELECTOR_CAP lists the supported selectors and the
supported packet header fields for each selector.
\field{cap_specific_data} is in the format \field{struct virtio_net_ff_cap_mask_data}.

\begin{lstlisting}[label={lst:Device Types / Network Device / Device Operation / Flow filter / Device and driver capabilities / VIRTIO-NET-FF-SELECTOR-CAP / virtio-net-ff-selector}]
struct virtio_net_ff_selector {
        u8 type;
        u8 flags;
        u8 reserved[2];
        u8 length;
        u8 reserved1[3];
        u8 mask[];
};

struct virtio_net_ff_cap_mask_data {
        u8 count;
        u8 reserved[7];
        struct virtio_net_ff_selector selectors[];
};

#define VIRTIO_NET_FF_MASK_F_PARTIAL_MASK (1 << 0)
\end{lstlisting}

\field{count} indicates number of valid entries in the \field{selectors} array.
\field{selectors[]} is an array of supported selectors. Within each array entry:
\field{type} specifies the type of the packet header, as defined in table
\ref{table:Device Types / Network Device / Device Operation / Flow filter / Device and driver capabilities / VIRTIO-NET-FF-SELECTOR-CAP / flow filter selector types}. \field{mask} specifies which fields of the
packet header can be matched in a flow filter rule.

Each \field{type} is also listed in table
\ref{table:Device Types / Network Device / Device Operation / Flow filter / Device and driver capabilities / VIRTIO-NET-FF-SELECTOR-CAP / flow filter selector types}. \field{mask} is a byte array
in network byte order. For example, when \field{type} is VIRTIO_NET_FF_MASK_TYPE_IPV6,
the \field{mask} is in the format \hyperref[intro:IPv6-Header-Format]{IPv6 Header Format}.

If partial masking is not set, then all bits in each field have to be either all 0s
to ignore this field or all 1s to match on this field. If partial masking is set,
then any combination of bits can bit set to match on these bits.
For example, when a selector \field{type} is VIRTIO_NET_FF_MASK_TYPE_ETH, if
\field{mask[0-12]} are zero and \field{mask[13-14]} are 0xff (all 1s), it
indicates that matching is only supported for \field{EtherType} of
\field{Ethernet MAC frame}, matching is not supported for
\field{Destination Address} and \field{Source Address}.

The entries in the array \field{selectors} are ordered by
\field{type}, with each \field{type} value only appearing once.

\field{length} is the length of a dynamic array \field{mask} in bytes.
\field{reserved} and \field{reserved1} are reserved and set to zero.

\begin{table}[H]
\caption{Flow filter selector types}
\label{table:Device Types / Network Device / Device Operation / Flow filter / Device and driver capabilities / VIRTIO-NET-FF-SELECTOR-CAP / flow filter selector types}
\begin{tabularx}{\textwidth}{ |l|X|X| }
\hline
Type & Name & Description \\
\hline \hline
0x0 & - & Reserved \\
\hline
0x1 & VIRTIO_NET_FF_MASK_TYPE_ETH & 14 bytes of frame header starting from destination address described in \hyperref[intro:IEEE 802.3-2022]{IEEE 802.3-2022} \\
\hline
0x2 & VIRTIO_NET_FF_MASK_TYPE_IPV4 & 20 bytes of \hyperref[intro:Internet-Header-Format]{IPv4: Internet Header Format} \\
\hline
0x3 & VIRTIO_NET_FF_MASK_TYPE_IPV6 & 40 bytes of \hyperref[intro:IPv6-Header-Format]{IPv6 Header Format} \\
\hline
0x4 & VIRTIO_NET_FF_MASK_TYPE_TCP & 20 bytes of \hyperref[intro:TCP-Header-Format]{TCP Header Format} \\
\hline
0x5 & VIRTIO_NET_FF_MASK_TYPE_UDP & 8 bytes of UDP header described in \hyperref[intro:UDP]{UDP} \\
\hline
0x6 - 0xFF & & Reserved for future \\
\hline
\end{tabularx}
\end{table}

When VIRTIO_NET_FF_MASK_F_PARTIAL_MASK (bit 0) is set, it indicates that
partial masking is supported for all the fields of the selector identified by \field{type}.

For the selector \field{type} VIRTIO_NET_FF_MASK_TYPE_IPV4, if a partial mask is unsupported,
then matching on an individual bit of \field{Flags} in the
\field{IPv4: Internet Header Format} is unsupported. \field{Flags} has to match as a whole
if it is supported.

For the selector \field{type} VIRTIO_NET_FF_MASK_TYPE_IPV4, \field{mask} includes fields
up to the \field{Destination Address}; that is, \field{Options} and
\field{Padding} are excluded.

For the selector \field{type} VIRTIO_NET_FF_MASK_TYPE_IPV6, the \field{Next Header} field
of the \field{mask} corresponds to the \field{Next Header} in the packet
when \field{IPv6 Extension Headers} are not present. When the packet includes
one or more \field{IPv6 Extension Headers}, the \field{Next Header} field of
the \field{mask} corresponds to the \field{Next Header} of the last
\field{IPv6 Extension Header} in the packet.

For the selector \field{type} VIRTIO_NET_FF_MASK_TYPE_TCP, \field{Control bits}
are treated as individual fields for matching; that is, matching individual
\field{Control bits} does not depend on the partial mask support.

\subparagraph{VIRTIO_NET_FF_ACTION_CAP}
\label{par:Device Types / Network Device / Device Operation / Flow filter / Device and driver capabilities / VIRTIO-NET-FF-ACTION-CAP}

The capability VIRTIO_NET_FF_ACTION_CAP lists the supported actions in a rule.
\field{cap_specific_data} is in the format \field{struct virtio_net_ff_cap_actions}.

\begin{lstlisting}
struct virtio_net_ff_actions {
        u8 count;
        u8 reserved[7];
        u8 actions[];
};
\end{lstlisting}

\field{actions} is an array listing all possible actions.
The entries in the array are ordered from the smallest to the largest,
with each supported value appearing exactly once. Each entry can have the
following values:

\begin{table}[H]
\caption{Flow filter rule actions}
\label{table:Device Types / Network Device / Device Operation / Flow filter / Device and driver capabilities / VIRTIO-NET-FF-ACTION-CAP / flow filter rule actions}
\begin{tabularx}{\textwidth}{ |l|X|X| }
\hline
Action & Name & Description \\
\hline \hline
0x0 & - & reserved \\
\hline
0x1 & VIRTIO_NET_FF_ACTION_DROP & Matching packet will be dropped by the device \\
\hline
0x2 & VIRTIO_NET_FF_ACTION_DIRECT_RX_VQ & Matching packet will be directed to a receive queue \\
\hline
0x3 - 0xFF & & Reserved for future \\
\hline
\end{tabularx}
\end{table}

\paragraph{Resource objects}
\label{par:Device Types / Network Device / Device Operation / Flow filter / Resource objects}

\subparagraph{VIRTIO_NET_RESOURCE_OBJ_FF_GROUP}\label{par:Device Types / Network Device / Device Operation / Flow filter / Resource objects / VIRTIO-NET-RESOURCE-OBJ-FF-GROUP}

A flow filter group contains between 0 and \field{rules_limit} rules, as specified by the
capability VIRTIO_NET_FF_RESOURCE_CAP. For the flow filter group object both
\field{resource_obj_specific_data} and
\field{resource_obj_specific_result} are in the format
\field{struct virtio_net_resource_obj_ff_group}.

\begin{lstlisting}
struct virtio_net_resource_obj_ff_group {
        le16 group_priority;
};
\end{lstlisting}

\field{group_priority} specifies the priority for the group. Each group has a
distinct priority. For each incoming packet, the device tries to apply rules
from groups from higher \field{group_priority} value to lower, until either a
rule matches the packet or all groups have been tried.

\subparagraph{VIRTIO_NET_RESOURCE_OBJ_FF_CLASSIFIER}\label{par:Device Types / Network Device / Device Operation / Flow filter / Resource objects / VIRTIO-NET-RESOURCE-OBJ-FF-CLASSIFIER}

A classifier is used to match a flow filter key against a packet. The
classifier defines the desired packet fields to match, and is represented by
the VIRTIO_NET_RESOURCE_OBJ_FF_CLASSIFIER device resource object.

For the flow filter classifier object both \field{resource_obj_specific_data} and
\field{resource_obj_specific_result} are in the format
\field{struct virtio_net_resource_obj_ff_classifier}.

\begin{lstlisting}
struct virtio_net_resource_obj_ff_classifier {
        u8 count;
        u8 reserved[7];
        struct virtio_net_ff_selector selectors[];
};
\end{lstlisting}

A classifier is an array of \field{selectors}. The number of selectors in the
array is indicated by \field{count}. The selector has a type that specifies
the header fields to be matched against, and a mask.
See \ref{lst:Device Types / Network Device / Device Operation / Flow filter / Device and driver capabilities / VIRTIO-NET-FF-SELECTOR-CAP / virtio-net-ff-selector}
for details about selectors.

The first selector is always VIRTIO_NET_FF_MASK_TYPE_ETH. When there are multiple
selectors, a second selector can be either VIRTIO_NET_FF_MASK_TYPE_IPV4
or VIRTIO_NET_FF_MASK_TYPE_IPV6. If the third selector exists, the third
selector can be either VIRTIO_NET_FF_MASK_TYPE_UDP or VIRTIO_NET_FF_MASK_TYPE_TCP.
For example, to match a Ethernet IPv6 UDP packet,
\field{selectors[0].type} is set to VIRTIO_NET_FF_MASK_TYPE_ETH, \field{selectors[1].type}
is set to VIRTIO_NET_FF_MASK_TYPE_IPV6 and \field{selectors[2].type} is
set to VIRTIO_NET_FF_MASK_TYPE_UDP; accordingly, \field{selectors[0].mask[0-13]} is
for Ethernet header fields, \field{selectors[1].mask[0-39]} is set for IPV6 header
and \field{selectors[2].mask[0-7]} is set for UDP header.

When there are multiple selectors, the type of the (N+1)\textsuperscript{th} selector
affects the mask of the (N)\textsuperscript{th} selector. If
\field{count} is 2 or more, all the mask bits within \field{selectors[0]}
corresponding to \field{EtherType} of an Ethernet header are set.

If \field{count} is more than 2:
\begin{itemize}
\item if \field{selector[1].type} is, VIRTIO_NET_FF_MASK_TYPE_IPV4, then, all the mask bits within
\field{selector[1]} for \field{Protocol} is set.
\item if \field{selector[1].type} is, VIRTIO_NET_FF_MASK_TYPE_IPV6, then, all the mask bits within
\field{selector[1]} for \field{Next Header} is set.
\end{itemize}

If for a given packet header field, a subset of bits of a field is to be matched,
and if the partial mask is supported, the flow filter
mask object can specify a mask which has fewer bits set than the packet header
field size. For example, a partial mask for the Ethernet header source mac
address can be of 1-bit for multicast detection instead of 48-bits.

\subparagraph{VIRTIO_NET_RESOURCE_OBJ_FF_RULE}\label{par:Device Types / Network Device / Device Operation / Flow filter / Resource objects / VIRTIO-NET-RESOURCE-OBJ-FF-RULE}

Each flow filter rule resource object comprises a key, a priority, and an action.
For the flow filter rule object,
\field{resource_obj_specific_data} and
\field{resource_obj_specific_result} are in the format
\field{struct virtio_net_resource_obj_ff_rule}.

\begin{lstlisting}
struct virtio_net_resource_obj_ff_rule {
        le32 group_id;
        le32 classifier_id;
        u8 rule_priority;
        u8 key_length; /* length of key in bytes */
        u8 action;
        u8 reserved;
        le16 vq_index;
        u8 reserved1[2];
        u8 keys[][];
};
\end{lstlisting}

\field{group_id} is the resource object ID of the flow filter group to which
this rule belongs. \field{classifier_id} is the resource object ID of the
classifier used to match a packet against the \field{key}.

\field{rule_priority} denotes the priority of the rule within the group
specified by the \field{group_id}.
Rules within the group are applied from the highest to the lowest priority
until a rule matches the packet and an
action is taken. Rules with the same priority can be applied in any order.

\field{reserved} and \field{reserved1} are reserved and set to 0.

\field{keys[][]} is an array of keys to match against packets, using
the classifier specified by \field{classifier_id}. Each entry (key) comprises
a byte array, and they are located one immediately after another.
The size (number of entries) of the array is exactly the same as that of
\field{selectors} in the classifier, or in other words, \field{count}
in the classifier.

\field{key_length} specifies the total length of \field{keys} in bytes.
In other words, it equals the sum total of \field{length} of all
selectors in \field{selectors} in the classifier specified by
\field{classifier_id}.

For example, if a classifier object's \field{selectors[0].type} is
VIRTIO_NET_FF_MASK_TYPE_ETH and \field{selectors[1].type} is
VIRTIO_NET_FF_MASK_TYPE_IPV6,
then selectors[0].length is 14 and selectors[1].length is 40.
Accordingly, the \field{key_length} is set to 54.
This setting indicates that the \field{key} array's length is 54 bytes
comprising a first byte array of 14 bytes for the
Ethernet MAC header in bytes 0-13, immediately followed by 40 bytes for the
IPv6 header in bytes 14-53.

When there are multiple selectors in the classifier object, the key bytes
for (N)\textsuperscript{th} selector are set so that
(N+1)\textsuperscript{th} selector can be matched.

If \field{count} is 2 or more, key bytes of \field{EtherType}
are set according to \hyperref[intro:IEEE 802 Ethertypes]{IEEE 802 Ethertypes}
for VIRTIO_NET_FF_MASK_TYPE_IPV4 or VIRTIO_NET_FF_MASK_TYPE_IPV6 respectively.

If \field{count} is more than 2, when \field{selector[1].type} is
VIRTIO_NET_FF_MASK_TYPE_IPV4 or VIRTIO_NET_FF_MASK_TYPE_IPV6, key
bytes of \field{Protocol} or \field{Next Header} is set as per
\field{Protocol Numbers} defined \hyperref[intro:IANA Protocol Numbers]{IANA Protocol Numbers}
respectively.

\field{action} is the action to take when a packet matches the
\field{key} using the \field{classifier_id}. Supported actions are described in
\ref{table:Device Types / Network Device / Device Operation / Flow filter / Device and driver capabilities / VIRTIO-NET-FF-ACTION-CAP / flow filter rule actions}.

\field{vq_index} specifies a receive virtqueue. When the \field{action} is set
to VIRTIO_NET_FF_ACTION_DIRECT_RX_VQ, and the packet matches the \field{key},
the matching packet is directed to this virtqueue.

Note that at most one action is ever taken for a given packet. If a rule is
applied and an action is taken, the action of other rules is not taken.

\devicenormative{\paragraph}{Flow filter}{Device Types / Network Device / Device Operation / Flow filter}

When the device supports flow filter operations,
\begin{itemize}
\item the device MUST set VIRTIO_NET_FF_RESOURCE_CAP, VIRTIO_NET_FF_SELECTOR_CAP
and VIRTIO_NET_FF_ACTION_CAP capability in the \field{supported_caps} in the
command VIRTIO_ADMIN_CMD_CAP_SUPPORT_QUERY.
\item the device MUST support the administration commands
VIRTIO_ADMIN_CMD_RESOURCE_OBJ_CREATE,
VIRTIO_ADMIN_CMD_RESOURCE_OBJ_MODIFY, VIRTIO_ADMIN_CMD_RESOURCE_OBJ_QUERY,
VIRTIO_ADMIN_CMD_RESOURCE_OBJ_DESTROY for the resource types
VIRTIO_NET_RESOURCE_OBJ_FF_GROUP, VIRTIO_NET_RESOURCE_OBJ_FF_CLASSIFIER and
VIRTIO_NET_RESOURCE_OBJ_FF_RULE.
\end{itemize}

When any of the VIRTIO_NET_FF_RESOURCE_CAP, VIRTIO_NET_FF_SELECTOR_CAP, or
VIRTIO_NET_FF_ACTION_CAP capability is disabled, the device SHOULD set
\field{status} to VIRTIO_ADMIN_STATUS_Q_INVALID_OPCODE for the commands
VIRTIO_ADMIN_CMD_RESOURCE_OBJ_CREATE,
VIRTIO_ADMIN_CMD_RESOURCE_OBJ_MODIFY, VIRTIO_ADMIN_CMD_RESOURCE_OBJ_QUERY,
and VIRTIO_ADMIN_CMD_RESOURCE_OBJ_DESTROY. These commands apply to the resource
\field{type} of VIRTIO_NET_RESOURCE_OBJ_FF_GROUP, VIRTIO_NET_RESOURCE_OBJ_FF_CLASSIFIER, and
VIRTIO_NET_RESOURCE_OBJ_FF_RULE.

The device SHOULD set \field{status} to VIRTIO_ADMIN_STATUS_EINVAL for the
command VIRTIO_ADMIN_CMD_RESOURCE_OBJ_CREATE when the resource \field{type}
is VIRTIO_NET_RESOURCE_OBJ_FF_GROUP, if a flow filter group already exists
with the supplied \field{group_priority}.

The device SHOULD set \field{status} to VIRTIO_ADMIN_STATUS_ENOSPC for the
command VIRTIO_ADMIN_CMD_RESOURCE_OBJ_CREATE when the resource \field{type}
is VIRTIO_NET_RESOURCE_OBJ_FF_GROUP, if the number of flow filter group
objects in the device exceeds the lower of the configured driver
capabilities \field{groups_limit} and \field{rules_per_group_limit}.

The device SHOULD set \field{status} to VIRTIO_ADMIN_STATUS_ENOSPC for the
command VIRTIO_ADMIN_CMD_RESOURCE_OBJ_CREATE when the resource \field{type} is
VIRTIO_NET_RESOURCE_OBJ_FF_CLASSIFIER, if the number of flow filter selector
objects in the device exceeds the configured driver capability
\field{selectors_limit}.

The device SHOULD set \field{status} to VIRTIO_ADMIN_STATUS_EBUSY for the
command VIRTIO_ADMIN_CMD_RESOURCE_OBJ_DESTROY for a flow filter group when
the flow filter group has one or more flow filter rules depending on it.

The device SHOULD set \field{status} to VIRTIO_ADMIN_STATUS_EBUSY for the
command VIRTIO_ADMIN_CMD_RESOURCE_OBJ_DESTROY for a flow filter classifier when
the flow filter classifier has one or more flow filter rules depending on it.

The device SHOULD fail the command VIRTIO_ADMIN_CMD_RESOURCE_OBJ_CREATE for the
flow filter rule resource object if,
\begin{itemize}
\item \field{vq_index} is not a valid receive virtqueue index for
the VIRTIO_NET_FF_ACTION_DIRECT_RX_VQ action,
\item \field{priority} is greater than or equal to
      \field{last_rule_priority},
\item \field{id} is greater than or equal to \field{rules_limit} or
      greater than or equal to \field{rules_per_group_limit}, whichever is lower,
\item the length of \field{keys} and the length of all the mask bytes of
      \field{selectors[].mask} as referred by \field{classifier_id} differs,
\item the supplied \field{action} is not supported in the capability VIRTIO_NET_FF_ACTION_CAP.
\end{itemize}

When the flow filter directs a packet to the virtqueue identified by
\field{vq_index} and if the receive virtqueue is reset, the device
MUST drop such packets.

Upon applying a flow filter rule to a packet, the device MUST STOP any further
application of rules and cease applying any other steering configurations.

For multiple flow filter groups, the device MUST apply the rules from
the group with the highest priority. If any rule from this group is applied,
the device MUST ignore the remaining groups. If none of the rules from the
highest priority group match, the device MUST apply the rules from
the group with the next highest priority, until either a rule matches or
all groups have been attempted.

The device MUST apply the rules within the group from the highest to the
lowest priority until a rule matches the packet, and the device MUST take
the action. If an action is taken, the device MUST not take any other
action for this packet.

The device MAY apply the rules with the same \field{rule_priority} in any
order within the group.

The device MUST process incoming packets in the following order:
\begin{itemize}
\item apply the steering configuration received using control virtqueue
      commands VIRTIO_NET_CTRL_RX, VIRTIO_NET_CTRL_MAC, and
      VIRTIO_NET_CTRL_VLAN.
\item apply flow filter rules if any.
\item if no filter rule is applied, apply the steering configuration
      received using the command VIRTIO_NET_CTRL_MQ_RSS_CONFIG
      or according to automatic receive steering.
\end{itemize}

When processing an incoming packet, if the packet is dropped at any stage, the device
MUST skip further processing.

When the device drops the packet due to the configuration done using the control
virtqueue commands VIRTIO_NET_CTRL_RX or VIRTIO_NET_CTRL_MAC or VIRTIO_NET_CTRL_VLAN,
the device MUST skip flow filter rules for this packet.

When the device performs flow filter match operations and if the operation
result did not have any match in all the groups, the receive packet processing
continues to next level, i.e. to apply configuration done using
VIRTIO_NET_CTRL_MQ_RSS_CONFIG command.

The device MUST support the creation of flow filter classifier objects
using the command VIRTIO_ADMIN_CMD_RESOURCE_OBJ_CREATE with \field{flags}
set to VIRTIO_NET_FF_MASK_F_PARTIAL_MASK;
this support is required even if all the bits of the masks are set for
a field in \field{selectors}, provided that partial masking is supported
for the selectors.

\drivernormative{\paragraph}{Flow filter}{Device Types / Network Device / Device Operation / Flow filter}

The driver MUST enable VIRTIO_NET_FF_RESOURCE_CAP, VIRTIO_NET_FF_SELECTOR_CAP,
and VIRTIO_NET_FF_ACTION_CAP capabilities to use flow filter.

The driver SHOULD NOT remove a flow filter group using the command
VIRTIO_ADMIN_CMD_RESOURCE_OBJ_DESTROY when one or more flow filter rules
depend on that group. The driver SHOULD only destroy the group after
all the associated rules have been destroyed.

The driver SHOULD NOT remove a flow filter classifier using the command
VIRTIO_ADMIN_CMD_RESOURCE_OBJ_DESTROY when one or more flow filter rules
depend on the classifier. The driver SHOULD only destroy the classifier
after all the associated rules have been destroyed.

The driver SHOULD NOT add multiple flow filter rules with the same
\field{rule_priority} within a flow filter group, as these rules MAY match
the same packet. The driver SHOULD assign different \field{rule_priority}
values to different flow filter rules if multiple rules may match a single
packet.

For the command VIRTIO_ADMIN_CMD_RESOURCE_OBJ_CREATE, when creating a resource
of \field{type} VIRTIO_NET_RESOURCE_OBJ_FF_CLASSIFIER, the driver MUST set:
\begin{itemize}
\item \field{selectors[0].type} to VIRTIO_NET_FF_MASK_TYPE_ETH.
\item \field{selectors[1].type} to VIRTIO_NET_FF_MASK_TYPE_IPV4 or
      VIRTIO_NET_FF_MASK_TYPE_IPV6 when \field{count} is more than 1,
\item \field{selectors[2].type} VIRTIO_NET_FF_MASK_TYPE_UDP or
      VIRTIO_NET_FF_MASK_TYPE_TCP when \field{count} is more than 2.
\end{itemize}

For the command VIRTIO_ADMIN_CMD_RESOURCE_OBJ_CREATE, when creating a resource
of \field{type} VIRTIO_NET_RESOURCE_OBJ_FF_CLASSIFIER, the driver MUST set:
\begin{itemize}
\item \field{selectors[0].mask} bytes to all 1s for the \field{EtherType}
       when \field{count} is 2 or more.
\item \field{selectors[1].mask} bytes to all 1s for \field{Protocol} or \field{Next Header}
       when \field{selector[1].type} is VIRTIO_NET_FF_MASK_TYPE_IPV4 or VIRTIO_NET_FF_MASK_TYPE_IPV6,
       and when \field{count} is more than 2.
\end{itemize}

For the command VIRTIO_ADMIN_CMD_RESOURCE_OBJ_CREATE, the resource \field{type}
VIRTIO_NET_RESOURCE_OBJ_FF_RULE, if the corresponding classifier object's
\field{count} is 2 or more, the driver MUST SET the \field{keys} bytes of
\field{EtherType} in accordance with
\hyperref[intro:IEEE 802 Ethertypes]{IEEE 802 Ethertypes}
for either VIRTIO_NET_FF_MASK_TYPE_IPV4 or VIRTIO_NET_FF_MASK_TYPE_IPV6.

For the command VIRTIO_ADMIN_CMD_RESOURCE_OBJ_CREATE, when creating a resource of
\field{type} VIRTIO_NET_RESOURCE_OBJ_FF_RULE, if the corresponding classifier
object's \field{count} is more than 2, and the \field{selector[1].type} is either
VIRTIO_NET_FF_MASK_TYPE_IPV4 or VIRTIO_NET_FF_MASK_TYPE_IPV6, the driver MUST
set the \field{keys} bytes for the \field{Protocol} or \field{Next Header}
according to \hyperref[intro:IANA Protocol Numbers]{IANA Protocol Numbers} respectively.

The driver SHOULD set all the bits for a field in the mask of a selector in both the
capability and the classifier object, unless the VIRTIO_NET_FF_MASK_F_PARTIAL_MASK
is enabled.

\subsubsection{Legacy Interface: Framing Requirements}\label{sec:Device
Types / Network Device / Legacy Interface: Framing Requirements}

When using legacy interfaces, transitional drivers which have not
negotiated VIRTIO_F_ANY_LAYOUT MUST use a single descriptor for the
\field{struct virtio_net_hdr} on both transmit and receive, with the
network data in the following descriptors.

Additionally, when using the control virtqueue (see \ref{sec:Device
Types / Network Device / Device Operation / Control Virtqueue})
, transitional drivers which have not
negotiated VIRTIO_F_ANY_LAYOUT MUST:
\begin{itemize}
\item for all commands, use a single 2-byte descriptor including the first two
fields: \field{class} and \field{command}
\item for all commands except VIRTIO_NET_CTRL_MAC_TABLE_SET
use a single descriptor including command-specific-data
with no padding.
\item for the VIRTIO_NET_CTRL_MAC_TABLE_SET command use exactly
two descriptors including command-specific-data with no padding:
the first of these descriptors MUST include the
virtio_net_ctrl_mac table structure for the unicast addresses with no padding,
the second of these descriptors MUST include the
virtio_net_ctrl_mac table structure for the multicast addresses
with no padding.
\item for all commands, use a single 1-byte descriptor for the
\field{ack} field
\end{itemize}

See \ref{sec:Basic
Facilities of a Virtio Device / Virtqueues / Message Framing}.

\section{Network Device}\label{sec:Device Types / Network Device}

The virtio network device is a virtual network interface controller.
It consists of a virtual Ethernet link which connects the device
to the Ethernet network. The device has transmit and receive
queues. The driver adds empty buffers to the receive virtqueue.
The device receives incoming packets from the link; the device
places these incoming packets in the receive virtqueue buffers.
The driver adds outgoing packets to the transmit virtqueue. The device
removes these packets from the transmit virtqueue and sends them to
the link. The device may have a control virtqueue. The driver
uses the control virtqueue to dynamically manipulate various
features of the initialized device.

\subsection{Device ID}\label{sec:Device Types / Network Device / Device ID}

 1

\subsection{Virtqueues}\label{sec:Device Types / Network Device / Virtqueues}

\begin{description}
\item[0] receiveq1
\item[1] transmitq1
\item[\ldots]
\item[2(N-1)] receiveqN
\item[2(N-1)+1] transmitqN
\item[2N] controlq
\end{description}

 N=1 if neither VIRTIO_NET_F_MQ nor VIRTIO_NET_F_RSS are negotiated, otherwise N is set by
 \field{max_virtqueue_pairs}.

controlq is optional; it only exists if VIRTIO_NET_F_CTRL_VQ is
negotiated.

\subsection{Feature bits}\label{sec:Device Types / Network Device / Feature bits}

\begin{description}
\item[VIRTIO_NET_F_CSUM (0)] Device handles packets with partial checksum offload.

\item[VIRTIO_NET_F_GUEST_CSUM (1)] Driver handles packets with partial checksum.

\item[VIRTIO_NET_F_CTRL_GUEST_OFFLOADS (2)] Control channel offloads
        reconfiguration support.

\item[VIRTIO_NET_F_MTU(3)] Device maximum MTU reporting is supported. If
    offered by the device, device advises driver about the value of
    its maximum MTU. If negotiated, the driver uses \field{mtu} as
    the maximum MTU value.

\item[VIRTIO_NET_F_MAC (5)] Device has given MAC address.

\item[VIRTIO_NET_F_GUEST_TSO4 (7)] Driver can receive TSOv4.

\item[VIRTIO_NET_F_GUEST_TSO6 (8)] Driver can receive TSOv6.

\item[VIRTIO_NET_F_GUEST_ECN (9)] Driver can receive TSO with ECN.

\item[VIRTIO_NET_F_GUEST_UFO (10)] Driver can receive UFO.

\item[VIRTIO_NET_F_HOST_TSO4 (11)] Device can receive TSOv4.

\item[VIRTIO_NET_F_HOST_TSO6 (12)] Device can receive TSOv6.

\item[VIRTIO_NET_F_HOST_ECN (13)] Device can receive TSO with ECN.

\item[VIRTIO_NET_F_HOST_UFO (14)] Device can receive UFO.

\item[VIRTIO_NET_F_MRG_RXBUF (15)] Driver can merge receive buffers.

\item[VIRTIO_NET_F_STATUS (16)] Configuration status field is
    available.

\item[VIRTIO_NET_F_CTRL_VQ (17)] Control channel is available.

\item[VIRTIO_NET_F_CTRL_RX (18)] Control channel RX mode support.

\item[VIRTIO_NET_F_CTRL_VLAN (19)] Control channel VLAN filtering.

\item[VIRTIO_NET_F_CTRL_RX_EXTRA (20)]	Control channel RX extra mode support.

\item[VIRTIO_NET_F_GUEST_ANNOUNCE(21)] Driver can send gratuitous
    packets.

\item[VIRTIO_NET_F_MQ(22)] Device supports multiqueue with automatic
    receive steering.

\item[VIRTIO_NET_F_CTRL_MAC_ADDR(23)] Set MAC address through control
    channel.

\item[VIRTIO_NET_F_DEVICE_STATS(50)] Device can provide device-level statistics
    to the driver through the control virtqueue.

\item[VIRTIO_NET_F_HASH_TUNNEL(51)] Device supports inner header hash for encapsulated packets.

\item[VIRTIO_NET_F_VQ_NOTF_COAL(52)] Device supports virtqueue notification coalescing.

\item[VIRTIO_NET_F_NOTF_COAL(53)] Device supports notifications coalescing.

\item[VIRTIO_NET_F_GUEST_USO4 (54)] Driver can receive USOv4 packets.

\item[VIRTIO_NET_F_GUEST_USO6 (55)] Driver can receive USOv6 packets.

\item[VIRTIO_NET_F_HOST_USO (56)] Device can receive USO packets. Unlike UFO
 (fragmenting the packet) the USO splits large UDP packet
 to several segments when each of these smaller packets has UDP header.

\item[VIRTIO_NET_F_HASH_REPORT(57)] Device can report per-packet hash
    value and a type of calculated hash.

\item[VIRTIO_NET_F_GUEST_HDRLEN(59)] Driver can provide the exact \field{hdr_len}
    value. Device benefits from knowing the exact header length.

\item[VIRTIO_NET_F_RSS(60)] Device supports RSS (receive-side scaling)
    with Toeplitz hash calculation and configurable hash
    parameters for receive steering.

\item[VIRTIO_NET_F_RSC_EXT(61)] Device can process duplicated ACKs
    and report number of coalesced segments and duplicated ACKs.

\item[VIRTIO_NET_F_STANDBY(62)] Device may act as a standby for a primary
    device with the same MAC address.

\item[VIRTIO_NET_F_SPEED_DUPLEX(63)] Device reports speed and duplex.

\item[VIRTIO_NET_F_RSS_CONTEXT(64)] Device supports multiple RSS contexts.

\item[VIRTIO_NET_F_GUEST_UDP_TUNNEL_GSO (65)] Driver can receive GSO packets
  carried by a UDP tunnel.

\item[VIRTIO_NET_F_GUEST_UDP_TUNNEL_GSO_CSUM (66)] Driver handles packets
  carried by a UDP tunnel with partial csum for the outer header.

\item[VIRTIO_NET_F_HOST_UDP_TUNNEL_GSO (67)] Device can receive GSO packets
  carried by a UDP tunnel.

\item[VIRTIO_NET_F_HOST_UDP_TUNNEL_GSO_CSUM (68)] Device handles packets
  carried by a UDP tunnel with partial csum for the outer header.
\end{description}

\subsubsection{Feature bit requirements}\label{sec:Device Types / Network Device / Feature bits / Feature bit requirements}

Some networking feature bits require other networking feature bits
(see \ref{drivernormative:Basic Facilities of a Virtio Device / Feature Bits}):

\begin{description}
\item[VIRTIO_NET_F_GUEST_TSO4] Requires VIRTIO_NET_F_GUEST_CSUM.
\item[VIRTIO_NET_F_GUEST_TSO6] Requires VIRTIO_NET_F_GUEST_CSUM.
\item[VIRTIO_NET_F_GUEST_ECN] Requires VIRTIO_NET_F_GUEST_TSO4 or VIRTIO_NET_F_GUEST_TSO6.
\item[VIRTIO_NET_F_GUEST_UFO] Requires VIRTIO_NET_F_GUEST_CSUM.
\item[VIRTIO_NET_F_GUEST_USO4] Requires VIRTIO_NET_F_GUEST_CSUM.
\item[VIRTIO_NET_F_GUEST_USO6] Requires VIRTIO_NET_F_GUEST_CSUM.
\item[VIRTIO_NET_F_GUEST_UDP_TUNNEL_GSO] Requires VIRTIO_NET_F_GUEST_TSO4, VIRTIO_NET_F_GUEST_TSO6,
   VIRTIO_NET_F_GUEST_USO4 and VIRTIO_NET_F_GUEST_USO6.
\item[VIRTIO_NET_F_GUEST_UDP_TUNNEL_GSO_CSUM] Requires VIRTIO_NET_F_GUEST_UDP_TUNNEL_GSO

\item[VIRTIO_NET_F_HOST_TSO4] Requires VIRTIO_NET_F_CSUM.
\item[VIRTIO_NET_F_HOST_TSO6] Requires VIRTIO_NET_F_CSUM.
\item[VIRTIO_NET_F_HOST_ECN] Requires VIRTIO_NET_F_HOST_TSO4 or VIRTIO_NET_F_HOST_TSO6.
\item[VIRTIO_NET_F_HOST_UFO] Requires VIRTIO_NET_F_CSUM.
\item[VIRTIO_NET_F_HOST_USO] Requires VIRTIO_NET_F_CSUM.
\item[VIRTIO_NET_F_HOST_UDP_TUNNEL_GSO] Requires VIRTIO_NET_F_HOST_TSO4, VIRTIO_NET_F_HOST_TSO6
   and VIRTIO_NET_F_HOST_USO.
\item[VIRTIO_NET_F_HOST_UDP_TUNNEL_GSO_CSUM] Requires VIRTIO_NET_F_HOST_UDP_TUNNEL_GSO

\item[VIRTIO_NET_F_CTRL_RX] Requires VIRTIO_NET_F_CTRL_VQ.
\item[VIRTIO_NET_F_CTRL_VLAN] Requires VIRTIO_NET_F_CTRL_VQ.
\item[VIRTIO_NET_F_GUEST_ANNOUNCE] Requires VIRTIO_NET_F_CTRL_VQ.
\item[VIRTIO_NET_F_MQ] Requires VIRTIO_NET_F_CTRL_VQ.
\item[VIRTIO_NET_F_CTRL_MAC_ADDR] Requires VIRTIO_NET_F_CTRL_VQ.
\item[VIRTIO_NET_F_NOTF_COAL] Requires VIRTIO_NET_F_CTRL_VQ.
\item[VIRTIO_NET_F_RSC_EXT] Requires VIRTIO_NET_F_HOST_TSO4 or VIRTIO_NET_F_HOST_TSO6.
\item[VIRTIO_NET_F_RSS] Requires VIRTIO_NET_F_CTRL_VQ.
\item[VIRTIO_NET_F_VQ_NOTF_COAL] Requires VIRTIO_NET_F_CTRL_VQ.
\item[VIRTIO_NET_F_HASH_TUNNEL] Requires VIRTIO_NET_F_CTRL_VQ along with VIRTIO_NET_F_RSS or VIRTIO_NET_F_HASH_REPORT.
\item[VIRTIO_NET_F_RSS_CONTEXT] Requires VIRTIO_NET_F_CTRL_VQ and VIRTIO_NET_F_RSS.
\end{description}

\begin{note}
The dependency between UDP_TUNNEL_GSO_CSUM and UDP_TUNNEL_GSO is intentionally
in the opposite direction with respect to the plain GSO features and the plain
checksum offload because UDP tunnel checksum offload gives very little gain
for non GSO packets and is quite complex to implement in H/W.
\end{note}

\subsubsection{Legacy Interface: Feature bits}\label{sec:Device Types / Network Device / Feature bits / Legacy Interface: Feature bits}
\begin{description}
\item[VIRTIO_NET_F_GSO (6)] Device handles packets with any GSO type. This was supposed to indicate segmentation offload support, but
upon further investigation it became clear that multiple bits were needed.
\item[VIRTIO_NET_F_GUEST_RSC4 (41)] Device coalesces TCPIP v4 packets. This was implemented by hypervisor patch for certification
purposes and current Windows driver depends on it. It will not function if virtio-net device reports this feature.
\item[VIRTIO_NET_F_GUEST_RSC6 (42)] Device coalesces TCPIP v6 packets. Similar to VIRTIO_NET_F_GUEST_RSC4.
\end{description}

\subsection{Device configuration layout}\label{sec:Device Types / Network Device / Device configuration layout}
\label{sec:Device Types / Block Device / Feature bits / Device configuration layout}

The network device has the following device configuration layout.
All of the device configuration fields are read-only for the driver.

\begin{lstlisting}
struct virtio_net_config {
        u8 mac[6];
        le16 status;
        le16 max_virtqueue_pairs;
        le16 mtu;
        le32 speed;
        u8 duplex;
        u8 rss_max_key_size;
        le16 rss_max_indirection_table_length;
        le32 supported_hash_types;
        le32 supported_tunnel_types;
};
\end{lstlisting}

The \field{mac} address field always exists (although it is only
valid if VIRTIO_NET_F_MAC is set).

The \field{status} only exists if VIRTIO_NET_F_STATUS is set.
Two bits are currently defined for the status field: VIRTIO_NET_S_LINK_UP
and VIRTIO_NET_S_ANNOUNCE.

\begin{lstlisting}
#define VIRTIO_NET_S_LINK_UP     1
#define VIRTIO_NET_S_ANNOUNCE    2
\end{lstlisting}

The following field, \field{max_virtqueue_pairs} only exists if
VIRTIO_NET_F_MQ or VIRTIO_NET_F_RSS is set. This field specifies the maximum number
of each of transmit and receive virtqueues (receiveq1\ldots receiveqN
and transmitq1\ldots transmitqN respectively) that can be configured once at least one of these features
is negotiated.

The following field, \field{mtu} only exists if VIRTIO_NET_F_MTU
is set. This field specifies the maximum MTU for the driver to
use.

The following two fields, \field{speed} and \field{duplex}, only
exist if VIRTIO_NET_F_SPEED_DUPLEX is set.

\field{speed} contains the device speed, in units of 1 MBit per
second, 0 to 0x7fffffff, or 0xffffffff for unknown speed.

\field{duplex} has the values of 0x01 for full duplex, 0x00 for
half duplex and 0xff for unknown duplex state.

Both \field{speed} and \field{duplex} can change, thus the driver
is expected to re-read these values after receiving a
configuration change notification.

The following field, \field{rss_max_key_size} only exists if VIRTIO_NET_F_RSS or VIRTIO_NET_F_HASH_REPORT is set.
It specifies the maximum supported length of RSS key in bytes.

The following field, \field{rss_max_indirection_table_length} only exists if VIRTIO_NET_F_RSS is set.
It specifies the maximum number of 16-bit entries in RSS indirection table.

The next field, \field{supported_hash_types} only exists if the device supports hash calculation,
i.e. if VIRTIO_NET_F_RSS or VIRTIO_NET_F_HASH_REPORT is set.

Field \field{supported_hash_types} contains the bitmask of supported hash types.
See \ref{sec:Device Types / Network Device / Device Operation / Processing of Incoming Packets / Hash calculation for incoming packets / Supported/enabled hash types} for details of supported hash types.

Field \field{supported_tunnel_types} only exists if the device supports inner header hash, i.e. if VIRTIO_NET_F_HASH_TUNNEL is set.

Field \field{supported_tunnel_types} contains the bitmask of encapsulation types supported by the device for inner header hash.
Encapsulation types are defined in \ref{sec:Device Types / Network Device / Device Operation / Processing of Incoming Packets /
Hash calculation for incoming packets / Encapsulation types supported/enabled for inner header hash}.

\devicenormative{\subsubsection}{Device configuration layout}{Device Types / Network Device / Device configuration layout}

The device MUST set \field{max_virtqueue_pairs} to between 1 and 0x8000 inclusive,
if it offers VIRTIO_NET_F_MQ.

The device MUST set \field{mtu} to between 68 and 65535 inclusive,
if it offers VIRTIO_NET_F_MTU.

The device SHOULD set \field{mtu} to at least 1280, if it offers
VIRTIO_NET_F_MTU.

The device MUST NOT modify \field{mtu} once it has been set.

The device MUST NOT pass received packets that exceed \field{mtu} (plus low
level ethernet header length) size with \field{gso_type} NONE or ECN
after VIRTIO_NET_F_MTU has been successfully negotiated.

The device MUST forward transmitted packets of up to \field{mtu} (plus low
level ethernet header length) size with \field{gso_type} NONE or ECN, and do
so without fragmentation, after VIRTIO_NET_F_MTU has been successfully
negotiated.

The device MUST set \field{rss_max_key_size} to at least 40, if it offers
VIRTIO_NET_F_RSS or VIRTIO_NET_F_HASH_REPORT.

The device MUST set \field{rss_max_indirection_table_length} to at least 128, if it offers
VIRTIO_NET_F_RSS.

If the driver negotiates the VIRTIO_NET_F_STANDBY feature, the device MAY act
as a standby device for a primary device with the same MAC address.

If VIRTIO_NET_F_SPEED_DUPLEX has been negotiated, \field{speed}
MUST contain the device speed, in units of 1 MBit per second, 0 to
0x7ffffffff, or 0xfffffffff for unknown.

If VIRTIO_NET_F_SPEED_DUPLEX has been negotiated, \field{duplex}
MUST have the values of 0x00 for full duplex, 0x01 for half
duplex, or 0xff for unknown.

If VIRTIO_NET_F_SPEED_DUPLEX and VIRTIO_NET_F_STATUS have both
been negotiated, the device SHOULD NOT change the \field{speed} and
\field{duplex} fields as long as VIRTIO_NET_S_LINK_UP is set in
the \field{status}.

The device SHOULD NOT offer VIRTIO_NET_F_HASH_REPORT if it
does not offer VIRTIO_NET_F_CTRL_VQ.

The device SHOULD NOT offer VIRTIO_NET_F_CTRL_RX_EXTRA if it
does not offer VIRTIO_NET_F_CTRL_VQ.

\drivernormative{\subsubsection}{Device configuration layout}{Device Types / Network Device / Device configuration layout}

The driver MUST NOT write to any of the device configuration fields.

A driver SHOULD negotiate VIRTIO_NET_F_MAC if the device offers it.
If the driver negotiates the VIRTIO_NET_F_MAC feature, the driver MUST set
the physical address of the NIC to \field{mac}.  Otherwise, it SHOULD
use a locally-administered MAC address (see \hyperref[intro:IEEE 802]{IEEE 802},
``9.2 48-bit universal LAN MAC addresses'').

If the driver does not negotiate the VIRTIO_NET_F_STATUS feature, it SHOULD
assume the link is active, otherwise it SHOULD read the link status from
the bottom bit of \field{status}.

A driver SHOULD negotiate VIRTIO_NET_F_MTU if the device offers it.

If the driver negotiates VIRTIO_NET_F_MTU, it MUST supply enough receive
buffers to receive at least one receive packet of size \field{mtu} (plus low
level ethernet header length) with \field{gso_type} NONE or ECN.

If the driver negotiates VIRTIO_NET_F_MTU, it MUST NOT transmit packets of
size exceeding the value of \field{mtu} (plus low level ethernet header length)
with \field{gso_type} NONE or ECN.

A driver SHOULD negotiate the VIRTIO_NET_F_STANDBY feature if the device offers it.

If VIRTIO_NET_F_SPEED_DUPLEX has been negotiated,
the driver MUST treat any value of \field{speed} above
0x7fffffff as well as any value of \field{duplex} not
matching 0x00 or 0x01 as an unknown value.

If VIRTIO_NET_F_SPEED_DUPLEX has been negotiated, the driver
SHOULD re-read \field{speed} and \field{duplex} after a
configuration change notification.

A driver SHOULD NOT negotiate VIRTIO_NET_F_HASH_REPORT if it
does not negotiate VIRTIO_NET_F_CTRL_VQ.

A driver SHOULD NOT negotiate VIRTIO_NET_F_CTRL_RX_EXTRA if it
does not negotiate VIRTIO_NET_F_CTRL_VQ.

\subsubsection{Legacy Interface: Device configuration layout}\label{sec:Device Types / Network Device / Device configuration layout / Legacy Interface: Device configuration layout}
\label{sec:Device Types / Block Device / Feature bits / Device configuration layout / Legacy Interface: Device configuration layout}
When using the legacy interface, transitional devices and drivers
MUST format \field{status} and
\field{max_virtqueue_pairs} in struct virtio_net_config
according to the native endian of the guest rather than
(necessarily when not using the legacy interface) little-endian.

When using the legacy interface, \field{mac} is driver-writable
which provided a way for drivers to update the MAC without
negotiating VIRTIO_NET_F_CTRL_MAC_ADDR.

\subsection{Device Initialization}\label{sec:Device Types / Network Device / Device Initialization}

A driver would perform a typical initialization routine like so:

\begin{enumerate}
\item Identify and initialize the receive and
  transmission virtqueues, up to N of each kind. If
  VIRTIO_NET_F_MQ feature bit is negotiated,
  N=\field{max_virtqueue_pairs}, otherwise identify N=1.

\item If the VIRTIO_NET_F_CTRL_VQ feature bit is negotiated,
  identify the control virtqueue.

\item Fill the receive queues with buffers: see \ref{sec:Device Types / Network Device / Device Operation / Setting Up Receive Buffers}.

\item Even with VIRTIO_NET_F_MQ, only receiveq1, transmitq1 and
  controlq are used by default.  The driver would send the
  VIRTIO_NET_CTRL_MQ_VQ_PAIRS_SET command specifying the
  number of the transmit and receive queues to use.

\item If the VIRTIO_NET_F_MAC feature bit is set, the configuration
  space \field{mac} entry indicates the ``physical'' address of the
  device, otherwise the driver would typically generate a random
  local MAC address.

\item If the VIRTIO_NET_F_STATUS feature bit is negotiated, the link
  status comes from the bottom bit of \field{status}.
  Otherwise, the driver assumes it's active.

\item A performant driver would indicate that it will generate checksumless
  packets by negotiating the VIRTIO_NET_F_CSUM feature.

\item If that feature is negotiated, a driver can use TCP segmentation or UDP
  segmentation/fragmentation offload by negotiating the VIRTIO_NET_F_HOST_TSO4 (IPv4
  TCP), VIRTIO_NET_F_HOST_TSO6 (IPv6 TCP), VIRTIO_NET_F_HOST_UFO
  (UDP fragmentation) and VIRTIO_NET_F_HOST_USO (UDP segmentation) features.

\item If the VIRTIO_NET_F_HOST_TSO6, VIRTIO_NET_F_HOST_TSO4 and VIRTIO_NET_F_HOST_USO
  segmentation features are negotiated, a driver can
  use TCP segmentation or UDP segmentation on top of UDP encapsulation
  offload, when the outer header does not require checksumming - e.g.
  the outer UDP checksum is zero - by negotiating the
  VIRTIO_NET_F_HOST_UDP_TUNNEL_GSO feature.
  GSO over UDP tunnels packets carry two sets of headers: the outer ones
  and the inner ones. The outer transport protocol is UDP, the inner
  could be either TCP or UDP. Only a single level of encapsulation
  offload is supported.

\item If VIRTIO_NET_F_HOST_UDP_TUNNEL_GSO is negotiated, a driver can
  additionally use TCP segmentation or UDP segmentation on top of UDP
  encapsulation with the outer header requiring checksum offload,
  negotiating the VIRTIO_NET_F_HOST_UDP_TUNNEL_GSO_CSUM feature.

\item The converse features are also available: a driver can save
  the virtual device some work by negotiating these features.\note{For example, a network packet transported between two guests on
the same system might not need checksumming at all, nor segmentation,
if both guests are amenable.}
   The VIRTIO_NET_F_GUEST_CSUM feature indicates that partially
  checksummed packets can be received, and if it can do that then
  the VIRTIO_NET_F_GUEST_TSO4, VIRTIO_NET_F_GUEST_TSO6,
  VIRTIO_NET_F_GUEST_UFO, VIRTIO_NET_F_GUEST_ECN, VIRTIO_NET_F_GUEST_USO4,
  VIRTIO_NET_F_GUEST_USO6 VIRTIO_NET_F_GUEST_UDP_TUNNEL_GSO and
  VIRTIO_NET_F_GUEST_UDP_TUNNEL_GSO_CSUM are the input equivalents of
  the features described above.
  See \ref{sec:Device Types / Network Device / Device Operation /
Setting Up Receive Buffers}~\nameref{sec:Device Types / Network
Device / Device Operation / Setting Up Receive Buffers} and
\ref{sec:Device Types / Network Device / Device Operation /
Processing of Incoming Packets}~\nameref{sec:Device Types /
Network Device / Device Operation / Processing of Incoming Packets} below.
\end{enumerate}

A truly minimal driver would only accept VIRTIO_NET_F_MAC and ignore
everything else.

\subsection{Device and driver capabilities}\label{sec:Device Types / Network Device / Device and driver capabilities}

The network device has the following capabilities.

\begin{tabularx}{\textwidth}{ |l||l|X| }
\hline
Identifier & Name & Description \\
\hline \hline
0x0800 & \hyperref[par:Device Types / Network Device / Device Operation / Flow filter / Device and driver capabilities / VIRTIO-NET-FF-RESOURCE-CAP]{VIRTIO_NET_FF_RESOURCE_CAP} & Flow filter resource capability \\
\hline
0x0801 & \hyperref[par:Device Types / Network Device / Device Operation / Flow filter / Device and driver capabilities / VIRTIO-NET-FF-SELECTOR-CAP]{VIRTIO_NET_FF_SELECTOR_CAP} & Flow filter classifier capability \\
\hline
0x0802 & \hyperref[par:Device Types / Network Device / Device Operation / Flow filter / Device and driver capabilities / VIRTIO-NET-FF-ACTION-CAP]{VIRTIO_NET_FF_ACTION_CAP} & Flow filter action capability \\
\hline
\end{tabularx}

\subsection{Device resource objects}\label{sec:Device Types / Network Device / Device resource objects}

The network device has the following resource objects.

\begin{tabularx}{\textwidth}{ |l||l|X| }
\hline
type & Name & Description \\
\hline \hline
0x0200 & \hyperref[par:Device Types / Network Device / Device Operation / Flow filter / Resource objects / VIRTIO-NET-RESOURCE-OBJ-FF-GROUP]{VIRTIO_NET_RESOURCE_OBJ_FF_GROUP} & Flow filter group resource object \\
\hline
0x0201 & \hyperref[par:Device Types / Network Device / Device Operation / Flow filter / Resource objects / VIRTIO-NET-RESOURCE-OBJ-FF-CLASSIFIER]{VIRTIO_NET_RESOURCE_OBJ_FF_CLASSIFIER} & Flow filter mask object \\
\hline
0x0202 & \hyperref[par:Device Types / Network Device / Device Operation / Flow filter / Resource objects / VIRTIO-NET-RESOURCE-OBJ-FF-RULE]{VIRTIO_NET_RESOURCE_OBJ_FF_RULE} & Flow filter rule object \\
\hline
\end{tabularx}

\subsection{Device parts}\label{sec:Device Types / Network Device / Device parts}

Network device parts represent the configuration done by the driver using control
virtqueue commands. Network device part is in the format of
\field{struct virtio_dev_part}.

\begin{tabularx}{\textwidth}{ |l||l|X| }
\hline
Type & Name & Description \\
\hline \hline
0x200 & VIRTIO_NET_DEV_PART_CVQ_CFG_PART & Represents device configuration done through a control virtqueue command, see \ref{sec:Device Types / Network Device / Device parts / VIRTIO-NET-DEV-PART-CVQ-CFG-PART} \\
\hline
0x201 - 0x5FF & - & reserved for future \\
\hline
\hline
\end{tabularx}

\subsubsection{VIRTIO_NET_DEV_PART_CVQ_CFG_PART}\label{sec:Device Types / Network Device / Device parts / VIRTIO-NET-DEV-PART-CVQ-CFG-PART}

For VIRTIO_NET_DEV_PART_CVQ_CFG_PART, \field{part_type} is set to 0x200. The
VIRTIO_NET_DEV_PART_CVQ_CFG_PART part indicates configuration performed by the
driver using a control virtqueue command.

\begin{lstlisting}
struct virtio_net_dev_part_cvq_selector {
        u8 class;
        u8 command;
        u8 reserved[6];
};
\end{lstlisting}

There is one device part of type VIRTIO_NET_DEV_PART_CVQ_CFG_PART for each
individual configuration. Each part is identified by a unique selector value.
The selector, \field{device_type_raw}, is in the format
\field{struct virtio_net_dev_part_cvq_selector}.

The selector consists of two fields: \field{class} and \field{command}. These
fields correspond to the \field{class} and \field{command} defined in
\field{struct virtio_net_ctrl}, as described in the relevant sections of
\ref{sec:Device Types / Network Device / Device Operation / Control Virtqueue}.

The value corresponding to each part’s selector follows the same format as the
respective \field{command-specific-data} described in the relevant sections of
\ref{sec:Device Types / Network Device / Device Operation / Control Virtqueue}.

For example, when the \field{class} is VIRTIO_NET_CTRL_MAC, the \field{command}
can be either VIRTIO_NET_CTRL_MAC_TABLE_SET or VIRTIO_NET_CTRL_MAC_ADDR_SET;
when \field{command} is set to VIRTIO_NET_CTRL_MAC_TABLE_SET, \field{value}
is in the format of \field{struct virtio_net_ctrl_mac}.

Supported selectors are listed in the table:

\begin{tabularx}{\textwidth}{ |l|X| }
\hline
Class selector & Command selector \\
\hline \hline
VIRTIO_NET_CTRL_RX & VIRTIO_NET_CTRL_RX_PROMISC \\
\hline
VIRTIO_NET_CTRL_RX & VIRTIO_NET_CTRL_RX_ALLMULTI \\
\hline
VIRTIO_NET_CTRL_RX & VIRTIO_NET_CTRL_RX_ALLUNI \\
\hline
VIRTIO_NET_CTRL_RX & VIRTIO_NET_CTRL_RX_NOMULTI \\
\hline
VIRTIO_NET_CTRL_RX & VIRTIO_NET_CTRL_RX_NOUNI \\
\hline
VIRTIO_NET_CTRL_RX & VIRTIO_NET_CTRL_RX_NOBCAST \\
\hline
VIRTIO_NET_CTRL_MAC & VIRTIO_NET_CTRL_MAC_TABLE_SET \\
\hline
VIRTIO_NET_CTRL_MAC & VIRTIO_NET_CTRL_MAC_ADDR_SET \\
\hline
VIRTIO_NET_CTRL_VLAN & VIRTIO_NET_CTRL_VLAN_ADD \\
\hline
VIRTIO_NET_CTRL_ANNOUNCE & VIRTIO_NET_CTRL_ANNOUNCE_ACK \\
\hline
VIRTIO_NET_CTRL_MQ & VIRTIO_NET_CTRL_MQ_VQ_PAIRS_SET \\
\hline
VIRTIO_NET_CTRL_MQ & VIRTIO_NET_CTRL_MQ_RSS_CONFIG \\
\hline
VIRTIO_NET_CTRL_MQ & VIRTIO_NET_CTRL_MQ_HASH_CONFIG \\
\hline
\hline
\end{tabularx}

For command selector VIRTIO_NET_CTRL_VLAN_ADD, device part consists of a whole
VLAN table.

\field{reserved} is reserved and set to zero.

\subsection{Device Operation}\label{sec:Device Types / Network Device / Device Operation}

Packets are transmitted by placing them in the
transmitq1\ldots transmitqN, and buffers for incoming packets are
placed in the receiveq1\ldots receiveqN. In each case, the packet
itself is preceded by a header:

\begin{lstlisting}
struct virtio_net_hdr {
#define VIRTIO_NET_HDR_F_NEEDS_CSUM    1
#define VIRTIO_NET_HDR_F_DATA_VALID    2
#define VIRTIO_NET_HDR_F_RSC_INFO      4
#define VIRTIO_NET_HDR_F_UDP_TUNNEL_CSUM 8
        u8 flags;
#define VIRTIO_NET_HDR_GSO_NONE        0
#define VIRTIO_NET_HDR_GSO_TCPV4       1
#define VIRTIO_NET_HDR_GSO_UDP         3
#define VIRTIO_NET_HDR_GSO_TCPV6       4
#define VIRTIO_NET_HDR_GSO_UDP_L4      5
#define VIRTIO_NET_HDR_GSO_UDP_TUNNEL_IPV4 0x20
#define VIRTIO_NET_HDR_GSO_UDP_TUNNEL_IPV6 0x40
#define VIRTIO_NET_HDR_GSO_ECN      0x80
        u8 gso_type;
        le16 hdr_len;
        le16 gso_size;
        le16 csum_start;
        le16 csum_offset;
        le16 num_buffers;
        le32 hash_value;        (Only if VIRTIO_NET_F_HASH_REPORT negotiated)
        le16 hash_report;       (Only if VIRTIO_NET_F_HASH_REPORT negotiated)
        le16 padding_reserved;  (Only if VIRTIO_NET_F_HASH_REPORT negotiated)
        le16 outer_th_offset    (Only if VIRTIO_NET_F_HOST_UDP_TUNNEL_GSO or VIRTIO_NET_F_GUEST_UDP_TUNNEL_GSO negotiated)
        le16 inner_nh_offset;   (Only if VIRTIO_NET_F_HOST_UDP_TUNNEL_GSO or VIRTIO_NET_F_GUEST_UDP_TUNNEL_GSO negotiated)
};
\end{lstlisting}

The controlq is used to control various device features described further in
section \ref{sec:Device Types / Network Device / Device Operation / Control Virtqueue}.

\subsubsection{Legacy Interface: Device Operation}\label{sec:Device Types / Network Device / Device Operation / Legacy Interface: Device Operation}
When using the legacy interface, transitional devices and drivers
MUST format the fields in \field{struct virtio_net_hdr}
according to the native endian of the guest rather than
(necessarily when not using the legacy interface) little-endian.

The legacy driver only presented \field{num_buffers} in the \field{struct virtio_net_hdr}
when VIRTIO_NET_F_MRG_RXBUF was negotiated; without that feature the
structure was 2 bytes shorter.

When using the legacy interface, the driver SHOULD ignore the
used length for the transmit queues
and the controlq queue.
\begin{note}
Historically, some devices put
the total descriptor length there, even though no data was
actually written.
\end{note}

\subsubsection{Packet Transmission}\label{sec:Device Types / Network Device / Device Operation / Packet Transmission}

Transmitting a single packet is simple, but varies depending on
the different features the driver negotiated.

\begin{enumerate}
\item The driver can send a completely checksummed packet.  In this case,
  \field{flags} will be zero, and \field{gso_type} will be VIRTIO_NET_HDR_GSO_NONE.

\item If the driver negotiated VIRTIO_NET_F_CSUM, it can skip
  checksumming the packet:
  \begin{itemize}
  \item \field{flags} has the VIRTIO_NET_HDR_F_NEEDS_CSUM set,

  \item \field{csum_start} is set to the offset within the packet to begin checksumming,
    and

  \item \field{csum_offset} indicates how many bytes after the csum_start the
    new (16 bit ones' complement) checksum is placed by the device.

  \item The TCP checksum field in the packet is set to the sum
    of the TCP pseudo header, so that replacing it by the ones'
    complement checksum of the TCP header and body will give the
    correct result.
  \end{itemize}

\begin{note}
For example, consider a partially checksummed TCP (IPv4) packet.
It will have a 14 byte ethernet header and 20 byte IP header
followed by the TCP header (with the TCP checksum field 16 bytes
into that header). \field{csum_start} will be 14+20 = 34 (the TCP
checksum includes the header), and \field{csum_offset} will be 16.
If the given packet has the VIRTIO_NET_HDR_GSO_UDP_TUNNEL_IPV4 bit or the
VIRTIO_NET_HDR_GSO_UDP_TUNNEL_IPV6 bit set,
the above checksum fields refer to the inner header checksum, see
the example below.
\end{note}

\item If the driver negotiated
  VIRTIO_NET_F_HOST_TSO4, TSO6, USO or UFO, and the packet requires
  TCP segmentation, UDP segmentation or fragmentation, then \field{gso_type}
  is set to VIRTIO_NET_HDR_GSO_TCPV4, TCPV6, UDP_L4 or UDP.
  (Otherwise, it is set to VIRTIO_NET_HDR_GSO_NONE). In this
  case, packets larger than 1514 bytes can be transmitted: the
  metadata indicates how to replicate the packet header to cut it
  into smaller packets. The other gso fields are set:

  \begin{itemize}
  \item If the VIRTIO_NET_F_GUEST_HDRLEN feature has been negotiated,
    \field{hdr_len} indicates the header length that needs to be replicated
    for each packet. It's the number of bytes from the beginning of the packet
    to the beginning of the transport payload.
    If the \field{gso_type} has the VIRTIO_NET_HDR_GSO_UDP_TUNNEL_IPV4 bit or
    VIRTIO_NET_HDR_GSO_UDP_TUNNEL_IPV6 bit set, \field{hdr_len} accounts for
    all the headers up to and including the inner transport.
    Otherwise, if the VIRTIO_NET_F_GUEST_HDRLEN feature has not been negotiated,
    \field{hdr_len} is a hint to the device as to how much of the header
    needs to be kept to copy into each packet, usually set to the
    length of the headers, including the transport header\footnote{Due to various bugs in implementations, this field is not useful
as a guarantee of the transport header size.
}.

  \begin{note}
  Some devices benefit from knowledge of the exact header length.
  \end{note}

  \item \field{gso_size} is the maximum size of each packet beyond that
    header (ie. MSS).

  \item If the driver negotiated the VIRTIO_NET_F_HOST_ECN feature,
    the VIRTIO_NET_HDR_GSO_ECN bit in \field{gso_type}
    indicates that the TCP packet has the ECN bit set\footnote{This case is not handled by some older hardware, so is called out
specifically in the protocol.}.
   \end{itemize}

\item If the driver negotiated the VIRTIO_NET_F_HOST_UDP_TUNNEL_GSO feature and the
  \field{gso_type} has the VIRTIO_NET_HDR_GSO_UDP_TUNNEL_IPV4 bit or
  VIRTIO_NET_HDR_GSO_UDP_TUNNEL_IPV6 bit set, the GSO protocol is encapsulated
  in a UDP tunnel.
  If the outer UDP header requires checksumming, the driver must have
  additionally negotiated the VIRTIO_NET_F_HOST_UDP_TUNNEL_GSO_CSUM feature
  and offloaded the outer checksum accordingly, otherwise
  the outer UDP header must not require checksum validation, i.e. the outer
  UDP checksum must be positive zero (0x0) as defined in UDP RFC 768.
  The other tunnel-related fields indicate how to replicate the packet
  headers to cut it into smaller packets:

  \begin{itemize}
  \item \field{outer_th_offset} field indicates the outer transport header within
      the packet. This field differs from \field{csum_start} as the latter
      points to the inner transport header within the packet.

  \item \field{inner_nh_offset} field indicates the inner network header within
      the packet.
  \end{itemize}

\begin{note}
For example, consider a partially checksummed TCP (IPv4) packet carried over a
Geneve UDP tunnel (again IPv4) with no tunnel options. The
only relevant variable related to the tunnel type is the tunnel header length.
The packet will have a 14 byte outer ethernet header, 20 byte outer IP header
followed by the 8 byte UDP header (with a 0 checksum value), 8 byte Geneve header,
14 byte inner ethernet header, 20 byte inner IP header
and the TCP header (with the TCP checksum field 16 bytes
into that header). \field{csum_start} will be 14+20+8+8+14+20 = 84 (the TCP
checksum includes the header), \field{csum_offset} will be 16.
\field{inner_nh_offset} will be 14+20+8+8+14 = 62, \field{outer_th_offset} will be
14+20+8 = 42 and \field{gso_type} will be
VIRTIO_NET_HDR_GSO_TCPV4 | VIRTIO_NET_HDR_GSO_UDP_TUNNEL_IPV4 = 0x21
\end{note}

\item If the driver negotiated the VIRTIO_NET_F_HOST_UDP_TUNNEL_GSO_CSUM feature,
  the transmitted packet is a GSO one encapsulated in a UDP tunnel, and
  the outer UDP header requires checksumming, the driver can skip checksumming
  the outer header:

  \begin{itemize}
  \item \field{flags} has the VIRTIO_NET_HDR_F_UDP_TUNNEL_CSUM set,

  \item The outer UDP checksum field in the packet is set to the sum
    of the UDP pseudo header, so that replacing it by the ones'
    complement checksum of the outer UDP header and payload will give the
    correct result.
  \end{itemize}

\item \field{num_buffers} is set to zero.  This field is unused on transmitted packets.

\item The header and packet are added as one output descriptor to the
  transmitq, and the device is notified of the new entry
  (see \ref{sec:Device Types / Network Device / Device Initialization}~\nameref{sec:Device Types / Network Device / Device Initialization}).
\end{enumerate}

\drivernormative{\paragraph}{Packet Transmission}{Device Types / Network Device / Device Operation / Packet Transmission}

For the transmit packet buffer, the driver MUST use the size of the
structure \field{struct virtio_net_hdr} same as the receive packet buffer.

The driver MUST set \field{num_buffers} to zero.

If VIRTIO_NET_F_CSUM is not negotiated, the driver MUST set
\field{flags} to zero and SHOULD supply a fully checksummed
packet to the device.

If VIRTIO_NET_F_HOST_TSO4 is negotiated, the driver MAY set
\field{gso_type} to VIRTIO_NET_HDR_GSO_TCPV4 to request TCPv4
segmentation, otherwise the driver MUST NOT set
\field{gso_type} to VIRTIO_NET_HDR_GSO_TCPV4.

If VIRTIO_NET_F_HOST_TSO6 is negotiated, the driver MAY set
\field{gso_type} to VIRTIO_NET_HDR_GSO_TCPV6 to request TCPv6
segmentation, otherwise the driver MUST NOT set
\field{gso_type} to VIRTIO_NET_HDR_GSO_TCPV6.

If VIRTIO_NET_F_HOST_UFO is negotiated, the driver MAY set
\field{gso_type} to VIRTIO_NET_HDR_GSO_UDP to request UDP
fragmentation, otherwise the driver MUST NOT set
\field{gso_type} to VIRTIO_NET_HDR_GSO_UDP.

If VIRTIO_NET_F_HOST_USO is negotiated, the driver MAY set
\field{gso_type} to VIRTIO_NET_HDR_GSO_UDP_L4 to request UDP
segmentation, otherwise the driver MUST NOT set
\field{gso_type} to VIRTIO_NET_HDR_GSO_UDP_L4.

The driver SHOULD NOT send to the device TCP packets requiring segmentation offload
which have the Explicit Congestion Notification bit set, unless the
VIRTIO_NET_F_HOST_ECN feature is negotiated, in which case the
driver MUST set the VIRTIO_NET_HDR_GSO_ECN bit in
\field{gso_type}.

If VIRTIO_NET_F_HOST_UDP_TUNNEL_GSO is negotiated, the driver MAY set
VIRTIO_NET_HDR_GSO_UDP_TUNNEL_IPV4 bit or the VIRTIO_NET_HDR_GSO_UDP_TUNNEL_IPV6 bit
in \field{gso_type} according to the inner network header protocol type
to request GSO packets over UDPv4 or UDPv6 tunnel segmentation,
otherwise the driver MUST NOT set either the
VIRTIO_NET_HDR_GSO_UDP_TUNNEL_IPV4 bit or the VIRTIO_NET_HDR_GSO_UDP_TUNNEL_IPV6 bit
in \field{gso_type}.

When requesting GSO segmentation over UDP tunnel, the driver MUST SET the
VIRTIO_NET_HDR_GSO_UDP_TUNNEL_IPV4 bit if the inner network header is IPv4, i.e. the
packet is a TCPv4 GSO one, otherwise, if the inner network header is IPv6, the driver
MUST SET the VIRTIO_NET_HDR_GSO_UDP_TUNNEL_IPV6 bit.

The driver MUST NOT send to the device GSO packets over UDP tunnel
requiring segmentation and outer UDP checksum offload, unless both the
VIRTIO_NET_F_HOST_UDP_TUNNEL_GSO and VIRTIO_NET_F_HOST_UDP_TUNNEL_GSO_CSUM features
are negotiated, in which case the driver MUST set either the
VIRTIO_NET_HDR_GSO_UDP_TUNNEL_IPV4 bit or the VIRTIO_NET_HDR_GSO_UDP_TUNNEL_IPV6
bit in the \field{gso_type} and the VIRTIO_NET_HDR_F_UDP_TUNNEL_CSUM bit in
the \field{flags}.

If VIRTIO_NET_F_HOST_UDP_TUNNEL_GSO_CSUM is not negotiated, the driver MUST not set
the VIRTIO_NET_HDR_F_UDP_TUNNEL_CSUM bit in the \field{flags} and
MUST NOT send to the device GSO packets over UDP tunnel
requiring segmentation and outer UDP checksum offload.

The driver MUST NOT set the VIRTIO_NET_HDR_GSO_UDP_TUNNEL_IPV4 bit or the
VIRTIO_NET_HDR_GSO_UDP_TUNNEL_IPV6 bit together with VIRTIO_NET_HDR_GSO_UDP, as the
latter is deprecated in favor of UDP_L4 and no new feature will support it.

The driver MUST NOT set the VIRTIO_NET_HDR_GSO_UDP_TUNNEL_IPV4 bit and the
VIRTIO_NET_HDR_GSO_UDP_TUNNEL_IPV6 bit together.

The driver MUST NOT set the VIRTIO_NET_HDR_F_UDP_TUNNEL_CSUM bit \field{flags}
without setting either the VIRTIO_NET_HDR_GSO_UDP_TUNNEL_IPV4 bit or
the VIRTIO_NET_HDR_GSO_UDP_TUNNEL_IPV6 bit in \field{gso_type}.

If the VIRTIO_NET_F_CSUM feature has been negotiated, the
driver MAY set the VIRTIO_NET_HDR_F_NEEDS_CSUM bit in
\field{flags}, if so:
\begin{enumerate}
\item the driver MUST validate the packet checksum at
	offset \field{csum_offset} from \field{csum_start} as well as all
	preceding offsets;
\begin{note}
If \field{gso_type} differs from VIRTIO_NET_HDR_GSO_NONE and the
VIRTIO_NET_HDR_GSO_UDP_TUNNEL_IPV4 bit or the VIRTIO_NET_HDR_GSO_UDP_TUNNEL_IPV6
bit are not set in \field{gso_type}, \field{csum_offset}
points to the only transport header present in the packet, and there are no
additional preceding checksums validated by VIRTIO_NET_HDR_F_NEEDS_CSUM.
\end{note}
\item the driver MUST set the packet checksum stored in the
	buffer to the TCP/UDP pseudo header;
\item the driver MUST set \field{csum_start} and
	\field{csum_offset} such that calculating a ones'
	complement checksum from \field{csum_start} up until the end of
	the packet and storing the result at offset \field{csum_offset}
	from  \field{csum_start} will result in a fully checksummed
	packet;
\end{enumerate}

If none of the VIRTIO_NET_F_HOST_TSO4, TSO6, USO or UFO options have
been negotiated, the driver MUST set \field{gso_type} to
VIRTIO_NET_HDR_GSO_NONE.

If \field{gso_type} differs from VIRTIO_NET_HDR_GSO_NONE, then
the driver MUST also set the VIRTIO_NET_HDR_F_NEEDS_CSUM bit in
\field{flags} and MUST set \field{gso_size} to indicate the
desired MSS.

If one of the VIRTIO_NET_F_HOST_TSO4, TSO6, USO or UFO options have
been negotiated:
\begin{itemize}
\item If the VIRTIO_NET_F_GUEST_HDRLEN feature has been negotiated,
	and \field{gso_type} differs from VIRTIO_NET_HDR_GSO_NONE,
	the driver MUST set \field{hdr_len} to a value equal to the length
	of the headers, including the transport header. If \field{gso_type}
	has the VIRTIO_NET_HDR_GSO_UDP_TUNNEL_IPV4 bit or the
	VIRTIO_NET_HDR_GSO_UDP_TUNNEL_IPV6 bit set, \field{hdr_len} includes
	the inner transport header.

\item If the VIRTIO_NET_F_GUEST_HDRLEN feature has not been negotiated,
	or \field{gso_type} is VIRTIO_NET_HDR_GSO_NONE,
	the driver SHOULD set \field{hdr_len} to a value
	not less than the length of the headers, including the transport
	header.
\end{itemize}

If the VIRTIO_NET_F_HOST_UDP_TUNNEL_GSO option has been negotiated, the
driver MAY set the VIRTIO_NET_HDR_GSO_UDP_TUNNEL_IPV4 bit or the
VIRTIO_NET_HDR_GSO_UDP_TUNNEL_IPV6 bit in \field{gso_type}, if so:
\begin{itemize}
\item the driver MUST set \field{outer_th_offset} to the outer UDP header
  offset and \field{inner_nh_offset} to the inner network header offset.
  The \field{csum_start} and \field{csum_offset} fields point respectively
  to the inner transport header and inner transport checksum field.
\end{itemize}

If the VIRTIO_NET_F_HOST_UDP_TUNNEL_GSO_CSUM feature has been negotiated,
and the VIRTIO_NET_HDR_GSO_UDP_TUNNEL_IPV4 bit or
VIRTIO_NET_HDR_GSO_UDP_TUNNEL_IPV6 bit in \field{gso_type} are set,
the driver MAY set the VIRTIO_NET_HDR_F_UDP_TUNNEL_CSUM bit in
\field{flags}, if so the driver MUST set the packet outer UDP header checksum
to the outer UDP pseudo header checksum.

\begin{note}
calculating a ones' complement checksum from \field{outer_th_offset}
up until the end of the packet and storing the result at offset 6
from \field{outer_th_offset} will result in a fully checksummed outer UDP packet;
\end{note}

If the VIRTIO_NET_HDR_GSO_UDP_TUNNEL_IPV4 bit or the
VIRTIO_NET_HDR_GSO_UDP_TUNNEL_IPV6 bit in \field{gso_type} are set
and the VIRTIO_NET_F_HOST_UDP_TUNNEL_GSO_CSUM feature has not
been negotiated, the
outer UDP header MUST NOT require checksum validation. That is, the
outer UDP checksum value MUST be 0 or the validated complete checksum
for such header.

\begin{note}
The valid complete checksum of the outer UDP header of individual segments
can be computed by the driver prior to segmentation only if the GSO packet
size is a multiple of \field{gso_size}, because then all segments
have the same size and thus all data included in the outer UDP
checksum is the same for every segment. These pre-computed segment
length and checksum fields are different from those of the GSO
packet.
In this scenario the outer UDP header of the GSO packet must carry the
segmented UDP packet length.
\end{note}

If the VIRTIO_NET_F_HOST_UDP_TUNNEL_GSO option has not
been negotiated, the driver MUST NOT set either the VIRTIO_NET_HDR_F_GSO_UDP_TUNNEL_IPV4
bit or the VIRTIO_NET_HDR_F_GSO_UDP_TUNNEL_IPV6 in \field{gso_type}.

If the VIRTIO_NET_F_HOST_UDP_TUNNEL_GSO_CSUM option has not been negotiated,
the driver MUST NOT set the VIRTIO_NET_HDR_F_UDP_TUNNEL_CSUM bit
in \field{flags}.

The driver SHOULD accept the VIRTIO_NET_F_GUEST_HDRLEN feature if it has
been offered, and if it's able to provide the exact header length.

The driver MUST NOT set the VIRTIO_NET_HDR_F_DATA_VALID and
VIRTIO_NET_HDR_F_RSC_INFO bits in \field{flags}.

The driver MUST NOT set the VIRTIO_NET_HDR_F_DATA_VALID bit in \field{flags}
together with the VIRTIO_NET_HDR_F_GSO_UDP_TUNNEL_IPV4 bit or the
VIRTIO_NET_HDR_F_GSO_UDP_TUNNEL_IPV6 bit in \field{gso_type}.

\devicenormative{\paragraph}{Packet Transmission}{Device Types / Network Device / Device Operation / Packet Transmission}
The device MUST ignore \field{flag} bits that it does not recognize.

If VIRTIO_NET_HDR_F_NEEDS_CSUM bit in \field{flags} is not set, the
device MUST NOT use the \field{csum_start} and \field{csum_offset}.

If one of the VIRTIO_NET_F_HOST_TSO4, TSO6, USO or UFO options have
been negotiated:
\begin{itemize}
\item If the VIRTIO_NET_F_GUEST_HDRLEN feature has been negotiated,
	and \field{gso_type} differs from VIRTIO_NET_HDR_GSO_NONE,
	the device MAY use \field{hdr_len} as the transport header size.

	\begin{note}
	Caution should be taken by the implementation so as to prevent
	a malicious driver from attacking the device by setting an incorrect hdr_len.
	\end{note}

\item If the VIRTIO_NET_F_GUEST_HDRLEN feature has not been negotiated,
	or \field{gso_type} is VIRTIO_NET_HDR_GSO_NONE,
	the device MAY use \field{hdr_len} only as a hint about the
	transport header size.
	The device MUST NOT rely on \field{hdr_len} to be correct.

	\begin{note}
	This is due to various bugs in implementations.
	\end{note}
\end{itemize}

If both the VIRTIO_NET_HDR_GSO_UDP_TUNNEL_IPV4 bit and
the VIRTIO_NET_HDR_GSO_UDP_TUNNEL_IPV6 bit in in \field{gso_type} are set,
the device MUST NOT accept the packet.

If the VIRTIO_NET_HDR_GSO_UDP_TUNNEL_IPV4 bit and the VIRTIO_NET_HDR_GSO_UDP_TUNNEL_IPV6
bit in \field{gso_type} are not set, the device MUST NOT use the
\field{outer_th_offset} and \field{inner_nh_offset}.

If either the VIRTIO_NET_HDR_GSO_UDP_TUNNEL_IPV4 bit or
the VIRTIO_NET_HDR_GSO_UDP_TUNNEL_IPV6 bit in \field{gso_type} are set, and any of
the following is true:
\begin{itemize}
\item the VIRTIO_NET_HDR_F_NEEDS_CSUM is not set in \field{flags}
\item the VIRTIO_NET_HDR_F_DATA_VALID is set in \field{flags}
\item the \field{gso_type} excluding the VIRTIO_NET_HDR_GSO_UDP_TUNNEL_IPV4
bit and the VIRTIO_NET_HDR_GSO_UDP_TUNNEL_IPV6 bit is VIRTIO_NET_HDR_GSO_NONE
\end{itemize}
the device MUST NOT accept the packet.

If the VIRTIO_NET_HDR_F_UDP_TUNNEL_CSUM bit in \field{flags} is set,
and both the bits VIRTIO_NET_HDR_GSO_UDP_TUNNEL_IPV4 and
VIRTIO_NET_HDR_GSO_UDP_TUNNEL_IPV6 in \field{gso_type} are not set,
the device MOST NOT accept the packet.

If VIRTIO_NET_HDR_F_NEEDS_CSUM is not set, the device MUST NOT
rely on the packet checksum being correct.
\paragraph{Packet Transmission Interrupt}\label{sec:Device Types / Network Device / Device Operation / Packet Transmission / Packet Transmission Interrupt}

Often a driver will suppress transmission virtqueue interrupts
and check for used packets in the transmit path of following
packets.

The normal behavior in this interrupt handler is to retrieve
used buffers from the virtqueue and free the corresponding
headers and packets.

\subsubsection{Setting Up Receive Buffers}\label{sec:Device Types / Network Device / Device Operation / Setting Up Receive Buffers}

It is generally a good idea to keep the receive virtqueue as
fully populated as possible: if it runs out, network performance
will suffer.

If the VIRTIO_NET_F_GUEST_TSO4, VIRTIO_NET_F_GUEST_TSO6,
VIRTIO_NET_F_GUEST_UFO, VIRTIO_NET_F_GUEST_USO4 or VIRTIO_NET_F_GUEST_USO6
features are used, the maximum incoming packet
will be 65589 bytes long (14 bytes of Ethernet header, plus 40 bytes of
the IPv6 header, plus 65535 bytes of maximum IPv6 payload including any
extension header), otherwise 1514 bytes.
When VIRTIO_NET_F_HASH_REPORT is not negotiated, the required receive buffer
size is either 65601 or 1526 bytes accounting for 20 bytes of
\field{struct virtio_net_hdr} followed by receive packet.
When VIRTIO_NET_F_HASH_REPORT is negotiated, the required receive buffer
size is either 65609 or 1534 bytes accounting for 12 bytes of
\field{struct virtio_net_hdr} followed by receive packet.

\drivernormative{\paragraph}{Setting Up Receive Buffers}{Device Types / Network Device / Device Operation / Setting Up Receive Buffers}

\begin{itemize}
\item If VIRTIO_NET_F_MRG_RXBUF is not negotiated:
  \begin{itemize}
    \item If VIRTIO_NET_F_GUEST_TSO4, VIRTIO_NET_F_GUEST_TSO6, VIRTIO_NET_F_GUEST_UFO,
	VIRTIO_NET_F_GUEST_USO4 or VIRTIO_NET_F_GUEST_USO6 are negotiated, the driver SHOULD populate
      the receive queue(s) with buffers of at least 65609 bytes if
      VIRTIO_NET_F_HASH_REPORT is negotiated, and of at least 65601 bytes if not.
    \item Otherwise, the driver SHOULD populate the receive queue(s)
      with buffers of at least 1534 bytes if VIRTIO_NET_F_HASH_REPORT
      is negotiated, and of at least 1526 bytes if not.
  \end{itemize}
\item If VIRTIO_NET_F_MRG_RXBUF is negotiated, each buffer MUST be at
least size of \field{struct virtio_net_hdr},
i.e. 20 bytes if VIRTIO_NET_F_HASH_REPORT is negotiated, and 12 bytes if not.
\end{itemize}

\begin{note}
Obviously each buffer can be split across multiple descriptor elements.
\end{note}

When calculating the size of \field{struct virtio_net_hdr}, the driver
MUST consider all the fields inclusive up to \field{padding_reserved},
i.e. 20 bytes if VIRTIO_NET_F_HASH_REPORT is negotiated, and 12 bytes if not.

If VIRTIO_NET_F_MQ is negotiated, each of receiveq1\ldots receiveqN
that will be used SHOULD be populated with receive buffers.

\devicenormative{\paragraph}{Setting Up Receive Buffers}{Device Types / Network Device / Device Operation / Setting Up Receive Buffers}

The device MUST set \field{num_buffers} to the number of descriptors used to
hold the incoming packet.

The device MUST use only a single descriptor if VIRTIO_NET_F_MRG_RXBUF
was not negotiated.
\begin{note}
{This means that \field{num_buffers} will always be 1
if VIRTIO_NET_F_MRG_RXBUF is not negotiated.}
\end{note}

\subsubsection{Processing of Incoming Packets}\label{sec:Device Types / Network Device / Device Operation / Processing of Incoming Packets}
\label{sec:Device Types / Network Device / Device Operation / Processing of Packets}%old label for latexdiff

When a packet is copied into a buffer in the receiveq, the
optimal path is to disable further used buffer notifications for the
receiveq and process packets until no more are found, then re-enable
them.

Processing incoming packets involves:

\begin{enumerate}
\item \field{num_buffers} indicates how many descriptors
  this packet is spread over (including this one): this will
  always be 1 if VIRTIO_NET_F_MRG_RXBUF was not negotiated.
  This allows receipt of large packets without having to allocate large
  buffers: a packet that does not fit in a single buffer can flow
  over to the next buffer, and so on. In this case, there will be
  at least \field{num_buffers} used buffers in the virtqueue, and the device
  chains them together to form a single packet in a way similar to
  how it would store it in a single buffer spread over multiple
  descriptors.
  The other buffers will not begin with a \field{struct virtio_net_hdr}.

\item If
  \field{num_buffers} is one, then the entire packet will be
  contained within this buffer, immediately following the struct
  virtio_net_hdr.
\item If the VIRTIO_NET_F_GUEST_CSUM feature was negotiated, the
  VIRTIO_NET_HDR_F_DATA_VALID bit in \field{flags} can be
  set: if so, device has validated the packet checksum.
  If the VIRTIO_NET_F_GUEST_UDP_TUNNEL_GSO_CSUM feature has been negotiated,
  and the VIRTIO_NET_HDR_F_UDP_TUNNEL_CSUM bit is set in \field{flags},
  both the outer UDP checksum and the inner transport checksum
  have been validated, otherwise only one level of checksums (the outer one
  in case of tunnels) has been validated.
\end{enumerate}

Additionally, VIRTIO_NET_F_GUEST_CSUM, TSO4, TSO6, UDP, UDP_TUNNEL
and ECN features enable receive checksum, large receive offload and ECN
support which are the input equivalents of the transmit checksum,
transmit segmentation offloading and ECN features, as described
in \ref{sec:Device Types / Network Device / Device Operation /
Packet Transmission}:
\begin{enumerate}
\item If the VIRTIO_NET_F_GUEST_TSO4, TSO6, UFO, USO4 or USO6 options were
  negotiated, then \field{gso_type} MAY be something other than
  VIRTIO_NET_HDR_GSO_NONE, and \field{gso_size} field indicates the
  desired MSS (see Packet Transmission point 2).
\item If the VIRTIO_NET_F_RSC_EXT option was negotiated (this
  implies one of VIRTIO_NET_F_GUEST_TSO4, TSO6), the
  device processes also duplicated ACK segments, reports
  number of coalesced TCP segments in \field{csum_start} field and
  number of duplicated ACK segments in \field{csum_offset} field
  and sets bit VIRTIO_NET_HDR_F_RSC_INFO in \field{flags}.
\item If the VIRTIO_NET_F_GUEST_CSUM feature was negotiated, the
  VIRTIO_NET_HDR_F_NEEDS_CSUM bit in \field{flags} can be
  set: if so, the packet checksum at offset \field{csum_offset}
  from \field{csum_start} and any preceding checksums
  have been validated.  The checksum on the packet is incomplete and
  if bit VIRTIO_NET_HDR_F_RSC_INFO is not set in \field{flags},
  then \field{csum_start} and \field{csum_offset} indicate how to calculate it
  (see Packet Transmission point 1).
\begin{note}
If \field{gso_type} differs from VIRTIO_NET_HDR_GSO_NONE and the
VIRTIO_NET_HDR_GSO_UDP_TUNNEL_IPV4 bit or the VIRTIO_NET_HDR_GSO_UDP_TUNNEL_IPV6
bit are not set, \field{csum_offset}
points to the only transport header present in the packet, and there are no
additional preceding checksums validated by VIRTIO_NET_HDR_F_NEEDS_CSUM.
\end{note}
\item If the VIRTIO_NET_F_GUEST_UDP_TUNNEL_GSO option was negotiated and
  \field{gso_type} is not VIRTIO_NET_HDR_GSO_NONE, the
  VIRTIO_NET_HDR_GSO_UDP_TUNNEL_IPV4 bit or the VIRTIO_NET_HDR_GSO_UDP_TUNNEL_IPV6
  bit MAY be set. In such case the \field{outer_th_offset} and
  \field{inner_nh_offset} fields indicate the corresponding
  headers information.
\item If the VIRTIO_NET_F_GUEST_UDP_TUNNEL_GSO_CSUM feature was
negotiated, and
  the VIRTIO_NET_HDR_GSO_UDP_TUNNEL_IPV4 bit or the VIRTIO_NET_HDR_GSO_UDP_TUNNEL_IPV6
  are set in \field{gso_type}, the VIRTIO_NET_HDR_F_UDP_TUNNEL_CSUM bit in the
  \field{flags} can be set: if so, the outer UDP checksum has been validated
  and the UDP header checksum at offset 6 from from \field{outer_th_offset}
  is set to the outer UDP pseudo header checksum.

\begin{note}
If the VIRTIO_NET_HDR_GSO_UDP_TUNNEL_IPV4 bit or VIRTIO_NET_HDR_GSO_UDP_TUNNEL_IPV6
bit are set in \field{gso_type}, the \field{csum_start} field refers to
the inner transport header offset (see Packet Transmission point 1).
If the VIRTIO_NET_HDR_F_UDP_TUNNEL_CSUM bit in \field{flags} is set both
the inner and the outer header checksums have been validated by
VIRTIO_NET_HDR_F_NEEDS_CSUM, otherwise only the inner transport header
checksum has been validated.
\end{note}
\end{enumerate}

If applicable, the device calculates per-packet hash for incoming packets as
defined in \ref{sec:Device Types / Network Device / Device Operation / Processing of Incoming Packets / Hash calculation for incoming packets}.

If applicable, the device reports hash information for incoming packets as
defined in \ref{sec:Device Types / Network Device / Device Operation / Processing of Incoming Packets / Hash reporting for incoming packets}.

\devicenormative{\paragraph}{Processing of Incoming Packets}{Device Types / Network Device / Device Operation / Processing of Incoming Packets}
\label{devicenormative:Device Types / Network Device / Device Operation / Processing of Packets}%old label for latexdiff

If VIRTIO_NET_F_MRG_RXBUF has not been negotiated, the device MUST set
\field{num_buffers} to 1.

If VIRTIO_NET_F_MRG_RXBUF has been negotiated, the device MUST set
\field{num_buffers} to indicate the number of buffers
the packet (including the header) is spread over.

If a receive packet is spread over multiple buffers, the device
MUST use all buffers but the last (i.e. the first \field{num_buffers} -
1 buffers) completely up to the full length of each buffer
supplied by the driver.

The device MUST use all buffers used by a single receive
packet together, such that at least \field{num_buffers} are
observed by driver as used.

If VIRTIO_NET_F_GUEST_CSUM is not negotiated, the device MUST set
\field{flags} to zero and SHOULD supply a fully checksummed
packet to the driver.

If VIRTIO_NET_F_GUEST_TSO4 is not negotiated, the device MUST NOT set
\field{gso_type} to VIRTIO_NET_HDR_GSO_TCPV4.

If VIRTIO_NET_F_GUEST_UDP is not negotiated, the device MUST NOT set
\field{gso_type} to VIRTIO_NET_HDR_GSO_UDP.

If VIRTIO_NET_F_GUEST_TSO6 is not negotiated, the device MUST NOT set
\field{gso_type} to VIRTIO_NET_HDR_GSO_TCPV6.

If none of VIRTIO_NET_F_GUEST_USO4 or VIRTIO_NET_F_GUEST_USO6 have been negotiated,
the device MUST NOT set \field{gso_type} to VIRTIO_NET_HDR_GSO_UDP_L4.

If VIRTIO_NET_F_GUEST_UDP_TUNNEL_GSO is not negotiated, the device MUST NOT set
either the VIRTIO_NET_HDR_GSO_UDP_TUNNEL_IPV4 bit or the
VIRTIO_NET_HDR_GSO_UDP_TUNNEL_IPV6 bit in \field{gso_type}.

If VIRTIO_NET_F_GUEST_UDP_TUNNEL_GSO_CSUM is not negotiated the device MUST NOT set
the VIRTIO_NET_HDR_F_UDP_TUNNEL_CSUM bit in \field{flags}.

The device SHOULD NOT send to the driver TCP packets requiring segmentation offload
which have the Explicit Congestion Notification bit set, unless the
VIRTIO_NET_F_GUEST_ECN feature is negotiated, in which case the
device MUST set the VIRTIO_NET_HDR_GSO_ECN bit in
\field{gso_type}.

If the VIRTIO_NET_F_GUEST_CSUM feature has been negotiated, the
device MAY set the VIRTIO_NET_HDR_F_NEEDS_CSUM bit in
\field{flags}, if so:
\begin{enumerate}
\item the device MUST validate the packet checksum at
	offset \field{csum_offset} from \field{csum_start} as well as all
	preceding offsets;
\item the device MUST set the packet checksum stored in the
	receive buffer to the TCP/UDP pseudo header;
\item the device MUST set \field{csum_start} and
	\field{csum_offset} such that calculating a ones'
	complement checksum from \field{csum_start} up until the
	end of the packet and storing the result at offset
	\field{csum_offset} from  \field{csum_start} will result in a
	fully checksummed packet;
\end{enumerate}

The device MUST NOT send to the driver GSO packets encapsulated in UDP
tunnel and requiring segmentation offload, unless the
VIRTIO_NET_F_GUEST_UDP_TUNNEL_GSO is negotiated, in which case the device MUST set
the VIRTIO_NET_HDR_GSO_UDP_TUNNEL_IPV4 bit or the VIRTIO_NET_HDR_GSO_UDP_TUNNEL_IPV6
bit in \field{gso_type} according to the inner network header protocol type,
MUST set the \field{outer_th_offset} and \field{inner_nh_offset} fields
to the corresponding header information, and the outer UDP header MUST NOT
require checksum offload.

If the VIRTIO_NET_F_GUEST_UDP_TUNNEL_GSO_CSUM feature has not been negotiated,
the device MUST NOT send the driver GSO packets encapsulated in UDP
tunnel and requiring segmentation and outer checksum offload.

If none of the VIRTIO_NET_F_GUEST_TSO4, TSO6, UFO, USO4 or USO6 options have
been negotiated, the device MUST set \field{gso_type} to
VIRTIO_NET_HDR_GSO_NONE.

If \field{gso_type} differs from VIRTIO_NET_HDR_GSO_NONE, then
the device MUST also set the VIRTIO_NET_HDR_F_NEEDS_CSUM bit in
\field{flags} MUST set \field{gso_size} to indicate the desired MSS.
If VIRTIO_NET_F_RSC_EXT was negotiated, the device MUST also
set VIRTIO_NET_HDR_F_RSC_INFO bit in \field{flags},
set \field{csum_start} to number of coalesced TCP segments and
set \field{csum_offset} to number of received duplicated ACK segments.

If VIRTIO_NET_F_RSC_EXT was not negotiated, the device MUST
not set VIRTIO_NET_HDR_F_RSC_INFO bit in \field{flags}.

If one of the VIRTIO_NET_F_GUEST_TSO4, TSO6, UFO, USO4 or USO6 options have
been negotiated, the device SHOULD set \field{hdr_len} to a value
not less than the length of the headers, including the transport
header. If \field{gso_type} has the VIRTIO_NET_HDR_GSO_UDP_TUNNEL_IPV4 bit
or the VIRTIO_NET_HDR_GSO_UDP_TUNNEL_IPV6 bit set, the referenced transport
header is the inner one.

If the VIRTIO_NET_F_GUEST_CSUM feature has been negotiated, the
device MAY set the VIRTIO_NET_HDR_F_DATA_VALID bit in
\field{flags}, if so, the device MUST validate the packet
checksum. If the VIRTIO_NET_F_GUEST_UDP_TUNNEL_GSO_CSUM feature has
been negotiated, and the VIRTIO_NET_HDR_F_UDP_TUNNEL_CSUM bit set in
\field{flags}, both the outer UDP checksum and the inner transport
checksum have been validated.
Otherwise level of checksum is validated: in case of multiple
encapsulated protocols the outermost one.

If either the VIRTIO_NET_HDR_GSO_UDP_TUNNEL_IPV4 bit or the
VIRTIO_NET_HDR_GSO_UDP_TUNNEL_IPV6 bit in \field{gso_type} are set,
the device MUST NOT set the VIRTIO_NET_HDR_F_DATA_VALID bit in
\field{flags}.

If the VIRTIO_NET_F_GUEST_UDP_TUNNEL_GSO_CSUM feature has been negotiated
and either the VIRTIO_NET_HDR_GSO_UDP_TUNNEL_IPV4 bit is set or the
VIRTIO_NET_HDR_GSO_UDP_TUNNEL_IPV6 bit is set in \field{gso_type}, the
device MAY set the VIRTIO_NET_HDR_F_UDP_TUNNEL_CSUM bit in
\field{flags}, if so the device MUST set the packet outer UDP checksum
stored in the receive buffer to the outer UDP pseudo header.

Otherwise, the VIRTIO_NET_F_GUEST_UDP_TUNNEL_GSO_CSUM feature has been
negotiated, either the VIRTIO_NET_HDR_GSO_UDP_TUNNEL_IPV4 bit is set or the
VIRTIO_NET_HDR_GSO_UDP_TUNNEL_IPV6 bit is set in \field{gso_type},
and the bit VIRTIO_NET_HDR_F_UDP_TUNNEL_CSUM is not set in
\field{flags}, the device MUST either provide a zero outer UDP header
checksum or a fully checksummed outer UDP header.

\drivernormative{\paragraph}{Processing of Incoming
Packets}{Device Types / Network Device / Device Operation /
Processing of Incoming Packets}

The driver MUST ignore \field{flag} bits that it does not recognize.

If VIRTIO_NET_HDR_F_NEEDS_CSUM bit in \field{flags} is not set or
if VIRTIO_NET_HDR_F_RSC_INFO bit \field{flags} is set, the
driver MUST NOT use the \field{csum_start} and \field{csum_offset}.

If one of the VIRTIO_NET_F_GUEST_TSO4, TSO6, UFO, USO4 or USO6 options have
been negotiated, the driver MAY use \field{hdr_len} only as a hint about the
transport header size.
The driver MUST NOT rely on \field{hdr_len} to be correct.
\begin{note}
This is due to various bugs in implementations.
\end{note}

If neither VIRTIO_NET_HDR_F_NEEDS_CSUM nor
VIRTIO_NET_HDR_F_DATA_VALID is set, the driver MUST NOT
rely on the packet checksum being correct.

If both the VIRTIO_NET_HDR_GSO_UDP_TUNNEL_IPV4 bit and
the VIRTIO_NET_HDR_GSO_UDP_TUNNEL_IPV6 bit in in \field{gso_type} are set,
the driver MUST NOT accept the packet.

If the VIRTIO_NET_HDR_GSO_UDP_TUNNEL_IPV4 bit or the VIRTIO_NET_HDR_GSO_UDP_TUNNEL_IPV6
bit in \field{gso_type} are not set, the driver MUST NOT use the
\field{outer_th_offset} and \field{inner_nh_offset}.

If either the VIRTIO_NET_HDR_GSO_UDP_TUNNEL_IPV4 bit or
the VIRTIO_NET_HDR_GSO_UDP_TUNNEL_IPV6 bit in \field{gso_type} are set, and any of
the following is true:
\begin{itemize}
\item the VIRTIO_NET_HDR_F_NEEDS_CSUM bit is not set in \field{flags}
\item the VIRTIO_NET_HDR_F_DATA_VALID bit is set in \field{flags}
\item the \field{gso_type} excluding the VIRTIO_NET_HDR_GSO_UDP_TUNNEL_IPV4
bit and the VIRTIO_NET_HDR_GSO_UDP_TUNNEL_IPV6 bit is VIRTIO_NET_HDR_GSO_NONE
\end{itemize}
the driver MUST NOT accept the packet.

If the VIRTIO_NET_HDR_F_UDP_TUNNEL_CSUM bit and the VIRTIO_NET_HDR_F_NEEDS_CSUM
bit in \field{flags} are set,
and both the bits VIRTIO_NET_HDR_GSO_UDP_TUNNEL_IPV4 and
VIRTIO_NET_HDR_GSO_UDP_TUNNEL_IPV6 in \field{gso_type} are not set,
the driver MOST NOT accept the packet.

\paragraph{Hash calculation for incoming packets}
\label{sec:Device Types / Network Device / Device Operation / Processing of Incoming Packets / Hash calculation for incoming packets}

A device attempts to calculate a per-packet hash in the following cases:
\begin{itemize}
\item The feature VIRTIO_NET_F_RSS was negotiated. The device uses the hash to determine the receive virtqueue to place incoming packets.
\item The feature VIRTIO_NET_F_HASH_REPORT was negotiated. The device reports the hash value and the hash type with the packet.
\end{itemize}

If the feature VIRTIO_NET_F_RSS was negotiated:
\begin{itemize}
\item The device uses \field{hash_types} of the virtio_net_rss_config structure as 'Enabled hash types' bitmask.
\item If additionally the feature VIRTIO_NET_F_HASH_TUNNEL was negotiated, the device uses \field{enabled_tunnel_types} of the
      virtnet_hash_tunnel structure as 'Encapsulation types enabled for inner header hash' bitmask.
\item The device uses a key as defined in \field{hash_key_data} and \field{hash_key_length} of the virtio_net_rss_config structure (see
\ref{sec:Device Types / Network Device / Device Operation / Control Virtqueue / Receive-side scaling (RSS) / Setting RSS parameters}).
\end{itemize}

If the feature VIRTIO_NET_F_RSS was not negotiated:
\begin{itemize}
\item The device uses \field{hash_types} of the virtio_net_hash_config structure as 'Enabled hash types' bitmask.
\item If additionally the feature VIRTIO_NET_F_HASH_TUNNEL was negotiated, the device uses \field{enabled_tunnel_types} of the
      virtnet_hash_tunnel structure as 'Encapsulation types enabled for inner header hash' bitmask.
\item The device uses a key as defined in \field{hash_key_data} and \field{hash_key_length} of the virtio_net_hash_config structure (see
\ref{sec:Device Types / Network Device / Device Operation / Control Virtqueue / Automatic receive steering in multiqueue mode / Hash calculation}).
\end{itemize}

Note that if the device offers VIRTIO_NET_F_HASH_REPORT, even if it supports only one pair of virtqueues, it MUST support
at least one of commands of VIRTIO_NET_CTRL_MQ class to configure reported hash parameters:
\begin{itemize}
\item If the device offers VIRTIO_NET_F_RSS, it MUST support VIRTIO_NET_CTRL_MQ_RSS_CONFIG command per
 \ref{sec:Device Types / Network Device / Device Operation / Control Virtqueue / Receive-side scaling (RSS) / Setting RSS parameters}.
\item Otherwise the device MUST support VIRTIO_NET_CTRL_MQ_HASH_CONFIG command per
 \ref{sec:Device Types / Network Device / Device Operation / Control Virtqueue / Automatic receive steering in multiqueue mode / Hash calculation}.
\end{itemize}

The per-packet hash calculation can depend on the IP packet type. See
\hyperref[intro:IP]{[IP]}, \hyperref[intro:UDP]{[UDP]} and \hyperref[intro:TCP]{[TCP]}.

\subparagraph{Supported/enabled hash types}
\label{sec:Device Types / Network Device / Device Operation / Processing of Incoming Packets / Hash calculation for incoming packets / Supported/enabled hash types}
Hash types applicable for IPv4 packets:
\begin{lstlisting}
#define VIRTIO_NET_HASH_TYPE_IPv4              (1 << 0)
#define VIRTIO_NET_HASH_TYPE_TCPv4             (1 << 1)
#define VIRTIO_NET_HASH_TYPE_UDPv4             (1 << 2)
\end{lstlisting}
Hash types applicable for IPv6 packets without extension headers
\begin{lstlisting}
#define VIRTIO_NET_HASH_TYPE_IPv6              (1 << 3)
#define VIRTIO_NET_HASH_TYPE_TCPv6             (1 << 4)
#define VIRTIO_NET_HASH_TYPE_UDPv6             (1 << 5)
\end{lstlisting}
Hash types applicable for IPv6 packets with extension headers
\begin{lstlisting}
#define VIRTIO_NET_HASH_TYPE_IP_EX             (1 << 6)
#define VIRTIO_NET_HASH_TYPE_TCP_EX            (1 << 7)
#define VIRTIO_NET_HASH_TYPE_UDP_EX            (1 << 8)
\end{lstlisting}

\subparagraph{IPv4 packets}
\label{sec:Device Types / Network Device / Device Operation / Processing of Incoming Packets / Hash calculation for incoming packets / IPv4 packets}
The device calculates the hash on IPv4 packets according to 'Enabled hash types' bitmask as follows:
\begin{itemize}
\item If VIRTIO_NET_HASH_TYPE_TCPv4 is set and the packet has
a TCP header, the hash is calculated over the following fields:
\begin{itemize}
\item Source IP address
\item Destination IP address
\item Source TCP port
\item Destination TCP port
\end{itemize}
\item Else if VIRTIO_NET_HASH_TYPE_UDPv4 is set and the
packet has a UDP header, the hash is calculated over the following fields:
\begin{itemize}
\item Source IP address
\item Destination IP address
\item Source UDP port
\item Destination UDP port
\end{itemize}
\item Else if VIRTIO_NET_HASH_TYPE_IPv4 is set, the hash is
calculated over the following fields:
\begin{itemize}
\item Source IP address
\item Destination IP address
\end{itemize}
\item Else the device does not calculate the hash
\end{itemize}

\subparagraph{IPv6 packets without extension header}
\label{sec:Device Types / Network Device / Device Operation / Processing of Incoming Packets / Hash calculation for incoming packets / IPv6 packets without extension header}
The device calculates the hash on IPv6 packets without extension
headers according to 'Enabled hash types' bitmask as follows:
\begin{itemize}
\item If VIRTIO_NET_HASH_TYPE_TCPv6 is set and the packet has
a TCPv6 header, the hash is calculated over the following fields:
\begin{itemize}
\item Source IPv6 address
\item Destination IPv6 address
\item Source TCP port
\item Destination TCP port
\end{itemize}
\item Else if VIRTIO_NET_HASH_TYPE_UDPv6 is set and the
packet has a UDPv6 header, the hash is calculated over the following fields:
\begin{itemize}
\item Source IPv6 address
\item Destination IPv6 address
\item Source UDP port
\item Destination UDP port
\end{itemize}
\item Else if VIRTIO_NET_HASH_TYPE_IPv6 is set, the hash is
calculated over the following fields:
\begin{itemize}
\item Source IPv6 address
\item Destination IPv6 address
\end{itemize}
\item Else the device does not calculate the hash
\end{itemize}

\subparagraph{IPv6 packets with extension header}
\label{sec:Device Types / Network Device / Device Operation / Processing of Incoming Packets / Hash calculation for incoming packets / IPv6 packets with extension header}
The device calculates the hash on IPv6 packets with extension
headers according to 'Enabled hash types' bitmask as follows:
\begin{itemize}
\item If VIRTIO_NET_HASH_TYPE_TCP_EX is set and the packet
has a TCPv6 header, the hash is calculated over the following fields:
\begin{itemize}
\item Home address from the home address option in the IPv6 destination options header. If the extension header is not present, use the Source IPv6 address.
\item IPv6 address that is contained in the Routing-Header-Type-2 from the associated extension header. If the extension header is not present, use the Destination IPv6 address.
\item Source TCP port
\item Destination TCP port
\end{itemize}
\item Else if VIRTIO_NET_HASH_TYPE_UDP_EX is set and the
packet has a UDPv6 header, the hash is calculated over the following fields:
\begin{itemize}
\item Home address from the home address option in the IPv6 destination options header. If the extension header is not present, use the Source IPv6 address.
\item IPv6 address that is contained in the Routing-Header-Type-2 from the associated extension header. If the extension header is not present, use the Destination IPv6 address.
\item Source UDP port
\item Destination UDP port
\end{itemize}
\item Else if VIRTIO_NET_HASH_TYPE_IP_EX is set, the hash is
calculated over the following fields:
\begin{itemize}
\item Home address from the home address option in the IPv6 destination options header. If the extension header is not present, use the Source IPv6 address.
\item IPv6 address that is contained in the Routing-Header-Type-2 from the associated extension header. If the extension header is not present, use the Destination IPv6 address.
\end{itemize}
\item Else skip IPv6 extension headers and calculate the hash as
defined for an IPv6 packet without extension headers
(see \ref{sec:Device Types / Network Device / Device Operation / Processing of Incoming Packets / Hash calculation for incoming packets / IPv6 packets without extension header}).
\end{itemize}

\paragraph{Inner Header Hash}
\label{sec:Device Types / Network Device / Device Operation / Processing of Incoming Packets / Inner Header Hash}

If VIRTIO_NET_F_HASH_TUNNEL has been negotiated, the driver can send the command
VIRTIO_NET_CTRL_HASH_TUNNEL_SET to configure the calculation of the inner header hash.

\begin{lstlisting}
struct virtnet_hash_tunnel {
    le32 enabled_tunnel_types;
};

#define VIRTIO_NET_CTRL_HASH_TUNNEL 7
 #define VIRTIO_NET_CTRL_HASH_TUNNEL_SET 0
\end{lstlisting}

Field \field{enabled_tunnel_types} contains the bitmask of encapsulation types enabled for inner header hash.
See \ref{sec:Device Types / Network Device / Device Operation / Processing of Incoming Packets /
Hash calculation for incoming packets / Encapsulation types supported/enabled for inner header hash}.

The class VIRTIO_NET_CTRL_HASH_TUNNEL has one command:
VIRTIO_NET_CTRL_HASH_TUNNEL_SET sets \field{enabled_tunnel_types} for the device using the
virtnet_hash_tunnel structure, which is read-only for the device.

Inner header hash is disabled by VIRTIO_NET_CTRL_HASH_TUNNEL_SET with \field{enabled_tunnel_types} set to 0.

Initially (before the driver sends any VIRTIO_NET_CTRL_HASH_TUNNEL_SET command) all
encapsulation types are disabled for inner header hash.

\subparagraph{Encapsulated packet}
\label{sec:Device Types / Network Device / Device Operation / Processing of Incoming Packets / Hash calculation for incoming packets / Encapsulated packet}

Multiple tunneling protocols allow encapsulating an inner, payload packet in an outer, encapsulated packet.
The encapsulated packet thus contains an outer header and an inner header, and the device calculates the
hash over either the inner header or the outer header.

If VIRTIO_NET_F_HASH_TUNNEL is negotiated and a received encapsulated packet's outer header matches one of the
encapsulation types enabled in \field{enabled_tunnel_types}, then the device uses the inner header for hash
calculations (only a single level of encapsulation is currently supported).

If VIRTIO_NET_F_HASH_TUNNEL is negotiated and a received packet's (outer) header does not match any encapsulation
types enabled in \field{enabled_tunnel_types}, then the device uses the outer header for hash calculations.

\subparagraph{Encapsulation types supported/enabled for inner header hash}
\label{sec:Device Types / Network Device / Device Operation / Processing of Incoming Packets /
Hash calculation for incoming packets / Encapsulation types supported/enabled for inner header hash}

Encapsulation types applicable for inner header hash:
\begin{lstlisting}[escapechar=|]
#define VIRTIO_NET_HASH_TUNNEL_TYPE_GRE_2784    (1 << 0) /* |\hyperref[intro:rfc2784]{[RFC2784]}| */
#define VIRTIO_NET_HASH_TUNNEL_TYPE_GRE_2890    (1 << 1) /* |\hyperref[intro:rfc2890]{[RFC2890]}| */
#define VIRTIO_NET_HASH_TUNNEL_TYPE_GRE_7676    (1 << 2) /* |\hyperref[intro:rfc7676]{[RFC7676]}| */
#define VIRTIO_NET_HASH_TUNNEL_TYPE_GRE_UDP     (1 << 3) /* |\hyperref[intro:rfc8086]{[GRE-in-UDP]}| */
#define VIRTIO_NET_HASH_TUNNEL_TYPE_VXLAN       (1 << 4) /* |\hyperref[intro:vxlan]{[VXLAN]}| */
#define VIRTIO_NET_HASH_TUNNEL_TYPE_VXLAN_GPE   (1 << 5) /* |\hyperref[intro:vxlan-gpe]{[VXLAN-GPE]}| */
#define VIRTIO_NET_HASH_TUNNEL_TYPE_GENEVE      (1 << 6) /* |\hyperref[intro:geneve]{[GENEVE]}| */
#define VIRTIO_NET_HASH_TUNNEL_TYPE_IPIP        (1 << 7) /* |\hyperref[intro:ipip]{[IPIP]}| */
#define VIRTIO_NET_HASH_TUNNEL_TYPE_NVGRE       (1 << 8) /* |\hyperref[intro:nvgre]{[NVGRE]}| */
\end{lstlisting}

\subparagraph{Advice}
Example uses of the inner header hash:
\begin{itemize}
\item Legacy tunneling protocols, lacking the outer header entropy, can use RSS with the inner header hash to
      distribute flows with identical outer but different inner headers across various queues, improving performance.
\item Identify an inner flow distributed across multiple outer tunnels.
\end{itemize}

As using the inner header hash completely discards the outer header entropy, care must be taken
if the inner header is controlled by an adversary, as the adversary can then intentionally create
configurations with insufficient entropy.

Besides disabling the inner header hash, mitigations would depend on how the hash is used. When the hash
use is limited to the RSS queue selection, the inner header hash may have quality of service (QoS) limitations.

\devicenormative{\subparagraph}{Inner Header Hash}{Device Types / Network Device / Device Operation / Control Virtqueue / Inner Header Hash}

If the (outer) header of the received packet does not match any encapsulation types enabled
in \field{enabled_tunnel_types}, the device MUST calculate the hash on the outer header.

If the device receives any bits in \field{enabled_tunnel_types} which are not set in \field{supported_tunnel_types},
it SHOULD respond to the VIRTIO_NET_CTRL_HASH_TUNNEL_SET command with VIRTIO_NET_ERR.

If the driver sets \field{enabled_tunnel_types} to 0 through VIRTIO_NET_CTRL_HASH_TUNNEL_SET or upon the device reset,
the device MUST disable the inner header hash for all encapsulation types.

\drivernormative{\subparagraph}{Inner Header Hash}{Device Types / Network Device / Device Operation / Control Virtqueue / Inner Header Hash}

The driver MUST have negotiated the VIRTIO_NET_F_HASH_TUNNEL feature when issuing the VIRTIO_NET_CTRL_HASH_TUNNEL_SET command.

The driver MUST NOT set any bits in \field{enabled_tunnel_types} which are not set in \field{supported_tunnel_types}.

The driver MUST ignore bits in \field{supported_tunnel_types} which are not documented in this specification.

\paragraph{Hash reporting for incoming packets}
\label{sec:Device Types / Network Device / Device Operation / Processing of Incoming Packets / Hash reporting for incoming packets}

If VIRTIO_NET_F_HASH_REPORT was negotiated and
 the device has calculated the hash for the packet, the device fills \field{hash_report} with the report type of calculated hash
and \field{hash_value} with the value of calculated hash.

If VIRTIO_NET_F_HASH_REPORT was negotiated but due to any reason the
hash was not calculated, the device sets \field{hash_report} to VIRTIO_NET_HASH_REPORT_NONE.

Possible values that the device can report in \field{hash_report} are defined below.
They correspond to supported hash types defined in
\ref{sec:Device Types / Network Device / Device Operation / Processing of Incoming Packets / Hash calculation for incoming packets / Supported/enabled hash types}
as follows:

VIRTIO_NET_HASH_TYPE_XXX = 1 << (VIRTIO_NET_HASH_REPORT_XXX - 1)

\begin{lstlisting}
#define VIRTIO_NET_HASH_REPORT_NONE            0
#define VIRTIO_NET_HASH_REPORT_IPv4            1
#define VIRTIO_NET_HASH_REPORT_TCPv4           2
#define VIRTIO_NET_HASH_REPORT_UDPv4           3
#define VIRTIO_NET_HASH_REPORT_IPv6            4
#define VIRTIO_NET_HASH_REPORT_TCPv6           5
#define VIRTIO_NET_HASH_REPORT_UDPv6           6
#define VIRTIO_NET_HASH_REPORT_IPv6_EX         7
#define VIRTIO_NET_HASH_REPORT_TCPv6_EX        8
#define VIRTIO_NET_HASH_REPORT_UDPv6_EX        9
\end{lstlisting}

\subsubsection{Control Virtqueue}\label{sec:Device Types / Network Device / Device Operation / Control Virtqueue}

The driver uses the control virtqueue (if VIRTIO_NET_F_CTRL_VQ is
negotiated) to send commands to manipulate various features of
the device which would not easily map into the configuration
space.

All commands are of the following form:

\begin{lstlisting}
struct virtio_net_ctrl {
        u8 class;
        u8 command;
        u8 command-specific-data[];
        u8 ack;
        u8 command-specific-result[];
};

/* ack values */
#define VIRTIO_NET_OK     0
#define VIRTIO_NET_ERR    1
\end{lstlisting}

The \field{class}, \field{command} and command-specific-data are set by the
driver, and the device sets the \field{ack} byte and optionally
\field{command-specific-result}. There is little the driver can
do except issue a diagnostic if \field{ack} is not VIRTIO_NET_OK.

The command VIRTIO_NET_CTRL_STATS_QUERY and VIRTIO_NET_CTRL_STATS_GET contain
\field{command-specific-result}.

\paragraph{Packet Receive Filtering}\label{sec:Device Types / Network Device / Device Operation / Control Virtqueue / Packet Receive Filtering}
\label{sec:Device Types / Network Device / Device Operation / Control Virtqueue / Setting Promiscuous Mode}%old label for latexdiff

If the VIRTIO_NET_F_CTRL_RX and VIRTIO_NET_F_CTRL_RX_EXTRA
features are negotiated, the driver can send control commands for
promiscuous mode, multicast, unicast and broadcast receiving.

\begin{note}
In general, these commands are best-effort: unwanted
packets could still arrive.
\end{note}

\begin{lstlisting}
#define VIRTIO_NET_CTRL_RX    0
 #define VIRTIO_NET_CTRL_RX_PROMISC      0
 #define VIRTIO_NET_CTRL_RX_ALLMULTI     1
 #define VIRTIO_NET_CTRL_RX_ALLUNI       2
 #define VIRTIO_NET_CTRL_RX_NOMULTI      3
 #define VIRTIO_NET_CTRL_RX_NOUNI        4
 #define VIRTIO_NET_CTRL_RX_NOBCAST      5
\end{lstlisting}


\devicenormative{\subparagraph}{Packet Receive Filtering}{Device Types / Network Device / Device Operation / Control Virtqueue / Packet Receive Filtering}

If the VIRTIO_NET_F_CTRL_RX feature has been negotiated,
the device MUST support the following VIRTIO_NET_CTRL_RX class
commands:
\begin{itemize}
\item VIRTIO_NET_CTRL_RX_PROMISC turns promiscuous mode on and
off. The command-specific-data is one byte containing 0 (off) or
1 (on). If promiscuous mode is on, the device SHOULD receive all
incoming packets.
This SHOULD take effect even if one of the other modes set by
a VIRTIO_NET_CTRL_RX class command is on.
\item VIRTIO_NET_CTRL_RX_ALLMULTI turns all-multicast receive on and
off. The command-specific-data is one byte containing 0 (off) or
1 (on). When all-multicast receive is on the device SHOULD allow
all incoming multicast packets.
\end{itemize}

If the VIRTIO_NET_F_CTRL_RX_EXTRA feature has been negotiated,
the device MUST support the following VIRTIO_NET_CTRL_RX class
commands:
\begin{itemize}
\item VIRTIO_NET_CTRL_RX_ALLUNI turns all-unicast receive on and
off. The command-specific-data is one byte containing 0 (off) or
1 (on). When all-unicast receive is on the device SHOULD allow
all incoming unicast packets.
\item VIRTIO_NET_CTRL_RX_NOMULTI suppresses multicast receive.
The command-specific-data is one byte containing 0 (multicast
receive allowed) or 1 (multicast receive suppressed).
When multicast receive is suppressed, the device SHOULD NOT
send multicast packets to the driver.
This SHOULD take effect even if VIRTIO_NET_CTRL_RX_ALLMULTI is on.
This filter SHOULD NOT apply to broadcast packets.
\item VIRTIO_NET_CTRL_RX_NOUNI suppresses unicast receive.
The command-specific-data is one byte containing 0 (unicast
receive allowed) or 1 (unicast receive suppressed).
When unicast receive is suppressed, the device SHOULD NOT
send unicast packets to the driver.
This SHOULD take effect even if VIRTIO_NET_CTRL_RX_ALLUNI is on.
\item VIRTIO_NET_CTRL_RX_NOBCAST suppresses broadcast receive.
The command-specific-data is one byte containing 0 (broadcast
receive allowed) or 1 (broadcast receive suppressed).
When broadcast receive is suppressed, the device SHOULD NOT
send broadcast packets to the driver.
This SHOULD take effect even if VIRTIO_NET_CTRL_RX_ALLMULTI is on.
\end{itemize}

\drivernormative{\subparagraph}{Packet Receive Filtering}{Device Types / Network Device / Device Operation / Control Virtqueue / Packet Receive Filtering}

If the VIRTIO_NET_F_CTRL_RX feature has not been negotiated,
the driver MUST NOT issue commands VIRTIO_NET_CTRL_RX_PROMISC or
VIRTIO_NET_CTRL_RX_ALLMULTI.

If the VIRTIO_NET_F_CTRL_RX_EXTRA feature has not been negotiated,
the driver MUST NOT issue commands
 VIRTIO_NET_CTRL_RX_ALLUNI,
 VIRTIO_NET_CTRL_RX_NOMULTI,
 VIRTIO_NET_CTRL_RX_NOUNI or
 VIRTIO_NET_CTRL_RX_NOBCAST.

\paragraph{Setting MAC Address Filtering}\label{sec:Device Types / Network Device / Device Operation / Control Virtqueue / Setting MAC Address Filtering}

If the VIRTIO_NET_F_CTRL_RX feature is negotiated, the driver can
send control commands for MAC address filtering.

\begin{lstlisting}
struct virtio_net_ctrl_mac {
        le32 entries;
        u8 macs[entries][6];
};

#define VIRTIO_NET_CTRL_MAC    1
 #define VIRTIO_NET_CTRL_MAC_TABLE_SET        0
 #define VIRTIO_NET_CTRL_MAC_ADDR_SET         1
\end{lstlisting}

The device can filter incoming packets by any number of destination
MAC addresses\footnote{Since there are no guarantees, it can use a hash filter or
silently switch to allmulti or promiscuous mode if it is given too
many addresses.
}. This table is set using the class
VIRTIO_NET_CTRL_MAC and the command VIRTIO_NET_CTRL_MAC_TABLE_SET. The
command-specific-data is two variable length tables of 6-byte MAC
addresses (as described in struct virtio_net_ctrl_mac). The first table contains unicast addresses, and the second
contains multicast addresses.

The VIRTIO_NET_CTRL_MAC_ADDR_SET command is used to set the
default MAC address which rx filtering
accepts (and if VIRTIO_NET_F_MAC has been negotiated,
this will be reflected in \field{mac} in config space).

The command-specific-data for VIRTIO_NET_CTRL_MAC_ADDR_SET is
the 6-byte MAC address.

\devicenormative{\subparagraph}{Setting MAC Address Filtering}{Device Types / Network Device / Device Operation / Control Virtqueue / Setting MAC Address Filtering}

The device MUST have an empty MAC filtering table on reset.

The device MUST update the MAC filtering table before it consumes
the VIRTIO_NET_CTRL_MAC_TABLE_SET command.

The device MUST update \field{mac} in config space before it consumes
the VIRTIO_NET_CTRL_MAC_ADDR_SET command, if VIRTIO_NET_F_MAC has
been negotiated.

The device SHOULD drop incoming packets which have a destination MAC which
matches neither the \field{mac} (or that set with VIRTIO_NET_CTRL_MAC_ADDR_SET)
nor the MAC filtering table.

\drivernormative{\subparagraph}{Setting MAC Address Filtering}{Device Types / Network Device / Device Operation / Control Virtqueue / Setting MAC Address Filtering}

If VIRTIO_NET_F_CTRL_RX has not been negotiated,
the driver MUST NOT issue VIRTIO_NET_CTRL_MAC class commands.

If VIRTIO_NET_F_CTRL_RX has been negotiated,
the driver SHOULD issue VIRTIO_NET_CTRL_MAC_ADDR_SET
to set the default mac if it is different from \field{mac}.

The driver MUST follow the VIRTIO_NET_CTRL_MAC_TABLE_SET command
by a le32 number, followed by that number of non-multicast
MAC addresses, followed by another le32 number, followed by
that number of multicast addresses.  Either number MAY be 0.

\subparagraph{Legacy Interface: Setting MAC Address Filtering}\label{sec:Device Types / Network Device / Device Operation / Control Virtqueue / Setting MAC Address Filtering / Legacy Interface: Setting MAC Address Filtering}
When using the legacy interface, transitional devices and drivers
MUST format \field{entries} in struct virtio_net_ctrl_mac
according to the native endian of the guest rather than
(necessarily when not using the legacy interface) little-endian.

Legacy drivers that didn't negotiate VIRTIO_NET_F_CTRL_MAC_ADDR
changed \field{mac} in config space when NIC is accepting
incoming packets. These drivers always wrote the mac value from
first to last byte, therefore after detecting such drivers,
a transitional device MAY defer MAC update, or MAY defer
processing incoming packets until driver writes the last byte
of \field{mac} in the config space.

\paragraph{VLAN Filtering}\label{sec:Device Types / Network Device / Device Operation / Control Virtqueue / VLAN Filtering}

If the driver negotiates the VIRTIO_NET_F_CTRL_VLAN feature, it
can control a VLAN filter table in the device. The VLAN filter
table applies only to VLAN tagged packets.

When VIRTIO_NET_F_CTRL_VLAN is negotiated, the device starts with
an empty VLAN filter table.

\begin{note}
Similar to the MAC address based filtering, the VLAN filtering
is also best-effort: unwanted packets could still arrive.
\end{note}

\begin{lstlisting}
#define VIRTIO_NET_CTRL_VLAN       2
 #define VIRTIO_NET_CTRL_VLAN_ADD             0
 #define VIRTIO_NET_CTRL_VLAN_DEL             1
\end{lstlisting}

Both the VIRTIO_NET_CTRL_VLAN_ADD and VIRTIO_NET_CTRL_VLAN_DEL
command take a little-endian 16-bit VLAN id as the command-specific-data.

VIRTIO_NET_CTRL_VLAN_ADD command adds the specified VLAN to the
VLAN filter table.

VIRTIO_NET_CTRL_VLAN_DEL command removes the specified VLAN from
the VLAN filter table.

\devicenormative{\subparagraph}{VLAN Filtering}{Device Types / Network Device / Device Operation / Control Virtqueue / VLAN Filtering}

When VIRTIO_NET_F_CTRL_VLAN is not negotiated, the device MUST
accept all VLAN tagged packets.

When VIRTIO_NET_F_CTRL_VLAN is negotiated, the device MUST
accept all VLAN tagged packets whose VLAN tag is present in
the VLAN filter table and SHOULD drop all VLAN tagged packets
whose VLAN tag is absent in the VLAN filter table.

\subparagraph{Legacy Interface: VLAN Filtering}\label{sec:Device Types / Network Device / Device Operation / Control Virtqueue / VLAN Filtering / Legacy Interface: VLAN Filtering}
When using the legacy interface, transitional devices and drivers
MUST format the VLAN id
according to the native endian of the guest rather than
(necessarily when not using the legacy interface) little-endian.

\paragraph{Gratuitous Packet Sending}\label{sec:Device Types / Network Device / Device Operation / Control Virtqueue / Gratuitous Packet Sending}

If the driver negotiates the VIRTIO_NET_F_GUEST_ANNOUNCE (depends
on VIRTIO_NET_F_CTRL_VQ), the device can ask the driver to send gratuitous
packets; this is usually done after the guest has been physically
migrated, and needs to announce its presence on the new network
links. (As hypervisor does not have the knowledge of guest
network configuration (eg. tagged vlan) it is simplest to prod
the guest in this way).

\begin{lstlisting}
#define VIRTIO_NET_CTRL_ANNOUNCE       3
 #define VIRTIO_NET_CTRL_ANNOUNCE_ACK             0
\end{lstlisting}

The driver checks VIRTIO_NET_S_ANNOUNCE bit in the device configuration \field{status} field
when it notices the changes of device configuration. The
command VIRTIO_NET_CTRL_ANNOUNCE_ACK is used to indicate that
driver has received the notification and device clears the
VIRTIO_NET_S_ANNOUNCE bit in \field{status}.

Processing this notification involves:

\begin{enumerate}
\item Sending the gratuitous packets (eg. ARP) or marking there are pending
  gratuitous packets to be sent and letting deferred routine to
  send them.

\item Sending VIRTIO_NET_CTRL_ANNOUNCE_ACK command through control
  vq.
\end{enumerate}

\drivernormative{\subparagraph}{Gratuitous Packet Sending}{Device Types / Network Device / Device Operation / Control Virtqueue / Gratuitous Packet Sending}

If the driver negotiates VIRTIO_NET_F_GUEST_ANNOUNCE, it SHOULD notify
network peers of its new location after it sees the VIRTIO_NET_S_ANNOUNCE bit
in \field{status}.  The driver MUST send a command on the command queue
with class VIRTIO_NET_CTRL_ANNOUNCE and command VIRTIO_NET_CTRL_ANNOUNCE_ACK.

\devicenormative{\subparagraph}{Gratuitous Packet Sending}{Device Types / Network Device / Device Operation / Control Virtqueue / Gratuitous Packet Sending}

If VIRTIO_NET_F_GUEST_ANNOUNCE is negotiated, the device MUST clear the
VIRTIO_NET_S_ANNOUNCE bit in \field{status} upon receipt of a command buffer
with class VIRTIO_NET_CTRL_ANNOUNCE and command VIRTIO_NET_CTRL_ANNOUNCE_ACK
before marking the buffer as used.

\paragraph{Device operation in multiqueue mode}\label{sec:Device Types / Network Device / Device Operation / Control Virtqueue / Device operation in multiqueue mode}

This specification defines the following modes that a device MAY implement for operation with multiple transmit/receive virtqueues:
\begin{itemize}
\item Automatic receive steering as defined in \ref{sec:Device Types / Network Device / Device Operation / Control Virtqueue / Automatic receive steering in multiqueue mode}.
 If a device supports this mode, it offers the VIRTIO_NET_F_MQ feature bit.
\item Receive-side scaling as defined in \ref{devicenormative:Device Types / Network Device / Device Operation / Control Virtqueue / Receive-side scaling (RSS) / RSS processing}.
 If a device supports this mode, it offers the VIRTIO_NET_F_RSS feature bit.
\end{itemize}

A device MAY support one of these features or both. The driver MAY negotiate any set of these features that the device supports.

Multiqueue is disabled by default.

The driver enables multiqueue by sending a command using \field{class} VIRTIO_NET_CTRL_MQ. The \field{command} selects the mode of multiqueue operation, as follows:
\begin{lstlisting}
#define VIRTIO_NET_CTRL_MQ    4
 #define VIRTIO_NET_CTRL_MQ_VQ_PAIRS_SET        0 (for automatic receive steering)
 #define VIRTIO_NET_CTRL_MQ_RSS_CONFIG          1 (for configurable receive steering)
 #define VIRTIO_NET_CTRL_MQ_HASH_CONFIG         2 (for configurable hash calculation)
\end{lstlisting}

If more than one multiqueue mode is negotiated, the resulting device configuration is defined by the last command sent by the driver.

\paragraph{Automatic receive steering in multiqueue mode}\label{sec:Device Types / Network Device / Device Operation / Control Virtqueue / Automatic receive steering in multiqueue mode}

If the driver negotiates the VIRTIO_NET_F_MQ feature bit (depends on VIRTIO_NET_F_CTRL_VQ), it MAY transmit outgoing packets on one
of the multiple transmitq1\ldots transmitqN and ask the device to
queue incoming packets into one of the multiple receiveq1\ldots receiveqN
depending on the packet flow.

The driver enables multiqueue by
sending the VIRTIO_NET_CTRL_MQ_VQ_PAIRS_SET command, specifying
the number of the transmit and receive queues to be used up to
\field{max_virtqueue_pairs}; subsequently,
transmitq1\ldots transmitqn and receiveq1\ldots receiveqn where
n=\field{virtqueue_pairs} MAY be used.
\begin{lstlisting}
struct virtio_net_ctrl_mq_pairs_set {
       le16 virtqueue_pairs;
};
#define VIRTIO_NET_CTRL_MQ_VQ_PAIRS_MIN        1
#define VIRTIO_NET_CTRL_MQ_VQ_PAIRS_MAX        0x8000

\end{lstlisting}

When multiqueue is enabled by VIRTIO_NET_CTRL_MQ_VQ_PAIRS_SET command, the device MUST use automatic receive steering
based on packet flow. Programming of the receive steering
classificator is implicit. After the driver transmitted a packet of a
flow on transmitqX, the device SHOULD cause incoming packets for that flow to
be steered to receiveqX. For uni-directional protocols, or where
no packets have been transmitted yet, the device MAY steer a packet
to a random queue out of the specified receiveq1\ldots receiveqn.

Multiqueue is disabled by VIRTIO_NET_CTRL_MQ_VQ_PAIRS_SET with \field{virtqueue_pairs} to 1 (this is
the default) and waiting for the device to use the command buffer.

\drivernormative{\subparagraph}{Automatic receive steering in multiqueue mode}{Device Types / Network Device / Device Operation / Control Virtqueue / Automatic receive steering in multiqueue mode}

The driver MUST configure the virtqueues before enabling them with the
VIRTIO_NET_CTRL_MQ_VQ_PAIRS_SET command.

The driver MUST NOT request a \field{virtqueue_pairs} of 0 or
greater than \field{max_virtqueue_pairs} in the device configuration space.

The driver MUST queue packets only on any transmitq1 before the
VIRTIO_NET_CTRL_MQ_VQ_PAIRS_SET command.

The driver MUST NOT queue packets on transmit queues greater than
\field{virtqueue_pairs} once it has placed the VIRTIO_NET_CTRL_MQ_VQ_PAIRS_SET command in the available ring.

\devicenormative{\subparagraph}{Automatic receive steering in multiqueue mode}{Device Types / Network Device / Device Operation / Control Virtqueue / Automatic receive steering in multiqueue mode}

After initialization of reset, the device MUST queue packets only on receiveq1.

The device MUST NOT queue packets on receive queues greater than
\field{virtqueue_pairs} once it has placed the
VIRTIO_NET_CTRL_MQ_VQ_PAIRS_SET command in a used buffer.

If the destination receive queue is being reset (See \ref{sec:Basic Facilities of a Virtio Device / Virtqueues / Virtqueue Reset}),
the device SHOULD re-select another random queue. If all receive queues are
being reset, the device MUST drop the packet.

\subparagraph{Legacy Interface: Automatic receive steering in multiqueue mode}\label{sec:Device Types / Network Device / Device Operation / Control Virtqueue / Automatic receive steering in multiqueue mode / Legacy Interface: Automatic receive steering in multiqueue mode}
When using the legacy interface, transitional devices and drivers
MUST format \field{virtqueue_pairs}
according to the native endian of the guest rather than
(necessarily when not using the legacy interface) little-endian.

\subparagraph{Hash calculation}\label{sec:Device Types / Network Device / Device Operation / Control Virtqueue / Automatic receive steering in multiqueue mode / Hash calculation}
If VIRTIO_NET_F_HASH_REPORT was negotiated and the device uses automatic receive steering,
the device MUST support a command to configure hash calculation parameters.

The driver provides parameters for hash calculation as follows:

\field{class} VIRTIO_NET_CTRL_MQ, \field{command} VIRTIO_NET_CTRL_MQ_HASH_CONFIG.

The \field{command-specific-data} has following format:
\begin{lstlisting}
struct virtio_net_hash_config {
    le32 hash_types;
    le16 reserved[4];
    u8 hash_key_length;
    u8 hash_key_data[hash_key_length];
};
\end{lstlisting}
Field \field{hash_types} contains a bitmask of allowed hash types as
defined in
\ref{sec:Device Types / Network Device / Device Operation / Processing of Incoming Packets / Hash calculation for incoming packets / Supported/enabled hash types}.
Initially the device has all hash types disabled and reports only VIRTIO_NET_HASH_REPORT_NONE.

Field \field{reserved} MUST contain zeroes. It is defined to make the structure to match the layout of virtio_net_rss_config structure,
defined in \ref{sec:Device Types / Network Device / Device Operation / Control Virtqueue / Receive-side scaling (RSS)}.

Fields \field{hash_key_length} and \field{hash_key_data} define the key to be used in hash calculation.

\paragraph{Receive-side scaling (RSS)}\label{sec:Device Types / Network Device / Device Operation / Control Virtqueue / Receive-side scaling (RSS)}
A device offers the feature VIRTIO_NET_F_RSS if it supports RSS receive steering with Toeplitz hash calculation and configurable parameters.

A driver queries RSS capabilities of the device by reading device configuration as defined in \ref{sec:Device Types / Network Device / Device configuration layout}

\subparagraph{Setting RSS parameters}\label{sec:Device Types / Network Device / Device Operation / Control Virtqueue / Receive-side scaling (RSS) / Setting RSS parameters}

Driver sends a VIRTIO_NET_CTRL_MQ_RSS_CONFIG command using the following format for \field{command-specific-data}:
\begin{lstlisting}
struct rss_rq_id {
   le16 vq_index_1_16: 15; /* Bits 1 to 16 of the virtqueue index */
   le16 reserved: 1; /* Set to zero */
};

struct virtio_net_rss_config {
    le32 hash_types;
    le16 indirection_table_mask;
    struct rss_rq_id unclassified_queue;
    struct rss_rq_id indirection_table[indirection_table_length];
    le16 max_tx_vq;
    u8 hash_key_length;
    u8 hash_key_data[hash_key_length];
};
\end{lstlisting}
Field \field{hash_types} contains a bitmask of allowed hash types as
defined in
\ref{sec:Device Types / Network Device / Device Operation / Processing of Incoming Packets / Hash calculation for incoming packets / Supported/enabled hash types}.

Field \field{indirection_table_mask} is a mask to be applied to
the calculated hash to produce an index in the
\field{indirection_table} array.
Number of entries in \field{indirection_table} is (\field{indirection_table_mask} + 1).

\field{rss_rq_id} is a receive virtqueue id. \field{vq_index_1_16}
consists of bits 1 to 16 of a virtqueue index. For example, a
\field{vq_index_1_16} value of 3 corresponds to virtqueue index 6,
which maps to receiveq4.

Field \field{unclassified_queue} specifies the receive virtqueue id in which to
place unclassified packets.

Field \field{indirection_table} is an array of receive virtqueues ids.

A driver sets \field{max_tx_vq} to inform a device how many transmit virtqueues it may use (transmitq1\ldots transmitq \field{max_tx_vq}).

Fields \field{hash_key_length} and \field{hash_key_data} define the key to be used in hash calculation.

\drivernormative{\subparagraph}{Setting RSS parameters}{Device Types / Network Device / Device Operation / Control Virtqueue / Receive-side scaling (RSS) }

A driver MUST NOT send the VIRTIO_NET_CTRL_MQ_RSS_CONFIG command if the feature VIRTIO_NET_F_RSS has not been negotiated.

A driver MUST fill the \field{indirection_table} array only with
enabled receive virtqueues ids.

The number of entries in \field{indirection_table} (\field{indirection_table_mask} + 1) MUST be a power of two.

A driver MUST use \field{indirection_table_mask} values that are less than \field{rss_max_indirection_table_length} reported by a device.

A driver MUST NOT set any VIRTIO_NET_HASH_TYPE_ flags that are not supported by a device.

\devicenormative{\subparagraph}{RSS processing}{Device Types / Network Device / Device Operation / Control Virtqueue / Receive-side scaling (RSS) / RSS processing}
The device MUST determine the destination queue for a network packet as follows:
\begin{itemize}
\item Calculate the hash of the packet as defined in \ref{sec:Device Types / Network Device / Device Operation / Processing of Incoming Packets / Hash calculation for incoming packets}.
\item If the device did not calculate the hash for the specific packet, the device directs the packet to the receiveq specified by \field{unclassified_queue} of virtio_net_rss_config structure.
\item Apply \field{indirection_table_mask} to the calculated hash
and use the result as the index in the indirection table to get
the destination receive virtqueue id.
\item If the destination receive queue is being reset (See \ref{sec:Basic Facilities of a Virtio Device / Virtqueues / Virtqueue Reset}), the device MUST drop the packet.
\end{itemize}

\paragraph{RSS Context}\label{sec:Device Types / Network Device / Device Operation / Control Virtqueue / RSS Context}

An RSS context consists of configurable parameters specified by \ref{sec:Device Types / Network Device
/ Device Operation / Control Virtqueue / Receive-side scaling (RSS)}.

The RSS configuration supported by VIRTIO_NET_F_RSS is considered the default RSS configuration.

The device offers the feature VIRTIO_NET_F_RSS_CONTEXT if it supports one or multiple RSS contexts
(excluding the default RSS configuration) and configurable parameters.

\subparagraph{Querying RSS Context Capability}\label{sec:Device Types / Network Device / Device Operation / Control Virtqueue / RSS Context / Querying RSS Context Capability}

\begin{lstlisting}
#define VIRTNET_RSS_CTX_CTRL 9
 #define VIRTNET_RSS_CTX_CTRL_CAP_GET  0
 #define VIRTNET_RSS_CTX_CTRL_ADD      1
 #define VIRTNET_RSS_CTX_CTRL_MOD      2
 #define VIRTNET_RSS_CTX_CTRL_DEL      3

struct virtnet_rss_ctx_cap {
    le16 max_rss_contexts;
}
\end{lstlisting}

Field \field{max_rss_contexts} specifies the maximum number of RSS contexts \ref{sec:Device Types / Network Device /
Device Operation / Control Virtqueue / RSS Context} supported by the device.

The driver queries the RSS context capability of the device by sending the command VIRTNET_RSS_CTX_CTRL_CAP_GET
with the structure virtnet_rss_ctx_cap.

For the command VIRTNET_RSS_CTX_CTRL_CAP_GET, the structure virtnet_rss_ctx_cap is write-only for the device.

\subparagraph{Setting RSS Context Parameters}\label{sec:Device Types / Network Device / Device Operation / Control Virtqueue / RSS Context / Setting RSS Context Parameters}

\begin{lstlisting}
struct virtnet_rss_ctx_add_modify {
    le16 rss_ctx_id;
    u8 reserved[6];
    struct virtio_net_rss_config rss;
};

struct virtnet_rss_ctx_del {
    le16 rss_ctx_id;
};
\end{lstlisting}

RSS context parameters:
\begin{itemize}
\item  \field{rss_ctx_id}: ID of the specific RSS context.
\item  \field{rss}: RSS context parameters of the specific RSS context whose id is \field{rss_ctx_id}.
\end{itemize}

\field{reserved} is reserved and it is ignored by the device.

If the feature VIRTIO_NET_F_RSS_CONTEXT has been negotiated, the driver can send the following
VIRTNET_RSS_CTX_CTRL class commands:
\begin{enumerate}
\item VIRTNET_RSS_CTX_CTRL_ADD: use the structure virtnet_rss_ctx_add_modify to
       add an RSS context configured as \field{rss} and id as \field{rss_ctx_id} for the device.
\item VIRTNET_RSS_CTX_CTRL_MOD: use the structure virtnet_rss_ctx_add_modify to
       configure parameters of the RSS context whose id is \field{rss_ctx_id} as \field{rss} for the device.
\item VIRTNET_RSS_CTX_CTRL_DEL: use the structure virtnet_rss_ctx_del to delete
       the RSS context whose id is \field{rss_ctx_id} for the device.
\end{enumerate}

For commands VIRTNET_RSS_CTX_CTRL_ADD and VIRTNET_RSS_CTX_CTRL_MOD, the structure virtnet_rss_ctx_add_modify is read-only for the device.
For the command VIRTNET_RSS_CTX_CTRL_DEL, the structure virtnet_rss_ctx_del is read-only for the device.

\devicenormative{\subparagraph}{RSS Context}{Device Types / Network Device / Device Operation / Control Virtqueue / RSS Context}

The device MUST set \field{max_rss_contexts} to at least 1 if it offers VIRTIO_NET_F_RSS_CONTEXT.

Upon reset, the device MUST clear all previously configured RSS contexts.

\drivernormative{\subparagraph}{RSS Context}{Device Types / Network Device / Device Operation / Control Virtqueue / RSS Context}

The driver MUST have negotiated the VIRTIO_NET_F_RSS_CONTEXT feature when issuing the VIRTNET_RSS_CTX_CTRL class commands.

The driver MUST set \field{rss_ctx_id} to between 1 and \field{max_rss_contexts} inclusive.

The driver MUST NOT send the command VIRTIO_NET_CTRL_MQ_VQ_PAIRS_SET when the device has successfully configured at least one RSS context.

\paragraph{Offloads State Configuration}\label{sec:Device Types / Network Device / Device Operation / Control Virtqueue / Offloads State Configuration}

If the VIRTIO_NET_F_CTRL_GUEST_OFFLOADS feature is negotiated, the driver can
send control commands for dynamic offloads state configuration.

\subparagraph{Setting Offloads State}\label{sec:Device Types / Network Device / Device Operation / Control Virtqueue / Offloads State Configuration / Setting Offloads State}

To configure the offloads, the following layout structure and
definitions are used:

\begin{lstlisting}
le64 offloads;

#define VIRTIO_NET_F_GUEST_CSUM       1
#define VIRTIO_NET_F_GUEST_TSO4       7
#define VIRTIO_NET_F_GUEST_TSO6       8
#define VIRTIO_NET_F_GUEST_ECN        9
#define VIRTIO_NET_F_GUEST_UFO        10
#define VIRTIO_NET_F_GUEST_UDP_TUNNEL_GSO  46
#define VIRTIO_NET_F_GUEST_UDP_TUNNEL_GSO_CSUM 47
#define VIRTIO_NET_F_GUEST_USO4       54
#define VIRTIO_NET_F_GUEST_USO6       55

#define VIRTIO_NET_CTRL_GUEST_OFFLOADS       5
 #define VIRTIO_NET_CTRL_GUEST_OFFLOADS_SET   0
\end{lstlisting}

The class VIRTIO_NET_CTRL_GUEST_OFFLOADS has one command:
VIRTIO_NET_CTRL_GUEST_OFFLOADS_SET applies the new offloads configuration.

le64 value passed as command data is a bitmask, bits set define
offloads to be enabled, bits cleared - offloads to be disabled.

There is a corresponding device feature for each offload. Upon feature
negotiation corresponding offload gets enabled to preserve backward
compatibility.

\drivernormative{\subparagraph}{Setting Offloads State}{Device Types / Network Device / Device Operation / Control Virtqueue / Offloads State Configuration / Setting Offloads State}

A driver MUST NOT enable an offload for which the appropriate feature
has not been negotiated.

\subparagraph{Legacy Interface: Setting Offloads State}\label{sec:Device Types / Network Device / Device Operation / Control Virtqueue / Offloads State Configuration / Setting Offloads State / Legacy Interface: Setting Offloads State}
When using the legacy interface, transitional devices and drivers
MUST format \field{offloads}
according to the native endian of the guest rather than
(necessarily when not using the legacy interface) little-endian.


\paragraph{Notifications Coalescing}\label{sec:Device Types / Network Device / Device Operation / Control Virtqueue / Notifications Coalescing}

If the VIRTIO_NET_F_NOTF_COAL feature is negotiated, the driver can
send commands VIRTIO_NET_CTRL_NOTF_COAL_TX_SET and VIRTIO_NET_CTRL_NOTF_COAL_RX_SET
for notification coalescing.

If the VIRTIO_NET_F_VQ_NOTF_COAL feature is negotiated, the driver can
send commands VIRTIO_NET_CTRL_NOTF_COAL_VQ_SET and VIRTIO_NET_CTRL_NOTF_COAL_VQ_GET
for virtqueue notification coalescing.

\begin{lstlisting}
struct virtio_net_ctrl_coal {
    le32 max_packets;
    le32 max_usecs;
};

struct virtio_net_ctrl_coal_vq {
    le16 vq_index;
    le16 reserved;
    struct virtio_net_ctrl_coal coal;
};

#define VIRTIO_NET_CTRL_NOTF_COAL 6
 #define VIRTIO_NET_CTRL_NOTF_COAL_TX_SET  0
 #define VIRTIO_NET_CTRL_NOTF_COAL_RX_SET 1
 #define VIRTIO_NET_CTRL_NOTF_COAL_VQ_SET 2
 #define VIRTIO_NET_CTRL_NOTF_COAL_VQ_GET 3
\end{lstlisting}

Coalescing parameters:
\begin{itemize}
\item \field{vq_index}: The virtqueue index of an enabled transmit or receive virtqueue.
\item \field{max_usecs} for RX: Maximum number of microseconds to delay a RX notification.
\item \field{max_usecs} for TX: Maximum number of microseconds to delay a TX notification.
\item \field{max_packets} for RX: Maximum number of packets to receive before a RX notification.
\item \field{max_packets} for TX: Maximum number of packets to send before a TX notification.
\end{itemize}

\field{reserved} is reserved and it is ignored by the device.

Read/Write attributes for coalescing parameters:
\begin{itemize}
\item For commands VIRTIO_NET_CTRL_NOTF_COAL_TX_SET and VIRTIO_NET_CTRL_NOTF_COAL_RX_SET, the structure virtio_net_ctrl_coal is write-only for the driver.
\item For the command VIRTIO_NET_CTRL_NOTF_COAL_VQ_SET, the structure virtio_net_ctrl_coal_vq is write-only for the driver.
\item For the command VIRTIO_NET_CTRL_NOTF_COAL_VQ_GET, \field{vq_index} and \field{reserved} are write-only
      for the driver, and the structure virtio_net_ctrl_coal is read-only for the driver.
\end{itemize}

The class VIRTIO_NET_CTRL_NOTF_COAL has the following commands:
\begin{enumerate}
\item VIRTIO_NET_CTRL_NOTF_COAL_TX_SET: use the structure virtio_net_ctrl_coal to set the \field{max_usecs} and \field{max_packets} parameters for all transmit virtqueues.
\item VIRTIO_NET_CTRL_NOTF_COAL_RX_SET: use the structure virtio_net_ctrl_coal to set the \field{max_usecs} and \field{max_packets} parameters for all receive virtqueues.
\item VIRTIO_NET_CTRL_NOTF_COAL_VQ_SET: use the structure virtio_net_ctrl_coal_vq to set the \field{max_usecs} and \field{max_packets} parameters
                                        for an enabled transmit/receive virtqueue whose index is \field{vq_index}.
\item VIRTIO_NET_CTRL_NOTF_COAL_VQ_GET: use the structure virtio_net_ctrl_coal_vq to get the \field{max_usecs} and \field{max_packets} parameters
                                        for an enabled transmit/receive virtqueue whose index is \field{vq_index}.
\end{enumerate}

The device may generate notifications more or less frequently than specified by set commands of the VIRTIO_NET_CTRL_NOTF_COAL class.

If coalescing parameters are being set, the device applies the last coalescing parameters set for a
virtqueue, regardless of the command used to set the parameters. Use the following command sequence
with two pairs of virtqueues as an example:
Each of the following commands sets \field{max_usecs} and \field{max_packets} parameters for virtqueues.
\begin{itemize}
\item Command1: VIRTIO_NET_CTRL_NOTF_COAL_RX_SET sets coalescing parameters for virtqueues having index 0 and index 2. Virtqueues having index 1 and index 3 retain their previous parameters.
\item Command2: VIRTIO_NET_CTRL_NOTF_COAL_VQ_SET with \field{vq_index} = 0 sets coalescing parameters for virtqueue having index 0. Virtqueue having index 2 retains the parameters from command1.
\item Command3: VIRTIO_NET_CTRL_NOTF_COAL_VQ_GET with \field{vq_index} = 0, the device responds with coalescing parameters of vq_index 0 set by command2.
\item Command4: VIRTIO_NET_CTRL_NOTF_COAL_VQ_SET with \field{vq_index} = 1 sets coalescing parameters for virtqueue having index 1. Virtqueue having index 3 retains its previous parameters.
\item Command5: VIRTIO_NET_CTRL_NOTF_COAL_TX_SET sets coalescing parameters for virtqueues having index 1 and index 3, and overrides the parameters set by command4.
\item Command6: VIRTIO_NET_CTRL_NOTF_COAL_VQ_GET with \field{vq_index} = 1, the device responds with coalescing parameters of index 1 set by command5.
\end{itemize}

\subparagraph{Operation}\label{sec:Device Types / Network Device / Device Operation / Control Virtqueue / Notifications Coalescing / Operation}

The device sends a used buffer notification once the notification conditions are met and if the notifications are not suppressed as explained in \ref{sec:Basic Facilities of a Virtio Device / Virtqueues / Used Buffer Notification Suppression}.

When the device has non-zero \field{max_usecs} and non-zero \field{max_packets}, it starts counting microseconds and packets upon receiving/sending a packet.
The device counts packets and microseconds for each receive virtqueue and transmit virtqueue separately.
In this case, the notification conditions are met when \field{max_usecs} microseconds elapse, or upon sending/receiving \field{max_packets} packets, whichever happens first.
Afterwards, the device waits for the next packet and starts counting packets and microseconds again.

When the device has \field{max_usecs} = 0 or \field{max_packets} = 0, the notification conditions are met after every packet received/sent.

\subparagraph{RX Example}\label{sec:Device Types / Network Device / Device Operation / Control Virtqueue / Notifications Coalescing / RX Example}

If, for example:
\begin{itemize}
\item \field{max_usecs} = 10.
\item \field{max_packets} = 15.
\end{itemize}
then each receive virtqueue of a device will operate as follows:
\begin{itemize}
\item The device will count packets received on each virtqueue until it accumulates 15, or until 10 microseconds elapsed since the first one was received.
\item If the notifications are not suppressed by the driver, the device will send an used buffer notification, otherwise, the device will not send an used buffer notification as long as the notifications are suppressed.
\end{itemize}

\subparagraph{TX Example}\label{sec:Device Types / Network Device / Device Operation / Control Virtqueue / Notifications Coalescing / TX Example}

If, for example:
\begin{itemize}
\item \field{max_usecs} = 10.
\item \field{max_packets} = 15.
\end{itemize}
then each transmit virtqueue of a device will operate as follows:
\begin{itemize}
\item The device will count packets sent on each virtqueue until it accumulates 15, or until 10 microseconds elapsed since the first one was sent.
\item If the notifications are not suppressed by the driver, the device will send an used buffer notification, otherwise, the device will not send an used buffer notification as long as the notifications are suppressed.
\end{itemize}

\subparagraph{Notifications When Coalescing Parameters Change}\label{sec:Device Types / Network Device / Device Operation / Control Virtqueue / Notifications Coalescing / Notifications When Coalescing Parameters Change}

When the coalescing parameters of a device change, the device needs to check if the new notification conditions are met and send a used buffer notification if so.

For example, \field{max_packets} = 15 for a device with a single transmit virtqueue: if the device sends 10 packets and afterwards receives a
VIRTIO_NET_CTRL_NOTF_COAL_TX_SET command with \field{max_packets} = 8, then the notification condition is immediately considered to be met;
the device needs to immediately send a used buffer notification, if the notifications are not suppressed by the driver.

\drivernormative{\subparagraph}{Notifications Coalescing}{Device Types / Network Device / Device Operation / Control Virtqueue / Notifications Coalescing}

The driver MUST set \field{vq_index} to the virtqueue index of an enabled transmit or receive virtqueue.

The driver MUST have negotiated the VIRTIO_NET_F_NOTF_COAL feature when issuing commands VIRTIO_NET_CTRL_NOTF_COAL_TX_SET and VIRTIO_NET_CTRL_NOTF_COAL_RX_SET.

The driver MUST have negotiated the VIRTIO_NET_F_VQ_NOTF_COAL feature when issuing commands VIRTIO_NET_CTRL_NOTF_COAL_VQ_SET and VIRTIO_NET_CTRL_NOTF_COAL_VQ_GET.

The driver MUST ignore the values of coalescing parameters received from the VIRTIO_NET_CTRL_NOTF_COAL_VQ_GET command if the device responds with VIRTIO_NET_ERR.

\devicenormative{\subparagraph}{Notifications Coalescing}{Device Types / Network Device / Device Operation / Control Virtqueue / Notifications Coalescing}

The device MUST ignore \field{reserved}.

The device SHOULD respond to VIRTIO_NET_CTRL_NOTF_COAL_TX_SET and VIRTIO_NET_CTRL_NOTF_COAL_RX_SET commands with VIRTIO_NET_ERR if it was not able to change the parameters.

The device MUST respond to the VIRTIO_NET_CTRL_NOTF_COAL_VQ_SET command with VIRTIO_NET_ERR if it was not able to change the parameters.

The device MUST respond to VIRTIO_NET_CTRL_NOTF_COAL_VQ_SET and VIRTIO_NET_CTRL_NOTF_COAL_VQ_GET commands with
VIRTIO_NET_ERR if the designated virtqueue is not an enabled transmit or receive virtqueue.

Upon disabling and re-enabling a transmit virtqueue, the device MUST set the coalescing parameters of the virtqueue
to those configured through the VIRTIO_NET_CTRL_NOTF_COAL_TX_SET command, or, if the driver did not set any TX coalescing parameters, to 0.

Upon disabling and re-enabling a receive virtqueue, the device MUST set the coalescing parameters of the virtqueue
to those configured through the VIRTIO_NET_CTRL_NOTF_COAL_RX_SET command, or, if the driver did not set any RX coalescing parameters, to 0.

The behavior of the device in response to set commands of the VIRTIO_NET_CTRL_NOTF_COAL class is best-effort:
the device MAY generate notifications more or less frequently than specified.

A device SHOULD NOT send used buffer notifications to the driver if the notifications are suppressed, even if the notification conditions are met.

Upon reset, a device MUST initialize all coalescing parameters to 0.

\paragraph{Device Statistics}\label{sec:Device Types / Network Device / Device Operation / Control Virtqueue / Device Statistics}

If the VIRTIO_NET_F_DEVICE_STATS feature is negotiated, the driver can obtain
device statistics from the device by using the following command.

Different types of virtqueues have different statistics. The statistics of the
receiveq are different from those of the transmitq.

The statistics of a certain type of virtqueue are also divided into multiple types
because different types require different features. This enables the expansion
of new statistics.

In one command, the driver can obtain the statistics of one or multiple virtqueues.
Additionally, the driver can obtain multiple type statistics of each virtqueue.

\subparagraph{Query Statistic Capabilities}\label{sec:Device Types / Network Device / Device Operation / Control Virtqueue / Device Statistics / Query Statistic Capabilities}

\begin{lstlisting}
#define VIRTIO_NET_CTRL_STATS         8
#define VIRTIO_NET_CTRL_STATS_QUERY   0
#define VIRTIO_NET_CTRL_STATS_GET     1

struct virtio_net_stats_capabilities {

#define VIRTIO_NET_STATS_TYPE_CVQ       (1 << 32)

#define VIRTIO_NET_STATS_TYPE_RX_BASIC  (1 << 0)
#define VIRTIO_NET_STATS_TYPE_RX_CSUM   (1 << 1)
#define VIRTIO_NET_STATS_TYPE_RX_GSO    (1 << 2)
#define VIRTIO_NET_STATS_TYPE_RX_SPEED  (1 << 3)

#define VIRTIO_NET_STATS_TYPE_TX_BASIC  (1 << 16)
#define VIRTIO_NET_STATS_TYPE_TX_CSUM   (1 << 17)
#define VIRTIO_NET_STATS_TYPE_TX_GSO    (1 << 18)
#define VIRTIO_NET_STATS_TYPE_TX_SPEED  (1 << 19)

    le64 supported_stats_types[1];
}
\end{lstlisting}

To obtain device statistic capability, use the VIRTIO_NET_CTRL_STATS_QUERY
command. When the command completes successfully, \field{command-specific-result}
is in the format of \field{struct virtio_net_stats_capabilities}.

\subparagraph{Get Statistics}\label{sec:Device Types / Network Device / Device Operation / Control Virtqueue / Device Statistics / Get Statistics}

\begin{lstlisting}
struct virtio_net_ctrl_queue_stats {
       struct {
           le16 vq_index;
           le16 reserved[3];
           le64 types_bitmap[1];
       } stats[];
};

struct virtio_net_stats_reply_hdr {
#define VIRTIO_NET_STATS_TYPE_REPLY_CVQ       32

#define VIRTIO_NET_STATS_TYPE_REPLY_RX_BASIC  0
#define VIRTIO_NET_STATS_TYPE_REPLY_RX_CSUM   1
#define VIRTIO_NET_STATS_TYPE_REPLY_RX_GSO    2
#define VIRTIO_NET_STATS_TYPE_REPLY_RX_SPEED  3

#define VIRTIO_NET_STATS_TYPE_REPLY_TX_BASIC  16
#define VIRTIO_NET_STATS_TYPE_REPLY_TX_CSUM   17
#define VIRTIO_NET_STATS_TYPE_REPLY_TX_GSO    18
#define VIRTIO_NET_STATS_TYPE_REPLY_TX_SPEED  19
    u8 type;
    u8 reserved;
    le16 vq_index;
    le16 reserved1;
    le16 size;
}
\end{lstlisting}

To obtain device statistics, use the VIRTIO_NET_CTRL_STATS_GET command with the
\field{command-specific-data} which is in the format of
\field{struct virtio_net_ctrl_queue_stats}. When the command completes
successfully, \field{command-specific-result} contains multiple statistic
results, each statistic result has the \field{struct virtio_net_stats_reply_hdr}
as the header.

The fields of the \field{struct virtio_net_ctrl_queue_stats}:
\begin{description}
    \item [vq_index]
        The index of the virtqueue to obtain the statistics.

    \item [types_bitmap]
        This is a bitmask of the types of statistics to be obtained. Therefore, a
        \field{stats} inside \field{struct virtio_net_ctrl_queue_stats} may
        indicate multiple statistic replies for the virtqueue.
\end{description}

The fields of the \field{struct virtio_net_stats_reply_hdr}:
\begin{description}
    \item [type]
        The type of the reply statistic.

    \item [vq_index]
        The virtqueue index of the reply statistic.

    \item [size]
        The number of bytes for the statistics entry including size of \field{struct virtio_net_stats_reply_hdr}.

\end{description}

\subparagraph{Controlq Statistics}\label{sec:Device Types / Network Device / Device Operation / Control Virtqueue / Device Statistics / Controlq Statistics}

The structure corresponding to the controlq statistics is
\field{struct virtio_net_stats_cvq}. The corresponding type is
VIRTIO_NET_STATS_TYPE_CVQ. This is for the controlq.

\begin{lstlisting}
struct virtio_net_stats_cvq {
    struct virtio_net_stats_reply_hdr hdr;

    le64 command_num;
    le64 ok_num;
};
\end{lstlisting}

\begin{description}
    \item [command_num]
        The number of commands received by the device including the current command.

    \item [ok_num]
        The number of commands completed successfully by the device including the current command.
\end{description}


\subparagraph{Receiveq Basic Statistics}\label{sec:Device Types / Network Device / Device Operation / Control Virtqueue / Device Statistics / Receiveq Basic Statistics}

The structure corresponding to the receiveq basic statistics is
\field{struct virtio_net_stats_rx_basic}. The corresponding type is
VIRTIO_NET_STATS_TYPE_RX_BASIC. This is for the receiveq.

Receiveq basic statistics do not require any feature. As long as the device supports
VIRTIO_NET_F_DEVICE_STATS, the following are the receiveq basic statistics.

\begin{lstlisting}
struct virtio_net_stats_rx_basic {
    struct virtio_net_stats_reply_hdr hdr;

    le64 rx_notifications;

    le64 rx_packets;
    le64 rx_bytes;

    le64 rx_interrupts;

    le64 rx_drops;
    le64 rx_drop_overruns;
};
\end{lstlisting}

The packets described below were all presented on the specified virtqueue.
\begin{description}
    \item [rx_notifications]
        The number of driver notifications received by the device for this
        receiveq.

    \item [rx_packets]
        This is the number of packets passed to the driver by the device.

    \item [rx_bytes]
        This is the bytes of packets passed to the driver by the device.

    \item [rx_interrupts]
        The number of interrupts generated by the device for this receiveq.

    \item [rx_drops]
        This is the number of packets dropped by the device. The count includes
        all types of packets dropped by the device.

    \item [rx_drop_overruns]
        This is the number of packets dropped by the device when no more
        descriptors were available.

\end{description}

\subparagraph{Transmitq Basic Statistics}\label{sec:Device Types / Network Device / Device Operation / Control Virtqueue / Device Statistics / Transmitq Basic Statistics}

The structure corresponding to the transmitq basic statistics is
\field{struct virtio_net_stats_tx_basic}. The corresponding type is
VIRTIO_NET_STATS_TYPE_TX_BASIC. This is for the transmitq.

Transmitq basic statistics do not require any feature. As long as the device supports
VIRTIO_NET_F_DEVICE_STATS, the following are the transmitq basic statistics.

\begin{lstlisting}
struct virtio_net_stats_tx_basic {
    struct virtio_net_stats_reply_hdr hdr;

    le64 tx_notifications;

    le64 tx_packets;
    le64 tx_bytes;

    le64 tx_interrupts;

    le64 tx_drops;
    le64 tx_drop_malformed;
};
\end{lstlisting}

The packets described below are all for a specific virtqueue.
\begin{description}
    \item [tx_notifications]
        The number of driver notifications received by the device for this
        transmitq.

    \item [tx_packets]
        This is the number of packets sent by the device (not the packets
        got from the driver).

    \item [tx_bytes]
        This is the number of bytes sent by the device for all the sent packets
        (not the bytes sent got from the driver).

    \item [tx_interrupts]
        The number of interrupts generated by the device for this transmitq.

    \item [tx_drops]
        The number of packets dropped by the device. The count includes all
        types of packets dropped by the device.

    \item [tx_drop_malformed]
        The number of packets dropped by the device, when the descriptors are
        malformed. For example, the buffer is too short.
\end{description}

\subparagraph{Receiveq CSUM Statistics}\label{sec:Device Types / Network Device / Device Operation / Control Virtqueue / Device Statistics / Receiveq CSUM Statistics}

The structure corresponding to the receiveq checksum statistics is
\field{struct virtio_net_stats_rx_csum}. The corresponding type is
VIRTIO_NET_STATS_TYPE_RX_CSUM. This is for the receiveq.

Only after the VIRTIO_NET_F_GUEST_CSUM is negotiated, the receiveq checksum
statistics can be obtained.

\begin{lstlisting}
struct virtio_net_stats_rx_csum {
    struct virtio_net_stats_reply_hdr hdr;

    le64 rx_csum_valid;
    le64 rx_needs_csum;
    le64 rx_csum_none;
    le64 rx_csum_bad;
};
\end{lstlisting}

The packets described below were all presented on the specified virtqueue.
\begin{description}
    \item [rx_csum_valid]
        The number of packets with VIRTIO_NET_HDR_F_DATA_VALID.

    \item [rx_needs_csum]
        The number of packets with VIRTIO_NET_HDR_F_NEEDS_CSUM.

    \item [rx_csum_none]
        The number of packets without hardware checksum. The packet here refers
        to the non-TCP/UDP packet that the device cannot recognize.

    \item [rx_csum_bad]
        The number of packets with checksum mismatch.

\end{description}

\subparagraph{Transmitq CSUM Statistics}\label{sec:Device Types / Network Device / Device Operation / Control Virtqueue / Device Statistics / Transmitq CSUM Statistics}

The structure corresponding to the transmitq checksum statistics is
\field{struct virtio_net_stats_tx_csum}. The corresponding type is
VIRTIO_NET_STATS_TYPE_TX_CSUM. This is for the transmitq.

Only after the VIRTIO_NET_F_CSUM is negotiated, the transmitq checksum
statistics can be obtained.

The following are the transmitq checksum statistics:

\begin{lstlisting}
struct virtio_net_stats_tx_csum {
    struct virtio_net_stats_reply_hdr hdr;

    le64 tx_csum_none;
    le64 tx_needs_csum;
};
\end{lstlisting}

The packets described below are all for a specific virtqueue.
\begin{description}
    \item [tx_csum_none]
        The number of packets which do not require hardware checksum.

    \item [tx_needs_csum]
        The number of packets which require checksum calculation by the device.

\end{description}

\subparagraph{Receiveq GSO Statistics}\label{sec:Device Types / Network Device / Device Operation / Control Virtqueue / Device Statistics / Receiveq GSO Statistics}

The structure corresponding to the receivq GSO statistics is
\field{struct virtio_net_stats_rx_gso}. The corresponding type is
VIRTIO_NET_STATS_TYPE_RX_GSO. This is for the receiveq.

If one or more of the VIRTIO_NET_F_GUEST_TSO4, VIRTIO_NET_F_GUEST_TSO6
have been negotiated, the receiveq GSO statistics can be obtained.

GSO packets refer to packets passed by the device to the driver where
\field{gso_type} is not VIRTIO_NET_HDR_GSO_NONE.

\begin{lstlisting}
struct virtio_net_stats_rx_gso {
    struct virtio_net_stats_reply_hdr hdr;

    le64 rx_gso_packets;
    le64 rx_gso_bytes;
    le64 rx_gso_packets_coalesced;
    le64 rx_gso_bytes_coalesced;
};
\end{lstlisting}

The packets described below were all presented on the specified virtqueue.
\begin{description}
    \item [rx_gso_packets]
        The number of the GSO packets received by the device.

    \item [rx_gso_bytes]
        The bytes of the GSO packets received by the device.
        This includes the header size of the GSO packet.

    \item [rx_gso_packets_coalesced]
        The number of the GSO packets coalesced by the device.

    \item [rx_gso_bytes_coalesced]
        The bytes of the GSO packets coalesced by the device.
        This includes the header size of the GSO packet.
\end{description}

\subparagraph{Transmitq GSO Statistics}\label{sec:Device Types / Network Device / Device Operation / Control Virtqueue / Device Statistics / Transmitq GSO Statistics}

The structure corresponding to the transmitq GSO statistics is
\field{struct virtio_net_stats_tx_gso}. The corresponding type is
VIRTIO_NET_STATS_TYPE_TX_GSO. This is for the transmitq.

If one or more of the VIRTIO_NET_F_HOST_TSO4, VIRTIO_NET_F_HOST_TSO6,
VIRTIO_NET_F_HOST_USO options have been negotiated, the transmitq GSO statistics
can be obtained.

GSO packets refer to packets passed by the driver to the device where
\field{gso_type} is not VIRTIO_NET_HDR_GSO_NONE.
See more \ref{sec:Device Types / Network Device / Device Operation / Packet
Transmission}.

\begin{lstlisting}
struct virtio_net_stats_tx_gso {
    struct virtio_net_stats_reply_hdr hdr;

    le64 tx_gso_packets;
    le64 tx_gso_bytes;
    le64 tx_gso_segments;
    le64 tx_gso_segments_bytes;
    le64 tx_gso_packets_noseg;
    le64 tx_gso_bytes_noseg;
};
\end{lstlisting}

The packets described below are all for a specific virtqueue.
\begin{description}
    \item [tx_gso_packets]
        The number of the GSO packets sent by the device.

    \item [tx_gso_bytes]
        The bytes of the GSO packets sent by the device.

    \item [tx_gso_segments]
        The number of segments prepared from GSO packets.

    \item [tx_gso_segments_bytes]
        The bytes of segments prepared from GSO packets.

    \item [tx_gso_packets_noseg]
        The number of the GSO packets without segmentation.

    \item [tx_gso_bytes_noseg]
        The bytes of the GSO packets without segmentation.

\end{description}

\subparagraph{Receiveq Speed Statistics}\label{sec:Device Types / Network Device / Device Operation / Control Virtqueue / Device Statistics / Receiveq Speed Statistics}

The structure corresponding to the receiveq speed statistics is
\field{struct virtio_net_stats_rx_speed}. The corresponding type is
VIRTIO_NET_STATS_TYPE_RX_SPEED. This is for the receiveq.

The device has the allowance for the speed. If VIRTIO_NET_F_SPEED_DUPLEX has
been negotiated, the driver can get this by \field{speed}. When the received
packets bitrate exceeds the \field{speed}, some packets may be dropped by the
device.

\begin{lstlisting}
struct virtio_net_stats_rx_speed {
    struct virtio_net_stats_reply_hdr hdr;

    le64 rx_packets_allowance_exceeded;
    le64 rx_bytes_allowance_exceeded;
};
\end{lstlisting}

The packets described below were all presented on the specified virtqueue.
\begin{description}
    \item [rx_packets_allowance_exceeded]
        The number of the packets dropped by the device due to the received
        packets bitrate exceeding the \field{speed}.

    \item [rx_bytes_allowance_exceeded]
        The bytes of the packets dropped by the device due to the received
        packets bitrate exceeding the \field{speed}.

\end{description}

\subparagraph{Transmitq Speed Statistics}\label{sec:Device Types / Network Device / Device Operation / Control Virtqueue / Device Statistics / Transmitq Speed Statistics}

The structure corresponding to the transmitq speed statistics is
\field{struct virtio_net_stats_tx_speed}. The corresponding type is
VIRTIO_NET_STATS_TYPE_TX_SPEED. This is for the transmitq.

The device has the allowance for the speed. If VIRTIO_NET_F_SPEED_DUPLEX has
been negotiated, the driver can get this by \field{speed}. When the transmit
packets bitrate exceeds the \field{speed}, some packets may be dropped by the
device.

\begin{lstlisting}
struct virtio_net_stats_tx_speed {
    struct virtio_net_stats_reply_hdr hdr;

    le64 tx_packets_allowance_exceeded;
    le64 tx_bytes_allowance_exceeded;
};
\end{lstlisting}

The packets described below were all presented on the specified virtqueue.
\begin{description}
    \item [tx_packets_allowance_exceeded]
        The number of the packets dropped by the device due to the transmit packets
        bitrate exceeding the \field{speed}.

    \item [tx_bytes_allowance_exceeded]
        The bytes of the packets dropped by the device due to the transmit packets
        bitrate exceeding the \field{speed}.

\end{description}

\devicenormative{\subparagraph}{Device Statistics}{Device Types / Network Device / Device Operation / Control Virtqueue / Device Statistics}

When the VIRTIO_NET_F_DEVICE_STATS feature is negotiated, the device MUST reply
to the command VIRTIO_NET_CTRL_STATS_QUERY with the
\field{struct virtio_net_stats_capabilities}. \field{supported_stats_types}
includes all the statistic types supported by the device.

If \field{struct virtio_net_ctrl_queue_stats} is incorrect (such as the
following), the device MUST set \field{ack} to VIRTIO_NET_ERR. Even if there is
only one error, the device MUST fail the entire command.
\begin{itemize}
    \item \field{vq_index} exceeds the queue range.
    \item \field{types_bitmap} contains unknown types.
    \item One or more of the bits present in \field{types_bitmap} is not valid
        for the specified virtqueue.
    \item The feature corresponding to the specified \field{types_bitmap} was
        not negotiated.
\end{itemize}

The device MUST set the actual size of the bytes occupied by the reply to the
\field{size} of the \field{hdr}. And the device MUST set the \field{type} and
the \field{vq_index} of the statistic header.

The \field{command-specific-result} buffer allocated by the driver may be
smaller or bigger than all the statistics specified by
\field{struct virtio_net_ctrl_queue_stats}. The device MUST fill up only upto
the valid bytes.

The statistics counter replied by the device MUST wrap around to zero by the
device on the overflow.

\drivernormative{\subparagraph}{Device Statistics}{Device Types / Network Device / Device Operation / Control Virtqueue / Device Statistics}

The types contained in the \field{types_bitmap} MUST be queried from the device
via command VIRTIO_NET_CTRL_STATS_QUERY.

\field{types_bitmap} in \field{struct virtio_net_ctrl_queue_stats} MUST be valid to the
vq specified by \field{vq_index}.

The \field{command-specific-result} buffer allocated by the driver MUST have
enough capacity to store all the statistics reply headers defined in
\field{struct virtio_net_ctrl_queue_stats}. If the
\field{command-specific-result} buffer is fully utilized by the device but some
replies are missed, it is possible that some statistics may exceed the capacity
of the driver's records. In such cases, the driver should allocate additional
space for the \field{command-specific-result} buffer.

\subsubsection{Flow filter}\label{sec:Device Types / Network Device / Device Operation / Flow filter}

A network device can support one or more flow filter rules. Each flow filter rule
is applied by matching a packet and then taking an action, such as directing the packet
to a specific receiveq or dropping the packet. An example of a match is
matching on specific source and destination IP addresses.

A flow filter rule is a device resource object that consists of a key,
a processing priority, and an action to either direct a packet to a
receive queue or drop the packet.

Each rule uses a classifier. The key is matched against the packet using
a classifier, defining which fields in the packet are matched.
A classifier resource object consists of one or more field selectors, each with
a type that specifies the header fields to be matched against, and a mask.
The mask can match whole fields or parts of a field in a header. Each
rule resource object depends on the classifier resource object.

When a packet is received, relevant fields are extracted
(in the same way) from both the packet and the key according to the
classifier. The resulting field contents are then compared -
if they are identical the rule action is taken, if they are not, the rule is ignored.

Multiple flow filter rules are part of a group. The rule resource object
depends on the group. Each rule within a
group has a rule priority, and each group also has a group priority. For a
packet, a group with the highest priority is selected first. Within a group,
rules are applied from highest to lowest priority, until one of the rules
matches the packet and an action is taken. If all the rules within a group
are ignored, the group with the next highest priority is selected, and so on.

The device and the driver indicates flow filter resource limits using the capability
\ref{par:Device Types / Network Device / Device Operation / Flow filter / Device and driver capabilities / VIRTIO-NET-FF-RESOURCE-CAP} specifying the limits on the number of flow filter rule,
group and classifier resource objects. The capability \ref{par:Device Types / Network Device / Device Operation / Flow filter / Device and driver capabilities / VIRTIO-NET-FF-SELECTOR-CAP} specifies which selectors the device supports.
The driver indicates the selectors it is using by setting the flow
filter selector capability, prior to adding any resource objects.

The capability \ref{par:Device Types / Network Device / Device Operation / Flow filter / Device and driver capabilities / VIRTIO-NET-FF-ACTION-CAP} specifies which actions the device supports.

The driver controls the flow filter rule, classifier and group resource objects using
administration commands described in
\ref{sec:Basic Facilities of a Virtio Device / Device groups / Group administration commands / Device resource objects}.

\paragraph{Packet processing order}\label{sec:sec:Device Types / Network Device / Device Operation / Flow filter / Packet processing order}

Note that flow filter rules are applied after MAC/VLAN filtering. Flow filter
rules take precedence over steering: if a flow filter rule results in an action,
the steering configuration does not apply. The steering configuration only applies
to packets for which no flow filter rule action was performed. For example,
incoming packets can be processed in the following order:

\begin{itemize}
\item apply steering configuration received using control virtqueue commands
      VIRTIO_NET_CTRL_RX, VIRTIO_NET_CTRL_MAC and VIRTIO_NET_CTRL_VLAN.
\item apply flow filter rules if any.
\item if no filter rule applied, apply steering configuration received using command
      VIRTIO_NET_CTRL_MQ_RSS_CONFIG or as per automatic receive steering.
\end{itemize}

Some incoming packet processing examples:
\begin{itemize}
\item If the packet is dropped by the flow filter rule, RSS
      steering is ignored for the packet.
\item If the packet is directed to a specific receiveq using flow filter rule,
      the RSS steering is ignored for the packet.
\item If a packet is dropped due to the VIRTIO_NET_CTRL_MAC configuration,
      both flow filter rules and the RSS steering are ignored for the packet.
\item If a packet does not match any flow filter rules,
      the RSS steering is used to select the receiveq for the packet (if enabled).
\item If there are two flow filter groups configured as group_A and group_B
      with respective group priorities as 4, and 5; flow filter rules of
      group_B are applied first having highest group priority, if there is a match,
      the flow filter rules of group_A are ignored; if there is no match for
      the flow filter rules in group_B, the flow filter rules of next level group_A are applied.
\end{itemize}

\paragraph{Device and driver capabilities}
\label{par:Device Types / Network Device / Device Operation / Flow filter / Device and driver capabilities}

\subparagraph{VIRTIO_NET_FF_RESOURCE_CAP}
\label{par:Device Types / Network Device / Device Operation / Flow filter / Device and driver capabilities / VIRTIO-NET-FF-RESOURCE-CAP}

The capability VIRTIO_NET_FF_RESOURCE_CAP indicates the flow filter resource limits.
\field{cap_specific_data} is in the format
\field{struct virtio_net_ff_cap_data}.

\begin{lstlisting}
struct virtio_net_ff_cap_data {
        le32 groups_limit;
        le32 selectors_limit;
        le32 rules_limit;
        le32 rules_per_group_limit;
        u8 last_rule_priority;
        u8 selectors_per_classifier_limit;
};
\end{lstlisting}

\field{groups_limit}, and \field{selectors_limit} represent the maximum
number of flow filter groups and selectors, respectively, that the driver can create.
 \field{rules_limit} is the maximum number of
flow fiilter rules that the driver can create across all the groups.
\field{rules_per_group_limit} is the maximum number of flow filter rules that the driver
can create for each flow filter group.

\field{last_rule_priority} is the highest priority that can be assigned to a
flow filter rule.

\field{selectors_per_classifier_limit} is the maximum number of selectors
that a classifier can have.

\subparagraph{VIRTIO_NET_FF_SELECTOR_CAP}
\label{par:Device Types / Network Device / Device Operation / Flow filter / Device and driver capabilities / VIRTIO-NET-FF-SELECTOR-CAP}

The capability VIRTIO_NET_FF_SELECTOR_CAP lists the supported selectors and the
supported packet header fields for each selector.
\field{cap_specific_data} is in the format \field{struct virtio_net_ff_cap_mask_data}.

\begin{lstlisting}[label={lst:Device Types / Network Device / Device Operation / Flow filter / Device and driver capabilities / VIRTIO-NET-FF-SELECTOR-CAP / virtio-net-ff-selector}]
struct virtio_net_ff_selector {
        u8 type;
        u8 flags;
        u8 reserved[2];
        u8 length;
        u8 reserved1[3];
        u8 mask[];
};

struct virtio_net_ff_cap_mask_data {
        u8 count;
        u8 reserved[7];
        struct virtio_net_ff_selector selectors[];
};

#define VIRTIO_NET_FF_MASK_F_PARTIAL_MASK (1 << 0)
\end{lstlisting}

\field{count} indicates number of valid entries in the \field{selectors} array.
\field{selectors[]} is an array of supported selectors. Within each array entry:
\field{type} specifies the type of the packet header, as defined in table
\ref{table:Device Types / Network Device / Device Operation / Flow filter / Device and driver capabilities / VIRTIO-NET-FF-SELECTOR-CAP / flow filter selector types}. \field{mask} specifies which fields of the
packet header can be matched in a flow filter rule.

Each \field{type} is also listed in table
\ref{table:Device Types / Network Device / Device Operation / Flow filter / Device and driver capabilities / VIRTIO-NET-FF-SELECTOR-CAP / flow filter selector types}. \field{mask} is a byte array
in network byte order. For example, when \field{type} is VIRTIO_NET_FF_MASK_TYPE_IPV6,
the \field{mask} is in the format \hyperref[intro:IPv6-Header-Format]{IPv6 Header Format}.

If partial masking is not set, then all bits in each field have to be either all 0s
to ignore this field or all 1s to match on this field. If partial masking is set,
then any combination of bits can bit set to match on these bits.
For example, when a selector \field{type} is VIRTIO_NET_FF_MASK_TYPE_ETH, if
\field{mask[0-12]} are zero and \field{mask[13-14]} are 0xff (all 1s), it
indicates that matching is only supported for \field{EtherType} of
\field{Ethernet MAC frame}, matching is not supported for
\field{Destination Address} and \field{Source Address}.

The entries in the array \field{selectors} are ordered by
\field{type}, with each \field{type} value only appearing once.

\field{length} is the length of a dynamic array \field{mask} in bytes.
\field{reserved} and \field{reserved1} are reserved and set to zero.

\begin{table}[H]
\caption{Flow filter selector types}
\label{table:Device Types / Network Device / Device Operation / Flow filter / Device and driver capabilities / VIRTIO-NET-FF-SELECTOR-CAP / flow filter selector types}
\begin{tabularx}{\textwidth}{ |l|X|X| }
\hline
Type & Name & Description \\
\hline \hline
0x0 & - & Reserved \\
\hline
0x1 & VIRTIO_NET_FF_MASK_TYPE_ETH & 14 bytes of frame header starting from destination address described in \hyperref[intro:IEEE 802.3-2022]{IEEE 802.3-2022} \\
\hline
0x2 & VIRTIO_NET_FF_MASK_TYPE_IPV4 & 20 bytes of \hyperref[intro:Internet-Header-Format]{IPv4: Internet Header Format} \\
\hline
0x3 & VIRTIO_NET_FF_MASK_TYPE_IPV6 & 40 bytes of \hyperref[intro:IPv6-Header-Format]{IPv6 Header Format} \\
\hline
0x4 & VIRTIO_NET_FF_MASK_TYPE_TCP & 20 bytes of \hyperref[intro:TCP-Header-Format]{TCP Header Format} \\
\hline
0x5 & VIRTIO_NET_FF_MASK_TYPE_UDP & 8 bytes of UDP header described in \hyperref[intro:UDP]{UDP} \\
\hline
0x6 - 0xFF & & Reserved for future \\
\hline
\end{tabularx}
\end{table}

When VIRTIO_NET_FF_MASK_F_PARTIAL_MASK (bit 0) is set, it indicates that
partial masking is supported for all the fields of the selector identified by \field{type}.

For the selector \field{type} VIRTIO_NET_FF_MASK_TYPE_IPV4, if a partial mask is unsupported,
then matching on an individual bit of \field{Flags} in the
\field{IPv4: Internet Header Format} is unsupported. \field{Flags} has to match as a whole
if it is supported.

For the selector \field{type} VIRTIO_NET_FF_MASK_TYPE_IPV4, \field{mask} includes fields
up to the \field{Destination Address}; that is, \field{Options} and
\field{Padding} are excluded.

For the selector \field{type} VIRTIO_NET_FF_MASK_TYPE_IPV6, the \field{Next Header} field
of the \field{mask} corresponds to the \field{Next Header} in the packet
when \field{IPv6 Extension Headers} are not present. When the packet includes
one or more \field{IPv6 Extension Headers}, the \field{Next Header} field of
the \field{mask} corresponds to the \field{Next Header} of the last
\field{IPv6 Extension Header} in the packet.

For the selector \field{type} VIRTIO_NET_FF_MASK_TYPE_TCP, \field{Control bits}
are treated as individual fields for matching; that is, matching individual
\field{Control bits} does not depend on the partial mask support.

\subparagraph{VIRTIO_NET_FF_ACTION_CAP}
\label{par:Device Types / Network Device / Device Operation / Flow filter / Device and driver capabilities / VIRTIO-NET-FF-ACTION-CAP}

The capability VIRTIO_NET_FF_ACTION_CAP lists the supported actions in a rule.
\field{cap_specific_data} is in the format \field{struct virtio_net_ff_cap_actions}.

\begin{lstlisting}
struct virtio_net_ff_actions {
        u8 count;
        u8 reserved[7];
        u8 actions[];
};
\end{lstlisting}

\field{actions} is an array listing all possible actions.
The entries in the array are ordered from the smallest to the largest,
with each supported value appearing exactly once. Each entry can have the
following values:

\begin{table}[H]
\caption{Flow filter rule actions}
\label{table:Device Types / Network Device / Device Operation / Flow filter / Device and driver capabilities / VIRTIO-NET-FF-ACTION-CAP / flow filter rule actions}
\begin{tabularx}{\textwidth}{ |l|X|X| }
\hline
Action & Name & Description \\
\hline \hline
0x0 & - & reserved \\
\hline
0x1 & VIRTIO_NET_FF_ACTION_DROP & Matching packet will be dropped by the device \\
\hline
0x2 & VIRTIO_NET_FF_ACTION_DIRECT_RX_VQ & Matching packet will be directed to a receive queue \\
\hline
0x3 - 0xFF & & Reserved for future \\
\hline
\end{tabularx}
\end{table}

\paragraph{Resource objects}
\label{par:Device Types / Network Device / Device Operation / Flow filter / Resource objects}

\subparagraph{VIRTIO_NET_RESOURCE_OBJ_FF_GROUP}\label{par:Device Types / Network Device / Device Operation / Flow filter / Resource objects / VIRTIO-NET-RESOURCE-OBJ-FF-GROUP}

A flow filter group contains between 0 and \field{rules_limit} rules, as specified by the
capability VIRTIO_NET_FF_RESOURCE_CAP. For the flow filter group object both
\field{resource_obj_specific_data} and
\field{resource_obj_specific_result} are in the format
\field{struct virtio_net_resource_obj_ff_group}.

\begin{lstlisting}
struct virtio_net_resource_obj_ff_group {
        le16 group_priority;
};
\end{lstlisting}

\field{group_priority} specifies the priority for the group. Each group has a
distinct priority. For each incoming packet, the device tries to apply rules
from groups from higher \field{group_priority} value to lower, until either a
rule matches the packet or all groups have been tried.

\subparagraph{VIRTIO_NET_RESOURCE_OBJ_FF_CLASSIFIER}\label{par:Device Types / Network Device / Device Operation / Flow filter / Resource objects / VIRTIO-NET-RESOURCE-OBJ-FF-CLASSIFIER}

A classifier is used to match a flow filter key against a packet. The
classifier defines the desired packet fields to match, and is represented by
the VIRTIO_NET_RESOURCE_OBJ_FF_CLASSIFIER device resource object.

For the flow filter classifier object both \field{resource_obj_specific_data} and
\field{resource_obj_specific_result} are in the format
\field{struct virtio_net_resource_obj_ff_classifier}.

\begin{lstlisting}
struct virtio_net_resource_obj_ff_classifier {
        u8 count;
        u8 reserved[7];
        struct virtio_net_ff_selector selectors[];
};
\end{lstlisting}

A classifier is an array of \field{selectors}. The number of selectors in the
array is indicated by \field{count}. The selector has a type that specifies
the header fields to be matched against, and a mask.
See \ref{lst:Device Types / Network Device / Device Operation / Flow filter / Device and driver capabilities / VIRTIO-NET-FF-SELECTOR-CAP / virtio-net-ff-selector}
for details about selectors.

The first selector is always VIRTIO_NET_FF_MASK_TYPE_ETH. When there are multiple
selectors, a second selector can be either VIRTIO_NET_FF_MASK_TYPE_IPV4
or VIRTIO_NET_FF_MASK_TYPE_IPV6. If the third selector exists, the third
selector can be either VIRTIO_NET_FF_MASK_TYPE_UDP or VIRTIO_NET_FF_MASK_TYPE_TCP.
For example, to match a Ethernet IPv6 UDP packet,
\field{selectors[0].type} is set to VIRTIO_NET_FF_MASK_TYPE_ETH, \field{selectors[1].type}
is set to VIRTIO_NET_FF_MASK_TYPE_IPV6 and \field{selectors[2].type} is
set to VIRTIO_NET_FF_MASK_TYPE_UDP; accordingly, \field{selectors[0].mask[0-13]} is
for Ethernet header fields, \field{selectors[1].mask[0-39]} is set for IPV6 header
and \field{selectors[2].mask[0-7]} is set for UDP header.

When there are multiple selectors, the type of the (N+1)\textsuperscript{th} selector
affects the mask of the (N)\textsuperscript{th} selector. If
\field{count} is 2 or more, all the mask bits within \field{selectors[0]}
corresponding to \field{EtherType} of an Ethernet header are set.

If \field{count} is more than 2:
\begin{itemize}
\item if \field{selector[1].type} is, VIRTIO_NET_FF_MASK_TYPE_IPV4, then, all the mask bits within
\field{selector[1]} for \field{Protocol} is set.
\item if \field{selector[1].type} is, VIRTIO_NET_FF_MASK_TYPE_IPV6, then, all the mask bits within
\field{selector[1]} for \field{Next Header} is set.
\end{itemize}

If for a given packet header field, a subset of bits of a field is to be matched,
and if the partial mask is supported, the flow filter
mask object can specify a mask which has fewer bits set than the packet header
field size. For example, a partial mask for the Ethernet header source mac
address can be of 1-bit for multicast detection instead of 48-bits.

\subparagraph{VIRTIO_NET_RESOURCE_OBJ_FF_RULE}\label{par:Device Types / Network Device / Device Operation / Flow filter / Resource objects / VIRTIO-NET-RESOURCE-OBJ-FF-RULE}

Each flow filter rule resource object comprises a key, a priority, and an action.
For the flow filter rule object,
\field{resource_obj_specific_data} and
\field{resource_obj_specific_result} are in the format
\field{struct virtio_net_resource_obj_ff_rule}.

\begin{lstlisting}
struct virtio_net_resource_obj_ff_rule {
        le32 group_id;
        le32 classifier_id;
        u8 rule_priority;
        u8 key_length; /* length of key in bytes */
        u8 action;
        u8 reserved;
        le16 vq_index;
        u8 reserved1[2];
        u8 keys[][];
};
\end{lstlisting}

\field{group_id} is the resource object ID of the flow filter group to which
this rule belongs. \field{classifier_id} is the resource object ID of the
classifier used to match a packet against the \field{key}.

\field{rule_priority} denotes the priority of the rule within the group
specified by the \field{group_id}.
Rules within the group are applied from the highest to the lowest priority
until a rule matches the packet and an
action is taken. Rules with the same priority can be applied in any order.

\field{reserved} and \field{reserved1} are reserved and set to 0.

\field{keys[][]} is an array of keys to match against packets, using
the classifier specified by \field{classifier_id}. Each entry (key) comprises
a byte array, and they are located one immediately after another.
The size (number of entries) of the array is exactly the same as that of
\field{selectors} in the classifier, or in other words, \field{count}
in the classifier.

\field{key_length} specifies the total length of \field{keys} in bytes.
In other words, it equals the sum total of \field{length} of all
selectors in \field{selectors} in the classifier specified by
\field{classifier_id}.

For example, if a classifier object's \field{selectors[0].type} is
VIRTIO_NET_FF_MASK_TYPE_ETH and \field{selectors[1].type} is
VIRTIO_NET_FF_MASK_TYPE_IPV6,
then selectors[0].length is 14 and selectors[1].length is 40.
Accordingly, the \field{key_length} is set to 54.
This setting indicates that the \field{key} array's length is 54 bytes
comprising a first byte array of 14 bytes for the
Ethernet MAC header in bytes 0-13, immediately followed by 40 bytes for the
IPv6 header in bytes 14-53.

When there are multiple selectors in the classifier object, the key bytes
for (N)\textsuperscript{th} selector are set so that
(N+1)\textsuperscript{th} selector can be matched.

If \field{count} is 2 or more, key bytes of \field{EtherType}
are set according to \hyperref[intro:IEEE 802 Ethertypes]{IEEE 802 Ethertypes}
for VIRTIO_NET_FF_MASK_TYPE_IPV4 or VIRTIO_NET_FF_MASK_TYPE_IPV6 respectively.

If \field{count} is more than 2, when \field{selector[1].type} is
VIRTIO_NET_FF_MASK_TYPE_IPV4 or VIRTIO_NET_FF_MASK_TYPE_IPV6, key
bytes of \field{Protocol} or \field{Next Header} is set as per
\field{Protocol Numbers} defined \hyperref[intro:IANA Protocol Numbers]{IANA Protocol Numbers}
respectively.

\field{action} is the action to take when a packet matches the
\field{key} using the \field{classifier_id}. Supported actions are described in
\ref{table:Device Types / Network Device / Device Operation / Flow filter / Device and driver capabilities / VIRTIO-NET-FF-ACTION-CAP / flow filter rule actions}.

\field{vq_index} specifies a receive virtqueue. When the \field{action} is set
to VIRTIO_NET_FF_ACTION_DIRECT_RX_VQ, and the packet matches the \field{key},
the matching packet is directed to this virtqueue.

Note that at most one action is ever taken for a given packet. If a rule is
applied and an action is taken, the action of other rules is not taken.

\devicenormative{\paragraph}{Flow filter}{Device Types / Network Device / Device Operation / Flow filter}

When the device supports flow filter operations,
\begin{itemize}
\item the device MUST set VIRTIO_NET_FF_RESOURCE_CAP, VIRTIO_NET_FF_SELECTOR_CAP
and VIRTIO_NET_FF_ACTION_CAP capability in the \field{supported_caps} in the
command VIRTIO_ADMIN_CMD_CAP_SUPPORT_QUERY.
\item the device MUST support the administration commands
VIRTIO_ADMIN_CMD_RESOURCE_OBJ_CREATE,
VIRTIO_ADMIN_CMD_RESOURCE_OBJ_MODIFY, VIRTIO_ADMIN_CMD_RESOURCE_OBJ_QUERY,
VIRTIO_ADMIN_CMD_RESOURCE_OBJ_DESTROY for the resource types
VIRTIO_NET_RESOURCE_OBJ_FF_GROUP, VIRTIO_NET_RESOURCE_OBJ_FF_CLASSIFIER and
VIRTIO_NET_RESOURCE_OBJ_FF_RULE.
\end{itemize}

When any of the VIRTIO_NET_FF_RESOURCE_CAP, VIRTIO_NET_FF_SELECTOR_CAP, or
VIRTIO_NET_FF_ACTION_CAP capability is disabled, the device SHOULD set
\field{status} to VIRTIO_ADMIN_STATUS_Q_INVALID_OPCODE for the commands
VIRTIO_ADMIN_CMD_RESOURCE_OBJ_CREATE,
VIRTIO_ADMIN_CMD_RESOURCE_OBJ_MODIFY, VIRTIO_ADMIN_CMD_RESOURCE_OBJ_QUERY,
and VIRTIO_ADMIN_CMD_RESOURCE_OBJ_DESTROY. These commands apply to the resource
\field{type} of VIRTIO_NET_RESOURCE_OBJ_FF_GROUP, VIRTIO_NET_RESOURCE_OBJ_FF_CLASSIFIER, and
VIRTIO_NET_RESOURCE_OBJ_FF_RULE.

The device SHOULD set \field{status} to VIRTIO_ADMIN_STATUS_EINVAL for the
command VIRTIO_ADMIN_CMD_RESOURCE_OBJ_CREATE when the resource \field{type}
is VIRTIO_NET_RESOURCE_OBJ_FF_GROUP, if a flow filter group already exists
with the supplied \field{group_priority}.

The device SHOULD set \field{status} to VIRTIO_ADMIN_STATUS_ENOSPC for the
command VIRTIO_ADMIN_CMD_RESOURCE_OBJ_CREATE when the resource \field{type}
is VIRTIO_NET_RESOURCE_OBJ_FF_GROUP, if the number of flow filter group
objects in the device exceeds the lower of the configured driver
capabilities \field{groups_limit} and \field{rules_per_group_limit}.

The device SHOULD set \field{status} to VIRTIO_ADMIN_STATUS_ENOSPC for the
command VIRTIO_ADMIN_CMD_RESOURCE_OBJ_CREATE when the resource \field{type} is
VIRTIO_NET_RESOURCE_OBJ_FF_CLASSIFIER, if the number of flow filter selector
objects in the device exceeds the configured driver capability
\field{selectors_limit}.

The device SHOULD set \field{status} to VIRTIO_ADMIN_STATUS_EBUSY for the
command VIRTIO_ADMIN_CMD_RESOURCE_OBJ_DESTROY for a flow filter group when
the flow filter group has one or more flow filter rules depending on it.

The device SHOULD set \field{status} to VIRTIO_ADMIN_STATUS_EBUSY for the
command VIRTIO_ADMIN_CMD_RESOURCE_OBJ_DESTROY for a flow filter classifier when
the flow filter classifier has one or more flow filter rules depending on it.

The device SHOULD fail the command VIRTIO_ADMIN_CMD_RESOURCE_OBJ_CREATE for the
flow filter rule resource object if,
\begin{itemize}
\item \field{vq_index} is not a valid receive virtqueue index for
the VIRTIO_NET_FF_ACTION_DIRECT_RX_VQ action,
\item \field{priority} is greater than or equal to
      \field{last_rule_priority},
\item \field{id} is greater than or equal to \field{rules_limit} or
      greater than or equal to \field{rules_per_group_limit}, whichever is lower,
\item the length of \field{keys} and the length of all the mask bytes of
      \field{selectors[].mask} as referred by \field{classifier_id} differs,
\item the supplied \field{action} is not supported in the capability VIRTIO_NET_FF_ACTION_CAP.
\end{itemize}

When the flow filter directs a packet to the virtqueue identified by
\field{vq_index} and if the receive virtqueue is reset, the device
MUST drop such packets.

Upon applying a flow filter rule to a packet, the device MUST STOP any further
application of rules and cease applying any other steering configurations.

For multiple flow filter groups, the device MUST apply the rules from
the group with the highest priority. If any rule from this group is applied,
the device MUST ignore the remaining groups. If none of the rules from the
highest priority group match, the device MUST apply the rules from
the group with the next highest priority, until either a rule matches or
all groups have been attempted.

The device MUST apply the rules within the group from the highest to the
lowest priority until a rule matches the packet, and the device MUST take
the action. If an action is taken, the device MUST not take any other
action for this packet.

The device MAY apply the rules with the same \field{rule_priority} in any
order within the group.

The device MUST process incoming packets in the following order:
\begin{itemize}
\item apply the steering configuration received using control virtqueue
      commands VIRTIO_NET_CTRL_RX, VIRTIO_NET_CTRL_MAC, and
      VIRTIO_NET_CTRL_VLAN.
\item apply flow filter rules if any.
\item if no filter rule is applied, apply the steering configuration
      received using the command VIRTIO_NET_CTRL_MQ_RSS_CONFIG
      or according to automatic receive steering.
\end{itemize}

When processing an incoming packet, if the packet is dropped at any stage, the device
MUST skip further processing.

When the device drops the packet due to the configuration done using the control
virtqueue commands VIRTIO_NET_CTRL_RX or VIRTIO_NET_CTRL_MAC or VIRTIO_NET_CTRL_VLAN,
the device MUST skip flow filter rules for this packet.

When the device performs flow filter match operations and if the operation
result did not have any match in all the groups, the receive packet processing
continues to next level, i.e. to apply configuration done using
VIRTIO_NET_CTRL_MQ_RSS_CONFIG command.

The device MUST support the creation of flow filter classifier objects
using the command VIRTIO_ADMIN_CMD_RESOURCE_OBJ_CREATE with \field{flags}
set to VIRTIO_NET_FF_MASK_F_PARTIAL_MASK;
this support is required even if all the bits of the masks are set for
a field in \field{selectors}, provided that partial masking is supported
for the selectors.

\drivernormative{\paragraph}{Flow filter}{Device Types / Network Device / Device Operation / Flow filter}

The driver MUST enable VIRTIO_NET_FF_RESOURCE_CAP, VIRTIO_NET_FF_SELECTOR_CAP,
and VIRTIO_NET_FF_ACTION_CAP capabilities to use flow filter.

The driver SHOULD NOT remove a flow filter group using the command
VIRTIO_ADMIN_CMD_RESOURCE_OBJ_DESTROY when one or more flow filter rules
depend on that group. The driver SHOULD only destroy the group after
all the associated rules have been destroyed.

The driver SHOULD NOT remove a flow filter classifier using the command
VIRTIO_ADMIN_CMD_RESOURCE_OBJ_DESTROY when one or more flow filter rules
depend on the classifier. The driver SHOULD only destroy the classifier
after all the associated rules have been destroyed.

The driver SHOULD NOT add multiple flow filter rules with the same
\field{rule_priority} within a flow filter group, as these rules MAY match
the same packet. The driver SHOULD assign different \field{rule_priority}
values to different flow filter rules if multiple rules may match a single
packet.

For the command VIRTIO_ADMIN_CMD_RESOURCE_OBJ_CREATE, when creating a resource
of \field{type} VIRTIO_NET_RESOURCE_OBJ_FF_CLASSIFIER, the driver MUST set:
\begin{itemize}
\item \field{selectors[0].type} to VIRTIO_NET_FF_MASK_TYPE_ETH.
\item \field{selectors[1].type} to VIRTIO_NET_FF_MASK_TYPE_IPV4 or
      VIRTIO_NET_FF_MASK_TYPE_IPV6 when \field{count} is more than 1,
\item \field{selectors[2].type} VIRTIO_NET_FF_MASK_TYPE_UDP or
      VIRTIO_NET_FF_MASK_TYPE_TCP when \field{count} is more than 2.
\end{itemize}

For the command VIRTIO_ADMIN_CMD_RESOURCE_OBJ_CREATE, when creating a resource
of \field{type} VIRTIO_NET_RESOURCE_OBJ_FF_CLASSIFIER, the driver MUST set:
\begin{itemize}
\item \field{selectors[0].mask} bytes to all 1s for the \field{EtherType}
       when \field{count} is 2 or more.
\item \field{selectors[1].mask} bytes to all 1s for \field{Protocol} or \field{Next Header}
       when \field{selector[1].type} is VIRTIO_NET_FF_MASK_TYPE_IPV4 or VIRTIO_NET_FF_MASK_TYPE_IPV6,
       and when \field{count} is more than 2.
\end{itemize}

For the command VIRTIO_ADMIN_CMD_RESOURCE_OBJ_CREATE, the resource \field{type}
VIRTIO_NET_RESOURCE_OBJ_FF_RULE, if the corresponding classifier object's
\field{count} is 2 or more, the driver MUST SET the \field{keys} bytes of
\field{EtherType} in accordance with
\hyperref[intro:IEEE 802 Ethertypes]{IEEE 802 Ethertypes}
for either VIRTIO_NET_FF_MASK_TYPE_IPV4 or VIRTIO_NET_FF_MASK_TYPE_IPV6.

For the command VIRTIO_ADMIN_CMD_RESOURCE_OBJ_CREATE, when creating a resource of
\field{type} VIRTIO_NET_RESOURCE_OBJ_FF_RULE, if the corresponding classifier
object's \field{count} is more than 2, and the \field{selector[1].type} is either
VIRTIO_NET_FF_MASK_TYPE_IPV4 or VIRTIO_NET_FF_MASK_TYPE_IPV6, the driver MUST
set the \field{keys} bytes for the \field{Protocol} or \field{Next Header}
according to \hyperref[intro:IANA Protocol Numbers]{IANA Protocol Numbers} respectively.

The driver SHOULD set all the bits for a field in the mask of a selector in both the
capability and the classifier object, unless the VIRTIO_NET_FF_MASK_F_PARTIAL_MASK
is enabled.

\subsubsection{Legacy Interface: Framing Requirements}\label{sec:Device
Types / Network Device / Legacy Interface: Framing Requirements}

When using legacy interfaces, transitional drivers which have not
negotiated VIRTIO_F_ANY_LAYOUT MUST use a single descriptor for the
\field{struct virtio_net_hdr} on both transmit and receive, with the
network data in the following descriptors.

Additionally, when using the control virtqueue (see \ref{sec:Device
Types / Network Device / Device Operation / Control Virtqueue})
, transitional drivers which have not
negotiated VIRTIO_F_ANY_LAYOUT MUST:
\begin{itemize}
\item for all commands, use a single 2-byte descriptor including the first two
fields: \field{class} and \field{command}
\item for all commands except VIRTIO_NET_CTRL_MAC_TABLE_SET
use a single descriptor including command-specific-data
with no padding.
\item for the VIRTIO_NET_CTRL_MAC_TABLE_SET command use exactly
two descriptors including command-specific-data with no padding:
the first of these descriptors MUST include the
virtio_net_ctrl_mac table structure for the unicast addresses with no padding,
the second of these descriptors MUST include the
virtio_net_ctrl_mac table structure for the multicast addresses
with no padding.
\item for all commands, use a single 1-byte descriptor for the
\field{ack} field
\end{itemize}

See \ref{sec:Basic
Facilities of a Virtio Device / Virtqueues / Message Framing}.

\section{Network Device}\label{sec:Device Types / Network Device}

The virtio network device is a virtual network interface controller.
It consists of a virtual Ethernet link which connects the device
to the Ethernet network. The device has transmit and receive
queues. The driver adds empty buffers to the receive virtqueue.
The device receives incoming packets from the link; the device
places these incoming packets in the receive virtqueue buffers.
The driver adds outgoing packets to the transmit virtqueue. The device
removes these packets from the transmit virtqueue and sends them to
the link. The device may have a control virtqueue. The driver
uses the control virtqueue to dynamically manipulate various
features of the initialized device.

\subsection{Device ID}\label{sec:Device Types / Network Device / Device ID}

 1

\subsection{Virtqueues}\label{sec:Device Types / Network Device / Virtqueues}

\begin{description}
\item[0] receiveq1
\item[1] transmitq1
\item[\ldots]
\item[2(N-1)] receiveqN
\item[2(N-1)+1] transmitqN
\item[2N] controlq
\end{description}

 N=1 if neither VIRTIO_NET_F_MQ nor VIRTIO_NET_F_RSS are negotiated, otherwise N is set by
 \field{max_virtqueue_pairs}.

controlq is optional; it only exists if VIRTIO_NET_F_CTRL_VQ is
negotiated.

\subsection{Feature bits}\label{sec:Device Types / Network Device / Feature bits}

\begin{description}
\item[VIRTIO_NET_F_CSUM (0)] Device handles packets with partial checksum offload.

\item[VIRTIO_NET_F_GUEST_CSUM (1)] Driver handles packets with partial checksum.

\item[VIRTIO_NET_F_CTRL_GUEST_OFFLOADS (2)] Control channel offloads
        reconfiguration support.

\item[VIRTIO_NET_F_MTU(3)] Device maximum MTU reporting is supported. If
    offered by the device, device advises driver about the value of
    its maximum MTU. If negotiated, the driver uses \field{mtu} as
    the maximum MTU value.

\item[VIRTIO_NET_F_MAC (5)] Device has given MAC address.

\item[VIRTIO_NET_F_GUEST_TSO4 (7)] Driver can receive TSOv4.

\item[VIRTIO_NET_F_GUEST_TSO6 (8)] Driver can receive TSOv6.

\item[VIRTIO_NET_F_GUEST_ECN (9)] Driver can receive TSO with ECN.

\item[VIRTIO_NET_F_GUEST_UFO (10)] Driver can receive UFO.

\item[VIRTIO_NET_F_HOST_TSO4 (11)] Device can receive TSOv4.

\item[VIRTIO_NET_F_HOST_TSO6 (12)] Device can receive TSOv6.

\item[VIRTIO_NET_F_HOST_ECN (13)] Device can receive TSO with ECN.

\item[VIRTIO_NET_F_HOST_UFO (14)] Device can receive UFO.

\item[VIRTIO_NET_F_MRG_RXBUF (15)] Driver can merge receive buffers.

\item[VIRTIO_NET_F_STATUS (16)] Configuration status field is
    available.

\item[VIRTIO_NET_F_CTRL_VQ (17)] Control channel is available.

\item[VIRTIO_NET_F_CTRL_RX (18)] Control channel RX mode support.

\item[VIRTIO_NET_F_CTRL_VLAN (19)] Control channel VLAN filtering.

\item[VIRTIO_NET_F_CTRL_RX_EXTRA (20)]	Control channel RX extra mode support.

\item[VIRTIO_NET_F_GUEST_ANNOUNCE(21)] Driver can send gratuitous
    packets.

\item[VIRTIO_NET_F_MQ(22)] Device supports multiqueue with automatic
    receive steering.

\item[VIRTIO_NET_F_CTRL_MAC_ADDR(23)] Set MAC address through control
    channel.

\item[VIRTIO_NET_F_DEVICE_STATS(50)] Device can provide device-level statistics
    to the driver through the control virtqueue.

\item[VIRTIO_NET_F_HASH_TUNNEL(51)] Device supports inner header hash for encapsulated packets.

\item[VIRTIO_NET_F_VQ_NOTF_COAL(52)] Device supports virtqueue notification coalescing.

\item[VIRTIO_NET_F_NOTF_COAL(53)] Device supports notifications coalescing.

\item[VIRTIO_NET_F_GUEST_USO4 (54)] Driver can receive USOv4 packets.

\item[VIRTIO_NET_F_GUEST_USO6 (55)] Driver can receive USOv6 packets.

\item[VIRTIO_NET_F_HOST_USO (56)] Device can receive USO packets. Unlike UFO
 (fragmenting the packet) the USO splits large UDP packet
 to several segments when each of these smaller packets has UDP header.

\item[VIRTIO_NET_F_HASH_REPORT(57)] Device can report per-packet hash
    value and a type of calculated hash.

\item[VIRTIO_NET_F_GUEST_HDRLEN(59)] Driver can provide the exact \field{hdr_len}
    value. Device benefits from knowing the exact header length.

\item[VIRTIO_NET_F_RSS(60)] Device supports RSS (receive-side scaling)
    with Toeplitz hash calculation and configurable hash
    parameters for receive steering.

\item[VIRTIO_NET_F_RSC_EXT(61)] Device can process duplicated ACKs
    and report number of coalesced segments and duplicated ACKs.

\item[VIRTIO_NET_F_STANDBY(62)] Device may act as a standby for a primary
    device with the same MAC address.

\item[VIRTIO_NET_F_SPEED_DUPLEX(63)] Device reports speed and duplex.

\item[VIRTIO_NET_F_RSS_CONTEXT(64)] Device supports multiple RSS contexts.

\item[VIRTIO_NET_F_GUEST_UDP_TUNNEL_GSO (65)] Driver can receive GSO packets
  carried by a UDP tunnel.

\item[VIRTIO_NET_F_GUEST_UDP_TUNNEL_GSO_CSUM (66)] Driver handles packets
  carried by a UDP tunnel with partial csum for the outer header.

\item[VIRTIO_NET_F_HOST_UDP_TUNNEL_GSO (67)] Device can receive GSO packets
  carried by a UDP tunnel.

\item[VIRTIO_NET_F_HOST_UDP_TUNNEL_GSO_CSUM (68)] Device handles packets
  carried by a UDP tunnel with partial csum for the outer header.
\end{description}

\subsubsection{Feature bit requirements}\label{sec:Device Types / Network Device / Feature bits / Feature bit requirements}

Some networking feature bits require other networking feature bits
(see \ref{drivernormative:Basic Facilities of a Virtio Device / Feature Bits}):

\begin{description}
\item[VIRTIO_NET_F_GUEST_TSO4] Requires VIRTIO_NET_F_GUEST_CSUM.
\item[VIRTIO_NET_F_GUEST_TSO6] Requires VIRTIO_NET_F_GUEST_CSUM.
\item[VIRTIO_NET_F_GUEST_ECN] Requires VIRTIO_NET_F_GUEST_TSO4 or VIRTIO_NET_F_GUEST_TSO6.
\item[VIRTIO_NET_F_GUEST_UFO] Requires VIRTIO_NET_F_GUEST_CSUM.
\item[VIRTIO_NET_F_GUEST_USO4] Requires VIRTIO_NET_F_GUEST_CSUM.
\item[VIRTIO_NET_F_GUEST_USO6] Requires VIRTIO_NET_F_GUEST_CSUM.
\item[VIRTIO_NET_F_GUEST_UDP_TUNNEL_GSO] Requires VIRTIO_NET_F_GUEST_TSO4, VIRTIO_NET_F_GUEST_TSO6,
   VIRTIO_NET_F_GUEST_USO4 and VIRTIO_NET_F_GUEST_USO6.
\item[VIRTIO_NET_F_GUEST_UDP_TUNNEL_GSO_CSUM] Requires VIRTIO_NET_F_GUEST_UDP_TUNNEL_GSO

\item[VIRTIO_NET_F_HOST_TSO4] Requires VIRTIO_NET_F_CSUM.
\item[VIRTIO_NET_F_HOST_TSO6] Requires VIRTIO_NET_F_CSUM.
\item[VIRTIO_NET_F_HOST_ECN] Requires VIRTIO_NET_F_HOST_TSO4 or VIRTIO_NET_F_HOST_TSO6.
\item[VIRTIO_NET_F_HOST_UFO] Requires VIRTIO_NET_F_CSUM.
\item[VIRTIO_NET_F_HOST_USO] Requires VIRTIO_NET_F_CSUM.
\item[VIRTIO_NET_F_HOST_UDP_TUNNEL_GSO] Requires VIRTIO_NET_F_HOST_TSO4, VIRTIO_NET_F_HOST_TSO6
   and VIRTIO_NET_F_HOST_USO.
\item[VIRTIO_NET_F_HOST_UDP_TUNNEL_GSO_CSUM] Requires VIRTIO_NET_F_HOST_UDP_TUNNEL_GSO

\item[VIRTIO_NET_F_CTRL_RX] Requires VIRTIO_NET_F_CTRL_VQ.
\item[VIRTIO_NET_F_CTRL_VLAN] Requires VIRTIO_NET_F_CTRL_VQ.
\item[VIRTIO_NET_F_GUEST_ANNOUNCE] Requires VIRTIO_NET_F_CTRL_VQ.
\item[VIRTIO_NET_F_MQ] Requires VIRTIO_NET_F_CTRL_VQ.
\item[VIRTIO_NET_F_CTRL_MAC_ADDR] Requires VIRTIO_NET_F_CTRL_VQ.
\item[VIRTIO_NET_F_NOTF_COAL] Requires VIRTIO_NET_F_CTRL_VQ.
\item[VIRTIO_NET_F_RSC_EXT] Requires VIRTIO_NET_F_HOST_TSO4 or VIRTIO_NET_F_HOST_TSO6.
\item[VIRTIO_NET_F_RSS] Requires VIRTIO_NET_F_CTRL_VQ.
\item[VIRTIO_NET_F_VQ_NOTF_COAL] Requires VIRTIO_NET_F_CTRL_VQ.
\item[VIRTIO_NET_F_HASH_TUNNEL] Requires VIRTIO_NET_F_CTRL_VQ along with VIRTIO_NET_F_RSS or VIRTIO_NET_F_HASH_REPORT.
\item[VIRTIO_NET_F_RSS_CONTEXT] Requires VIRTIO_NET_F_CTRL_VQ and VIRTIO_NET_F_RSS.
\end{description}

\begin{note}
The dependency between UDP_TUNNEL_GSO_CSUM and UDP_TUNNEL_GSO is intentionally
in the opposite direction with respect to the plain GSO features and the plain
checksum offload because UDP tunnel checksum offload gives very little gain
for non GSO packets and is quite complex to implement in H/W.
\end{note}

\subsubsection{Legacy Interface: Feature bits}\label{sec:Device Types / Network Device / Feature bits / Legacy Interface: Feature bits}
\begin{description}
\item[VIRTIO_NET_F_GSO (6)] Device handles packets with any GSO type. This was supposed to indicate segmentation offload support, but
upon further investigation it became clear that multiple bits were needed.
\item[VIRTIO_NET_F_GUEST_RSC4 (41)] Device coalesces TCPIP v4 packets. This was implemented by hypervisor patch for certification
purposes and current Windows driver depends on it. It will not function if virtio-net device reports this feature.
\item[VIRTIO_NET_F_GUEST_RSC6 (42)] Device coalesces TCPIP v6 packets. Similar to VIRTIO_NET_F_GUEST_RSC4.
\end{description}

\subsection{Device configuration layout}\label{sec:Device Types / Network Device / Device configuration layout}
\label{sec:Device Types / Block Device / Feature bits / Device configuration layout}

The network device has the following device configuration layout.
All of the device configuration fields are read-only for the driver.

\begin{lstlisting}
struct virtio_net_config {
        u8 mac[6];
        le16 status;
        le16 max_virtqueue_pairs;
        le16 mtu;
        le32 speed;
        u8 duplex;
        u8 rss_max_key_size;
        le16 rss_max_indirection_table_length;
        le32 supported_hash_types;
        le32 supported_tunnel_types;
};
\end{lstlisting}

The \field{mac} address field always exists (although it is only
valid if VIRTIO_NET_F_MAC is set).

The \field{status} only exists if VIRTIO_NET_F_STATUS is set.
Two bits are currently defined for the status field: VIRTIO_NET_S_LINK_UP
and VIRTIO_NET_S_ANNOUNCE.

\begin{lstlisting}
#define VIRTIO_NET_S_LINK_UP     1
#define VIRTIO_NET_S_ANNOUNCE    2
\end{lstlisting}

The following field, \field{max_virtqueue_pairs} only exists if
VIRTIO_NET_F_MQ or VIRTIO_NET_F_RSS is set. This field specifies the maximum number
of each of transmit and receive virtqueues (receiveq1\ldots receiveqN
and transmitq1\ldots transmitqN respectively) that can be configured once at least one of these features
is negotiated.

The following field, \field{mtu} only exists if VIRTIO_NET_F_MTU
is set. This field specifies the maximum MTU for the driver to
use.

The following two fields, \field{speed} and \field{duplex}, only
exist if VIRTIO_NET_F_SPEED_DUPLEX is set.

\field{speed} contains the device speed, in units of 1 MBit per
second, 0 to 0x7fffffff, or 0xffffffff for unknown speed.

\field{duplex} has the values of 0x01 for full duplex, 0x00 for
half duplex and 0xff for unknown duplex state.

Both \field{speed} and \field{duplex} can change, thus the driver
is expected to re-read these values after receiving a
configuration change notification.

The following field, \field{rss_max_key_size} only exists if VIRTIO_NET_F_RSS or VIRTIO_NET_F_HASH_REPORT is set.
It specifies the maximum supported length of RSS key in bytes.

The following field, \field{rss_max_indirection_table_length} only exists if VIRTIO_NET_F_RSS is set.
It specifies the maximum number of 16-bit entries in RSS indirection table.

The next field, \field{supported_hash_types} only exists if the device supports hash calculation,
i.e. if VIRTIO_NET_F_RSS or VIRTIO_NET_F_HASH_REPORT is set.

Field \field{supported_hash_types} contains the bitmask of supported hash types.
See \ref{sec:Device Types / Network Device / Device Operation / Processing of Incoming Packets / Hash calculation for incoming packets / Supported/enabled hash types} for details of supported hash types.

Field \field{supported_tunnel_types} only exists if the device supports inner header hash, i.e. if VIRTIO_NET_F_HASH_TUNNEL is set.

Field \field{supported_tunnel_types} contains the bitmask of encapsulation types supported by the device for inner header hash.
Encapsulation types are defined in \ref{sec:Device Types / Network Device / Device Operation / Processing of Incoming Packets /
Hash calculation for incoming packets / Encapsulation types supported/enabled for inner header hash}.

\devicenormative{\subsubsection}{Device configuration layout}{Device Types / Network Device / Device configuration layout}

The device MUST set \field{max_virtqueue_pairs} to between 1 and 0x8000 inclusive,
if it offers VIRTIO_NET_F_MQ.

The device MUST set \field{mtu} to between 68 and 65535 inclusive,
if it offers VIRTIO_NET_F_MTU.

The device SHOULD set \field{mtu} to at least 1280, if it offers
VIRTIO_NET_F_MTU.

The device MUST NOT modify \field{mtu} once it has been set.

The device MUST NOT pass received packets that exceed \field{mtu} (plus low
level ethernet header length) size with \field{gso_type} NONE or ECN
after VIRTIO_NET_F_MTU has been successfully negotiated.

The device MUST forward transmitted packets of up to \field{mtu} (plus low
level ethernet header length) size with \field{gso_type} NONE or ECN, and do
so without fragmentation, after VIRTIO_NET_F_MTU has been successfully
negotiated.

The device MUST set \field{rss_max_key_size} to at least 40, if it offers
VIRTIO_NET_F_RSS or VIRTIO_NET_F_HASH_REPORT.

The device MUST set \field{rss_max_indirection_table_length} to at least 128, if it offers
VIRTIO_NET_F_RSS.

If the driver negotiates the VIRTIO_NET_F_STANDBY feature, the device MAY act
as a standby device for a primary device with the same MAC address.

If VIRTIO_NET_F_SPEED_DUPLEX has been negotiated, \field{speed}
MUST contain the device speed, in units of 1 MBit per second, 0 to
0x7ffffffff, or 0xfffffffff for unknown.

If VIRTIO_NET_F_SPEED_DUPLEX has been negotiated, \field{duplex}
MUST have the values of 0x00 for full duplex, 0x01 for half
duplex, or 0xff for unknown.

If VIRTIO_NET_F_SPEED_DUPLEX and VIRTIO_NET_F_STATUS have both
been negotiated, the device SHOULD NOT change the \field{speed} and
\field{duplex} fields as long as VIRTIO_NET_S_LINK_UP is set in
the \field{status}.

The device SHOULD NOT offer VIRTIO_NET_F_HASH_REPORT if it
does not offer VIRTIO_NET_F_CTRL_VQ.

The device SHOULD NOT offer VIRTIO_NET_F_CTRL_RX_EXTRA if it
does not offer VIRTIO_NET_F_CTRL_VQ.

\drivernormative{\subsubsection}{Device configuration layout}{Device Types / Network Device / Device configuration layout}

The driver MUST NOT write to any of the device configuration fields.

A driver SHOULD negotiate VIRTIO_NET_F_MAC if the device offers it.
If the driver negotiates the VIRTIO_NET_F_MAC feature, the driver MUST set
the physical address of the NIC to \field{mac}.  Otherwise, it SHOULD
use a locally-administered MAC address (see \hyperref[intro:IEEE 802]{IEEE 802},
``9.2 48-bit universal LAN MAC addresses'').

If the driver does not negotiate the VIRTIO_NET_F_STATUS feature, it SHOULD
assume the link is active, otherwise it SHOULD read the link status from
the bottom bit of \field{status}.

A driver SHOULD negotiate VIRTIO_NET_F_MTU if the device offers it.

If the driver negotiates VIRTIO_NET_F_MTU, it MUST supply enough receive
buffers to receive at least one receive packet of size \field{mtu} (plus low
level ethernet header length) with \field{gso_type} NONE or ECN.

If the driver negotiates VIRTIO_NET_F_MTU, it MUST NOT transmit packets of
size exceeding the value of \field{mtu} (plus low level ethernet header length)
with \field{gso_type} NONE or ECN.

A driver SHOULD negotiate the VIRTIO_NET_F_STANDBY feature if the device offers it.

If VIRTIO_NET_F_SPEED_DUPLEX has been negotiated,
the driver MUST treat any value of \field{speed} above
0x7fffffff as well as any value of \field{duplex} not
matching 0x00 or 0x01 as an unknown value.

If VIRTIO_NET_F_SPEED_DUPLEX has been negotiated, the driver
SHOULD re-read \field{speed} and \field{duplex} after a
configuration change notification.

A driver SHOULD NOT negotiate VIRTIO_NET_F_HASH_REPORT if it
does not negotiate VIRTIO_NET_F_CTRL_VQ.

A driver SHOULD NOT negotiate VIRTIO_NET_F_CTRL_RX_EXTRA if it
does not negotiate VIRTIO_NET_F_CTRL_VQ.

\subsubsection{Legacy Interface: Device configuration layout}\label{sec:Device Types / Network Device / Device configuration layout / Legacy Interface: Device configuration layout}
\label{sec:Device Types / Block Device / Feature bits / Device configuration layout / Legacy Interface: Device configuration layout}
When using the legacy interface, transitional devices and drivers
MUST format \field{status} and
\field{max_virtqueue_pairs} in struct virtio_net_config
according to the native endian of the guest rather than
(necessarily when not using the legacy interface) little-endian.

When using the legacy interface, \field{mac} is driver-writable
which provided a way for drivers to update the MAC without
negotiating VIRTIO_NET_F_CTRL_MAC_ADDR.

\subsection{Device Initialization}\label{sec:Device Types / Network Device / Device Initialization}

A driver would perform a typical initialization routine like so:

\begin{enumerate}
\item Identify and initialize the receive and
  transmission virtqueues, up to N of each kind. If
  VIRTIO_NET_F_MQ feature bit is negotiated,
  N=\field{max_virtqueue_pairs}, otherwise identify N=1.

\item If the VIRTIO_NET_F_CTRL_VQ feature bit is negotiated,
  identify the control virtqueue.

\item Fill the receive queues with buffers: see \ref{sec:Device Types / Network Device / Device Operation / Setting Up Receive Buffers}.

\item Even with VIRTIO_NET_F_MQ, only receiveq1, transmitq1 and
  controlq are used by default.  The driver would send the
  VIRTIO_NET_CTRL_MQ_VQ_PAIRS_SET command specifying the
  number of the transmit and receive queues to use.

\item If the VIRTIO_NET_F_MAC feature bit is set, the configuration
  space \field{mac} entry indicates the ``physical'' address of the
  device, otherwise the driver would typically generate a random
  local MAC address.

\item If the VIRTIO_NET_F_STATUS feature bit is negotiated, the link
  status comes from the bottom bit of \field{status}.
  Otherwise, the driver assumes it's active.

\item A performant driver would indicate that it will generate checksumless
  packets by negotiating the VIRTIO_NET_F_CSUM feature.

\item If that feature is negotiated, a driver can use TCP segmentation or UDP
  segmentation/fragmentation offload by negotiating the VIRTIO_NET_F_HOST_TSO4 (IPv4
  TCP), VIRTIO_NET_F_HOST_TSO6 (IPv6 TCP), VIRTIO_NET_F_HOST_UFO
  (UDP fragmentation) and VIRTIO_NET_F_HOST_USO (UDP segmentation) features.

\item If the VIRTIO_NET_F_HOST_TSO6, VIRTIO_NET_F_HOST_TSO4 and VIRTIO_NET_F_HOST_USO
  segmentation features are negotiated, a driver can
  use TCP segmentation or UDP segmentation on top of UDP encapsulation
  offload, when the outer header does not require checksumming - e.g.
  the outer UDP checksum is zero - by negotiating the
  VIRTIO_NET_F_HOST_UDP_TUNNEL_GSO feature.
  GSO over UDP tunnels packets carry two sets of headers: the outer ones
  and the inner ones. The outer transport protocol is UDP, the inner
  could be either TCP or UDP. Only a single level of encapsulation
  offload is supported.

\item If VIRTIO_NET_F_HOST_UDP_TUNNEL_GSO is negotiated, a driver can
  additionally use TCP segmentation or UDP segmentation on top of UDP
  encapsulation with the outer header requiring checksum offload,
  negotiating the VIRTIO_NET_F_HOST_UDP_TUNNEL_GSO_CSUM feature.

\item The converse features are also available: a driver can save
  the virtual device some work by negotiating these features.\note{For example, a network packet transported between two guests on
the same system might not need checksumming at all, nor segmentation,
if both guests are amenable.}
   The VIRTIO_NET_F_GUEST_CSUM feature indicates that partially
  checksummed packets can be received, and if it can do that then
  the VIRTIO_NET_F_GUEST_TSO4, VIRTIO_NET_F_GUEST_TSO6,
  VIRTIO_NET_F_GUEST_UFO, VIRTIO_NET_F_GUEST_ECN, VIRTIO_NET_F_GUEST_USO4,
  VIRTIO_NET_F_GUEST_USO6 VIRTIO_NET_F_GUEST_UDP_TUNNEL_GSO and
  VIRTIO_NET_F_GUEST_UDP_TUNNEL_GSO_CSUM are the input equivalents of
  the features described above.
  See \ref{sec:Device Types / Network Device / Device Operation /
Setting Up Receive Buffers}~\nameref{sec:Device Types / Network
Device / Device Operation / Setting Up Receive Buffers} and
\ref{sec:Device Types / Network Device / Device Operation /
Processing of Incoming Packets}~\nameref{sec:Device Types /
Network Device / Device Operation / Processing of Incoming Packets} below.
\end{enumerate}

A truly minimal driver would only accept VIRTIO_NET_F_MAC and ignore
everything else.

\subsection{Device and driver capabilities}\label{sec:Device Types / Network Device / Device and driver capabilities}

The network device has the following capabilities.

\begin{tabularx}{\textwidth}{ |l||l|X| }
\hline
Identifier & Name & Description \\
\hline \hline
0x0800 & \hyperref[par:Device Types / Network Device / Device Operation / Flow filter / Device and driver capabilities / VIRTIO-NET-FF-RESOURCE-CAP]{VIRTIO_NET_FF_RESOURCE_CAP} & Flow filter resource capability \\
\hline
0x0801 & \hyperref[par:Device Types / Network Device / Device Operation / Flow filter / Device and driver capabilities / VIRTIO-NET-FF-SELECTOR-CAP]{VIRTIO_NET_FF_SELECTOR_CAP} & Flow filter classifier capability \\
\hline
0x0802 & \hyperref[par:Device Types / Network Device / Device Operation / Flow filter / Device and driver capabilities / VIRTIO-NET-FF-ACTION-CAP]{VIRTIO_NET_FF_ACTION_CAP} & Flow filter action capability \\
\hline
\end{tabularx}

\subsection{Device resource objects}\label{sec:Device Types / Network Device / Device resource objects}

The network device has the following resource objects.

\begin{tabularx}{\textwidth}{ |l||l|X| }
\hline
type & Name & Description \\
\hline \hline
0x0200 & \hyperref[par:Device Types / Network Device / Device Operation / Flow filter / Resource objects / VIRTIO-NET-RESOURCE-OBJ-FF-GROUP]{VIRTIO_NET_RESOURCE_OBJ_FF_GROUP} & Flow filter group resource object \\
\hline
0x0201 & \hyperref[par:Device Types / Network Device / Device Operation / Flow filter / Resource objects / VIRTIO-NET-RESOURCE-OBJ-FF-CLASSIFIER]{VIRTIO_NET_RESOURCE_OBJ_FF_CLASSIFIER} & Flow filter mask object \\
\hline
0x0202 & \hyperref[par:Device Types / Network Device / Device Operation / Flow filter / Resource objects / VIRTIO-NET-RESOURCE-OBJ-FF-RULE]{VIRTIO_NET_RESOURCE_OBJ_FF_RULE} & Flow filter rule object \\
\hline
\end{tabularx}

\subsection{Device parts}\label{sec:Device Types / Network Device / Device parts}

Network device parts represent the configuration done by the driver using control
virtqueue commands. Network device part is in the format of
\field{struct virtio_dev_part}.

\begin{tabularx}{\textwidth}{ |l||l|X| }
\hline
Type & Name & Description \\
\hline \hline
0x200 & VIRTIO_NET_DEV_PART_CVQ_CFG_PART & Represents device configuration done through a control virtqueue command, see \ref{sec:Device Types / Network Device / Device parts / VIRTIO-NET-DEV-PART-CVQ-CFG-PART} \\
\hline
0x201 - 0x5FF & - & reserved for future \\
\hline
\hline
\end{tabularx}

\subsubsection{VIRTIO_NET_DEV_PART_CVQ_CFG_PART}\label{sec:Device Types / Network Device / Device parts / VIRTIO-NET-DEV-PART-CVQ-CFG-PART}

For VIRTIO_NET_DEV_PART_CVQ_CFG_PART, \field{part_type} is set to 0x200. The
VIRTIO_NET_DEV_PART_CVQ_CFG_PART part indicates configuration performed by the
driver using a control virtqueue command.

\begin{lstlisting}
struct virtio_net_dev_part_cvq_selector {
        u8 class;
        u8 command;
        u8 reserved[6];
};
\end{lstlisting}

There is one device part of type VIRTIO_NET_DEV_PART_CVQ_CFG_PART for each
individual configuration. Each part is identified by a unique selector value.
The selector, \field{device_type_raw}, is in the format
\field{struct virtio_net_dev_part_cvq_selector}.

The selector consists of two fields: \field{class} and \field{command}. These
fields correspond to the \field{class} and \field{command} defined in
\field{struct virtio_net_ctrl}, as described in the relevant sections of
\ref{sec:Device Types / Network Device / Device Operation / Control Virtqueue}.

The value corresponding to each part’s selector follows the same format as the
respective \field{command-specific-data} described in the relevant sections of
\ref{sec:Device Types / Network Device / Device Operation / Control Virtqueue}.

For example, when the \field{class} is VIRTIO_NET_CTRL_MAC, the \field{command}
can be either VIRTIO_NET_CTRL_MAC_TABLE_SET or VIRTIO_NET_CTRL_MAC_ADDR_SET;
when \field{command} is set to VIRTIO_NET_CTRL_MAC_TABLE_SET, \field{value}
is in the format of \field{struct virtio_net_ctrl_mac}.

Supported selectors are listed in the table:

\begin{tabularx}{\textwidth}{ |l|X| }
\hline
Class selector & Command selector \\
\hline \hline
VIRTIO_NET_CTRL_RX & VIRTIO_NET_CTRL_RX_PROMISC \\
\hline
VIRTIO_NET_CTRL_RX & VIRTIO_NET_CTRL_RX_ALLMULTI \\
\hline
VIRTIO_NET_CTRL_RX & VIRTIO_NET_CTRL_RX_ALLUNI \\
\hline
VIRTIO_NET_CTRL_RX & VIRTIO_NET_CTRL_RX_NOMULTI \\
\hline
VIRTIO_NET_CTRL_RX & VIRTIO_NET_CTRL_RX_NOUNI \\
\hline
VIRTIO_NET_CTRL_RX & VIRTIO_NET_CTRL_RX_NOBCAST \\
\hline
VIRTIO_NET_CTRL_MAC & VIRTIO_NET_CTRL_MAC_TABLE_SET \\
\hline
VIRTIO_NET_CTRL_MAC & VIRTIO_NET_CTRL_MAC_ADDR_SET \\
\hline
VIRTIO_NET_CTRL_VLAN & VIRTIO_NET_CTRL_VLAN_ADD \\
\hline
VIRTIO_NET_CTRL_ANNOUNCE & VIRTIO_NET_CTRL_ANNOUNCE_ACK \\
\hline
VIRTIO_NET_CTRL_MQ & VIRTIO_NET_CTRL_MQ_VQ_PAIRS_SET \\
\hline
VIRTIO_NET_CTRL_MQ & VIRTIO_NET_CTRL_MQ_RSS_CONFIG \\
\hline
VIRTIO_NET_CTRL_MQ & VIRTIO_NET_CTRL_MQ_HASH_CONFIG \\
\hline
\hline
\end{tabularx}

For command selector VIRTIO_NET_CTRL_VLAN_ADD, device part consists of a whole
VLAN table.

\field{reserved} is reserved and set to zero.

\subsection{Device Operation}\label{sec:Device Types / Network Device / Device Operation}

Packets are transmitted by placing them in the
transmitq1\ldots transmitqN, and buffers for incoming packets are
placed in the receiveq1\ldots receiveqN. In each case, the packet
itself is preceded by a header:

\begin{lstlisting}
struct virtio_net_hdr {
#define VIRTIO_NET_HDR_F_NEEDS_CSUM    1
#define VIRTIO_NET_HDR_F_DATA_VALID    2
#define VIRTIO_NET_HDR_F_RSC_INFO      4
#define VIRTIO_NET_HDR_F_UDP_TUNNEL_CSUM 8
        u8 flags;
#define VIRTIO_NET_HDR_GSO_NONE        0
#define VIRTIO_NET_HDR_GSO_TCPV4       1
#define VIRTIO_NET_HDR_GSO_UDP         3
#define VIRTIO_NET_HDR_GSO_TCPV6       4
#define VIRTIO_NET_HDR_GSO_UDP_L4      5
#define VIRTIO_NET_HDR_GSO_UDP_TUNNEL_IPV4 0x20
#define VIRTIO_NET_HDR_GSO_UDP_TUNNEL_IPV6 0x40
#define VIRTIO_NET_HDR_GSO_ECN      0x80
        u8 gso_type;
        le16 hdr_len;
        le16 gso_size;
        le16 csum_start;
        le16 csum_offset;
        le16 num_buffers;
        le32 hash_value;        (Only if VIRTIO_NET_F_HASH_REPORT negotiated)
        le16 hash_report;       (Only if VIRTIO_NET_F_HASH_REPORT negotiated)
        le16 padding_reserved;  (Only if VIRTIO_NET_F_HASH_REPORT negotiated)
        le16 outer_th_offset    (Only if VIRTIO_NET_F_HOST_UDP_TUNNEL_GSO or VIRTIO_NET_F_GUEST_UDP_TUNNEL_GSO negotiated)
        le16 inner_nh_offset;   (Only if VIRTIO_NET_F_HOST_UDP_TUNNEL_GSO or VIRTIO_NET_F_GUEST_UDP_TUNNEL_GSO negotiated)
};
\end{lstlisting}

The controlq is used to control various device features described further in
section \ref{sec:Device Types / Network Device / Device Operation / Control Virtqueue}.

\subsubsection{Legacy Interface: Device Operation}\label{sec:Device Types / Network Device / Device Operation / Legacy Interface: Device Operation}
When using the legacy interface, transitional devices and drivers
MUST format the fields in \field{struct virtio_net_hdr}
according to the native endian of the guest rather than
(necessarily when not using the legacy interface) little-endian.

The legacy driver only presented \field{num_buffers} in the \field{struct virtio_net_hdr}
when VIRTIO_NET_F_MRG_RXBUF was negotiated; without that feature the
structure was 2 bytes shorter.

When using the legacy interface, the driver SHOULD ignore the
used length for the transmit queues
and the controlq queue.
\begin{note}
Historically, some devices put
the total descriptor length there, even though no data was
actually written.
\end{note}

\subsubsection{Packet Transmission}\label{sec:Device Types / Network Device / Device Operation / Packet Transmission}

Transmitting a single packet is simple, but varies depending on
the different features the driver negotiated.

\begin{enumerate}
\item The driver can send a completely checksummed packet.  In this case,
  \field{flags} will be zero, and \field{gso_type} will be VIRTIO_NET_HDR_GSO_NONE.

\item If the driver negotiated VIRTIO_NET_F_CSUM, it can skip
  checksumming the packet:
  \begin{itemize}
  \item \field{flags} has the VIRTIO_NET_HDR_F_NEEDS_CSUM set,

  \item \field{csum_start} is set to the offset within the packet to begin checksumming,
    and

  \item \field{csum_offset} indicates how many bytes after the csum_start the
    new (16 bit ones' complement) checksum is placed by the device.

  \item The TCP checksum field in the packet is set to the sum
    of the TCP pseudo header, so that replacing it by the ones'
    complement checksum of the TCP header and body will give the
    correct result.
  \end{itemize}

\begin{note}
For example, consider a partially checksummed TCP (IPv4) packet.
It will have a 14 byte ethernet header and 20 byte IP header
followed by the TCP header (with the TCP checksum field 16 bytes
into that header). \field{csum_start} will be 14+20 = 34 (the TCP
checksum includes the header), and \field{csum_offset} will be 16.
If the given packet has the VIRTIO_NET_HDR_GSO_UDP_TUNNEL_IPV4 bit or the
VIRTIO_NET_HDR_GSO_UDP_TUNNEL_IPV6 bit set,
the above checksum fields refer to the inner header checksum, see
the example below.
\end{note}

\item If the driver negotiated
  VIRTIO_NET_F_HOST_TSO4, TSO6, USO or UFO, and the packet requires
  TCP segmentation, UDP segmentation or fragmentation, then \field{gso_type}
  is set to VIRTIO_NET_HDR_GSO_TCPV4, TCPV6, UDP_L4 or UDP.
  (Otherwise, it is set to VIRTIO_NET_HDR_GSO_NONE). In this
  case, packets larger than 1514 bytes can be transmitted: the
  metadata indicates how to replicate the packet header to cut it
  into smaller packets. The other gso fields are set:

  \begin{itemize}
  \item If the VIRTIO_NET_F_GUEST_HDRLEN feature has been negotiated,
    \field{hdr_len} indicates the header length that needs to be replicated
    for each packet. It's the number of bytes from the beginning of the packet
    to the beginning of the transport payload.
    If the \field{gso_type} has the VIRTIO_NET_HDR_GSO_UDP_TUNNEL_IPV4 bit or
    VIRTIO_NET_HDR_GSO_UDP_TUNNEL_IPV6 bit set, \field{hdr_len} accounts for
    all the headers up to and including the inner transport.
    Otherwise, if the VIRTIO_NET_F_GUEST_HDRLEN feature has not been negotiated,
    \field{hdr_len} is a hint to the device as to how much of the header
    needs to be kept to copy into each packet, usually set to the
    length of the headers, including the transport header\footnote{Due to various bugs in implementations, this field is not useful
as a guarantee of the transport header size.
}.

  \begin{note}
  Some devices benefit from knowledge of the exact header length.
  \end{note}

  \item \field{gso_size} is the maximum size of each packet beyond that
    header (ie. MSS).

  \item If the driver negotiated the VIRTIO_NET_F_HOST_ECN feature,
    the VIRTIO_NET_HDR_GSO_ECN bit in \field{gso_type}
    indicates that the TCP packet has the ECN bit set\footnote{This case is not handled by some older hardware, so is called out
specifically in the protocol.}.
   \end{itemize}

\item If the driver negotiated the VIRTIO_NET_F_HOST_UDP_TUNNEL_GSO feature and the
  \field{gso_type} has the VIRTIO_NET_HDR_GSO_UDP_TUNNEL_IPV4 bit or
  VIRTIO_NET_HDR_GSO_UDP_TUNNEL_IPV6 bit set, the GSO protocol is encapsulated
  in a UDP tunnel.
  If the outer UDP header requires checksumming, the driver must have
  additionally negotiated the VIRTIO_NET_F_HOST_UDP_TUNNEL_GSO_CSUM feature
  and offloaded the outer checksum accordingly, otherwise
  the outer UDP header must not require checksum validation, i.e. the outer
  UDP checksum must be positive zero (0x0) as defined in UDP RFC 768.
  The other tunnel-related fields indicate how to replicate the packet
  headers to cut it into smaller packets:

  \begin{itemize}
  \item \field{outer_th_offset} field indicates the outer transport header within
      the packet. This field differs from \field{csum_start} as the latter
      points to the inner transport header within the packet.

  \item \field{inner_nh_offset} field indicates the inner network header within
      the packet.
  \end{itemize}

\begin{note}
For example, consider a partially checksummed TCP (IPv4) packet carried over a
Geneve UDP tunnel (again IPv4) with no tunnel options. The
only relevant variable related to the tunnel type is the tunnel header length.
The packet will have a 14 byte outer ethernet header, 20 byte outer IP header
followed by the 8 byte UDP header (with a 0 checksum value), 8 byte Geneve header,
14 byte inner ethernet header, 20 byte inner IP header
and the TCP header (with the TCP checksum field 16 bytes
into that header). \field{csum_start} will be 14+20+8+8+14+20 = 84 (the TCP
checksum includes the header), \field{csum_offset} will be 16.
\field{inner_nh_offset} will be 14+20+8+8+14 = 62, \field{outer_th_offset} will be
14+20+8 = 42 and \field{gso_type} will be
VIRTIO_NET_HDR_GSO_TCPV4 | VIRTIO_NET_HDR_GSO_UDP_TUNNEL_IPV4 = 0x21
\end{note}

\item If the driver negotiated the VIRTIO_NET_F_HOST_UDP_TUNNEL_GSO_CSUM feature,
  the transmitted packet is a GSO one encapsulated in a UDP tunnel, and
  the outer UDP header requires checksumming, the driver can skip checksumming
  the outer header:

  \begin{itemize}
  \item \field{flags} has the VIRTIO_NET_HDR_F_UDP_TUNNEL_CSUM set,

  \item The outer UDP checksum field in the packet is set to the sum
    of the UDP pseudo header, so that replacing it by the ones'
    complement checksum of the outer UDP header and payload will give the
    correct result.
  \end{itemize}

\item \field{num_buffers} is set to zero.  This field is unused on transmitted packets.

\item The header and packet are added as one output descriptor to the
  transmitq, and the device is notified of the new entry
  (see \ref{sec:Device Types / Network Device / Device Initialization}~\nameref{sec:Device Types / Network Device / Device Initialization}).
\end{enumerate}

\drivernormative{\paragraph}{Packet Transmission}{Device Types / Network Device / Device Operation / Packet Transmission}

For the transmit packet buffer, the driver MUST use the size of the
structure \field{struct virtio_net_hdr} same as the receive packet buffer.

The driver MUST set \field{num_buffers} to zero.

If VIRTIO_NET_F_CSUM is not negotiated, the driver MUST set
\field{flags} to zero and SHOULD supply a fully checksummed
packet to the device.

If VIRTIO_NET_F_HOST_TSO4 is negotiated, the driver MAY set
\field{gso_type} to VIRTIO_NET_HDR_GSO_TCPV4 to request TCPv4
segmentation, otherwise the driver MUST NOT set
\field{gso_type} to VIRTIO_NET_HDR_GSO_TCPV4.

If VIRTIO_NET_F_HOST_TSO6 is negotiated, the driver MAY set
\field{gso_type} to VIRTIO_NET_HDR_GSO_TCPV6 to request TCPv6
segmentation, otherwise the driver MUST NOT set
\field{gso_type} to VIRTIO_NET_HDR_GSO_TCPV6.

If VIRTIO_NET_F_HOST_UFO is negotiated, the driver MAY set
\field{gso_type} to VIRTIO_NET_HDR_GSO_UDP to request UDP
fragmentation, otherwise the driver MUST NOT set
\field{gso_type} to VIRTIO_NET_HDR_GSO_UDP.

If VIRTIO_NET_F_HOST_USO is negotiated, the driver MAY set
\field{gso_type} to VIRTIO_NET_HDR_GSO_UDP_L4 to request UDP
segmentation, otherwise the driver MUST NOT set
\field{gso_type} to VIRTIO_NET_HDR_GSO_UDP_L4.

The driver SHOULD NOT send to the device TCP packets requiring segmentation offload
which have the Explicit Congestion Notification bit set, unless the
VIRTIO_NET_F_HOST_ECN feature is negotiated, in which case the
driver MUST set the VIRTIO_NET_HDR_GSO_ECN bit in
\field{gso_type}.

If VIRTIO_NET_F_HOST_UDP_TUNNEL_GSO is negotiated, the driver MAY set
VIRTIO_NET_HDR_GSO_UDP_TUNNEL_IPV4 bit or the VIRTIO_NET_HDR_GSO_UDP_TUNNEL_IPV6 bit
in \field{gso_type} according to the inner network header protocol type
to request GSO packets over UDPv4 or UDPv6 tunnel segmentation,
otherwise the driver MUST NOT set either the
VIRTIO_NET_HDR_GSO_UDP_TUNNEL_IPV4 bit or the VIRTIO_NET_HDR_GSO_UDP_TUNNEL_IPV6 bit
in \field{gso_type}.

When requesting GSO segmentation over UDP tunnel, the driver MUST SET the
VIRTIO_NET_HDR_GSO_UDP_TUNNEL_IPV4 bit if the inner network header is IPv4, i.e. the
packet is a TCPv4 GSO one, otherwise, if the inner network header is IPv6, the driver
MUST SET the VIRTIO_NET_HDR_GSO_UDP_TUNNEL_IPV6 bit.

The driver MUST NOT send to the device GSO packets over UDP tunnel
requiring segmentation and outer UDP checksum offload, unless both the
VIRTIO_NET_F_HOST_UDP_TUNNEL_GSO and VIRTIO_NET_F_HOST_UDP_TUNNEL_GSO_CSUM features
are negotiated, in which case the driver MUST set either the
VIRTIO_NET_HDR_GSO_UDP_TUNNEL_IPV4 bit or the VIRTIO_NET_HDR_GSO_UDP_TUNNEL_IPV6
bit in the \field{gso_type} and the VIRTIO_NET_HDR_F_UDP_TUNNEL_CSUM bit in
the \field{flags}.

If VIRTIO_NET_F_HOST_UDP_TUNNEL_GSO_CSUM is not negotiated, the driver MUST not set
the VIRTIO_NET_HDR_F_UDP_TUNNEL_CSUM bit in the \field{flags} and
MUST NOT send to the device GSO packets over UDP tunnel
requiring segmentation and outer UDP checksum offload.

The driver MUST NOT set the VIRTIO_NET_HDR_GSO_UDP_TUNNEL_IPV4 bit or the
VIRTIO_NET_HDR_GSO_UDP_TUNNEL_IPV6 bit together with VIRTIO_NET_HDR_GSO_UDP, as the
latter is deprecated in favor of UDP_L4 and no new feature will support it.

The driver MUST NOT set the VIRTIO_NET_HDR_GSO_UDP_TUNNEL_IPV4 bit and the
VIRTIO_NET_HDR_GSO_UDP_TUNNEL_IPV6 bit together.

The driver MUST NOT set the VIRTIO_NET_HDR_F_UDP_TUNNEL_CSUM bit \field{flags}
without setting either the VIRTIO_NET_HDR_GSO_UDP_TUNNEL_IPV4 bit or
the VIRTIO_NET_HDR_GSO_UDP_TUNNEL_IPV6 bit in \field{gso_type}.

If the VIRTIO_NET_F_CSUM feature has been negotiated, the
driver MAY set the VIRTIO_NET_HDR_F_NEEDS_CSUM bit in
\field{flags}, if so:
\begin{enumerate}
\item the driver MUST validate the packet checksum at
	offset \field{csum_offset} from \field{csum_start} as well as all
	preceding offsets;
\begin{note}
If \field{gso_type} differs from VIRTIO_NET_HDR_GSO_NONE and the
VIRTIO_NET_HDR_GSO_UDP_TUNNEL_IPV4 bit or the VIRTIO_NET_HDR_GSO_UDP_TUNNEL_IPV6
bit are not set in \field{gso_type}, \field{csum_offset}
points to the only transport header present in the packet, and there are no
additional preceding checksums validated by VIRTIO_NET_HDR_F_NEEDS_CSUM.
\end{note}
\item the driver MUST set the packet checksum stored in the
	buffer to the TCP/UDP pseudo header;
\item the driver MUST set \field{csum_start} and
	\field{csum_offset} such that calculating a ones'
	complement checksum from \field{csum_start} up until the end of
	the packet and storing the result at offset \field{csum_offset}
	from  \field{csum_start} will result in a fully checksummed
	packet;
\end{enumerate}

If none of the VIRTIO_NET_F_HOST_TSO4, TSO6, USO or UFO options have
been negotiated, the driver MUST set \field{gso_type} to
VIRTIO_NET_HDR_GSO_NONE.

If \field{gso_type} differs from VIRTIO_NET_HDR_GSO_NONE, then
the driver MUST also set the VIRTIO_NET_HDR_F_NEEDS_CSUM bit in
\field{flags} and MUST set \field{gso_size} to indicate the
desired MSS.

If one of the VIRTIO_NET_F_HOST_TSO4, TSO6, USO or UFO options have
been negotiated:
\begin{itemize}
\item If the VIRTIO_NET_F_GUEST_HDRLEN feature has been negotiated,
	and \field{gso_type} differs from VIRTIO_NET_HDR_GSO_NONE,
	the driver MUST set \field{hdr_len} to a value equal to the length
	of the headers, including the transport header. If \field{gso_type}
	has the VIRTIO_NET_HDR_GSO_UDP_TUNNEL_IPV4 bit or the
	VIRTIO_NET_HDR_GSO_UDP_TUNNEL_IPV6 bit set, \field{hdr_len} includes
	the inner transport header.

\item If the VIRTIO_NET_F_GUEST_HDRLEN feature has not been negotiated,
	or \field{gso_type} is VIRTIO_NET_HDR_GSO_NONE,
	the driver SHOULD set \field{hdr_len} to a value
	not less than the length of the headers, including the transport
	header.
\end{itemize}

If the VIRTIO_NET_F_HOST_UDP_TUNNEL_GSO option has been negotiated, the
driver MAY set the VIRTIO_NET_HDR_GSO_UDP_TUNNEL_IPV4 bit or the
VIRTIO_NET_HDR_GSO_UDP_TUNNEL_IPV6 bit in \field{gso_type}, if so:
\begin{itemize}
\item the driver MUST set \field{outer_th_offset} to the outer UDP header
  offset and \field{inner_nh_offset} to the inner network header offset.
  The \field{csum_start} and \field{csum_offset} fields point respectively
  to the inner transport header and inner transport checksum field.
\end{itemize}

If the VIRTIO_NET_F_HOST_UDP_TUNNEL_GSO_CSUM feature has been negotiated,
and the VIRTIO_NET_HDR_GSO_UDP_TUNNEL_IPV4 bit or
VIRTIO_NET_HDR_GSO_UDP_TUNNEL_IPV6 bit in \field{gso_type} are set,
the driver MAY set the VIRTIO_NET_HDR_F_UDP_TUNNEL_CSUM bit in
\field{flags}, if so the driver MUST set the packet outer UDP header checksum
to the outer UDP pseudo header checksum.

\begin{note}
calculating a ones' complement checksum from \field{outer_th_offset}
up until the end of the packet and storing the result at offset 6
from \field{outer_th_offset} will result in a fully checksummed outer UDP packet;
\end{note}

If the VIRTIO_NET_HDR_GSO_UDP_TUNNEL_IPV4 bit or the
VIRTIO_NET_HDR_GSO_UDP_TUNNEL_IPV6 bit in \field{gso_type} are set
and the VIRTIO_NET_F_HOST_UDP_TUNNEL_GSO_CSUM feature has not
been negotiated, the
outer UDP header MUST NOT require checksum validation. That is, the
outer UDP checksum value MUST be 0 or the validated complete checksum
for such header.

\begin{note}
The valid complete checksum of the outer UDP header of individual segments
can be computed by the driver prior to segmentation only if the GSO packet
size is a multiple of \field{gso_size}, because then all segments
have the same size and thus all data included in the outer UDP
checksum is the same for every segment. These pre-computed segment
length and checksum fields are different from those of the GSO
packet.
In this scenario the outer UDP header of the GSO packet must carry the
segmented UDP packet length.
\end{note}

If the VIRTIO_NET_F_HOST_UDP_TUNNEL_GSO option has not
been negotiated, the driver MUST NOT set either the VIRTIO_NET_HDR_F_GSO_UDP_TUNNEL_IPV4
bit or the VIRTIO_NET_HDR_F_GSO_UDP_TUNNEL_IPV6 in \field{gso_type}.

If the VIRTIO_NET_F_HOST_UDP_TUNNEL_GSO_CSUM option has not been negotiated,
the driver MUST NOT set the VIRTIO_NET_HDR_F_UDP_TUNNEL_CSUM bit
in \field{flags}.

The driver SHOULD accept the VIRTIO_NET_F_GUEST_HDRLEN feature if it has
been offered, and if it's able to provide the exact header length.

The driver MUST NOT set the VIRTIO_NET_HDR_F_DATA_VALID and
VIRTIO_NET_HDR_F_RSC_INFO bits in \field{flags}.

The driver MUST NOT set the VIRTIO_NET_HDR_F_DATA_VALID bit in \field{flags}
together with the VIRTIO_NET_HDR_F_GSO_UDP_TUNNEL_IPV4 bit or the
VIRTIO_NET_HDR_F_GSO_UDP_TUNNEL_IPV6 bit in \field{gso_type}.

\devicenormative{\paragraph}{Packet Transmission}{Device Types / Network Device / Device Operation / Packet Transmission}
The device MUST ignore \field{flag} bits that it does not recognize.

If VIRTIO_NET_HDR_F_NEEDS_CSUM bit in \field{flags} is not set, the
device MUST NOT use the \field{csum_start} and \field{csum_offset}.

If one of the VIRTIO_NET_F_HOST_TSO4, TSO6, USO or UFO options have
been negotiated:
\begin{itemize}
\item If the VIRTIO_NET_F_GUEST_HDRLEN feature has been negotiated,
	and \field{gso_type} differs from VIRTIO_NET_HDR_GSO_NONE,
	the device MAY use \field{hdr_len} as the transport header size.

	\begin{note}
	Caution should be taken by the implementation so as to prevent
	a malicious driver from attacking the device by setting an incorrect hdr_len.
	\end{note}

\item If the VIRTIO_NET_F_GUEST_HDRLEN feature has not been negotiated,
	or \field{gso_type} is VIRTIO_NET_HDR_GSO_NONE,
	the device MAY use \field{hdr_len} only as a hint about the
	transport header size.
	The device MUST NOT rely on \field{hdr_len} to be correct.

	\begin{note}
	This is due to various bugs in implementations.
	\end{note}
\end{itemize}

If both the VIRTIO_NET_HDR_GSO_UDP_TUNNEL_IPV4 bit and
the VIRTIO_NET_HDR_GSO_UDP_TUNNEL_IPV6 bit in in \field{gso_type} are set,
the device MUST NOT accept the packet.

If the VIRTIO_NET_HDR_GSO_UDP_TUNNEL_IPV4 bit and the VIRTIO_NET_HDR_GSO_UDP_TUNNEL_IPV6
bit in \field{gso_type} are not set, the device MUST NOT use the
\field{outer_th_offset} and \field{inner_nh_offset}.

If either the VIRTIO_NET_HDR_GSO_UDP_TUNNEL_IPV4 bit or
the VIRTIO_NET_HDR_GSO_UDP_TUNNEL_IPV6 bit in \field{gso_type} are set, and any of
the following is true:
\begin{itemize}
\item the VIRTIO_NET_HDR_F_NEEDS_CSUM is not set in \field{flags}
\item the VIRTIO_NET_HDR_F_DATA_VALID is set in \field{flags}
\item the \field{gso_type} excluding the VIRTIO_NET_HDR_GSO_UDP_TUNNEL_IPV4
bit and the VIRTIO_NET_HDR_GSO_UDP_TUNNEL_IPV6 bit is VIRTIO_NET_HDR_GSO_NONE
\end{itemize}
the device MUST NOT accept the packet.

If the VIRTIO_NET_HDR_F_UDP_TUNNEL_CSUM bit in \field{flags} is set,
and both the bits VIRTIO_NET_HDR_GSO_UDP_TUNNEL_IPV4 and
VIRTIO_NET_HDR_GSO_UDP_TUNNEL_IPV6 in \field{gso_type} are not set,
the device MOST NOT accept the packet.

If VIRTIO_NET_HDR_F_NEEDS_CSUM is not set, the device MUST NOT
rely on the packet checksum being correct.
\paragraph{Packet Transmission Interrupt}\label{sec:Device Types / Network Device / Device Operation / Packet Transmission / Packet Transmission Interrupt}

Often a driver will suppress transmission virtqueue interrupts
and check for used packets in the transmit path of following
packets.

The normal behavior in this interrupt handler is to retrieve
used buffers from the virtqueue and free the corresponding
headers and packets.

\subsubsection{Setting Up Receive Buffers}\label{sec:Device Types / Network Device / Device Operation / Setting Up Receive Buffers}

It is generally a good idea to keep the receive virtqueue as
fully populated as possible: if it runs out, network performance
will suffer.

If the VIRTIO_NET_F_GUEST_TSO4, VIRTIO_NET_F_GUEST_TSO6,
VIRTIO_NET_F_GUEST_UFO, VIRTIO_NET_F_GUEST_USO4 or VIRTIO_NET_F_GUEST_USO6
features are used, the maximum incoming packet
will be 65589 bytes long (14 bytes of Ethernet header, plus 40 bytes of
the IPv6 header, plus 65535 bytes of maximum IPv6 payload including any
extension header), otherwise 1514 bytes.
When VIRTIO_NET_F_HASH_REPORT is not negotiated, the required receive buffer
size is either 65601 or 1526 bytes accounting for 20 bytes of
\field{struct virtio_net_hdr} followed by receive packet.
When VIRTIO_NET_F_HASH_REPORT is negotiated, the required receive buffer
size is either 65609 or 1534 bytes accounting for 12 bytes of
\field{struct virtio_net_hdr} followed by receive packet.

\drivernormative{\paragraph}{Setting Up Receive Buffers}{Device Types / Network Device / Device Operation / Setting Up Receive Buffers}

\begin{itemize}
\item If VIRTIO_NET_F_MRG_RXBUF is not negotiated:
  \begin{itemize}
    \item If VIRTIO_NET_F_GUEST_TSO4, VIRTIO_NET_F_GUEST_TSO6, VIRTIO_NET_F_GUEST_UFO,
	VIRTIO_NET_F_GUEST_USO4 or VIRTIO_NET_F_GUEST_USO6 are negotiated, the driver SHOULD populate
      the receive queue(s) with buffers of at least 65609 bytes if
      VIRTIO_NET_F_HASH_REPORT is negotiated, and of at least 65601 bytes if not.
    \item Otherwise, the driver SHOULD populate the receive queue(s)
      with buffers of at least 1534 bytes if VIRTIO_NET_F_HASH_REPORT
      is negotiated, and of at least 1526 bytes if not.
  \end{itemize}
\item If VIRTIO_NET_F_MRG_RXBUF is negotiated, each buffer MUST be at
least size of \field{struct virtio_net_hdr},
i.e. 20 bytes if VIRTIO_NET_F_HASH_REPORT is negotiated, and 12 bytes if not.
\end{itemize}

\begin{note}
Obviously each buffer can be split across multiple descriptor elements.
\end{note}

When calculating the size of \field{struct virtio_net_hdr}, the driver
MUST consider all the fields inclusive up to \field{padding_reserved},
i.e. 20 bytes if VIRTIO_NET_F_HASH_REPORT is negotiated, and 12 bytes if not.

If VIRTIO_NET_F_MQ is negotiated, each of receiveq1\ldots receiveqN
that will be used SHOULD be populated with receive buffers.

\devicenormative{\paragraph}{Setting Up Receive Buffers}{Device Types / Network Device / Device Operation / Setting Up Receive Buffers}

The device MUST set \field{num_buffers} to the number of descriptors used to
hold the incoming packet.

The device MUST use only a single descriptor if VIRTIO_NET_F_MRG_RXBUF
was not negotiated.
\begin{note}
{This means that \field{num_buffers} will always be 1
if VIRTIO_NET_F_MRG_RXBUF is not negotiated.}
\end{note}

\subsubsection{Processing of Incoming Packets}\label{sec:Device Types / Network Device / Device Operation / Processing of Incoming Packets}
\label{sec:Device Types / Network Device / Device Operation / Processing of Packets}%old label for latexdiff

When a packet is copied into a buffer in the receiveq, the
optimal path is to disable further used buffer notifications for the
receiveq and process packets until no more are found, then re-enable
them.

Processing incoming packets involves:

\begin{enumerate}
\item \field{num_buffers} indicates how many descriptors
  this packet is spread over (including this one): this will
  always be 1 if VIRTIO_NET_F_MRG_RXBUF was not negotiated.
  This allows receipt of large packets without having to allocate large
  buffers: a packet that does not fit in a single buffer can flow
  over to the next buffer, and so on. In this case, there will be
  at least \field{num_buffers} used buffers in the virtqueue, and the device
  chains them together to form a single packet in a way similar to
  how it would store it in a single buffer spread over multiple
  descriptors.
  The other buffers will not begin with a \field{struct virtio_net_hdr}.

\item If
  \field{num_buffers} is one, then the entire packet will be
  contained within this buffer, immediately following the struct
  virtio_net_hdr.
\item If the VIRTIO_NET_F_GUEST_CSUM feature was negotiated, the
  VIRTIO_NET_HDR_F_DATA_VALID bit in \field{flags} can be
  set: if so, device has validated the packet checksum.
  If the VIRTIO_NET_F_GUEST_UDP_TUNNEL_GSO_CSUM feature has been negotiated,
  and the VIRTIO_NET_HDR_F_UDP_TUNNEL_CSUM bit is set in \field{flags},
  both the outer UDP checksum and the inner transport checksum
  have been validated, otherwise only one level of checksums (the outer one
  in case of tunnels) has been validated.
\end{enumerate}

Additionally, VIRTIO_NET_F_GUEST_CSUM, TSO4, TSO6, UDP, UDP_TUNNEL
and ECN features enable receive checksum, large receive offload and ECN
support which are the input equivalents of the transmit checksum,
transmit segmentation offloading and ECN features, as described
in \ref{sec:Device Types / Network Device / Device Operation /
Packet Transmission}:
\begin{enumerate}
\item If the VIRTIO_NET_F_GUEST_TSO4, TSO6, UFO, USO4 or USO6 options were
  negotiated, then \field{gso_type} MAY be something other than
  VIRTIO_NET_HDR_GSO_NONE, and \field{gso_size} field indicates the
  desired MSS (see Packet Transmission point 2).
\item If the VIRTIO_NET_F_RSC_EXT option was negotiated (this
  implies one of VIRTIO_NET_F_GUEST_TSO4, TSO6), the
  device processes also duplicated ACK segments, reports
  number of coalesced TCP segments in \field{csum_start} field and
  number of duplicated ACK segments in \field{csum_offset} field
  and sets bit VIRTIO_NET_HDR_F_RSC_INFO in \field{flags}.
\item If the VIRTIO_NET_F_GUEST_CSUM feature was negotiated, the
  VIRTIO_NET_HDR_F_NEEDS_CSUM bit in \field{flags} can be
  set: if so, the packet checksum at offset \field{csum_offset}
  from \field{csum_start} and any preceding checksums
  have been validated.  The checksum on the packet is incomplete and
  if bit VIRTIO_NET_HDR_F_RSC_INFO is not set in \field{flags},
  then \field{csum_start} and \field{csum_offset} indicate how to calculate it
  (see Packet Transmission point 1).
\begin{note}
If \field{gso_type} differs from VIRTIO_NET_HDR_GSO_NONE and the
VIRTIO_NET_HDR_GSO_UDP_TUNNEL_IPV4 bit or the VIRTIO_NET_HDR_GSO_UDP_TUNNEL_IPV6
bit are not set, \field{csum_offset}
points to the only transport header present in the packet, and there are no
additional preceding checksums validated by VIRTIO_NET_HDR_F_NEEDS_CSUM.
\end{note}
\item If the VIRTIO_NET_F_GUEST_UDP_TUNNEL_GSO option was negotiated and
  \field{gso_type} is not VIRTIO_NET_HDR_GSO_NONE, the
  VIRTIO_NET_HDR_GSO_UDP_TUNNEL_IPV4 bit or the VIRTIO_NET_HDR_GSO_UDP_TUNNEL_IPV6
  bit MAY be set. In such case the \field{outer_th_offset} and
  \field{inner_nh_offset} fields indicate the corresponding
  headers information.
\item If the VIRTIO_NET_F_GUEST_UDP_TUNNEL_GSO_CSUM feature was
negotiated, and
  the VIRTIO_NET_HDR_GSO_UDP_TUNNEL_IPV4 bit or the VIRTIO_NET_HDR_GSO_UDP_TUNNEL_IPV6
  are set in \field{gso_type}, the VIRTIO_NET_HDR_F_UDP_TUNNEL_CSUM bit in the
  \field{flags} can be set: if so, the outer UDP checksum has been validated
  and the UDP header checksum at offset 6 from from \field{outer_th_offset}
  is set to the outer UDP pseudo header checksum.

\begin{note}
If the VIRTIO_NET_HDR_GSO_UDP_TUNNEL_IPV4 bit or VIRTIO_NET_HDR_GSO_UDP_TUNNEL_IPV6
bit are set in \field{gso_type}, the \field{csum_start} field refers to
the inner transport header offset (see Packet Transmission point 1).
If the VIRTIO_NET_HDR_F_UDP_TUNNEL_CSUM bit in \field{flags} is set both
the inner and the outer header checksums have been validated by
VIRTIO_NET_HDR_F_NEEDS_CSUM, otherwise only the inner transport header
checksum has been validated.
\end{note}
\end{enumerate}

If applicable, the device calculates per-packet hash for incoming packets as
defined in \ref{sec:Device Types / Network Device / Device Operation / Processing of Incoming Packets / Hash calculation for incoming packets}.

If applicable, the device reports hash information for incoming packets as
defined in \ref{sec:Device Types / Network Device / Device Operation / Processing of Incoming Packets / Hash reporting for incoming packets}.

\devicenormative{\paragraph}{Processing of Incoming Packets}{Device Types / Network Device / Device Operation / Processing of Incoming Packets}
\label{devicenormative:Device Types / Network Device / Device Operation / Processing of Packets}%old label for latexdiff

If VIRTIO_NET_F_MRG_RXBUF has not been negotiated, the device MUST set
\field{num_buffers} to 1.

If VIRTIO_NET_F_MRG_RXBUF has been negotiated, the device MUST set
\field{num_buffers} to indicate the number of buffers
the packet (including the header) is spread over.

If a receive packet is spread over multiple buffers, the device
MUST use all buffers but the last (i.e. the first \field{num_buffers} -
1 buffers) completely up to the full length of each buffer
supplied by the driver.

The device MUST use all buffers used by a single receive
packet together, such that at least \field{num_buffers} are
observed by driver as used.

If VIRTIO_NET_F_GUEST_CSUM is not negotiated, the device MUST set
\field{flags} to zero and SHOULD supply a fully checksummed
packet to the driver.

If VIRTIO_NET_F_GUEST_TSO4 is not negotiated, the device MUST NOT set
\field{gso_type} to VIRTIO_NET_HDR_GSO_TCPV4.

If VIRTIO_NET_F_GUEST_UDP is not negotiated, the device MUST NOT set
\field{gso_type} to VIRTIO_NET_HDR_GSO_UDP.

If VIRTIO_NET_F_GUEST_TSO6 is not negotiated, the device MUST NOT set
\field{gso_type} to VIRTIO_NET_HDR_GSO_TCPV6.

If none of VIRTIO_NET_F_GUEST_USO4 or VIRTIO_NET_F_GUEST_USO6 have been negotiated,
the device MUST NOT set \field{gso_type} to VIRTIO_NET_HDR_GSO_UDP_L4.

If VIRTIO_NET_F_GUEST_UDP_TUNNEL_GSO is not negotiated, the device MUST NOT set
either the VIRTIO_NET_HDR_GSO_UDP_TUNNEL_IPV4 bit or the
VIRTIO_NET_HDR_GSO_UDP_TUNNEL_IPV6 bit in \field{gso_type}.

If VIRTIO_NET_F_GUEST_UDP_TUNNEL_GSO_CSUM is not negotiated the device MUST NOT set
the VIRTIO_NET_HDR_F_UDP_TUNNEL_CSUM bit in \field{flags}.

The device SHOULD NOT send to the driver TCP packets requiring segmentation offload
which have the Explicit Congestion Notification bit set, unless the
VIRTIO_NET_F_GUEST_ECN feature is negotiated, in which case the
device MUST set the VIRTIO_NET_HDR_GSO_ECN bit in
\field{gso_type}.

If the VIRTIO_NET_F_GUEST_CSUM feature has been negotiated, the
device MAY set the VIRTIO_NET_HDR_F_NEEDS_CSUM bit in
\field{flags}, if so:
\begin{enumerate}
\item the device MUST validate the packet checksum at
	offset \field{csum_offset} from \field{csum_start} as well as all
	preceding offsets;
\item the device MUST set the packet checksum stored in the
	receive buffer to the TCP/UDP pseudo header;
\item the device MUST set \field{csum_start} and
	\field{csum_offset} such that calculating a ones'
	complement checksum from \field{csum_start} up until the
	end of the packet and storing the result at offset
	\field{csum_offset} from  \field{csum_start} will result in a
	fully checksummed packet;
\end{enumerate}

The device MUST NOT send to the driver GSO packets encapsulated in UDP
tunnel and requiring segmentation offload, unless the
VIRTIO_NET_F_GUEST_UDP_TUNNEL_GSO is negotiated, in which case the device MUST set
the VIRTIO_NET_HDR_GSO_UDP_TUNNEL_IPV4 bit or the VIRTIO_NET_HDR_GSO_UDP_TUNNEL_IPV6
bit in \field{gso_type} according to the inner network header protocol type,
MUST set the \field{outer_th_offset} and \field{inner_nh_offset} fields
to the corresponding header information, and the outer UDP header MUST NOT
require checksum offload.

If the VIRTIO_NET_F_GUEST_UDP_TUNNEL_GSO_CSUM feature has not been negotiated,
the device MUST NOT send the driver GSO packets encapsulated in UDP
tunnel and requiring segmentation and outer checksum offload.

If none of the VIRTIO_NET_F_GUEST_TSO4, TSO6, UFO, USO4 or USO6 options have
been negotiated, the device MUST set \field{gso_type} to
VIRTIO_NET_HDR_GSO_NONE.

If \field{gso_type} differs from VIRTIO_NET_HDR_GSO_NONE, then
the device MUST also set the VIRTIO_NET_HDR_F_NEEDS_CSUM bit in
\field{flags} MUST set \field{gso_size} to indicate the desired MSS.
If VIRTIO_NET_F_RSC_EXT was negotiated, the device MUST also
set VIRTIO_NET_HDR_F_RSC_INFO bit in \field{flags},
set \field{csum_start} to number of coalesced TCP segments and
set \field{csum_offset} to number of received duplicated ACK segments.

If VIRTIO_NET_F_RSC_EXT was not negotiated, the device MUST
not set VIRTIO_NET_HDR_F_RSC_INFO bit in \field{flags}.

If one of the VIRTIO_NET_F_GUEST_TSO4, TSO6, UFO, USO4 or USO6 options have
been negotiated, the device SHOULD set \field{hdr_len} to a value
not less than the length of the headers, including the transport
header. If \field{gso_type} has the VIRTIO_NET_HDR_GSO_UDP_TUNNEL_IPV4 bit
or the VIRTIO_NET_HDR_GSO_UDP_TUNNEL_IPV6 bit set, the referenced transport
header is the inner one.

If the VIRTIO_NET_F_GUEST_CSUM feature has been negotiated, the
device MAY set the VIRTIO_NET_HDR_F_DATA_VALID bit in
\field{flags}, if so, the device MUST validate the packet
checksum. If the VIRTIO_NET_F_GUEST_UDP_TUNNEL_GSO_CSUM feature has
been negotiated, and the VIRTIO_NET_HDR_F_UDP_TUNNEL_CSUM bit set in
\field{flags}, both the outer UDP checksum and the inner transport
checksum have been validated.
Otherwise level of checksum is validated: in case of multiple
encapsulated protocols the outermost one.

If either the VIRTIO_NET_HDR_GSO_UDP_TUNNEL_IPV4 bit or the
VIRTIO_NET_HDR_GSO_UDP_TUNNEL_IPV6 bit in \field{gso_type} are set,
the device MUST NOT set the VIRTIO_NET_HDR_F_DATA_VALID bit in
\field{flags}.

If the VIRTIO_NET_F_GUEST_UDP_TUNNEL_GSO_CSUM feature has been negotiated
and either the VIRTIO_NET_HDR_GSO_UDP_TUNNEL_IPV4 bit is set or the
VIRTIO_NET_HDR_GSO_UDP_TUNNEL_IPV6 bit is set in \field{gso_type}, the
device MAY set the VIRTIO_NET_HDR_F_UDP_TUNNEL_CSUM bit in
\field{flags}, if so the device MUST set the packet outer UDP checksum
stored in the receive buffer to the outer UDP pseudo header.

Otherwise, the VIRTIO_NET_F_GUEST_UDP_TUNNEL_GSO_CSUM feature has been
negotiated, either the VIRTIO_NET_HDR_GSO_UDP_TUNNEL_IPV4 bit is set or the
VIRTIO_NET_HDR_GSO_UDP_TUNNEL_IPV6 bit is set in \field{gso_type},
and the bit VIRTIO_NET_HDR_F_UDP_TUNNEL_CSUM is not set in
\field{flags}, the device MUST either provide a zero outer UDP header
checksum or a fully checksummed outer UDP header.

\drivernormative{\paragraph}{Processing of Incoming
Packets}{Device Types / Network Device / Device Operation /
Processing of Incoming Packets}

The driver MUST ignore \field{flag} bits that it does not recognize.

If VIRTIO_NET_HDR_F_NEEDS_CSUM bit in \field{flags} is not set or
if VIRTIO_NET_HDR_F_RSC_INFO bit \field{flags} is set, the
driver MUST NOT use the \field{csum_start} and \field{csum_offset}.

If one of the VIRTIO_NET_F_GUEST_TSO4, TSO6, UFO, USO4 or USO6 options have
been negotiated, the driver MAY use \field{hdr_len} only as a hint about the
transport header size.
The driver MUST NOT rely on \field{hdr_len} to be correct.
\begin{note}
This is due to various bugs in implementations.
\end{note}

If neither VIRTIO_NET_HDR_F_NEEDS_CSUM nor
VIRTIO_NET_HDR_F_DATA_VALID is set, the driver MUST NOT
rely on the packet checksum being correct.

If both the VIRTIO_NET_HDR_GSO_UDP_TUNNEL_IPV4 bit and
the VIRTIO_NET_HDR_GSO_UDP_TUNNEL_IPV6 bit in in \field{gso_type} are set,
the driver MUST NOT accept the packet.

If the VIRTIO_NET_HDR_GSO_UDP_TUNNEL_IPV4 bit or the VIRTIO_NET_HDR_GSO_UDP_TUNNEL_IPV6
bit in \field{gso_type} are not set, the driver MUST NOT use the
\field{outer_th_offset} and \field{inner_nh_offset}.

If either the VIRTIO_NET_HDR_GSO_UDP_TUNNEL_IPV4 bit or
the VIRTIO_NET_HDR_GSO_UDP_TUNNEL_IPV6 bit in \field{gso_type} are set, and any of
the following is true:
\begin{itemize}
\item the VIRTIO_NET_HDR_F_NEEDS_CSUM bit is not set in \field{flags}
\item the VIRTIO_NET_HDR_F_DATA_VALID bit is set in \field{flags}
\item the \field{gso_type} excluding the VIRTIO_NET_HDR_GSO_UDP_TUNNEL_IPV4
bit and the VIRTIO_NET_HDR_GSO_UDP_TUNNEL_IPV6 bit is VIRTIO_NET_HDR_GSO_NONE
\end{itemize}
the driver MUST NOT accept the packet.

If the VIRTIO_NET_HDR_F_UDP_TUNNEL_CSUM bit and the VIRTIO_NET_HDR_F_NEEDS_CSUM
bit in \field{flags} are set,
and both the bits VIRTIO_NET_HDR_GSO_UDP_TUNNEL_IPV4 and
VIRTIO_NET_HDR_GSO_UDP_TUNNEL_IPV6 in \field{gso_type} are not set,
the driver MOST NOT accept the packet.

\paragraph{Hash calculation for incoming packets}
\label{sec:Device Types / Network Device / Device Operation / Processing of Incoming Packets / Hash calculation for incoming packets}

A device attempts to calculate a per-packet hash in the following cases:
\begin{itemize}
\item The feature VIRTIO_NET_F_RSS was negotiated. The device uses the hash to determine the receive virtqueue to place incoming packets.
\item The feature VIRTIO_NET_F_HASH_REPORT was negotiated. The device reports the hash value and the hash type with the packet.
\end{itemize}

If the feature VIRTIO_NET_F_RSS was negotiated:
\begin{itemize}
\item The device uses \field{hash_types} of the virtio_net_rss_config structure as 'Enabled hash types' bitmask.
\item If additionally the feature VIRTIO_NET_F_HASH_TUNNEL was negotiated, the device uses \field{enabled_tunnel_types} of the
      virtnet_hash_tunnel structure as 'Encapsulation types enabled for inner header hash' bitmask.
\item The device uses a key as defined in \field{hash_key_data} and \field{hash_key_length} of the virtio_net_rss_config structure (see
\ref{sec:Device Types / Network Device / Device Operation / Control Virtqueue / Receive-side scaling (RSS) / Setting RSS parameters}).
\end{itemize}

If the feature VIRTIO_NET_F_RSS was not negotiated:
\begin{itemize}
\item The device uses \field{hash_types} of the virtio_net_hash_config structure as 'Enabled hash types' bitmask.
\item If additionally the feature VIRTIO_NET_F_HASH_TUNNEL was negotiated, the device uses \field{enabled_tunnel_types} of the
      virtnet_hash_tunnel structure as 'Encapsulation types enabled for inner header hash' bitmask.
\item The device uses a key as defined in \field{hash_key_data} and \field{hash_key_length} of the virtio_net_hash_config structure (see
\ref{sec:Device Types / Network Device / Device Operation / Control Virtqueue / Automatic receive steering in multiqueue mode / Hash calculation}).
\end{itemize}

Note that if the device offers VIRTIO_NET_F_HASH_REPORT, even if it supports only one pair of virtqueues, it MUST support
at least one of commands of VIRTIO_NET_CTRL_MQ class to configure reported hash parameters:
\begin{itemize}
\item If the device offers VIRTIO_NET_F_RSS, it MUST support VIRTIO_NET_CTRL_MQ_RSS_CONFIG command per
 \ref{sec:Device Types / Network Device / Device Operation / Control Virtqueue / Receive-side scaling (RSS) / Setting RSS parameters}.
\item Otherwise the device MUST support VIRTIO_NET_CTRL_MQ_HASH_CONFIG command per
 \ref{sec:Device Types / Network Device / Device Operation / Control Virtqueue / Automatic receive steering in multiqueue mode / Hash calculation}.
\end{itemize}

The per-packet hash calculation can depend on the IP packet type. See
\hyperref[intro:IP]{[IP]}, \hyperref[intro:UDP]{[UDP]} and \hyperref[intro:TCP]{[TCP]}.

\subparagraph{Supported/enabled hash types}
\label{sec:Device Types / Network Device / Device Operation / Processing of Incoming Packets / Hash calculation for incoming packets / Supported/enabled hash types}
Hash types applicable for IPv4 packets:
\begin{lstlisting}
#define VIRTIO_NET_HASH_TYPE_IPv4              (1 << 0)
#define VIRTIO_NET_HASH_TYPE_TCPv4             (1 << 1)
#define VIRTIO_NET_HASH_TYPE_UDPv4             (1 << 2)
\end{lstlisting}
Hash types applicable for IPv6 packets without extension headers
\begin{lstlisting}
#define VIRTIO_NET_HASH_TYPE_IPv6              (1 << 3)
#define VIRTIO_NET_HASH_TYPE_TCPv6             (1 << 4)
#define VIRTIO_NET_HASH_TYPE_UDPv6             (1 << 5)
\end{lstlisting}
Hash types applicable for IPv6 packets with extension headers
\begin{lstlisting}
#define VIRTIO_NET_HASH_TYPE_IP_EX             (1 << 6)
#define VIRTIO_NET_HASH_TYPE_TCP_EX            (1 << 7)
#define VIRTIO_NET_HASH_TYPE_UDP_EX            (1 << 8)
\end{lstlisting}

\subparagraph{IPv4 packets}
\label{sec:Device Types / Network Device / Device Operation / Processing of Incoming Packets / Hash calculation for incoming packets / IPv4 packets}
The device calculates the hash on IPv4 packets according to 'Enabled hash types' bitmask as follows:
\begin{itemize}
\item If VIRTIO_NET_HASH_TYPE_TCPv4 is set and the packet has
a TCP header, the hash is calculated over the following fields:
\begin{itemize}
\item Source IP address
\item Destination IP address
\item Source TCP port
\item Destination TCP port
\end{itemize}
\item Else if VIRTIO_NET_HASH_TYPE_UDPv4 is set and the
packet has a UDP header, the hash is calculated over the following fields:
\begin{itemize}
\item Source IP address
\item Destination IP address
\item Source UDP port
\item Destination UDP port
\end{itemize}
\item Else if VIRTIO_NET_HASH_TYPE_IPv4 is set, the hash is
calculated over the following fields:
\begin{itemize}
\item Source IP address
\item Destination IP address
\end{itemize}
\item Else the device does not calculate the hash
\end{itemize}

\subparagraph{IPv6 packets without extension header}
\label{sec:Device Types / Network Device / Device Operation / Processing of Incoming Packets / Hash calculation for incoming packets / IPv6 packets without extension header}
The device calculates the hash on IPv6 packets without extension
headers according to 'Enabled hash types' bitmask as follows:
\begin{itemize}
\item If VIRTIO_NET_HASH_TYPE_TCPv6 is set and the packet has
a TCPv6 header, the hash is calculated over the following fields:
\begin{itemize}
\item Source IPv6 address
\item Destination IPv6 address
\item Source TCP port
\item Destination TCP port
\end{itemize}
\item Else if VIRTIO_NET_HASH_TYPE_UDPv6 is set and the
packet has a UDPv6 header, the hash is calculated over the following fields:
\begin{itemize}
\item Source IPv6 address
\item Destination IPv6 address
\item Source UDP port
\item Destination UDP port
\end{itemize}
\item Else if VIRTIO_NET_HASH_TYPE_IPv6 is set, the hash is
calculated over the following fields:
\begin{itemize}
\item Source IPv6 address
\item Destination IPv6 address
\end{itemize}
\item Else the device does not calculate the hash
\end{itemize}

\subparagraph{IPv6 packets with extension header}
\label{sec:Device Types / Network Device / Device Operation / Processing of Incoming Packets / Hash calculation for incoming packets / IPv6 packets with extension header}
The device calculates the hash on IPv6 packets with extension
headers according to 'Enabled hash types' bitmask as follows:
\begin{itemize}
\item If VIRTIO_NET_HASH_TYPE_TCP_EX is set and the packet
has a TCPv6 header, the hash is calculated over the following fields:
\begin{itemize}
\item Home address from the home address option in the IPv6 destination options header. If the extension header is not present, use the Source IPv6 address.
\item IPv6 address that is contained in the Routing-Header-Type-2 from the associated extension header. If the extension header is not present, use the Destination IPv6 address.
\item Source TCP port
\item Destination TCP port
\end{itemize}
\item Else if VIRTIO_NET_HASH_TYPE_UDP_EX is set and the
packet has a UDPv6 header, the hash is calculated over the following fields:
\begin{itemize}
\item Home address from the home address option in the IPv6 destination options header. If the extension header is not present, use the Source IPv6 address.
\item IPv6 address that is contained in the Routing-Header-Type-2 from the associated extension header. If the extension header is not present, use the Destination IPv6 address.
\item Source UDP port
\item Destination UDP port
\end{itemize}
\item Else if VIRTIO_NET_HASH_TYPE_IP_EX is set, the hash is
calculated over the following fields:
\begin{itemize}
\item Home address from the home address option in the IPv6 destination options header. If the extension header is not present, use the Source IPv6 address.
\item IPv6 address that is contained in the Routing-Header-Type-2 from the associated extension header. If the extension header is not present, use the Destination IPv6 address.
\end{itemize}
\item Else skip IPv6 extension headers and calculate the hash as
defined for an IPv6 packet without extension headers
(see \ref{sec:Device Types / Network Device / Device Operation / Processing of Incoming Packets / Hash calculation for incoming packets / IPv6 packets without extension header}).
\end{itemize}

\paragraph{Inner Header Hash}
\label{sec:Device Types / Network Device / Device Operation / Processing of Incoming Packets / Inner Header Hash}

If VIRTIO_NET_F_HASH_TUNNEL has been negotiated, the driver can send the command
VIRTIO_NET_CTRL_HASH_TUNNEL_SET to configure the calculation of the inner header hash.

\begin{lstlisting}
struct virtnet_hash_tunnel {
    le32 enabled_tunnel_types;
};

#define VIRTIO_NET_CTRL_HASH_TUNNEL 7
 #define VIRTIO_NET_CTRL_HASH_TUNNEL_SET 0
\end{lstlisting}

Field \field{enabled_tunnel_types} contains the bitmask of encapsulation types enabled for inner header hash.
See \ref{sec:Device Types / Network Device / Device Operation / Processing of Incoming Packets /
Hash calculation for incoming packets / Encapsulation types supported/enabled for inner header hash}.

The class VIRTIO_NET_CTRL_HASH_TUNNEL has one command:
VIRTIO_NET_CTRL_HASH_TUNNEL_SET sets \field{enabled_tunnel_types} for the device using the
virtnet_hash_tunnel structure, which is read-only for the device.

Inner header hash is disabled by VIRTIO_NET_CTRL_HASH_TUNNEL_SET with \field{enabled_tunnel_types} set to 0.

Initially (before the driver sends any VIRTIO_NET_CTRL_HASH_TUNNEL_SET command) all
encapsulation types are disabled for inner header hash.

\subparagraph{Encapsulated packet}
\label{sec:Device Types / Network Device / Device Operation / Processing of Incoming Packets / Hash calculation for incoming packets / Encapsulated packet}

Multiple tunneling protocols allow encapsulating an inner, payload packet in an outer, encapsulated packet.
The encapsulated packet thus contains an outer header and an inner header, and the device calculates the
hash over either the inner header or the outer header.

If VIRTIO_NET_F_HASH_TUNNEL is negotiated and a received encapsulated packet's outer header matches one of the
encapsulation types enabled in \field{enabled_tunnel_types}, then the device uses the inner header for hash
calculations (only a single level of encapsulation is currently supported).

If VIRTIO_NET_F_HASH_TUNNEL is negotiated and a received packet's (outer) header does not match any encapsulation
types enabled in \field{enabled_tunnel_types}, then the device uses the outer header for hash calculations.

\subparagraph{Encapsulation types supported/enabled for inner header hash}
\label{sec:Device Types / Network Device / Device Operation / Processing of Incoming Packets /
Hash calculation for incoming packets / Encapsulation types supported/enabled for inner header hash}

Encapsulation types applicable for inner header hash:
\begin{lstlisting}[escapechar=|]
#define VIRTIO_NET_HASH_TUNNEL_TYPE_GRE_2784    (1 << 0) /* |\hyperref[intro:rfc2784]{[RFC2784]}| */
#define VIRTIO_NET_HASH_TUNNEL_TYPE_GRE_2890    (1 << 1) /* |\hyperref[intro:rfc2890]{[RFC2890]}| */
#define VIRTIO_NET_HASH_TUNNEL_TYPE_GRE_7676    (1 << 2) /* |\hyperref[intro:rfc7676]{[RFC7676]}| */
#define VIRTIO_NET_HASH_TUNNEL_TYPE_GRE_UDP     (1 << 3) /* |\hyperref[intro:rfc8086]{[GRE-in-UDP]}| */
#define VIRTIO_NET_HASH_TUNNEL_TYPE_VXLAN       (1 << 4) /* |\hyperref[intro:vxlan]{[VXLAN]}| */
#define VIRTIO_NET_HASH_TUNNEL_TYPE_VXLAN_GPE   (1 << 5) /* |\hyperref[intro:vxlan-gpe]{[VXLAN-GPE]}| */
#define VIRTIO_NET_HASH_TUNNEL_TYPE_GENEVE      (1 << 6) /* |\hyperref[intro:geneve]{[GENEVE]}| */
#define VIRTIO_NET_HASH_TUNNEL_TYPE_IPIP        (1 << 7) /* |\hyperref[intro:ipip]{[IPIP]}| */
#define VIRTIO_NET_HASH_TUNNEL_TYPE_NVGRE       (1 << 8) /* |\hyperref[intro:nvgre]{[NVGRE]}| */
\end{lstlisting}

\subparagraph{Advice}
Example uses of the inner header hash:
\begin{itemize}
\item Legacy tunneling protocols, lacking the outer header entropy, can use RSS with the inner header hash to
      distribute flows with identical outer but different inner headers across various queues, improving performance.
\item Identify an inner flow distributed across multiple outer tunnels.
\end{itemize}

As using the inner header hash completely discards the outer header entropy, care must be taken
if the inner header is controlled by an adversary, as the adversary can then intentionally create
configurations with insufficient entropy.

Besides disabling the inner header hash, mitigations would depend on how the hash is used. When the hash
use is limited to the RSS queue selection, the inner header hash may have quality of service (QoS) limitations.

\devicenormative{\subparagraph}{Inner Header Hash}{Device Types / Network Device / Device Operation / Control Virtqueue / Inner Header Hash}

If the (outer) header of the received packet does not match any encapsulation types enabled
in \field{enabled_tunnel_types}, the device MUST calculate the hash on the outer header.

If the device receives any bits in \field{enabled_tunnel_types} which are not set in \field{supported_tunnel_types},
it SHOULD respond to the VIRTIO_NET_CTRL_HASH_TUNNEL_SET command with VIRTIO_NET_ERR.

If the driver sets \field{enabled_tunnel_types} to 0 through VIRTIO_NET_CTRL_HASH_TUNNEL_SET or upon the device reset,
the device MUST disable the inner header hash for all encapsulation types.

\drivernormative{\subparagraph}{Inner Header Hash}{Device Types / Network Device / Device Operation / Control Virtqueue / Inner Header Hash}

The driver MUST have negotiated the VIRTIO_NET_F_HASH_TUNNEL feature when issuing the VIRTIO_NET_CTRL_HASH_TUNNEL_SET command.

The driver MUST NOT set any bits in \field{enabled_tunnel_types} which are not set in \field{supported_tunnel_types}.

The driver MUST ignore bits in \field{supported_tunnel_types} which are not documented in this specification.

\paragraph{Hash reporting for incoming packets}
\label{sec:Device Types / Network Device / Device Operation / Processing of Incoming Packets / Hash reporting for incoming packets}

If VIRTIO_NET_F_HASH_REPORT was negotiated and
 the device has calculated the hash for the packet, the device fills \field{hash_report} with the report type of calculated hash
and \field{hash_value} with the value of calculated hash.

If VIRTIO_NET_F_HASH_REPORT was negotiated but due to any reason the
hash was not calculated, the device sets \field{hash_report} to VIRTIO_NET_HASH_REPORT_NONE.

Possible values that the device can report in \field{hash_report} are defined below.
They correspond to supported hash types defined in
\ref{sec:Device Types / Network Device / Device Operation / Processing of Incoming Packets / Hash calculation for incoming packets / Supported/enabled hash types}
as follows:

VIRTIO_NET_HASH_TYPE_XXX = 1 << (VIRTIO_NET_HASH_REPORT_XXX - 1)

\begin{lstlisting}
#define VIRTIO_NET_HASH_REPORT_NONE            0
#define VIRTIO_NET_HASH_REPORT_IPv4            1
#define VIRTIO_NET_HASH_REPORT_TCPv4           2
#define VIRTIO_NET_HASH_REPORT_UDPv4           3
#define VIRTIO_NET_HASH_REPORT_IPv6            4
#define VIRTIO_NET_HASH_REPORT_TCPv6           5
#define VIRTIO_NET_HASH_REPORT_UDPv6           6
#define VIRTIO_NET_HASH_REPORT_IPv6_EX         7
#define VIRTIO_NET_HASH_REPORT_TCPv6_EX        8
#define VIRTIO_NET_HASH_REPORT_UDPv6_EX        9
\end{lstlisting}

\subsubsection{Control Virtqueue}\label{sec:Device Types / Network Device / Device Operation / Control Virtqueue}

The driver uses the control virtqueue (if VIRTIO_NET_F_CTRL_VQ is
negotiated) to send commands to manipulate various features of
the device which would not easily map into the configuration
space.

All commands are of the following form:

\begin{lstlisting}
struct virtio_net_ctrl {
        u8 class;
        u8 command;
        u8 command-specific-data[];
        u8 ack;
        u8 command-specific-result[];
};

/* ack values */
#define VIRTIO_NET_OK     0
#define VIRTIO_NET_ERR    1
\end{lstlisting}

The \field{class}, \field{command} and command-specific-data are set by the
driver, and the device sets the \field{ack} byte and optionally
\field{command-specific-result}. There is little the driver can
do except issue a diagnostic if \field{ack} is not VIRTIO_NET_OK.

The command VIRTIO_NET_CTRL_STATS_QUERY and VIRTIO_NET_CTRL_STATS_GET contain
\field{command-specific-result}.

\paragraph{Packet Receive Filtering}\label{sec:Device Types / Network Device / Device Operation / Control Virtqueue / Packet Receive Filtering}
\label{sec:Device Types / Network Device / Device Operation / Control Virtqueue / Setting Promiscuous Mode}%old label for latexdiff

If the VIRTIO_NET_F_CTRL_RX and VIRTIO_NET_F_CTRL_RX_EXTRA
features are negotiated, the driver can send control commands for
promiscuous mode, multicast, unicast and broadcast receiving.

\begin{note}
In general, these commands are best-effort: unwanted
packets could still arrive.
\end{note}

\begin{lstlisting}
#define VIRTIO_NET_CTRL_RX    0
 #define VIRTIO_NET_CTRL_RX_PROMISC      0
 #define VIRTIO_NET_CTRL_RX_ALLMULTI     1
 #define VIRTIO_NET_CTRL_RX_ALLUNI       2
 #define VIRTIO_NET_CTRL_RX_NOMULTI      3
 #define VIRTIO_NET_CTRL_RX_NOUNI        4
 #define VIRTIO_NET_CTRL_RX_NOBCAST      5
\end{lstlisting}


\devicenormative{\subparagraph}{Packet Receive Filtering}{Device Types / Network Device / Device Operation / Control Virtqueue / Packet Receive Filtering}

If the VIRTIO_NET_F_CTRL_RX feature has been negotiated,
the device MUST support the following VIRTIO_NET_CTRL_RX class
commands:
\begin{itemize}
\item VIRTIO_NET_CTRL_RX_PROMISC turns promiscuous mode on and
off. The command-specific-data is one byte containing 0 (off) or
1 (on). If promiscuous mode is on, the device SHOULD receive all
incoming packets.
This SHOULD take effect even if one of the other modes set by
a VIRTIO_NET_CTRL_RX class command is on.
\item VIRTIO_NET_CTRL_RX_ALLMULTI turns all-multicast receive on and
off. The command-specific-data is one byte containing 0 (off) or
1 (on). When all-multicast receive is on the device SHOULD allow
all incoming multicast packets.
\end{itemize}

If the VIRTIO_NET_F_CTRL_RX_EXTRA feature has been negotiated,
the device MUST support the following VIRTIO_NET_CTRL_RX class
commands:
\begin{itemize}
\item VIRTIO_NET_CTRL_RX_ALLUNI turns all-unicast receive on and
off. The command-specific-data is one byte containing 0 (off) or
1 (on). When all-unicast receive is on the device SHOULD allow
all incoming unicast packets.
\item VIRTIO_NET_CTRL_RX_NOMULTI suppresses multicast receive.
The command-specific-data is one byte containing 0 (multicast
receive allowed) or 1 (multicast receive suppressed).
When multicast receive is suppressed, the device SHOULD NOT
send multicast packets to the driver.
This SHOULD take effect even if VIRTIO_NET_CTRL_RX_ALLMULTI is on.
This filter SHOULD NOT apply to broadcast packets.
\item VIRTIO_NET_CTRL_RX_NOUNI suppresses unicast receive.
The command-specific-data is one byte containing 0 (unicast
receive allowed) or 1 (unicast receive suppressed).
When unicast receive is suppressed, the device SHOULD NOT
send unicast packets to the driver.
This SHOULD take effect even if VIRTIO_NET_CTRL_RX_ALLUNI is on.
\item VIRTIO_NET_CTRL_RX_NOBCAST suppresses broadcast receive.
The command-specific-data is one byte containing 0 (broadcast
receive allowed) or 1 (broadcast receive suppressed).
When broadcast receive is suppressed, the device SHOULD NOT
send broadcast packets to the driver.
This SHOULD take effect even if VIRTIO_NET_CTRL_RX_ALLMULTI is on.
\end{itemize}

\drivernormative{\subparagraph}{Packet Receive Filtering}{Device Types / Network Device / Device Operation / Control Virtqueue / Packet Receive Filtering}

If the VIRTIO_NET_F_CTRL_RX feature has not been negotiated,
the driver MUST NOT issue commands VIRTIO_NET_CTRL_RX_PROMISC or
VIRTIO_NET_CTRL_RX_ALLMULTI.

If the VIRTIO_NET_F_CTRL_RX_EXTRA feature has not been negotiated,
the driver MUST NOT issue commands
 VIRTIO_NET_CTRL_RX_ALLUNI,
 VIRTIO_NET_CTRL_RX_NOMULTI,
 VIRTIO_NET_CTRL_RX_NOUNI or
 VIRTIO_NET_CTRL_RX_NOBCAST.

\paragraph{Setting MAC Address Filtering}\label{sec:Device Types / Network Device / Device Operation / Control Virtqueue / Setting MAC Address Filtering}

If the VIRTIO_NET_F_CTRL_RX feature is negotiated, the driver can
send control commands for MAC address filtering.

\begin{lstlisting}
struct virtio_net_ctrl_mac {
        le32 entries;
        u8 macs[entries][6];
};

#define VIRTIO_NET_CTRL_MAC    1
 #define VIRTIO_NET_CTRL_MAC_TABLE_SET        0
 #define VIRTIO_NET_CTRL_MAC_ADDR_SET         1
\end{lstlisting}

The device can filter incoming packets by any number of destination
MAC addresses\footnote{Since there are no guarantees, it can use a hash filter or
silently switch to allmulti or promiscuous mode if it is given too
many addresses.
}. This table is set using the class
VIRTIO_NET_CTRL_MAC and the command VIRTIO_NET_CTRL_MAC_TABLE_SET. The
command-specific-data is two variable length tables of 6-byte MAC
addresses (as described in struct virtio_net_ctrl_mac). The first table contains unicast addresses, and the second
contains multicast addresses.

The VIRTIO_NET_CTRL_MAC_ADDR_SET command is used to set the
default MAC address which rx filtering
accepts (and if VIRTIO_NET_F_MAC has been negotiated,
this will be reflected in \field{mac} in config space).

The command-specific-data for VIRTIO_NET_CTRL_MAC_ADDR_SET is
the 6-byte MAC address.

\devicenormative{\subparagraph}{Setting MAC Address Filtering}{Device Types / Network Device / Device Operation / Control Virtqueue / Setting MAC Address Filtering}

The device MUST have an empty MAC filtering table on reset.

The device MUST update the MAC filtering table before it consumes
the VIRTIO_NET_CTRL_MAC_TABLE_SET command.

The device MUST update \field{mac} in config space before it consumes
the VIRTIO_NET_CTRL_MAC_ADDR_SET command, if VIRTIO_NET_F_MAC has
been negotiated.

The device SHOULD drop incoming packets which have a destination MAC which
matches neither the \field{mac} (or that set with VIRTIO_NET_CTRL_MAC_ADDR_SET)
nor the MAC filtering table.

\drivernormative{\subparagraph}{Setting MAC Address Filtering}{Device Types / Network Device / Device Operation / Control Virtqueue / Setting MAC Address Filtering}

If VIRTIO_NET_F_CTRL_RX has not been negotiated,
the driver MUST NOT issue VIRTIO_NET_CTRL_MAC class commands.

If VIRTIO_NET_F_CTRL_RX has been negotiated,
the driver SHOULD issue VIRTIO_NET_CTRL_MAC_ADDR_SET
to set the default mac if it is different from \field{mac}.

The driver MUST follow the VIRTIO_NET_CTRL_MAC_TABLE_SET command
by a le32 number, followed by that number of non-multicast
MAC addresses, followed by another le32 number, followed by
that number of multicast addresses.  Either number MAY be 0.

\subparagraph{Legacy Interface: Setting MAC Address Filtering}\label{sec:Device Types / Network Device / Device Operation / Control Virtqueue / Setting MAC Address Filtering / Legacy Interface: Setting MAC Address Filtering}
When using the legacy interface, transitional devices and drivers
MUST format \field{entries} in struct virtio_net_ctrl_mac
according to the native endian of the guest rather than
(necessarily when not using the legacy interface) little-endian.

Legacy drivers that didn't negotiate VIRTIO_NET_F_CTRL_MAC_ADDR
changed \field{mac} in config space when NIC is accepting
incoming packets. These drivers always wrote the mac value from
first to last byte, therefore after detecting such drivers,
a transitional device MAY defer MAC update, or MAY defer
processing incoming packets until driver writes the last byte
of \field{mac} in the config space.

\paragraph{VLAN Filtering}\label{sec:Device Types / Network Device / Device Operation / Control Virtqueue / VLAN Filtering}

If the driver negotiates the VIRTIO_NET_F_CTRL_VLAN feature, it
can control a VLAN filter table in the device. The VLAN filter
table applies only to VLAN tagged packets.

When VIRTIO_NET_F_CTRL_VLAN is negotiated, the device starts with
an empty VLAN filter table.

\begin{note}
Similar to the MAC address based filtering, the VLAN filtering
is also best-effort: unwanted packets could still arrive.
\end{note}

\begin{lstlisting}
#define VIRTIO_NET_CTRL_VLAN       2
 #define VIRTIO_NET_CTRL_VLAN_ADD             0
 #define VIRTIO_NET_CTRL_VLAN_DEL             1
\end{lstlisting}

Both the VIRTIO_NET_CTRL_VLAN_ADD and VIRTIO_NET_CTRL_VLAN_DEL
command take a little-endian 16-bit VLAN id as the command-specific-data.

VIRTIO_NET_CTRL_VLAN_ADD command adds the specified VLAN to the
VLAN filter table.

VIRTIO_NET_CTRL_VLAN_DEL command removes the specified VLAN from
the VLAN filter table.

\devicenormative{\subparagraph}{VLAN Filtering}{Device Types / Network Device / Device Operation / Control Virtqueue / VLAN Filtering}

When VIRTIO_NET_F_CTRL_VLAN is not negotiated, the device MUST
accept all VLAN tagged packets.

When VIRTIO_NET_F_CTRL_VLAN is negotiated, the device MUST
accept all VLAN tagged packets whose VLAN tag is present in
the VLAN filter table and SHOULD drop all VLAN tagged packets
whose VLAN tag is absent in the VLAN filter table.

\subparagraph{Legacy Interface: VLAN Filtering}\label{sec:Device Types / Network Device / Device Operation / Control Virtqueue / VLAN Filtering / Legacy Interface: VLAN Filtering}
When using the legacy interface, transitional devices and drivers
MUST format the VLAN id
according to the native endian of the guest rather than
(necessarily when not using the legacy interface) little-endian.

\paragraph{Gratuitous Packet Sending}\label{sec:Device Types / Network Device / Device Operation / Control Virtqueue / Gratuitous Packet Sending}

If the driver negotiates the VIRTIO_NET_F_GUEST_ANNOUNCE (depends
on VIRTIO_NET_F_CTRL_VQ), the device can ask the driver to send gratuitous
packets; this is usually done after the guest has been physically
migrated, and needs to announce its presence on the new network
links. (As hypervisor does not have the knowledge of guest
network configuration (eg. tagged vlan) it is simplest to prod
the guest in this way).

\begin{lstlisting}
#define VIRTIO_NET_CTRL_ANNOUNCE       3
 #define VIRTIO_NET_CTRL_ANNOUNCE_ACK             0
\end{lstlisting}

The driver checks VIRTIO_NET_S_ANNOUNCE bit in the device configuration \field{status} field
when it notices the changes of device configuration. The
command VIRTIO_NET_CTRL_ANNOUNCE_ACK is used to indicate that
driver has received the notification and device clears the
VIRTIO_NET_S_ANNOUNCE bit in \field{status}.

Processing this notification involves:

\begin{enumerate}
\item Sending the gratuitous packets (eg. ARP) or marking there are pending
  gratuitous packets to be sent and letting deferred routine to
  send them.

\item Sending VIRTIO_NET_CTRL_ANNOUNCE_ACK command through control
  vq.
\end{enumerate}

\drivernormative{\subparagraph}{Gratuitous Packet Sending}{Device Types / Network Device / Device Operation / Control Virtqueue / Gratuitous Packet Sending}

If the driver negotiates VIRTIO_NET_F_GUEST_ANNOUNCE, it SHOULD notify
network peers of its new location after it sees the VIRTIO_NET_S_ANNOUNCE bit
in \field{status}.  The driver MUST send a command on the command queue
with class VIRTIO_NET_CTRL_ANNOUNCE and command VIRTIO_NET_CTRL_ANNOUNCE_ACK.

\devicenormative{\subparagraph}{Gratuitous Packet Sending}{Device Types / Network Device / Device Operation / Control Virtqueue / Gratuitous Packet Sending}

If VIRTIO_NET_F_GUEST_ANNOUNCE is negotiated, the device MUST clear the
VIRTIO_NET_S_ANNOUNCE bit in \field{status} upon receipt of a command buffer
with class VIRTIO_NET_CTRL_ANNOUNCE and command VIRTIO_NET_CTRL_ANNOUNCE_ACK
before marking the buffer as used.

\paragraph{Device operation in multiqueue mode}\label{sec:Device Types / Network Device / Device Operation / Control Virtqueue / Device operation in multiqueue mode}

This specification defines the following modes that a device MAY implement for operation with multiple transmit/receive virtqueues:
\begin{itemize}
\item Automatic receive steering as defined in \ref{sec:Device Types / Network Device / Device Operation / Control Virtqueue / Automatic receive steering in multiqueue mode}.
 If a device supports this mode, it offers the VIRTIO_NET_F_MQ feature bit.
\item Receive-side scaling as defined in \ref{devicenormative:Device Types / Network Device / Device Operation / Control Virtqueue / Receive-side scaling (RSS) / RSS processing}.
 If a device supports this mode, it offers the VIRTIO_NET_F_RSS feature bit.
\end{itemize}

A device MAY support one of these features or both. The driver MAY negotiate any set of these features that the device supports.

Multiqueue is disabled by default.

The driver enables multiqueue by sending a command using \field{class} VIRTIO_NET_CTRL_MQ. The \field{command} selects the mode of multiqueue operation, as follows:
\begin{lstlisting}
#define VIRTIO_NET_CTRL_MQ    4
 #define VIRTIO_NET_CTRL_MQ_VQ_PAIRS_SET        0 (for automatic receive steering)
 #define VIRTIO_NET_CTRL_MQ_RSS_CONFIG          1 (for configurable receive steering)
 #define VIRTIO_NET_CTRL_MQ_HASH_CONFIG         2 (for configurable hash calculation)
\end{lstlisting}

If more than one multiqueue mode is negotiated, the resulting device configuration is defined by the last command sent by the driver.

\paragraph{Automatic receive steering in multiqueue mode}\label{sec:Device Types / Network Device / Device Operation / Control Virtqueue / Automatic receive steering in multiqueue mode}

If the driver negotiates the VIRTIO_NET_F_MQ feature bit (depends on VIRTIO_NET_F_CTRL_VQ), it MAY transmit outgoing packets on one
of the multiple transmitq1\ldots transmitqN and ask the device to
queue incoming packets into one of the multiple receiveq1\ldots receiveqN
depending on the packet flow.

The driver enables multiqueue by
sending the VIRTIO_NET_CTRL_MQ_VQ_PAIRS_SET command, specifying
the number of the transmit and receive queues to be used up to
\field{max_virtqueue_pairs}; subsequently,
transmitq1\ldots transmitqn and receiveq1\ldots receiveqn where
n=\field{virtqueue_pairs} MAY be used.
\begin{lstlisting}
struct virtio_net_ctrl_mq_pairs_set {
       le16 virtqueue_pairs;
};
#define VIRTIO_NET_CTRL_MQ_VQ_PAIRS_MIN        1
#define VIRTIO_NET_CTRL_MQ_VQ_PAIRS_MAX        0x8000

\end{lstlisting}

When multiqueue is enabled by VIRTIO_NET_CTRL_MQ_VQ_PAIRS_SET command, the device MUST use automatic receive steering
based on packet flow. Programming of the receive steering
classificator is implicit. After the driver transmitted a packet of a
flow on transmitqX, the device SHOULD cause incoming packets for that flow to
be steered to receiveqX. For uni-directional protocols, or where
no packets have been transmitted yet, the device MAY steer a packet
to a random queue out of the specified receiveq1\ldots receiveqn.

Multiqueue is disabled by VIRTIO_NET_CTRL_MQ_VQ_PAIRS_SET with \field{virtqueue_pairs} to 1 (this is
the default) and waiting for the device to use the command buffer.

\drivernormative{\subparagraph}{Automatic receive steering in multiqueue mode}{Device Types / Network Device / Device Operation / Control Virtqueue / Automatic receive steering in multiqueue mode}

The driver MUST configure the virtqueues before enabling them with the
VIRTIO_NET_CTRL_MQ_VQ_PAIRS_SET command.

The driver MUST NOT request a \field{virtqueue_pairs} of 0 or
greater than \field{max_virtqueue_pairs} in the device configuration space.

The driver MUST queue packets only on any transmitq1 before the
VIRTIO_NET_CTRL_MQ_VQ_PAIRS_SET command.

The driver MUST NOT queue packets on transmit queues greater than
\field{virtqueue_pairs} once it has placed the VIRTIO_NET_CTRL_MQ_VQ_PAIRS_SET command in the available ring.

\devicenormative{\subparagraph}{Automatic receive steering in multiqueue mode}{Device Types / Network Device / Device Operation / Control Virtqueue / Automatic receive steering in multiqueue mode}

After initialization of reset, the device MUST queue packets only on receiveq1.

The device MUST NOT queue packets on receive queues greater than
\field{virtqueue_pairs} once it has placed the
VIRTIO_NET_CTRL_MQ_VQ_PAIRS_SET command in a used buffer.

If the destination receive queue is being reset (See \ref{sec:Basic Facilities of a Virtio Device / Virtqueues / Virtqueue Reset}),
the device SHOULD re-select another random queue. If all receive queues are
being reset, the device MUST drop the packet.

\subparagraph{Legacy Interface: Automatic receive steering in multiqueue mode}\label{sec:Device Types / Network Device / Device Operation / Control Virtqueue / Automatic receive steering in multiqueue mode / Legacy Interface: Automatic receive steering in multiqueue mode}
When using the legacy interface, transitional devices and drivers
MUST format \field{virtqueue_pairs}
according to the native endian of the guest rather than
(necessarily when not using the legacy interface) little-endian.

\subparagraph{Hash calculation}\label{sec:Device Types / Network Device / Device Operation / Control Virtqueue / Automatic receive steering in multiqueue mode / Hash calculation}
If VIRTIO_NET_F_HASH_REPORT was negotiated and the device uses automatic receive steering,
the device MUST support a command to configure hash calculation parameters.

The driver provides parameters for hash calculation as follows:

\field{class} VIRTIO_NET_CTRL_MQ, \field{command} VIRTIO_NET_CTRL_MQ_HASH_CONFIG.

The \field{command-specific-data} has following format:
\begin{lstlisting}
struct virtio_net_hash_config {
    le32 hash_types;
    le16 reserved[4];
    u8 hash_key_length;
    u8 hash_key_data[hash_key_length];
};
\end{lstlisting}
Field \field{hash_types} contains a bitmask of allowed hash types as
defined in
\ref{sec:Device Types / Network Device / Device Operation / Processing of Incoming Packets / Hash calculation for incoming packets / Supported/enabled hash types}.
Initially the device has all hash types disabled and reports only VIRTIO_NET_HASH_REPORT_NONE.

Field \field{reserved} MUST contain zeroes. It is defined to make the structure to match the layout of virtio_net_rss_config structure,
defined in \ref{sec:Device Types / Network Device / Device Operation / Control Virtqueue / Receive-side scaling (RSS)}.

Fields \field{hash_key_length} and \field{hash_key_data} define the key to be used in hash calculation.

\paragraph{Receive-side scaling (RSS)}\label{sec:Device Types / Network Device / Device Operation / Control Virtqueue / Receive-side scaling (RSS)}
A device offers the feature VIRTIO_NET_F_RSS if it supports RSS receive steering with Toeplitz hash calculation and configurable parameters.

A driver queries RSS capabilities of the device by reading device configuration as defined in \ref{sec:Device Types / Network Device / Device configuration layout}

\subparagraph{Setting RSS parameters}\label{sec:Device Types / Network Device / Device Operation / Control Virtqueue / Receive-side scaling (RSS) / Setting RSS parameters}

Driver sends a VIRTIO_NET_CTRL_MQ_RSS_CONFIG command using the following format for \field{command-specific-data}:
\begin{lstlisting}
struct rss_rq_id {
   le16 vq_index_1_16: 15; /* Bits 1 to 16 of the virtqueue index */
   le16 reserved: 1; /* Set to zero */
};

struct virtio_net_rss_config {
    le32 hash_types;
    le16 indirection_table_mask;
    struct rss_rq_id unclassified_queue;
    struct rss_rq_id indirection_table[indirection_table_length];
    le16 max_tx_vq;
    u8 hash_key_length;
    u8 hash_key_data[hash_key_length];
};
\end{lstlisting}
Field \field{hash_types} contains a bitmask of allowed hash types as
defined in
\ref{sec:Device Types / Network Device / Device Operation / Processing of Incoming Packets / Hash calculation for incoming packets / Supported/enabled hash types}.

Field \field{indirection_table_mask} is a mask to be applied to
the calculated hash to produce an index in the
\field{indirection_table} array.
Number of entries in \field{indirection_table} is (\field{indirection_table_mask} + 1).

\field{rss_rq_id} is a receive virtqueue id. \field{vq_index_1_16}
consists of bits 1 to 16 of a virtqueue index. For example, a
\field{vq_index_1_16} value of 3 corresponds to virtqueue index 6,
which maps to receiveq4.

Field \field{unclassified_queue} specifies the receive virtqueue id in which to
place unclassified packets.

Field \field{indirection_table} is an array of receive virtqueues ids.

A driver sets \field{max_tx_vq} to inform a device how many transmit virtqueues it may use (transmitq1\ldots transmitq \field{max_tx_vq}).

Fields \field{hash_key_length} and \field{hash_key_data} define the key to be used in hash calculation.

\drivernormative{\subparagraph}{Setting RSS parameters}{Device Types / Network Device / Device Operation / Control Virtqueue / Receive-side scaling (RSS) }

A driver MUST NOT send the VIRTIO_NET_CTRL_MQ_RSS_CONFIG command if the feature VIRTIO_NET_F_RSS has not been negotiated.

A driver MUST fill the \field{indirection_table} array only with
enabled receive virtqueues ids.

The number of entries in \field{indirection_table} (\field{indirection_table_mask} + 1) MUST be a power of two.

A driver MUST use \field{indirection_table_mask} values that are less than \field{rss_max_indirection_table_length} reported by a device.

A driver MUST NOT set any VIRTIO_NET_HASH_TYPE_ flags that are not supported by a device.

\devicenormative{\subparagraph}{RSS processing}{Device Types / Network Device / Device Operation / Control Virtqueue / Receive-side scaling (RSS) / RSS processing}
The device MUST determine the destination queue for a network packet as follows:
\begin{itemize}
\item Calculate the hash of the packet as defined in \ref{sec:Device Types / Network Device / Device Operation / Processing of Incoming Packets / Hash calculation for incoming packets}.
\item If the device did not calculate the hash for the specific packet, the device directs the packet to the receiveq specified by \field{unclassified_queue} of virtio_net_rss_config structure.
\item Apply \field{indirection_table_mask} to the calculated hash
and use the result as the index in the indirection table to get
the destination receive virtqueue id.
\item If the destination receive queue is being reset (See \ref{sec:Basic Facilities of a Virtio Device / Virtqueues / Virtqueue Reset}), the device MUST drop the packet.
\end{itemize}

\paragraph{RSS Context}\label{sec:Device Types / Network Device / Device Operation / Control Virtqueue / RSS Context}

An RSS context consists of configurable parameters specified by \ref{sec:Device Types / Network Device
/ Device Operation / Control Virtqueue / Receive-side scaling (RSS)}.

The RSS configuration supported by VIRTIO_NET_F_RSS is considered the default RSS configuration.

The device offers the feature VIRTIO_NET_F_RSS_CONTEXT if it supports one or multiple RSS contexts
(excluding the default RSS configuration) and configurable parameters.

\subparagraph{Querying RSS Context Capability}\label{sec:Device Types / Network Device / Device Operation / Control Virtqueue / RSS Context / Querying RSS Context Capability}

\begin{lstlisting}
#define VIRTNET_RSS_CTX_CTRL 9
 #define VIRTNET_RSS_CTX_CTRL_CAP_GET  0
 #define VIRTNET_RSS_CTX_CTRL_ADD      1
 #define VIRTNET_RSS_CTX_CTRL_MOD      2
 #define VIRTNET_RSS_CTX_CTRL_DEL      3

struct virtnet_rss_ctx_cap {
    le16 max_rss_contexts;
}
\end{lstlisting}

Field \field{max_rss_contexts} specifies the maximum number of RSS contexts \ref{sec:Device Types / Network Device /
Device Operation / Control Virtqueue / RSS Context} supported by the device.

The driver queries the RSS context capability of the device by sending the command VIRTNET_RSS_CTX_CTRL_CAP_GET
with the structure virtnet_rss_ctx_cap.

For the command VIRTNET_RSS_CTX_CTRL_CAP_GET, the structure virtnet_rss_ctx_cap is write-only for the device.

\subparagraph{Setting RSS Context Parameters}\label{sec:Device Types / Network Device / Device Operation / Control Virtqueue / RSS Context / Setting RSS Context Parameters}

\begin{lstlisting}
struct virtnet_rss_ctx_add_modify {
    le16 rss_ctx_id;
    u8 reserved[6];
    struct virtio_net_rss_config rss;
};

struct virtnet_rss_ctx_del {
    le16 rss_ctx_id;
};
\end{lstlisting}

RSS context parameters:
\begin{itemize}
\item  \field{rss_ctx_id}: ID of the specific RSS context.
\item  \field{rss}: RSS context parameters of the specific RSS context whose id is \field{rss_ctx_id}.
\end{itemize}

\field{reserved} is reserved and it is ignored by the device.

If the feature VIRTIO_NET_F_RSS_CONTEXT has been negotiated, the driver can send the following
VIRTNET_RSS_CTX_CTRL class commands:
\begin{enumerate}
\item VIRTNET_RSS_CTX_CTRL_ADD: use the structure virtnet_rss_ctx_add_modify to
       add an RSS context configured as \field{rss} and id as \field{rss_ctx_id} for the device.
\item VIRTNET_RSS_CTX_CTRL_MOD: use the structure virtnet_rss_ctx_add_modify to
       configure parameters of the RSS context whose id is \field{rss_ctx_id} as \field{rss} for the device.
\item VIRTNET_RSS_CTX_CTRL_DEL: use the structure virtnet_rss_ctx_del to delete
       the RSS context whose id is \field{rss_ctx_id} for the device.
\end{enumerate}

For commands VIRTNET_RSS_CTX_CTRL_ADD and VIRTNET_RSS_CTX_CTRL_MOD, the structure virtnet_rss_ctx_add_modify is read-only for the device.
For the command VIRTNET_RSS_CTX_CTRL_DEL, the structure virtnet_rss_ctx_del is read-only for the device.

\devicenormative{\subparagraph}{RSS Context}{Device Types / Network Device / Device Operation / Control Virtqueue / RSS Context}

The device MUST set \field{max_rss_contexts} to at least 1 if it offers VIRTIO_NET_F_RSS_CONTEXT.

Upon reset, the device MUST clear all previously configured RSS contexts.

\drivernormative{\subparagraph}{RSS Context}{Device Types / Network Device / Device Operation / Control Virtqueue / RSS Context}

The driver MUST have negotiated the VIRTIO_NET_F_RSS_CONTEXT feature when issuing the VIRTNET_RSS_CTX_CTRL class commands.

The driver MUST set \field{rss_ctx_id} to between 1 and \field{max_rss_contexts} inclusive.

The driver MUST NOT send the command VIRTIO_NET_CTRL_MQ_VQ_PAIRS_SET when the device has successfully configured at least one RSS context.

\paragraph{Offloads State Configuration}\label{sec:Device Types / Network Device / Device Operation / Control Virtqueue / Offloads State Configuration}

If the VIRTIO_NET_F_CTRL_GUEST_OFFLOADS feature is negotiated, the driver can
send control commands for dynamic offloads state configuration.

\subparagraph{Setting Offloads State}\label{sec:Device Types / Network Device / Device Operation / Control Virtqueue / Offloads State Configuration / Setting Offloads State}

To configure the offloads, the following layout structure and
definitions are used:

\begin{lstlisting}
le64 offloads;

#define VIRTIO_NET_F_GUEST_CSUM       1
#define VIRTIO_NET_F_GUEST_TSO4       7
#define VIRTIO_NET_F_GUEST_TSO6       8
#define VIRTIO_NET_F_GUEST_ECN        9
#define VIRTIO_NET_F_GUEST_UFO        10
#define VIRTIO_NET_F_GUEST_UDP_TUNNEL_GSO  46
#define VIRTIO_NET_F_GUEST_UDP_TUNNEL_GSO_CSUM 47
#define VIRTIO_NET_F_GUEST_USO4       54
#define VIRTIO_NET_F_GUEST_USO6       55

#define VIRTIO_NET_CTRL_GUEST_OFFLOADS       5
 #define VIRTIO_NET_CTRL_GUEST_OFFLOADS_SET   0
\end{lstlisting}

The class VIRTIO_NET_CTRL_GUEST_OFFLOADS has one command:
VIRTIO_NET_CTRL_GUEST_OFFLOADS_SET applies the new offloads configuration.

le64 value passed as command data is a bitmask, bits set define
offloads to be enabled, bits cleared - offloads to be disabled.

There is a corresponding device feature for each offload. Upon feature
negotiation corresponding offload gets enabled to preserve backward
compatibility.

\drivernormative{\subparagraph}{Setting Offloads State}{Device Types / Network Device / Device Operation / Control Virtqueue / Offloads State Configuration / Setting Offloads State}

A driver MUST NOT enable an offload for which the appropriate feature
has not been negotiated.

\subparagraph{Legacy Interface: Setting Offloads State}\label{sec:Device Types / Network Device / Device Operation / Control Virtqueue / Offloads State Configuration / Setting Offloads State / Legacy Interface: Setting Offloads State}
When using the legacy interface, transitional devices and drivers
MUST format \field{offloads}
according to the native endian of the guest rather than
(necessarily when not using the legacy interface) little-endian.


\paragraph{Notifications Coalescing}\label{sec:Device Types / Network Device / Device Operation / Control Virtqueue / Notifications Coalescing}

If the VIRTIO_NET_F_NOTF_COAL feature is negotiated, the driver can
send commands VIRTIO_NET_CTRL_NOTF_COAL_TX_SET and VIRTIO_NET_CTRL_NOTF_COAL_RX_SET
for notification coalescing.

If the VIRTIO_NET_F_VQ_NOTF_COAL feature is negotiated, the driver can
send commands VIRTIO_NET_CTRL_NOTF_COAL_VQ_SET and VIRTIO_NET_CTRL_NOTF_COAL_VQ_GET
for virtqueue notification coalescing.

\begin{lstlisting}
struct virtio_net_ctrl_coal {
    le32 max_packets;
    le32 max_usecs;
};

struct virtio_net_ctrl_coal_vq {
    le16 vq_index;
    le16 reserved;
    struct virtio_net_ctrl_coal coal;
};

#define VIRTIO_NET_CTRL_NOTF_COAL 6
 #define VIRTIO_NET_CTRL_NOTF_COAL_TX_SET  0
 #define VIRTIO_NET_CTRL_NOTF_COAL_RX_SET 1
 #define VIRTIO_NET_CTRL_NOTF_COAL_VQ_SET 2
 #define VIRTIO_NET_CTRL_NOTF_COAL_VQ_GET 3
\end{lstlisting}

Coalescing parameters:
\begin{itemize}
\item \field{vq_index}: The virtqueue index of an enabled transmit or receive virtqueue.
\item \field{max_usecs} for RX: Maximum number of microseconds to delay a RX notification.
\item \field{max_usecs} for TX: Maximum number of microseconds to delay a TX notification.
\item \field{max_packets} for RX: Maximum number of packets to receive before a RX notification.
\item \field{max_packets} for TX: Maximum number of packets to send before a TX notification.
\end{itemize}

\field{reserved} is reserved and it is ignored by the device.

Read/Write attributes for coalescing parameters:
\begin{itemize}
\item For commands VIRTIO_NET_CTRL_NOTF_COAL_TX_SET and VIRTIO_NET_CTRL_NOTF_COAL_RX_SET, the structure virtio_net_ctrl_coal is write-only for the driver.
\item For the command VIRTIO_NET_CTRL_NOTF_COAL_VQ_SET, the structure virtio_net_ctrl_coal_vq is write-only for the driver.
\item For the command VIRTIO_NET_CTRL_NOTF_COAL_VQ_GET, \field{vq_index} and \field{reserved} are write-only
      for the driver, and the structure virtio_net_ctrl_coal is read-only for the driver.
\end{itemize}

The class VIRTIO_NET_CTRL_NOTF_COAL has the following commands:
\begin{enumerate}
\item VIRTIO_NET_CTRL_NOTF_COAL_TX_SET: use the structure virtio_net_ctrl_coal to set the \field{max_usecs} and \field{max_packets} parameters for all transmit virtqueues.
\item VIRTIO_NET_CTRL_NOTF_COAL_RX_SET: use the structure virtio_net_ctrl_coal to set the \field{max_usecs} and \field{max_packets} parameters for all receive virtqueues.
\item VIRTIO_NET_CTRL_NOTF_COAL_VQ_SET: use the structure virtio_net_ctrl_coal_vq to set the \field{max_usecs} and \field{max_packets} parameters
                                        for an enabled transmit/receive virtqueue whose index is \field{vq_index}.
\item VIRTIO_NET_CTRL_NOTF_COAL_VQ_GET: use the structure virtio_net_ctrl_coal_vq to get the \field{max_usecs} and \field{max_packets} parameters
                                        for an enabled transmit/receive virtqueue whose index is \field{vq_index}.
\end{enumerate}

The device may generate notifications more or less frequently than specified by set commands of the VIRTIO_NET_CTRL_NOTF_COAL class.

If coalescing parameters are being set, the device applies the last coalescing parameters set for a
virtqueue, regardless of the command used to set the parameters. Use the following command sequence
with two pairs of virtqueues as an example:
Each of the following commands sets \field{max_usecs} and \field{max_packets} parameters for virtqueues.
\begin{itemize}
\item Command1: VIRTIO_NET_CTRL_NOTF_COAL_RX_SET sets coalescing parameters for virtqueues having index 0 and index 2. Virtqueues having index 1 and index 3 retain their previous parameters.
\item Command2: VIRTIO_NET_CTRL_NOTF_COAL_VQ_SET with \field{vq_index} = 0 sets coalescing parameters for virtqueue having index 0. Virtqueue having index 2 retains the parameters from command1.
\item Command3: VIRTIO_NET_CTRL_NOTF_COAL_VQ_GET with \field{vq_index} = 0, the device responds with coalescing parameters of vq_index 0 set by command2.
\item Command4: VIRTIO_NET_CTRL_NOTF_COAL_VQ_SET with \field{vq_index} = 1 sets coalescing parameters for virtqueue having index 1. Virtqueue having index 3 retains its previous parameters.
\item Command5: VIRTIO_NET_CTRL_NOTF_COAL_TX_SET sets coalescing parameters for virtqueues having index 1 and index 3, and overrides the parameters set by command4.
\item Command6: VIRTIO_NET_CTRL_NOTF_COAL_VQ_GET with \field{vq_index} = 1, the device responds with coalescing parameters of index 1 set by command5.
\end{itemize}

\subparagraph{Operation}\label{sec:Device Types / Network Device / Device Operation / Control Virtqueue / Notifications Coalescing / Operation}

The device sends a used buffer notification once the notification conditions are met and if the notifications are not suppressed as explained in \ref{sec:Basic Facilities of a Virtio Device / Virtqueues / Used Buffer Notification Suppression}.

When the device has non-zero \field{max_usecs} and non-zero \field{max_packets}, it starts counting microseconds and packets upon receiving/sending a packet.
The device counts packets and microseconds for each receive virtqueue and transmit virtqueue separately.
In this case, the notification conditions are met when \field{max_usecs} microseconds elapse, or upon sending/receiving \field{max_packets} packets, whichever happens first.
Afterwards, the device waits for the next packet and starts counting packets and microseconds again.

When the device has \field{max_usecs} = 0 or \field{max_packets} = 0, the notification conditions are met after every packet received/sent.

\subparagraph{RX Example}\label{sec:Device Types / Network Device / Device Operation / Control Virtqueue / Notifications Coalescing / RX Example}

If, for example:
\begin{itemize}
\item \field{max_usecs} = 10.
\item \field{max_packets} = 15.
\end{itemize}
then each receive virtqueue of a device will operate as follows:
\begin{itemize}
\item The device will count packets received on each virtqueue until it accumulates 15, or until 10 microseconds elapsed since the first one was received.
\item If the notifications are not suppressed by the driver, the device will send an used buffer notification, otherwise, the device will not send an used buffer notification as long as the notifications are suppressed.
\end{itemize}

\subparagraph{TX Example}\label{sec:Device Types / Network Device / Device Operation / Control Virtqueue / Notifications Coalescing / TX Example}

If, for example:
\begin{itemize}
\item \field{max_usecs} = 10.
\item \field{max_packets} = 15.
\end{itemize}
then each transmit virtqueue of a device will operate as follows:
\begin{itemize}
\item The device will count packets sent on each virtqueue until it accumulates 15, or until 10 microseconds elapsed since the first one was sent.
\item If the notifications are not suppressed by the driver, the device will send an used buffer notification, otherwise, the device will not send an used buffer notification as long as the notifications are suppressed.
\end{itemize}

\subparagraph{Notifications When Coalescing Parameters Change}\label{sec:Device Types / Network Device / Device Operation / Control Virtqueue / Notifications Coalescing / Notifications When Coalescing Parameters Change}

When the coalescing parameters of a device change, the device needs to check if the new notification conditions are met and send a used buffer notification if so.

For example, \field{max_packets} = 15 for a device with a single transmit virtqueue: if the device sends 10 packets and afterwards receives a
VIRTIO_NET_CTRL_NOTF_COAL_TX_SET command with \field{max_packets} = 8, then the notification condition is immediately considered to be met;
the device needs to immediately send a used buffer notification, if the notifications are not suppressed by the driver.

\drivernormative{\subparagraph}{Notifications Coalescing}{Device Types / Network Device / Device Operation / Control Virtqueue / Notifications Coalescing}

The driver MUST set \field{vq_index} to the virtqueue index of an enabled transmit or receive virtqueue.

The driver MUST have negotiated the VIRTIO_NET_F_NOTF_COAL feature when issuing commands VIRTIO_NET_CTRL_NOTF_COAL_TX_SET and VIRTIO_NET_CTRL_NOTF_COAL_RX_SET.

The driver MUST have negotiated the VIRTIO_NET_F_VQ_NOTF_COAL feature when issuing commands VIRTIO_NET_CTRL_NOTF_COAL_VQ_SET and VIRTIO_NET_CTRL_NOTF_COAL_VQ_GET.

The driver MUST ignore the values of coalescing parameters received from the VIRTIO_NET_CTRL_NOTF_COAL_VQ_GET command if the device responds with VIRTIO_NET_ERR.

\devicenormative{\subparagraph}{Notifications Coalescing}{Device Types / Network Device / Device Operation / Control Virtqueue / Notifications Coalescing}

The device MUST ignore \field{reserved}.

The device SHOULD respond to VIRTIO_NET_CTRL_NOTF_COAL_TX_SET and VIRTIO_NET_CTRL_NOTF_COAL_RX_SET commands with VIRTIO_NET_ERR if it was not able to change the parameters.

The device MUST respond to the VIRTIO_NET_CTRL_NOTF_COAL_VQ_SET command with VIRTIO_NET_ERR if it was not able to change the parameters.

The device MUST respond to VIRTIO_NET_CTRL_NOTF_COAL_VQ_SET and VIRTIO_NET_CTRL_NOTF_COAL_VQ_GET commands with
VIRTIO_NET_ERR if the designated virtqueue is not an enabled transmit or receive virtqueue.

Upon disabling and re-enabling a transmit virtqueue, the device MUST set the coalescing parameters of the virtqueue
to those configured through the VIRTIO_NET_CTRL_NOTF_COAL_TX_SET command, or, if the driver did not set any TX coalescing parameters, to 0.

Upon disabling and re-enabling a receive virtqueue, the device MUST set the coalescing parameters of the virtqueue
to those configured through the VIRTIO_NET_CTRL_NOTF_COAL_RX_SET command, or, if the driver did not set any RX coalescing parameters, to 0.

The behavior of the device in response to set commands of the VIRTIO_NET_CTRL_NOTF_COAL class is best-effort:
the device MAY generate notifications more or less frequently than specified.

A device SHOULD NOT send used buffer notifications to the driver if the notifications are suppressed, even if the notification conditions are met.

Upon reset, a device MUST initialize all coalescing parameters to 0.

\paragraph{Device Statistics}\label{sec:Device Types / Network Device / Device Operation / Control Virtqueue / Device Statistics}

If the VIRTIO_NET_F_DEVICE_STATS feature is negotiated, the driver can obtain
device statistics from the device by using the following command.

Different types of virtqueues have different statistics. The statistics of the
receiveq are different from those of the transmitq.

The statistics of a certain type of virtqueue are also divided into multiple types
because different types require different features. This enables the expansion
of new statistics.

In one command, the driver can obtain the statistics of one or multiple virtqueues.
Additionally, the driver can obtain multiple type statistics of each virtqueue.

\subparagraph{Query Statistic Capabilities}\label{sec:Device Types / Network Device / Device Operation / Control Virtqueue / Device Statistics / Query Statistic Capabilities}

\begin{lstlisting}
#define VIRTIO_NET_CTRL_STATS         8
#define VIRTIO_NET_CTRL_STATS_QUERY   0
#define VIRTIO_NET_CTRL_STATS_GET     1

struct virtio_net_stats_capabilities {

#define VIRTIO_NET_STATS_TYPE_CVQ       (1 << 32)

#define VIRTIO_NET_STATS_TYPE_RX_BASIC  (1 << 0)
#define VIRTIO_NET_STATS_TYPE_RX_CSUM   (1 << 1)
#define VIRTIO_NET_STATS_TYPE_RX_GSO    (1 << 2)
#define VIRTIO_NET_STATS_TYPE_RX_SPEED  (1 << 3)

#define VIRTIO_NET_STATS_TYPE_TX_BASIC  (1 << 16)
#define VIRTIO_NET_STATS_TYPE_TX_CSUM   (1 << 17)
#define VIRTIO_NET_STATS_TYPE_TX_GSO    (1 << 18)
#define VIRTIO_NET_STATS_TYPE_TX_SPEED  (1 << 19)

    le64 supported_stats_types[1];
}
\end{lstlisting}

To obtain device statistic capability, use the VIRTIO_NET_CTRL_STATS_QUERY
command. When the command completes successfully, \field{command-specific-result}
is in the format of \field{struct virtio_net_stats_capabilities}.

\subparagraph{Get Statistics}\label{sec:Device Types / Network Device / Device Operation / Control Virtqueue / Device Statistics / Get Statistics}

\begin{lstlisting}
struct virtio_net_ctrl_queue_stats {
       struct {
           le16 vq_index;
           le16 reserved[3];
           le64 types_bitmap[1];
       } stats[];
};

struct virtio_net_stats_reply_hdr {
#define VIRTIO_NET_STATS_TYPE_REPLY_CVQ       32

#define VIRTIO_NET_STATS_TYPE_REPLY_RX_BASIC  0
#define VIRTIO_NET_STATS_TYPE_REPLY_RX_CSUM   1
#define VIRTIO_NET_STATS_TYPE_REPLY_RX_GSO    2
#define VIRTIO_NET_STATS_TYPE_REPLY_RX_SPEED  3

#define VIRTIO_NET_STATS_TYPE_REPLY_TX_BASIC  16
#define VIRTIO_NET_STATS_TYPE_REPLY_TX_CSUM   17
#define VIRTIO_NET_STATS_TYPE_REPLY_TX_GSO    18
#define VIRTIO_NET_STATS_TYPE_REPLY_TX_SPEED  19
    u8 type;
    u8 reserved;
    le16 vq_index;
    le16 reserved1;
    le16 size;
}
\end{lstlisting}

To obtain device statistics, use the VIRTIO_NET_CTRL_STATS_GET command with the
\field{command-specific-data} which is in the format of
\field{struct virtio_net_ctrl_queue_stats}. When the command completes
successfully, \field{command-specific-result} contains multiple statistic
results, each statistic result has the \field{struct virtio_net_stats_reply_hdr}
as the header.

The fields of the \field{struct virtio_net_ctrl_queue_stats}:
\begin{description}
    \item [vq_index]
        The index of the virtqueue to obtain the statistics.

    \item [types_bitmap]
        This is a bitmask of the types of statistics to be obtained. Therefore, a
        \field{stats} inside \field{struct virtio_net_ctrl_queue_stats} may
        indicate multiple statistic replies for the virtqueue.
\end{description}

The fields of the \field{struct virtio_net_stats_reply_hdr}:
\begin{description}
    \item [type]
        The type of the reply statistic.

    \item [vq_index]
        The virtqueue index of the reply statistic.

    \item [size]
        The number of bytes for the statistics entry including size of \field{struct virtio_net_stats_reply_hdr}.

\end{description}

\subparagraph{Controlq Statistics}\label{sec:Device Types / Network Device / Device Operation / Control Virtqueue / Device Statistics / Controlq Statistics}

The structure corresponding to the controlq statistics is
\field{struct virtio_net_stats_cvq}. The corresponding type is
VIRTIO_NET_STATS_TYPE_CVQ. This is for the controlq.

\begin{lstlisting}
struct virtio_net_stats_cvq {
    struct virtio_net_stats_reply_hdr hdr;

    le64 command_num;
    le64 ok_num;
};
\end{lstlisting}

\begin{description}
    \item [command_num]
        The number of commands received by the device including the current command.

    \item [ok_num]
        The number of commands completed successfully by the device including the current command.
\end{description}


\subparagraph{Receiveq Basic Statistics}\label{sec:Device Types / Network Device / Device Operation / Control Virtqueue / Device Statistics / Receiveq Basic Statistics}

The structure corresponding to the receiveq basic statistics is
\field{struct virtio_net_stats_rx_basic}. The corresponding type is
VIRTIO_NET_STATS_TYPE_RX_BASIC. This is for the receiveq.

Receiveq basic statistics do not require any feature. As long as the device supports
VIRTIO_NET_F_DEVICE_STATS, the following are the receiveq basic statistics.

\begin{lstlisting}
struct virtio_net_stats_rx_basic {
    struct virtio_net_stats_reply_hdr hdr;

    le64 rx_notifications;

    le64 rx_packets;
    le64 rx_bytes;

    le64 rx_interrupts;

    le64 rx_drops;
    le64 rx_drop_overruns;
};
\end{lstlisting}

The packets described below were all presented on the specified virtqueue.
\begin{description}
    \item [rx_notifications]
        The number of driver notifications received by the device for this
        receiveq.

    \item [rx_packets]
        This is the number of packets passed to the driver by the device.

    \item [rx_bytes]
        This is the bytes of packets passed to the driver by the device.

    \item [rx_interrupts]
        The number of interrupts generated by the device for this receiveq.

    \item [rx_drops]
        This is the number of packets dropped by the device. The count includes
        all types of packets dropped by the device.

    \item [rx_drop_overruns]
        This is the number of packets dropped by the device when no more
        descriptors were available.

\end{description}

\subparagraph{Transmitq Basic Statistics}\label{sec:Device Types / Network Device / Device Operation / Control Virtqueue / Device Statistics / Transmitq Basic Statistics}

The structure corresponding to the transmitq basic statistics is
\field{struct virtio_net_stats_tx_basic}. The corresponding type is
VIRTIO_NET_STATS_TYPE_TX_BASIC. This is for the transmitq.

Transmitq basic statistics do not require any feature. As long as the device supports
VIRTIO_NET_F_DEVICE_STATS, the following are the transmitq basic statistics.

\begin{lstlisting}
struct virtio_net_stats_tx_basic {
    struct virtio_net_stats_reply_hdr hdr;

    le64 tx_notifications;

    le64 tx_packets;
    le64 tx_bytes;

    le64 tx_interrupts;

    le64 tx_drops;
    le64 tx_drop_malformed;
};
\end{lstlisting}

The packets described below are all for a specific virtqueue.
\begin{description}
    \item [tx_notifications]
        The number of driver notifications received by the device for this
        transmitq.

    \item [tx_packets]
        This is the number of packets sent by the device (not the packets
        got from the driver).

    \item [tx_bytes]
        This is the number of bytes sent by the device for all the sent packets
        (not the bytes sent got from the driver).

    \item [tx_interrupts]
        The number of interrupts generated by the device for this transmitq.

    \item [tx_drops]
        The number of packets dropped by the device. The count includes all
        types of packets dropped by the device.

    \item [tx_drop_malformed]
        The number of packets dropped by the device, when the descriptors are
        malformed. For example, the buffer is too short.
\end{description}

\subparagraph{Receiveq CSUM Statistics}\label{sec:Device Types / Network Device / Device Operation / Control Virtqueue / Device Statistics / Receiveq CSUM Statistics}

The structure corresponding to the receiveq checksum statistics is
\field{struct virtio_net_stats_rx_csum}. The corresponding type is
VIRTIO_NET_STATS_TYPE_RX_CSUM. This is for the receiveq.

Only after the VIRTIO_NET_F_GUEST_CSUM is negotiated, the receiveq checksum
statistics can be obtained.

\begin{lstlisting}
struct virtio_net_stats_rx_csum {
    struct virtio_net_stats_reply_hdr hdr;

    le64 rx_csum_valid;
    le64 rx_needs_csum;
    le64 rx_csum_none;
    le64 rx_csum_bad;
};
\end{lstlisting}

The packets described below were all presented on the specified virtqueue.
\begin{description}
    \item [rx_csum_valid]
        The number of packets with VIRTIO_NET_HDR_F_DATA_VALID.

    \item [rx_needs_csum]
        The number of packets with VIRTIO_NET_HDR_F_NEEDS_CSUM.

    \item [rx_csum_none]
        The number of packets without hardware checksum. The packet here refers
        to the non-TCP/UDP packet that the device cannot recognize.

    \item [rx_csum_bad]
        The number of packets with checksum mismatch.

\end{description}

\subparagraph{Transmitq CSUM Statistics}\label{sec:Device Types / Network Device / Device Operation / Control Virtqueue / Device Statistics / Transmitq CSUM Statistics}

The structure corresponding to the transmitq checksum statistics is
\field{struct virtio_net_stats_tx_csum}. The corresponding type is
VIRTIO_NET_STATS_TYPE_TX_CSUM. This is for the transmitq.

Only after the VIRTIO_NET_F_CSUM is negotiated, the transmitq checksum
statistics can be obtained.

The following are the transmitq checksum statistics:

\begin{lstlisting}
struct virtio_net_stats_tx_csum {
    struct virtio_net_stats_reply_hdr hdr;

    le64 tx_csum_none;
    le64 tx_needs_csum;
};
\end{lstlisting}

The packets described below are all for a specific virtqueue.
\begin{description}
    \item [tx_csum_none]
        The number of packets which do not require hardware checksum.

    \item [tx_needs_csum]
        The number of packets which require checksum calculation by the device.

\end{description}

\subparagraph{Receiveq GSO Statistics}\label{sec:Device Types / Network Device / Device Operation / Control Virtqueue / Device Statistics / Receiveq GSO Statistics}

The structure corresponding to the receivq GSO statistics is
\field{struct virtio_net_stats_rx_gso}. The corresponding type is
VIRTIO_NET_STATS_TYPE_RX_GSO. This is for the receiveq.

If one or more of the VIRTIO_NET_F_GUEST_TSO4, VIRTIO_NET_F_GUEST_TSO6
have been negotiated, the receiveq GSO statistics can be obtained.

GSO packets refer to packets passed by the device to the driver where
\field{gso_type} is not VIRTIO_NET_HDR_GSO_NONE.

\begin{lstlisting}
struct virtio_net_stats_rx_gso {
    struct virtio_net_stats_reply_hdr hdr;

    le64 rx_gso_packets;
    le64 rx_gso_bytes;
    le64 rx_gso_packets_coalesced;
    le64 rx_gso_bytes_coalesced;
};
\end{lstlisting}

The packets described below were all presented on the specified virtqueue.
\begin{description}
    \item [rx_gso_packets]
        The number of the GSO packets received by the device.

    \item [rx_gso_bytes]
        The bytes of the GSO packets received by the device.
        This includes the header size of the GSO packet.

    \item [rx_gso_packets_coalesced]
        The number of the GSO packets coalesced by the device.

    \item [rx_gso_bytes_coalesced]
        The bytes of the GSO packets coalesced by the device.
        This includes the header size of the GSO packet.
\end{description}

\subparagraph{Transmitq GSO Statistics}\label{sec:Device Types / Network Device / Device Operation / Control Virtqueue / Device Statistics / Transmitq GSO Statistics}

The structure corresponding to the transmitq GSO statistics is
\field{struct virtio_net_stats_tx_gso}. The corresponding type is
VIRTIO_NET_STATS_TYPE_TX_GSO. This is for the transmitq.

If one or more of the VIRTIO_NET_F_HOST_TSO4, VIRTIO_NET_F_HOST_TSO6,
VIRTIO_NET_F_HOST_USO options have been negotiated, the transmitq GSO statistics
can be obtained.

GSO packets refer to packets passed by the driver to the device where
\field{gso_type} is not VIRTIO_NET_HDR_GSO_NONE.
See more \ref{sec:Device Types / Network Device / Device Operation / Packet
Transmission}.

\begin{lstlisting}
struct virtio_net_stats_tx_gso {
    struct virtio_net_stats_reply_hdr hdr;

    le64 tx_gso_packets;
    le64 tx_gso_bytes;
    le64 tx_gso_segments;
    le64 tx_gso_segments_bytes;
    le64 tx_gso_packets_noseg;
    le64 tx_gso_bytes_noseg;
};
\end{lstlisting}

The packets described below are all for a specific virtqueue.
\begin{description}
    \item [tx_gso_packets]
        The number of the GSO packets sent by the device.

    \item [tx_gso_bytes]
        The bytes of the GSO packets sent by the device.

    \item [tx_gso_segments]
        The number of segments prepared from GSO packets.

    \item [tx_gso_segments_bytes]
        The bytes of segments prepared from GSO packets.

    \item [tx_gso_packets_noseg]
        The number of the GSO packets without segmentation.

    \item [tx_gso_bytes_noseg]
        The bytes of the GSO packets without segmentation.

\end{description}

\subparagraph{Receiveq Speed Statistics}\label{sec:Device Types / Network Device / Device Operation / Control Virtqueue / Device Statistics / Receiveq Speed Statistics}

The structure corresponding to the receiveq speed statistics is
\field{struct virtio_net_stats_rx_speed}. The corresponding type is
VIRTIO_NET_STATS_TYPE_RX_SPEED. This is for the receiveq.

The device has the allowance for the speed. If VIRTIO_NET_F_SPEED_DUPLEX has
been negotiated, the driver can get this by \field{speed}. When the received
packets bitrate exceeds the \field{speed}, some packets may be dropped by the
device.

\begin{lstlisting}
struct virtio_net_stats_rx_speed {
    struct virtio_net_stats_reply_hdr hdr;

    le64 rx_packets_allowance_exceeded;
    le64 rx_bytes_allowance_exceeded;
};
\end{lstlisting}

The packets described below were all presented on the specified virtqueue.
\begin{description}
    \item [rx_packets_allowance_exceeded]
        The number of the packets dropped by the device due to the received
        packets bitrate exceeding the \field{speed}.

    \item [rx_bytes_allowance_exceeded]
        The bytes of the packets dropped by the device due to the received
        packets bitrate exceeding the \field{speed}.

\end{description}

\subparagraph{Transmitq Speed Statistics}\label{sec:Device Types / Network Device / Device Operation / Control Virtqueue / Device Statistics / Transmitq Speed Statistics}

The structure corresponding to the transmitq speed statistics is
\field{struct virtio_net_stats_tx_speed}. The corresponding type is
VIRTIO_NET_STATS_TYPE_TX_SPEED. This is for the transmitq.

The device has the allowance for the speed. If VIRTIO_NET_F_SPEED_DUPLEX has
been negotiated, the driver can get this by \field{speed}. When the transmit
packets bitrate exceeds the \field{speed}, some packets may be dropped by the
device.

\begin{lstlisting}
struct virtio_net_stats_tx_speed {
    struct virtio_net_stats_reply_hdr hdr;

    le64 tx_packets_allowance_exceeded;
    le64 tx_bytes_allowance_exceeded;
};
\end{lstlisting}

The packets described below were all presented on the specified virtqueue.
\begin{description}
    \item [tx_packets_allowance_exceeded]
        The number of the packets dropped by the device due to the transmit packets
        bitrate exceeding the \field{speed}.

    \item [tx_bytes_allowance_exceeded]
        The bytes of the packets dropped by the device due to the transmit packets
        bitrate exceeding the \field{speed}.

\end{description}

\devicenormative{\subparagraph}{Device Statistics}{Device Types / Network Device / Device Operation / Control Virtqueue / Device Statistics}

When the VIRTIO_NET_F_DEVICE_STATS feature is negotiated, the device MUST reply
to the command VIRTIO_NET_CTRL_STATS_QUERY with the
\field{struct virtio_net_stats_capabilities}. \field{supported_stats_types}
includes all the statistic types supported by the device.

If \field{struct virtio_net_ctrl_queue_stats} is incorrect (such as the
following), the device MUST set \field{ack} to VIRTIO_NET_ERR. Even if there is
only one error, the device MUST fail the entire command.
\begin{itemize}
    \item \field{vq_index} exceeds the queue range.
    \item \field{types_bitmap} contains unknown types.
    \item One or more of the bits present in \field{types_bitmap} is not valid
        for the specified virtqueue.
    \item The feature corresponding to the specified \field{types_bitmap} was
        not negotiated.
\end{itemize}

The device MUST set the actual size of the bytes occupied by the reply to the
\field{size} of the \field{hdr}. And the device MUST set the \field{type} and
the \field{vq_index} of the statistic header.

The \field{command-specific-result} buffer allocated by the driver may be
smaller or bigger than all the statistics specified by
\field{struct virtio_net_ctrl_queue_stats}. The device MUST fill up only upto
the valid bytes.

The statistics counter replied by the device MUST wrap around to zero by the
device on the overflow.

\drivernormative{\subparagraph}{Device Statistics}{Device Types / Network Device / Device Operation / Control Virtqueue / Device Statistics}

The types contained in the \field{types_bitmap} MUST be queried from the device
via command VIRTIO_NET_CTRL_STATS_QUERY.

\field{types_bitmap} in \field{struct virtio_net_ctrl_queue_stats} MUST be valid to the
vq specified by \field{vq_index}.

The \field{command-specific-result} buffer allocated by the driver MUST have
enough capacity to store all the statistics reply headers defined in
\field{struct virtio_net_ctrl_queue_stats}. If the
\field{command-specific-result} buffer is fully utilized by the device but some
replies are missed, it is possible that some statistics may exceed the capacity
of the driver's records. In such cases, the driver should allocate additional
space for the \field{command-specific-result} buffer.

\subsubsection{Flow filter}\label{sec:Device Types / Network Device / Device Operation / Flow filter}

A network device can support one or more flow filter rules. Each flow filter rule
is applied by matching a packet and then taking an action, such as directing the packet
to a specific receiveq or dropping the packet. An example of a match is
matching on specific source and destination IP addresses.

A flow filter rule is a device resource object that consists of a key,
a processing priority, and an action to either direct a packet to a
receive queue or drop the packet.

Each rule uses a classifier. The key is matched against the packet using
a classifier, defining which fields in the packet are matched.
A classifier resource object consists of one or more field selectors, each with
a type that specifies the header fields to be matched against, and a mask.
The mask can match whole fields or parts of a field in a header. Each
rule resource object depends on the classifier resource object.

When a packet is received, relevant fields are extracted
(in the same way) from both the packet and the key according to the
classifier. The resulting field contents are then compared -
if they are identical the rule action is taken, if they are not, the rule is ignored.

Multiple flow filter rules are part of a group. The rule resource object
depends on the group. Each rule within a
group has a rule priority, and each group also has a group priority. For a
packet, a group with the highest priority is selected first. Within a group,
rules are applied from highest to lowest priority, until one of the rules
matches the packet and an action is taken. If all the rules within a group
are ignored, the group with the next highest priority is selected, and so on.

The device and the driver indicates flow filter resource limits using the capability
\ref{par:Device Types / Network Device / Device Operation / Flow filter / Device and driver capabilities / VIRTIO-NET-FF-RESOURCE-CAP} specifying the limits on the number of flow filter rule,
group and classifier resource objects. The capability \ref{par:Device Types / Network Device / Device Operation / Flow filter / Device and driver capabilities / VIRTIO-NET-FF-SELECTOR-CAP} specifies which selectors the device supports.
The driver indicates the selectors it is using by setting the flow
filter selector capability, prior to adding any resource objects.

The capability \ref{par:Device Types / Network Device / Device Operation / Flow filter / Device and driver capabilities / VIRTIO-NET-FF-ACTION-CAP} specifies which actions the device supports.

The driver controls the flow filter rule, classifier and group resource objects using
administration commands described in
\ref{sec:Basic Facilities of a Virtio Device / Device groups / Group administration commands / Device resource objects}.

\paragraph{Packet processing order}\label{sec:sec:Device Types / Network Device / Device Operation / Flow filter / Packet processing order}

Note that flow filter rules are applied after MAC/VLAN filtering. Flow filter
rules take precedence over steering: if a flow filter rule results in an action,
the steering configuration does not apply. The steering configuration only applies
to packets for which no flow filter rule action was performed. For example,
incoming packets can be processed in the following order:

\begin{itemize}
\item apply steering configuration received using control virtqueue commands
      VIRTIO_NET_CTRL_RX, VIRTIO_NET_CTRL_MAC and VIRTIO_NET_CTRL_VLAN.
\item apply flow filter rules if any.
\item if no filter rule applied, apply steering configuration received using command
      VIRTIO_NET_CTRL_MQ_RSS_CONFIG or as per automatic receive steering.
\end{itemize}

Some incoming packet processing examples:
\begin{itemize}
\item If the packet is dropped by the flow filter rule, RSS
      steering is ignored for the packet.
\item If the packet is directed to a specific receiveq using flow filter rule,
      the RSS steering is ignored for the packet.
\item If a packet is dropped due to the VIRTIO_NET_CTRL_MAC configuration,
      both flow filter rules and the RSS steering are ignored for the packet.
\item If a packet does not match any flow filter rules,
      the RSS steering is used to select the receiveq for the packet (if enabled).
\item If there are two flow filter groups configured as group_A and group_B
      with respective group priorities as 4, and 5; flow filter rules of
      group_B are applied first having highest group priority, if there is a match,
      the flow filter rules of group_A are ignored; if there is no match for
      the flow filter rules in group_B, the flow filter rules of next level group_A are applied.
\end{itemize}

\paragraph{Device and driver capabilities}
\label{par:Device Types / Network Device / Device Operation / Flow filter / Device and driver capabilities}

\subparagraph{VIRTIO_NET_FF_RESOURCE_CAP}
\label{par:Device Types / Network Device / Device Operation / Flow filter / Device and driver capabilities / VIRTIO-NET-FF-RESOURCE-CAP}

The capability VIRTIO_NET_FF_RESOURCE_CAP indicates the flow filter resource limits.
\field{cap_specific_data} is in the format
\field{struct virtio_net_ff_cap_data}.

\begin{lstlisting}
struct virtio_net_ff_cap_data {
        le32 groups_limit;
        le32 selectors_limit;
        le32 rules_limit;
        le32 rules_per_group_limit;
        u8 last_rule_priority;
        u8 selectors_per_classifier_limit;
};
\end{lstlisting}

\field{groups_limit}, and \field{selectors_limit} represent the maximum
number of flow filter groups and selectors, respectively, that the driver can create.
 \field{rules_limit} is the maximum number of
flow fiilter rules that the driver can create across all the groups.
\field{rules_per_group_limit} is the maximum number of flow filter rules that the driver
can create for each flow filter group.

\field{last_rule_priority} is the highest priority that can be assigned to a
flow filter rule.

\field{selectors_per_classifier_limit} is the maximum number of selectors
that a classifier can have.

\subparagraph{VIRTIO_NET_FF_SELECTOR_CAP}
\label{par:Device Types / Network Device / Device Operation / Flow filter / Device and driver capabilities / VIRTIO-NET-FF-SELECTOR-CAP}

The capability VIRTIO_NET_FF_SELECTOR_CAP lists the supported selectors and the
supported packet header fields for each selector.
\field{cap_specific_data} is in the format \field{struct virtio_net_ff_cap_mask_data}.

\begin{lstlisting}[label={lst:Device Types / Network Device / Device Operation / Flow filter / Device and driver capabilities / VIRTIO-NET-FF-SELECTOR-CAP / virtio-net-ff-selector}]
struct virtio_net_ff_selector {
        u8 type;
        u8 flags;
        u8 reserved[2];
        u8 length;
        u8 reserved1[3];
        u8 mask[];
};

struct virtio_net_ff_cap_mask_data {
        u8 count;
        u8 reserved[7];
        struct virtio_net_ff_selector selectors[];
};

#define VIRTIO_NET_FF_MASK_F_PARTIAL_MASK (1 << 0)
\end{lstlisting}

\field{count} indicates number of valid entries in the \field{selectors} array.
\field{selectors[]} is an array of supported selectors. Within each array entry:
\field{type} specifies the type of the packet header, as defined in table
\ref{table:Device Types / Network Device / Device Operation / Flow filter / Device and driver capabilities / VIRTIO-NET-FF-SELECTOR-CAP / flow filter selector types}. \field{mask} specifies which fields of the
packet header can be matched in a flow filter rule.

Each \field{type} is also listed in table
\ref{table:Device Types / Network Device / Device Operation / Flow filter / Device and driver capabilities / VIRTIO-NET-FF-SELECTOR-CAP / flow filter selector types}. \field{mask} is a byte array
in network byte order. For example, when \field{type} is VIRTIO_NET_FF_MASK_TYPE_IPV6,
the \field{mask} is in the format \hyperref[intro:IPv6-Header-Format]{IPv6 Header Format}.

If partial masking is not set, then all bits in each field have to be either all 0s
to ignore this field or all 1s to match on this field. If partial masking is set,
then any combination of bits can bit set to match on these bits.
For example, when a selector \field{type} is VIRTIO_NET_FF_MASK_TYPE_ETH, if
\field{mask[0-12]} are zero and \field{mask[13-14]} are 0xff (all 1s), it
indicates that matching is only supported for \field{EtherType} of
\field{Ethernet MAC frame}, matching is not supported for
\field{Destination Address} and \field{Source Address}.

The entries in the array \field{selectors} are ordered by
\field{type}, with each \field{type} value only appearing once.

\field{length} is the length of a dynamic array \field{mask} in bytes.
\field{reserved} and \field{reserved1} are reserved and set to zero.

\begin{table}[H]
\caption{Flow filter selector types}
\label{table:Device Types / Network Device / Device Operation / Flow filter / Device and driver capabilities / VIRTIO-NET-FF-SELECTOR-CAP / flow filter selector types}
\begin{tabularx}{\textwidth}{ |l|X|X| }
\hline
Type & Name & Description \\
\hline \hline
0x0 & - & Reserved \\
\hline
0x1 & VIRTIO_NET_FF_MASK_TYPE_ETH & 14 bytes of frame header starting from destination address described in \hyperref[intro:IEEE 802.3-2022]{IEEE 802.3-2022} \\
\hline
0x2 & VIRTIO_NET_FF_MASK_TYPE_IPV4 & 20 bytes of \hyperref[intro:Internet-Header-Format]{IPv4: Internet Header Format} \\
\hline
0x3 & VIRTIO_NET_FF_MASK_TYPE_IPV6 & 40 bytes of \hyperref[intro:IPv6-Header-Format]{IPv6 Header Format} \\
\hline
0x4 & VIRTIO_NET_FF_MASK_TYPE_TCP & 20 bytes of \hyperref[intro:TCP-Header-Format]{TCP Header Format} \\
\hline
0x5 & VIRTIO_NET_FF_MASK_TYPE_UDP & 8 bytes of UDP header described in \hyperref[intro:UDP]{UDP} \\
\hline
0x6 - 0xFF & & Reserved for future \\
\hline
\end{tabularx}
\end{table}

When VIRTIO_NET_FF_MASK_F_PARTIAL_MASK (bit 0) is set, it indicates that
partial masking is supported for all the fields of the selector identified by \field{type}.

For the selector \field{type} VIRTIO_NET_FF_MASK_TYPE_IPV4, if a partial mask is unsupported,
then matching on an individual bit of \field{Flags} in the
\field{IPv4: Internet Header Format} is unsupported. \field{Flags} has to match as a whole
if it is supported.

For the selector \field{type} VIRTIO_NET_FF_MASK_TYPE_IPV4, \field{mask} includes fields
up to the \field{Destination Address}; that is, \field{Options} and
\field{Padding} are excluded.

For the selector \field{type} VIRTIO_NET_FF_MASK_TYPE_IPV6, the \field{Next Header} field
of the \field{mask} corresponds to the \field{Next Header} in the packet
when \field{IPv6 Extension Headers} are not present. When the packet includes
one or more \field{IPv6 Extension Headers}, the \field{Next Header} field of
the \field{mask} corresponds to the \field{Next Header} of the last
\field{IPv6 Extension Header} in the packet.

For the selector \field{type} VIRTIO_NET_FF_MASK_TYPE_TCP, \field{Control bits}
are treated as individual fields for matching; that is, matching individual
\field{Control bits} does not depend on the partial mask support.

\subparagraph{VIRTIO_NET_FF_ACTION_CAP}
\label{par:Device Types / Network Device / Device Operation / Flow filter / Device and driver capabilities / VIRTIO-NET-FF-ACTION-CAP}

The capability VIRTIO_NET_FF_ACTION_CAP lists the supported actions in a rule.
\field{cap_specific_data} is in the format \field{struct virtio_net_ff_cap_actions}.

\begin{lstlisting}
struct virtio_net_ff_actions {
        u8 count;
        u8 reserved[7];
        u8 actions[];
};
\end{lstlisting}

\field{actions} is an array listing all possible actions.
The entries in the array are ordered from the smallest to the largest,
with each supported value appearing exactly once. Each entry can have the
following values:

\begin{table}[H]
\caption{Flow filter rule actions}
\label{table:Device Types / Network Device / Device Operation / Flow filter / Device and driver capabilities / VIRTIO-NET-FF-ACTION-CAP / flow filter rule actions}
\begin{tabularx}{\textwidth}{ |l|X|X| }
\hline
Action & Name & Description \\
\hline \hline
0x0 & - & reserved \\
\hline
0x1 & VIRTIO_NET_FF_ACTION_DROP & Matching packet will be dropped by the device \\
\hline
0x2 & VIRTIO_NET_FF_ACTION_DIRECT_RX_VQ & Matching packet will be directed to a receive queue \\
\hline
0x3 - 0xFF & & Reserved for future \\
\hline
\end{tabularx}
\end{table}

\paragraph{Resource objects}
\label{par:Device Types / Network Device / Device Operation / Flow filter / Resource objects}

\subparagraph{VIRTIO_NET_RESOURCE_OBJ_FF_GROUP}\label{par:Device Types / Network Device / Device Operation / Flow filter / Resource objects / VIRTIO-NET-RESOURCE-OBJ-FF-GROUP}

A flow filter group contains between 0 and \field{rules_limit} rules, as specified by the
capability VIRTIO_NET_FF_RESOURCE_CAP. For the flow filter group object both
\field{resource_obj_specific_data} and
\field{resource_obj_specific_result} are in the format
\field{struct virtio_net_resource_obj_ff_group}.

\begin{lstlisting}
struct virtio_net_resource_obj_ff_group {
        le16 group_priority;
};
\end{lstlisting}

\field{group_priority} specifies the priority for the group. Each group has a
distinct priority. For each incoming packet, the device tries to apply rules
from groups from higher \field{group_priority} value to lower, until either a
rule matches the packet or all groups have been tried.

\subparagraph{VIRTIO_NET_RESOURCE_OBJ_FF_CLASSIFIER}\label{par:Device Types / Network Device / Device Operation / Flow filter / Resource objects / VIRTIO-NET-RESOURCE-OBJ-FF-CLASSIFIER}

A classifier is used to match a flow filter key against a packet. The
classifier defines the desired packet fields to match, and is represented by
the VIRTIO_NET_RESOURCE_OBJ_FF_CLASSIFIER device resource object.

For the flow filter classifier object both \field{resource_obj_specific_data} and
\field{resource_obj_specific_result} are in the format
\field{struct virtio_net_resource_obj_ff_classifier}.

\begin{lstlisting}
struct virtio_net_resource_obj_ff_classifier {
        u8 count;
        u8 reserved[7];
        struct virtio_net_ff_selector selectors[];
};
\end{lstlisting}

A classifier is an array of \field{selectors}. The number of selectors in the
array is indicated by \field{count}. The selector has a type that specifies
the header fields to be matched against, and a mask.
See \ref{lst:Device Types / Network Device / Device Operation / Flow filter / Device and driver capabilities / VIRTIO-NET-FF-SELECTOR-CAP / virtio-net-ff-selector}
for details about selectors.

The first selector is always VIRTIO_NET_FF_MASK_TYPE_ETH. When there are multiple
selectors, a second selector can be either VIRTIO_NET_FF_MASK_TYPE_IPV4
or VIRTIO_NET_FF_MASK_TYPE_IPV6. If the third selector exists, the third
selector can be either VIRTIO_NET_FF_MASK_TYPE_UDP or VIRTIO_NET_FF_MASK_TYPE_TCP.
For example, to match a Ethernet IPv6 UDP packet,
\field{selectors[0].type} is set to VIRTIO_NET_FF_MASK_TYPE_ETH, \field{selectors[1].type}
is set to VIRTIO_NET_FF_MASK_TYPE_IPV6 and \field{selectors[2].type} is
set to VIRTIO_NET_FF_MASK_TYPE_UDP; accordingly, \field{selectors[0].mask[0-13]} is
for Ethernet header fields, \field{selectors[1].mask[0-39]} is set for IPV6 header
and \field{selectors[2].mask[0-7]} is set for UDP header.

When there are multiple selectors, the type of the (N+1)\textsuperscript{th} selector
affects the mask of the (N)\textsuperscript{th} selector. If
\field{count} is 2 or more, all the mask bits within \field{selectors[0]}
corresponding to \field{EtherType} of an Ethernet header are set.

If \field{count} is more than 2:
\begin{itemize}
\item if \field{selector[1].type} is, VIRTIO_NET_FF_MASK_TYPE_IPV4, then, all the mask bits within
\field{selector[1]} for \field{Protocol} is set.
\item if \field{selector[1].type} is, VIRTIO_NET_FF_MASK_TYPE_IPV6, then, all the mask bits within
\field{selector[1]} for \field{Next Header} is set.
\end{itemize}

If for a given packet header field, a subset of bits of a field is to be matched,
and if the partial mask is supported, the flow filter
mask object can specify a mask which has fewer bits set than the packet header
field size. For example, a partial mask for the Ethernet header source mac
address can be of 1-bit for multicast detection instead of 48-bits.

\subparagraph{VIRTIO_NET_RESOURCE_OBJ_FF_RULE}\label{par:Device Types / Network Device / Device Operation / Flow filter / Resource objects / VIRTIO-NET-RESOURCE-OBJ-FF-RULE}

Each flow filter rule resource object comprises a key, a priority, and an action.
For the flow filter rule object,
\field{resource_obj_specific_data} and
\field{resource_obj_specific_result} are in the format
\field{struct virtio_net_resource_obj_ff_rule}.

\begin{lstlisting}
struct virtio_net_resource_obj_ff_rule {
        le32 group_id;
        le32 classifier_id;
        u8 rule_priority;
        u8 key_length; /* length of key in bytes */
        u8 action;
        u8 reserved;
        le16 vq_index;
        u8 reserved1[2];
        u8 keys[][];
};
\end{lstlisting}

\field{group_id} is the resource object ID of the flow filter group to which
this rule belongs. \field{classifier_id} is the resource object ID of the
classifier used to match a packet against the \field{key}.

\field{rule_priority} denotes the priority of the rule within the group
specified by the \field{group_id}.
Rules within the group are applied from the highest to the lowest priority
until a rule matches the packet and an
action is taken. Rules with the same priority can be applied in any order.

\field{reserved} and \field{reserved1} are reserved and set to 0.

\field{keys[][]} is an array of keys to match against packets, using
the classifier specified by \field{classifier_id}. Each entry (key) comprises
a byte array, and they are located one immediately after another.
The size (number of entries) of the array is exactly the same as that of
\field{selectors} in the classifier, or in other words, \field{count}
in the classifier.

\field{key_length} specifies the total length of \field{keys} in bytes.
In other words, it equals the sum total of \field{length} of all
selectors in \field{selectors} in the classifier specified by
\field{classifier_id}.

For example, if a classifier object's \field{selectors[0].type} is
VIRTIO_NET_FF_MASK_TYPE_ETH and \field{selectors[1].type} is
VIRTIO_NET_FF_MASK_TYPE_IPV6,
then selectors[0].length is 14 and selectors[1].length is 40.
Accordingly, the \field{key_length} is set to 54.
This setting indicates that the \field{key} array's length is 54 bytes
comprising a first byte array of 14 bytes for the
Ethernet MAC header in bytes 0-13, immediately followed by 40 bytes for the
IPv6 header in bytes 14-53.

When there are multiple selectors in the classifier object, the key bytes
for (N)\textsuperscript{th} selector are set so that
(N+1)\textsuperscript{th} selector can be matched.

If \field{count} is 2 or more, key bytes of \field{EtherType}
are set according to \hyperref[intro:IEEE 802 Ethertypes]{IEEE 802 Ethertypes}
for VIRTIO_NET_FF_MASK_TYPE_IPV4 or VIRTIO_NET_FF_MASK_TYPE_IPV6 respectively.

If \field{count} is more than 2, when \field{selector[1].type} is
VIRTIO_NET_FF_MASK_TYPE_IPV4 or VIRTIO_NET_FF_MASK_TYPE_IPV6, key
bytes of \field{Protocol} or \field{Next Header} is set as per
\field{Protocol Numbers} defined \hyperref[intro:IANA Protocol Numbers]{IANA Protocol Numbers}
respectively.

\field{action} is the action to take when a packet matches the
\field{key} using the \field{classifier_id}. Supported actions are described in
\ref{table:Device Types / Network Device / Device Operation / Flow filter / Device and driver capabilities / VIRTIO-NET-FF-ACTION-CAP / flow filter rule actions}.

\field{vq_index} specifies a receive virtqueue. When the \field{action} is set
to VIRTIO_NET_FF_ACTION_DIRECT_RX_VQ, and the packet matches the \field{key},
the matching packet is directed to this virtqueue.

Note that at most one action is ever taken for a given packet. If a rule is
applied and an action is taken, the action of other rules is not taken.

\devicenormative{\paragraph}{Flow filter}{Device Types / Network Device / Device Operation / Flow filter}

When the device supports flow filter operations,
\begin{itemize}
\item the device MUST set VIRTIO_NET_FF_RESOURCE_CAP, VIRTIO_NET_FF_SELECTOR_CAP
and VIRTIO_NET_FF_ACTION_CAP capability in the \field{supported_caps} in the
command VIRTIO_ADMIN_CMD_CAP_SUPPORT_QUERY.
\item the device MUST support the administration commands
VIRTIO_ADMIN_CMD_RESOURCE_OBJ_CREATE,
VIRTIO_ADMIN_CMD_RESOURCE_OBJ_MODIFY, VIRTIO_ADMIN_CMD_RESOURCE_OBJ_QUERY,
VIRTIO_ADMIN_CMD_RESOURCE_OBJ_DESTROY for the resource types
VIRTIO_NET_RESOURCE_OBJ_FF_GROUP, VIRTIO_NET_RESOURCE_OBJ_FF_CLASSIFIER and
VIRTIO_NET_RESOURCE_OBJ_FF_RULE.
\end{itemize}

When any of the VIRTIO_NET_FF_RESOURCE_CAP, VIRTIO_NET_FF_SELECTOR_CAP, or
VIRTIO_NET_FF_ACTION_CAP capability is disabled, the device SHOULD set
\field{status} to VIRTIO_ADMIN_STATUS_Q_INVALID_OPCODE for the commands
VIRTIO_ADMIN_CMD_RESOURCE_OBJ_CREATE,
VIRTIO_ADMIN_CMD_RESOURCE_OBJ_MODIFY, VIRTIO_ADMIN_CMD_RESOURCE_OBJ_QUERY,
and VIRTIO_ADMIN_CMD_RESOURCE_OBJ_DESTROY. These commands apply to the resource
\field{type} of VIRTIO_NET_RESOURCE_OBJ_FF_GROUP, VIRTIO_NET_RESOURCE_OBJ_FF_CLASSIFIER, and
VIRTIO_NET_RESOURCE_OBJ_FF_RULE.

The device SHOULD set \field{status} to VIRTIO_ADMIN_STATUS_EINVAL for the
command VIRTIO_ADMIN_CMD_RESOURCE_OBJ_CREATE when the resource \field{type}
is VIRTIO_NET_RESOURCE_OBJ_FF_GROUP, if a flow filter group already exists
with the supplied \field{group_priority}.

The device SHOULD set \field{status} to VIRTIO_ADMIN_STATUS_ENOSPC for the
command VIRTIO_ADMIN_CMD_RESOURCE_OBJ_CREATE when the resource \field{type}
is VIRTIO_NET_RESOURCE_OBJ_FF_GROUP, if the number of flow filter group
objects in the device exceeds the lower of the configured driver
capabilities \field{groups_limit} and \field{rules_per_group_limit}.

The device SHOULD set \field{status} to VIRTIO_ADMIN_STATUS_ENOSPC for the
command VIRTIO_ADMIN_CMD_RESOURCE_OBJ_CREATE when the resource \field{type} is
VIRTIO_NET_RESOURCE_OBJ_FF_CLASSIFIER, if the number of flow filter selector
objects in the device exceeds the configured driver capability
\field{selectors_limit}.

The device SHOULD set \field{status} to VIRTIO_ADMIN_STATUS_EBUSY for the
command VIRTIO_ADMIN_CMD_RESOURCE_OBJ_DESTROY for a flow filter group when
the flow filter group has one or more flow filter rules depending on it.

The device SHOULD set \field{status} to VIRTIO_ADMIN_STATUS_EBUSY for the
command VIRTIO_ADMIN_CMD_RESOURCE_OBJ_DESTROY for a flow filter classifier when
the flow filter classifier has one or more flow filter rules depending on it.

The device SHOULD fail the command VIRTIO_ADMIN_CMD_RESOURCE_OBJ_CREATE for the
flow filter rule resource object if,
\begin{itemize}
\item \field{vq_index} is not a valid receive virtqueue index for
the VIRTIO_NET_FF_ACTION_DIRECT_RX_VQ action,
\item \field{priority} is greater than or equal to
      \field{last_rule_priority},
\item \field{id} is greater than or equal to \field{rules_limit} or
      greater than or equal to \field{rules_per_group_limit}, whichever is lower,
\item the length of \field{keys} and the length of all the mask bytes of
      \field{selectors[].mask} as referred by \field{classifier_id} differs,
\item the supplied \field{action} is not supported in the capability VIRTIO_NET_FF_ACTION_CAP.
\end{itemize}

When the flow filter directs a packet to the virtqueue identified by
\field{vq_index} and if the receive virtqueue is reset, the device
MUST drop such packets.

Upon applying a flow filter rule to a packet, the device MUST STOP any further
application of rules and cease applying any other steering configurations.

For multiple flow filter groups, the device MUST apply the rules from
the group with the highest priority. If any rule from this group is applied,
the device MUST ignore the remaining groups. If none of the rules from the
highest priority group match, the device MUST apply the rules from
the group with the next highest priority, until either a rule matches or
all groups have been attempted.

The device MUST apply the rules within the group from the highest to the
lowest priority until a rule matches the packet, and the device MUST take
the action. If an action is taken, the device MUST not take any other
action for this packet.

The device MAY apply the rules with the same \field{rule_priority} in any
order within the group.

The device MUST process incoming packets in the following order:
\begin{itemize}
\item apply the steering configuration received using control virtqueue
      commands VIRTIO_NET_CTRL_RX, VIRTIO_NET_CTRL_MAC, and
      VIRTIO_NET_CTRL_VLAN.
\item apply flow filter rules if any.
\item if no filter rule is applied, apply the steering configuration
      received using the command VIRTIO_NET_CTRL_MQ_RSS_CONFIG
      or according to automatic receive steering.
\end{itemize}

When processing an incoming packet, if the packet is dropped at any stage, the device
MUST skip further processing.

When the device drops the packet due to the configuration done using the control
virtqueue commands VIRTIO_NET_CTRL_RX or VIRTIO_NET_CTRL_MAC or VIRTIO_NET_CTRL_VLAN,
the device MUST skip flow filter rules for this packet.

When the device performs flow filter match operations and if the operation
result did not have any match in all the groups, the receive packet processing
continues to next level, i.e. to apply configuration done using
VIRTIO_NET_CTRL_MQ_RSS_CONFIG command.

The device MUST support the creation of flow filter classifier objects
using the command VIRTIO_ADMIN_CMD_RESOURCE_OBJ_CREATE with \field{flags}
set to VIRTIO_NET_FF_MASK_F_PARTIAL_MASK;
this support is required even if all the bits of the masks are set for
a field in \field{selectors}, provided that partial masking is supported
for the selectors.

\drivernormative{\paragraph}{Flow filter}{Device Types / Network Device / Device Operation / Flow filter}

The driver MUST enable VIRTIO_NET_FF_RESOURCE_CAP, VIRTIO_NET_FF_SELECTOR_CAP,
and VIRTIO_NET_FF_ACTION_CAP capabilities to use flow filter.

The driver SHOULD NOT remove a flow filter group using the command
VIRTIO_ADMIN_CMD_RESOURCE_OBJ_DESTROY when one or more flow filter rules
depend on that group. The driver SHOULD only destroy the group after
all the associated rules have been destroyed.

The driver SHOULD NOT remove a flow filter classifier using the command
VIRTIO_ADMIN_CMD_RESOURCE_OBJ_DESTROY when one or more flow filter rules
depend on the classifier. The driver SHOULD only destroy the classifier
after all the associated rules have been destroyed.

The driver SHOULD NOT add multiple flow filter rules with the same
\field{rule_priority} within a flow filter group, as these rules MAY match
the same packet. The driver SHOULD assign different \field{rule_priority}
values to different flow filter rules if multiple rules may match a single
packet.

For the command VIRTIO_ADMIN_CMD_RESOURCE_OBJ_CREATE, when creating a resource
of \field{type} VIRTIO_NET_RESOURCE_OBJ_FF_CLASSIFIER, the driver MUST set:
\begin{itemize}
\item \field{selectors[0].type} to VIRTIO_NET_FF_MASK_TYPE_ETH.
\item \field{selectors[1].type} to VIRTIO_NET_FF_MASK_TYPE_IPV4 or
      VIRTIO_NET_FF_MASK_TYPE_IPV6 when \field{count} is more than 1,
\item \field{selectors[2].type} VIRTIO_NET_FF_MASK_TYPE_UDP or
      VIRTIO_NET_FF_MASK_TYPE_TCP when \field{count} is more than 2.
\end{itemize}

For the command VIRTIO_ADMIN_CMD_RESOURCE_OBJ_CREATE, when creating a resource
of \field{type} VIRTIO_NET_RESOURCE_OBJ_FF_CLASSIFIER, the driver MUST set:
\begin{itemize}
\item \field{selectors[0].mask} bytes to all 1s for the \field{EtherType}
       when \field{count} is 2 or more.
\item \field{selectors[1].mask} bytes to all 1s for \field{Protocol} or \field{Next Header}
       when \field{selector[1].type} is VIRTIO_NET_FF_MASK_TYPE_IPV4 or VIRTIO_NET_FF_MASK_TYPE_IPV6,
       and when \field{count} is more than 2.
\end{itemize}

For the command VIRTIO_ADMIN_CMD_RESOURCE_OBJ_CREATE, the resource \field{type}
VIRTIO_NET_RESOURCE_OBJ_FF_RULE, if the corresponding classifier object's
\field{count} is 2 or more, the driver MUST SET the \field{keys} bytes of
\field{EtherType} in accordance with
\hyperref[intro:IEEE 802 Ethertypes]{IEEE 802 Ethertypes}
for either VIRTIO_NET_FF_MASK_TYPE_IPV4 or VIRTIO_NET_FF_MASK_TYPE_IPV6.

For the command VIRTIO_ADMIN_CMD_RESOURCE_OBJ_CREATE, when creating a resource of
\field{type} VIRTIO_NET_RESOURCE_OBJ_FF_RULE, if the corresponding classifier
object's \field{count} is more than 2, and the \field{selector[1].type} is either
VIRTIO_NET_FF_MASK_TYPE_IPV4 or VIRTIO_NET_FF_MASK_TYPE_IPV6, the driver MUST
set the \field{keys} bytes for the \field{Protocol} or \field{Next Header}
according to \hyperref[intro:IANA Protocol Numbers]{IANA Protocol Numbers} respectively.

The driver SHOULD set all the bits for a field in the mask of a selector in both the
capability and the classifier object, unless the VIRTIO_NET_FF_MASK_F_PARTIAL_MASK
is enabled.

\subsubsection{Legacy Interface: Framing Requirements}\label{sec:Device
Types / Network Device / Legacy Interface: Framing Requirements}

When using legacy interfaces, transitional drivers which have not
negotiated VIRTIO_F_ANY_LAYOUT MUST use a single descriptor for the
\field{struct virtio_net_hdr} on both transmit and receive, with the
network data in the following descriptors.

Additionally, when using the control virtqueue (see \ref{sec:Device
Types / Network Device / Device Operation / Control Virtqueue})
, transitional drivers which have not
negotiated VIRTIO_F_ANY_LAYOUT MUST:
\begin{itemize}
\item for all commands, use a single 2-byte descriptor including the first two
fields: \field{class} and \field{command}
\item for all commands except VIRTIO_NET_CTRL_MAC_TABLE_SET
use a single descriptor including command-specific-data
with no padding.
\item for the VIRTIO_NET_CTRL_MAC_TABLE_SET command use exactly
two descriptors including command-specific-data with no padding:
the first of these descriptors MUST include the
virtio_net_ctrl_mac table structure for the unicast addresses with no padding,
the second of these descriptors MUST include the
virtio_net_ctrl_mac table structure for the multicast addresses
with no padding.
\item for all commands, use a single 1-byte descriptor for the
\field{ack} field
\end{itemize}

See \ref{sec:Basic
Facilities of a Virtio Device / Virtqueues / Message Framing}.

\section{Network Device}\label{sec:Device Types / Network Device}

The virtio network device is a virtual network interface controller.
It consists of a virtual Ethernet link which connects the device
to the Ethernet network. The device has transmit and receive
queues. The driver adds empty buffers to the receive virtqueue.
The device receives incoming packets from the link; the device
places these incoming packets in the receive virtqueue buffers.
The driver adds outgoing packets to the transmit virtqueue. The device
removes these packets from the transmit virtqueue and sends them to
the link. The device may have a control virtqueue. The driver
uses the control virtqueue to dynamically manipulate various
features of the initialized device.

\subsection{Device ID}\label{sec:Device Types / Network Device / Device ID}

 1

\subsection{Virtqueues}\label{sec:Device Types / Network Device / Virtqueues}

\begin{description}
\item[0] receiveq1
\item[1] transmitq1
\item[\ldots]
\item[2(N-1)] receiveqN
\item[2(N-1)+1] transmitqN
\item[2N] controlq
\end{description}

 N=1 if neither VIRTIO_NET_F_MQ nor VIRTIO_NET_F_RSS are negotiated, otherwise N is set by
 \field{max_virtqueue_pairs}.

controlq is optional; it only exists if VIRTIO_NET_F_CTRL_VQ is
negotiated.

\subsection{Feature bits}\label{sec:Device Types / Network Device / Feature bits}

\begin{description}
\item[VIRTIO_NET_F_CSUM (0)] Device handles packets with partial checksum offload.

\item[VIRTIO_NET_F_GUEST_CSUM (1)] Driver handles packets with partial checksum.

\item[VIRTIO_NET_F_CTRL_GUEST_OFFLOADS (2)] Control channel offloads
        reconfiguration support.

\item[VIRTIO_NET_F_MTU(3)] Device maximum MTU reporting is supported. If
    offered by the device, device advises driver about the value of
    its maximum MTU. If negotiated, the driver uses \field{mtu} as
    the maximum MTU value.

\item[VIRTIO_NET_F_MAC (5)] Device has given MAC address.

\item[VIRTIO_NET_F_GUEST_TSO4 (7)] Driver can receive TSOv4.

\item[VIRTIO_NET_F_GUEST_TSO6 (8)] Driver can receive TSOv6.

\item[VIRTIO_NET_F_GUEST_ECN (9)] Driver can receive TSO with ECN.

\item[VIRTIO_NET_F_GUEST_UFO (10)] Driver can receive UFO.

\item[VIRTIO_NET_F_HOST_TSO4 (11)] Device can receive TSOv4.

\item[VIRTIO_NET_F_HOST_TSO6 (12)] Device can receive TSOv6.

\item[VIRTIO_NET_F_HOST_ECN (13)] Device can receive TSO with ECN.

\item[VIRTIO_NET_F_HOST_UFO (14)] Device can receive UFO.

\item[VIRTIO_NET_F_MRG_RXBUF (15)] Driver can merge receive buffers.

\item[VIRTIO_NET_F_STATUS (16)] Configuration status field is
    available.

\item[VIRTIO_NET_F_CTRL_VQ (17)] Control channel is available.

\item[VIRTIO_NET_F_CTRL_RX (18)] Control channel RX mode support.

\item[VIRTIO_NET_F_CTRL_VLAN (19)] Control channel VLAN filtering.

\item[VIRTIO_NET_F_CTRL_RX_EXTRA (20)]	Control channel RX extra mode support.

\item[VIRTIO_NET_F_GUEST_ANNOUNCE(21)] Driver can send gratuitous
    packets.

\item[VIRTIO_NET_F_MQ(22)] Device supports multiqueue with automatic
    receive steering.

\item[VIRTIO_NET_F_CTRL_MAC_ADDR(23)] Set MAC address through control
    channel.

\item[VIRTIO_NET_F_DEVICE_STATS(50)] Device can provide device-level statistics
    to the driver through the control virtqueue.

\item[VIRTIO_NET_F_HASH_TUNNEL(51)] Device supports inner header hash for encapsulated packets.

\item[VIRTIO_NET_F_VQ_NOTF_COAL(52)] Device supports virtqueue notification coalescing.

\item[VIRTIO_NET_F_NOTF_COAL(53)] Device supports notifications coalescing.

\item[VIRTIO_NET_F_GUEST_USO4 (54)] Driver can receive USOv4 packets.

\item[VIRTIO_NET_F_GUEST_USO6 (55)] Driver can receive USOv6 packets.

\item[VIRTIO_NET_F_HOST_USO (56)] Device can receive USO packets. Unlike UFO
 (fragmenting the packet) the USO splits large UDP packet
 to several segments when each of these smaller packets has UDP header.

\item[VIRTIO_NET_F_HASH_REPORT(57)] Device can report per-packet hash
    value and a type of calculated hash.

\item[VIRTIO_NET_F_GUEST_HDRLEN(59)] Driver can provide the exact \field{hdr_len}
    value. Device benefits from knowing the exact header length.

\item[VIRTIO_NET_F_RSS(60)] Device supports RSS (receive-side scaling)
    with Toeplitz hash calculation and configurable hash
    parameters for receive steering.

\item[VIRTIO_NET_F_RSC_EXT(61)] Device can process duplicated ACKs
    and report number of coalesced segments and duplicated ACKs.

\item[VIRTIO_NET_F_STANDBY(62)] Device may act as a standby for a primary
    device with the same MAC address.

\item[VIRTIO_NET_F_SPEED_DUPLEX(63)] Device reports speed and duplex.

\item[VIRTIO_NET_F_RSS_CONTEXT(64)] Device supports multiple RSS contexts.

\item[VIRTIO_NET_F_GUEST_UDP_TUNNEL_GSO (65)] Driver can receive GSO packets
  carried by a UDP tunnel.

\item[VIRTIO_NET_F_GUEST_UDP_TUNNEL_GSO_CSUM (66)] Driver handles packets
  carried by a UDP tunnel with partial csum for the outer header.

\item[VIRTIO_NET_F_HOST_UDP_TUNNEL_GSO (67)] Device can receive GSO packets
  carried by a UDP tunnel.

\item[VIRTIO_NET_F_HOST_UDP_TUNNEL_GSO_CSUM (68)] Device handles packets
  carried by a UDP tunnel with partial csum for the outer header.
\end{description}

\subsubsection{Feature bit requirements}\label{sec:Device Types / Network Device / Feature bits / Feature bit requirements}

Some networking feature bits require other networking feature bits
(see \ref{drivernormative:Basic Facilities of a Virtio Device / Feature Bits}):

\begin{description}
\item[VIRTIO_NET_F_GUEST_TSO4] Requires VIRTIO_NET_F_GUEST_CSUM.
\item[VIRTIO_NET_F_GUEST_TSO6] Requires VIRTIO_NET_F_GUEST_CSUM.
\item[VIRTIO_NET_F_GUEST_ECN] Requires VIRTIO_NET_F_GUEST_TSO4 or VIRTIO_NET_F_GUEST_TSO6.
\item[VIRTIO_NET_F_GUEST_UFO] Requires VIRTIO_NET_F_GUEST_CSUM.
\item[VIRTIO_NET_F_GUEST_USO4] Requires VIRTIO_NET_F_GUEST_CSUM.
\item[VIRTIO_NET_F_GUEST_USO6] Requires VIRTIO_NET_F_GUEST_CSUM.
\item[VIRTIO_NET_F_GUEST_UDP_TUNNEL_GSO] Requires VIRTIO_NET_F_GUEST_TSO4, VIRTIO_NET_F_GUEST_TSO6,
   VIRTIO_NET_F_GUEST_USO4 and VIRTIO_NET_F_GUEST_USO6.
\item[VIRTIO_NET_F_GUEST_UDP_TUNNEL_GSO_CSUM] Requires VIRTIO_NET_F_GUEST_UDP_TUNNEL_GSO

\item[VIRTIO_NET_F_HOST_TSO4] Requires VIRTIO_NET_F_CSUM.
\item[VIRTIO_NET_F_HOST_TSO6] Requires VIRTIO_NET_F_CSUM.
\item[VIRTIO_NET_F_HOST_ECN] Requires VIRTIO_NET_F_HOST_TSO4 or VIRTIO_NET_F_HOST_TSO6.
\item[VIRTIO_NET_F_HOST_UFO] Requires VIRTIO_NET_F_CSUM.
\item[VIRTIO_NET_F_HOST_USO] Requires VIRTIO_NET_F_CSUM.
\item[VIRTIO_NET_F_HOST_UDP_TUNNEL_GSO] Requires VIRTIO_NET_F_HOST_TSO4, VIRTIO_NET_F_HOST_TSO6
   and VIRTIO_NET_F_HOST_USO.
\item[VIRTIO_NET_F_HOST_UDP_TUNNEL_GSO_CSUM] Requires VIRTIO_NET_F_HOST_UDP_TUNNEL_GSO

\item[VIRTIO_NET_F_CTRL_RX] Requires VIRTIO_NET_F_CTRL_VQ.
\item[VIRTIO_NET_F_CTRL_VLAN] Requires VIRTIO_NET_F_CTRL_VQ.
\item[VIRTIO_NET_F_GUEST_ANNOUNCE] Requires VIRTIO_NET_F_CTRL_VQ.
\item[VIRTIO_NET_F_MQ] Requires VIRTIO_NET_F_CTRL_VQ.
\item[VIRTIO_NET_F_CTRL_MAC_ADDR] Requires VIRTIO_NET_F_CTRL_VQ.
\item[VIRTIO_NET_F_NOTF_COAL] Requires VIRTIO_NET_F_CTRL_VQ.
\item[VIRTIO_NET_F_RSC_EXT] Requires VIRTIO_NET_F_HOST_TSO4 or VIRTIO_NET_F_HOST_TSO6.
\item[VIRTIO_NET_F_RSS] Requires VIRTIO_NET_F_CTRL_VQ.
\item[VIRTIO_NET_F_VQ_NOTF_COAL] Requires VIRTIO_NET_F_CTRL_VQ.
\item[VIRTIO_NET_F_HASH_TUNNEL] Requires VIRTIO_NET_F_CTRL_VQ along with VIRTIO_NET_F_RSS or VIRTIO_NET_F_HASH_REPORT.
\item[VIRTIO_NET_F_RSS_CONTEXT] Requires VIRTIO_NET_F_CTRL_VQ and VIRTIO_NET_F_RSS.
\end{description}

\begin{note}
The dependency between UDP_TUNNEL_GSO_CSUM and UDP_TUNNEL_GSO is intentionally
in the opposite direction with respect to the plain GSO features and the plain
checksum offload because UDP tunnel checksum offload gives very little gain
for non GSO packets and is quite complex to implement in H/W.
\end{note}

\subsubsection{Legacy Interface: Feature bits}\label{sec:Device Types / Network Device / Feature bits / Legacy Interface: Feature bits}
\begin{description}
\item[VIRTIO_NET_F_GSO (6)] Device handles packets with any GSO type. This was supposed to indicate segmentation offload support, but
upon further investigation it became clear that multiple bits were needed.
\item[VIRTIO_NET_F_GUEST_RSC4 (41)] Device coalesces TCPIP v4 packets. This was implemented by hypervisor patch for certification
purposes and current Windows driver depends on it. It will not function if virtio-net device reports this feature.
\item[VIRTIO_NET_F_GUEST_RSC6 (42)] Device coalesces TCPIP v6 packets. Similar to VIRTIO_NET_F_GUEST_RSC4.
\end{description}

\subsection{Device configuration layout}\label{sec:Device Types / Network Device / Device configuration layout}
\label{sec:Device Types / Block Device / Feature bits / Device configuration layout}

The network device has the following device configuration layout.
All of the device configuration fields are read-only for the driver.

\begin{lstlisting}
struct virtio_net_config {
        u8 mac[6];
        le16 status;
        le16 max_virtqueue_pairs;
        le16 mtu;
        le32 speed;
        u8 duplex;
        u8 rss_max_key_size;
        le16 rss_max_indirection_table_length;
        le32 supported_hash_types;
        le32 supported_tunnel_types;
};
\end{lstlisting}

The \field{mac} address field always exists (although it is only
valid if VIRTIO_NET_F_MAC is set).

The \field{status} only exists if VIRTIO_NET_F_STATUS is set.
Two bits are currently defined for the status field: VIRTIO_NET_S_LINK_UP
and VIRTIO_NET_S_ANNOUNCE.

\begin{lstlisting}
#define VIRTIO_NET_S_LINK_UP     1
#define VIRTIO_NET_S_ANNOUNCE    2
\end{lstlisting}

The following field, \field{max_virtqueue_pairs} only exists if
VIRTIO_NET_F_MQ or VIRTIO_NET_F_RSS is set. This field specifies the maximum number
of each of transmit and receive virtqueues (receiveq1\ldots receiveqN
and transmitq1\ldots transmitqN respectively) that can be configured once at least one of these features
is negotiated.

The following field, \field{mtu} only exists if VIRTIO_NET_F_MTU
is set. This field specifies the maximum MTU for the driver to
use.

The following two fields, \field{speed} and \field{duplex}, only
exist if VIRTIO_NET_F_SPEED_DUPLEX is set.

\field{speed} contains the device speed, in units of 1 MBit per
second, 0 to 0x7fffffff, or 0xffffffff for unknown speed.

\field{duplex} has the values of 0x01 for full duplex, 0x00 for
half duplex and 0xff for unknown duplex state.

Both \field{speed} and \field{duplex} can change, thus the driver
is expected to re-read these values after receiving a
configuration change notification.

The following field, \field{rss_max_key_size} only exists if VIRTIO_NET_F_RSS or VIRTIO_NET_F_HASH_REPORT is set.
It specifies the maximum supported length of RSS key in bytes.

The following field, \field{rss_max_indirection_table_length} only exists if VIRTIO_NET_F_RSS is set.
It specifies the maximum number of 16-bit entries in RSS indirection table.

The next field, \field{supported_hash_types} only exists if the device supports hash calculation,
i.e. if VIRTIO_NET_F_RSS or VIRTIO_NET_F_HASH_REPORT is set.

Field \field{supported_hash_types} contains the bitmask of supported hash types.
See \ref{sec:Device Types / Network Device / Device Operation / Processing of Incoming Packets / Hash calculation for incoming packets / Supported/enabled hash types} for details of supported hash types.

Field \field{supported_tunnel_types} only exists if the device supports inner header hash, i.e. if VIRTIO_NET_F_HASH_TUNNEL is set.

Field \field{supported_tunnel_types} contains the bitmask of encapsulation types supported by the device for inner header hash.
Encapsulation types are defined in \ref{sec:Device Types / Network Device / Device Operation / Processing of Incoming Packets /
Hash calculation for incoming packets / Encapsulation types supported/enabled for inner header hash}.

\devicenormative{\subsubsection}{Device configuration layout}{Device Types / Network Device / Device configuration layout}

The device MUST set \field{max_virtqueue_pairs} to between 1 and 0x8000 inclusive,
if it offers VIRTIO_NET_F_MQ.

The device MUST set \field{mtu} to between 68 and 65535 inclusive,
if it offers VIRTIO_NET_F_MTU.

The device SHOULD set \field{mtu} to at least 1280, if it offers
VIRTIO_NET_F_MTU.

The device MUST NOT modify \field{mtu} once it has been set.

The device MUST NOT pass received packets that exceed \field{mtu} (plus low
level ethernet header length) size with \field{gso_type} NONE or ECN
after VIRTIO_NET_F_MTU has been successfully negotiated.

The device MUST forward transmitted packets of up to \field{mtu} (plus low
level ethernet header length) size with \field{gso_type} NONE or ECN, and do
so without fragmentation, after VIRTIO_NET_F_MTU has been successfully
negotiated.

The device MUST set \field{rss_max_key_size} to at least 40, if it offers
VIRTIO_NET_F_RSS or VIRTIO_NET_F_HASH_REPORT.

The device MUST set \field{rss_max_indirection_table_length} to at least 128, if it offers
VIRTIO_NET_F_RSS.

If the driver negotiates the VIRTIO_NET_F_STANDBY feature, the device MAY act
as a standby device for a primary device with the same MAC address.

If VIRTIO_NET_F_SPEED_DUPLEX has been negotiated, \field{speed}
MUST contain the device speed, in units of 1 MBit per second, 0 to
0x7ffffffff, or 0xfffffffff for unknown.

If VIRTIO_NET_F_SPEED_DUPLEX has been negotiated, \field{duplex}
MUST have the values of 0x00 for full duplex, 0x01 for half
duplex, or 0xff for unknown.

If VIRTIO_NET_F_SPEED_DUPLEX and VIRTIO_NET_F_STATUS have both
been negotiated, the device SHOULD NOT change the \field{speed} and
\field{duplex} fields as long as VIRTIO_NET_S_LINK_UP is set in
the \field{status}.

The device SHOULD NOT offer VIRTIO_NET_F_HASH_REPORT if it
does not offer VIRTIO_NET_F_CTRL_VQ.

The device SHOULD NOT offer VIRTIO_NET_F_CTRL_RX_EXTRA if it
does not offer VIRTIO_NET_F_CTRL_VQ.

\drivernormative{\subsubsection}{Device configuration layout}{Device Types / Network Device / Device configuration layout}

The driver MUST NOT write to any of the device configuration fields.

A driver SHOULD negotiate VIRTIO_NET_F_MAC if the device offers it.
If the driver negotiates the VIRTIO_NET_F_MAC feature, the driver MUST set
the physical address of the NIC to \field{mac}.  Otherwise, it SHOULD
use a locally-administered MAC address (see \hyperref[intro:IEEE 802]{IEEE 802},
``9.2 48-bit universal LAN MAC addresses'').

If the driver does not negotiate the VIRTIO_NET_F_STATUS feature, it SHOULD
assume the link is active, otherwise it SHOULD read the link status from
the bottom bit of \field{status}.

A driver SHOULD negotiate VIRTIO_NET_F_MTU if the device offers it.

If the driver negotiates VIRTIO_NET_F_MTU, it MUST supply enough receive
buffers to receive at least one receive packet of size \field{mtu} (plus low
level ethernet header length) with \field{gso_type} NONE or ECN.

If the driver negotiates VIRTIO_NET_F_MTU, it MUST NOT transmit packets of
size exceeding the value of \field{mtu} (plus low level ethernet header length)
with \field{gso_type} NONE or ECN.

A driver SHOULD negotiate the VIRTIO_NET_F_STANDBY feature if the device offers it.

If VIRTIO_NET_F_SPEED_DUPLEX has been negotiated,
the driver MUST treat any value of \field{speed} above
0x7fffffff as well as any value of \field{duplex} not
matching 0x00 or 0x01 as an unknown value.

If VIRTIO_NET_F_SPEED_DUPLEX has been negotiated, the driver
SHOULD re-read \field{speed} and \field{duplex} after a
configuration change notification.

A driver SHOULD NOT negotiate VIRTIO_NET_F_HASH_REPORT if it
does not negotiate VIRTIO_NET_F_CTRL_VQ.

A driver SHOULD NOT negotiate VIRTIO_NET_F_CTRL_RX_EXTRA if it
does not negotiate VIRTIO_NET_F_CTRL_VQ.

\subsubsection{Legacy Interface: Device configuration layout}\label{sec:Device Types / Network Device / Device configuration layout / Legacy Interface: Device configuration layout}
\label{sec:Device Types / Block Device / Feature bits / Device configuration layout / Legacy Interface: Device configuration layout}
When using the legacy interface, transitional devices and drivers
MUST format \field{status} and
\field{max_virtqueue_pairs} in struct virtio_net_config
according to the native endian of the guest rather than
(necessarily when not using the legacy interface) little-endian.

When using the legacy interface, \field{mac} is driver-writable
which provided a way for drivers to update the MAC without
negotiating VIRTIO_NET_F_CTRL_MAC_ADDR.

\subsection{Device Initialization}\label{sec:Device Types / Network Device / Device Initialization}

A driver would perform a typical initialization routine like so:

\begin{enumerate}
\item Identify and initialize the receive and
  transmission virtqueues, up to N of each kind. If
  VIRTIO_NET_F_MQ feature bit is negotiated,
  N=\field{max_virtqueue_pairs}, otherwise identify N=1.

\item If the VIRTIO_NET_F_CTRL_VQ feature bit is negotiated,
  identify the control virtqueue.

\item Fill the receive queues with buffers: see \ref{sec:Device Types / Network Device / Device Operation / Setting Up Receive Buffers}.

\item Even with VIRTIO_NET_F_MQ, only receiveq1, transmitq1 and
  controlq are used by default.  The driver would send the
  VIRTIO_NET_CTRL_MQ_VQ_PAIRS_SET command specifying the
  number of the transmit and receive queues to use.

\item If the VIRTIO_NET_F_MAC feature bit is set, the configuration
  space \field{mac} entry indicates the ``physical'' address of the
  device, otherwise the driver would typically generate a random
  local MAC address.

\item If the VIRTIO_NET_F_STATUS feature bit is negotiated, the link
  status comes from the bottom bit of \field{status}.
  Otherwise, the driver assumes it's active.

\item A performant driver would indicate that it will generate checksumless
  packets by negotiating the VIRTIO_NET_F_CSUM feature.

\item If that feature is negotiated, a driver can use TCP segmentation or UDP
  segmentation/fragmentation offload by negotiating the VIRTIO_NET_F_HOST_TSO4 (IPv4
  TCP), VIRTIO_NET_F_HOST_TSO6 (IPv6 TCP), VIRTIO_NET_F_HOST_UFO
  (UDP fragmentation) and VIRTIO_NET_F_HOST_USO (UDP segmentation) features.

\item If the VIRTIO_NET_F_HOST_TSO6, VIRTIO_NET_F_HOST_TSO4 and VIRTIO_NET_F_HOST_USO
  segmentation features are negotiated, a driver can
  use TCP segmentation or UDP segmentation on top of UDP encapsulation
  offload, when the outer header does not require checksumming - e.g.
  the outer UDP checksum is zero - by negotiating the
  VIRTIO_NET_F_HOST_UDP_TUNNEL_GSO feature.
  GSO over UDP tunnels packets carry two sets of headers: the outer ones
  and the inner ones. The outer transport protocol is UDP, the inner
  could be either TCP or UDP. Only a single level of encapsulation
  offload is supported.

\item If VIRTIO_NET_F_HOST_UDP_TUNNEL_GSO is negotiated, a driver can
  additionally use TCP segmentation or UDP segmentation on top of UDP
  encapsulation with the outer header requiring checksum offload,
  negotiating the VIRTIO_NET_F_HOST_UDP_TUNNEL_GSO_CSUM feature.

\item The converse features are also available: a driver can save
  the virtual device some work by negotiating these features.\note{For example, a network packet transported between two guests on
the same system might not need checksumming at all, nor segmentation,
if both guests are amenable.}
   The VIRTIO_NET_F_GUEST_CSUM feature indicates that partially
  checksummed packets can be received, and if it can do that then
  the VIRTIO_NET_F_GUEST_TSO4, VIRTIO_NET_F_GUEST_TSO6,
  VIRTIO_NET_F_GUEST_UFO, VIRTIO_NET_F_GUEST_ECN, VIRTIO_NET_F_GUEST_USO4,
  VIRTIO_NET_F_GUEST_USO6 VIRTIO_NET_F_GUEST_UDP_TUNNEL_GSO and
  VIRTIO_NET_F_GUEST_UDP_TUNNEL_GSO_CSUM are the input equivalents of
  the features described above.
  See \ref{sec:Device Types / Network Device / Device Operation /
Setting Up Receive Buffers}~\nameref{sec:Device Types / Network
Device / Device Operation / Setting Up Receive Buffers} and
\ref{sec:Device Types / Network Device / Device Operation /
Processing of Incoming Packets}~\nameref{sec:Device Types /
Network Device / Device Operation / Processing of Incoming Packets} below.
\end{enumerate}

A truly minimal driver would only accept VIRTIO_NET_F_MAC and ignore
everything else.

\subsection{Device and driver capabilities}\label{sec:Device Types / Network Device / Device and driver capabilities}

The network device has the following capabilities.

\begin{tabularx}{\textwidth}{ |l||l|X| }
\hline
Identifier & Name & Description \\
\hline \hline
0x0800 & \hyperref[par:Device Types / Network Device / Device Operation / Flow filter / Device and driver capabilities / VIRTIO-NET-FF-RESOURCE-CAP]{VIRTIO_NET_FF_RESOURCE_CAP} & Flow filter resource capability \\
\hline
0x0801 & \hyperref[par:Device Types / Network Device / Device Operation / Flow filter / Device and driver capabilities / VIRTIO-NET-FF-SELECTOR-CAP]{VIRTIO_NET_FF_SELECTOR_CAP} & Flow filter classifier capability \\
\hline
0x0802 & \hyperref[par:Device Types / Network Device / Device Operation / Flow filter / Device and driver capabilities / VIRTIO-NET-FF-ACTION-CAP]{VIRTIO_NET_FF_ACTION_CAP} & Flow filter action capability \\
\hline
\end{tabularx}

\subsection{Device resource objects}\label{sec:Device Types / Network Device / Device resource objects}

The network device has the following resource objects.

\begin{tabularx}{\textwidth}{ |l||l|X| }
\hline
type & Name & Description \\
\hline \hline
0x0200 & \hyperref[par:Device Types / Network Device / Device Operation / Flow filter / Resource objects / VIRTIO-NET-RESOURCE-OBJ-FF-GROUP]{VIRTIO_NET_RESOURCE_OBJ_FF_GROUP} & Flow filter group resource object \\
\hline
0x0201 & \hyperref[par:Device Types / Network Device / Device Operation / Flow filter / Resource objects / VIRTIO-NET-RESOURCE-OBJ-FF-CLASSIFIER]{VIRTIO_NET_RESOURCE_OBJ_FF_CLASSIFIER} & Flow filter mask object \\
\hline
0x0202 & \hyperref[par:Device Types / Network Device / Device Operation / Flow filter / Resource objects / VIRTIO-NET-RESOURCE-OBJ-FF-RULE]{VIRTIO_NET_RESOURCE_OBJ_FF_RULE} & Flow filter rule object \\
\hline
\end{tabularx}

\subsection{Device parts}\label{sec:Device Types / Network Device / Device parts}

Network device parts represent the configuration done by the driver using control
virtqueue commands. Network device part is in the format of
\field{struct virtio_dev_part}.

\begin{tabularx}{\textwidth}{ |l||l|X| }
\hline
Type & Name & Description \\
\hline \hline
0x200 & VIRTIO_NET_DEV_PART_CVQ_CFG_PART & Represents device configuration done through a control virtqueue command, see \ref{sec:Device Types / Network Device / Device parts / VIRTIO-NET-DEV-PART-CVQ-CFG-PART} \\
\hline
0x201 - 0x5FF & - & reserved for future \\
\hline
\hline
\end{tabularx}

\subsubsection{VIRTIO_NET_DEV_PART_CVQ_CFG_PART}\label{sec:Device Types / Network Device / Device parts / VIRTIO-NET-DEV-PART-CVQ-CFG-PART}

For VIRTIO_NET_DEV_PART_CVQ_CFG_PART, \field{part_type} is set to 0x200. The
VIRTIO_NET_DEV_PART_CVQ_CFG_PART part indicates configuration performed by the
driver using a control virtqueue command.

\begin{lstlisting}
struct virtio_net_dev_part_cvq_selector {
        u8 class;
        u8 command;
        u8 reserved[6];
};
\end{lstlisting}

There is one device part of type VIRTIO_NET_DEV_PART_CVQ_CFG_PART for each
individual configuration. Each part is identified by a unique selector value.
The selector, \field{device_type_raw}, is in the format
\field{struct virtio_net_dev_part_cvq_selector}.

The selector consists of two fields: \field{class} and \field{command}. These
fields correspond to the \field{class} and \field{command} defined in
\field{struct virtio_net_ctrl}, as described in the relevant sections of
\ref{sec:Device Types / Network Device / Device Operation / Control Virtqueue}.

The value corresponding to each part’s selector follows the same format as the
respective \field{command-specific-data} described in the relevant sections of
\ref{sec:Device Types / Network Device / Device Operation / Control Virtqueue}.

For example, when the \field{class} is VIRTIO_NET_CTRL_MAC, the \field{command}
can be either VIRTIO_NET_CTRL_MAC_TABLE_SET or VIRTIO_NET_CTRL_MAC_ADDR_SET;
when \field{command} is set to VIRTIO_NET_CTRL_MAC_TABLE_SET, \field{value}
is in the format of \field{struct virtio_net_ctrl_mac}.

Supported selectors are listed in the table:

\begin{tabularx}{\textwidth}{ |l|X| }
\hline
Class selector & Command selector \\
\hline \hline
VIRTIO_NET_CTRL_RX & VIRTIO_NET_CTRL_RX_PROMISC \\
\hline
VIRTIO_NET_CTRL_RX & VIRTIO_NET_CTRL_RX_ALLMULTI \\
\hline
VIRTIO_NET_CTRL_RX & VIRTIO_NET_CTRL_RX_ALLUNI \\
\hline
VIRTIO_NET_CTRL_RX & VIRTIO_NET_CTRL_RX_NOMULTI \\
\hline
VIRTIO_NET_CTRL_RX & VIRTIO_NET_CTRL_RX_NOUNI \\
\hline
VIRTIO_NET_CTRL_RX & VIRTIO_NET_CTRL_RX_NOBCAST \\
\hline
VIRTIO_NET_CTRL_MAC & VIRTIO_NET_CTRL_MAC_TABLE_SET \\
\hline
VIRTIO_NET_CTRL_MAC & VIRTIO_NET_CTRL_MAC_ADDR_SET \\
\hline
VIRTIO_NET_CTRL_VLAN & VIRTIO_NET_CTRL_VLAN_ADD \\
\hline
VIRTIO_NET_CTRL_ANNOUNCE & VIRTIO_NET_CTRL_ANNOUNCE_ACK \\
\hline
VIRTIO_NET_CTRL_MQ & VIRTIO_NET_CTRL_MQ_VQ_PAIRS_SET \\
\hline
VIRTIO_NET_CTRL_MQ & VIRTIO_NET_CTRL_MQ_RSS_CONFIG \\
\hline
VIRTIO_NET_CTRL_MQ & VIRTIO_NET_CTRL_MQ_HASH_CONFIG \\
\hline
\hline
\end{tabularx}

For command selector VIRTIO_NET_CTRL_VLAN_ADD, device part consists of a whole
VLAN table.

\field{reserved} is reserved and set to zero.

\subsection{Device Operation}\label{sec:Device Types / Network Device / Device Operation}

Packets are transmitted by placing them in the
transmitq1\ldots transmitqN, and buffers for incoming packets are
placed in the receiveq1\ldots receiveqN. In each case, the packet
itself is preceded by a header:

\begin{lstlisting}
struct virtio_net_hdr {
#define VIRTIO_NET_HDR_F_NEEDS_CSUM    1
#define VIRTIO_NET_HDR_F_DATA_VALID    2
#define VIRTIO_NET_HDR_F_RSC_INFO      4
#define VIRTIO_NET_HDR_F_UDP_TUNNEL_CSUM 8
        u8 flags;
#define VIRTIO_NET_HDR_GSO_NONE        0
#define VIRTIO_NET_HDR_GSO_TCPV4       1
#define VIRTIO_NET_HDR_GSO_UDP         3
#define VIRTIO_NET_HDR_GSO_TCPV6       4
#define VIRTIO_NET_HDR_GSO_UDP_L4      5
#define VIRTIO_NET_HDR_GSO_UDP_TUNNEL_IPV4 0x20
#define VIRTIO_NET_HDR_GSO_UDP_TUNNEL_IPV6 0x40
#define VIRTIO_NET_HDR_GSO_ECN      0x80
        u8 gso_type;
        le16 hdr_len;
        le16 gso_size;
        le16 csum_start;
        le16 csum_offset;
        le16 num_buffers;
        le32 hash_value;        (Only if VIRTIO_NET_F_HASH_REPORT negotiated)
        le16 hash_report;       (Only if VIRTIO_NET_F_HASH_REPORT negotiated)
        le16 padding_reserved;  (Only if VIRTIO_NET_F_HASH_REPORT negotiated)
        le16 outer_th_offset    (Only if VIRTIO_NET_F_HOST_UDP_TUNNEL_GSO or VIRTIO_NET_F_GUEST_UDP_TUNNEL_GSO negotiated)
        le16 inner_nh_offset;   (Only if VIRTIO_NET_F_HOST_UDP_TUNNEL_GSO or VIRTIO_NET_F_GUEST_UDP_TUNNEL_GSO negotiated)
};
\end{lstlisting}

The controlq is used to control various device features described further in
section \ref{sec:Device Types / Network Device / Device Operation / Control Virtqueue}.

\subsubsection{Legacy Interface: Device Operation}\label{sec:Device Types / Network Device / Device Operation / Legacy Interface: Device Operation}
When using the legacy interface, transitional devices and drivers
MUST format the fields in \field{struct virtio_net_hdr}
according to the native endian of the guest rather than
(necessarily when not using the legacy interface) little-endian.

The legacy driver only presented \field{num_buffers} in the \field{struct virtio_net_hdr}
when VIRTIO_NET_F_MRG_RXBUF was negotiated; without that feature the
structure was 2 bytes shorter.

When using the legacy interface, the driver SHOULD ignore the
used length for the transmit queues
and the controlq queue.
\begin{note}
Historically, some devices put
the total descriptor length there, even though no data was
actually written.
\end{note}

\subsubsection{Packet Transmission}\label{sec:Device Types / Network Device / Device Operation / Packet Transmission}

Transmitting a single packet is simple, but varies depending on
the different features the driver negotiated.

\begin{enumerate}
\item The driver can send a completely checksummed packet.  In this case,
  \field{flags} will be zero, and \field{gso_type} will be VIRTIO_NET_HDR_GSO_NONE.

\item If the driver negotiated VIRTIO_NET_F_CSUM, it can skip
  checksumming the packet:
  \begin{itemize}
  \item \field{flags} has the VIRTIO_NET_HDR_F_NEEDS_CSUM set,

  \item \field{csum_start} is set to the offset within the packet to begin checksumming,
    and

  \item \field{csum_offset} indicates how many bytes after the csum_start the
    new (16 bit ones' complement) checksum is placed by the device.

  \item The TCP checksum field in the packet is set to the sum
    of the TCP pseudo header, so that replacing it by the ones'
    complement checksum of the TCP header and body will give the
    correct result.
  \end{itemize}

\begin{note}
For example, consider a partially checksummed TCP (IPv4) packet.
It will have a 14 byte ethernet header and 20 byte IP header
followed by the TCP header (with the TCP checksum field 16 bytes
into that header). \field{csum_start} will be 14+20 = 34 (the TCP
checksum includes the header), and \field{csum_offset} will be 16.
If the given packet has the VIRTIO_NET_HDR_GSO_UDP_TUNNEL_IPV4 bit or the
VIRTIO_NET_HDR_GSO_UDP_TUNNEL_IPV6 bit set,
the above checksum fields refer to the inner header checksum, see
the example below.
\end{note}

\item If the driver negotiated
  VIRTIO_NET_F_HOST_TSO4, TSO6, USO or UFO, and the packet requires
  TCP segmentation, UDP segmentation or fragmentation, then \field{gso_type}
  is set to VIRTIO_NET_HDR_GSO_TCPV4, TCPV6, UDP_L4 or UDP.
  (Otherwise, it is set to VIRTIO_NET_HDR_GSO_NONE). In this
  case, packets larger than 1514 bytes can be transmitted: the
  metadata indicates how to replicate the packet header to cut it
  into smaller packets. The other gso fields are set:

  \begin{itemize}
  \item If the VIRTIO_NET_F_GUEST_HDRLEN feature has been negotiated,
    \field{hdr_len} indicates the header length that needs to be replicated
    for each packet. It's the number of bytes from the beginning of the packet
    to the beginning of the transport payload.
    If the \field{gso_type} has the VIRTIO_NET_HDR_GSO_UDP_TUNNEL_IPV4 bit or
    VIRTIO_NET_HDR_GSO_UDP_TUNNEL_IPV6 bit set, \field{hdr_len} accounts for
    all the headers up to and including the inner transport.
    Otherwise, if the VIRTIO_NET_F_GUEST_HDRLEN feature has not been negotiated,
    \field{hdr_len} is a hint to the device as to how much of the header
    needs to be kept to copy into each packet, usually set to the
    length of the headers, including the transport header\footnote{Due to various bugs in implementations, this field is not useful
as a guarantee of the transport header size.
}.

  \begin{note}
  Some devices benefit from knowledge of the exact header length.
  \end{note}

  \item \field{gso_size} is the maximum size of each packet beyond that
    header (ie. MSS).

  \item If the driver negotiated the VIRTIO_NET_F_HOST_ECN feature,
    the VIRTIO_NET_HDR_GSO_ECN bit in \field{gso_type}
    indicates that the TCP packet has the ECN bit set\footnote{This case is not handled by some older hardware, so is called out
specifically in the protocol.}.
   \end{itemize}

\item If the driver negotiated the VIRTIO_NET_F_HOST_UDP_TUNNEL_GSO feature and the
  \field{gso_type} has the VIRTIO_NET_HDR_GSO_UDP_TUNNEL_IPV4 bit or
  VIRTIO_NET_HDR_GSO_UDP_TUNNEL_IPV6 bit set, the GSO protocol is encapsulated
  in a UDP tunnel.
  If the outer UDP header requires checksumming, the driver must have
  additionally negotiated the VIRTIO_NET_F_HOST_UDP_TUNNEL_GSO_CSUM feature
  and offloaded the outer checksum accordingly, otherwise
  the outer UDP header must not require checksum validation, i.e. the outer
  UDP checksum must be positive zero (0x0) as defined in UDP RFC 768.
  The other tunnel-related fields indicate how to replicate the packet
  headers to cut it into smaller packets:

  \begin{itemize}
  \item \field{outer_th_offset} field indicates the outer transport header within
      the packet. This field differs from \field{csum_start} as the latter
      points to the inner transport header within the packet.

  \item \field{inner_nh_offset} field indicates the inner network header within
      the packet.
  \end{itemize}

\begin{note}
For example, consider a partially checksummed TCP (IPv4) packet carried over a
Geneve UDP tunnel (again IPv4) with no tunnel options. The
only relevant variable related to the tunnel type is the tunnel header length.
The packet will have a 14 byte outer ethernet header, 20 byte outer IP header
followed by the 8 byte UDP header (with a 0 checksum value), 8 byte Geneve header,
14 byte inner ethernet header, 20 byte inner IP header
and the TCP header (with the TCP checksum field 16 bytes
into that header). \field{csum_start} will be 14+20+8+8+14+20 = 84 (the TCP
checksum includes the header), \field{csum_offset} will be 16.
\field{inner_nh_offset} will be 14+20+8+8+14 = 62, \field{outer_th_offset} will be
14+20+8 = 42 and \field{gso_type} will be
VIRTIO_NET_HDR_GSO_TCPV4 | VIRTIO_NET_HDR_GSO_UDP_TUNNEL_IPV4 = 0x21
\end{note}

\item If the driver negotiated the VIRTIO_NET_F_HOST_UDP_TUNNEL_GSO_CSUM feature,
  the transmitted packet is a GSO one encapsulated in a UDP tunnel, and
  the outer UDP header requires checksumming, the driver can skip checksumming
  the outer header:

  \begin{itemize}
  \item \field{flags} has the VIRTIO_NET_HDR_F_UDP_TUNNEL_CSUM set,

  \item The outer UDP checksum field in the packet is set to the sum
    of the UDP pseudo header, so that replacing it by the ones'
    complement checksum of the outer UDP header and payload will give the
    correct result.
  \end{itemize}

\item \field{num_buffers} is set to zero.  This field is unused on transmitted packets.

\item The header and packet are added as one output descriptor to the
  transmitq, and the device is notified of the new entry
  (see \ref{sec:Device Types / Network Device / Device Initialization}~\nameref{sec:Device Types / Network Device / Device Initialization}).
\end{enumerate}

\drivernormative{\paragraph}{Packet Transmission}{Device Types / Network Device / Device Operation / Packet Transmission}

For the transmit packet buffer, the driver MUST use the size of the
structure \field{struct virtio_net_hdr} same as the receive packet buffer.

The driver MUST set \field{num_buffers} to zero.

If VIRTIO_NET_F_CSUM is not negotiated, the driver MUST set
\field{flags} to zero and SHOULD supply a fully checksummed
packet to the device.

If VIRTIO_NET_F_HOST_TSO4 is negotiated, the driver MAY set
\field{gso_type} to VIRTIO_NET_HDR_GSO_TCPV4 to request TCPv4
segmentation, otherwise the driver MUST NOT set
\field{gso_type} to VIRTIO_NET_HDR_GSO_TCPV4.

If VIRTIO_NET_F_HOST_TSO6 is negotiated, the driver MAY set
\field{gso_type} to VIRTIO_NET_HDR_GSO_TCPV6 to request TCPv6
segmentation, otherwise the driver MUST NOT set
\field{gso_type} to VIRTIO_NET_HDR_GSO_TCPV6.

If VIRTIO_NET_F_HOST_UFO is negotiated, the driver MAY set
\field{gso_type} to VIRTIO_NET_HDR_GSO_UDP to request UDP
fragmentation, otherwise the driver MUST NOT set
\field{gso_type} to VIRTIO_NET_HDR_GSO_UDP.

If VIRTIO_NET_F_HOST_USO is negotiated, the driver MAY set
\field{gso_type} to VIRTIO_NET_HDR_GSO_UDP_L4 to request UDP
segmentation, otherwise the driver MUST NOT set
\field{gso_type} to VIRTIO_NET_HDR_GSO_UDP_L4.

The driver SHOULD NOT send to the device TCP packets requiring segmentation offload
which have the Explicit Congestion Notification bit set, unless the
VIRTIO_NET_F_HOST_ECN feature is negotiated, in which case the
driver MUST set the VIRTIO_NET_HDR_GSO_ECN bit in
\field{gso_type}.

If VIRTIO_NET_F_HOST_UDP_TUNNEL_GSO is negotiated, the driver MAY set
VIRTIO_NET_HDR_GSO_UDP_TUNNEL_IPV4 bit or the VIRTIO_NET_HDR_GSO_UDP_TUNNEL_IPV6 bit
in \field{gso_type} according to the inner network header protocol type
to request GSO packets over UDPv4 or UDPv6 tunnel segmentation,
otherwise the driver MUST NOT set either the
VIRTIO_NET_HDR_GSO_UDP_TUNNEL_IPV4 bit or the VIRTIO_NET_HDR_GSO_UDP_TUNNEL_IPV6 bit
in \field{gso_type}.

When requesting GSO segmentation over UDP tunnel, the driver MUST SET the
VIRTIO_NET_HDR_GSO_UDP_TUNNEL_IPV4 bit if the inner network header is IPv4, i.e. the
packet is a TCPv4 GSO one, otherwise, if the inner network header is IPv6, the driver
MUST SET the VIRTIO_NET_HDR_GSO_UDP_TUNNEL_IPV6 bit.

The driver MUST NOT send to the device GSO packets over UDP tunnel
requiring segmentation and outer UDP checksum offload, unless both the
VIRTIO_NET_F_HOST_UDP_TUNNEL_GSO and VIRTIO_NET_F_HOST_UDP_TUNNEL_GSO_CSUM features
are negotiated, in which case the driver MUST set either the
VIRTIO_NET_HDR_GSO_UDP_TUNNEL_IPV4 bit or the VIRTIO_NET_HDR_GSO_UDP_TUNNEL_IPV6
bit in the \field{gso_type} and the VIRTIO_NET_HDR_F_UDP_TUNNEL_CSUM bit in
the \field{flags}.

If VIRTIO_NET_F_HOST_UDP_TUNNEL_GSO_CSUM is not negotiated, the driver MUST not set
the VIRTIO_NET_HDR_F_UDP_TUNNEL_CSUM bit in the \field{flags} and
MUST NOT send to the device GSO packets over UDP tunnel
requiring segmentation and outer UDP checksum offload.

The driver MUST NOT set the VIRTIO_NET_HDR_GSO_UDP_TUNNEL_IPV4 bit or the
VIRTIO_NET_HDR_GSO_UDP_TUNNEL_IPV6 bit together with VIRTIO_NET_HDR_GSO_UDP, as the
latter is deprecated in favor of UDP_L4 and no new feature will support it.

The driver MUST NOT set the VIRTIO_NET_HDR_GSO_UDP_TUNNEL_IPV4 bit and the
VIRTIO_NET_HDR_GSO_UDP_TUNNEL_IPV6 bit together.

The driver MUST NOT set the VIRTIO_NET_HDR_F_UDP_TUNNEL_CSUM bit \field{flags}
without setting either the VIRTIO_NET_HDR_GSO_UDP_TUNNEL_IPV4 bit or
the VIRTIO_NET_HDR_GSO_UDP_TUNNEL_IPV6 bit in \field{gso_type}.

If the VIRTIO_NET_F_CSUM feature has been negotiated, the
driver MAY set the VIRTIO_NET_HDR_F_NEEDS_CSUM bit in
\field{flags}, if so:
\begin{enumerate}
\item the driver MUST validate the packet checksum at
	offset \field{csum_offset} from \field{csum_start} as well as all
	preceding offsets;
\begin{note}
If \field{gso_type} differs from VIRTIO_NET_HDR_GSO_NONE and the
VIRTIO_NET_HDR_GSO_UDP_TUNNEL_IPV4 bit or the VIRTIO_NET_HDR_GSO_UDP_TUNNEL_IPV6
bit are not set in \field{gso_type}, \field{csum_offset}
points to the only transport header present in the packet, and there are no
additional preceding checksums validated by VIRTIO_NET_HDR_F_NEEDS_CSUM.
\end{note}
\item the driver MUST set the packet checksum stored in the
	buffer to the TCP/UDP pseudo header;
\item the driver MUST set \field{csum_start} and
	\field{csum_offset} such that calculating a ones'
	complement checksum from \field{csum_start} up until the end of
	the packet and storing the result at offset \field{csum_offset}
	from  \field{csum_start} will result in a fully checksummed
	packet;
\end{enumerate}

If none of the VIRTIO_NET_F_HOST_TSO4, TSO6, USO or UFO options have
been negotiated, the driver MUST set \field{gso_type} to
VIRTIO_NET_HDR_GSO_NONE.

If \field{gso_type} differs from VIRTIO_NET_HDR_GSO_NONE, then
the driver MUST also set the VIRTIO_NET_HDR_F_NEEDS_CSUM bit in
\field{flags} and MUST set \field{gso_size} to indicate the
desired MSS.

If one of the VIRTIO_NET_F_HOST_TSO4, TSO6, USO or UFO options have
been negotiated:
\begin{itemize}
\item If the VIRTIO_NET_F_GUEST_HDRLEN feature has been negotiated,
	and \field{gso_type} differs from VIRTIO_NET_HDR_GSO_NONE,
	the driver MUST set \field{hdr_len} to a value equal to the length
	of the headers, including the transport header. If \field{gso_type}
	has the VIRTIO_NET_HDR_GSO_UDP_TUNNEL_IPV4 bit or the
	VIRTIO_NET_HDR_GSO_UDP_TUNNEL_IPV6 bit set, \field{hdr_len} includes
	the inner transport header.

\item If the VIRTIO_NET_F_GUEST_HDRLEN feature has not been negotiated,
	or \field{gso_type} is VIRTIO_NET_HDR_GSO_NONE,
	the driver SHOULD set \field{hdr_len} to a value
	not less than the length of the headers, including the transport
	header.
\end{itemize}

If the VIRTIO_NET_F_HOST_UDP_TUNNEL_GSO option has been negotiated, the
driver MAY set the VIRTIO_NET_HDR_GSO_UDP_TUNNEL_IPV4 bit or the
VIRTIO_NET_HDR_GSO_UDP_TUNNEL_IPV6 bit in \field{gso_type}, if so:
\begin{itemize}
\item the driver MUST set \field{outer_th_offset} to the outer UDP header
  offset and \field{inner_nh_offset} to the inner network header offset.
  The \field{csum_start} and \field{csum_offset} fields point respectively
  to the inner transport header and inner transport checksum field.
\end{itemize}

If the VIRTIO_NET_F_HOST_UDP_TUNNEL_GSO_CSUM feature has been negotiated,
and the VIRTIO_NET_HDR_GSO_UDP_TUNNEL_IPV4 bit or
VIRTIO_NET_HDR_GSO_UDP_TUNNEL_IPV6 bit in \field{gso_type} are set,
the driver MAY set the VIRTIO_NET_HDR_F_UDP_TUNNEL_CSUM bit in
\field{flags}, if so the driver MUST set the packet outer UDP header checksum
to the outer UDP pseudo header checksum.

\begin{note}
calculating a ones' complement checksum from \field{outer_th_offset}
up until the end of the packet and storing the result at offset 6
from \field{outer_th_offset} will result in a fully checksummed outer UDP packet;
\end{note}

If the VIRTIO_NET_HDR_GSO_UDP_TUNNEL_IPV4 bit or the
VIRTIO_NET_HDR_GSO_UDP_TUNNEL_IPV6 bit in \field{gso_type} are set
and the VIRTIO_NET_F_HOST_UDP_TUNNEL_GSO_CSUM feature has not
been negotiated, the
outer UDP header MUST NOT require checksum validation. That is, the
outer UDP checksum value MUST be 0 or the validated complete checksum
for such header.

\begin{note}
The valid complete checksum of the outer UDP header of individual segments
can be computed by the driver prior to segmentation only if the GSO packet
size is a multiple of \field{gso_size}, because then all segments
have the same size and thus all data included in the outer UDP
checksum is the same for every segment. These pre-computed segment
length and checksum fields are different from those of the GSO
packet.
In this scenario the outer UDP header of the GSO packet must carry the
segmented UDP packet length.
\end{note}

If the VIRTIO_NET_F_HOST_UDP_TUNNEL_GSO option has not
been negotiated, the driver MUST NOT set either the VIRTIO_NET_HDR_F_GSO_UDP_TUNNEL_IPV4
bit or the VIRTIO_NET_HDR_F_GSO_UDP_TUNNEL_IPV6 in \field{gso_type}.

If the VIRTIO_NET_F_HOST_UDP_TUNNEL_GSO_CSUM option has not been negotiated,
the driver MUST NOT set the VIRTIO_NET_HDR_F_UDP_TUNNEL_CSUM bit
in \field{flags}.

The driver SHOULD accept the VIRTIO_NET_F_GUEST_HDRLEN feature if it has
been offered, and if it's able to provide the exact header length.

The driver MUST NOT set the VIRTIO_NET_HDR_F_DATA_VALID and
VIRTIO_NET_HDR_F_RSC_INFO bits in \field{flags}.

The driver MUST NOT set the VIRTIO_NET_HDR_F_DATA_VALID bit in \field{flags}
together with the VIRTIO_NET_HDR_F_GSO_UDP_TUNNEL_IPV4 bit or the
VIRTIO_NET_HDR_F_GSO_UDP_TUNNEL_IPV6 bit in \field{gso_type}.

\devicenormative{\paragraph}{Packet Transmission}{Device Types / Network Device / Device Operation / Packet Transmission}
The device MUST ignore \field{flag} bits that it does not recognize.

If VIRTIO_NET_HDR_F_NEEDS_CSUM bit in \field{flags} is not set, the
device MUST NOT use the \field{csum_start} and \field{csum_offset}.

If one of the VIRTIO_NET_F_HOST_TSO4, TSO6, USO or UFO options have
been negotiated:
\begin{itemize}
\item If the VIRTIO_NET_F_GUEST_HDRLEN feature has been negotiated,
	and \field{gso_type} differs from VIRTIO_NET_HDR_GSO_NONE,
	the device MAY use \field{hdr_len} as the transport header size.

	\begin{note}
	Caution should be taken by the implementation so as to prevent
	a malicious driver from attacking the device by setting an incorrect hdr_len.
	\end{note}

\item If the VIRTIO_NET_F_GUEST_HDRLEN feature has not been negotiated,
	or \field{gso_type} is VIRTIO_NET_HDR_GSO_NONE,
	the device MAY use \field{hdr_len} only as a hint about the
	transport header size.
	The device MUST NOT rely on \field{hdr_len} to be correct.

	\begin{note}
	This is due to various bugs in implementations.
	\end{note}
\end{itemize}

If both the VIRTIO_NET_HDR_GSO_UDP_TUNNEL_IPV4 bit and
the VIRTIO_NET_HDR_GSO_UDP_TUNNEL_IPV6 bit in in \field{gso_type} are set,
the device MUST NOT accept the packet.

If the VIRTIO_NET_HDR_GSO_UDP_TUNNEL_IPV4 bit and the VIRTIO_NET_HDR_GSO_UDP_TUNNEL_IPV6
bit in \field{gso_type} are not set, the device MUST NOT use the
\field{outer_th_offset} and \field{inner_nh_offset}.

If either the VIRTIO_NET_HDR_GSO_UDP_TUNNEL_IPV4 bit or
the VIRTIO_NET_HDR_GSO_UDP_TUNNEL_IPV6 bit in \field{gso_type} are set, and any of
the following is true:
\begin{itemize}
\item the VIRTIO_NET_HDR_F_NEEDS_CSUM is not set in \field{flags}
\item the VIRTIO_NET_HDR_F_DATA_VALID is set in \field{flags}
\item the \field{gso_type} excluding the VIRTIO_NET_HDR_GSO_UDP_TUNNEL_IPV4
bit and the VIRTIO_NET_HDR_GSO_UDP_TUNNEL_IPV6 bit is VIRTIO_NET_HDR_GSO_NONE
\end{itemize}
the device MUST NOT accept the packet.

If the VIRTIO_NET_HDR_F_UDP_TUNNEL_CSUM bit in \field{flags} is set,
and both the bits VIRTIO_NET_HDR_GSO_UDP_TUNNEL_IPV4 and
VIRTIO_NET_HDR_GSO_UDP_TUNNEL_IPV6 in \field{gso_type} are not set,
the device MOST NOT accept the packet.

If VIRTIO_NET_HDR_F_NEEDS_CSUM is not set, the device MUST NOT
rely on the packet checksum being correct.
\paragraph{Packet Transmission Interrupt}\label{sec:Device Types / Network Device / Device Operation / Packet Transmission / Packet Transmission Interrupt}

Often a driver will suppress transmission virtqueue interrupts
and check for used packets in the transmit path of following
packets.

The normal behavior in this interrupt handler is to retrieve
used buffers from the virtqueue and free the corresponding
headers and packets.

\subsubsection{Setting Up Receive Buffers}\label{sec:Device Types / Network Device / Device Operation / Setting Up Receive Buffers}

It is generally a good idea to keep the receive virtqueue as
fully populated as possible: if it runs out, network performance
will suffer.

If the VIRTIO_NET_F_GUEST_TSO4, VIRTIO_NET_F_GUEST_TSO6,
VIRTIO_NET_F_GUEST_UFO, VIRTIO_NET_F_GUEST_USO4 or VIRTIO_NET_F_GUEST_USO6
features are used, the maximum incoming packet
will be 65589 bytes long (14 bytes of Ethernet header, plus 40 bytes of
the IPv6 header, plus 65535 bytes of maximum IPv6 payload including any
extension header), otherwise 1514 bytes.
When VIRTIO_NET_F_HASH_REPORT is not negotiated, the required receive buffer
size is either 65601 or 1526 bytes accounting for 20 bytes of
\field{struct virtio_net_hdr} followed by receive packet.
When VIRTIO_NET_F_HASH_REPORT is negotiated, the required receive buffer
size is either 65609 or 1534 bytes accounting for 12 bytes of
\field{struct virtio_net_hdr} followed by receive packet.

\drivernormative{\paragraph}{Setting Up Receive Buffers}{Device Types / Network Device / Device Operation / Setting Up Receive Buffers}

\begin{itemize}
\item If VIRTIO_NET_F_MRG_RXBUF is not negotiated:
  \begin{itemize}
    \item If VIRTIO_NET_F_GUEST_TSO4, VIRTIO_NET_F_GUEST_TSO6, VIRTIO_NET_F_GUEST_UFO,
	VIRTIO_NET_F_GUEST_USO4 or VIRTIO_NET_F_GUEST_USO6 are negotiated, the driver SHOULD populate
      the receive queue(s) with buffers of at least 65609 bytes if
      VIRTIO_NET_F_HASH_REPORT is negotiated, and of at least 65601 bytes if not.
    \item Otherwise, the driver SHOULD populate the receive queue(s)
      with buffers of at least 1534 bytes if VIRTIO_NET_F_HASH_REPORT
      is negotiated, and of at least 1526 bytes if not.
  \end{itemize}
\item If VIRTIO_NET_F_MRG_RXBUF is negotiated, each buffer MUST be at
least size of \field{struct virtio_net_hdr},
i.e. 20 bytes if VIRTIO_NET_F_HASH_REPORT is negotiated, and 12 bytes if not.
\end{itemize}

\begin{note}
Obviously each buffer can be split across multiple descriptor elements.
\end{note}

When calculating the size of \field{struct virtio_net_hdr}, the driver
MUST consider all the fields inclusive up to \field{padding_reserved},
i.e. 20 bytes if VIRTIO_NET_F_HASH_REPORT is negotiated, and 12 bytes if not.

If VIRTIO_NET_F_MQ is negotiated, each of receiveq1\ldots receiveqN
that will be used SHOULD be populated with receive buffers.

\devicenormative{\paragraph}{Setting Up Receive Buffers}{Device Types / Network Device / Device Operation / Setting Up Receive Buffers}

The device MUST set \field{num_buffers} to the number of descriptors used to
hold the incoming packet.

The device MUST use only a single descriptor if VIRTIO_NET_F_MRG_RXBUF
was not negotiated.
\begin{note}
{This means that \field{num_buffers} will always be 1
if VIRTIO_NET_F_MRG_RXBUF is not negotiated.}
\end{note}

\subsubsection{Processing of Incoming Packets}\label{sec:Device Types / Network Device / Device Operation / Processing of Incoming Packets}
\label{sec:Device Types / Network Device / Device Operation / Processing of Packets}%old label for latexdiff

When a packet is copied into a buffer in the receiveq, the
optimal path is to disable further used buffer notifications for the
receiveq and process packets until no more are found, then re-enable
them.

Processing incoming packets involves:

\begin{enumerate}
\item \field{num_buffers} indicates how many descriptors
  this packet is spread over (including this one): this will
  always be 1 if VIRTIO_NET_F_MRG_RXBUF was not negotiated.
  This allows receipt of large packets without having to allocate large
  buffers: a packet that does not fit in a single buffer can flow
  over to the next buffer, and so on. In this case, there will be
  at least \field{num_buffers} used buffers in the virtqueue, and the device
  chains them together to form a single packet in a way similar to
  how it would store it in a single buffer spread over multiple
  descriptors.
  The other buffers will not begin with a \field{struct virtio_net_hdr}.

\item If
  \field{num_buffers} is one, then the entire packet will be
  contained within this buffer, immediately following the struct
  virtio_net_hdr.
\item If the VIRTIO_NET_F_GUEST_CSUM feature was negotiated, the
  VIRTIO_NET_HDR_F_DATA_VALID bit in \field{flags} can be
  set: if so, device has validated the packet checksum.
  If the VIRTIO_NET_F_GUEST_UDP_TUNNEL_GSO_CSUM feature has been negotiated,
  and the VIRTIO_NET_HDR_F_UDP_TUNNEL_CSUM bit is set in \field{flags},
  both the outer UDP checksum and the inner transport checksum
  have been validated, otherwise only one level of checksums (the outer one
  in case of tunnels) has been validated.
\end{enumerate}

Additionally, VIRTIO_NET_F_GUEST_CSUM, TSO4, TSO6, UDP, UDP_TUNNEL
and ECN features enable receive checksum, large receive offload and ECN
support which are the input equivalents of the transmit checksum,
transmit segmentation offloading and ECN features, as described
in \ref{sec:Device Types / Network Device / Device Operation /
Packet Transmission}:
\begin{enumerate}
\item If the VIRTIO_NET_F_GUEST_TSO4, TSO6, UFO, USO4 or USO6 options were
  negotiated, then \field{gso_type} MAY be something other than
  VIRTIO_NET_HDR_GSO_NONE, and \field{gso_size} field indicates the
  desired MSS (see Packet Transmission point 2).
\item If the VIRTIO_NET_F_RSC_EXT option was negotiated (this
  implies one of VIRTIO_NET_F_GUEST_TSO4, TSO6), the
  device processes also duplicated ACK segments, reports
  number of coalesced TCP segments in \field{csum_start} field and
  number of duplicated ACK segments in \field{csum_offset} field
  and sets bit VIRTIO_NET_HDR_F_RSC_INFO in \field{flags}.
\item If the VIRTIO_NET_F_GUEST_CSUM feature was negotiated, the
  VIRTIO_NET_HDR_F_NEEDS_CSUM bit in \field{flags} can be
  set: if so, the packet checksum at offset \field{csum_offset}
  from \field{csum_start} and any preceding checksums
  have been validated.  The checksum on the packet is incomplete and
  if bit VIRTIO_NET_HDR_F_RSC_INFO is not set in \field{flags},
  then \field{csum_start} and \field{csum_offset} indicate how to calculate it
  (see Packet Transmission point 1).
\begin{note}
If \field{gso_type} differs from VIRTIO_NET_HDR_GSO_NONE and the
VIRTIO_NET_HDR_GSO_UDP_TUNNEL_IPV4 bit or the VIRTIO_NET_HDR_GSO_UDP_TUNNEL_IPV6
bit are not set, \field{csum_offset}
points to the only transport header present in the packet, and there are no
additional preceding checksums validated by VIRTIO_NET_HDR_F_NEEDS_CSUM.
\end{note}
\item If the VIRTIO_NET_F_GUEST_UDP_TUNNEL_GSO option was negotiated and
  \field{gso_type} is not VIRTIO_NET_HDR_GSO_NONE, the
  VIRTIO_NET_HDR_GSO_UDP_TUNNEL_IPV4 bit or the VIRTIO_NET_HDR_GSO_UDP_TUNNEL_IPV6
  bit MAY be set. In such case the \field{outer_th_offset} and
  \field{inner_nh_offset} fields indicate the corresponding
  headers information.
\item If the VIRTIO_NET_F_GUEST_UDP_TUNNEL_GSO_CSUM feature was
negotiated, and
  the VIRTIO_NET_HDR_GSO_UDP_TUNNEL_IPV4 bit or the VIRTIO_NET_HDR_GSO_UDP_TUNNEL_IPV6
  are set in \field{gso_type}, the VIRTIO_NET_HDR_F_UDP_TUNNEL_CSUM bit in the
  \field{flags} can be set: if so, the outer UDP checksum has been validated
  and the UDP header checksum at offset 6 from from \field{outer_th_offset}
  is set to the outer UDP pseudo header checksum.

\begin{note}
If the VIRTIO_NET_HDR_GSO_UDP_TUNNEL_IPV4 bit or VIRTIO_NET_HDR_GSO_UDP_TUNNEL_IPV6
bit are set in \field{gso_type}, the \field{csum_start} field refers to
the inner transport header offset (see Packet Transmission point 1).
If the VIRTIO_NET_HDR_F_UDP_TUNNEL_CSUM bit in \field{flags} is set both
the inner and the outer header checksums have been validated by
VIRTIO_NET_HDR_F_NEEDS_CSUM, otherwise only the inner transport header
checksum has been validated.
\end{note}
\end{enumerate}

If applicable, the device calculates per-packet hash for incoming packets as
defined in \ref{sec:Device Types / Network Device / Device Operation / Processing of Incoming Packets / Hash calculation for incoming packets}.

If applicable, the device reports hash information for incoming packets as
defined in \ref{sec:Device Types / Network Device / Device Operation / Processing of Incoming Packets / Hash reporting for incoming packets}.

\devicenormative{\paragraph}{Processing of Incoming Packets}{Device Types / Network Device / Device Operation / Processing of Incoming Packets}
\label{devicenormative:Device Types / Network Device / Device Operation / Processing of Packets}%old label for latexdiff

If VIRTIO_NET_F_MRG_RXBUF has not been negotiated, the device MUST set
\field{num_buffers} to 1.

If VIRTIO_NET_F_MRG_RXBUF has been negotiated, the device MUST set
\field{num_buffers} to indicate the number of buffers
the packet (including the header) is spread over.

If a receive packet is spread over multiple buffers, the device
MUST use all buffers but the last (i.e. the first \field{num_buffers} -
1 buffers) completely up to the full length of each buffer
supplied by the driver.

The device MUST use all buffers used by a single receive
packet together, such that at least \field{num_buffers} are
observed by driver as used.

If VIRTIO_NET_F_GUEST_CSUM is not negotiated, the device MUST set
\field{flags} to zero and SHOULD supply a fully checksummed
packet to the driver.

If VIRTIO_NET_F_GUEST_TSO4 is not negotiated, the device MUST NOT set
\field{gso_type} to VIRTIO_NET_HDR_GSO_TCPV4.

If VIRTIO_NET_F_GUEST_UDP is not negotiated, the device MUST NOT set
\field{gso_type} to VIRTIO_NET_HDR_GSO_UDP.

If VIRTIO_NET_F_GUEST_TSO6 is not negotiated, the device MUST NOT set
\field{gso_type} to VIRTIO_NET_HDR_GSO_TCPV6.

If none of VIRTIO_NET_F_GUEST_USO4 or VIRTIO_NET_F_GUEST_USO6 have been negotiated,
the device MUST NOT set \field{gso_type} to VIRTIO_NET_HDR_GSO_UDP_L4.

If VIRTIO_NET_F_GUEST_UDP_TUNNEL_GSO is not negotiated, the device MUST NOT set
either the VIRTIO_NET_HDR_GSO_UDP_TUNNEL_IPV4 bit or the
VIRTIO_NET_HDR_GSO_UDP_TUNNEL_IPV6 bit in \field{gso_type}.

If VIRTIO_NET_F_GUEST_UDP_TUNNEL_GSO_CSUM is not negotiated the device MUST NOT set
the VIRTIO_NET_HDR_F_UDP_TUNNEL_CSUM bit in \field{flags}.

The device SHOULD NOT send to the driver TCP packets requiring segmentation offload
which have the Explicit Congestion Notification bit set, unless the
VIRTIO_NET_F_GUEST_ECN feature is negotiated, in which case the
device MUST set the VIRTIO_NET_HDR_GSO_ECN bit in
\field{gso_type}.

If the VIRTIO_NET_F_GUEST_CSUM feature has been negotiated, the
device MAY set the VIRTIO_NET_HDR_F_NEEDS_CSUM bit in
\field{flags}, if so:
\begin{enumerate}
\item the device MUST validate the packet checksum at
	offset \field{csum_offset} from \field{csum_start} as well as all
	preceding offsets;
\item the device MUST set the packet checksum stored in the
	receive buffer to the TCP/UDP pseudo header;
\item the device MUST set \field{csum_start} and
	\field{csum_offset} such that calculating a ones'
	complement checksum from \field{csum_start} up until the
	end of the packet and storing the result at offset
	\field{csum_offset} from  \field{csum_start} will result in a
	fully checksummed packet;
\end{enumerate}

The device MUST NOT send to the driver GSO packets encapsulated in UDP
tunnel and requiring segmentation offload, unless the
VIRTIO_NET_F_GUEST_UDP_TUNNEL_GSO is negotiated, in which case the device MUST set
the VIRTIO_NET_HDR_GSO_UDP_TUNNEL_IPV4 bit or the VIRTIO_NET_HDR_GSO_UDP_TUNNEL_IPV6
bit in \field{gso_type} according to the inner network header protocol type,
MUST set the \field{outer_th_offset} and \field{inner_nh_offset} fields
to the corresponding header information, and the outer UDP header MUST NOT
require checksum offload.

If the VIRTIO_NET_F_GUEST_UDP_TUNNEL_GSO_CSUM feature has not been negotiated,
the device MUST NOT send the driver GSO packets encapsulated in UDP
tunnel and requiring segmentation and outer checksum offload.

If none of the VIRTIO_NET_F_GUEST_TSO4, TSO6, UFO, USO4 or USO6 options have
been negotiated, the device MUST set \field{gso_type} to
VIRTIO_NET_HDR_GSO_NONE.

If \field{gso_type} differs from VIRTIO_NET_HDR_GSO_NONE, then
the device MUST also set the VIRTIO_NET_HDR_F_NEEDS_CSUM bit in
\field{flags} MUST set \field{gso_size} to indicate the desired MSS.
If VIRTIO_NET_F_RSC_EXT was negotiated, the device MUST also
set VIRTIO_NET_HDR_F_RSC_INFO bit in \field{flags},
set \field{csum_start} to number of coalesced TCP segments and
set \field{csum_offset} to number of received duplicated ACK segments.

If VIRTIO_NET_F_RSC_EXT was not negotiated, the device MUST
not set VIRTIO_NET_HDR_F_RSC_INFO bit in \field{flags}.

If one of the VIRTIO_NET_F_GUEST_TSO4, TSO6, UFO, USO4 or USO6 options have
been negotiated, the device SHOULD set \field{hdr_len} to a value
not less than the length of the headers, including the transport
header. If \field{gso_type} has the VIRTIO_NET_HDR_GSO_UDP_TUNNEL_IPV4 bit
or the VIRTIO_NET_HDR_GSO_UDP_TUNNEL_IPV6 bit set, the referenced transport
header is the inner one.

If the VIRTIO_NET_F_GUEST_CSUM feature has been negotiated, the
device MAY set the VIRTIO_NET_HDR_F_DATA_VALID bit in
\field{flags}, if so, the device MUST validate the packet
checksum. If the VIRTIO_NET_F_GUEST_UDP_TUNNEL_GSO_CSUM feature has
been negotiated, and the VIRTIO_NET_HDR_F_UDP_TUNNEL_CSUM bit set in
\field{flags}, both the outer UDP checksum and the inner transport
checksum have been validated.
Otherwise level of checksum is validated: in case of multiple
encapsulated protocols the outermost one.

If either the VIRTIO_NET_HDR_GSO_UDP_TUNNEL_IPV4 bit or the
VIRTIO_NET_HDR_GSO_UDP_TUNNEL_IPV6 bit in \field{gso_type} are set,
the device MUST NOT set the VIRTIO_NET_HDR_F_DATA_VALID bit in
\field{flags}.

If the VIRTIO_NET_F_GUEST_UDP_TUNNEL_GSO_CSUM feature has been negotiated
and either the VIRTIO_NET_HDR_GSO_UDP_TUNNEL_IPV4 bit is set or the
VIRTIO_NET_HDR_GSO_UDP_TUNNEL_IPV6 bit is set in \field{gso_type}, the
device MAY set the VIRTIO_NET_HDR_F_UDP_TUNNEL_CSUM bit in
\field{flags}, if so the device MUST set the packet outer UDP checksum
stored in the receive buffer to the outer UDP pseudo header.

Otherwise, the VIRTIO_NET_F_GUEST_UDP_TUNNEL_GSO_CSUM feature has been
negotiated, either the VIRTIO_NET_HDR_GSO_UDP_TUNNEL_IPV4 bit is set or the
VIRTIO_NET_HDR_GSO_UDP_TUNNEL_IPV6 bit is set in \field{gso_type},
and the bit VIRTIO_NET_HDR_F_UDP_TUNNEL_CSUM is not set in
\field{flags}, the device MUST either provide a zero outer UDP header
checksum or a fully checksummed outer UDP header.

\drivernormative{\paragraph}{Processing of Incoming
Packets}{Device Types / Network Device / Device Operation /
Processing of Incoming Packets}

The driver MUST ignore \field{flag} bits that it does not recognize.

If VIRTIO_NET_HDR_F_NEEDS_CSUM bit in \field{flags} is not set or
if VIRTIO_NET_HDR_F_RSC_INFO bit \field{flags} is set, the
driver MUST NOT use the \field{csum_start} and \field{csum_offset}.

If one of the VIRTIO_NET_F_GUEST_TSO4, TSO6, UFO, USO4 or USO6 options have
been negotiated, the driver MAY use \field{hdr_len} only as a hint about the
transport header size.
The driver MUST NOT rely on \field{hdr_len} to be correct.
\begin{note}
This is due to various bugs in implementations.
\end{note}

If neither VIRTIO_NET_HDR_F_NEEDS_CSUM nor
VIRTIO_NET_HDR_F_DATA_VALID is set, the driver MUST NOT
rely on the packet checksum being correct.

If both the VIRTIO_NET_HDR_GSO_UDP_TUNNEL_IPV4 bit and
the VIRTIO_NET_HDR_GSO_UDP_TUNNEL_IPV6 bit in in \field{gso_type} are set,
the driver MUST NOT accept the packet.

If the VIRTIO_NET_HDR_GSO_UDP_TUNNEL_IPV4 bit or the VIRTIO_NET_HDR_GSO_UDP_TUNNEL_IPV6
bit in \field{gso_type} are not set, the driver MUST NOT use the
\field{outer_th_offset} and \field{inner_nh_offset}.

If either the VIRTIO_NET_HDR_GSO_UDP_TUNNEL_IPV4 bit or
the VIRTIO_NET_HDR_GSO_UDP_TUNNEL_IPV6 bit in \field{gso_type} are set, and any of
the following is true:
\begin{itemize}
\item the VIRTIO_NET_HDR_F_NEEDS_CSUM bit is not set in \field{flags}
\item the VIRTIO_NET_HDR_F_DATA_VALID bit is set in \field{flags}
\item the \field{gso_type} excluding the VIRTIO_NET_HDR_GSO_UDP_TUNNEL_IPV4
bit and the VIRTIO_NET_HDR_GSO_UDP_TUNNEL_IPV6 bit is VIRTIO_NET_HDR_GSO_NONE
\end{itemize}
the driver MUST NOT accept the packet.

If the VIRTIO_NET_HDR_F_UDP_TUNNEL_CSUM bit and the VIRTIO_NET_HDR_F_NEEDS_CSUM
bit in \field{flags} are set,
and both the bits VIRTIO_NET_HDR_GSO_UDP_TUNNEL_IPV4 and
VIRTIO_NET_HDR_GSO_UDP_TUNNEL_IPV6 in \field{gso_type} are not set,
the driver MOST NOT accept the packet.

\paragraph{Hash calculation for incoming packets}
\label{sec:Device Types / Network Device / Device Operation / Processing of Incoming Packets / Hash calculation for incoming packets}

A device attempts to calculate a per-packet hash in the following cases:
\begin{itemize}
\item The feature VIRTIO_NET_F_RSS was negotiated. The device uses the hash to determine the receive virtqueue to place incoming packets.
\item The feature VIRTIO_NET_F_HASH_REPORT was negotiated. The device reports the hash value and the hash type with the packet.
\end{itemize}

If the feature VIRTIO_NET_F_RSS was negotiated:
\begin{itemize}
\item The device uses \field{hash_types} of the virtio_net_rss_config structure as 'Enabled hash types' bitmask.
\item If additionally the feature VIRTIO_NET_F_HASH_TUNNEL was negotiated, the device uses \field{enabled_tunnel_types} of the
      virtnet_hash_tunnel structure as 'Encapsulation types enabled for inner header hash' bitmask.
\item The device uses a key as defined in \field{hash_key_data} and \field{hash_key_length} of the virtio_net_rss_config structure (see
\ref{sec:Device Types / Network Device / Device Operation / Control Virtqueue / Receive-side scaling (RSS) / Setting RSS parameters}).
\end{itemize}

If the feature VIRTIO_NET_F_RSS was not negotiated:
\begin{itemize}
\item The device uses \field{hash_types} of the virtio_net_hash_config structure as 'Enabled hash types' bitmask.
\item If additionally the feature VIRTIO_NET_F_HASH_TUNNEL was negotiated, the device uses \field{enabled_tunnel_types} of the
      virtnet_hash_tunnel structure as 'Encapsulation types enabled for inner header hash' bitmask.
\item The device uses a key as defined in \field{hash_key_data} and \field{hash_key_length} of the virtio_net_hash_config structure (see
\ref{sec:Device Types / Network Device / Device Operation / Control Virtqueue / Automatic receive steering in multiqueue mode / Hash calculation}).
\end{itemize}

Note that if the device offers VIRTIO_NET_F_HASH_REPORT, even if it supports only one pair of virtqueues, it MUST support
at least one of commands of VIRTIO_NET_CTRL_MQ class to configure reported hash parameters:
\begin{itemize}
\item If the device offers VIRTIO_NET_F_RSS, it MUST support VIRTIO_NET_CTRL_MQ_RSS_CONFIG command per
 \ref{sec:Device Types / Network Device / Device Operation / Control Virtqueue / Receive-side scaling (RSS) / Setting RSS parameters}.
\item Otherwise the device MUST support VIRTIO_NET_CTRL_MQ_HASH_CONFIG command per
 \ref{sec:Device Types / Network Device / Device Operation / Control Virtqueue / Automatic receive steering in multiqueue mode / Hash calculation}.
\end{itemize}

The per-packet hash calculation can depend on the IP packet type. See
\hyperref[intro:IP]{[IP]}, \hyperref[intro:UDP]{[UDP]} and \hyperref[intro:TCP]{[TCP]}.

\subparagraph{Supported/enabled hash types}
\label{sec:Device Types / Network Device / Device Operation / Processing of Incoming Packets / Hash calculation for incoming packets / Supported/enabled hash types}
Hash types applicable for IPv4 packets:
\begin{lstlisting}
#define VIRTIO_NET_HASH_TYPE_IPv4              (1 << 0)
#define VIRTIO_NET_HASH_TYPE_TCPv4             (1 << 1)
#define VIRTIO_NET_HASH_TYPE_UDPv4             (1 << 2)
\end{lstlisting}
Hash types applicable for IPv6 packets without extension headers
\begin{lstlisting}
#define VIRTIO_NET_HASH_TYPE_IPv6              (1 << 3)
#define VIRTIO_NET_HASH_TYPE_TCPv6             (1 << 4)
#define VIRTIO_NET_HASH_TYPE_UDPv6             (1 << 5)
\end{lstlisting}
Hash types applicable for IPv6 packets with extension headers
\begin{lstlisting}
#define VIRTIO_NET_HASH_TYPE_IP_EX             (1 << 6)
#define VIRTIO_NET_HASH_TYPE_TCP_EX            (1 << 7)
#define VIRTIO_NET_HASH_TYPE_UDP_EX            (1 << 8)
\end{lstlisting}

\subparagraph{IPv4 packets}
\label{sec:Device Types / Network Device / Device Operation / Processing of Incoming Packets / Hash calculation for incoming packets / IPv4 packets}
The device calculates the hash on IPv4 packets according to 'Enabled hash types' bitmask as follows:
\begin{itemize}
\item If VIRTIO_NET_HASH_TYPE_TCPv4 is set and the packet has
a TCP header, the hash is calculated over the following fields:
\begin{itemize}
\item Source IP address
\item Destination IP address
\item Source TCP port
\item Destination TCP port
\end{itemize}
\item Else if VIRTIO_NET_HASH_TYPE_UDPv4 is set and the
packet has a UDP header, the hash is calculated over the following fields:
\begin{itemize}
\item Source IP address
\item Destination IP address
\item Source UDP port
\item Destination UDP port
\end{itemize}
\item Else if VIRTIO_NET_HASH_TYPE_IPv4 is set, the hash is
calculated over the following fields:
\begin{itemize}
\item Source IP address
\item Destination IP address
\end{itemize}
\item Else the device does not calculate the hash
\end{itemize}

\subparagraph{IPv6 packets without extension header}
\label{sec:Device Types / Network Device / Device Operation / Processing of Incoming Packets / Hash calculation for incoming packets / IPv6 packets without extension header}
The device calculates the hash on IPv6 packets without extension
headers according to 'Enabled hash types' bitmask as follows:
\begin{itemize}
\item If VIRTIO_NET_HASH_TYPE_TCPv6 is set and the packet has
a TCPv6 header, the hash is calculated over the following fields:
\begin{itemize}
\item Source IPv6 address
\item Destination IPv6 address
\item Source TCP port
\item Destination TCP port
\end{itemize}
\item Else if VIRTIO_NET_HASH_TYPE_UDPv6 is set and the
packet has a UDPv6 header, the hash is calculated over the following fields:
\begin{itemize}
\item Source IPv6 address
\item Destination IPv6 address
\item Source UDP port
\item Destination UDP port
\end{itemize}
\item Else if VIRTIO_NET_HASH_TYPE_IPv6 is set, the hash is
calculated over the following fields:
\begin{itemize}
\item Source IPv6 address
\item Destination IPv6 address
\end{itemize}
\item Else the device does not calculate the hash
\end{itemize}

\subparagraph{IPv6 packets with extension header}
\label{sec:Device Types / Network Device / Device Operation / Processing of Incoming Packets / Hash calculation for incoming packets / IPv6 packets with extension header}
The device calculates the hash on IPv6 packets with extension
headers according to 'Enabled hash types' bitmask as follows:
\begin{itemize}
\item If VIRTIO_NET_HASH_TYPE_TCP_EX is set and the packet
has a TCPv6 header, the hash is calculated over the following fields:
\begin{itemize}
\item Home address from the home address option in the IPv6 destination options header. If the extension header is not present, use the Source IPv6 address.
\item IPv6 address that is contained in the Routing-Header-Type-2 from the associated extension header. If the extension header is not present, use the Destination IPv6 address.
\item Source TCP port
\item Destination TCP port
\end{itemize}
\item Else if VIRTIO_NET_HASH_TYPE_UDP_EX is set and the
packet has a UDPv6 header, the hash is calculated over the following fields:
\begin{itemize}
\item Home address from the home address option in the IPv6 destination options header. If the extension header is not present, use the Source IPv6 address.
\item IPv6 address that is contained in the Routing-Header-Type-2 from the associated extension header. If the extension header is not present, use the Destination IPv6 address.
\item Source UDP port
\item Destination UDP port
\end{itemize}
\item Else if VIRTIO_NET_HASH_TYPE_IP_EX is set, the hash is
calculated over the following fields:
\begin{itemize}
\item Home address from the home address option in the IPv6 destination options header. If the extension header is not present, use the Source IPv6 address.
\item IPv6 address that is contained in the Routing-Header-Type-2 from the associated extension header. If the extension header is not present, use the Destination IPv6 address.
\end{itemize}
\item Else skip IPv6 extension headers and calculate the hash as
defined for an IPv6 packet without extension headers
(see \ref{sec:Device Types / Network Device / Device Operation / Processing of Incoming Packets / Hash calculation for incoming packets / IPv6 packets without extension header}).
\end{itemize}

\paragraph{Inner Header Hash}
\label{sec:Device Types / Network Device / Device Operation / Processing of Incoming Packets / Inner Header Hash}

If VIRTIO_NET_F_HASH_TUNNEL has been negotiated, the driver can send the command
VIRTIO_NET_CTRL_HASH_TUNNEL_SET to configure the calculation of the inner header hash.

\begin{lstlisting}
struct virtnet_hash_tunnel {
    le32 enabled_tunnel_types;
};

#define VIRTIO_NET_CTRL_HASH_TUNNEL 7
 #define VIRTIO_NET_CTRL_HASH_TUNNEL_SET 0
\end{lstlisting}

Field \field{enabled_tunnel_types} contains the bitmask of encapsulation types enabled for inner header hash.
See \ref{sec:Device Types / Network Device / Device Operation / Processing of Incoming Packets /
Hash calculation for incoming packets / Encapsulation types supported/enabled for inner header hash}.

The class VIRTIO_NET_CTRL_HASH_TUNNEL has one command:
VIRTIO_NET_CTRL_HASH_TUNNEL_SET sets \field{enabled_tunnel_types} for the device using the
virtnet_hash_tunnel structure, which is read-only for the device.

Inner header hash is disabled by VIRTIO_NET_CTRL_HASH_TUNNEL_SET with \field{enabled_tunnel_types} set to 0.

Initially (before the driver sends any VIRTIO_NET_CTRL_HASH_TUNNEL_SET command) all
encapsulation types are disabled for inner header hash.

\subparagraph{Encapsulated packet}
\label{sec:Device Types / Network Device / Device Operation / Processing of Incoming Packets / Hash calculation for incoming packets / Encapsulated packet}

Multiple tunneling protocols allow encapsulating an inner, payload packet in an outer, encapsulated packet.
The encapsulated packet thus contains an outer header and an inner header, and the device calculates the
hash over either the inner header or the outer header.

If VIRTIO_NET_F_HASH_TUNNEL is negotiated and a received encapsulated packet's outer header matches one of the
encapsulation types enabled in \field{enabled_tunnel_types}, then the device uses the inner header for hash
calculations (only a single level of encapsulation is currently supported).

If VIRTIO_NET_F_HASH_TUNNEL is negotiated and a received packet's (outer) header does not match any encapsulation
types enabled in \field{enabled_tunnel_types}, then the device uses the outer header for hash calculations.

\subparagraph{Encapsulation types supported/enabled for inner header hash}
\label{sec:Device Types / Network Device / Device Operation / Processing of Incoming Packets /
Hash calculation for incoming packets / Encapsulation types supported/enabled for inner header hash}

Encapsulation types applicable for inner header hash:
\begin{lstlisting}[escapechar=|]
#define VIRTIO_NET_HASH_TUNNEL_TYPE_GRE_2784    (1 << 0) /* |\hyperref[intro:rfc2784]{[RFC2784]}| */
#define VIRTIO_NET_HASH_TUNNEL_TYPE_GRE_2890    (1 << 1) /* |\hyperref[intro:rfc2890]{[RFC2890]}| */
#define VIRTIO_NET_HASH_TUNNEL_TYPE_GRE_7676    (1 << 2) /* |\hyperref[intro:rfc7676]{[RFC7676]}| */
#define VIRTIO_NET_HASH_TUNNEL_TYPE_GRE_UDP     (1 << 3) /* |\hyperref[intro:rfc8086]{[GRE-in-UDP]}| */
#define VIRTIO_NET_HASH_TUNNEL_TYPE_VXLAN       (1 << 4) /* |\hyperref[intro:vxlan]{[VXLAN]}| */
#define VIRTIO_NET_HASH_TUNNEL_TYPE_VXLAN_GPE   (1 << 5) /* |\hyperref[intro:vxlan-gpe]{[VXLAN-GPE]}| */
#define VIRTIO_NET_HASH_TUNNEL_TYPE_GENEVE      (1 << 6) /* |\hyperref[intro:geneve]{[GENEVE]}| */
#define VIRTIO_NET_HASH_TUNNEL_TYPE_IPIP        (1 << 7) /* |\hyperref[intro:ipip]{[IPIP]}| */
#define VIRTIO_NET_HASH_TUNNEL_TYPE_NVGRE       (1 << 8) /* |\hyperref[intro:nvgre]{[NVGRE]}| */
\end{lstlisting}

\subparagraph{Advice}
Example uses of the inner header hash:
\begin{itemize}
\item Legacy tunneling protocols, lacking the outer header entropy, can use RSS with the inner header hash to
      distribute flows with identical outer but different inner headers across various queues, improving performance.
\item Identify an inner flow distributed across multiple outer tunnels.
\end{itemize}

As using the inner header hash completely discards the outer header entropy, care must be taken
if the inner header is controlled by an adversary, as the adversary can then intentionally create
configurations with insufficient entropy.

Besides disabling the inner header hash, mitigations would depend on how the hash is used. When the hash
use is limited to the RSS queue selection, the inner header hash may have quality of service (QoS) limitations.

\devicenormative{\subparagraph}{Inner Header Hash}{Device Types / Network Device / Device Operation / Control Virtqueue / Inner Header Hash}

If the (outer) header of the received packet does not match any encapsulation types enabled
in \field{enabled_tunnel_types}, the device MUST calculate the hash on the outer header.

If the device receives any bits in \field{enabled_tunnel_types} which are not set in \field{supported_tunnel_types},
it SHOULD respond to the VIRTIO_NET_CTRL_HASH_TUNNEL_SET command with VIRTIO_NET_ERR.

If the driver sets \field{enabled_tunnel_types} to 0 through VIRTIO_NET_CTRL_HASH_TUNNEL_SET or upon the device reset,
the device MUST disable the inner header hash for all encapsulation types.

\drivernormative{\subparagraph}{Inner Header Hash}{Device Types / Network Device / Device Operation / Control Virtqueue / Inner Header Hash}

The driver MUST have negotiated the VIRTIO_NET_F_HASH_TUNNEL feature when issuing the VIRTIO_NET_CTRL_HASH_TUNNEL_SET command.

The driver MUST NOT set any bits in \field{enabled_tunnel_types} which are not set in \field{supported_tunnel_types}.

The driver MUST ignore bits in \field{supported_tunnel_types} which are not documented in this specification.

\paragraph{Hash reporting for incoming packets}
\label{sec:Device Types / Network Device / Device Operation / Processing of Incoming Packets / Hash reporting for incoming packets}

If VIRTIO_NET_F_HASH_REPORT was negotiated and
 the device has calculated the hash for the packet, the device fills \field{hash_report} with the report type of calculated hash
and \field{hash_value} with the value of calculated hash.

If VIRTIO_NET_F_HASH_REPORT was negotiated but due to any reason the
hash was not calculated, the device sets \field{hash_report} to VIRTIO_NET_HASH_REPORT_NONE.

Possible values that the device can report in \field{hash_report} are defined below.
They correspond to supported hash types defined in
\ref{sec:Device Types / Network Device / Device Operation / Processing of Incoming Packets / Hash calculation for incoming packets / Supported/enabled hash types}
as follows:

VIRTIO_NET_HASH_TYPE_XXX = 1 << (VIRTIO_NET_HASH_REPORT_XXX - 1)

\begin{lstlisting}
#define VIRTIO_NET_HASH_REPORT_NONE            0
#define VIRTIO_NET_HASH_REPORT_IPv4            1
#define VIRTIO_NET_HASH_REPORT_TCPv4           2
#define VIRTIO_NET_HASH_REPORT_UDPv4           3
#define VIRTIO_NET_HASH_REPORT_IPv6            4
#define VIRTIO_NET_HASH_REPORT_TCPv6           5
#define VIRTIO_NET_HASH_REPORT_UDPv6           6
#define VIRTIO_NET_HASH_REPORT_IPv6_EX         7
#define VIRTIO_NET_HASH_REPORT_TCPv6_EX        8
#define VIRTIO_NET_HASH_REPORT_UDPv6_EX        9
\end{lstlisting}

\subsubsection{Control Virtqueue}\label{sec:Device Types / Network Device / Device Operation / Control Virtqueue}

The driver uses the control virtqueue (if VIRTIO_NET_F_CTRL_VQ is
negotiated) to send commands to manipulate various features of
the device which would not easily map into the configuration
space.

All commands are of the following form:

\begin{lstlisting}
struct virtio_net_ctrl {
        u8 class;
        u8 command;
        u8 command-specific-data[];
        u8 ack;
        u8 command-specific-result[];
};

/* ack values */
#define VIRTIO_NET_OK     0
#define VIRTIO_NET_ERR    1
\end{lstlisting}

The \field{class}, \field{command} and command-specific-data are set by the
driver, and the device sets the \field{ack} byte and optionally
\field{command-specific-result}. There is little the driver can
do except issue a diagnostic if \field{ack} is not VIRTIO_NET_OK.

The command VIRTIO_NET_CTRL_STATS_QUERY and VIRTIO_NET_CTRL_STATS_GET contain
\field{command-specific-result}.

\paragraph{Packet Receive Filtering}\label{sec:Device Types / Network Device / Device Operation / Control Virtqueue / Packet Receive Filtering}
\label{sec:Device Types / Network Device / Device Operation / Control Virtqueue / Setting Promiscuous Mode}%old label for latexdiff

If the VIRTIO_NET_F_CTRL_RX and VIRTIO_NET_F_CTRL_RX_EXTRA
features are negotiated, the driver can send control commands for
promiscuous mode, multicast, unicast and broadcast receiving.

\begin{note}
In general, these commands are best-effort: unwanted
packets could still arrive.
\end{note}

\begin{lstlisting}
#define VIRTIO_NET_CTRL_RX    0
 #define VIRTIO_NET_CTRL_RX_PROMISC      0
 #define VIRTIO_NET_CTRL_RX_ALLMULTI     1
 #define VIRTIO_NET_CTRL_RX_ALLUNI       2
 #define VIRTIO_NET_CTRL_RX_NOMULTI      3
 #define VIRTIO_NET_CTRL_RX_NOUNI        4
 #define VIRTIO_NET_CTRL_RX_NOBCAST      5
\end{lstlisting}


\devicenormative{\subparagraph}{Packet Receive Filtering}{Device Types / Network Device / Device Operation / Control Virtqueue / Packet Receive Filtering}

If the VIRTIO_NET_F_CTRL_RX feature has been negotiated,
the device MUST support the following VIRTIO_NET_CTRL_RX class
commands:
\begin{itemize}
\item VIRTIO_NET_CTRL_RX_PROMISC turns promiscuous mode on and
off. The command-specific-data is one byte containing 0 (off) or
1 (on). If promiscuous mode is on, the device SHOULD receive all
incoming packets.
This SHOULD take effect even if one of the other modes set by
a VIRTIO_NET_CTRL_RX class command is on.
\item VIRTIO_NET_CTRL_RX_ALLMULTI turns all-multicast receive on and
off. The command-specific-data is one byte containing 0 (off) or
1 (on). When all-multicast receive is on the device SHOULD allow
all incoming multicast packets.
\end{itemize}

If the VIRTIO_NET_F_CTRL_RX_EXTRA feature has been negotiated,
the device MUST support the following VIRTIO_NET_CTRL_RX class
commands:
\begin{itemize}
\item VIRTIO_NET_CTRL_RX_ALLUNI turns all-unicast receive on and
off. The command-specific-data is one byte containing 0 (off) or
1 (on). When all-unicast receive is on the device SHOULD allow
all incoming unicast packets.
\item VIRTIO_NET_CTRL_RX_NOMULTI suppresses multicast receive.
The command-specific-data is one byte containing 0 (multicast
receive allowed) or 1 (multicast receive suppressed).
When multicast receive is suppressed, the device SHOULD NOT
send multicast packets to the driver.
This SHOULD take effect even if VIRTIO_NET_CTRL_RX_ALLMULTI is on.
This filter SHOULD NOT apply to broadcast packets.
\item VIRTIO_NET_CTRL_RX_NOUNI suppresses unicast receive.
The command-specific-data is one byte containing 0 (unicast
receive allowed) or 1 (unicast receive suppressed).
When unicast receive is suppressed, the device SHOULD NOT
send unicast packets to the driver.
This SHOULD take effect even if VIRTIO_NET_CTRL_RX_ALLUNI is on.
\item VIRTIO_NET_CTRL_RX_NOBCAST suppresses broadcast receive.
The command-specific-data is one byte containing 0 (broadcast
receive allowed) or 1 (broadcast receive suppressed).
When broadcast receive is suppressed, the device SHOULD NOT
send broadcast packets to the driver.
This SHOULD take effect even if VIRTIO_NET_CTRL_RX_ALLMULTI is on.
\end{itemize}

\drivernormative{\subparagraph}{Packet Receive Filtering}{Device Types / Network Device / Device Operation / Control Virtqueue / Packet Receive Filtering}

If the VIRTIO_NET_F_CTRL_RX feature has not been negotiated,
the driver MUST NOT issue commands VIRTIO_NET_CTRL_RX_PROMISC or
VIRTIO_NET_CTRL_RX_ALLMULTI.

If the VIRTIO_NET_F_CTRL_RX_EXTRA feature has not been negotiated,
the driver MUST NOT issue commands
 VIRTIO_NET_CTRL_RX_ALLUNI,
 VIRTIO_NET_CTRL_RX_NOMULTI,
 VIRTIO_NET_CTRL_RX_NOUNI or
 VIRTIO_NET_CTRL_RX_NOBCAST.

\paragraph{Setting MAC Address Filtering}\label{sec:Device Types / Network Device / Device Operation / Control Virtqueue / Setting MAC Address Filtering}

If the VIRTIO_NET_F_CTRL_RX feature is negotiated, the driver can
send control commands for MAC address filtering.

\begin{lstlisting}
struct virtio_net_ctrl_mac {
        le32 entries;
        u8 macs[entries][6];
};

#define VIRTIO_NET_CTRL_MAC    1
 #define VIRTIO_NET_CTRL_MAC_TABLE_SET        0
 #define VIRTIO_NET_CTRL_MAC_ADDR_SET         1
\end{lstlisting}

The device can filter incoming packets by any number of destination
MAC addresses\footnote{Since there are no guarantees, it can use a hash filter or
silently switch to allmulti or promiscuous mode if it is given too
many addresses.
}. This table is set using the class
VIRTIO_NET_CTRL_MAC and the command VIRTIO_NET_CTRL_MAC_TABLE_SET. The
command-specific-data is two variable length tables of 6-byte MAC
addresses (as described in struct virtio_net_ctrl_mac). The first table contains unicast addresses, and the second
contains multicast addresses.

The VIRTIO_NET_CTRL_MAC_ADDR_SET command is used to set the
default MAC address which rx filtering
accepts (and if VIRTIO_NET_F_MAC has been negotiated,
this will be reflected in \field{mac} in config space).

The command-specific-data for VIRTIO_NET_CTRL_MAC_ADDR_SET is
the 6-byte MAC address.

\devicenormative{\subparagraph}{Setting MAC Address Filtering}{Device Types / Network Device / Device Operation / Control Virtqueue / Setting MAC Address Filtering}

The device MUST have an empty MAC filtering table on reset.

The device MUST update the MAC filtering table before it consumes
the VIRTIO_NET_CTRL_MAC_TABLE_SET command.

The device MUST update \field{mac} in config space before it consumes
the VIRTIO_NET_CTRL_MAC_ADDR_SET command, if VIRTIO_NET_F_MAC has
been negotiated.

The device SHOULD drop incoming packets which have a destination MAC which
matches neither the \field{mac} (or that set with VIRTIO_NET_CTRL_MAC_ADDR_SET)
nor the MAC filtering table.

\drivernormative{\subparagraph}{Setting MAC Address Filtering}{Device Types / Network Device / Device Operation / Control Virtqueue / Setting MAC Address Filtering}

If VIRTIO_NET_F_CTRL_RX has not been negotiated,
the driver MUST NOT issue VIRTIO_NET_CTRL_MAC class commands.

If VIRTIO_NET_F_CTRL_RX has been negotiated,
the driver SHOULD issue VIRTIO_NET_CTRL_MAC_ADDR_SET
to set the default mac if it is different from \field{mac}.

The driver MUST follow the VIRTIO_NET_CTRL_MAC_TABLE_SET command
by a le32 number, followed by that number of non-multicast
MAC addresses, followed by another le32 number, followed by
that number of multicast addresses.  Either number MAY be 0.

\subparagraph{Legacy Interface: Setting MAC Address Filtering}\label{sec:Device Types / Network Device / Device Operation / Control Virtqueue / Setting MAC Address Filtering / Legacy Interface: Setting MAC Address Filtering}
When using the legacy interface, transitional devices and drivers
MUST format \field{entries} in struct virtio_net_ctrl_mac
according to the native endian of the guest rather than
(necessarily when not using the legacy interface) little-endian.

Legacy drivers that didn't negotiate VIRTIO_NET_F_CTRL_MAC_ADDR
changed \field{mac} in config space when NIC is accepting
incoming packets. These drivers always wrote the mac value from
first to last byte, therefore after detecting such drivers,
a transitional device MAY defer MAC update, or MAY defer
processing incoming packets until driver writes the last byte
of \field{mac} in the config space.

\paragraph{VLAN Filtering}\label{sec:Device Types / Network Device / Device Operation / Control Virtqueue / VLAN Filtering}

If the driver negotiates the VIRTIO_NET_F_CTRL_VLAN feature, it
can control a VLAN filter table in the device. The VLAN filter
table applies only to VLAN tagged packets.

When VIRTIO_NET_F_CTRL_VLAN is negotiated, the device starts with
an empty VLAN filter table.

\begin{note}
Similar to the MAC address based filtering, the VLAN filtering
is also best-effort: unwanted packets could still arrive.
\end{note}

\begin{lstlisting}
#define VIRTIO_NET_CTRL_VLAN       2
 #define VIRTIO_NET_CTRL_VLAN_ADD             0
 #define VIRTIO_NET_CTRL_VLAN_DEL             1
\end{lstlisting}

Both the VIRTIO_NET_CTRL_VLAN_ADD and VIRTIO_NET_CTRL_VLAN_DEL
command take a little-endian 16-bit VLAN id as the command-specific-data.

VIRTIO_NET_CTRL_VLAN_ADD command adds the specified VLAN to the
VLAN filter table.

VIRTIO_NET_CTRL_VLAN_DEL command removes the specified VLAN from
the VLAN filter table.

\devicenormative{\subparagraph}{VLAN Filtering}{Device Types / Network Device / Device Operation / Control Virtqueue / VLAN Filtering}

When VIRTIO_NET_F_CTRL_VLAN is not negotiated, the device MUST
accept all VLAN tagged packets.

When VIRTIO_NET_F_CTRL_VLAN is negotiated, the device MUST
accept all VLAN tagged packets whose VLAN tag is present in
the VLAN filter table and SHOULD drop all VLAN tagged packets
whose VLAN tag is absent in the VLAN filter table.

\subparagraph{Legacy Interface: VLAN Filtering}\label{sec:Device Types / Network Device / Device Operation / Control Virtqueue / VLAN Filtering / Legacy Interface: VLAN Filtering}
When using the legacy interface, transitional devices and drivers
MUST format the VLAN id
according to the native endian of the guest rather than
(necessarily when not using the legacy interface) little-endian.

\paragraph{Gratuitous Packet Sending}\label{sec:Device Types / Network Device / Device Operation / Control Virtqueue / Gratuitous Packet Sending}

If the driver negotiates the VIRTIO_NET_F_GUEST_ANNOUNCE (depends
on VIRTIO_NET_F_CTRL_VQ), the device can ask the driver to send gratuitous
packets; this is usually done after the guest has been physically
migrated, and needs to announce its presence on the new network
links. (As hypervisor does not have the knowledge of guest
network configuration (eg. tagged vlan) it is simplest to prod
the guest in this way).

\begin{lstlisting}
#define VIRTIO_NET_CTRL_ANNOUNCE       3
 #define VIRTIO_NET_CTRL_ANNOUNCE_ACK             0
\end{lstlisting}

The driver checks VIRTIO_NET_S_ANNOUNCE bit in the device configuration \field{status} field
when it notices the changes of device configuration. The
command VIRTIO_NET_CTRL_ANNOUNCE_ACK is used to indicate that
driver has received the notification and device clears the
VIRTIO_NET_S_ANNOUNCE bit in \field{status}.

Processing this notification involves:

\begin{enumerate}
\item Sending the gratuitous packets (eg. ARP) or marking there are pending
  gratuitous packets to be sent and letting deferred routine to
  send them.

\item Sending VIRTIO_NET_CTRL_ANNOUNCE_ACK command through control
  vq.
\end{enumerate}

\drivernormative{\subparagraph}{Gratuitous Packet Sending}{Device Types / Network Device / Device Operation / Control Virtqueue / Gratuitous Packet Sending}

If the driver negotiates VIRTIO_NET_F_GUEST_ANNOUNCE, it SHOULD notify
network peers of its new location after it sees the VIRTIO_NET_S_ANNOUNCE bit
in \field{status}.  The driver MUST send a command on the command queue
with class VIRTIO_NET_CTRL_ANNOUNCE and command VIRTIO_NET_CTRL_ANNOUNCE_ACK.

\devicenormative{\subparagraph}{Gratuitous Packet Sending}{Device Types / Network Device / Device Operation / Control Virtqueue / Gratuitous Packet Sending}

If VIRTIO_NET_F_GUEST_ANNOUNCE is negotiated, the device MUST clear the
VIRTIO_NET_S_ANNOUNCE bit in \field{status} upon receipt of a command buffer
with class VIRTIO_NET_CTRL_ANNOUNCE and command VIRTIO_NET_CTRL_ANNOUNCE_ACK
before marking the buffer as used.

\paragraph{Device operation in multiqueue mode}\label{sec:Device Types / Network Device / Device Operation / Control Virtqueue / Device operation in multiqueue mode}

This specification defines the following modes that a device MAY implement for operation with multiple transmit/receive virtqueues:
\begin{itemize}
\item Automatic receive steering as defined in \ref{sec:Device Types / Network Device / Device Operation / Control Virtqueue / Automatic receive steering in multiqueue mode}.
 If a device supports this mode, it offers the VIRTIO_NET_F_MQ feature bit.
\item Receive-side scaling as defined in \ref{devicenormative:Device Types / Network Device / Device Operation / Control Virtqueue / Receive-side scaling (RSS) / RSS processing}.
 If a device supports this mode, it offers the VIRTIO_NET_F_RSS feature bit.
\end{itemize}

A device MAY support one of these features or both. The driver MAY negotiate any set of these features that the device supports.

Multiqueue is disabled by default.

The driver enables multiqueue by sending a command using \field{class} VIRTIO_NET_CTRL_MQ. The \field{command} selects the mode of multiqueue operation, as follows:
\begin{lstlisting}
#define VIRTIO_NET_CTRL_MQ    4
 #define VIRTIO_NET_CTRL_MQ_VQ_PAIRS_SET        0 (for automatic receive steering)
 #define VIRTIO_NET_CTRL_MQ_RSS_CONFIG          1 (for configurable receive steering)
 #define VIRTIO_NET_CTRL_MQ_HASH_CONFIG         2 (for configurable hash calculation)
\end{lstlisting}

If more than one multiqueue mode is negotiated, the resulting device configuration is defined by the last command sent by the driver.

\paragraph{Automatic receive steering in multiqueue mode}\label{sec:Device Types / Network Device / Device Operation / Control Virtqueue / Automatic receive steering in multiqueue mode}

If the driver negotiates the VIRTIO_NET_F_MQ feature bit (depends on VIRTIO_NET_F_CTRL_VQ), it MAY transmit outgoing packets on one
of the multiple transmitq1\ldots transmitqN and ask the device to
queue incoming packets into one of the multiple receiveq1\ldots receiveqN
depending on the packet flow.

The driver enables multiqueue by
sending the VIRTIO_NET_CTRL_MQ_VQ_PAIRS_SET command, specifying
the number of the transmit and receive queues to be used up to
\field{max_virtqueue_pairs}; subsequently,
transmitq1\ldots transmitqn and receiveq1\ldots receiveqn where
n=\field{virtqueue_pairs} MAY be used.
\begin{lstlisting}
struct virtio_net_ctrl_mq_pairs_set {
       le16 virtqueue_pairs;
};
#define VIRTIO_NET_CTRL_MQ_VQ_PAIRS_MIN        1
#define VIRTIO_NET_CTRL_MQ_VQ_PAIRS_MAX        0x8000

\end{lstlisting}

When multiqueue is enabled by VIRTIO_NET_CTRL_MQ_VQ_PAIRS_SET command, the device MUST use automatic receive steering
based on packet flow. Programming of the receive steering
classificator is implicit. After the driver transmitted a packet of a
flow on transmitqX, the device SHOULD cause incoming packets for that flow to
be steered to receiveqX. For uni-directional protocols, or where
no packets have been transmitted yet, the device MAY steer a packet
to a random queue out of the specified receiveq1\ldots receiveqn.

Multiqueue is disabled by VIRTIO_NET_CTRL_MQ_VQ_PAIRS_SET with \field{virtqueue_pairs} to 1 (this is
the default) and waiting for the device to use the command buffer.

\drivernormative{\subparagraph}{Automatic receive steering in multiqueue mode}{Device Types / Network Device / Device Operation / Control Virtqueue / Automatic receive steering in multiqueue mode}

The driver MUST configure the virtqueues before enabling them with the
VIRTIO_NET_CTRL_MQ_VQ_PAIRS_SET command.

The driver MUST NOT request a \field{virtqueue_pairs} of 0 or
greater than \field{max_virtqueue_pairs} in the device configuration space.

The driver MUST queue packets only on any transmitq1 before the
VIRTIO_NET_CTRL_MQ_VQ_PAIRS_SET command.

The driver MUST NOT queue packets on transmit queues greater than
\field{virtqueue_pairs} once it has placed the VIRTIO_NET_CTRL_MQ_VQ_PAIRS_SET command in the available ring.

\devicenormative{\subparagraph}{Automatic receive steering in multiqueue mode}{Device Types / Network Device / Device Operation / Control Virtqueue / Automatic receive steering in multiqueue mode}

After initialization of reset, the device MUST queue packets only on receiveq1.

The device MUST NOT queue packets on receive queues greater than
\field{virtqueue_pairs} once it has placed the
VIRTIO_NET_CTRL_MQ_VQ_PAIRS_SET command in a used buffer.

If the destination receive queue is being reset (See \ref{sec:Basic Facilities of a Virtio Device / Virtqueues / Virtqueue Reset}),
the device SHOULD re-select another random queue. If all receive queues are
being reset, the device MUST drop the packet.

\subparagraph{Legacy Interface: Automatic receive steering in multiqueue mode}\label{sec:Device Types / Network Device / Device Operation / Control Virtqueue / Automatic receive steering in multiqueue mode / Legacy Interface: Automatic receive steering in multiqueue mode}
When using the legacy interface, transitional devices and drivers
MUST format \field{virtqueue_pairs}
according to the native endian of the guest rather than
(necessarily when not using the legacy interface) little-endian.

\subparagraph{Hash calculation}\label{sec:Device Types / Network Device / Device Operation / Control Virtqueue / Automatic receive steering in multiqueue mode / Hash calculation}
If VIRTIO_NET_F_HASH_REPORT was negotiated and the device uses automatic receive steering,
the device MUST support a command to configure hash calculation parameters.

The driver provides parameters for hash calculation as follows:

\field{class} VIRTIO_NET_CTRL_MQ, \field{command} VIRTIO_NET_CTRL_MQ_HASH_CONFIG.

The \field{command-specific-data} has following format:
\begin{lstlisting}
struct virtio_net_hash_config {
    le32 hash_types;
    le16 reserved[4];
    u8 hash_key_length;
    u8 hash_key_data[hash_key_length];
};
\end{lstlisting}
Field \field{hash_types} contains a bitmask of allowed hash types as
defined in
\ref{sec:Device Types / Network Device / Device Operation / Processing of Incoming Packets / Hash calculation for incoming packets / Supported/enabled hash types}.
Initially the device has all hash types disabled and reports only VIRTIO_NET_HASH_REPORT_NONE.

Field \field{reserved} MUST contain zeroes. It is defined to make the structure to match the layout of virtio_net_rss_config structure,
defined in \ref{sec:Device Types / Network Device / Device Operation / Control Virtqueue / Receive-side scaling (RSS)}.

Fields \field{hash_key_length} and \field{hash_key_data} define the key to be used in hash calculation.

\paragraph{Receive-side scaling (RSS)}\label{sec:Device Types / Network Device / Device Operation / Control Virtqueue / Receive-side scaling (RSS)}
A device offers the feature VIRTIO_NET_F_RSS if it supports RSS receive steering with Toeplitz hash calculation and configurable parameters.

A driver queries RSS capabilities of the device by reading device configuration as defined in \ref{sec:Device Types / Network Device / Device configuration layout}

\subparagraph{Setting RSS parameters}\label{sec:Device Types / Network Device / Device Operation / Control Virtqueue / Receive-side scaling (RSS) / Setting RSS parameters}

Driver sends a VIRTIO_NET_CTRL_MQ_RSS_CONFIG command using the following format for \field{command-specific-data}:
\begin{lstlisting}
struct rss_rq_id {
   le16 vq_index_1_16: 15; /* Bits 1 to 16 of the virtqueue index */
   le16 reserved: 1; /* Set to zero */
};

struct virtio_net_rss_config {
    le32 hash_types;
    le16 indirection_table_mask;
    struct rss_rq_id unclassified_queue;
    struct rss_rq_id indirection_table[indirection_table_length];
    le16 max_tx_vq;
    u8 hash_key_length;
    u8 hash_key_data[hash_key_length];
};
\end{lstlisting}
Field \field{hash_types} contains a bitmask of allowed hash types as
defined in
\ref{sec:Device Types / Network Device / Device Operation / Processing of Incoming Packets / Hash calculation for incoming packets / Supported/enabled hash types}.

Field \field{indirection_table_mask} is a mask to be applied to
the calculated hash to produce an index in the
\field{indirection_table} array.
Number of entries in \field{indirection_table} is (\field{indirection_table_mask} + 1).

\field{rss_rq_id} is a receive virtqueue id. \field{vq_index_1_16}
consists of bits 1 to 16 of a virtqueue index. For example, a
\field{vq_index_1_16} value of 3 corresponds to virtqueue index 6,
which maps to receiveq4.

Field \field{unclassified_queue} specifies the receive virtqueue id in which to
place unclassified packets.

Field \field{indirection_table} is an array of receive virtqueues ids.

A driver sets \field{max_tx_vq} to inform a device how many transmit virtqueues it may use (transmitq1\ldots transmitq \field{max_tx_vq}).

Fields \field{hash_key_length} and \field{hash_key_data} define the key to be used in hash calculation.

\drivernormative{\subparagraph}{Setting RSS parameters}{Device Types / Network Device / Device Operation / Control Virtqueue / Receive-side scaling (RSS) }

A driver MUST NOT send the VIRTIO_NET_CTRL_MQ_RSS_CONFIG command if the feature VIRTIO_NET_F_RSS has not been negotiated.

A driver MUST fill the \field{indirection_table} array only with
enabled receive virtqueues ids.

The number of entries in \field{indirection_table} (\field{indirection_table_mask} + 1) MUST be a power of two.

A driver MUST use \field{indirection_table_mask} values that are less than \field{rss_max_indirection_table_length} reported by a device.

A driver MUST NOT set any VIRTIO_NET_HASH_TYPE_ flags that are not supported by a device.

\devicenormative{\subparagraph}{RSS processing}{Device Types / Network Device / Device Operation / Control Virtqueue / Receive-side scaling (RSS) / RSS processing}
The device MUST determine the destination queue for a network packet as follows:
\begin{itemize}
\item Calculate the hash of the packet as defined in \ref{sec:Device Types / Network Device / Device Operation / Processing of Incoming Packets / Hash calculation for incoming packets}.
\item If the device did not calculate the hash for the specific packet, the device directs the packet to the receiveq specified by \field{unclassified_queue} of virtio_net_rss_config structure.
\item Apply \field{indirection_table_mask} to the calculated hash
and use the result as the index in the indirection table to get
the destination receive virtqueue id.
\item If the destination receive queue is being reset (See \ref{sec:Basic Facilities of a Virtio Device / Virtqueues / Virtqueue Reset}), the device MUST drop the packet.
\end{itemize}

\paragraph{RSS Context}\label{sec:Device Types / Network Device / Device Operation / Control Virtqueue / RSS Context}

An RSS context consists of configurable parameters specified by \ref{sec:Device Types / Network Device
/ Device Operation / Control Virtqueue / Receive-side scaling (RSS)}.

The RSS configuration supported by VIRTIO_NET_F_RSS is considered the default RSS configuration.

The device offers the feature VIRTIO_NET_F_RSS_CONTEXT if it supports one or multiple RSS contexts
(excluding the default RSS configuration) and configurable parameters.

\subparagraph{Querying RSS Context Capability}\label{sec:Device Types / Network Device / Device Operation / Control Virtqueue / RSS Context / Querying RSS Context Capability}

\begin{lstlisting}
#define VIRTNET_RSS_CTX_CTRL 9
 #define VIRTNET_RSS_CTX_CTRL_CAP_GET  0
 #define VIRTNET_RSS_CTX_CTRL_ADD      1
 #define VIRTNET_RSS_CTX_CTRL_MOD      2
 #define VIRTNET_RSS_CTX_CTRL_DEL      3

struct virtnet_rss_ctx_cap {
    le16 max_rss_contexts;
}
\end{lstlisting}

Field \field{max_rss_contexts} specifies the maximum number of RSS contexts \ref{sec:Device Types / Network Device /
Device Operation / Control Virtqueue / RSS Context} supported by the device.

The driver queries the RSS context capability of the device by sending the command VIRTNET_RSS_CTX_CTRL_CAP_GET
with the structure virtnet_rss_ctx_cap.

For the command VIRTNET_RSS_CTX_CTRL_CAP_GET, the structure virtnet_rss_ctx_cap is write-only for the device.

\subparagraph{Setting RSS Context Parameters}\label{sec:Device Types / Network Device / Device Operation / Control Virtqueue / RSS Context / Setting RSS Context Parameters}

\begin{lstlisting}
struct virtnet_rss_ctx_add_modify {
    le16 rss_ctx_id;
    u8 reserved[6];
    struct virtio_net_rss_config rss;
};

struct virtnet_rss_ctx_del {
    le16 rss_ctx_id;
};
\end{lstlisting}

RSS context parameters:
\begin{itemize}
\item  \field{rss_ctx_id}: ID of the specific RSS context.
\item  \field{rss}: RSS context parameters of the specific RSS context whose id is \field{rss_ctx_id}.
\end{itemize}

\field{reserved} is reserved and it is ignored by the device.

If the feature VIRTIO_NET_F_RSS_CONTEXT has been negotiated, the driver can send the following
VIRTNET_RSS_CTX_CTRL class commands:
\begin{enumerate}
\item VIRTNET_RSS_CTX_CTRL_ADD: use the structure virtnet_rss_ctx_add_modify to
       add an RSS context configured as \field{rss} and id as \field{rss_ctx_id} for the device.
\item VIRTNET_RSS_CTX_CTRL_MOD: use the structure virtnet_rss_ctx_add_modify to
       configure parameters of the RSS context whose id is \field{rss_ctx_id} as \field{rss} for the device.
\item VIRTNET_RSS_CTX_CTRL_DEL: use the structure virtnet_rss_ctx_del to delete
       the RSS context whose id is \field{rss_ctx_id} for the device.
\end{enumerate}

For commands VIRTNET_RSS_CTX_CTRL_ADD and VIRTNET_RSS_CTX_CTRL_MOD, the structure virtnet_rss_ctx_add_modify is read-only for the device.
For the command VIRTNET_RSS_CTX_CTRL_DEL, the structure virtnet_rss_ctx_del is read-only for the device.

\devicenormative{\subparagraph}{RSS Context}{Device Types / Network Device / Device Operation / Control Virtqueue / RSS Context}

The device MUST set \field{max_rss_contexts} to at least 1 if it offers VIRTIO_NET_F_RSS_CONTEXT.

Upon reset, the device MUST clear all previously configured RSS contexts.

\drivernormative{\subparagraph}{RSS Context}{Device Types / Network Device / Device Operation / Control Virtqueue / RSS Context}

The driver MUST have negotiated the VIRTIO_NET_F_RSS_CONTEXT feature when issuing the VIRTNET_RSS_CTX_CTRL class commands.

The driver MUST set \field{rss_ctx_id} to between 1 and \field{max_rss_contexts} inclusive.

The driver MUST NOT send the command VIRTIO_NET_CTRL_MQ_VQ_PAIRS_SET when the device has successfully configured at least one RSS context.

\paragraph{Offloads State Configuration}\label{sec:Device Types / Network Device / Device Operation / Control Virtqueue / Offloads State Configuration}

If the VIRTIO_NET_F_CTRL_GUEST_OFFLOADS feature is negotiated, the driver can
send control commands for dynamic offloads state configuration.

\subparagraph{Setting Offloads State}\label{sec:Device Types / Network Device / Device Operation / Control Virtqueue / Offloads State Configuration / Setting Offloads State}

To configure the offloads, the following layout structure and
definitions are used:

\begin{lstlisting}
le64 offloads;

#define VIRTIO_NET_F_GUEST_CSUM       1
#define VIRTIO_NET_F_GUEST_TSO4       7
#define VIRTIO_NET_F_GUEST_TSO6       8
#define VIRTIO_NET_F_GUEST_ECN        9
#define VIRTIO_NET_F_GUEST_UFO        10
#define VIRTIO_NET_F_GUEST_UDP_TUNNEL_GSO  46
#define VIRTIO_NET_F_GUEST_UDP_TUNNEL_GSO_CSUM 47
#define VIRTIO_NET_F_GUEST_USO4       54
#define VIRTIO_NET_F_GUEST_USO6       55

#define VIRTIO_NET_CTRL_GUEST_OFFLOADS       5
 #define VIRTIO_NET_CTRL_GUEST_OFFLOADS_SET   0
\end{lstlisting}

The class VIRTIO_NET_CTRL_GUEST_OFFLOADS has one command:
VIRTIO_NET_CTRL_GUEST_OFFLOADS_SET applies the new offloads configuration.

le64 value passed as command data is a bitmask, bits set define
offloads to be enabled, bits cleared - offloads to be disabled.

There is a corresponding device feature for each offload. Upon feature
negotiation corresponding offload gets enabled to preserve backward
compatibility.

\drivernormative{\subparagraph}{Setting Offloads State}{Device Types / Network Device / Device Operation / Control Virtqueue / Offloads State Configuration / Setting Offloads State}

A driver MUST NOT enable an offload for which the appropriate feature
has not been negotiated.

\subparagraph{Legacy Interface: Setting Offloads State}\label{sec:Device Types / Network Device / Device Operation / Control Virtqueue / Offloads State Configuration / Setting Offloads State / Legacy Interface: Setting Offloads State}
When using the legacy interface, transitional devices and drivers
MUST format \field{offloads}
according to the native endian of the guest rather than
(necessarily when not using the legacy interface) little-endian.


\paragraph{Notifications Coalescing}\label{sec:Device Types / Network Device / Device Operation / Control Virtqueue / Notifications Coalescing}

If the VIRTIO_NET_F_NOTF_COAL feature is negotiated, the driver can
send commands VIRTIO_NET_CTRL_NOTF_COAL_TX_SET and VIRTIO_NET_CTRL_NOTF_COAL_RX_SET
for notification coalescing.

If the VIRTIO_NET_F_VQ_NOTF_COAL feature is negotiated, the driver can
send commands VIRTIO_NET_CTRL_NOTF_COAL_VQ_SET and VIRTIO_NET_CTRL_NOTF_COAL_VQ_GET
for virtqueue notification coalescing.

\begin{lstlisting}
struct virtio_net_ctrl_coal {
    le32 max_packets;
    le32 max_usecs;
};

struct virtio_net_ctrl_coal_vq {
    le16 vq_index;
    le16 reserved;
    struct virtio_net_ctrl_coal coal;
};

#define VIRTIO_NET_CTRL_NOTF_COAL 6
 #define VIRTIO_NET_CTRL_NOTF_COAL_TX_SET  0
 #define VIRTIO_NET_CTRL_NOTF_COAL_RX_SET 1
 #define VIRTIO_NET_CTRL_NOTF_COAL_VQ_SET 2
 #define VIRTIO_NET_CTRL_NOTF_COAL_VQ_GET 3
\end{lstlisting}

Coalescing parameters:
\begin{itemize}
\item \field{vq_index}: The virtqueue index of an enabled transmit or receive virtqueue.
\item \field{max_usecs} for RX: Maximum number of microseconds to delay a RX notification.
\item \field{max_usecs} for TX: Maximum number of microseconds to delay a TX notification.
\item \field{max_packets} for RX: Maximum number of packets to receive before a RX notification.
\item \field{max_packets} for TX: Maximum number of packets to send before a TX notification.
\end{itemize}

\field{reserved} is reserved and it is ignored by the device.

Read/Write attributes for coalescing parameters:
\begin{itemize}
\item For commands VIRTIO_NET_CTRL_NOTF_COAL_TX_SET and VIRTIO_NET_CTRL_NOTF_COAL_RX_SET, the structure virtio_net_ctrl_coal is write-only for the driver.
\item For the command VIRTIO_NET_CTRL_NOTF_COAL_VQ_SET, the structure virtio_net_ctrl_coal_vq is write-only for the driver.
\item For the command VIRTIO_NET_CTRL_NOTF_COAL_VQ_GET, \field{vq_index} and \field{reserved} are write-only
      for the driver, and the structure virtio_net_ctrl_coal is read-only for the driver.
\end{itemize}

The class VIRTIO_NET_CTRL_NOTF_COAL has the following commands:
\begin{enumerate}
\item VIRTIO_NET_CTRL_NOTF_COAL_TX_SET: use the structure virtio_net_ctrl_coal to set the \field{max_usecs} and \field{max_packets} parameters for all transmit virtqueues.
\item VIRTIO_NET_CTRL_NOTF_COAL_RX_SET: use the structure virtio_net_ctrl_coal to set the \field{max_usecs} and \field{max_packets} parameters for all receive virtqueues.
\item VIRTIO_NET_CTRL_NOTF_COAL_VQ_SET: use the structure virtio_net_ctrl_coal_vq to set the \field{max_usecs} and \field{max_packets} parameters
                                        for an enabled transmit/receive virtqueue whose index is \field{vq_index}.
\item VIRTIO_NET_CTRL_NOTF_COAL_VQ_GET: use the structure virtio_net_ctrl_coal_vq to get the \field{max_usecs} and \field{max_packets} parameters
                                        for an enabled transmit/receive virtqueue whose index is \field{vq_index}.
\end{enumerate}

The device may generate notifications more or less frequently than specified by set commands of the VIRTIO_NET_CTRL_NOTF_COAL class.

If coalescing parameters are being set, the device applies the last coalescing parameters set for a
virtqueue, regardless of the command used to set the parameters. Use the following command sequence
with two pairs of virtqueues as an example:
Each of the following commands sets \field{max_usecs} and \field{max_packets} parameters for virtqueues.
\begin{itemize}
\item Command1: VIRTIO_NET_CTRL_NOTF_COAL_RX_SET sets coalescing parameters for virtqueues having index 0 and index 2. Virtqueues having index 1 and index 3 retain their previous parameters.
\item Command2: VIRTIO_NET_CTRL_NOTF_COAL_VQ_SET with \field{vq_index} = 0 sets coalescing parameters for virtqueue having index 0. Virtqueue having index 2 retains the parameters from command1.
\item Command3: VIRTIO_NET_CTRL_NOTF_COAL_VQ_GET with \field{vq_index} = 0, the device responds with coalescing parameters of vq_index 0 set by command2.
\item Command4: VIRTIO_NET_CTRL_NOTF_COAL_VQ_SET with \field{vq_index} = 1 sets coalescing parameters for virtqueue having index 1. Virtqueue having index 3 retains its previous parameters.
\item Command5: VIRTIO_NET_CTRL_NOTF_COAL_TX_SET sets coalescing parameters for virtqueues having index 1 and index 3, and overrides the parameters set by command4.
\item Command6: VIRTIO_NET_CTRL_NOTF_COAL_VQ_GET with \field{vq_index} = 1, the device responds with coalescing parameters of index 1 set by command5.
\end{itemize}

\subparagraph{Operation}\label{sec:Device Types / Network Device / Device Operation / Control Virtqueue / Notifications Coalescing / Operation}

The device sends a used buffer notification once the notification conditions are met and if the notifications are not suppressed as explained in \ref{sec:Basic Facilities of a Virtio Device / Virtqueues / Used Buffer Notification Suppression}.

When the device has non-zero \field{max_usecs} and non-zero \field{max_packets}, it starts counting microseconds and packets upon receiving/sending a packet.
The device counts packets and microseconds for each receive virtqueue and transmit virtqueue separately.
In this case, the notification conditions are met when \field{max_usecs} microseconds elapse, or upon sending/receiving \field{max_packets} packets, whichever happens first.
Afterwards, the device waits for the next packet and starts counting packets and microseconds again.

When the device has \field{max_usecs} = 0 or \field{max_packets} = 0, the notification conditions are met after every packet received/sent.

\subparagraph{RX Example}\label{sec:Device Types / Network Device / Device Operation / Control Virtqueue / Notifications Coalescing / RX Example}

If, for example:
\begin{itemize}
\item \field{max_usecs} = 10.
\item \field{max_packets} = 15.
\end{itemize}
then each receive virtqueue of a device will operate as follows:
\begin{itemize}
\item The device will count packets received on each virtqueue until it accumulates 15, or until 10 microseconds elapsed since the first one was received.
\item If the notifications are not suppressed by the driver, the device will send an used buffer notification, otherwise, the device will not send an used buffer notification as long as the notifications are suppressed.
\end{itemize}

\subparagraph{TX Example}\label{sec:Device Types / Network Device / Device Operation / Control Virtqueue / Notifications Coalescing / TX Example}

If, for example:
\begin{itemize}
\item \field{max_usecs} = 10.
\item \field{max_packets} = 15.
\end{itemize}
then each transmit virtqueue of a device will operate as follows:
\begin{itemize}
\item The device will count packets sent on each virtqueue until it accumulates 15, or until 10 microseconds elapsed since the first one was sent.
\item If the notifications are not suppressed by the driver, the device will send an used buffer notification, otherwise, the device will not send an used buffer notification as long as the notifications are suppressed.
\end{itemize}

\subparagraph{Notifications When Coalescing Parameters Change}\label{sec:Device Types / Network Device / Device Operation / Control Virtqueue / Notifications Coalescing / Notifications When Coalescing Parameters Change}

When the coalescing parameters of a device change, the device needs to check if the new notification conditions are met and send a used buffer notification if so.

For example, \field{max_packets} = 15 for a device with a single transmit virtqueue: if the device sends 10 packets and afterwards receives a
VIRTIO_NET_CTRL_NOTF_COAL_TX_SET command with \field{max_packets} = 8, then the notification condition is immediately considered to be met;
the device needs to immediately send a used buffer notification, if the notifications are not suppressed by the driver.

\drivernormative{\subparagraph}{Notifications Coalescing}{Device Types / Network Device / Device Operation / Control Virtqueue / Notifications Coalescing}

The driver MUST set \field{vq_index} to the virtqueue index of an enabled transmit or receive virtqueue.

The driver MUST have negotiated the VIRTIO_NET_F_NOTF_COAL feature when issuing commands VIRTIO_NET_CTRL_NOTF_COAL_TX_SET and VIRTIO_NET_CTRL_NOTF_COAL_RX_SET.

The driver MUST have negotiated the VIRTIO_NET_F_VQ_NOTF_COAL feature when issuing commands VIRTIO_NET_CTRL_NOTF_COAL_VQ_SET and VIRTIO_NET_CTRL_NOTF_COAL_VQ_GET.

The driver MUST ignore the values of coalescing parameters received from the VIRTIO_NET_CTRL_NOTF_COAL_VQ_GET command if the device responds with VIRTIO_NET_ERR.

\devicenormative{\subparagraph}{Notifications Coalescing}{Device Types / Network Device / Device Operation / Control Virtqueue / Notifications Coalescing}

The device MUST ignore \field{reserved}.

The device SHOULD respond to VIRTIO_NET_CTRL_NOTF_COAL_TX_SET and VIRTIO_NET_CTRL_NOTF_COAL_RX_SET commands with VIRTIO_NET_ERR if it was not able to change the parameters.

The device MUST respond to the VIRTIO_NET_CTRL_NOTF_COAL_VQ_SET command with VIRTIO_NET_ERR if it was not able to change the parameters.

The device MUST respond to VIRTIO_NET_CTRL_NOTF_COAL_VQ_SET and VIRTIO_NET_CTRL_NOTF_COAL_VQ_GET commands with
VIRTIO_NET_ERR if the designated virtqueue is not an enabled transmit or receive virtqueue.

Upon disabling and re-enabling a transmit virtqueue, the device MUST set the coalescing parameters of the virtqueue
to those configured through the VIRTIO_NET_CTRL_NOTF_COAL_TX_SET command, or, if the driver did not set any TX coalescing parameters, to 0.

Upon disabling and re-enabling a receive virtqueue, the device MUST set the coalescing parameters of the virtqueue
to those configured through the VIRTIO_NET_CTRL_NOTF_COAL_RX_SET command, or, if the driver did not set any RX coalescing parameters, to 0.

The behavior of the device in response to set commands of the VIRTIO_NET_CTRL_NOTF_COAL class is best-effort:
the device MAY generate notifications more or less frequently than specified.

A device SHOULD NOT send used buffer notifications to the driver if the notifications are suppressed, even if the notification conditions are met.

Upon reset, a device MUST initialize all coalescing parameters to 0.

\paragraph{Device Statistics}\label{sec:Device Types / Network Device / Device Operation / Control Virtqueue / Device Statistics}

If the VIRTIO_NET_F_DEVICE_STATS feature is negotiated, the driver can obtain
device statistics from the device by using the following command.

Different types of virtqueues have different statistics. The statistics of the
receiveq are different from those of the transmitq.

The statistics of a certain type of virtqueue are also divided into multiple types
because different types require different features. This enables the expansion
of new statistics.

In one command, the driver can obtain the statistics of one or multiple virtqueues.
Additionally, the driver can obtain multiple type statistics of each virtqueue.

\subparagraph{Query Statistic Capabilities}\label{sec:Device Types / Network Device / Device Operation / Control Virtqueue / Device Statistics / Query Statistic Capabilities}

\begin{lstlisting}
#define VIRTIO_NET_CTRL_STATS         8
#define VIRTIO_NET_CTRL_STATS_QUERY   0
#define VIRTIO_NET_CTRL_STATS_GET     1

struct virtio_net_stats_capabilities {

#define VIRTIO_NET_STATS_TYPE_CVQ       (1 << 32)

#define VIRTIO_NET_STATS_TYPE_RX_BASIC  (1 << 0)
#define VIRTIO_NET_STATS_TYPE_RX_CSUM   (1 << 1)
#define VIRTIO_NET_STATS_TYPE_RX_GSO    (1 << 2)
#define VIRTIO_NET_STATS_TYPE_RX_SPEED  (1 << 3)

#define VIRTIO_NET_STATS_TYPE_TX_BASIC  (1 << 16)
#define VIRTIO_NET_STATS_TYPE_TX_CSUM   (1 << 17)
#define VIRTIO_NET_STATS_TYPE_TX_GSO    (1 << 18)
#define VIRTIO_NET_STATS_TYPE_TX_SPEED  (1 << 19)

    le64 supported_stats_types[1];
}
\end{lstlisting}

To obtain device statistic capability, use the VIRTIO_NET_CTRL_STATS_QUERY
command. When the command completes successfully, \field{command-specific-result}
is in the format of \field{struct virtio_net_stats_capabilities}.

\subparagraph{Get Statistics}\label{sec:Device Types / Network Device / Device Operation / Control Virtqueue / Device Statistics / Get Statistics}

\begin{lstlisting}
struct virtio_net_ctrl_queue_stats {
       struct {
           le16 vq_index;
           le16 reserved[3];
           le64 types_bitmap[1];
       } stats[];
};

struct virtio_net_stats_reply_hdr {
#define VIRTIO_NET_STATS_TYPE_REPLY_CVQ       32

#define VIRTIO_NET_STATS_TYPE_REPLY_RX_BASIC  0
#define VIRTIO_NET_STATS_TYPE_REPLY_RX_CSUM   1
#define VIRTIO_NET_STATS_TYPE_REPLY_RX_GSO    2
#define VIRTIO_NET_STATS_TYPE_REPLY_RX_SPEED  3

#define VIRTIO_NET_STATS_TYPE_REPLY_TX_BASIC  16
#define VIRTIO_NET_STATS_TYPE_REPLY_TX_CSUM   17
#define VIRTIO_NET_STATS_TYPE_REPLY_TX_GSO    18
#define VIRTIO_NET_STATS_TYPE_REPLY_TX_SPEED  19
    u8 type;
    u8 reserved;
    le16 vq_index;
    le16 reserved1;
    le16 size;
}
\end{lstlisting}

To obtain device statistics, use the VIRTIO_NET_CTRL_STATS_GET command with the
\field{command-specific-data} which is in the format of
\field{struct virtio_net_ctrl_queue_stats}. When the command completes
successfully, \field{command-specific-result} contains multiple statistic
results, each statistic result has the \field{struct virtio_net_stats_reply_hdr}
as the header.

The fields of the \field{struct virtio_net_ctrl_queue_stats}:
\begin{description}
    \item [vq_index]
        The index of the virtqueue to obtain the statistics.

    \item [types_bitmap]
        This is a bitmask of the types of statistics to be obtained. Therefore, a
        \field{stats} inside \field{struct virtio_net_ctrl_queue_stats} may
        indicate multiple statistic replies for the virtqueue.
\end{description}

The fields of the \field{struct virtio_net_stats_reply_hdr}:
\begin{description}
    \item [type]
        The type of the reply statistic.

    \item [vq_index]
        The virtqueue index of the reply statistic.

    \item [size]
        The number of bytes for the statistics entry including size of \field{struct virtio_net_stats_reply_hdr}.

\end{description}

\subparagraph{Controlq Statistics}\label{sec:Device Types / Network Device / Device Operation / Control Virtqueue / Device Statistics / Controlq Statistics}

The structure corresponding to the controlq statistics is
\field{struct virtio_net_stats_cvq}. The corresponding type is
VIRTIO_NET_STATS_TYPE_CVQ. This is for the controlq.

\begin{lstlisting}
struct virtio_net_stats_cvq {
    struct virtio_net_stats_reply_hdr hdr;

    le64 command_num;
    le64 ok_num;
};
\end{lstlisting}

\begin{description}
    \item [command_num]
        The number of commands received by the device including the current command.

    \item [ok_num]
        The number of commands completed successfully by the device including the current command.
\end{description}


\subparagraph{Receiveq Basic Statistics}\label{sec:Device Types / Network Device / Device Operation / Control Virtqueue / Device Statistics / Receiveq Basic Statistics}

The structure corresponding to the receiveq basic statistics is
\field{struct virtio_net_stats_rx_basic}. The corresponding type is
VIRTIO_NET_STATS_TYPE_RX_BASIC. This is for the receiveq.

Receiveq basic statistics do not require any feature. As long as the device supports
VIRTIO_NET_F_DEVICE_STATS, the following are the receiveq basic statistics.

\begin{lstlisting}
struct virtio_net_stats_rx_basic {
    struct virtio_net_stats_reply_hdr hdr;

    le64 rx_notifications;

    le64 rx_packets;
    le64 rx_bytes;

    le64 rx_interrupts;

    le64 rx_drops;
    le64 rx_drop_overruns;
};
\end{lstlisting}

The packets described below were all presented on the specified virtqueue.
\begin{description}
    \item [rx_notifications]
        The number of driver notifications received by the device for this
        receiveq.

    \item [rx_packets]
        This is the number of packets passed to the driver by the device.

    \item [rx_bytes]
        This is the bytes of packets passed to the driver by the device.

    \item [rx_interrupts]
        The number of interrupts generated by the device for this receiveq.

    \item [rx_drops]
        This is the number of packets dropped by the device. The count includes
        all types of packets dropped by the device.

    \item [rx_drop_overruns]
        This is the number of packets dropped by the device when no more
        descriptors were available.

\end{description}

\subparagraph{Transmitq Basic Statistics}\label{sec:Device Types / Network Device / Device Operation / Control Virtqueue / Device Statistics / Transmitq Basic Statistics}

The structure corresponding to the transmitq basic statistics is
\field{struct virtio_net_stats_tx_basic}. The corresponding type is
VIRTIO_NET_STATS_TYPE_TX_BASIC. This is for the transmitq.

Transmitq basic statistics do not require any feature. As long as the device supports
VIRTIO_NET_F_DEVICE_STATS, the following are the transmitq basic statistics.

\begin{lstlisting}
struct virtio_net_stats_tx_basic {
    struct virtio_net_stats_reply_hdr hdr;

    le64 tx_notifications;

    le64 tx_packets;
    le64 tx_bytes;

    le64 tx_interrupts;

    le64 tx_drops;
    le64 tx_drop_malformed;
};
\end{lstlisting}

The packets described below are all for a specific virtqueue.
\begin{description}
    \item [tx_notifications]
        The number of driver notifications received by the device for this
        transmitq.

    \item [tx_packets]
        This is the number of packets sent by the device (not the packets
        got from the driver).

    \item [tx_bytes]
        This is the number of bytes sent by the device for all the sent packets
        (not the bytes sent got from the driver).

    \item [tx_interrupts]
        The number of interrupts generated by the device for this transmitq.

    \item [tx_drops]
        The number of packets dropped by the device. The count includes all
        types of packets dropped by the device.

    \item [tx_drop_malformed]
        The number of packets dropped by the device, when the descriptors are
        malformed. For example, the buffer is too short.
\end{description}

\subparagraph{Receiveq CSUM Statistics}\label{sec:Device Types / Network Device / Device Operation / Control Virtqueue / Device Statistics / Receiveq CSUM Statistics}

The structure corresponding to the receiveq checksum statistics is
\field{struct virtio_net_stats_rx_csum}. The corresponding type is
VIRTIO_NET_STATS_TYPE_RX_CSUM. This is for the receiveq.

Only after the VIRTIO_NET_F_GUEST_CSUM is negotiated, the receiveq checksum
statistics can be obtained.

\begin{lstlisting}
struct virtio_net_stats_rx_csum {
    struct virtio_net_stats_reply_hdr hdr;

    le64 rx_csum_valid;
    le64 rx_needs_csum;
    le64 rx_csum_none;
    le64 rx_csum_bad;
};
\end{lstlisting}

The packets described below were all presented on the specified virtqueue.
\begin{description}
    \item [rx_csum_valid]
        The number of packets with VIRTIO_NET_HDR_F_DATA_VALID.

    \item [rx_needs_csum]
        The number of packets with VIRTIO_NET_HDR_F_NEEDS_CSUM.

    \item [rx_csum_none]
        The number of packets without hardware checksum. The packet here refers
        to the non-TCP/UDP packet that the device cannot recognize.

    \item [rx_csum_bad]
        The number of packets with checksum mismatch.

\end{description}

\subparagraph{Transmitq CSUM Statistics}\label{sec:Device Types / Network Device / Device Operation / Control Virtqueue / Device Statistics / Transmitq CSUM Statistics}

The structure corresponding to the transmitq checksum statistics is
\field{struct virtio_net_stats_tx_csum}. The corresponding type is
VIRTIO_NET_STATS_TYPE_TX_CSUM. This is for the transmitq.

Only after the VIRTIO_NET_F_CSUM is negotiated, the transmitq checksum
statistics can be obtained.

The following are the transmitq checksum statistics:

\begin{lstlisting}
struct virtio_net_stats_tx_csum {
    struct virtio_net_stats_reply_hdr hdr;

    le64 tx_csum_none;
    le64 tx_needs_csum;
};
\end{lstlisting}

The packets described below are all for a specific virtqueue.
\begin{description}
    \item [tx_csum_none]
        The number of packets which do not require hardware checksum.

    \item [tx_needs_csum]
        The number of packets which require checksum calculation by the device.

\end{description}

\subparagraph{Receiveq GSO Statistics}\label{sec:Device Types / Network Device / Device Operation / Control Virtqueue / Device Statistics / Receiveq GSO Statistics}

The structure corresponding to the receivq GSO statistics is
\field{struct virtio_net_stats_rx_gso}. The corresponding type is
VIRTIO_NET_STATS_TYPE_RX_GSO. This is for the receiveq.

If one or more of the VIRTIO_NET_F_GUEST_TSO4, VIRTIO_NET_F_GUEST_TSO6
have been negotiated, the receiveq GSO statistics can be obtained.

GSO packets refer to packets passed by the device to the driver where
\field{gso_type} is not VIRTIO_NET_HDR_GSO_NONE.

\begin{lstlisting}
struct virtio_net_stats_rx_gso {
    struct virtio_net_stats_reply_hdr hdr;

    le64 rx_gso_packets;
    le64 rx_gso_bytes;
    le64 rx_gso_packets_coalesced;
    le64 rx_gso_bytes_coalesced;
};
\end{lstlisting}

The packets described below were all presented on the specified virtqueue.
\begin{description}
    \item [rx_gso_packets]
        The number of the GSO packets received by the device.

    \item [rx_gso_bytes]
        The bytes of the GSO packets received by the device.
        This includes the header size of the GSO packet.

    \item [rx_gso_packets_coalesced]
        The number of the GSO packets coalesced by the device.

    \item [rx_gso_bytes_coalesced]
        The bytes of the GSO packets coalesced by the device.
        This includes the header size of the GSO packet.
\end{description}

\subparagraph{Transmitq GSO Statistics}\label{sec:Device Types / Network Device / Device Operation / Control Virtqueue / Device Statistics / Transmitq GSO Statistics}

The structure corresponding to the transmitq GSO statistics is
\field{struct virtio_net_stats_tx_gso}. The corresponding type is
VIRTIO_NET_STATS_TYPE_TX_GSO. This is for the transmitq.

If one or more of the VIRTIO_NET_F_HOST_TSO4, VIRTIO_NET_F_HOST_TSO6,
VIRTIO_NET_F_HOST_USO options have been negotiated, the transmitq GSO statistics
can be obtained.

GSO packets refer to packets passed by the driver to the device where
\field{gso_type} is not VIRTIO_NET_HDR_GSO_NONE.
See more \ref{sec:Device Types / Network Device / Device Operation / Packet
Transmission}.

\begin{lstlisting}
struct virtio_net_stats_tx_gso {
    struct virtio_net_stats_reply_hdr hdr;

    le64 tx_gso_packets;
    le64 tx_gso_bytes;
    le64 tx_gso_segments;
    le64 tx_gso_segments_bytes;
    le64 tx_gso_packets_noseg;
    le64 tx_gso_bytes_noseg;
};
\end{lstlisting}

The packets described below are all for a specific virtqueue.
\begin{description}
    \item [tx_gso_packets]
        The number of the GSO packets sent by the device.

    \item [tx_gso_bytes]
        The bytes of the GSO packets sent by the device.

    \item [tx_gso_segments]
        The number of segments prepared from GSO packets.

    \item [tx_gso_segments_bytes]
        The bytes of segments prepared from GSO packets.

    \item [tx_gso_packets_noseg]
        The number of the GSO packets without segmentation.

    \item [tx_gso_bytes_noseg]
        The bytes of the GSO packets without segmentation.

\end{description}

\subparagraph{Receiveq Speed Statistics}\label{sec:Device Types / Network Device / Device Operation / Control Virtqueue / Device Statistics / Receiveq Speed Statistics}

The structure corresponding to the receiveq speed statistics is
\field{struct virtio_net_stats_rx_speed}. The corresponding type is
VIRTIO_NET_STATS_TYPE_RX_SPEED. This is for the receiveq.

The device has the allowance for the speed. If VIRTIO_NET_F_SPEED_DUPLEX has
been negotiated, the driver can get this by \field{speed}. When the received
packets bitrate exceeds the \field{speed}, some packets may be dropped by the
device.

\begin{lstlisting}
struct virtio_net_stats_rx_speed {
    struct virtio_net_stats_reply_hdr hdr;

    le64 rx_packets_allowance_exceeded;
    le64 rx_bytes_allowance_exceeded;
};
\end{lstlisting}

The packets described below were all presented on the specified virtqueue.
\begin{description}
    \item [rx_packets_allowance_exceeded]
        The number of the packets dropped by the device due to the received
        packets bitrate exceeding the \field{speed}.

    \item [rx_bytes_allowance_exceeded]
        The bytes of the packets dropped by the device due to the received
        packets bitrate exceeding the \field{speed}.

\end{description}

\subparagraph{Transmitq Speed Statistics}\label{sec:Device Types / Network Device / Device Operation / Control Virtqueue / Device Statistics / Transmitq Speed Statistics}

The structure corresponding to the transmitq speed statistics is
\field{struct virtio_net_stats_tx_speed}. The corresponding type is
VIRTIO_NET_STATS_TYPE_TX_SPEED. This is for the transmitq.

The device has the allowance for the speed. If VIRTIO_NET_F_SPEED_DUPLEX has
been negotiated, the driver can get this by \field{speed}. When the transmit
packets bitrate exceeds the \field{speed}, some packets may be dropped by the
device.

\begin{lstlisting}
struct virtio_net_stats_tx_speed {
    struct virtio_net_stats_reply_hdr hdr;

    le64 tx_packets_allowance_exceeded;
    le64 tx_bytes_allowance_exceeded;
};
\end{lstlisting}

The packets described below were all presented on the specified virtqueue.
\begin{description}
    \item [tx_packets_allowance_exceeded]
        The number of the packets dropped by the device due to the transmit packets
        bitrate exceeding the \field{speed}.

    \item [tx_bytes_allowance_exceeded]
        The bytes of the packets dropped by the device due to the transmit packets
        bitrate exceeding the \field{speed}.

\end{description}

\devicenormative{\subparagraph}{Device Statistics}{Device Types / Network Device / Device Operation / Control Virtqueue / Device Statistics}

When the VIRTIO_NET_F_DEVICE_STATS feature is negotiated, the device MUST reply
to the command VIRTIO_NET_CTRL_STATS_QUERY with the
\field{struct virtio_net_stats_capabilities}. \field{supported_stats_types}
includes all the statistic types supported by the device.

If \field{struct virtio_net_ctrl_queue_stats} is incorrect (such as the
following), the device MUST set \field{ack} to VIRTIO_NET_ERR. Even if there is
only one error, the device MUST fail the entire command.
\begin{itemize}
    \item \field{vq_index} exceeds the queue range.
    \item \field{types_bitmap} contains unknown types.
    \item One or more of the bits present in \field{types_bitmap} is not valid
        for the specified virtqueue.
    \item The feature corresponding to the specified \field{types_bitmap} was
        not negotiated.
\end{itemize}

The device MUST set the actual size of the bytes occupied by the reply to the
\field{size} of the \field{hdr}. And the device MUST set the \field{type} and
the \field{vq_index} of the statistic header.

The \field{command-specific-result} buffer allocated by the driver may be
smaller or bigger than all the statistics specified by
\field{struct virtio_net_ctrl_queue_stats}. The device MUST fill up only upto
the valid bytes.

The statistics counter replied by the device MUST wrap around to zero by the
device on the overflow.

\drivernormative{\subparagraph}{Device Statistics}{Device Types / Network Device / Device Operation / Control Virtqueue / Device Statistics}

The types contained in the \field{types_bitmap} MUST be queried from the device
via command VIRTIO_NET_CTRL_STATS_QUERY.

\field{types_bitmap} in \field{struct virtio_net_ctrl_queue_stats} MUST be valid to the
vq specified by \field{vq_index}.

The \field{command-specific-result} buffer allocated by the driver MUST have
enough capacity to store all the statistics reply headers defined in
\field{struct virtio_net_ctrl_queue_stats}. If the
\field{command-specific-result} buffer is fully utilized by the device but some
replies are missed, it is possible that some statistics may exceed the capacity
of the driver's records. In such cases, the driver should allocate additional
space for the \field{command-specific-result} buffer.

\subsubsection{Flow filter}\label{sec:Device Types / Network Device / Device Operation / Flow filter}

A network device can support one or more flow filter rules. Each flow filter rule
is applied by matching a packet and then taking an action, such as directing the packet
to a specific receiveq or dropping the packet. An example of a match is
matching on specific source and destination IP addresses.

A flow filter rule is a device resource object that consists of a key,
a processing priority, and an action to either direct a packet to a
receive queue or drop the packet.

Each rule uses a classifier. The key is matched against the packet using
a classifier, defining which fields in the packet are matched.
A classifier resource object consists of one or more field selectors, each with
a type that specifies the header fields to be matched against, and a mask.
The mask can match whole fields or parts of a field in a header. Each
rule resource object depends on the classifier resource object.

When a packet is received, relevant fields are extracted
(in the same way) from both the packet and the key according to the
classifier. The resulting field contents are then compared -
if they are identical the rule action is taken, if they are not, the rule is ignored.

Multiple flow filter rules are part of a group. The rule resource object
depends on the group. Each rule within a
group has a rule priority, and each group also has a group priority. For a
packet, a group with the highest priority is selected first. Within a group,
rules are applied from highest to lowest priority, until one of the rules
matches the packet and an action is taken. If all the rules within a group
are ignored, the group with the next highest priority is selected, and so on.

The device and the driver indicates flow filter resource limits using the capability
\ref{par:Device Types / Network Device / Device Operation / Flow filter / Device and driver capabilities / VIRTIO-NET-FF-RESOURCE-CAP} specifying the limits on the number of flow filter rule,
group and classifier resource objects. The capability \ref{par:Device Types / Network Device / Device Operation / Flow filter / Device and driver capabilities / VIRTIO-NET-FF-SELECTOR-CAP} specifies which selectors the device supports.
The driver indicates the selectors it is using by setting the flow
filter selector capability, prior to adding any resource objects.

The capability \ref{par:Device Types / Network Device / Device Operation / Flow filter / Device and driver capabilities / VIRTIO-NET-FF-ACTION-CAP} specifies which actions the device supports.

The driver controls the flow filter rule, classifier and group resource objects using
administration commands described in
\ref{sec:Basic Facilities of a Virtio Device / Device groups / Group administration commands / Device resource objects}.

\paragraph{Packet processing order}\label{sec:sec:Device Types / Network Device / Device Operation / Flow filter / Packet processing order}

Note that flow filter rules are applied after MAC/VLAN filtering. Flow filter
rules take precedence over steering: if a flow filter rule results in an action,
the steering configuration does not apply. The steering configuration only applies
to packets for which no flow filter rule action was performed. For example,
incoming packets can be processed in the following order:

\begin{itemize}
\item apply steering configuration received using control virtqueue commands
      VIRTIO_NET_CTRL_RX, VIRTIO_NET_CTRL_MAC and VIRTIO_NET_CTRL_VLAN.
\item apply flow filter rules if any.
\item if no filter rule applied, apply steering configuration received using command
      VIRTIO_NET_CTRL_MQ_RSS_CONFIG or as per automatic receive steering.
\end{itemize}

Some incoming packet processing examples:
\begin{itemize}
\item If the packet is dropped by the flow filter rule, RSS
      steering is ignored for the packet.
\item If the packet is directed to a specific receiveq using flow filter rule,
      the RSS steering is ignored for the packet.
\item If a packet is dropped due to the VIRTIO_NET_CTRL_MAC configuration,
      both flow filter rules and the RSS steering are ignored for the packet.
\item If a packet does not match any flow filter rules,
      the RSS steering is used to select the receiveq for the packet (if enabled).
\item If there are two flow filter groups configured as group_A and group_B
      with respective group priorities as 4, and 5; flow filter rules of
      group_B are applied first having highest group priority, if there is a match,
      the flow filter rules of group_A are ignored; if there is no match for
      the flow filter rules in group_B, the flow filter rules of next level group_A are applied.
\end{itemize}

\paragraph{Device and driver capabilities}
\label{par:Device Types / Network Device / Device Operation / Flow filter / Device and driver capabilities}

\subparagraph{VIRTIO_NET_FF_RESOURCE_CAP}
\label{par:Device Types / Network Device / Device Operation / Flow filter / Device and driver capabilities / VIRTIO-NET-FF-RESOURCE-CAP}

The capability VIRTIO_NET_FF_RESOURCE_CAP indicates the flow filter resource limits.
\field{cap_specific_data} is in the format
\field{struct virtio_net_ff_cap_data}.

\begin{lstlisting}
struct virtio_net_ff_cap_data {
        le32 groups_limit;
        le32 selectors_limit;
        le32 rules_limit;
        le32 rules_per_group_limit;
        u8 last_rule_priority;
        u8 selectors_per_classifier_limit;
};
\end{lstlisting}

\field{groups_limit}, and \field{selectors_limit} represent the maximum
number of flow filter groups and selectors, respectively, that the driver can create.
 \field{rules_limit} is the maximum number of
flow fiilter rules that the driver can create across all the groups.
\field{rules_per_group_limit} is the maximum number of flow filter rules that the driver
can create for each flow filter group.

\field{last_rule_priority} is the highest priority that can be assigned to a
flow filter rule.

\field{selectors_per_classifier_limit} is the maximum number of selectors
that a classifier can have.

\subparagraph{VIRTIO_NET_FF_SELECTOR_CAP}
\label{par:Device Types / Network Device / Device Operation / Flow filter / Device and driver capabilities / VIRTIO-NET-FF-SELECTOR-CAP}

The capability VIRTIO_NET_FF_SELECTOR_CAP lists the supported selectors and the
supported packet header fields for each selector.
\field{cap_specific_data} is in the format \field{struct virtio_net_ff_cap_mask_data}.

\begin{lstlisting}[label={lst:Device Types / Network Device / Device Operation / Flow filter / Device and driver capabilities / VIRTIO-NET-FF-SELECTOR-CAP / virtio-net-ff-selector}]
struct virtio_net_ff_selector {
        u8 type;
        u8 flags;
        u8 reserved[2];
        u8 length;
        u8 reserved1[3];
        u8 mask[];
};

struct virtio_net_ff_cap_mask_data {
        u8 count;
        u8 reserved[7];
        struct virtio_net_ff_selector selectors[];
};

#define VIRTIO_NET_FF_MASK_F_PARTIAL_MASK (1 << 0)
\end{lstlisting}

\field{count} indicates number of valid entries in the \field{selectors} array.
\field{selectors[]} is an array of supported selectors. Within each array entry:
\field{type} specifies the type of the packet header, as defined in table
\ref{table:Device Types / Network Device / Device Operation / Flow filter / Device and driver capabilities / VIRTIO-NET-FF-SELECTOR-CAP / flow filter selector types}. \field{mask} specifies which fields of the
packet header can be matched in a flow filter rule.

Each \field{type} is also listed in table
\ref{table:Device Types / Network Device / Device Operation / Flow filter / Device and driver capabilities / VIRTIO-NET-FF-SELECTOR-CAP / flow filter selector types}. \field{mask} is a byte array
in network byte order. For example, when \field{type} is VIRTIO_NET_FF_MASK_TYPE_IPV6,
the \field{mask} is in the format \hyperref[intro:IPv6-Header-Format]{IPv6 Header Format}.

If partial masking is not set, then all bits in each field have to be either all 0s
to ignore this field or all 1s to match on this field. If partial masking is set,
then any combination of bits can bit set to match on these bits.
For example, when a selector \field{type} is VIRTIO_NET_FF_MASK_TYPE_ETH, if
\field{mask[0-12]} are zero and \field{mask[13-14]} are 0xff (all 1s), it
indicates that matching is only supported for \field{EtherType} of
\field{Ethernet MAC frame}, matching is not supported for
\field{Destination Address} and \field{Source Address}.

The entries in the array \field{selectors} are ordered by
\field{type}, with each \field{type} value only appearing once.

\field{length} is the length of a dynamic array \field{mask} in bytes.
\field{reserved} and \field{reserved1} are reserved and set to zero.

\begin{table}[H]
\caption{Flow filter selector types}
\label{table:Device Types / Network Device / Device Operation / Flow filter / Device and driver capabilities / VIRTIO-NET-FF-SELECTOR-CAP / flow filter selector types}
\begin{tabularx}{\textwidth}{ |l|X|X| }
\hline
Type & Name & Description \\
\hline \hline
0x0 & - & Reserved \\
\hline
0x1 & VIRTIO_NET_FF_MASK_TYPE_ETH & 14 bytes of frame header starting from destination address described in \hyperref[intro:IEEE 802.3-2022]{IEEE 802.3-2022} \\
\hline
0x2 & VIRTIO_NET_FF_MASK_TYPE_IPV4 & 20 bytes of \hyperref[intro:Internet-Header-Format]{IPv4: Internet Header Format} \\
\hline
0x3 & VIRTIO_NET_FF_MASK_TYPE_IPV6 & 40 bytes of \hyperref[intro:IPv6-Header-Format]{IPv6 Header Format} \\
\hline
0x4 & VIRTIO_NET_FF_MASK_TYPE_TCP & 20 bytes of \hyperref[intro:TCP-Header-Format]{TCP Header Format} \\
\hline
0x5 & VIRTIO_NET_FF_MASK_TYPE_UDP & 8 bytes of UDP header described in \hyperref[intro:UDP]{UDP} \\
\hline
0x6 - 0xFF & & Reserved for future \\
\hline
\end{tabularx}
\end{table}

When VIRTIO_NET_FF_MASK_F_PARTIAL_MASK (bit 0) is set, it indicates that
partial masking is supported for all the fields of the selector identified by \field{type}.

For the selector \field{type} VIRTIO_NET_FF_MASK_TYPE_IPV4, if a partial mask is unsupported,
then matching on an individual bit of \field{Flags} in the
\field{IPv4: Internet Header Format} is unsupported. \field{Flags} has to match as a whole
if it is supported.

For the selector \field{type} VIRTIO_NET_FF_MASK_TYPE_IPV4, \field{mask} includes fields
up to the \field{Destination Address}; that is, \field{Options} and
\field{Padding} are excluded.

For the selector \field{type} VIRTIO_NET_FF_MASK_TYPE_IPV6, the \field{Next Header} field
of the \field{mask} corresponds to the \field{Next Header} in the packet
when \field{IPv6 Extension Headers} are not present. When the packet includes
one or more \field{IPv6 Extension Headers}, the \field{Next Header} field of
the \field{mask} corresponds to the \field{Next Header} of the last
\field{IPv6 Extension Header} in the packet.

For the selector \field{type} VIRTIO_NET_FF_MASK_TYPE_TCP, \field{Control bits}
are treated as individual fields for matching; that is, matching individual
\field{Control bits} does not depend on the partial mask support.

\subparagraph{VIRTIO_NET_FF_ACTION_CAP}
\label{par:Device Types / Network Device / Device Operation / Flow filter / Device and driver capabilities / VIRTIO-NET-FF-ACTION-CAP}

The capability VIRTIO_NET_FF_ACTION_CAP lists the supported actions in a rule.
\field{cap_specific_data} is in the format \field{struct virtio_net_ff_cap_actions}.

\begin{lstlisting}
struct virtio_net_ff_actions {
        u8 count;
        u8 reserved[7];
        u8 actions[];
};
\end{lstlisting}

\field{actions} is an array listing all possible actions.
The entries in the array are ordered from the smallest to the largest,
with each supported value appearing exactly once. Each entry can have the
following values:

\begin{table}[H]
\caption{Flow filter rule actions}
\label{table:Device Types / Network Device / Device Operation / Flow filter / Device and driver capabilities / VIRTIO-NET-FF-ACTION-CAP / flow filter rule actions}
\begin{tabularx}{\textwidth}{ |l|X|X| }
\hline
Action & Name & Description \\
\hline \hline
0x0 & - & reserved \\
\hline
0x1 & VIRTIO_NET_FF_ACTION_DROP & Matching packet will be dropped by the device \\
\hline
0x2 & VIRTIO_NET_FF_ACTION_DIRECT_RX_VQ & Matching packet will be directed to a receive queue \\
\hline
0x3 - 0xFF & & Reserved for future \\
\hline
\end{tabularx}
\end{table}

\paragraph{Resource objects}
\label{par:Device Types / Network Device / Device Operation / Flow filter / Resource objects}

\subparagraph{VIRTIO_NET_RESOURCE_OBJ_FF_GROUP}\label{par:Device Types / Network Device / Device Operation / Flow filter / Resource objects / VIRTIO-NET-RESOURCE-OBJ-FF-GROUP}

A flow filter group contains between 0 and \field{rules_limit} rules, as specified by the
capability VIRTIO_NET_FF_RESOURCE_CAP. For the flow filter group object both
\field{resource_obj_specific_data} and
\field{resource_obj_specific_result} are in the format
\field{struct virtio_net_resource_obj_ff_group}.

\begin{lstlisting}
struct virtio_net_resource_obj_ff_group {
        le16 group_priority;
};
\end{lstlisting}

\field{group_priority} specifies the priority for the group. Each group has a
distinct priority. For each incoming packet, the device tries to apply rules
from groups from higher \field{group_priority} value to lower, until either a
rule matches the packet or all groups have been tried.

\subparagraph{VIRTIO_NET_RESOURCE_OBJ_FF_CLASSIFIER}\label{par:Device Types / Network Device / Device Operation / Flow filter / Resource objects / VIRTIO-NET-RESOURCE-OBJ-FF-CLASSIFIER}

A classifier is used to match a flow filter key against a packet. The
classifier defines the desired packet fields to match, and is represented by
the VIRTIO_NET_RESOURCE_OBJ_FF_CLASSIFIER device resource object.

For the flow filter classifier object both \field{resource_obj_specific_data} and
\field{resource_obj_specific_result} are in the format
\field{struct virtio_net_resource_obj_ff_classifier}.

\begin{lstlisting}
struct virtio_net_resource_obj_ff_classifier {
        u8 count;
        u8 reserved[7];
        struct virtio_net_ff_selector selectors[];
};
\end{lstlisting}

A classifier is an array of \field{selectors}. The number of selectors in the
array is indicated by \field{count}. The selector has a type that specifies
the header fields to be matched against, and a mask.
See \ref{lst:Device Types / Network Device / Device Operation / Flow filter / Device and driver capabilities / VIRTIO-NET-FF-SELECTOR-CAP / virtio-net-ff-selector}
for details about selectors.

The first selector is always VIRTIO_NET_FF_MASK_TYPE_ETH. When there are multiple
selectors, a second selector can be either VIRTIO_NET_FF_MASK_TYPE_IPV4
or VIRTIO_NET_FF_MASK_TYPE_IPV6. If the third selector exists, the third
selector can be either VIRTIO_NET_FF_MASK_TYPE_UDP or VIRTIO_NET_FF_MASK_TYPE_TCP.
For example, to match a Ethernet IPv6 UDP packet,
\field{selectors[0].type} is set to VIRTIO_NET_FF_MASK_TYPE_ETH, \field{selectors[1].type}
is set to VIRTIO_NET_FF_MASK_TYPE_IPV6 and \field{selectors[2].type} is
set to VIRTIO_NET_FF_MASK_TYPE_UDP; accordingly, \field{selectors[0].mask[0-13]} is
for Ethernet header fields, \field{selectors[1].mask[0-39]} is set for IPV6 header
and \field{selectors[2].mask[0-7]} is set for UDP header.

When there are multiple selectors, the type of the (N+1)\textsuperscript{th} selector
affects the mask of the (N)\textsuperscript{th} selector. If
\field{count} is 2 or more, all the mask bits within \field{selectors[0]}
corresponding to \field{EtherType} of an Ethernet header are set.

If \field{count} is more than 2:
\begin{itemize}
\item if \field{selector[1].type} is, VIRTIO_NET_FF_MASK_TYPE_IPV4, then, all the mask bits within
\field{selector[1]} for \field{Protocol} is set.
\item if \field{selector[1].type} is, VIRTIO_NET_FF_MASK_TYPE_IPV6, then, all the mask bits within
\field{selector[1]} for \field{Next Header} is set.
\end{itemize}

If for a given packet header field, a subset of bits of a field is to be matched,
and if the partial mask is supported, the flow filter
mask object can specify a mask which has fewer bits set than the packet header
field size. For example, a partial mask for the Ethernet header source mac
address can be of 1-bit for multicast detection instead of 48-bits.

\subparagraph{VIRTIO_NET_RESOURCE_OBJ_FF_RULE}\label{par:Device Types / Network Device / Device Operation / Flow filter / Resource objects / VIRTIO-NET-RESOURCE-OBJ-FF-RULE}

Each flow filter rule resource object comprises a key, a priority, and an action.
For the flow filter rule object,
\field{resource_obj_specific_data} and
\field{resource_obj_specific_result} are in the format
\field{struct virtio_net_resource_obj_ff_rule}.

\begin{lstlisting}
struct virtio_net_resource_obj_ff_rule {
        le32 group_id;
        le32 classifier_id;
        u8 rule_priority;
        u8 key_length; /* length of key in bytes */
        u8 action;
        u8 reserved;
        le16 vq_index;
        u8 reserved1[2];
        u8 keys[][];
};
\end{lstlisting}

\field{group_id} is the resource object ID of the flow filter group to which
this rule belongs. \field{classifier_id} is the resource object ID of the
classifier used to match a packet against the \field{key}.

\field{rule_priority} denotes the priority of the rule within the group
specified by the \field{group_id}.
Rules within the group are applied from the highest to the lowest priority
until a rule matches the packet and an
action is taken. Rules with the same priority can be applied in any order.

\field{reserved} and \field{reserved1} are reserved and set to 0.

\field{keys[][]} is an array of keys to match against packets, using
the classifier specified by \field{classifier_id}. Each entry (key) comprises
a byte array, and they are located one immediately after another.
The size (number of entries) of the array is exactly the same as that of
\field{selectors} in the classifier, or in other words, \field{count}
in the classifier.

\field{key_length} specifies the total length of \field{keys} in bytes.
In other words, it equals the sum total of \field{length} of all
selectors in \field{selectors} in the classifier specified by
\field{classifier_id}.

For example, if a classifier object's \field{selectors[0].type} is
VIRTIO_NET_FF_MASK_TYPE_ETH and \field{selectors[1].type} is
VIRTIO_NET_FF_MASK_TYPE_IPV6,
then selectors[0].length is 14 and selectors[1].length is 40.
Accordingly, the \field{key_length} is set to 54.
This setting indicates that the \field{key} array's length is 54 bytes
comprising a first byte array of 14 bytes for the
Ethernet MAC header in bytes 0-13, immediately followed by 40 bytes for the
IPv6 header in bytes 14-53.

When there are multiple selectors in the classifier object, the key bytes
for (N)\textsuperscript{th} selector are set so that
(N+1)\textsuperscript{th} selector can be matched.

If \field{count} is 2 or more, key bytes of \field{EtherType}
are set according to \hyperref[intro:IEEE 802 Ethertypes]{IEEE 802 Ethertypes}
for VIRTIO_NET_FF_MASK_TYPE_IPV4 or VIRTIO_NET_FF_MASK_TYPE_IPV6 respectively.

If \field{count} is more than 2, when \field{selector[1].type} is
VIRTIO_NET_FF_MASK_TYPE_IPV4 or VIRTIO_NET_FF_MASK_TYPE_IPV6, key
bytes of \field{Protocol} or \field{Next Header} is set as per
\field{Protocol Numbers} defined \hyperref[intro:IANA Protocol Numbers]{IANA Protocol Numbers}
respectively.

\field{action} is the action to take when a packet matches the
\field{key} using the \field{classifier_id}. Supported actions are described in
\ref{table:Device Types / Network Device / Device Operation / Flow filter / Device and driver capabilities / VIRTIO-NET-FF-ACTION-CAP / flow filter rule actions}.

\field{vq_index} specifies a receive virtqueue. When the \field{action} is set
to VIRTIO_NET_FF_ACTION_DIRECT_RX_VQ, and the packet matches the \field{key},
the matching packet is directed to this virtqueue.

Note that at most one action is ever taken for a given packet. If a rule is
applied and an action is taken, the action of other rules is not taken.

\devicenormative{\paragraph}{Flow filter}{Device Types / Network Device / Device Operation / Flow filter}

When the device supports flow filter operations,
\begin{itemize}
\item the device MUST set VIRTIO_NET_FF_RESOURCE_CAP, VIRTIO_NET_FF_SELECTOR_CAP
and VIRTIO_NET_FF_ACTION_CAP capability in the \field{supported_caps} in the
command VIRTIO_ADMIN_CMD_CAP_SUPPORT_QUERY.
\item the device MUST support the administration commands
VIRTIO_ADMIN_CMD_RESOURCE_OBJ_CREATE,
VIRTIO_ADMIN_CMD_RESOURCE_OBJ_MODIFY, VIRTIO_ADMIN_CMD_RESOURCE_OBJ_QUERY,
VIRTIO_ADMIN_CMD_RESOURCE_OBJ_DESTROY for the resource types
VIRTIO_NET_RESOURCE_OBJ_FF_GROUP, VIRTIO_NET_RESOURCE_OBJ_FF_CLASSIFIER and
VIRTIO_NET_RESOURCE_OBJ_FF_RULE.
\end{itemize}

When any of the VIRTIO_NET_FF_RESOURCE_CAP, VIRTIO_NET_FF_SELECTOR_CAP, or
VIRTIO_NET_FF_ACTION_CAP capability is disabled, the device SHOULD set
\field{status} to VIRTIO_ADMIN_STATUS_Q_INVALID_OPCODE for the commands
VIRTIO_ADMIN_CMD_RESOURCE_OBJ_CREATE,
VIRTIO_ADMIN_CMD_RESOURCE_OBJ_MODIFY, VIRTIO_ADMIN_CMD_RESOURCE_OBJ_QUERY,
and VIRTIO_ADMIN_CMD_RESOURCE_OBJ_DESTROY. These commands apply to the resource
\field{type} of VIRTIO_NET_RESOURCE_OBJ_FF_GROUP, VIRTIO_NET_RESOURCE_OBJ_FF_CLASSIFIER, and
VIRTIO_NET_RESOURCE_OBJ_FF_RULE.

The device SHOULD set \field{status} to VIRTIO_ADMIN_STATUS_EINVAL for the
command VIRTIO_ADMIN_CMD_RESOURCE_OBJ_CREATE when the resource \field{type}
is VIRTIO_NET_RESOURCE_OBJ_FF_GROUP, if a flow filter group already exists
with the supplied \field{group_priority}.

The device SHOULD set \field{status} to VIRTIO_ADMIN_STATUS_ENOSPC for the
command VIRTIO_ADMIN_CMD_RESOURCE_OBJ_CREATE when the resource \field{type}
is VIRTIO_NET_RESOURCE_OBJ_FF_GROUP, if the number of flow filter group
objects in the device exceeds the lower of the configured driver
capabilities \field{groups_limit} and \field{rules_per_group_limit}.

The device SHOULD set \field{status} to VIRTIO_ADMIN_STATUS_ENOSPC for the
command VIRTIO_ADMIN_CMD_RESOURCE_OBJ_CREATE when the resource \field{type} is
VIRTIO_NET_RESOURCE_OBJ_FF_CLASSIFIER, if the number of flow filter selector
objects in the device exceeds the configured driver capability
\field{selectors_limit}.

The device SHOULD set \field{status} to VIRTIO_ADMIN_STATUS_EBUSY for the
command VIRTIO_ADMIN_CMD_RESOURCE_OBJ_DESTROY for a flow filter group when
the flow filter group has one or more flow filter rules depending on it.

The device SHOULD set \field{status} to VIRTIO_ADMIN_STATUS_EBUSY for the
command VIRTIO_ADMIN_CMD_RESOURCE_OBJ_DESTROY for a flow filter classifier when
the flow filter classifier has one or more flow filter rules depending on it.

The device SHOULD fail the command VIRTIO_ADMIN_CMD_RESOURCE_OBJ_CREATE for the
flow filter rule resource object if,
\begin{itemize}
\item \field{vq_index} is not a valid receive virtqueue index for
the VIRTIO_NET_FF_ACTION_DIRECT_RX_VQ action,
\item \field{priority} is greater than or equal to
      \field{last_rule_priority},
\item \field{id} is greater than or equal to \field{rules_limit} or
      greater than or equal to \field{rules_per_group_limit}, whichever is lower,
\item the length of \field{keys} and the length of all the mask bytes of
      \field{selectors[].mask} as referred by \field{classifier_id} differs,
\item the supplied \field{action} is not supported in the capability VIRTIO_NET_FF_ACTION_CAP.
\end{itemize}

When the flow filter directs a packet to the virtqueue identified by
\field{vq_index} and if the receive virtqueue is reset, the device
MUST drop such packets.

Upon applying a flow filter rule to a packet, the device MUST STOP any further
application of rules and cease applying any other steering configurations.

For multiple flow filter groups, the device MUST apply the rules from
the group with the highest priority. If any rule from this group is applied,
the device MUST ignore the remaining groups. If none of the rules from the
highest priority group match, the device MUST apply the rules from
the group with the next highest priority, until either a rule matches or
all groups have been attempted.

The device MUST apply the rules within the group from the highest to the
lowest priority until a rule matches the packet, and the device MUST take
the action. If an action is taken, the device MUST not take any other
action for this packet.

The device MAY apply the rules with the same \field{rule_priority} in any
order within the group.

The device MUST process incoming packets in the following order:
\begin{itemize}
\item apply the steering configuration received using control virtqueue
      commands VIRTIO_NET_CTRL_RX, VIRTIO_NET_CTRL_MAC, and
      VIRTIO_NET_CTRL_VLAN.
\item apply flow filter rules if any.
\item if no filter rule is applied, apply the steering configuration
      received using the command VIRTIO_NET_CTRL_MQ_RSS_CONFIG
      or according to automatic receive steering.
\end{itemize}

When processing an incoming packet, if the packet is dropped at any stage, the device
MUST skip further processing.

When the device drops the packet due to the configuration done using the control
virtqueue commands VIRTIO_NET_CTRL_RX or VIRTIO_NET_CTRL_MAC or VIRTIO_NET_CTRL_VLAN,
the device MUST skip flow filter rules for this packet.

When the device performs flow filter match operations and if the operation
result did not have any match in all the groups, the receive packet processing
continues to next level, i.e. to apply configuration done using
VIRTIO_NET_CTRL_MQ_RSS_CONFIG command.

The device MUST support the creation of flow filter classifier objects
using the command VIRTIO_ADMIN_CMD_RESOURCE_OBJ_CREATE with \field{flags}
set to VIRTIO_NET_FF_MASK_F_PARTIAL_MASK;
this support is required even if all the bits of the masks are set for
a field in \field{selectors}, provided that partial masking is supported
for the selectors.

\drivernormative{\paragraph}{Flow filter}{Device Types / Network Device / Device Operation / Flow filter}

The driver MUST enable VIRTIO_NET_FF_RESOURCE_CAP, VIRTIO_NET_FF_SELECTOR_CAP,
and VIRTIO_NET_FF_ACTION_CAP capabilities to use flow filter.

The driver SHOULD NOT remove a flow filter group using the command
VIRTIO_ADMIN_CMD_RESOURCE_OBJ_DESTROY when one or more flow filter rules
depend on that group. The driver SHOULD only destroy the group after
all the associated rules have been destroyed.

The driver SHOULD NOT remove a flow filter classifier using the command
VIRTIO_ADMIN_CMD_RESOURCE_OBJ_DESTROY when one or more flow filter rules
depend on the classifier. The driver SHOULD only destroy the classifier
after all the associated rules have been destroyed.

The driver SHOULD NOT add multiple flow filter rules with the same
\field{rule_priority} within a flow filter group, as these rules MAY match
the same packet. The driver SHOULD assign different \field{rule_priority}
values to different flow filter rules if multiple rules may match a single
packet.

For the command VIRTIO_ADMIN_CMD_RESOURCE_OBJ_CREATE, when creating a resource
of \field{type} VIRTIO_NET_RESOURCE_OBJ_FF_CLASSIFIER, the driver MUST set:
\begin{itemize}
\item \field{selectors[0].type} to VIRTIO_NET_FF_MASK_TYPE_ETH.
\item \field{selectors[1].type} to VIRTIO_NET_FF_MASK_TYPE_IPV4 or
      VIRTIO_NET_FF_MASK_TYPE_IPV6 when \field{count} is more than 1,
\item \field{selectors[2].type} VIRTIO_NET_FF_MASK_TYPE_UDP or
      VIRTIO_NET_FF_MASK_TYPE_TCP when \field{count} is more than 2.
\end{itemize}

For the command VIRTIO_ADMIN_CMD_RESOURCE_OBJ_CREATE, when creating a resource
of \field{type} VIRTIO_NET_RESOURCE_OBJ_FF_CLASSIFIER, the driver MUST set:
\begin{itemize}
\item \field{selectors[0].mask} bytes to all 1s for the \field{EtherType}
       when \field{count} is 2 or more.
\item \field{selectors[1].mask} bytes to all 1s for \field{Protocol} or \field{Next Header}
       when \field{selector[1].type} is VIRTIO_NET_FF_MASK_TYPE_IPV4 or VIRTIO_NET_FF_MASK_TYPE_IPV6,
       and when \field{count} is more than 2.
\end{itemize}

For the command VIRTIO_ADMIN_CMD_RESOURCE_OBJ_CREATE, the resource \field{type}
VIRTIO_NET_RESOURCE_OBJ_FF_RULE, if the corresponding classifier object's
\field{count} is 2 or more, the driver MUST SET the \field{keys} bytes of
\field{EtherType} in accordance with
\hyperref[intro:IEEE 802 Ethertypes]{IEEE 802 Ethertypes}
for either VIRTIO_NET_FF_MASK_TYPE_IPV4 or VIRTIO_NET_FF_MASK_TYPE_IPV6.

For the command VIRTIO_ADMIN_CMD_RESOURCE_OBJ_CREATE, when creating a resource of
\field{type} VIRTIO_NET_RESOURCE_OBJ_FF_RULE, if the corresponding classifier
object's \field{count} is more than 2, and the \field{selector[1].type} is either
VIRTIO_NET_FF_MASK_TYPE_IPV4 or VIRTIO_NET_FF_MASK_TYPE_IPV6, the driver MUST
set the \field{keys} bytes for the \field{Protocol} or \field{Next Header}
according to \hyperref[intro:IANA Protocol Numbers]{IANA Protocol Numbers} respectively.

The driver SHOULD set all the bits for a field in the mask of a selector in both the
capability and the classifier object, unless the VIRTIO_NET_FF_MASK_F_PARTIAL_MASK
is enabled.

\subsubsection{Legacy Interface: Framing Requirements}\label{sec:Device
Types / Network Device / Legacy Interface: Framing Requirements}

When using legacy interfaces, transitional drivers which have not
negotiated VIRTIO_F_ANY_LAYOUT MUST use a single descriptor for the
\field{struct virtio_net_hdr} on both transmit and receive, with the
network data in the following descriptors.

Additionally, when using the control virtqueue (see \ref{sec:Device
Types / Network Device / Device Operation / Control Virtqueue})
, transitional drivers which have not
negotiated VIRTIO_F_ANY_LAYOUT MUST:
\begin{itemize}
\item for all commands, use a single 2-byte descriptor including the first two
fields: \field{class} and \field{command}
\item for all commands except VIRTIO_NET_CTRL_MAC_TABLE_SET
use a single descriptor including command-specific-data
with no padding.
\item for the VIRTIO_NET_CTRL_MAC_TABLE_SET command use exactly
two descriptors including command-specific-data with no padding:
the first of these descriptors MUST include the
virtio_net_ctrl_mac table structure for the unicast addresses with no padding,
the second of these descriptors MUST include the
virtio_net_ctrl_mac table structure for the multicast addresses
with no padding.
\item for all commands, use a single 1-byte descriptor for the
\field{ack} field
\end{itemize}

See \ref{sec:Basic
Facilities of a Virtio Device / Virtqueues / Message Framing}.

\section{Network Device}\label{sec:Device Types / Network Device}

The virtio network device is a virtual network interface controller.
It consists of a virtual Ethernet link which connects the device
to the Ethernet network. The device has transmit and receive
queues. The driver adds empty buffers to the receive virtqueue.
The device receives incoming packets from the link; the device
places these incoming packets in the receive virtqueue buffers.
The driver adds outgoing packets to the transmit virtqueue. The device
removes these packets from the transmit virtqueue and sends them to
the link. The device may have a control virtqueue. The driver
uses the control virtqueue to dynamically manipulate various
features of the initialized device.

\subsection{Device ID}\label{sec:Device Types / Network Device / Device ID}

 1

\subsection{Virtqueues}\label{sec:Device Types / Network Device / Virtqueues}

\begin{description}
\item[0] receiveq1
\item[1] transmitq1
\item[\ldots]
\item[2(N-1)] receiveqN
\item[2(N-1)+1] transmitqN
\item[2N] controlq
\end{description}

 N=1 if neither VIRTIO_NET_F_MQ nor VIRTIO_NET_F_RSS are negotiated, otherwise N is set by
 \field{max_virtqueue_pairs}.

controlq is optional; it only exists if VIRTIO_NET_F_CTRL_VQ is
negotiated.

\subsection{Feature bits}\label{sec:Device Types / Network Device / Feature bits}

\begin{description}
\item[VIRTIO_NET_F_CSUM (0)] Device handles packets with partial checksum offload.

\item[VIRTIO_NET_F_GUEST_CSUM (1)] Driver handles packets with partial checksum.

\item[VIRTIO_NET_F_CTRL_GUEST_OFFLOADS (2)] Control channel offloads
        reconfiguration support.

\item[VIRTIO_NET_F_MTU(3)] Device maximum MTU reporting is supported. If
    offered by the device, device advises driver about the value of
    its maximum MTU. If negotiated, the driver uses \field{mtu} as
    the maximum MTU value.

\item[VIRTIO_NET_F_MAC (5)] Device has given MAC address.

\item[VIRTIO_NET_F_GUEST_TSO4 (7)] Driver can receive TSOv4.

\item[VIRTIO_NET_F_GUEST_TSO6 (8)] Driver can receive TSOv6.

\item[VIRTIO_NET_F_GUEST_ECN (9)] Driver can receive TSO with ECN.

\item[VIRTIO_NET_F_GUEST_UFO (10)] Driver can receive UFO.

\item[VIRTIO_NET_F_HOST_TSO4 (11)] Device can receive TSOv4.

\item[VIRTIO_NET_F_HOST_TSO6 (12)] Device can receive TSOv6.

\item[VIRTIO_NET_F_HOST_ECN (13)] Device can receive TSO with ECN.

\item[VIRTIO_NET_F_HOST_UFO (14)] Device can receive UFO.

\item[VIRTIO_NET_F_MRG_RXBUF (15)] Driver can merge receive buffers.

\item[VIRTIO_NET_F_STATUS (16)] Configuration status field is
    available.

\item[VIRTIO_NET_F_CTRL_VQ (17)] Control channel is available.

\item[VIRTIO_NET_F_CTRL_RX (18)] Control channel RX mode support.

\item[VIRTIO_NET_F_CTRL_VLAN (19)] Control channel VLAN filtering.

\item[VIRTIO_NET_F_CTRL_RX_EXTRA (20)]	Control channel RX extra mode support.

\item[VIRTIO_NET_F_GUEST_ANNOUNCE(21)] Driver can send gratuitous
    packets.

\item[VIRTIO_NET_F_MQ(22)] Device supports multiqueue with automatic
    receive steering.

\item[VIRTIO_NET_F_CTRL_MAC_ADDR(23)] Set MAC address through control
    channel.

\item[VIRTIO_NET_F_DEVICE_STATS(50)] Device can provide device-level statistics
    to the driver through the control virtqueue.

\item[VIRTIO_NET_F_HASH_TUNNEL(51)] Device supports inner header hash for encapsulated packets.

\item[VIRTIO_NET_F_VQ_NOTF_COAL(52)] Device supports virtqueue notification coalescing.

\item[VIRTIO_NET_F_NOTF_COAL(53)] Device supports notifications coalescing.

\item[VIRTIO_NET_F_GUEST_USO4 (54)] Driver can receive USOv4 packets.

\item[VIRTIO_NET_F_GUEST_USO6 (55)] Driver can receive USOv6 packets.

\item[VIRTIO_NET_F_HOST_USO (56)] Device can receive USO packets. Unlike UFO
 (fragmenting the packet) the USO splits large UDP packet
 to several segments when each of these smaller packets has UDP header.

\item[VIRTIO_NET_F_HASH_REPORT(57)] Device can report per-packet hash
    value and a type of calculated hash.

\item[VIRTIO_NET_F_GUEST_HDRLEN(59)] Driver can provide the exact \field{hdr_len}
    value. Device benefits from knowing the exact header length.

\item[VIRTIO_NET_F_RSS(60)] Device supports RSS (receive-side scaling)
    with Toeplitz hash calculation and configurable hash
    parameters for receive steering.

\item[VIRTIO_NET_F_RSC_EXT(61)] Device can process duplicated ACKs
    and report number of coalesced segments and duplicated ACKs.

\item[VIRTIO_NET_F_STANDBY(62)] Device may act as a standby for a primary
    device with the same MAC address.

\item[VIRTIO_NET_F_SPEED_DUPLEX(63)] Device reports speed and duplex.

\item[VIRTIO_NET_F_RSS_CONTEXT(64)] Device supports multiple RSS contexts.

\item[VIRTIO_NET_F_GUEST_UDP_TUNNEL_GSO (65)] Driver can receive GSO packets
  carried by a UDP tunnel.

\item[VIRTIO_NET_F_GUEST_UDP_TUNNEL_GSO_CSUM (66)] Driver handles packets
  carried by a UDP tunnel with partial csum for the outer header.

\item[VIRTIO_NET_F_HOST_UDP_TUNNEL_GSO (67)] Device can receive GSO packets
  carried by a UDP tunnel.

\item[VIRTIO_NET_F_HOST_UDP_TUNNEL_GSO_CSUM (68)] Device handles packets
  carried by a UDP tunnel with partial csum for the outer header.
\end{description}

\subsubsection{Feature bit requirements}\label{sec:Device Types / Network Device / Feature bits / Feature bit requirements}

Some networking feature bits require other networking feature bits
(see \ref{drivernormative:Basic Facilities of a Virtio Device / Feature Bits}):

\begin{description}
\item[VIRTIO_NET_F_GUEST_TSO4] Requires VIRTIO_NET_F_GUEST_CSUM.
\item[VIRTIO_NET_F_GUEST_TSO6] Requires VIRTIO_NET_F_GUEST_CSUM.
\item[VIRTIO_NET_F_GUEST_ECN] Requires VIRTIO_NET_F_GUEST_TSO4 or VIRTIO_NET_F_GUEST_TSO6.
\item[VIRTIO_NET_F_GUEST_UFO] Requires VIRTIO_NET_F_GUEST_CSUM.
\item[VIRTIO_NET_F_GUEST_USO4] Requires VIRTIO_NET_F_GUEST_CSUM.
\item[VIRTIO_NET_F_GUEST_USO6] Requires VIRTIO_NET_F_GUEST_CSUM.
\item[VIRTIO_NET_F_GUEST_UDP_TUNNEL_GSO] Requires VIRTIO_NET_F_GUEST_TSO4, VIRTIO_NET_F_GUEST_TSO6,
   VIRTIO_NET_F_GUEST_USO4 and VIRTIO_NET_F_GUEST_USO6.
\item[VIRTIO_NET_F_GUEST_UDP_TUNNEL_GSO_CSUM] Requires VIRTIO_NET_F_GUEST_UDP_TUNNEL_GSO

\item[VIRTIO_NET_F_HOST_TSO4] Requires VIRTIO_NET_F_CSUM.
\item[VIRTIO_NET_F_HOST_TSO6] Requires VIRTIO_NET_F_CSUM.
\item[VIRTIO_NET_F_HOST_ECN] Requires VIRTIO_NET_F_HOST_TSO4 or VIRTIO_NET_F_HOST_TSO6.
\item[VIRTIO_NET_F_HOST_UFO] Requires VIRTIO_NET_F_CSUM.
\item[VIRTIO_NET_F_HOST_USO] Requires VIRTIO_NET_F_CSUM.
\item[VIRTIO_NET_F_HOST_UDP_TUNNEL_GSO] Requires VIRTIO_NET_F_HOST_TSO4, VIRTIO_NET_F_HOST_TSO6
   and VIRTIO_NET_F_HOST_USO.
\item[VIRTIO_NET_F_HOST_UDP_TUNNEL_GSO_CSUM] Requires VIRTIO_NET_F_HOST_UDP_TUNNEL_GSO

\item[VIRTIO_NET_F_CTRL_RX] Requires VIRTIO_NET_F_CTRL_VQ.
\item[VIRTIO_NET_F_CTRL_VLAN] Requires VIRTIO_NET_F_CTRL_VQ.
\item[VIRTIO_NET_F_GUEST_ANNOUNCE] Requires VIRTIO_NET_F_CTRL_VQ.
\item[VIRTIO_NET_F_MQ] Requires VIRTIO_NET_F_CTRL_VQ.
\item[VIRTIO_NET_F_CTRL_MAC_ADDR] Requires VIRTIO_NET_F_CTRL_VQ.
\item[VIRTIO_NET_F_NOTF_COAL] Requires VIRTIO_NET_F_CTRL_VQ.
\item[VIRTIO_NET_F_RSC_EXT] Requires VIRTIO_NET_F_HOST_TSO4 or VIRTIO_NET_F_HOST_TSO6.
\item[VIRTIO_NET_F_RSS] Requires VIRTIO_NET_F_CTRL_VQ.
\item[VIRTIO_NET_F_VQ_NOTF_COAL] Requires VIRTIO_NET_F_CTRL_VQ.
\item[VIRTIO_NET_F_HASH_TUNNEL] Requires VIRTIO_NET_F_CTRL_VQ along with VIRTIO_NET_F_RSS or VIRTIO_NET_F_HASH_REPORT.
\item[VIRTIO_NET_F_RSS_CONTEXT] Requires VIRTIO_NET_F_CTRL_VQ and VIRTIO_NET_F_RSS.
\end{description}

\begin{note}
The dependency between UDP_TUNNEL_GSO_CSUM and UDP_TUNNEL_GSO is intentionally
in the opposite direction with respect to the plain GSO features and the plain
checksum offload because UDP tunnel checksum offload gives very little gain
for non GSO packets and is quite complex to implement in H/W.
\end{note}

\subsubsection{Legacy Interface: Feature bits}\label{sec:Device Types / Network Device / Feature bits / Legacy Interface: Feature bits}
\begin{description}
\item[VIRTIO_NET_F_GSO (6)] Device handles packets with any GSO type. This was supposed to indicate segmentation offload support, but
upon further investigation it became clear that multiple bits were needed.
\item[VIRTIO_NET_F_GUEST_RSC4 (41)] Device coalesces TCPIP v4 packets. This was implemented by hypervisor patch for certification
purposes and current Windows driver depends on it. It will not function if virtio-net device reports this feature.
\item[VIRTIO_NET_F_GUEST_RSC6 (42)] Device coalesces TCPIP v6 packets. Similar to VIRTIO_NET_F_GUEST_RSC4.
\end{description}

\subsection{Device configuration layout}\label{sec:Device Types / Network Device / Device configuration layout}
\label{sec:Device Types / Block Device / Feature bits / Device configuration layout}

The network device has the following device configuration layout.
All of the device configuration fields are read-only for the driver.

\begin{lstlisting}
struct virtio_net_config {
        u8 mac[6];
        le16 status;
        le16 max_virtqueue_pairs;
        le16 mtu;
        le32 speed;
        u8 duplex;
        u8 rss_max_key_size;
        le16 rss_max_indirection_table_length;
        le32 supported_hash_types;
        le32 supported_tunnel_types;
};
\end{lstlisting}

The \field{mac} address field always exists (although it is only
valid if VIRTIO_NET_F_MAC is set).

The \field{status} only exists if VIRTIO_NET_F_STATUS is set.
Two bits are currently defined for the status field: VIRTIO_NET_S_LINK_UP
and VIRTIO_NET_S_ANNOUNCE.

\begin{lstlisting}
#define VIRTIO_NET_S_LINK_UP     1
#define VIRTIO_NET_S_ANNOUNCE    2
\end{lstlisting}

The following field, \field{max_virtqueue_pairs} only exists if
VIRTIO_NET_F_MQ or VIRTIO_NET_F_RSS is set. This field specifies the maximum number
of each of transmit and receive virtqueues (receiveq1\ldots receiveqN
and transmitq1\ldots transmitqN respectively) that can be configured once at least one of these features
is negotiated.

The following field, \field{mtu} only exists if VIRTIO_NET_F_MTU
is set. This field specifies the maximum MTU for the driver to
use.

The following two fields, \field{speed} and \field{duplex}, only
exist if VIRTIO_NET_F_SPEED_DUPLEX is set.

\field{speed} contains the device speed, in units of 1 MBit per
second, 0 to 0x7fffffff, or 0xffffffff for unknown speed.

\field{duplex} has the values of 0x01 for full duplex, 0x00 for
half duplex and 0xff for unknown duplex state.

Both \field{speed} and \field{duplex} can change, thus the driver
is expected to re-read these values after receiving a
configuration change notification.

The following field, \field{rss_max_key_size} only exists if VIRTIO_NET_F_RSS or VIRTIO_NET_F_HASH_REPORT is set.
It specifies the maximum supported length of RSS key in bytes.

The following field, \field{rss_max_indirection_table_length} only exists if VIRTIO_NET_F_RSS is set.
It specifies the maximum number of 16-bit entries in RSS indirection table.

The next field, \field{supported_hash_types} only exists if the device supports hash calculation,
i.e. if VIRTIO_NET_F_RSS or VIRTIO_NET_F_HASH_REPORT is set.

Field \field{supported_hash_types} contains the bitmask of supported hash types.
See \ref{sec:Device Types / Network Device / Device Operation / Processing of Incoming Packets / Hash calculation for incoming packets / Supported/enabled hash types} for details of supported hash types.

Field \field{supported_tunnel_types} only exists if the device supports inner header hash, i.e. if VIRTIO_NET_F_HASH_TUNNEL is set.

Field \field{supported_tunnel_types} contains the bitmask of encapsulation types supported by the device for inner header hash.
Encapsulation types are defined in \ref{sec:Device Types / Network Device / Device Operation / Processing of Incoming Packets /
Hash calculation for incoming packets / Encapsulation types supported/enabled for inner header hash}.

\devicenormative{\subsubsection}{Device configuration layout}{Device Types / Network Device / Device configuration layout}

The device MUST set \field{max_virtqueue_pairs} to between 1 and 0x8000 inclusive,
if it offers VIRTIO_NET_F_MQ.

The device MUST set \field{mtu} to between 68 and 65535 inclusive,
if it offers VIRTIO_NET_F_MTU.

The device SHOULD set \field{mtu} to at least 1280, if it offers
VIRTIO_NET_F_MTU.

The device MUST NOT modify \field{mtu} once it has been set.

The device MUST NOT pass received packets that exceed \field{mtu} (plus low
level ethernet header length) size with \field{gso_type} NONE or ECN
after VIRTIO_NET_F_MTU has been successfully negotiated.

The device MUST forward transmitted packets of up to \field{mtu} (plus low
level ethernet header length) size with \field{gso_type} NONE or ECN, and do
so without fragmentation, after VIRTIO_NET_F_MTU has been successfully
negotiated.

The device MUST set \field{rss_max_key_size} to at least 40, if it offers
VIRTIO_NET_F_RSS or VIRTIO_NET_F_HASH_REPORT.

The device MUST set \field{rss_max_indirection_table_length} to at least 128, if it offers
VIRTIO_NET_F_RSS.

If the driver negotiates the VIRTIO_NET_F_STANDBY feature, the device MAY act
as a standby device for a primary device with the same MAC address.

If VIRTIO_NET_F_SPEED_DUPLEX has been negotiated, \field{speed}
MUST contain the device speed, in units of 1 MBit per second, 0 to
0x7ffffffff, or 0xfffffffff for unknown.

If VIRTIO_NET_F_SPEED_DUPLEX has been negotiated, \field{duplex}
MUST have the values of 0x00 for full duplex, 0x01 for half
duplex, or 0xff for unknown.

If VIRTIO_NET_F_SPEED_DUPLEX and VIRTIO_NET_F_STATUS have both
been negotiated, the device SHOULD NOT change the \field{speed} and
\field{duplex} fields as long as VIRTIO_NET_S_LINK_UP is set in
the \field{status}.

The device SHOULD NOT offer VIRTIO_NET_F_HASH_REPORT if it
does not offer VIRTIO_NET_F_CTRL_VQ.

The device SHOULD NOT offer VIRTIO_NET_F_CTRL_RX_EXTRA if it
does not offer VIRTIO_NET_F_CTRL_VQ.

\drivernormative{\subsubsection}{Device configuration layout}{Device Types / Network Device / Device configuration layout}

The driver MUST NOT write to any of the device configuration fields.

A driver SHOULD negotiate VIRTIO_NET_F_MAC if the device offers it.
If the driver negotiates the VIRTIO_NET_F_MAC feature, the driver MUST set
the physical address of the NIC to \field{mac}.  Otherwise, it SHOULD
use a locally-administered MAC address (see \hyperref[intro:IEEE 802]{IEEE 802},
``9.2 48-bit universal LAN MAC addresses'').

If the driver does not negotiate the VIRTIO_NET_F_STATUS feature, it SHOULD
assume the link is active, otherwise it SHOULD read the link status from
the bottom bit of \field{status}.

A driver SHOULD negotiate VIRTIO_NET_F_MTU if the device offers it.

If the driver negotiates VIRTIO_NET_F_MTU, it MUST supply enough receive
buffers to receive at least one receive packet of size \field{mtu} (plus low
level ethernet header length) with \field{gso_type} NONE or ECN.

If the driver negotiates VIRTIO_NET_F_MTU, it MUST NOT transmit packets of
size exceeding the value of \field{mtu} (plus low level ethernet header length)
with \field{gso_type} NONE or ECN.

A driver SHOULD negotiate the VIRTIO_NET_F_STANDBY feature if the device offers it.

If VIRTIO_NET_F_SPEED_DUPLEX has been negotiated,
the driver MUST treat any value of \field{speed} above
0x7fffffff as well as any value of \field{duplex} not
matching 0x00 or 0x01 as an unknown value.

If VIRTIO_NET_F_SPEED_DUPLEX has been negotiated, the driver
SHOULD re-read \field{speed} and \field{duplex} after a
configuration change notification.

A driver SHOULD NOT negotiate VIRTIO_NET_F_HASH_REPORT if it
does not negotiate VIRTIO_NET_F_CTRL_VQ.

A driver SHOULD NOT negotiate VIRTIO_NET_F_CTRL_RX_EXTRA if it
does not negotiate VIRTIO_NET_F_CTRL_VQ.

\subsubsection{Legacy Interface: Device configuration layout}\label{sec:Device Types / Network Device / Device configuration layout / Legacy Interface: Device configuration layout}
\label{sec:Device Types / Block Device / Feature bits / Device configuration layout / Legacy Interface: Device configuration layout}
When using the legacy interface, transitional devices and drivers
MUST format \field{status} and
\field{max_virtqueue_pairs} in struct virtio_net_config
according to the native endian of the guest rather than
(necessarily when not using the legacy interface) little-endian.

When using the legacy interface, \field{mac} is driver-writable
which provided a way for drivers to update the MAC without
negotiating VIRTIO_NET_F_CTRL_MAC_ADDR.

\subsection{Device Initialization}\label{sec:Device Types / Network Device / Device Initialization}

A driver would perform a typical initialization routine like so:

\begin{enumerate}
\item Identify and initialize the receive and
  transmission virtqueues, up to N of each kind. If
  VIRTIO_NET_F_MQ feature bit is negotiated,
  N=\field{max_virtqueue_pairs}, otherwise identify N=1.

\item If the VIRTIO_NET_F_CTRL_VQ feature bit is negotiated,
  identify the control virtqueue.

\item Fill the receive queues with buffers: see \ref{sec:Device Types / Network Device / Device Operation / Setting Up Receive Buffers}.

\item Even with VIRTIO_NET_F_MQ, only receiveq1, transmitq1 and
  controlq are used by default.  The driver would send the
  VIRTIO_NET_CTRL_MQ_VQ_PAIRS_SET command specifying the
  number of the transmit and receive queues to use.

\item If the VIRTIO_NET_F_MAC feature bit is set, the configuration
  space \field{mac} entry indicates the ``physical'' address of the
  device, otherwise the driver would typically generate a random
  local MAC address.

\item If the VIRTIO_NET_F_STATUS feature bit is negotiated, the link
  status comes from the bottom bit of \field{status}.
  Otherwise, the driver assumes it's active.

\item A performant driver would indicate that it will generate checksumless
  packets by negotiating the VIRTIO_NET_F_CSUM feature.

\item If that feature is negotiated, a driver can use TCP segmentation or UDP
  segmentation/fragmentation offload by negotiating the VIRTIO_NET_F_HOST_TSO4 (IPv4
  TCP), VIRTIO_NET_F_HOST_TSO6 (IPv6 TCP), VIRTIO_NET_F_HOST_UFO
  (UDP fragmentation) and VIRTIO_NET_F_HOST_USO (UDP segmentation) features.

\item If the VIRTIO_NET_F_HOST_TSO6, VIRTIO_NET_F_HOST_TSO4 and VIRTIO_NET_F_HOST_USO
  segmentation features are negotiated, a driver can
  use TCP segmentation or UDP segmentation on top of UDP encapsulation
  offload, when the outer header does not require checksumming - e.g.
  the outer UDP checksum is zero - by negotiating the
  VIRTIO_NET_F_HOST_UDP_TUNNEL_GSO feature.
  GSO over UDP tunnels packets carry two sets of headers: the outer ones
  and the inner ones. The outer transport protocol is UDP, the inner
  could be either TCP or UDP. Only a single level of encapsulation
  offload is supported.

\item If VIRTIO_NET_F_HOST_UDP_TUNNEL_GSO is negotiated, a driver can
  additionally use TCP segmentation or UDP segmentation on top of UDP
  encapsulation with the outer header requiring checksum offload,
  negotiating the VIRTIO_NET_F_HOST_UDP_TUNNEL_GSO_CSUM feature.

\item The converse features are also available: a driver can save
  the virtual device some work by negotiating these features.\note{For example, a network packet transported between two guests on
the same system might not need checksumming at all, nor segmentation,
if both guests are amenable.}
   The VIRTIO_NET_F_GUEST_CSUM feature indicates that partially
  checksummed packets can be received, and if it can do that then
  the VIRTIO_NET_F_GUEST_TSO4, VIRTIO_NET_F_GUEST_TSO6,
  VIRTIO_NET_F_GUEST_UFO, VIRTIO_NET_F_GUEST_ECN, VIRTIO_NET_F_GUEST_USO4,
  VIRTIO_NET_F_GUEST_USO6 VIRTIO_NET_F_GUEST_UDP_TUNNEL_GSO and
  VIRTIO_NET_F_GUEST_UDP_TUNNEL_GSO_CSUM are the input equivalents of
  the features described above.
  See \ref{sec:Device Types / Network Device / Device Operation /
Setting Up Receive Buffers}~\nameref{sec:Device Types / Network
Device / Device Operation / Setting Up Receive Buffers} and
\ref{sec:Device Types / Network Device / Device Operation /
Processing of Incoming Packets}~\nameref{sec:Device Types /
Network Device / Device Operation / Processing of Incoming Packets} below.
\end{enumerate}

A truly minimal driver would only accept VIRTIO_NET_F_MAC and ignore
everything else.

\subsection{Device and driver capabilities}\label{sec:Device Types / Network Device / Device and driver capabilities}

The network device has the following capabilities.

\begin{tabularx}{\textwidth}{ |l||l|X| }
\hline
Identifier & Name & Description \\
\hline \hline
0x0800 & \hyperref[par:Device Types / Network Device / Device Operation / Flow filter / Device and driver capabilities / VIRTIO-NET-FF-RESOURCE-CAP]{VIRTIO_NET_FF_RESOURCE_CAP} & Flow filter resource capability \\
\hline
0x0801 & \hyperref[par:Device Types / Network Device / Device Operation / Flow filter / Device and driver capabilities / VIRTIO-NET-FF-SELECTOR-CAP]{VIRTIO_NET_FF_SELECTOR_CAP} & Flow filter classifier capability \\
\hline
0x0802 & \hyperref[par:Device Types / Network Device / Device Operation / Flow filter / Device and driver capabilities / VIRTIO-NET-FF-ACTION-CAP]{VIRTIO_NET_FF_ACTION_CAP} & Flow filter action capability \\
\hline
\end{tabularx}

\subsection{Device resource objects}\label{sec:Device Types / Network Device / Device resource objects}

The network device has the following resource objects.

\begin{tabularx}{\textwidth}{ |l||l|X| }
\hline
type & Name & Description \\
\hline \hline
0x0200 & \hyperref[par:Device Types / Network Device / Device Operation / Flow filter / Resource objects / VIRTIO-NET-RESOURCE-OBJ-FF-GROUP]{VIRTIO_NET_RESOURCE_OBJ_FF_GROUP} & Flow filter group resource object \\
\hline
0x0201 & \hyperref[par:Device Types / Network Device / Device Operation / Flow filter / Resource objects / VIRTIO-NET-RESOURCE-OBJ-FF-CLASSIFIER]{VIRTIO_NET_RESOURCE_OBJ_FF_CLASSIFIER} & Flow filter mask object \\
\hline
0x0202 & \hyperref[par:Device Types / Network Device / Device Operation / Flow filter / Resource objects / VIRTIO-NET-RESOURCE-OBJ-FF-RULE]{VIRTIO_NET_RESOURCE_OBJ_FF_RULE} & Flow filter rule object \\
\hline
\end{tabularx}

\subsection{Device parts}\label{sec:Device Types / Network Device / Device parts}

Network device parts represent the configuration done by the driver using control
virtqueue commands. Network device part is in the format of
\field{struct virtio_dev_part}.

\begin{tabularx}{\textwidth}{ |l||l|X| }
\hline
Type & Name & Description \\
\hline \hline
0x200 & VIRTIO_NET_DEV_PART_CVQ_CFG_PART & Represents device configuration done through a control virtqueue command, see \ref{sec:Device Types / Network Device / Device parts / VIRTIO-NET-DEV-PART-CVQ-CFG-PART} \\
\hline
0x201 - 0x5FF & - & reserved for future \\
\hline
\hline
\end{tabularx}

\subsubsection{VIRTIO_NET_DEV_PART_CVQ_CFG_PART}\label{sec:Device Types / Network Device / Device parts / VIRTIO-NET-DEV-PART-CVQ-CFG-PART}

For VIRTIO_NET_DEV_PART_CVQ_CFG_PART, \field{part_type} is set to 0x200. The
VIRTIO_NET_DEV_PART_CVQ_CFG_PART part indicates configuration performed by the
driver using a control virtqueue command.

\begin{lstlisting}
struct virtio_net_dev_part_cvq_selector {
        u8 class;
        u8 command;
        u8 reserved[6];
};
\end{lstlisting}

There is one device part of type VIRTIO_NET_DEV_PART_CVQ_CFG_PART for each
individual configuration. Each part is identified by a unique selector value.
The selector, \field{device_type_raw}, is in the format
\field{struct virtio_net_dev_part_cvq_selector}.

The selector consists of two fields: \field{class} and \field{command}. These
fields correspond to the \field{class} and \field{command} defined in
\field{struct virtio_net_ctrl}, as described in the relevant sections of
\ref{sec:Device Types / Network Device / Device Operation / Control Virtqueue}.

The value corresponding to each part’s selector follows the same format as the
respective \field{command-specific-data} described in the relevant sections of
\ref{sec:Device Types / Network Device / Device Operation / Control Virtqueue}.

For example, when the \field{class} is VIRTIO_NET_CTRL_MAC, the \field{command}
can be either VIRTIO_NET_CTRL_MAC_TABLE_SET or VIRTIO_NET_CTRL_MAC_ADDR_SET;
when \field{command} is set to VIRTIO_NET_CTRL_MAC_TABLE_SET, \field{value}
is in the format of \field{struct virtio_net_ctrl_mac}.

Supported selectors are listed in the table:

\begin{tabularx}{\textwidth}{ |l|X| }
\hline
Class selector & Command selector \\
\hline \hline
VIRTIO_NET_CTRL_RX & VIRTIO_NET_CTRL_RX_PROMISC \\
\hline
VIRTIO_NET_CTRL_RX & VIRTIO_NET_CTRL_RX_ALLMULTI \\
\hline
VIRTIO_NET_CTRL_RX & VIRTIO_NET_CTRL_RX_ALLUNI \\
\hline
VIRTIO_NET_CTRL_RX & VIRTIO_NET_CTRL_RX_NOMULTI \\
\hline
VIRTIO_NET_CTRL_RX & VIRTIO_NET_CTRL_RX_NOUNI \\
\hline
VIRTIO_NET_CTRL_RX & VIRTIO_NET_CTRL_RX_NOBCAST \\
\hline
VIRTIO_NET_CTRL_MAC & VIRTIO_NET_CTRL_MAC_TABLE_SET \\
\hline
VIRTIO_NET_CTRL_MAC & VIRTIO_NET_CTRL_MAC_ADDR_SET \\
\hline
VIRTIO_NET_CTRL_VLAN & VIRTIO_NET_CTRL_VLAN_ADD \\
\hline
VIRTIO_NET_CTRL_ANNOUNCE & VIRTIO_NET_CTRL_ANNOUNCE_ACK \\
\hline
VIRTIO_NET_CTRL_MQ & VIRTIO_NET_CTRL_MQ_VQ_PAIRS_SET \\
\hline
VIRTIO_NET_CTRL_MQ & VIRTIO_NET_CTRL_MQ_RSS_CONFIG \\
\hline
VIRTIO_NET_CTRL_MQ & VIRTIO_NET_CTRL_MQ_HASH_CONFIG \\
\hline
\hline
\end{tabularx}

For command selector VIRTIO_NET_CTRL_VLAN_ADD, device part consists of a whole
VLAN table.

\field{reserved} is reserved and set to zero.

\subsection{Device Operation}\label{sec:Device Types / Network Device / Device Operation}

Packets are transmitted by placing them in the
transmitq1\ldots transmitqN, and buffers for incoming packets are
placed in the receiveq1\ldots receiveqN. In each case, the packet
itself is preceded by a header:

\begin{lstlisting}
struct virtio_net_hdr {
#define VIRTIO_NET_HDR_F_NEEDS_CSUM    1
#define VIRTIO_NET_HDR_F_DATA_VALID    2
#define VIRTIO_NET_HDR_F_RSC_INFO      4
#define VIRTIO_NET_HDR_F_UDP_TUNNEL_CSUM 8
        u8 flags;
#define VIRTIO_NET_HDR_GSO_NONE        0
#define VIRTIO_NET_HDR_GSO_TCPV4       1
#define VIRTIO_NET_HDR_GSO_UDP         3
#define VIRTIO_NET_HDR_GSO_TCPV6       4
#define VIRTIO_NET_HDR_GSO_UDP_L4      5
#define VIRTIO_NET_HDR_GSO_UDP_TUNNEL_IPV4 0x20
#define VIRTIO_NET_HDR_GSO_UDP_TUNNEL_IPV6 0x40
#define VIRTIO_NET_HDR_GSO_ECN      0x80
        u8 gso_type;
        le16 hdr_len;
        le16 gso_size;
        le16 csum_start;
        le16 csum_offset;
        le16 num_buffers;
        le32 hash_value;        (Only if VIRTIO_NET_F_HASH_REPORT negotiated)
        le16 hash_report;       (Only if VIRTIO_NET_F_HASH_REPORT negotiated)
        le16 padding_reserved;  (Only if VIRTIO_NET_F_HASH_REPORT negotiated)
        le16 outer_th_offset    (Only if VIRTIO_NET_F_HOST_UDP_TUNNEL_GSO or VIRTIO_NET_F_GUEST_UDP_TUNNEL_GSO negotiated)
        le16 inner_nh_offset;   (Only if VIRTIO_NET_F_HOST_UDP_TUNNEL_GSO or VIRTIO_NET_F_GUEST_UDP_TUNNEL_GSO negotiated)
};
\end{lstlisting}

The controlq is used to control various device features described further in
section \ref{sec:Device Types / Network Device / Device Operation / Control Virtqueue}.

\subsubsection{Legacy Interface: Device Operation}\label{sec:Device Types / Network Device / Device Operation / Legacy Interface: Device Operation}
When using the legacy interface, transitional devices and drivers
MUST format the fields in \field{struct virtio_net_hdr}
according to the native endian of the guest rather than
(necessarily when not using the legacy interface) little-endian.

The legacy driver only presented \field{num_buffers} in the \field{struct virtio_net_hdr}
when VIRTIO_NET_F_MRG_RXBUF was negotiated; without that feature the
structure was 2 bytes shorter.

When using the legacy interface, the driver SHOULD ignore the
used length for the transmit queues
and the controlq queue.
\begin{note}
Historically, some devices put
the total descriptor length there, even though no data was
actually written.
\end{note}

\subsubsection{Packet Transmission}\label{sec:Device Types / Network Device / Device Operation / Packet Transmission}

Transmitting a single packet is simple, but varies depending on
the different features the driver negotiated.

\begin{enumerate}
\item The driver can send a completely checksummed packet.  In this case,
  \field{flags} will be zero, and \field{gso_type} will be VIRTIO_NET_HDR_GSO_NONE.

\item If the driver negotiated VIRTIO_NET_F_CSUM, it can skip
  checksumming the packet:
  \begin{itemize}
  \item \field{flags} has the VIRTIO_NET_HDR_F_NEEDS_CSUM set,

  \item \field{csum_start} is set to the offset within the packet to begin checksumming,
    and

  \item \field{csum_offset} indicates how many bytes after the csum_start the
    new (16 bit ones' complement) checksum is placed by the device.

  \item The TCP checksum field in the packet is set to the sum
    of the TCP pseudo header, so that replacing it by the ones'
    complement checksum of the TCP header and body will give the
    correct result.
  \end{itemize}

\begin{note}
For example, consider a partially checksummed TCP (IPv4) packet.
It will have a 14 byte ethernet header and 20 byte IP header
followed by the TCP header (with the TCP checksum field 16 bytes
into that header). \field{csum_start} will be 14+20 = 34 (the TCP
checksum includes the header), and \field{csum_offset} will be 16.
If the given packet has the VIRTIO_NET_HDR_GSO_UDP_TUNNEL_IPV4 bit or the
VIRTIO_NET_HDR_GSO_UDP_TUNNEL_IPV6 bit set,
the above checksum fields refer to the inner header checksum, see
the example below.
\end{note}

\item If the driver negotiated
  VIRTIO_NET_F_HOST_TSO4, TSO6, USO or UFO, and the packet requires
  TCP segmentation, UDP segmentation or fragmentation, then \field{gso_type}
  is set to VIRTIO_NET_HDR_GSO_TCPV4, TCPV6, UDP_L4 or UDP.
  (Otherwise, it is set to VIRTIO_NET_HDR_GSO_NONE). In this
  case, packets larger than 1514 bytes can be transmitted: the
  metadata indicates how to replicate the packet header to cut it
  into smaller packets. The other gso fields are set:

  \begin{itemize}
  \item If the VIRTIO_NET_F_GUEST_HDRLEN feature has been negotiated,
    \field{hdr_len} indicates the header length that needs to be replicated
    for each packet. It's the number of bytes from the beginning of the packet
    to the beginning of the transport payload.
    If the \field{gso_type} has the VIRTIO_NET_HDR_GSO_UDP_TUNNEL_IPV4 bit or
    VIRTIO_NET_HDR_GSO_UDP_TUNNEL_IPV6 bit set, \field{hdr_len} accounts for
    all the headers up to and including the inner transport.
    Otherwise, if the VIRTIO_NET_F_GUEST_HDRLEN feature has not been negotiated,
    \field{hdr_len} is a hint to the device as to how much of the header
    needs to be kept to copy into each packet, usually set to the
    length of the headers, including the transport header\footnote{Due to various bugs in implementations, this field is not useful
as a guarantee of the transport header size.
}.

  \begin{note}
  Some devices benefit from knowledge of the exact header length.
  \end{note}

  \item \field{gso_size} is the maximum size of each packet beyond that
    header (ie. MSS).

  \item If the driver negotiated the VIRTIO_NET_F_HOST_ECN feature,
    the VIRTIO_NET_HDR_GSO_ECN bit in \field{gso_type}
    indicates that the TCP packet has the ECN bit set\footnote{This case is not handled by some older hardware, so is called out
specifically in the protocol.}.
   \end{itemize}

\item If the driver negotiated the VIRTIO_NET_F_HOST_UDP_TUNNEL_GSO feature and the
  \field{gso_type} has the VIRTIO_NET_HDR_GSO_UDP_TUNNEL_IPV4 bit or
  VIRTIO_NET_HDR_GSO_UDP_TUNNEL_IPV6 bit set, the GSO protocol is encapsulated
  in a UDP tunnel.
  If the outer UDP header requires checksumming, the driver must have
  additionally negotiated the VIRTIO_NET_F_HOST_UDP_TUNNEL_GSO_CSUM feature
  and offloaded the outer checksum accordingly, otherwise
  the outer UDP header must not require checksum validation, i.e. the outer
  UDP checksum must be positive zero (0x0) as defined in UDP RFC 768.
  The other tunnel-related fields indicate how to replicate the packet
  headers to cut it into smaller packets:

  \begin{itemize}
  \item \field{outer_th_offset} field indicates the outer transport header within
      the packet. This field differs from \field{csum_start} as the latter
      points to the inner transport header within the packet.

  \item \field{inner_nh_offset} field indicates the inner network header within
      the packet.
  \end{itemize}

\begin{note}
For example, consider a partially checksummed TCP (IPv4) packet carried over a
Geneve UDP tunnel (again IPv4) with no tunnel options. The
only relevant variable related to the tunnel type is the tunnel header length.
The packet will have a 14 byte outer ethernet header, 20 byte outer IP header
followed by the 8 byte UDP header (with a 0 checksum value), 8 byte Geneve header,
14 byte inner ethernet header, 20 byte inner IP header
and the TCP header (with the TCP checksum field 16 bytes
into that header). \field{csum_start} will be 14+20+8+8+14+20 = 84 (the TCP
checksum includes the header), \field{csum_offset} will be 16.
\field{inner_nh_offset} will be 14+20+8+8+14 = 62, \field{outer_th_offset} will be
14+20+8 = 42 and \field{gso_type} will be
VIRTIO_NET_HDR_GSO_TCPV4 | VIRTIO_NET_HDR_GSO_UDP_TUNNEL_IPV4 = 0x21
\end{note}

\item If the driver negotiated the VIRTIO_NET_F_HOST_UDP_TUNNEL_GSO_CSUM feature,
  the transmitted packet is a GSO one encapsulated in a UDP tunnel, and
  the outer UDP header requires checksumming, the driver can skip checksumming
  the outer header:

  \begin{itemize}
  \item \field{flags} has the VIRTIO_NET_HDR_F_UDP_TUNNEL_CSUM set,

  \item The outer UDP checksum field in the packet is set to the sum
    of the UDP pseudo header, so that replacing it by the ones'
    complement checksum of the outer UDP header and payload will give the
    correct result.
  \end{itemize}

\item \field{num_buffers} is set to zero.  This field is unused on transmitted packets.

\item The header and packet are added as one output descriptor to the
  transmitq, and the device is notified of the new entry
  (see \ref{sec:Device Types / Network Device / Device Initialization}~\nameref{sec:Device Types / Network Device / Device Initialization}).
\end{enumerate}

\drivernormative{\paragraph}{Packet Transmission}{Device Types / Network Device / Device Operation / Packet Transmission}

For the transmit packet buffer, the driver MUST use the size of the
structure \field{struct virtio_net_hdr} same as the receive packet buffer.

The driver MUST set \field{num_buffers} to zero.

If VIRTIO_NET_F_CSUM is not negotiated, the driver MUST set
\field{flags} to zero and SHOULD supply a fully checksummed
packet to the device.

If VIRTIO_NET_F_HOST_TSO4 is negotiated, the driver MAY set
\field{gso_type} to VIRTIO_NET_HDR_GSO_TCPV4 to request TCPv4
segmentation, otherwise the driver MUST NOT set
\field{gso_type} to VIRTIO_NET_HDR_GSO_TCPV4.

If VIRTIO_NET_F_HOST_TSO6 is negotiated, the driver MAY set
\field{gso_type} to VIRTIO_NET_HDR_GSO_TCPV6 to request TCPv6
segmentation, otherwise the driver MUST NOT set
\field{gso_type} to VIRTIO_NET_HDR_GSO_TCPV6.

If VIRTIO_NET_F_HOST_UFO is negotiated, the driver MAY set
\field{gso_type} to VIRTIO_NET_HDR_GSO_UDP to request UDP
fragmentation, otherwise the driver MUST NOT set
\field{gso_type} to VIRTIO_NET_HDR_GSO_UDP.

If VIRTIO_NET_F_HOST_USO is negotiated, the driver MAY set
\field{gso_type} to VIRTIO_NET_HDR_GSO_UDP_L4 to request UDP
segmentation, otherwise the driver MUST NOT set
\field{gso_type} to VIRTIO_NET_HDR_GSO_UDP_L4.

The driver SHOULD NOT send to the device TCP packets requiring segmentation offload
which have the Explicit Congestion Notification bit set, unless the
VIRTIO_NET_F_HOST_ECN feature is negotiated, in which case the
driver MUST set the VIRTIO_NET_HDR_GSO_ECN bit in
\field{gso_type}.

If VIRTIO_NET_F_HOST_UDP_TUNNEL_GSO is negotiated, the driver MAY set
VIRTIO_NET_HDR_GSO_UDP_TUNNEL_IPV4 bit or the VIRTIO_NET_HDR_GSO_UDP_TUNNEL_IPV6 bit
in \field{gso_type} according to the inner network header protocol type
to request GSO packets over UDPv4 or UDPv6 tunnel segmentation,
otherwise the driver MUST NOT set either the
VIRTIO_NET_HDR_GSO_UDP_TUNNEL_IPV4 bit or the VIRTIO_NET_HDR_GSO_UDP_TUNNEL_IPV6 bit
in \field{gso_type}.

When requesting GSO segmentation over UDP tunnel, the driver MUST SET the
VIRTIO_NET_HDR_GSO_UDP_TUNNEL_IPV4 bit if the inner network header is IPv4, i.e. the
packet is a TCPv4 GSO one, otherwise, if the inner network header is IPv6, the driver
MUST SET the VIRTIO_NET_HDR_GSO_UDP_TUNNEL_IPV6 bit.

The driver MUST NOT send to the device GSO packets over UDP tunnel
requiring segmentation and outer UDP checksum offload, unless both the
VIRTIO_NET_F_HOST_UDP_TUNNEL_GSO and VIRTIO_NET_F_HOST_UDP_TUNNEL_GSO_CSUM features
are negotiated, in which case the driver MUST set either the
VIRTIO_NET_HDR_GSO_UDP_TUNNEL_IPV4 bit or the VIRTIO_NET_HDR_GSO_UDP_TUNNEL_IPV6
bit in the \field{gso_type} and the VIRTIO_NET_HDR_F_UDP_TUNNEL_CSUM bit in
the \field{flags}.

If VIRTIO_NET_F_HOST_UDP_TUNNEL_GSO_CSUM is not negotiated, the driver MUST not set
the VIRTIO_NET_HDR_F_UDP_TUNNEL_CSUM bit in the \field{flags} and
MUST NOT send to the device GSO packets over UDP tunnel
requiring segmentation and outer UDP checksum offload.

The driver MUST NOT set the VIRTIO_NET_HDR_GSO_UDP_TUNNEL_IPV4 bit or the
VIRTIO_NET_HDR_GSO_UDP_TUNNEL_IPV6 bit together with VIRTIO_NET_HDR_GSO_UDP, as the
latter is deprecated in favor of UDP_L4 and no new feature will support it.

The driver MUST NOT set the VIRTIO_NET_HDR_GSO_UDP_TUNNEL_IPV4 bit and the
VIRTIO_NET_HDR_GSO_UDP_TUNNEL_IPV6 bit together.

The driver MUST NOT set the VIRTIO_NET_HDR_F_UDP_TUNNEL_CSUM bit \field{flags}
without setting either the VIRTIO_NET_HDR_GSO_UDP_TUNNEL_IPV4 bit or
the VIRTIO_NET_HDR_GSO_UDP_TUNNEL_IPV6 bit in \field{gso_type}.

If the VIRTIO_NET_F_CSUM feature has been negotiated, the
driver MAY set the VIRTIO_NET_HDR_F_NEEDS_CSUM bit in
\field{flags}, if so:
\begin{enumerate}
\item the driver MUST validate the packet checksum at
	offset \field{csum_offset} from \field{csum_start} as well as all
	preceding offsets;
\begin{note}
If \field{gso_type} differs from VIRTIO_NET_HDR_GSO_NONE and the
VIRTIO_NET_HDR_GSO_UDP_TUNNEL_IPV4 bit or the VIRTIO_NET_HDR_GSO_UDP_TUNNEL_IPV6
bit are not set in \field{gso_type}, \field{csum_offset}
points to the only transport header present in the packet, and there are no
additional preceding checksums validated by VIRTIO_NET_HDR_F_NEEDS_CSUM.
\end{note}
\item the driver MUST set the packet checksum stored in the
	buffer to the TCP/UDP pseudo header;
\item the driver MUST set \field{csum_start} and
	\field{csum_offset} such that calculating a ones'
	complement checksum from \field{csum_start} up until the end of
	the packet and storing the result at offset \field{csum_offset}
	from  \field{csum_start} will result in a fully checksummed
	packet;
\end{enumerate}

If none of the VIRTIO_NET_F_HOST_TSO4, TSO6, USO or UFO options have
been negotiated, the driver MUST set \field{gso_type} to
VIRTIO_NET_HDR_GSO_NONE.

If \field{gso_type} differs from VIRTIO_NET_HDR_GSO_NONE, then
the driver MUST also set the VIRTIO_NET_HDR_F_NEEDS_CSUM bit in
\field{flags} and MUST set \field{gso_size} to indicate the
desired MSS.

If one of the VIRTIO_NET_F_HOST_TSO4, TSO6, USO or UFO options have
been negotiated:
\begin{itemize}
\item If the VIRTIO_NET_F_GUEST_HDRLEN feature has been negotiated,
	and \field{gso_type} differs from VIRTIO_NET_HDR_GSO_NONE,
	the driver MUST set \field{hdr_len} to a value equal to the length
	of the headers, including the transport header. If \field{gso_type}
	has the VIRTIO_NET_HDR_GSO_UDP_TUNNEL_IPV4 bit or the
	VIRTIO_NET_HDR_GSO_UDP_TUNNEL_IPV6 bit set, \field{hdr_len} includes
	the inner transport header.

\item If the VIRTIO_NET_F_GUEST_HDRLEN feature has not been negotiated,
	or \field{gso_type} is VIRTIO_NET_HDR_GSO_NONE,
	the driver SHOULD set \field{hdr_len} to a value
	not less than the length of the headers, including the transport
	header.
\end{itemize}

If the VIRTIO_NET_F_HOST_UDP_TUNNEL_GSO option has been negotiated, the
driver MAY set the VIRTIO_NET_HDR_GSO_UDP_TUNNEL_IPV4 bit or the
VIRTIO_NET_HDR_GSO_UDP_TUNNEL_IPV6 bit in \field{gso_type}, if so:
\begin{itemize}
\item the driver MUST set \field{outer_th_offset} to the outer UDP header
  offset and \field{inner_nh_offset} to the inner network header offset.
  The \field{csum_start} and \field{csum_offset} fields point respectively
  to the inner transport header and inner transport checksum field.
\end{itemize}

If the VIRTIO_NET_F_HOST_UDP_TUNNEL_GSO_CSUM feature has been negotiated,
and the VIRTIO_NET_HDR_GSO_UDP_TUNNEL_IPV4 bit or
VIRTIO_NET_HDR_GSO_UDP_TUNNEL_IPV6 bit in \field{gso_type} are set,
the driver MAY set the VIRTIO_NET_HDR_F_UDP_TUNNEL_CSUM bit in
\field{flags}, if so the driver MUST set the packet outer UDP header checksum
to the outer UDP pseudo header checksum.

\begin{note}
calculating a ones' complement checksum from \field{outer_th_offset}
up until the end of the packet and storing the result at offset 6
from \field{outer_th_offset} will result in a fully checksummed outer UDP packet;
\end{note}

If the VIRTIO_NET_HDR_GSO_UDP_TUNNEL_IPV4 bit or the
VIRTIO_NET_HDR_GSO_UDP_TUNNEL_IPV6 bit in \field{gso_type} are set
and the VIRTIO_NET_F_HOST_UDP_TUNNEL_GSO_CSUM feature has not
been negotiated, the
outer UDP header MUST NOT require checksum validation. That is, the
outer UDP checksum value MUST be 0 or the validated complete checksum
for such header.

\begin{note}
The valid complete checksum of the outer UDP header of individual segments
can be computed by the driver prior to segmentation only if the GSO packet
size is a multiple of \field{gso_size}, because then all segments
have the same size and thus all data included in the outer UDP
checksum is the same for every segment. These pre-computed segment
length and checksum fields are different from those of the GSO
packet.
In this scenario the outer UDP header of the GSO packet must carry the
segmented UDP packet length.
\end{note}

If the VIRTIO_NET_F_HOST_UDP_TUNNEL_GSO option has not
been negotiated, the driver MUST NOT set either the VIRTIO_NET_HDR_F_GSO_UDP_TUNNEL_IPV4
bit or the VIRTIO_NET_HDR_F_GSO_UDP_TUNNEL_IPV6 in \field{gso_type}.

If the VIRTIO_NET_F_HOST_UDP_TUNNEL_GSO_CSUM option has not been negotiated,
the driver MUST NOT set the VIRTIO_NET_HDR_F_UDP_TUNNEL_CSUM bit
in \field{flags}.

The driver SHOULD accept the VIRTIO_NET_F_GUEST_HDRLEN feature if it has
been offered, and if it's able to provide the exact header length.

The driver MUST NOT set the VIRTIO_NET_HDR_F_DATA_VALID and
VIRTIO_NET_HDR_F_RSC_INFO bits in \field{flags}.

The driver MUST NOT set the VIRTIO_NET_HDR_F_DATA_VALID bit in \field{flags}
together with the VIRTIO_NET_HDR_F_GSO_UDP_TUNNEL_IPV4 bit or the
VIRTIO_NET_HDR_F_GSO_UDP_TUNNEL_IPV6 bit in \field{gso_type}.

\devicenormative{\paragraph}{Packet Transmission}{Device Types / Network Device / Device Operation / Packet Transmission}
The device MUST ignore \field{flag} bits that it does not recognize.

If VIRTIO_NET_HDR_F_NEEDS_CSUM bit in \field{flags} is not set, the
device MUST NOT use the \field{csum_start} and \field{csum_offset}.

If one of the VIRTIO_NET_F_HOST_TSO4, TSO6, USO or UFO options have
been negotiated:
\begin{itemize}
\item If the VIRTIO_NET_F_GUEST_HDRLEN feature has been negotiated,
	and \field{gso_type} differs from VIRTIO_NET_HDR_GSO_NONE,
	the device MAY use \field{hdr_len} as the transport header size.

	\begin{note}
	Caution should be taken by the implementation so as to prevent
	a malicious driver from attacking the device by setting an incorrect hdr_len.
	\end{note}

\item If the VIRTIO_NET_F_GUEST_HDRLEN feature has not been negotiated,
	or \field{gso_type} is VIRTIO_NET_HDR_GSO_NONE,
	the device MAY use \field{hdr_len} only as a hint about the
	transport header size.
	The device MUST NOT rely on \field{hdr_len} to be correct.

	\begin{note}
	This is due to various bugs in implementations.
	\end{note}
\end{itemize}

If both the VIRTIO_NET_HDR_GSO_UDP_TUNNEL_IPV4 bit and
the VIRTIO_NET_HDR_GSO_UDP_TUNNEL_IPV6 bit in in \field{gso_type} are set,
the device MUST NOT accept the packet.

If the VIRTIO_NET_HDR_GSO_UDP_TUNNEL_IPV4 bit and the VIRTIO_NET_HDR_GSO_UDP_TUNNEL_IPV6
bit in \field{gso_type} are not set, the device MUST NOT use the
\field{outer_th_offset} and \field{inner_nh_offset}.

If either the VIRTIO_NET_HDR_GSO_UDP_TUNNEL_IPV4 bit or
the VIRTIO_NET_HDR_GSO_UDP_TUNNEL_IPV6 bit in \field{gso_type} are set, and any of
the following is true:
\begin{itemize}
\item the VIRTIO_NET_HDR_F_NEEDS_CSUM is not set in \field{flags}
\item the VIRTIO_NET_HDR_F_DATA_VALID is set in \field{flags}
\item the \field{gso_type} excluding the VIRTIO_NET_HDR_GSO_UDP_TUNNEL_IPV4
bit and the VIRTIO_NET_HDR_GSO_UDP_TUNNEL_IPV6 bit is VIRTIO_NET_HDR_GSO_NONE
\end{itemize}
the device MUST NOT accept the packet.

If the VIRTIO_NET_HDR_F_UDP_TUNNEL_CSUM bit in \field{flags} is set,
and both the bits VIRTIO_NET_HDR_GSO_UDP_TUNNEL_IPV4 and
VIRTIO_NET_HDR_GSO_UDP_TUNNEL_IPV6 in \field{gso_type} are not set,
the device MOST NOT accept the packet.

If VIRTIO_NET_HDR_F_NEEDS_CSUM is not set, the device MUST NOT
rely on the packet checksum being correct.
\paragraph{Packet Transmission Interrupt}\label{sec:Device Types / Network Device / Device Operation / Packet Transmission / Packet Transmission Interrupt}

Often a driver will suppress transmission virtqueue interrupts
and check for used packets in the transmit path of following
packets.

The normal behavior in this interrupt handler is to retrieve
used buffers from the virtqueue and free the corresponding
headers and packets.

\subsubsection{Setting Up Receive Buffers}\label{sec:Device Types / Network Device / Device Operation / Setting Up Receive Buffers}

It is generally a good idea to keep the receive virtqueue as
fully populated as possible: if it runs out, network performance
will suffer.

If the VIRTIO_NET_F_GUEST_TSO4, VIRTIO_NET_F_GUEST_TSO6,
VIRTIO_NET_F_GUEST_UFO, VIRTIO_NET_F_GUEST_USO4 or VIRTIO_NET_F_GUEST_USO6
features are used, the maximum incoming packet
will be 65589 bytes long (14 bytes of Ethernet header, plus 40 bytes of
the IPv6 header, plus 65535 bytes of maximum IPv6 payload including any
extension header), otherwise 1514 bytes.
When VIRTIO_NET_F_HASH_REPORT is not negotiated, the required receive buffer
size is either 65601 or 1526 bytes accounting for 20 bytes of
\field{struct virtio_net_hdr} followed by receive packet.
When VIRTIO_NET_F_HASH_REPORT is negotiated, the required receive buffer
size is either 65609 or 1534 bytes accounting for 12 bytes of
\field{struct virtio_net_hdr} followed by receive packet.

\drivernormative{\paragraph}{Setting Up Receive Buffers}{Device Types / Network Device / Device Operation / Setting Up Receive Buffers}

\begin{itemize}
\item If VIRTIO_NET_F_MRG_RXBUF is not negotiated:
  \begin{itemize}
    \item If VIRTIO_NET_F_GUEST_TSO4, VIRTIO_NET_F_GUEST_TSO6, VIRTIO_NET_F_GUEST_UFO,
	VIRTIO_NET_F_GUEST_USO4 or VIRTIO_NET_F_GUEST_USO6 are negotiated, the driver SHOULD populate
      the receive queue(s) with buffers of at least 65609 bytes if
      VIRTIO_NET_F_HASH_REPORT is negotiated, and of at least 65601 bytes if not.
    \item Otherwise, the driver SHOULD populate the receive queue(s)
      with buffers of at least 1534 bytes if VIRTIO_NET_F_HASH_REPORT
      is negotiated, and of at least 1526 bytes if not.
  \end{itemize}
\item If VIRTIO_NET_F_MRG_RXBUF is negotiated, each buffer MUST be at
least size of \field{struct virtio_net_hdr},
i.e. 20 bytes if VIRTIO_NET_F_HASH_REPORT is negotiated, and 12 bytes if not.
\end{itemize}

\begin{note}
Obviously each buffer can be split across multiple descriptor elements.
\end{note}

When calculating the size of \field{struct virtio_net_hdr}, the driver
MUST consider all the fields inclusive up to \field{padding_reserved},
i.e. 20 bytes if VIRTIO_NET_F_HASH_REPORT is negotiated, and 12 bytes if not.

If VIRTIO_NET_F_MQ is negotiated, each of receiveq1\ldots receiveqN
that will be used SHOULD be populated with receive buffers.

\devicenormative{\paragraph}{Setting Up Receive Buffers}{Device Types / Network Device / Device Operation / Setting Up Receive Buffers}

The device MUST set \field{num_buffers} to the number of descriptors used to
hold the incoming packet.

The device MUST use only a single descriptor if VIRTIO_NET_F_MRG_RXBUF
was not negotiated.
\begin{note}
{This means that \field{num_buffers} will always be 1
if VIRTIO_NET_F_MRG_RXBUF is not negotiated.}
\end{note}

\subsubsection{Processing of Incoming Packets}\label{sec:Device Types / Network Device / Device Operation / Processing of Incoming Packets}
\label{sec:Device Types / Network Device / Device Operation / Processing of Packets}%old label for latexdiff

When a packet is copied into a buffer in the receiveq, the
optimal path is to disable further used buffer notifications for the
receiveq and process packets until no more are found, then re-enable
them.

Processing incoming packets involves:

\begin{enumerate}
\item \field{num_buffers} indicates how many descriptors
  this packet is spread over (including this one): this will
  always be 1 if VIRTIO_NET_F_MRG_RXBUF was not negotiated.
  This allows receipt of large packets without having to allocate large
  buffers: a packet that does not fit in a single buffer can flow
  over to the next buffer, and so on. In this case, there will be
  at least \field{num_buffers} used buffers in the virtqueue, and the device
  chains them together to form a single packet in a way similar to
  how it would store it in a single buffer spread over multiple
  descriptors.
  The other buffers will not begin with a \field{struct virtio_net_hdr}.

\item If
  \field{num_buffers} is one, then the entire packet will be
  contained within this buffer, immediately following the struct
  virtio_net_hdr.
\item If the VIRTIO_NET_F_GUEST_CSUM feature was negotiated, the
  VIRTIO_NET_HDR_F_DATA_VALID bit in \field{flags} can be
  set: if so, device has validated the packet checksum.
  If the VIRTIO_NET_F_GUEST_UDP_TUNNEL_GSO_CSUM feature has been negotiated,
  and the VIRTIO_NET_HDR_F_UDP_TUNNEL_CSUM bit is set in \field{flags},
  both the outer UDP checksum and the inner transport checksum
  have been validated, otherwise only one level of checksums (the outer one
  in case of tunnels) has been validated.
\end{enumerate}

Additionally, VIRTIO_NET_F_GUEST_CSUM, TSO4, TSO6, UDP, UDP_TUNNEL
and ECN features enable receive checksum, large receive offload and ECN
support which are the input equivalents of the transmit checksum,
transmit segmentation offloading and ECN features, as described
in \ref{sec:Device Types / Network Device / Device Operation /
Packet Transmission}:
\begin{enumerate}
\item If the VIRTIO_NET_F_GUEST_TSO4, TSO6, UFO, USO4 or USO6 options were
  negotiated, then \field{gso_type} MAY be something other than
  VIRTIO_NET_HDR_GSO_NONE, and \field{gso_size} field indicates the
  desired MSS (see Packet Transmission point 2).
\item If the VIRTIO_NET_F_RSC_EXT option was negotiated (this
  implies one of VIRTIO_NET_F_GUEST_TSO4, TSO6), the
  device processes also duplicated ACK segments, reports
  number of coalesced TCP segments in \field{csum_start} field and
  number of duplicated ACK segments in \field{csum_offset} field
  and sets bit VIRTIO_NET_HDR_F_RSC_INFO in \field{flags}.
\item If the VIRTIO_NET_F_GUEST_CSUM feature was negotiated, the
  VIRTIO_NET_HDR_F_NEEDS_CSUM bit in \field{flags} can be
  set: if so, the packet checksum at offset \field{csum_offset}
  from \field{csum_start} and any preceding checksums
  have been validated.  The checksum on the packet is incomplete and
  if bit VIRTIO_NET_HDR_F_RSC_INFO is not set in \field{flags},
  then \field{csum_start} and \field{csum_offset} indicate how to calculate it
  (see Packet Transmission point 1).
\begin{note}
If \field{gso_type} differs from VIRTIO_NET_HDR_GSO_NONE and the
VIRTIO_NET_HDR_GSO_UDP_TUNNEL_IPV4 bit or the VIRTIO_NET_HDR_GSO_UDP_TUNNEL_IPV6
bit are not set, \field{csum_offset}
points to the only transport header present in the packet, and there are no
additional preceding checksums validated by VIRTIO_NET_HDR_F_NEEDS_CSUM.
\end{note}
\item If the VIRTIO_NET_F_GUEST_UDP_TUNNEL_GSO option was negotiated and
  \field{gso_type} is not VIRTIO_NET_HDR_GSO_NONE, the
  VIRTIO_NET_HDR_GSO_UDP_TUNNEL_IPV4 bit or the VIRTIO_NET_HDR_GSO_UDP_TUNNEL_IPV6
  bit MAY be set. In such case the \field{outer_th_offset} and
  \field{inner_nh_offset} fields indicate the corresponding
  headers information.
\item If the VIRTIO_NET_F_GUEST_UDP_TUNNEL_GSO_CSUM feature was
negotiated, and
  the VIRTIO_NET_HDR_GSO_UDP_TUNNEL_IPV4 bit or the VIRTIO_NET_HDR_GSO_UDP_TUNNEL_IPV6
  are set in \field{gso_type}, the VIRTIO_NET_HDR_F_UDP_TUNNEL_CSUM bit in the
  \field{flags} can be set: if so, the outer UDP checksum has been validated
  and the UDP header checksum at offset 6 from from \field{outer_th_offset}
  is set to the outer UDP pseudo header checksum.

\begin{note}
If the VIRTIO_NET_HDR_GSO_UDP_TUNNEL_IPV4 bit or VIRTIO_NET_HDR_GSO_UDP_TUNNEL_IPV6
bit are set in \field{gso_type}, the \field{csum_start} field refers to
the inner transport header offset (see Packet Transmission point 1).
If the VIRTIO_NET_HDR_F_UDP_TUNNEL_CSUM bit in \field{flags} is set both
the inner and the outer header checksums have been validated by
VIRTIO_NET_HDR_F_NEEDS_CSUM, otherwise only the inner transport header
checksum has been validated.
\end{note}
\end{enumerate}

If applicable, the device calculates per-packet hash for incoming packets as
defined in \ref{sec:Device Types / Network Device / Device Operation / Processing of Incoming Packets / Hash calculation for incoming packets}.

If applicable, the device reports hash information for incoming packets as
defined in \ref{sec:Device Types / Network Device / Device Operation / Processing of Incoming Packets / Hash reporting for incoming packets}.

\devicenormative{\paragraph}{Processing of Incoming Packets}{Device Types / Network Device / Device Operation / Processing of Incoming Packets}
\label{devicenormative:Device Types / Network Device / Device Operation / Processing of Packets}%old label for latexdiff

If VIRTIO_NET_F_MRG_RXBUF has not been negotiated, the device MUST set
\field{num_buffers} to 1.

If VIRTIO_NET_F_MRG_RXBUF has been negotiated, the device MUST set
\field{num_buffers} to indicate the number of buffers
the packet (including the header) is spread over.

If a receive packet is spread over multiple buffers, the device
MUST use all buffers but the last (i.e. the first \field{num_buffers} -
1 buffers) completely up to the full length of each buffer
supplied by the driver.

The device MUST use all buffers used by a single receive
packet together, such that at least \field{num_buffers} are
observed by driver as used.

If VIRTIO_NET_F_GUEST_CSUM is not negotiated, the device MUST set
\field{flags} to zero and SHOULD supply a fully checksummed
packet to the driver.

If VIRTIO_NET_F_GUEST_TSO4 is not negotiated, the device MUST NOT set
\field{gso_type} to VIRTIO_NET_HDR_GSO_TCPV4.

If VIRTIO_NET_F_GUEST_UDP is not negotiated, the device MUST NOT set
\field{gso_type} to VIRTIO_NET_HDR_GSO_UDP.

If VIRTIO_NET_F_GUEST_TSO6 is not negotiated, the device MUST NOT set
\field{gso_type} to VIRTIO_NET_HDR_GSO_TCPV6.

If none of VIRTIO_NET_F_GUEST_USO4 or VIRTIO_NET_F_GUEST_USO6 have been negotiated,
the device MUST NOT set \field{gso_type} to VIRTIO_NET_HDR_GSO_UDP_L4.

If VIRTIO_NET_F_GUEST_UDP_TUNNEL_GSO is not negotiated, the device MUST NOT set
either the VIRTIO_NET_HDR_GSO_UDP_TUNNEL_IPV4 bit or the
VIRTIO_NET_HDR_GSO_UDP_TUNNEL_IPV6 bit in \field{gso_type}.

If VIRTIO_NET_F_GUEST_UDP_TUNNEL_GSO_CSUM is not negotiated the device MUST NOT set
the VIRTIO_NET_HDR_F_UDP_TUNNEL_CSUM bit in \field{flags}.

The device SHOULD NOT send to the driver TCP packets requiring segmentation offload
which have the Explicit Congestion Notification bit set, unless the
VIRTIO_NET_F_GUEST_ECN feature is negotiated, in which case the
device MUST set the VIRTIO_NET_HDR_GSO_ECN bit in
\field{gso_type}.

If the VIRTIO_NET_F_GUEST_CSUM feature has been negotiated, the
device MAY set the VIRTIO_NET_HDR_F_NEEDS_CSUM bit in
\field{flags}, if so:
\begin{enumerate}
\item the device MUST validate the packet checksum at
	offset \field{csum_offset} from \field{csum_start} as well as all
	preceding offsets;
\item the device MUST set the packet checksum stored in the
	receive buffer to the TCP/UDP pseudo header;
\item the device MUST set \field{csum_start} and
	\field{csum_offset} such that calculating a ones'
	complement checksum from \field{csum_start} up until the
	end of the packet and storing the result at offset
	\field{csum_offset} from  \field{csum_start} will result in a
	fully checksummed packet;
\end{enumerate}

The device MUST NOT send to the driver GSO packets encapsulated in UDP
tunnel and requiring segmentation offload, unless the
VIRTIO_NET_F_GUEST_UDP_TUNNEL_GSO is negotiated, in which case the device MUST set
the VIRTIO_NET_HDR_GSO_UDP_TUNNEL_IPV4 bit or the VIRTIO_NET_HDR_GSO_UDP_TUNNEL_IPV6
bit in \field{gso_type} according to the inner network header protocol type,
MUST set the \field{outer_th_offset} and \field{inner_nh_offset} fields
to the corresponding header information, and the outer UDP header MUST NOT
require checksum offload.

If the VIRTIO_NET_F_GUEST_UDP_TUNNEL_GSO_CSUM feature has not been negotiated,
the device MUST NOT send the driver GSO packets encapsulated in UDP
tunnel and requiring segmentation and outer checksum offload.

If none of the VIRTIO_NET_F_GUEST_TSO4, TSO6, UFO, USO4 or USO6 options have
been negotiated, the device MUST set \field{gso_type} to
VIRTIO_NET_HDR_GSO_NONE.

If \field{gso_type} differs from VIRTIO_NET_HDR_GSO_NONE, then
the device MUST also set the VIRTIO_NET_HDR_F_NEEDS_CSUM bit in
\field{flags} MUST set \field{gso_size} to indicate the desired MSS.
If VIRTIO_NET_F_RSC_EXT was negotiated, the device MUST also
set VIRTIO_NET_HDR_F_RSC_INFO bit in \field{flags},
set \field{csum_start} to number of coalesced TCP segments and
set \field{csum_offset} to number of received duplicated ACK segments.

If VIRTIO_NET_F_RSC_EXT was not negotiated, the device MUST
not set VIRTIO_NET_HDR_F_RSC_INFO bit in \field{flags}.

If one of the VIRTIO_NET_F_GUEST_TSO4, TSO6, UFO, USO4 or USO6 options have
been negotiated, the device SHOULD set \field{hdr_len} to a value
not less than the length of the headers, including the transport
header. If \field{gso_type} has the VIRTIO_NET_HDR_GSO_UDP_TUNNEL_IPV4 bit
or the VIRTIO_NET_HDR_GSO_UDP_TUNNEL_IPV6 bit set, the referenced transport
header is the inner one.

If the VIRTIO_NET_F_GUEST_CSUM feature has been negotiated, the
device MAY set the VIRTIO_NET_HDR_F_DATA_VALID bit in
\field{flags}, if so, the device MUST validate the packet
checksum. If the VIRTIO_NET_F_GUEST_UDP_TUNNEL_GSO_CSUM feature has
been negotiated, and the VIRTIO_NET_HDR_F_UDP_TUNNEL_CSUM bit set in
\field{flags}, both the outer UDP checksum and the inner transport
checksum have been validated.
Otherwise level of checksum is validated: in case of multiple
encapsulated protocols the outermost one.

If either the VIRTIO_NET_HDR_GSO_UDP_TUNNEL_IPV4 bit or the
VIRTIO_NET_HDR_GSO_UDP_TUNNEL_IPV6 bit in \field{gso_type} are set,
the device MUST NOT set the VIRTIO_NET_HDR_F_DATA_VALID bit in
\field{flags}.

If the VIRTIO_NET_F_GUEST_UDP_TUNNEL_GSO_CSUM feature has been negotiated
and either the VIRTIO_NET_HDR_GSO_UDP_TUNNEL_IPV4 bit is set or the
VIRTIO_NET_HDR_GSO_UDP_TUNNEL_IPV6 bit is set in \field{gso_type}, the
device MAY set the VIRTIO_NET_HDR_F_UDP_TUNNEL_CSUM bit in
\field{flags}, if so the device MUST set the packet outer UDP checksum
stored in the receive buffer to the outer UDP pseudo header.

Otherwise, the VIRTIO_NET_F_GUEST_UDP_TUNNEL_GSO_CSUM feature has been
negotiated, either the VIRTIO_NET_HDR_GSO_UDP_TUNNEL_IPV4 bit is set or the
VIRTIO_NET_HDR_GSO_UDP_TUNNEL_IPV6 bit is set in \field{gso_type},
and the bit VIRTIO_NET_HDR_F_UDP_TUNNEL_CSUM is not set in
\field{flags}, the device MUST either provide a zero outer UDP header
checksum or a fully checksummed outer UDP header.

\drivernormative{\paragraph}{Processing of Incoming
Packets}{Device Types / Network Device / Device Operation /
Processing of Incoming Packets}

The driver MUST ignore \field{flag} bits that it does not recognize.

If VIRTIO_NET_HDR_F_NEEDS_CSUM bit in \field{flags} is not set or
if VIRTIO_NET_HDR_F_RSC_INFO bit \field{flags} is set, the
driver MUST NOT use the \field{csum_start} and \field{csum_offset}.

If one of the VIRTIO_NET_F_GUEST_TSO4, TSO6, UFO, USO4 or USO6 options have
been negotiated, the driver MAY use \field{hdr_len} only as a hint about the
transport header size.
The driver MUST NOT rely on \field{hdr_len} to be correct.
\begin{note}
This is due to various bugs in implementations.
\end{note}

If neither VIRTIO_NET_HDR_F_NEEDS_CSUM nor
VIRTIO_NET_HDR_F_DATA_VALID is set, the driver MUST NOT
rely on the packet checksum being correct.

If both the VIRTIO_NET_HDR_GSO_UDP_TUNNEL_IPV4 bit and
the VIRTIO_NET_HDR_GSO_UDP_TUNNEL_IPV6 bit in in \field{gso_type} are set,
the driver MUST NOT accept the packet.

If the VIRTIO_NET_HDR_GSO_UDP_TUNNEL_IPV4 bit or the VIRTIO_NET_HDR_GSO_UDP_TUNNEL_IPV6
bit in \field{gso_type} are not set, the driver MUST NOT use the
\field{outer_th_offset} and \field{inner_nh_offset}.

If either the VIRTIO_NET_HDR_GSO_UDP_TUNNEL_IPV4 bit or
the VIRTIO_NET_HDR_GSO_UDP_TUNNEL_IPV6 bit in \field{gso_type} are set, and any of
the following is true:
\begin{itemize}
\item the VIRTIO_NET_HDR_F_NEEDS_CSUM bit is not set in \field{flags}
\item the VIRTIO_NET_HDR_F_DATA_VALID bit is set in \field{flags}
\item the \field{gso_type} excluding the VIRTIO_NET_HDR_GSO_UDP_TUNNEL_IPV4
bit and the VIRTIO_NET_HDR_GSO_UDP_TUNNEL_IPV6 bit is VIRTIO_NET_HDR_GSO_NONE
\end{itemize}
the driver MUST NOT accept the packet.

If the VIRTIO_NET_HDR_F_UDP_TUNNEL_CSUM bit and the VIRTIO_NET_HDR_F_NEEDS_CSUM
bit in \field{flags} are set,
and both the bits VIRTIO_NET_HDR_GSO_UDP_TUNNEL_IPV4 and
VIRTIO_NET_HDR_GSO_UDP_TUNNEL_IPV6 in \field{gso_type} are not set,
the driver MOST NOT accept the packet.

\paragraph{Hash calculation for incoming packets}
\label{sec:Device Types / Network Device / Device Operation / Processing of Incoming Packets / Hash calculation for incoming packets}

A device attempts to calculate a per-packet hash in the following cases:
\begin{itemize}
\item The feature VIRTIO_NET_F_RSS was negotiated. The device uses the hash to determine the receive virtqueue to place incoming packets.
\item The feature VIRTIO_NET_F_HASH_REPORT was negotiated. The device reports the hash value and the hash type with the packet.
\end{itemize}

If the feature VIRTIO_NET_F_RSS was negotiated:
\begin{itemize}
\item The device uses \field{hash_types} of the virtio_net_rss_config structure as 'Enabled hash types' bitmask.
\item If additionally the feature VIRTIO_NET_F_HASH_TUNNEL was negotiated, the device uses \field{enabled_tunnel_types} of the
      virtnet_hash_tunnel structure as 'Encapsulation types enabled for inner header hash' bitmask.
\item The device uses a key as defined in \field{hash_key_data} and \field{hash_key_length} of the virtio_net_rss_config structure (see
\ref{sec:Device Types / Network Device / Device Operation / Control Virtqueue / Receive-side scaling (RSS) / Setting RSS parameters}).
\end{itemize}

If the feature VIRTIO_NET_F_RSS was not negotiated:
\begin{itemize}
\item The device uses \field{hash_types} of the virtio_net_hash_config structure as 'Enabled hash types' bitmask.
\item If additionally the feature VIRTIO_NET_F_HASH_TUNNEL was negotiated, the device uses \field{enabled_tunnel_types} of the
      virtnet_hash_tunnel structure as 'Encapsulation types enabled for inner header hash' bitmask.
\item The device uses a key as defined in \field{hash_key_data} and \field{hash_key_length} of the virtio_net_hash_config structure (see
\ref{sec:Device Types / Network Device / Device Operation / Control Virtqueue / Automatic receive steering in multiqueue mode / Hash calculation}).
\end{itemize}

Note that if the device offers VIRTIO_NET_F_HASH_REPORT, even if it supports only one pair of virtqueues, it MUST support
at least one of commands of VIRTIO_NET_CTRL_MQ class to configure reported hash parameters:
\begin{itemize}
\item If the device offers VIRTIO_NET_F_RSS, it MUST support VIRTIO_NET_CTRL_MQ_RSS_CONFIG command per
 \ref{sec:Device Types / Network Device / Device Operation / Control Virtqueue / Receive-side scaling (RSS) / Setting RSS parameters}.
\item Otherwise the device MUST support VIRTIO_NET_CTRL_MQ_HASH_CONFIG command per
 \ref{sec:Device Types / Network Device / Device Operation / Control Virtqueue / Automatic receive steering in multiqueue mode / Hash calculation}.
\end{itemize}

The per-packet hash calculation can depend on the IP packet type. See
\hyperref[intro:IP]{[IP]}, \hyperref[intro:UDP]{[UDP]} and \hyperref[intro:TCP]{[TCP]}.

\subparagraph{Supported/enabled hash types}
\label{sec:Device Types / Network Device / Device Operation / Processing of Incoming Packets / Hash calculation for incoming packets / Supported/enabled hash types}
Hash types applicable for IPv4 packets:
\begin{lstlisting}
#define VIRTIO_NET_HASH_TYPE_IPv4              (1 << 0)
#define VIRTIO_NET_HASH_TYPE_TCPv4             (1 << 1)
#define VIRTIO_NET_HASH_TYPE_UDPv4             (1 << 2)
\end{lstlisting}
Hash types applicable for IPv6 packets without extension headers
\begin{lstlisting}
#define VIRTIO_NET_HASH_TYPE_IPv6              (1 << 3)
#define VIRTIO_NET_HASH_TYPE_TCPv6             (1 << 4)
#define VIRTIO_NET_HASH_TYPE_UDPv6             (1 << 5)
\end{lstlisting}
Hash types applicable for IPv6 packets with extension headers
\begin{lstlisting}
#define VIRTIO_NET_HASH_TYPE_IP_EX             (1 << 6)
#define VIRTIO_NET_HASH_TYPE_TCP_EX            (1 << 7)
#define VIRTIO_NET_HASH_TYPE_UDP_EX            (1 << 8)
\end{lstlisting}

\subparagraph{IPv4 packets}
\label{sec:Device Types / Network Device / Device Operation / Processing of Incoming Packets / Hash calculation for incoming packets / IPv4 packets}
The device calculates the hash on IPv4 packets according to 'Enabled hash types' bitmask as follows:
\begin{itemize}
\item If VIRTIO_NET_HASH_TYPE_TCPv4 is set and the packet has
a TCP header, the hash is calculated over the following fields:
\begin{itemize}
\item Source IP address
\item Destination IP address
\item Source TCP port
\item Destination TCP port
\end{itemize}
\item Else if VIRTIO_NET_HASH_TYPE_UDPv4 is set and the
packet has a UDP header, the hash is calculated over the following fields:
\begin{itemize}
\item Source IP address
\item Destination IP address
\item Source UDP port
\item Destination UDP port
\end{itemize}
\item Else if VIRTIO_NET_HASH_TYPE_IPv4 is set, the hash is
calculated over the following fields:
\begin{itemize}
\item Source IP address
\item Destination IP address
\end{itemize}
\item Else the device does not calculate the hash
\end{itemize}

\subparagraph{IPv6 packets without extension header}
\label{sec:Device Types / Network Device / Device Operation / Processing of Incoming Packets / Hash calculation for incoming packets / IPv6 packets without extension header}
The device calculates the hash on IPv6 packets without extension
headers according to 'Enabled hash types' bitmask as follows:
\begin{itemize}
\item If VIRTIO_NET_HASH_TYPE_TCPv6 is set and the packet has
a TCPv6 header, the hash is calculated over the following fields:
\begin{itemize}
\item Source IPv6 address
\item Destination IPv6 address
\item Source TCP port
\item Destination TCP port
\end{itemize}
\item Else if VIRTIO_NET_HASH_TYPE_UDPv6 is set and the
packet has a UDPv6 header, the hash is calculated over the following fields:
\begin{itemize}
\item Source IPv6 address
\item Destination IPv6 address
\item Source UDP port
\item Destination UDP port
\end{itemize}
\item Else if VIRTIO_NET_HASH_TYPE_IPv6 is set, the hash is
calculated over the following fields:
\begin{itemize}
\item Source IPv6 address
\item Destination IPv6 address
\end{itemize}
\item Else the device does not calculate the hash
\end{itemize}

\subparagraph{IPv6 packets with extension header}
\label{sec:Device Types / Network Device / Device Operation / Processing of Incoming Packets / Hash calculation for incoming packets / IPv6 packets with extension header}
The device calculates the hash on IPv6 packets with extension
headers according to 'Enabled hash types' bitmask as follows:
\begin{itemize}
\item If VIRTIO_NET_HASH_TYPE_TCP_EX is set and the packet
has a TCPv6 header, the hash is calculated over the following fields:
\begin{itemize}
\item Home address from the home address option in the IPv6 destination options header. If the extension header is not present, use the Source IPv6 address.
\item IPv6 address that is contained in the Routing-Header-Type-2 from the associated extension header. If the extension header is not present, use the Destination IPv6 address.
\item Source TCP port
\item Destination TCP port
\end{itemize}
\item Else if VIRTIO_NET_HASH_TYPE_UDP_EX is set and the
packet has a UDPv6 header, the hash is calculated over the following fields:
\begin{itemize}
\item Home address from the home address option in the IPv6 destination options header. If the extension header is not present, use the Source IPv6 address.
\item IPv6 address that is contained in the Routing-Header-Type-2 from the associated extension header. If the extension header is not present, use the Destination IPv6 address.
\item Source UDP port
\item Destination UDP port
\end{itemize}
\item Else if VIRTIO_NET_HASH_TYPE_IP_EX is set, the hash is
calculated over the following fields:
\begin{itemize}
\item Home address from the home address option in the IPv6 destination options header. If the extension header is not present, use the Source IPv6 address.
\item IPv6 address that is contained in the Routing-Header-Type-2 from the associated extension header. If the extension header is not present, use the Destination IPv6 address.
\end{itemize}
\item Else skip IPv6 extension headers and calculate the hash as
defined for an IPv6 packet without extension headers
(see \ref{sec:Device Types / Network Device / Device Operation / Processing of Incoming Packets / Hash calculation for incoming packets / IPv6 packets without extension header}).
\end{itemize}

\paragraph{Inner Header Hash}
\label{sec:Device Types / Network Device / Device Operation / Processing of Incoming Packets / Inner Header Hash}

If VIRTIO_NET_F_HASH_TUNNEL has been negotiated, the driver can send the command
VIRTIO_NET_CTRL_HASH_TUNNEL_SET to configure the calculation of the inner header hash.

\begin{lstlisting}
struct virtnet_hash_tunnel {
    le32 enabled_tunnel_types;
};

#define VIRTIO_NET_CTRL_HASH_TUNNEL 7
 #define VIRTIO_NET_CTRL_HASH_TUNNEL_SET 0
\end{lstlisting}

Field \field{enabled_tunnel_types} contains the bitmask of encapsulation types enabled for inner header hash.
See \ref{sec:Device Types / Network Device / Device Operation / Processing of Incoming Packets /
Hash calculation for incoming packets / Encapsulation types supported/enabled for inner header hash}.

The class VIRTIO_NET_CTRL_HASH_TUNNEL has one command:
VIRTIO_NET_CTRL_HASH_TUNNEL_SET sets \field{enabled_tunnel_types} for the device using the
virtnet_hash_tunnel structure, which is read-only for the device.

Inner header hash is disabled by VIRTIO_NET_CTRL_HASH_TUNNEL_SET with \field{enabled_tunnel_types} set to 0.

Initially (before the driver sends any VIRTIO_NET_CTRL_HASH_TUNNEL_SET command) all
encapsulation types are disabled for inner header hash.

\subparagraph{Encapsulated packet}
\label{sec:Device Types / Network Device / Device Operation / Processing of Incoming Packets / Hash calculation for incoming packets / Encapsulated packet}

Multiple tunneling protocols allow encapsulating an inner, payload packet in an outer, encapsulated packet.
The encapsulated packet thus contains an outer header and an inner header, and the device calculates the
hash over either the inner header or the outer header.

If VIRTIO_NET_F_HASH_TUNNEL is negotiated and a received encapsulated packet's outer header matches one of the
encapsulation types enabled in \field{enabled_tunnel_types}, then the device uses the inner header for hash
calculations (only a single level of encapsulation is currently supported).

If VIRTIO_NET_F_HASH_TUNNEL is negotiated and a received packet's (outer) header does not match any encapsulation
types enabled in \field{enabled_tunnel_types}, then the device uses the outer header for hash calculations.

\subparagraph{Encapsulation types supported/enabled for inner header hash}
\label{sec:Device Types / Network Device / Device Operation / Processing of Incoming Packets /
Hash calculation for incoming packets / Encapsulation types supported/enabled for inner header hash}

Encapsulation types applicable for inner header hash:
\begin{lstlisting}[escapechar=|]
#define VIRTIO_NET_HASH_TUNNEL_TYPE_GRE_2784    (1 << 0) /* |\hyperref[intro:rfc2784]{[RFC2784]}| */
#define VIRTIO_NET_HASH_TUNNEL_TYPE_GRE_2890    (1 << 1) /* |\hyperref[intro:rfc2890]{[RFC2890]}| */
#define VIRTIO_NET_HASH_TUNNEL_TYPE_GRE_7676    (1 << 2) /* |\hyperref[intro:rfc7676]{[RFC7676]}| */
#define VIRTIO_NET_HASH_TUNNEL_TYPE_GRE_UDP     (1 << 3) /* |\hyperref[intro:rfc8086]{[GRE-in-UDP]}| */
#define VIRTIO_NET_HASH_TUNNEL_TYPE_VXLAN       (1 << 4) /* |\hyperref[intro:vxlan]{[VXLAN]}| */
#define VIRTIO_NET_HASH_TUNNEL_TYPE_VXLAN_GPE   (1 << 5) /* |\hyperref[intro:vxlan-gpe]{[VXLAN-GPE]}| */
#define VIRTIO_NET_HASH_TUNNEL_TYPE_GENEVE      (1 << 6) /* |\hyperref[intro:geneve]{[GENEVE]}| */
#define VIRTIO_NET_HASH_TUNNEL_TYPE_IPIP        (1 << 7) /* |\hyperref[intro:ipip]{[IPIP]}| */
#define VIRTIO_NET_HASH_TUNNEL_TYPE_NVGRE       (1 << 8) /* |\hyperref[intro:nvgre]{[NVGRE]}| */
\end{lstlisting}

\subparagraph{Advice}
Example uses of the inner header hash:
\begin{itemize}
\item Legacy tunneling protocols, lacking the outer header entropy, can use RSS with the inner header hash to
      distribute flows with identical outer but different inner headers across various queues, improving performance.
\item Identify an inner flow distributed across multiple outer tunnels.
\end{itemize}

As using the inner header hash completely discards the outer header entropy, care must be taken
if the inner header is controlled by an adversary, as the adversary can then intentionally create
configurations with insufficient entropy.

Besides disabling the inner header hash, mitigations would depend on how the hash is used. When the hash
use is limited to the RSS queue selection, the inner header hash may have quality of service (QoS) limitations.

\devicenormative{\subparagraph}{Inner Header Hash}{Device Types / Network Device / Device Operation / Control Virtqueue / Inner Header Hash}

If the (outer) header of the received packet does not match any encapsulation types enabled
in \field{enabled_tunnel_types}, the device MUST calculate the hash on the outer header.

If the device receives any bits in \field{enabled_tunnel_types} which are not set in \field{supported_tunnel_types},
it SHOULD respond to the VIRTIO_NET_CTRL_HASH_TUNNEL_SET command with VIRTIO_NET_ERR.

If the driver sets \field{enabled_tunnel_types} to 0 through VIRTIO_NET_CTRL_HASH_TUNNEL_SET or upon the device reset,
the device MUST disable the inner header hash for all encapsulation types.

\drivernormative{\subparagraph}{Inner Header Hash}{Device Types / Network Device / Device Operation / Control Virtqueue / Inner Header Hash}

The driver MUST have negotiated the VIRTIO_NET_F_HASH_TUNNEL feature when issuing the VIRTIO_NET_CTRL_HASH_TUNNEL_SET command.

The driver MUST NOT set any bits in \field{enabled_tunnel_types} which are not set in \field{supported_tunnel_types}.

The driver MUST ignore bits in \field{supported_tunnel_types} which are not documented in this specification.

\paragraph{Hash reporting for incoming packets}
\label{sec:Device Types / Network Device / Device Operation / Processing of Incoming Packets / Hash reporting for incoming packets}

If VIRTIO_NET_F_HASH_REPORT was negotiated and
 the device has calculated the hash for the packet, the device fills \field{hash_report} with the report type of calculated hash
and \field{hash_value} with the value of calculated hash.

If VIRTIO_NET_F_HASH_REPORT was negotiated but due to any reason the
hash was not calculated, the device sets \field{hash_report} to VIRTIO_NET_HASH_REPORT_NONE.

Possible values that the device can report in \field{hash_report} are defined below.
They correspond to supported hash types defined in
\ref{sec:Device Types / Network Device / Device Operation / Processing of Incoming Packets / Hash calculation for incoming packets / Supported/enabled hash types}
as follows:

VIRTIO_NET_HASH_TYPE_XXX = 1 << (VIRTIO_NET_HASH_REPORT_XXX - 1)

\begin{lstlisting}
#define VIRTIO_NET_HASH_REPORT_NONE            0
#define VIRTIO_NET_HASH_REPORT_IPv4            1
#define VIRTIO_NET_HASH_REPORT_TCPv4           2
#define VIRTIO_NET_HASH_REPORT_UDPv4           3
#define VIRTIO_NET_HASH_REPORT_IPv6            4
#define VIRTIO_NET_HASH_REPORT_TCPv6           5
#define VIRTIO_NET_HASH_REPORT_UDPv6           6
#define VIRTIO_NET_HASH_REPORT_IPv6_EX         7
#define VIRTIO_NET_HASH_REPORT_TCPv6_EX        8
#define VIRTIO_NET_HASH_REPORT_UDPv6_EX        9
\end{lstlisting}

\subsubsection{Control Virtqueue}\label{sec:Device Types / Network Device / Device Operation / Control Virtqueue}

The driver uses the control virtqueue (if VIRTIO_NET_F_CTRL_VQ is
negotiated) to send commands to manipulate various features of
the device which would not easily map into the configuration
space.

All commands are of the following form:

\begin{lstlisting}
struct virtio_net_ctrl {
        u8 class;
        u8 command;
        u8 command-specific-data[];
        u8 ack;
        u8 command-specific-result[];
};

/* ack values */
#define VIRTIO_NET_OK     0
#define VIRTIO_NET_ERR    1
\end{lstlisting}

The \field{class}, \field{command} and command-specific-data are set by the
driver, and the device sets the \field{ack} byte and optionally
\field{command-specific-result}. There is little the driver can
do except issue a diagnostic if \field{ack} is not VIRTIO_NET_OK.

The command VIRTIO_NET_CTRL_STATS_QUERY and VIRTIO_NET_CTRL_STATS_GET contain
\field{command-specific-result}.

\paragraph{Packet Receive Filtering}\label{sec:Device Types / Network Device / Device Operation / Control Virtqueue / Packet Receive Filtering}
\label{sec:Device Types / Network Device / Device Operation / Control Virtqueue / Setting Promiscuous Mode}%old label for latexdiff

If the VIRTIO_NET_F_CTRL_RX and VIRTIO_NET_F_CTRL_RX_EXTRA
features are negotiated, the driver can send control commands for
promiscuous mode, multicast, unicast and broadcast receiving.

\begin{note}
In general, these commands are best-effort: unwanted
packets could still arrive.
\end{note}

\begin{lstlisting}
#define VIRTIO_NET_CTRL_RX    0
 #define VIRTIO_NET_CTRL_RX_PROMISC      0
 #define VIRTIO_NET_CTRL_RX_ALLMULTI     1
 #define VIRTIO_NET_CTRL_RX_ALLUNI       2
 #define VIRTIO_NET_CTRL_RX_NOMULTI      3
 #define VIRTIO_NET_CTRL_RX_NOUNI        4
 #define VIRTIO_NET_CTRL_RX_NOBCAST      5
\end{lstlisting}


\devicenormative{\subparagraph}{Packet Receive Filtering}{Device Types / Network Device / Device Operation / Control Virtqueue / Packet Receive Filtering}

If the VIRTIO_NET_F_CTRL_RX feature has been negotiated,
the device MUST support the following VIRTIO_NET_CTRL_RX class
commands:
\begin{itemize}
\item VIRTIO_NET_CTRL_RX_PROMISC turns promiscuous mode on and
off. The command-specific-data is one byte containing 0 (off) or
1 (on). If promiscuous mode is on, the device SHOULD receive all
incoming packets.
This SHOULD take effect even if one of the other modes set by
a VIRTIO_NET_CTRL_RX class command is on.
\item VIRTIO_NET_CTRL_RX_ALLMULTI turns all-multicast receive on and
off. The command-specific-data is one byte containing 0 (off) or
1 (on). When all-multicast receive is on the device SHOULD allow
all incoming multicast packets.
\end{itemize}

If the VIRTIO_NET_F_CTRL_RX_EXTRA feature has been negotiated,
the device MUST support the following VIRTIO_NET_CTRL_RX class
commands:
\begin{itemize}
\item VIRTIO_NET_CTRL_RX_ALLUNI turns all-unicast receive on and
off. The command-specific-data is one byte containing 0 (off) or
1 (on). When all-unicast receive is on the device SHOULD allow
all incoming unicast packets.
\item VIRTIO_NET_CTRL_RX_NOMULTI suppresses multicast receive.
The command-specific-data is one byte containing 0 (multicast
receive allowed) or 1 (multicast receive suppressed).
When multicast receive is suppressed, the device SHOULD NOT
send multicast packets to the driver.
This SHOULD take effect even if VIRTIO_NET_CTRL_RX_ALLMULTI is on.
This filter SHOULD NOT apply to broadcast packets.
\item VIRTIO_NET_CTRL_RX_NOUNI suppresses unicast receive.
The command-specific-data is one byte containing 0 (unicast
receive allowed) or 1 (unicast receive suppressed).
When unicast receive is suppressed, the device SHOULD NOT
send unicast packets to the driver.
This SHOULD take effect even if VIRTIO_NET_CTRL_RX_ALLUNI is on.
\item VIRTIO_NET_CTRL_RX_NOBCAST suppresses broadcast receive.
The command-specific-data is one byte containing 0 (broadcast
receive allowed) or 1 (broadcast receive suppressed).
When broadcast receive is suppressed, the device SHOULD NOT
send broadcast packets to the driver.
This SHOULD take effect even if VIRTIO_NET_CTRL_RX_ALLMULTI is on.
\end{itemize}

\drivernormative{\subparagraph}{Packet Receive Filtering}{Device Types / Network Device / Device Operation / Control Virtqueue / Packet Receive Filtering}

If the VIRTIO_NET_F_CTRL_RX feature has not been negotiated,
the driver MUST NOT issue commands VIRTIO_NET_CTRL_RX_PROMISC or
VIRTIO_NET_CTRL_RX_ALLMULTI.

If the VIRTIO_NET_F_CTRL_RX_EXTRA feature has not been negotiated,
the driver MUST NOT issue commands
 VIRTIO_NET_CTRL_RX_ALLUNI,
 VIRTIO_NET_CTRL_RX_NOMULTI,
 VIRTIO_NET_CTRL_RX_NOUNI or
 VIRTIO_NET_CTRL_RX_NOBCAST.

\paragraph{Setting MAC Address Filtering}\label{sec:Device Types / Network Device / Device Operation / Control Virtqueue / Setting MAC Address Filtering}

If the VIRTIO_NET_F_CTRL_RX feature is negotiated, the driver can
send control commands for MAC address filtering.

\begin{lstlisting}
struct virtio_net_ctrl_mac {
        le32 entries;
        u8 macs[entries][6];
};

#define VIRTIO_NET_CTRL_MAC    1
 #define VIRTIO_NET_CTRL_MAC_TABLE_SET        0
 #define VIRTIO_NET_CTRL_MAC_ADDR_SET         1
\end{lstlisting}

The device can filter incoming packets by any number of destination
MAC addresses\footnote{Since there are no guarantees, it can use a hash filter or
silently switch to allmulti or promiscuous mode if it is given too
many addresses.
}. This table is set using the class
VIRTIO_NET_CTRL_MAC and the command VIRTIO_NET_CTRL_MAC_TABLE_SET. The
command-specific-data is two variable length tables of 6-byte MAC
addresses (as described in struct virtio_net_ctrl_mac). The first table contains unicast addresses, and the second
contains multicast addresses.

The VIRTIO_NET_CTRL_MAC_ADDR_SET command is used to set the
default MAC address which rx filtering
accepts (and if VIRTIO_NET_F_MAC has been negotiated,
this will be reflected in \field{mac} in config space).

The command-specific-data for VIRTIO_NET_CTRL_MAC_ADDR_SET is
the 6-byte MAC address.

\devicenormative{\subparagraph}{Setting MAC Address Filtering}{Device Types / Network Device / Device Operation / Control Virtqueue / Setting MAC Address Filtering}

The device MUST have an empty MAC filtering table on reset.

The device MUST update the MAC filtering table before it consumes
the VIRTIO_NET_CTRL_MAC_TABLE_SET command.

The device MUST update \field{mac} in config space before it consumes
the VIRTIO_NET_CTRL_MAC_ADDR_SET command, if VIRTIO_NET_F_MAC has
been negotiated.

The device SHOULD drop incoming packets which have a destination MAC which
matches neither the \field{mac} (or that set with VIRTIO_NET_CTRL_MAC_ADDR_SET)
nor the MAC filtering table.

\drivernormative{\subparagraph}{Setting MAC Address Filtering}{Device Types / Network Device / Device Operation / Control Virtqueue / Setting MAC Address Filtering}

If VIRTIO_NET_F_CTRL_RX has not been negotiated,
the driver MUST NOT issue VIRTIO_NET_CTRL_MAC class commands.

If VIRTIO_NET_F_CTRL_RX has been negotiated,
the driver SHOULD issue VIRTIO_NET_CTRL_MAC_ADDR_SET
to set the default mac if it is different from \field{mac}.

The driver MUST follow the VIRTIO_NET_CTRL_MAC_TABLE_SET command
by a le32 number, followed by that number of non-multicast
MAC addresses, followed by another le32 number, followed by
that number of multicast addresses.  Either number MAY be 0.

\subparagraph{Legacy Interface: Setting MAC Address Filtering}\label{sec:Device Types / Network Device / Device Operation / Control Virtqueue / Setting MAC Address Filtering / Legacy Interface: Setting MAC Address Filtering}
When using the legacy interface, transitional devices and drivers
MUST format \field{entries} in struct virtio_net_ctrl_mac
according to the native endian of the guest rather than
(necessarily when not using the legacy interface) little-endian.

Legacy drivers that didn't negotiate VIRTIO_NET_F_CTRL_MAC_ADDR
changed \field{mac} in config space when NIC is accepting
incoming packets. These drivers always wrote the mac value from
first to last byte, therefore after detecting such drivers,
a transitional device MAY defer MAC update, or MAY defer
processing incoming packets until driver writes the last byte
of \field{mac} in the config space.

\paragraph{VLAN Filtering}\label{sec:Device Types / Network Device / Device Operation / Control Virtqueue / VLAN Filtering}

If the driver negotiates the VIRTIO_NET_F_CTRL_VLAN feature, it
can control a VLAN filter table in the device. The VLAN filter
table applies only to VLAN tagged packets.

When VIRTIO_NET_F_CTRL_VLAN is negotiated, the device starts with
an empty VLAN filter table.

\begin{note}
Similar to the MAC address based filtering, the VLAN filtering
is also best-effort: unwanted packets could still arrive.
\end{note}

\begin{lstlisting}
#define VIRTIO_NET_CTRL_VLAN       2
 #define VIRTIO_NET_CTRL_VLAN_ADD             0
 #define VIRTIO_NET_CTRL_VLAN_DEL             1
\end{lstlisting}

Both the VIRTIO_NET_CTRL_VLAN_ADD and VIRTIO_NET_CTRL_VLAN_DEL
command take a little-endian 16-bit VLAN id as the command-specific-data.

VIRTIO_NET_CTRL_VLAN_ADD command adds the specified VLAN to the
VLAN filter table.

VIRTIO_NET_CTRL_VLAN_DEL command removes the specified VLAN from
the VLAN filter table.

\devicenormative{\subparagraph}{VLAN Filtering}{Device Types / Network Device / Device Operation / Control Virtqueue / VLAN Filtering}

When VIRTIO_NET_F_CTRL_VLAN is not negotiated, the device MUST
accept all VLAN tagged packets.

When VIRTIO_NET_F_CTRL_VLAN is negotiated, the device MUST
accept all VLAN tagged packets whose VLAN tag is present in
the VLAN filter table and SHOULD drop all VLAN tagged packets
whose VLAN tag is absent in the VLAN filter table.

\subparagraph{Legacy Interface: VLAN Filtering}\label{sec:Device Types / Network Device / Device Operation / Control Virtqueue / VLAN Filtering / Legacy Interface: VLAN Filtering}
When using the legacy interface, transitional devices and drivers
MUST format the VLAN id
according to the native endian of the guest rather than
(necessarily when not using the legacy interface) little-endian.

\paragraph{Gratuitous Packet Sending}\label{sec:Device Types / Network Device / Device Operation / Control Virtqueue / Gratuitous Packet Sending}

If the driver negotiates the VIRTIO_NET_F_GUEST_ANNOUNCE (depends
on VIRTIO_NET_F_CTRL_VQ), the device can ask the driver to send gratuitous
packets; this is usually done after the guest has been physically
migrated, and needs to announce its presence on the new network
links. (As hypervisor does not have the knowledge of guest
network configuration (eg. tagged vlan) it is simplest to prod
the guest in this way).

\begin{lstlisting}
#define VIRTIO_NET_CTRL_ANNOUNCE       3
 #define VIRTIO_NET_CTRL_ANNOUNCE_ACK             0
\end{lstlisting}

The driver checks VIRTIO_NET_S_ANNOUNCE bit in the device configuration \field{status} field
when it notices the changes of device configuration. The
command VIRTIO_NET_CTRL_ANNOUNCE_ACK is used to indicate that
driver has received the notification and device clears the
VIRTIO_NET_S_ANNOUNCE bit in \field{status}.

Processing this notification involves:

\begin{enumerate}
\item Sending the gratuitous packets (eg. ARP) or marking there are pending
  gratuitous packets to be sent and letting deferred routine to
  send them.

\item Sending VIRTIO_NET_CTRL_ANNOUNCE_ACK command through control
  vq.
\end{enumerate}

\drivernormative{\subparagraph}{Gratuitous Packet Sending}{Device Types / Network Device / Device Operation / Control Virtqueue / Gratuitous Packet Sending}

If the driver negotiates VIRTIO_NET_F_GUEST_ANNOUNCE, it SHOULD notify
network peers of its new location after it sees the VIRTIO_NET_S_ANNOUNCE bit
in \field{status}.  The driver MUST send a command on the command queue
with class VIRTIO_NET_CTRL_ANNOUNCE and command VIRTIO_NET_CTRL_ANNOUNCE_ACK.

\devicenormative{\subparagraph}{Gratuitous Packet Sending}{Device Types / Network Device / Device Operation / Control Virtqueue / Gratuitous Packet Sending}

If VIRTIO_NET_F_GUEST_ANNOUNCE is negotiated, the device MUST clear the
VIRTIO_NET_S_ANNOUNCE bit in \field{status} upon receipt of a command buffer
with class VIRTIO_NET_CTRL_ANNOUNCE and command VIRTIO_NET_CTRL_ANNOUNCE_ACK
before marking the buffer as used.

\paragraph{Device operation in multiqueue mode}\label{sec:Device Types / Network Device / Device Operation / Control Virtqueue / Device operation in multiqueue mode}

This specification defines the following modes that a device MAY implement for operation with multiple transmit/receive virtqueues:
\begin{itemize}
\item Automatic receive steering as defined in \ref{sec:Device Types / Network Device / Device Operation / Control Virtqueue / Automatic receive steering in multiqueue mode}.
 If a device supports this mode, it offers the VIRTIO_NET_F_MQ feature bit.
\item Receive-side scaling as defined in \ref{devicenormative:Device Types / Network Device / Device Operation / Control Virtqueue / Receive-side scaling (RSS) / RSS processing}.
 If a device supports this mode, it offers the VIRTIO_NET_F_RSS feature bit.
\end{itemize}

A device MAY support one of these features or both. The driver MAY negotiate any set of these features that the device supports.

Multiqueue is disabled by default.

The driver enables multiqueue by sending a command using \field{class} VIRTIO_NET_CTRL_MQ. The \field{command} selects the mode of multiqueue operation, as follows:
\begin{lstlisting}
#define VIRTIO_NET_CTRL_MQ    4
 #define VIRTIO_NET_CTRL_MQ_VQ_PAIRS_SET        0 (for automatic receive steering)
 #define VIRTIO_NET_CTRL_MQ_RSS_CONFIG          1 (for configurable receive steering)
 #define VIRTIO_NET_CTRL_MQ_HASH_CONFIG         2 (for configurable hash calculation)
\end{lstlisting}

If more than one multiqueue mode is negotiated, the resulting device configuration is defined by the last command sent by the driver.

\paragraph{Automatic receive steering in multiqueue mode}\label{sec:Device Types / Network Device / Device Operation / Control Virtqueue / Automatic receive steering in multiqueue mode}

If the driver negotiates the VIRTIO_NET_F_MQ feature bit (depends on VIRTIO_NET_F_CTRL_VQ), it MAY transmit outgoing packets on one
of the multiple transmitq1\ldots transmitqN and ask the device to
queue incoming packets into one of the multiple receiveq1\ldots receiveqN
depending on the packet flow.

The driver enables multiqueue by
sending the VIRTIO_NET_CTRL_MQ_VQ_PAIRS_SET command, specifying
the number of the transmit and receive queues to be used up to
\field{max_virtqueue_pairs}; subsequently,
transmitq1\ldots transmitqn and receiveq1\ldots receiveqn where
n=\field{virtqueue_pairs} MAY be used.
\begin{lstlisting}
struct virtio_net_ctrl_mq_pairs_set {
       le16 virtqueue_pairs;
};
#define VIRTIO_NET_CTRL_MQ_VQ_PAIRS_MIN        1
#define VIRTIO_NET_CTRL_MQ_VQ_PAIRS_MAX        0x8000

\end{lstlisting}

When multiqueue is enabled by VIRTIO_NET_CTRL_MQ_VQ_PAIRS_SET command, the device MUST use automatic receive steering
based on packet flow. Programming of the receive steering
classificator is implicit. After the driver transmitted a packet of a
flow on transmitqX, the device SHOULD cause incoming packets for that flow to
be steered to receiveqX. For uni-directional protocols, or where
no packets have been transmitted yet, the device MAY steer a packet
to a random queue out of the specified receiveq1\ldots receiveqn.

Multiqueue is disabled by VIRTIO_NET_CTRL_MQ_VQ_PAIRS_SET with \field{virtqueue_pairs} to 1 (this is
the default) and waiting for the device to use the command buffer.

\drivernormative{\subparagraph}{Automatic receive steering in multiqueue mode}{Device Types / Network Device / Device Operation / Control Virtqueue / Automatic receive steering in multiqueue mode}

The driver MUST configure the virtqueues before enabling them with the
VIRTIO_NET_CTRL_MQ_VQ_PAIRS_SET command.

The driver MUST NOT request a \field{virtqueue_pairs} of 0 or
greater than \field{max_virtqueue_pairs} in the device configuration space.

The driver MUST queue packets only on any transmitq1 before the
VIRTIO_NET_CTRL_MQ_VQ_PAIRS_SET command.

The driver MUST NOT queue packets on transmit queues greater than
\field{virtqueue_pairs} once it has placed the VIRTIO_NET_CTRL_MQ_VQ_PAIRS_SET command in the available ring.

\devicenormative{\subparagraph}{Automatic receive steering in multiqueue mode}{Device Types / Network Device / Device Operation / Control Virtqueue / Automatic receive steering in multiqueue mode}

After initialization of reset, the device MUST queue packets only on receiveq1.

The device MUST NOT queue packets on receive queues greater than
\field{virtqueue_pairs} once it has placed the
VIRTIO_NET_CTRL_MQ_VQ_PAIRS_SET command in a used buffer.

If the destination receive queue is being reset (See \ref{sec:Basic Facilities of a Virtio Device / Virtqueues / Virtqueue Reset}),
the device SHOULD re-select another random queue. If all receive queues are
being reset, the device MUST drop the packet.

\subparagraph{Legacy Interface: Automatic receive steering in multiqueue mode}\label{sec:Device Types / Network Device / Device Operation / Control Virtqueue / Automatic receive steering in multiqueue mode / Legacy Interface: Automatic receive steering in multiqueue mode}
When using the legacy interface, transitional devices and drivers
MUST format \field{virtqueue_pairs}
according to the native endian of the guest rather than
(necessarily when not using the legacy interface) little-endian.

\subparagraph{Hash calculation}\label{sec:Device Types / Network Device / Device Operation / Control Virtqueue / Automatic receive steering in multiqueue mode / Hash calculation}
If VIRTIO_NET_F_HASH_REPORT was negotiated and the device uses automatic receive steering,
the device MUST support a command to configure hash calculation parameters.

The driver provides parameters for hash calculation as follows:

\field{class} VIRTIO_NET_CTRL_MQ, \field{command} VIRTIO_NET_CTRL_MQ_HASH_CONFIG.

The \field{command-specific-data} has following format:
\begin{lstlisting}
struct virtio_net_hash_config {
    le32 hash_types;
    le16 reserved[4];
    u8 hash_key_length;
    u8 hash_key_data[hash_key_length];
};
\end{lstlisting}
Field \field{hash_types} contains a bitmask of allowed hash types as
defined in
\ref{sec:Device Types / Network Device / Device Operation / Processing of Incoming Packets / Hash calculation for incoming packets / Supported/enabled hash types}.
Initially the device has all hash types disabled and reports only VIRTIO_NET_HASH_REPORT_NONE.

Field \field{reserved} MUST contain zeroes. It is defined to make the structure to match the layout of virtio_net_rss_config structure,
defined in \ref{sec:Device Types / Network Device / Device Operation / Control Virtqueue / Receive-side scaling (RSS)}.

Fields \field{hash_key_length} and \field{hash_key_data} define the key to be used in hash calculation.

\paragraph{Receive-side scaling (RSS)}\label{sec:Device Types / Network Device / Device Operation / Control Virtqueue / Receive-side scaling (RSS)}
A device offers the feature VIRTIO_NET_F_RSS if it supports RSS receive steering with Toeplitz hash calculation and configurable parameters.

A driver queries RSS capabilities of the device by reading device configuration as defined in \ref{sec:Device Types / Network Device / Device configuration layout}

\subparagraph{Setting RSS parameters}\label{sec:Device Types / Network Device / Device Operation / Control Virtqueue / Receive-side scaling (RSS) / Setting RSS parameters}

Driver sends a VIRTIO_NET_CTRL_MQ_RSS_CONFIG command using the following format for \field{command-specific-data}:
\begin{lstlisting}
struct rss_rq_id {
   le16 vq_index_1_16: 15; /* Bits 1 to 16 of the virtqueue index */
   le16 reserved: 1; /* Set to zero */
};

struct virtio_net_rss_config {
    le32 hash_types;
    le16 indirection_table_mask;
    struct rss_rq_id unclassified_queue;
    struct rss_rq_id indirection_table[indirection_table_length];
    le16 max_tx_vq;
    u8 hash_key_length;
    u8 hash_key_data[hash_key_length];
};
\end{lstlisting}
Field \field{hash_types} contains a bitmask of allowed hash types as
defined in
\ref{sec:Device Types / Network Device / Device Operation / Processing of Incoming Packets / Hash calculation for incoming packets / Supported/enabled hash types}.

Field \field{indirection_table_mask} is a mask to be applied to
the calculated hash to produce an index in the
\field{indirection_table} array.
Number of entries in \field{indirection_table} is (\field{indirection_table_mask} + 1).

\field{rss_rq_id} is a receive virtqueue id. \field{vq_index_1_16}
consists of bits 1 to 16 of a virtqueue index. For example, a
\field{vq_index_1_16} value of 3 corresponds to virtqueue index 6,
which maps to receiveq4.

Field \field{unclassified_queue} specifies the receive virtqueue id in which to
place unclassified packets.

Field \field{indirection_table} is an array of receive virtqueues ids.

A driver sets \field{max_tx_vq} to inform a device how many transmit virtqueues it may use (transmitq1\ldots transmitq \field{max_tx_vq}).

Fields \field{hash_key_length} and \field{hash_key_data} define the key to be used in hash calculation.

\drivernormative{\subparagraph}{Setting RSS parameters}{Device Types / Network Device / Device Operation / Control Virtqueue / Receive-side scaling (RSS) }

A driver MUST NOT send the VIRTIO_NET_CTRL_MQ_RSS_CONFIG command if the feature VIRTIO_NET_F_RSS has not been negotiated.

A driver MUST fill the \field{indirection_table} array only with
enabled receive virtqueues ids.

The number of entries in \field{indirection_table} (\field{indirection_table_mask} + 1) MUST be a power of two.

A driver MUST use \field{indirection_table_mask} values that are less than \field{rss_max_indirection_table_length} reported by a device.

A driver MUST NOT set any VIRTIO_NET_HASH_TYPE_ flags that are not supported by a device.

\devicenormative{\subparagraph}{RSS processing}{Device Types / Network Device / Device Operation / Control Virtqueue / Receive-side scaling (RSS) / RSS processing}
The device MUST determine the destination queue for a network packet as follows:
\begin{itemize}
\item Calculate the hash of the packet as defined in \ref{sec:Device Types / Network Device / Device Operation / Processing of Incoming Packets / Hash calculation for incoming packets}.
\item If the device did not calculate the hash for the specific packet, the device directs the packet to the receiveq specified by \field{unclassified_queue} of virtio_net_rss_config structure.
\item Apply \field{indirection_table_mask} to the calculated hash
and use the result as the index in the indirection table to get
the destination receive virtqueue id.
\item If the destination receive queue is being reset (See \ref{sec:Basic Facilities of a Virtio Device / Virtqueues / Virtqueue Reset}), the device MUST drop the packet.
\end{itemize}

\paragraph{RSS Context}\label{sec:Device Types / Network Device / Device Operation / Control Virtqueue / RSS Context}

An RSS context consists of configurable parameters specified by \ref{sec:Device Types / Network Device
/ Device Operation / Control Virtqueue / Receive-side scaling (RSS)}.

The RSS configuration supported by VIRTIO_NET_F_RSS is considered the default RSS configuration.

The device offers the feature VIRTIO_NET_F_RSS_CONTEXT if it supports one or multiple RSS contexts
(excluding the default RSS configuration) and configurable parameters.

\subparagraph{Querying RSS Context Capability}\label{sec:Device Types / Network Device / Device Operation / Control Virtqueue / RSS Context / Querying RSS Context Capability}

\begin{lstlisting}
#define VIRTNET_RSS_CTX_CTRL 9
 #define VIRTNET_RSS_CTX_CTRL_CAP_GET  0
 #define VIRTNET_RSS_CTX_CTRL_ADD      1
 #define VIRTNET_RSS_CTX_CTRL_MOD      2
 #define VIRTNET_RSS_CTX_CTRL_DEL      3

struct virtnet_rss_ctx_cap {
    le16 max_rss_contexts;
}
\end{lstlisting}

Field \field{max_rss_contexts} specifies the maximum number of RSS contexts \ref{sec:Device Types / Network Device /
Device Operation / Control Virtqueue / RSS Context} supported by the device.

The driver queries the RSS context capability of the device by sending the command VIRTNET_RSS_CTX_CTRL_CAP_GET
with the structure virtnet_rss_ctx_cap.

For the command VIRTNET_RSS_CTX_CTRL_CAP_GET, the structure virtnet_rss_ctx_cap is write-only for the device.

\subparagraph{Setting RSS Context Parameters}\label{sec:Device Types / Network Device / Device Operation / Control Virtqueue / RSS Context / Setting RSS Context Parameters}

\begin{lstlisting}
struct virtnet_rss_ctx_add_modify {
    le16 rss_ctx_id;
    u8 reserved[6];
    struct virtio_net_rss_config rss;
};

struct virtnet_rss_ctx_del {
    le16 rss_ctx_id;
};
\end{lstlisting}

RSS context parameters:
\begin{itemize}
\item  \field{rss_ctx_id}: ID of the specific RSS context.
\item  \field{rss}: RSS context parameters of the specific RSS context whose id is \field{rss_ctx_id}.
\end{itemize}

\field{reserved} is reserved and it is ignored by the device.

If the feature VIRTIO_NET_F_RSS_CONTEXT has been negotiated, the driver can send the following
VIRTNET_RSS_CTX_CTRL class commands:
\begin{enumerate}
\item VIRTNET_RSS_CTX_CTRL_ADD: use the structure virtnet_rss_ctx_add_modify to
       add an RSS context configured as \field{rss} and id as \field{rss_ctx_id} for the device.
\item VIRTNET_RSS_CTX_CTRL_MOD: use the structure virtnet_rss_ctx_add_modify to
       configure parameters of the RSS context whose id is \field{rss_ctx_id} as \field{rss} for the device.
\item VIRTNET_RSS_CTX_CTRL_DEL: use the structure virtnet_rss_ctx_del to delete
       the RSS context whose id is \field{rss_ctx_id} for the device.
\end{enumerate}

For commands VIRTNET_RSS_CTX_CTRL_ADD and VIRTNET_RSS_CTX_CTRL_MOD, the structure virtnet_rss_ctx_add_modify is read-only for the device.
For the command VIRTNET_RSS_CTX_CTRL_DEL, the structure virtnet_rss_ctx_del is read-only for the device.

\devicenormative{\subparagraph}{RSS Context}{Device Types / Network Device / Device Operation / Control Virtqueue / RSS Context}

The device MUST set \field{max_rss_contexts} to at least 1 if it offers VIRTIO_NET_F_RSS_CONTEXT.

Upon reset, the device MUST clear all previously configured RSS contexts.

\drivernormative{\subparagraph}{RSS Context}{Device Types / Network Device / Device Operation / Control Virtqueue / RSS Context}

The driver MUST have negotiated the VIRTIO_NET_F_RSS_CONTEXT feature when issuing the VIRTNET_RSS_CTX_CTRL class commands.

The driver MUST set \field{rss_ctx_id} to between 1 and \field{max_rss_contexts} inclusive.

The driver MUST NOT send the command VIRTIO_NET_CTRL_MQ_VQ_PAIRS_SET when the device has successfully configured at least one RSS context.

\paragraph{Offloads State Configuration}\label{sec:Device Types / Network Device / Device Operation / Control Virtqueue / Offloads State Configuration}

If the VIRTIO_NET_F_CTRL_GUEST_OFFLOADS feature is negotiated, the driver can
send control commands for dynamic offloads state configuration.

\subparagraph{Setting Offloads State}\label{sec:Device Types / Network Device / Device Operation / Control Virtqueue / Offloads State Configuration / Setting Offloads State}

To configure the offloads, the following layout structure and
definitions are used:

\begin{lstlisting}
le64 offloads;

#define VIRTIO_NET_F_GUEST_CSUM       1
#define VIRTIO_NET_F_GUEST_TSO4       7
#define VIRTIO_NET_F_GUEST_TSO6       8
#define VIRTIO_NET_F_GUEST_ECN        9
#define VIRTIO_NET_F_GUEST_UFO        10
#define VIRTIO_NET_F_GUEST_UDP_TUNNEL_GSO  46
#define VIRTIO_NET_F_GUEST_UDP_TUNNEL_GSO_CSUM 47
#define VIRTIO_NET_F_GUEST_USO4       54
#define VIRTIO_NET_F_GUEST_USO6       55

#define VIRTIO_NET_CTRL_GUEST_OFFLOADS       5
 #define VIRTIO_NET_CTRL_GUEST_OFFLOADS_SET   0
\end{lstlisting}

The class VIRTIO_NET_CTRL_GUEST_OFFLOADS has one command:
VIRTIO_NET_CTRL_GUEST_OFFLOADS_SET applies the new offloads configuration.

le64 value passed as command data is a bitmask, bits set define
offloads to be enabled, bits cleared - offloads to be disabled.

There is a corresponding device feature for each offload. Upon feature
negotiation corresponding offload gets enabled to preserve backward
compatibility.

\drivernormative{\subparagraph}{Setting Offloads State}{Device Types / Network Device / Device Operation / Control Virtqueue / Offloads State Configuration / Setting Offloads State}

A driver MUST NOT enable an offload for which the appropriate feature
has not been negotiated.

\subparagraph{Legacy Interface: Setting Offloads State}\label{sec:Device Types / Network Device / Device Operation / Control Virtqueue / Offloads State Configuration / Setting Offloads State / Legacy Interface: Setting Offloads State}
When using the legacy interface, transitional devices and drivers
MUST format \field{offloads}
according to the native endian of the guest rather than
(necessarily when not using the legacy interface) little-endian.


\paragraph{Notifications Coalescing}\label{sec:Device Types / Network Device / Device Operation / Control Virtqueue / Notifications Coalescing}

If the VIRTIO_NET_F_NOTF_COAL feature is negotiated, the driver can
send commands VIRTIO_NET_CTRL_NOTF_COAL_TX_SET and VIRTIO_NET_CTRL_NOTF_COAL_RX_SET
for notification coalescing.

If the VIRTIO_NET_F_VQ_NOTF_COAL feature is negotiated, the driver can
send commands VIRTIO_NET_CTRL_NOTF_COAL_VQ_SET and VIRTIO_NET_CTRL_NOTF_COAL_VQ_GET
for virtqueue notification coalescing.

\begin{lstlisting}
struct virtio_net_ctrl_coal {
    le32 max_packets;
    le32 max_usecs;
};

struct virtio_net_ctrl_coal_vq {
    le16 vq_index;
    le16 reserved;
    struct virtio_net_ctrl_coal coal;
};

#define VIRTIO_NET_CTRL_NOTF_COAL 6
 #define VIRTIO_NET_CTRL_NOTF_COAL_TX_SET  0
 #define VIRTIO_NET_CTRL_NOTF_COAL_RX_SET 1
 #define VIRTIO_NET_CTRL_NOTF_COAL_VQ_SET 2
 #define VIRTIO_NET_CTRL_NOTF_COAL_VQ_GET 3
\end{lstlisting}

Coalescing parameters:
\begin{itemize}
\item \field{vq_index}: The virtqueue index of an enabled transmit or receive virtqueue.
\item \field{max_usecs} for RX: Maximum number of microseconds to delay a RX notification.
\item \field{max_usecs} for TX: Maximum number of microseconds to delay a TX notification.
\item \field{max_packets} for RX: Maximum number of packets to receive before a RX notification.
\item \field{max_packets} for TX: Maximum number of packets to send before a TX notification.
\end{itemize}

\field{reserved} is reserved and it is ignored by the device.

Read/Write attributes for coalescing parameters:
\begin{itemize}
\item For commands VIRTIO_NET_CTRL_NOTF_COAL_TX_SET and VIRTIO_NET_CTRL_NOTF_COAL_RX_SET, the structure virtio_net_ctrl_coal is write-only for the driver.
\item For the command VIRTIO_NET_CTRL_NOTF_COAL_VQ_SET, the structure virtio_net_ctrl_coal_vq is write-only for the driver.
\item For the command VIRTIO_NET_CTRL_NOTF_COAL_VQ_GET, \field{vq_index} and \field{reserved} are write-only
      for the driver, and the structure virtio_net_ctrl_coal is read-only for the driver.
\end{itemize}

The class VIRTIO_NET_CTRL_NOTF_COAL has the following commands:
\begin{enumerate}
\item VIRTIO_NET_CTRL_NOTF_COAL_TX_SET: use the structure virtio_net_ctrl_coal to set the \field{max_usecs} and \field{max_packets} parameters for all transmit virtqueues.
\item VIRTIO_NET_CTRL_NOTF_COAL_RX_SET: use the structure virtio_net_ctrl_coal to set the \field{max_usecs} and \field{max_packets} parameters for all receive virtqueues.
\item VIRTIO_NET_CTRL_NOTF_COAL_VQ_SET: use the structure virtio_net_ctrl_coal_vq to set the \field{max_usecs} and \field{max_packets} parameters
                                        for an enabled transmit/receive virtqueue whose index is \field{vq_index}.
\item VIRTIO_NET_CTRL_NOTF_COAL_VQ_GET: use the structure virtio_net_ctrl_coal_vq to get the \field{max_usecs} and \field{max_packets} parameters
                                        for an enabled transmit/receive virtqueue whose index is \field{vq_index}.
\end{enumerate}

The device may generate notifications more or less frequently than specified by set commands of the VIRTIO_NET_CTRL_NOTF_COAL class.

If coalescing parameters are being set, the device applies the last coalescing parameters set for a
virtqueue, regardless of the command used to set the parameters. Use the following command sequence
with two pairs of virtqueues as an example:
Each of the following commands sets \field{max_usecs} and \field{max_packets} parameters for virtqueues.
\begin{itemize}
\item Command1: VIRTIO_NET_CTRL_NOTF_COAL_RX_SET sets coalescing parameters for virtqueues having index 0 and index 2. Virtqueues having index 1 and index 3 retain their previous parameters.
\item Command2: VIRTIO_NET_CTRL_NOTF_COAL_VQ_SET with \field{vq_index} = 0 sets coalescing parameters for virtqueue having index 0. Virtqueue having index 2 retains the parameters from command1.
\item Command3: VIRTIO_NET_CTRL_NOTF_COAL_VQ_GET with \field{vq_index} = 0, the device responds with coalescing parameters of vq_index 0 set by command2.
\item Command4: VIRTIO_NET_CTRL_NOTF_COAL_VQ_SET with \field{vq_index} = 1 sets coalescing parameters for virtqueue having index 1. Virtqueue having index 3 retains its previous parameters.
\item Command5: VIRTIO_NET_CTRL_NOTF_COAL_TX_SET sets coalescing parameters for virtqueues having index 1 and index 3, and overrides the parameters set by command4.
\item Command6: VIRTIO_NET_CTRL_NOTF_COAL_VQ_GET with \field{vq_index} = 1, the device responds with coalescing parameters of index 1 set by command5.
\end{itemize}

\subparagraph{Operation}\label{sec:Device Types / Network Device / Device Operation / Control Virtqueue / Notifications Coalescing / Operation}

The device sends a used buffer notification once the notification conditions are met and if the notifications are not suppressed as explained in \ref{sec:Basic Facilities of a Virtio Device / Virtqueues / Used Buffer Notification Suppression}.

When the device has non-zero \field{max_usecs} and non-zero \field{max_packets}, it starts counting microseconds and packets upon receiving/sending a packet.
The device counts packets and microseconds for each receive virtqueue and transmit virtqueue separately.
In this case, the notification conditions are met when \field{max_usecs} microseconds elapse, or upon sending/receiving \field{max_packets} packets, whichever happens first.
Afterwards, the device waits for the next packet and starts counting packets and microseconds again.

When the device has \field{max_usecs} = 0 or \field{max_packets} = 0, the notification conditions are met after every packet received/sent.

\subparagraph{RX Example}\label{sec:Device Types / Network Device / Device Operation / Control Virtqueue / Notifications Coalescing / RX Example}

If, for example:
\begin{itemize}
\item \field{max_usecs} = 10.
\item \field{max_packets} = 15.
\end{itemize}
then each receive virtqueue of a device will operate as follows:
\begin{itemize}
\item The device will count packets received on each virtqueue until it accumulates 15, or until 10 microseconds elapsed since the first one was received.
\item If the notifications are not suppressed by the driver, the device will send an used buffer notification, otherwise, the device will not send an used buffer notification as long as the notifications are suppressed.
\end{itemize}

\subparagraph{TX Example}\label{sec:Device Types / Network Device / Device Operation / Control Virtqueue / Notifications Coalescing / TX Example}

If, for example:
\begin{itemize}
\item \field{max_usecs} = 10.
\item \field{max_packets} = 15.
\end{itemize}
then each transmit virtqueue of a device will operate as follows:
\begin{itemize}
\item The device will count packets sent on each virtqueue until it accumulates 15, or until 10 microseconds elapsed since the first one was sent.
\item If the notifications are not suppressed by the driver, the device will send an used buffer notification, otherwise, the device will not send an used buffer notification as long as the notifications are suppressed.
\end{itemize}

\subparagraph{Notifications When Coalescing Parameters Change}\label{sec:Device Types / Network Device / Device Operation / Control Virtqueue / Notifications Coalescing / Notifications When Coalescing Parameters Change}

When the coalescing parameters of a device change, the device needs to check if the new notification conditions are met and send a used buffer notification if so.

For example, \field{max_packets} = 15 for a device with a single transmit virtqueue: if the device sends 10 packets and afterwards receives a
VIRTIO_NET_CTRL_NOTF_COAL_TX_SET command with \field{max_packets} = 8, then the notification condition is immediately considered to be met;
the device needs to immediately send a used buffer notification, if the notifications are not suppressed by the driver.

\drivernormative{\subparagraph}{Notifications Coalescing}{Device Types / Network Device / Device Operation / Control Virtqueue / Notifications Coalescing}

The driver MUST set \field{vq_index} to the virtqueue index of an enabled transmit or receive virtqueue.

The driver MUST have negotiated the VIRTIO_NET_F_NOTF_COAL feature when issuing commands VIRTIO_NET_CTRL_NOTF_COAL_TX_SET and VIRTIO_NET_CTRL_NOTF_COAL_RX_SET.

The driver MUST have negotiated the VIRTIO_NET_F_VQ_NOTF_COAL feature when issuing commands VIRTIO_NET_CTRL_NOTF_COAL_VQ_SET and VIRTIO_NET_CTRL_NOTF_COAL_VQ_GET.

The driver MUST ignore the values of coalescing parameters received from the VIRTIO_NET_CTRL_NOTF_COAL_VQ_GET command if the device responds with VIRTIO_NET_ERR.

\devicenormative{\subparagraph}{Notifications Coalescing}{Device Types / Network Device / Device Operation / Control Virtqueue / Notifications Coalescing}

The device MUST ignore \field{reserved}.

The device SHOULD respond to VIRTIO_NET_CTRL_NOTF_COAL_TX_SET and VIRTIO_NET_CTRL_NOTF_COAL_RX_SET commands with VIRTIO_NET_ERR if it was not able to change the parameters.

The device MUST respond to the VIRTIO_NET_CTRL_NOTF_COAL_VQ_SET command with VIRTIO_NET_ERR if it was not able to change the parameters.

The device MUST respond to VIRTIO_NET_CTRL_NOTF_COAL_VQ_SET and VIRTIO_NET_CTRL_NOTF_COAL_VQ_GET commands with
VIRTIO_NET_ERR if the designated virtqueue is not an enabled transmit or receive virtqueue.

Upon disabling and re-enabling a transmit virtqueue, the device MUST set the coalescing parameters of the virtqueue
to those configured through the VIRTIO_NET_CTRL_NOTF_COAL_TX_SET command, or, if the driver did not set any TX coalescing parameters, to 0.

Upon disabling and re-enabling a receive virtqueue, the device MUST set the coalescing parameters of the virtqueue
to those configured through the VIRTIO_NET_CTRL_NOTF_COAL_RX_SET command, or, if the driver did not set any RX coalescing parameters, to 0.

The behavior of the device in response to set commands of the VIRTIO_NET_CTRL_NOTF_COAL class is best-effort:
the device MAY generate notifications more or less frequently than specified.

A device SHOULD NOT send used buffer notifications to the driver if the notifications are suppressed, even if the notification conditions are met.

Upon reset, a device MUST initialize all coalescing parameters to 0.

\paragraph{Device Statistics}\label{sec:Device Types / Network Device / Device Operation / Control Virtqueue / Device Statistics}

If the VIRTIO_NET_F_DEVICE_STATS feature is negotiated, the driver can obtain
device statistics from the device by using the following command.

Different types of virtqueues have different statistics. The statistics of the
receiveq are different from those of the transmitq.

The statistics of a certain type of virtqueue are also divided into multiple types
because different types require different features. This enables the expansion
of new statistics.

In one command, the driver can obtain the statistics of one or multiple virtqueues.
Additionally, the driver can obtain multiple type statistics of each virtqueue.

\subparagraph{Query Statistic Capabilities}\label{sec:Device Types / Network Device / Device Operation / Control Virtqueue / Device Statistics / Query Statistic Capabilities}

\begin{lstlisting}
#define VIRTIO_NET_CTRL_STATS         8
#define VIRTIO_NET_CTRL_STATS_QUERY   0
#define VIRTIO_NET_CTRL_STATS_GET     1

struct virtio_net_stats_capabilities {

#define VIRTIO_NET_STATS_TYPE_CVQ       (1 << 32)

#define VIRTIO_NET_STATS_TYPE_RX_BASIC  (1 << 0)
#define VIRTIO_NET_STATS_TYPE_RX_CSUM   (1 << 1)
#define VIRTIO_NET_STATS_TYPE_RX_GSO    (1 << 2)
#define VIRTIO_NET_STATS_TYPE_RX_SPEED  (1 << 3)

#define VIRTIO_NET_STATS_TYPE_TX_BASIC  (1 << 16)
#define VIRTIO_NET_STATS_TYPE_TX_CSUM   (1 << 17)
#define VIRTIO_NET_STATS_TYPE_TX_GSO    (1 << 18)
#define VIRTIO_NET_STATS_TYPE_TX_SPEED  (1 << 19)

    le64 supported_stats_types[1];
}
\end{lstlisting}

To obtain device statistic capability, use the VIRTIO_NET_CTRL_STATS_QUERY
command. When the command completes successfully, \field{command-specific-result}
is in the format of \field{struct virtio_net_stats_capabilities}.

\subparagraph{Get Statistics}\label{sec:Device Types / Network Device / Device Operation / Control Virtqueue / Device Statistics / Get Statistics}

\begin{lstlisting}
struct virtio_net_ctrl_queue_stats {
       struct {
           le16 vq_index;
           le16 reserved[3];
           le64 types_bitmap[1];
       } stats[];
};

struct virtio_net_stats_reply_hdr {
#define VIRTIO_NET_STATS_TYPE_REPLY_CVQ       32

#define VIRTIO_NET_STATS_TYPE_REPLY_RX_BASIC  0
#define VIRTIO_NET_STATS_TYPE_REPLY_RX_CSUM   1
#define VIRTIO_NET_STATS_TYPE_REPLY_RX_GSO    2
#define VIRTIO_NET_STATS_TYPE_REPLY_RX_SPEED  3

#define VIRTIO_NET_STATS_TYPE_REPLY_TX_BASIC  16
#define VIRTIO_NET_STATS_TYPE_REPLY_TX_CSUM   17
#define VIRTIO_NET_STATS_TYPE_REPLY_TX_GSO    18
#define VIRTIO_NET_STATS_TYPE_REPLY_TX_SPEED  19
    u8 type;
    u8 reserved;
    le16 vq_index;
    le16 reserved1;
    le16 size;
}
\end{lstlisting}

To obtain device statistics, use the VIRTIO_NET_CTRL_STATS_GET command with the
\field{command-specific-data} which is in the format of
\field{struct virtio_net_ctrl_queue_stats}. When the command completes
successfully, \field{command-specific-result} contains multiple statistic
results, each statistic result has the \field{struct virtio_net_stats_reply_hdr}
as the header.

The fields of the \field{struct virtio_net_ctrl_queue_stats}:
\begin{description}
    \item [vq_index]
        The index of the virtqueue to obtain the statistics.

    \item [types_bitmap]
        This is a bitmask of the types of statistics to be obtained. Therefore, a
        \field{stats} inside \field{struct virtio_net_ctrl_queue_stats} may
        indicate multiple statistic replies for the virtqueue.
\end{description}

The fields of the \field{struct virtio_net_stats_reply_hdr}:
\begin{description}
    \item [type]
        The type of the reply statistic.

    \item [vq_index]
        The virtqueue index of the reply statistic.

    \item [size]
        The number of bytes for the statistics entry including size of \field{struct virtio_net_stats_reply_hdr}.

\end{description}

\subparagraph{Controlq Statistics}\label{sec:Device Types / Network Device / Device Operation / Control Virtqueue / Device Statistics / Controlq Statistics}

The structure corresponding to the controlq statistics is
\field{struct virtio_net_stats_cvq}. The corresponding type is
VIRTIO_NET_STATS_TYPE_CVQ. This is for the controlq.

\begin{lstlisting}
struct virtio_net_stats_cvq {
    struct virtio_net_stats_reply_hdr hdr;

    le64 command_num;
    le64 ok_num;
};
\end{lstlisting}

\begin{description}
    \item [command_num]
        The number of commands received by the device including the current command.

    \item [ok_num]
        The number of commands completed successfully by the device including the current command.
\end{description}


\subparagraph{Receiveq Basic Statistics}\label{sec:Device Types / Network Device / Device Operation / Control Virtqueue / Device Statistics / Receiveq Basic Statistics}

The structure corresponding to the receiveq basic statistics is
\field{struct virtio_net_stats_rx_basic}. The corresponding type is
VIRTIO_NET_STATS_TYPE_RX_BASIC. This is for the receiveq.

Receiveq basic statistics do not require any feature. As long as the device supports
VIRTIO_NET_F_DEVICE_STATS, the following are the receiveq basic statistics.

\begin{lstlisting}
struct virtio_net_stats_rx_basic {
    struct virtio_net_stats_reply_hdr hdr;

    le64 rx_notifications;

    le64 rx_packets;
    le64 rx_bytes;

    le64 rx_interrupts;

    le64 rx_drops;
    le64 rx_drop_overruns;
};
\end{lstlisting}

The packets described below were all presented on the specified virtqueue.
\begin{description}
    \item [rx_notifications]
        The number of driver notifications received by the device for this
        receiveq.

    \item [rx_packets]
        This is the number of packets passed to the driver by the device.

    \item [rx_bytes]
        This is the bytes of packets passed to the driver by the device.

    \item [rx_interrupts]
        The number of interrupts generated by the device for this receiveq.

    \item [rx_drops]
        This is the number of packets dropped by the device. The count includes
        all types of packets dropped by the device.

    \item [rx_drop_overruns]
        This is the number of packets dropped by the device when no more
        descriptors were available.

\end{description}

\subparagraph{Transmitq Basic Statistics}\label{sec:Device Types / Network Device / Device Operation / Control Virtqueue / Device Statistics / Transmitq Basic Statistics}

The structure corresponding to the transmitq basic statistics is
\field{struct virtio_net_stats_tx_basic}. The corresponding type is
VIRTIO_NET_STATS_TYPE_TX_BASIC. This is for the transmitq.

Transmitq basic statistics do not require any feature. As long as the device supports
VIRTIO_NET_F_DEVICE_STATS, the following are the transmitq basic statistics.

\begin{lstlisting}
struct virtio_net_stats_tx_basic {
    struct virtio_net_stats_reply_hdr hdr;

    le64 tx_notifications;

    le64 tx_packets;
    le64 tx_bytes;

    le64 tx_interrupts;

    le64 tx_drops;
    le64 tx_drop_malformed;
};
\end{lstlisting}

The packets described below are all for a specific virtqueue.
\begin{description}
    \item [tx_notifications]
        The number of driver notifications received by the device for this
        transmitq.

    \item [tx_packets]
        This is the number of packets sent by the device (not the packets
        got from the driver).

    \item [tx_bytes]
        This is the number of bytes sent by the device for all the sent packets
        (not the bytes sent got from the driver).

    \item [tx_interrupts]
        The number of interrupts generated by the device for this transmitq.

    \item [tx_drops]
        The number of packets dropped by the device. The count includes all
        types of packets dropped by the device.

    \item [tx_drop_malformed]
        The number of packets dropped by the device, when the descriptors are
        malformed. For example, the buffer is too short.
\end{description}

\subparagraph{Receiveq CSUM Statistics}\label{sec:Device Types / Network Device / Device Operation / Control Virtqueue / Device Statistics / Receiveq CSUM Statistics}

The structure corresponding to the receiveq checksum statistics is
\field{struct virtio_net_stats_rx_csum}. The corresponding type is
VIRTIO_NET_STATS_TYPE_RX_CSUM. This is for the receiveq.

Only after the VIRTIO_NET_F_GUEST_CSUM is negotiated, the receiveq checksum
statistics can be obtained.

\begin{lstlisting}
struct virtio_net_stats_rx_csum {
    struct virtio_net_stats_reply_hdr hdr;

    le64 rx_csum_valid;
    le64 rx_needs_csum;
    le64 rx_csum_none;
    le64 rx_csum_bad;
};
\end{lstlisting}

The packets described below were all presented on the specified virtqueue.
\begin{description}
    \item [rx_csum_valid]
        The number of packets with VIRTIO_NET_HDR_F_DATA_VALID.

    \item [rx_needs_csum]
        The number of packets with VIRTIO_NET_HDR_F_NEEDS_CSUM.

    \item [rx_csum_none]
        The number of packets without hardware checksum. The packet here refers
        to the non-TCP/UDP packet that the device cannot recognize.

    \item [rx_csum_bad]
        The number of packets with checksum mismatch.

\end{description}

\subparagraph{Transmitq CSUM Statistics}\label{sec:Device Types / Network Device / Device Operation / Control Virtqueue / Device Statistics / Transmitq CSUM Statistics}

The structure corresponding to the transmitq checksum statistics is
\field{struct virtio_net_stats_tx_csum}. The corresponding type is
VIRTIO_NET_STATS_TYPE_TX_CSUM. This is for the transmitq.

Only after the VIRTIO_NET_F_CSUM is negotiated, the transmitq checksum
statistics can be obtained.

The following are the transmitq checksum statistics:

\begin{lstlisting}
struct virtio_net_stats_tx_csum {
    struct virtio_net_stats_reply_hdr hdr;

    le64 tx_csum_none;
    le64 tx_needs_csum;
};
\end{lstlisting}

The packets described below are all for a specific virtqueue.
\begin{description}
    \item [tx_csum_none]
        The number of packets which do not require hardware checksum.

    \item [tx_needs_csum]
        The number of packets which require checksum calculation by the device.

\end{description}

\subparagraph{Receiveq GSO Statistics}\label{sec:Device Types / Network Device / Device Operation / Control Virtqueue / Device Statistics / Receiveq GSO Statistics}

The structure corresponding to the receivq GSO statistics is
\field{struct virtio_net_stats_rx_gso}. The corresponding type is
VIRTIO_NET_STATS_TYPE_RX_GSO. This is for the receiveq.

If one or more of the VIRTIO_NET_F_GUEST_TSO4, VIRTIO_NET_F_GUEST_TSO6
have been negotiated, the receiveq GSO statistics can be obtained.

GSO packets refer to packets passed by the device to the driver where
\field{gso_type} is not VIRTIO_NET_HDR_GSO_NONE.

\begin{lstlisting}
struct virtio_net_stats_rx_gso {
    struct virtio_net_stats_reply_hdr hdr;

    le64 rx_gso_packets;
    le64 rx_gso_bytes;
    le64 rx_gso_packets_coalesced;
    le64 rx_gso_bytes_coalesced;
};
\end{lstlisting}

The packets described below were all presented on the specified virtqueue.
\begin{description}
    \item [rx_gso_packets]
        The number of the GSO packets received by the device.

    \item [rx_gso_bytes]
        The bytes of the GSO packets received by the device.
        This includes the header size of the GSO packet.

    \item [rx_gso_packets_coalesced]
        The number of the GSO packets coalesced by the device.

    \item [rx_gso_bytes_coalesced]
        The bytes of the GSO packets coalesced by the device.
        This includes the header size of the GSO packet.
\end{description}

\subparagraph{Transmitq GSO Statistics}\label{sec:Device Types / Network Device / Device Operation / Control Virtqueue / Device Statistics / Transmitq GSO Statistics}

The structure corresponding to the transmitq GSO statistics is
\field{struct virtio_net_stats_tx_gso}. The corresponding type is
VIRTIO_NET_STATS_TYPE_TX_GSO. This is for the transmitq.

If one or more of the VIRTIO_NET_F_HOST_TSO4, VIRTIO_NET_F_HOST_TSO6,
VIRTIO_NET_F_HOST_USO options have been negotiated, the transmitq GSO statistics
can be obtained.

GSO packets refer to packets passed by the driver to the device where
\field{gso_type} is not VIRTIO_NET_HDR_GSO_NONE.
See more \ref{sec:Device Types / Network Device / Device Operation / Packet
Transmission}.

\begin{lstlisting}
struct virtio_net_stats_tx_gso {
    struct virtio_net_stats_reply_hdr hdr;

    le64 tx_gso_packets;
    le64 tx_gso_bytes;
    le64 tx_gso_segments;
    le64 tx_gso_segments_bytes;
    le64 tx_gso_packets_noseg;
    le64 tx_gso_bytes_noseg;
};
\end{lstlisting}

The packets described below are all for a specific virtqueue.
\begin{description}
    \item [tx_gso_packets]
        The number of the GSO packets sent by the device.

    \item [tx_gso_bytes]
        The bytes of the GSO packets sent by the device.

    \item [tx_gso_segments]
        The number of segments prepared from GSO packets.

    \item [tx_gso_segments_bytes]
        The bytes of segments prepared from GSO packets.

    \item [tx_gso_packets_noseg]
        The number of the GSO packets without segmentation.

    \item [tx_gso_bytes_noseg]
        The bytes of the GSO packets without segmentation.

\end{description}

\subparagraph{Receiveq Speed Statistics}\label{sec:Device Types / Network Device / Device Operation / Control Virtqueue / Device Statistics / Receiveq Speed Statistics}

The structure corresponding to the receiveq speed statistics is
\field{struct virtio_net_stats_rx_speed}. The corresponding type is
VIRTIO_NET_STATS_TYPE_RX_SPEED. This is for the receiveq.

The device has the allowance for the speed. If VIRTIO_NET_F_SPEED_DUPLEX has
been negotiated, the driver can get this by \field{speed}. When the received
packets bitrate exceeds the \field{speed}, some packets may be dropped by the
device.

\begin{lstlisting}
struct virtio_net_stats_rx_speed {
    struct virtio_net_stats_reply_hdr hdr;

    le64 rx_packets_allowance_exceeded;
    le64 rx_bytes_allowance_exceeded;
};
\end{lstlisting}

The packets described below were all presented on the specified virtqueue.
\begin{description}
    \item [rx_packets_allowance_exceeded]
        The number of the packets dropped by the device due to the received
        packets bitrate exceeding the \field{speed}.

    \item [rx_bytes_allowance_exceeded]
        The bytes of the packets dropped by the device due to the received
        packets bitrate exceeding the \field{speed}.

\end{description}

\subparagraph{Transmitq Speed Statistics}\label{sec:Device Types / Network Device / Device Operation / Control Virtqueue / Device Statistics / Transmitq Speed Statistics}

The structure corresponding to the transmitq speed statistics is
\field{struct virtio_net_stats_tx_speed}. The corresponding type is
VIRTIO_NET_STATS_TYPE_TX_SPEED. This is for the transmitq.

The device has the allowance for the speed. If VIRTIO_NET_F_SPEED_DUPLEX has
been negotiated, the driver can get this by \field{speed}. When the transmit
packets bitrate exceeds the \field{speed}, some packets may be dropped by the
device.

\begin{lstlisting}
struct virtio_net_stats_tx_speed {
    struct virtio_net_stats_reply_hdr hdr;

    le64 tx_packets_allowance_exceeded;
    le64 tx_bytes_allowance_exceeded;
};
\end{lstlisting}

The packets described below were all presented on the specified virtqueue.
\begin{description}
    \item [tx_packets_allowance_exceeded]
        The number of the packets dropped by the device due to the transmit packets
        bitrate exceeding the \field{speed}.

    \item [tx_bytes_allowance_exceeded]
        The bytes of the packets dropped by the device due to the transmit packets
        bitrate exceeding the \field{speed}.

\end{description}

\devicenormative{\subparagraph}{Device Statistics}{Device Types / Network Device / Device Operation / Control Virtqueue / Device Statistics}

When the VIRTIO_NET_F_DEVICE_STATS feature is negotiated, the device MUST reply
to the command VIRTIO_NET_CTRL_STATS_QUERY with the
\field{struct virtio_net_stats_capabilities}. \field{supported_stats_types}
includes all the statistic types supported by the device.

If \field{struct virtio_net_ctrl_queue_stats} is incorrect (such as the
following), the device MUST set \field{ack} to VIRTIO_NET_ERR. Even if there is
only one error, the device MUST fail the entire command.
\begin{itemize}
    \item \field{vq_index} exceeds the queue range.
    \item \field{types_bitmap} contains unknown types.
    \item One or more of the bits present in \field{types_bitmap} is not valid
        for the specified virtqueue.
    \item The feature corresponding to the specified \field{types_bitmap} was
        not negotiated.
\end{itemize}

The device MUST set the actual size of the bytes occupied by the reply to the
\field{size} of the \field{hdr}. And the device MUST set the \field{type} and
the \field{vq_index} of the statistic header.

The \field{command-specific-result} buffer allocated by the driver may be
smaller or bigger than all the statistics specified by
\field{struct virtio_net_ctrl_queue_stats}. The device MUST fill up only upto
the valid bytes.

The statistics counter replied by the device MUST wrap around to zero by the
device on the overflow.

\drivernormative{\subparagraph}{Device Statistics}{Device Types / Network Device / Device Operation / Control Virtqueue / Device Statistics}

The types contained in the \field{types_bitmap} MUST be queried from the device
via command VIRTIO_NET_CTRL_STATS_QUERY.

\field{types_bitmap} in \field{struct virtio_net_ctrl_queue_stats} MUST be valid to the
vq specified by \field{vq_index}.

The \field{command-specific-result} buffer allocated by the driver MUST have
enough capacity to store all the statistics reply headers defined in
\field{struct virtio_net_ctrl_queue_stats}. If the
\field{command-specific-result} buffer is fully utilized by the device but some
replies are missed, it is possible that some statistics may exceed the capacity
of the driver's records. In such cases, the driver should allocate additional
space for the \field{command-specific-result} buffer.

\subsubsection{Flow filter}\label{sec:Device Types / Network Device / Device Operation / Flow filter}

A network device can support one or more flow filter rules. Each flow filter rule
is applied by matching a packet and then taking an action, such as directing the packet
to a specific receiveq or dropping the packet. An example of a match is
matching on specific source and destination IP addresses.

A flow filter rule is a device resource object that consists of a key,
a processing priority, and an action to either direct a packet to a
receive queue or drop the packet.

Each rule uses a classifier. The key is matched against the packet using
a classifier, defining which fields in the packet are matched.
A classifier resource object consists of one or more field selectors, each with
a type that specifies the header fields to be matched against, and a mask.
The mask can match whole fields or parts of a field in a header. Each
rule resource object depends on the classifier resource object.

When a packet is received, relevant fields are extracted
(in the same way) from both the packet and the key according to the
classifier. The resulting field contents are then compared -
if they are identical the rule action is taken, if they are not, the rule is ignored.

Multiple flow filter rules are part of a group. The rule resource object
depends on the group. Each rule within a
group has a rule priority, and each group also has a group priority. For a
packet, a group with the highest priority is selected first. Within a group,
rules are applied from highest to lowest priority, until one of the rules
matches the packet and an action is taken. If all the rules within a group
are ignored, the group with the next highest priority is selected, and so on.

The device and the driver indicates flow filter resource limits using the capability
\ref{par:Device Types / Network Device / Device Operation / Flow filter / Device and driver capabilities / VIRTIO-NET-FF-RESOURCE-CAP} specifying the limits on the number of flow filter rule,
group and classifier resource objects. The capability \ref{par:Device Types / Network Device / Device Operation / Flow filter / Device and driver capabilities / VIRTIO-NET-FF-SELECTOR-CAP} specifies which selectors the device supports.
The driver indicates the selectors it is using by setting the flow
filter selector capability, prior to adding any resource objects.

The capability \ref{par:Device Types / Network Device / Device Operation / Flow filter / Device and driver capabilities / VIRTIO-NET-FF-ACTION-CAP} specifies which actions the device supports.

The driver controls the flow filter rule, classifier and group resource objects using
administration commands described in
\ref{sec:Basic Facilities of a Virtio Device / Device groups / Group administration commands / Device resource objects}.

\paragraph{Packet processing order}\label{sec:sec:Device Types / Network Device / Device Operation / Flow filter / Packet processing order}

Note that flow filter rules are applied after MAC/VLAN filtering. Flow filter
rules take precedence over steering: if a flow filter rule results in an action,
the steering configuration does not apply. The steering configuration only applies
to packets for which no flow filter rule action was performed. For example,
incoming packets can be processed in the following order:

\begin{itemize}
\item apply steering configuration received using control virtqueue commands
      VIRTIO_NET_CTRL_RX, VIRTIO_NET_CTRL_MAC and VIRTIO_NET_CTRL_VLAN.
\item apply flow filter rules if any.
\item if no filter rule applied, apply steering configuration received using command
      VIRTIO_NET_CTRL_MQ_RSS_CONFIG or as per automatic receive steering.
\end{itemize}

Some incoming packet processing examples:
\begin{itemize}
\item If the packet is dropped by the flow filter rule, RSS
      steering is ignored for the packet.
\item If the packet is directed to a specific receiveq using flow filter rule,
      the RSS steering is ignored for the packet.
\item If a packet is dropped due to the VIRTIO_NET_CTRL_MAC configuration,
      both flow filter rules and the RSS steering are ignored for the packet.
\item If a packet does not match any flow filter rules,
      the RSS steering is used to select the receiveq for the packet (if enabled).
\item If there are two flow filter groups configured as group_A and group_B
      with respective group priorities as 4, and 5; flow filter rules of
      group_B are applied first having highest group priority, if there is a match,
      the flow filter rules of group_A are ignored; if there is no match for
      the flow filter rules in group_B, the flow filter rules of next level group_A are applied.
\end{itemize}

\paragraph{Device and driver capabilities}
\label{par:Device Types / Network Device / Device Operation / Flow filter / Device and driver capabilities}

\subparagraph{VIRTIO_NET_FF_RESOURCE_CAP}
\label{par:Device Types / Network Device / Device Operation / Flow filter / Device and driver capabilities / VIRTIO-NET-FF-RESOURCE-CAP}

The capability VIRTIO_NET_FF_RESOURCE_CAP indicates the flow filter resource limits.
\field{cap_specific_data} is in the format
\field{struct virtio_net_ff_cap_data}.

\begin{lstlisting}
struct virtio_net_ff_cap_data {
        le32 groups_limit;
        le32 selectors_limit;
        le32 rules_limit;
        le32 rules_per_group_limit;
        u8 last_rule_priority;
        u8 selectors_per_classifier_limit;
};
\end{lstlisting}

\field{groups_limit}, and \field{selectors_limit} represent the maximum
number of flow filter groups and selectors, respectively, that the driver can create.
 \field{rules_limit} is the maximum number of
flow fiilter rules that the driver can create across all the groups.
\field{rules_per_group_limit} is the maximum number of flow filter rules that the driver
can create for each flow filter group.

\field{last_rule_priority} is the highest priority that can be assigned to a
flow filter rule.

\field{selectors_per_classifier_limit} is the maximum number of selectors
that a classifier can have.

\subparagraph{VIRTIO_NET_FF_SELECTOR_CAP}
\label{par:Device Types / Network Device / Device Operation / Flow filter / Device and driver capabilities / VIRTIO-NET-FF-SELECTOR-CAP}

The capability VIRTIO_NET_FF_SELECTOR_CAP lists the supported selectors and the
supported packet header fields for each selector.
\field{cap_specific_data} is in the format \field{struct virtio_net_ff_cap_mask_data}.

\begin{lstlisting}[label={lst:Device Types / Network Device / Device Operation / Flow filter / Device and driver capabilities / VIRTIO-NET-FF-SELECTOR-CAP / virtio-net-ff-selector}]
struct virtio_net_ff_selector {
        u8 type;
        u8 flags;
        u8 reserved[2];
        u8 length;
        u8 reserved1[3];
        u8 mask[];
};

struct virtio_net_ff_cap_mask_data {
        u8 count;
        u8 reserved[7];
        struct virtio_net_ff_selector selectors[];
};

#define VIRTIO_NET_FF_MASK_F_PARTIAL_MASK (1 << 0)
\end{lstlisting}

\field{count} indicates number of valid entries in the \field{selectors} array.
\field{selectors[]} is an array of supported selectors. Within each array entry:
\field{type} specifies the type of the packet header, as defined in table
\ref{table:Device Types / Network Device / Device Operation / Flow filter / Device and driver capabilities / VIRTIO-NET-FF-SELECTOR-CAP / flow filter selector types}. \field{mask} specifies which fields of the
packet header can be matched in a flow filter rule.

Each \field{type} is also listed in table
\ref{table:Device Types / Network Device / Device Operation / Flow filter / Device and driver capabilities / VIRTIO-NET-FF-SELECTOR-CAP / flow filter selector types}. \field{mask} is a byte array
in network byte order. For example, when \field{type} is VIRTIO_NET_FF_MASK_TYPE_IPV6,
the \field{mask} is in the format \hyperref[intro:IPv6-Header-Format]{IPv6 Header Format}.

If partial masking is not set, then all bits in each field have to be either all 0s
to ignore this field or all 1s to match on this field. If partial masking is set,
then any combination of bits can bit set to match on these bits.
For example, when a selector \field{type} is VIRTIO_NET_FF_MASK_TYPE_ETH, if
\field{mask[0-12]} are zero and \field{mask[13-14]} are 0xff (all 1s), it
indicates that matching is only supported for \field{EtherType} of
\field{Ethernet MAC frame}, matching is not supported for
\field{Destination Address} and \field{Source Address}.

The entries in the array \field{selectors} are ordered by
\field{type}, with each \field{type} value only appearing once.

\field{length} is the length of a dynamic array \field{mask} in bytes.
\field{reserved} and \field{reserved1} are reserved and set to zero.

\begin{table}[H]
\caption{Flow filter selector types}
\label{table:Device Types / Network Device / Device Operation / Flow filter / Device and driver capabilities / VIRTIO-NET-FF-SELECTOR-CAP / flow filter selector types}
\begin{tabularx}{\textwidth}{ |l|X|X| }
\hline
Type & Name & Description \\
\hline \hline
0x0 & - & Reserved \\
\hline
0x1 & VIRTIO_NET_FF_MASK_TYPE_ETH & 14 bytes of frame header starting from destination address described in \hyperref[intro:IEEE 802.3-2022]{IEEE 802.3-2022} \\
\hline
0x2 & VIRTIO_NET_FF_MASK_TYPE_IPV4 & 20 bytes of \hyperref[intro:Internet-Header-Format]{IPv4: Internet Header Format} \\
\hline
0x3 & VIRTIO_NET_FF_MASK_TYPE_IPV6 & 40 bytes of \hyperref[intro:IPv6-Header-Format]{IPv6 Header Format} \\
\hline
0x4 & VIRTIO_NET_FF_MASK_TYPE_TCP & 20 bytes of \hyperref[intro:TCP-Header-Format]{TCP Header Format} \\
\hline
0x5 & VIRTIO_NET_FF_MASK_TYPE_UDP & 8 bytes of UDP header described in \hyperref[intro:UDP]{UDP} \\
\hline
0x6 - 0xFF & & Reserved for future \\
\hline
\end{tabularx}
\end{table}

When VIRTIO_NET_FF_MASK_F_PARTIAL_MASK (bit 0) is set, it indicates that
partial masking is supported for all the fields of the selector identified by \field{type}.

For the selector \field{type} VIRTIO_NET_FF_MASK_TYPE_IPV4, if a partial mask is unsupported,
then matching on an individual bit of \field{Flags} in the
\field{IPv4: Internet Header Format} is unsupported. \field{Flags} has to match as a whole
if it is supported.

For the selector \field{type} VIRTIO_NET_FF_MASK_TYPE_IPV4, \field{mask} includes fields
up to the \field{Destination Address}; that is, \field{Options} and
\field{Padding} are excluded.

For the selector \field{type} VIRTIO_NET_FF_MASK_TYPE_IPV6, the \field{Next Header} field
of the \field{mask} corresponds to the \field{Next Header} in the packet
when \field{IPv6 Extension Headers} are not present. When the packet includes
one or more \field{IPv6 Extension Headers}, the \field{Next Header} field of
the \field{mask} corresponds to the \field{Next Header} of the last
\field{IPv6 Extension Header} in the packet.

For the selector \field{type} VIRTIO_NET_FF_MASK_TYPE_TCP, \field{Control bits}
are treated as individual fields for matching; that is, matching individual
\field{Control bits} does not depend on the partial mask support.

\subparagraph{VIRTIO_NET_FF_ACTION_CAP}
\label{par:Device Types / Network Device / Device Operation / Flow filter / Device and driver capabilities / VIRTIO-NET-FF-ACTION-CAP}

The capability VIRTIO_NET_FF_ACTION_CAP lists the supported actions in a rule.
\field{cap_specific_data} is in the format \field{struct virtio_net_ff_cap_actions}.

\begin{lstlisting}
struct virtio_net_ff_actions {
        u8 count;
        u8 reserved[7];
        u8 actions[];
};
\end{lstlisting}

\field{actions} is an array listing all possible actions.
The entries in the array are ordered from the smallest to the largest,
with each supported value appearing exactly once. Each entry can have the
following values:

\begin{table}[H]
\caption{Flow filter rule actions}
\label{table:Device Types / Network Device / Device Operation / Flow filter / Device and driver capabilities / VIRTIO-NET-FF-ACTION-CAP / flow filter rule actions}
\begin{tabularx}{\textwidth}{ |l|X|X| }
\hline
Action & Name & Description \\
\hline \hline
0x0 & - & reserved \\
\hline
0x1 & VIRTIO_NET_FF_ACTION_DROP & Matching packet will be dropped by the device \\
\hline
0x2 & VIRTIO_NET_FF_ACTION_DIRECT_RX_VQ & Matching packet will be directed to a receive queue \\
\hline
0x3 - 0xFF & & Reserved for future \\
\hline
\end{tabularx}
\end{table}

\paragraph{Resource objects}
\label{par:Device Types / Network Device / Device Operation / Flow filter / Resource objects}

\subparagraph{VIRTIO_NET_RESOURCE_OBJ_FF_GROUP}\label{par:Device Types / Network Device / Device Operation / Flow filter / Resource objects / VIRTIO-NET-RESOURCE-OBJ-FF-GROUP}

A flow filter group contains between 0 and \field{rules_limit} rules, as specified by the
capability VIRTIO_NET_FF_RESOURCE_CAP. For the flow filter group object both
\field{resource_obj_specific_data} and
\field{resource_obj_specific_result} are in the format
\field{struct virtio_net_resource_obj_ff_group}.

\begin{lstlisting}
struct virtio_net_resource_obj_ff_group {
        le16 group_priority;
};
\end{lstlisting}

\field{group_priority} specifies the priority for the group. Each group has a
distinct priority. For each incoming packet, the device tries to apply rules
from groups from higher \field{group_priority} value to lower, until either a
rule matches the packet or all groups have been tried.

\subparagraph{VIRTIO_NET_RESOURCE_OBJ_FF_CLASSIFIER}\label{par:Device Types / Network Device / Device Operation / Flow filter / Resource objects / VIRTIO-NET-RESOURCE-OBJ-FF-CLASSIFIER}

A classifier is used to match a flow filter key against a packet. The
classifier defines the desired packet fields to match, and is represented by
the VIRTIO_NET_RESOURCE_OBJ_FF_CLASSIFIER device resource object.

For the flow filter classifier object both \field{resource_obj_specific_data} and
\field{resource_obj_specific_result} are in the format
\field{struct virtio_net_resource_obj_ff_classifier}.

\begin{lstlisting}
struct virtio_net_resource_obj_ff_classifier {
        u8 count;
        u8 reserved[7];
        struct virtio_net_ff_selector selectors[];
};
\end{lstlisting}

A classifier is an array of \field{selectors}. The number of selectors in the
array is indicated by \field{count}. The selector has a type that specifies
the header fields to be matched against, and a mask.
See \ref{lst:Device Types / Network Device / Device Operation / Flow filter / Device and driver capabilities / VIRTIO-NET-FF-SELECTOR-CAP / virtio-net-ff-selector}
for details about selectors.

The first selector is always VIRTIO_NET_FF_MASK_TYPE_ETH. When there are multiple
selectors, a second selector can be either VIRTIO_NET_FF_MASK_TYPE_IPV4
or VIRTIO_NET_FF_MASK_TYPE_IPV6. If the third selector exists, the third
selector can be either VIRTIO_NET_FF_MASK_TYPE_UDP or VIRTIO_NET_FF_MASK_TYPE_TCP.
For example, to match a Ethernet IPv6 UDP packet,
\field{selectors[0].type} is set to VIRTIO_NET_FF_MASK_TYPE_ETH, \field{selectors[1].type}
is set to VIRTIO_NET_FF_MASK_TYPE_IPV6 and \field{selectors[2].type} is
set to VIRTIO_NET_FF_MASK_TYPE_UDP; accordingly, \field{selectors[0].mask[0-13]} is
for Ethernet header fields, \field{selectors[1].mask[0-39]} is set for IPV6 header
and \field{selectors[2].mask[0-7]} is set for UDP header.

When there are multiple selectors, the type of the (N+1)\textsuperscript{th} selector
affects the mask of the (N)\textsuperscript{th} selector. If
\field{count} is 2 or more, all the mask bits within \field{selectors[0]}
corresponding to \field{EtherType} of an Ethernet header are set.

If \field{count} is more than 2:
\begin{itemize}
\item if \field{selector[1].type} is, VIRTIO_NET_FF_MASK_TYPE_IPV4, then, all the mask bits within
\field{selector[1]} for \field{Protocol} is set.
\item if \field{selector[1].type} is, VIRTIO_NET_FF_MASK_TYPE_IPV6, then, all the mask bits within
\field{selector[1]} for \field{Next Header} is set.
\end{itemize}

If for a given packet header field, a subset of bits of a field is to be matched,
and if the partial mask is supported, the flow filter
mask object can specify a mask which has fewer bits set than the packet header
field size. For example, a partial mask for the Ethernet header source mac
address can be of 1-bit for multicast detection instead of 48-bits.

\subparagraph{VIRTIO_NET_RESOURCE_OBJ_FF_RULE}\label{par:Device Types / Network Device / Device Operation / Flow filter / Resource objects / VIRTIO-NET-RESOURCE-OBJ-FF-RULE}

Each flow filter rule resource object comprises a key, a priority, and an action.
For the flow filter rule object,
\field{resource_obj_specific_data} and
\field{resource_obj_specific_result} are in the format
\field{struct virtio_net_resource_obj_ff_rule}.

\begin{lstlisting}
struct virtio_net_resource_obj_ff_rule {
        le32 group_id;
        le32 classifier_id;
        u8 rule_priority;
        u8 key_length; /* length of key in bytes */
        u8 action;
        u8 reserved;
        le16 vq_index;
        u8 reserved1[2];
        u8 keys[][];
};
\end{lstlisting}

\field{group_id} is the resource object ID of the flow filter group to which
this rule belongs. \field{classifier_id} is the resource object ID of the
classifier used to match a packet against the \field{key}.

\field{rule_priority} denotes the priority of the rule within the group
specified by the \field{group_id}.
Rules within the group are applied from the highest to the lowest priority
until a rule matches the packet and an
action is taken. Rules with the same priority can be applied in any order.

\field{reserved} and \field{reserved1} are reserved and set to 0.

\field{keys[][]} is an array of keys to match against packets, using
the classifier specified by \field{classifier_id}. Each entry (key) comprises
a byte array, and they are located one immediately after another.
The size (number of entries) of the array is exactly the same as that of
\field{selectors} in the classifier, or in other words, \field{count}
in the classifier.

\field{key_length} specifies the total length of \field{keys} in bytes.
In other words, it equals the sum total of \field{length} of all
selectors in \field{selectors} in the classifier specified by
\field{classifier_id}.

For example, if a classifier object's \field{selectors[0].type} is
VIRTIO_NET_FF_MASK_TYPE_ETH and \field{selectors[1].type} is
VIRTIO_NET_FF_MASK_TYPE_IPV6,
then selectors[0].length is 14 and selectors[1].length is 40.
Accordingly, the \field{key_length} is set to 54.
This setting indicates that the \field{key} array's length is 54 bytes
comprising a first byte array of 14 bytes for the
Ethernet MAC header in bytes 0-13, immediately followed by 40 bytes for the
IPv6 header in bytes 14-53.

When there are multiple selectors in the classifier object, the key bytes
for (N)\textsuperscript{th} selector are set so that
(N+1)\textsuperscript{th} selector can be matched.

If \field{count} is 2 or more, key bytes of \field{EtherType}
are set according to \hyperref[intro:IEEE 802 Ethertypes]{IEEE 802 Ethertypes}
for VIRTIO_NET_FF_MASK_TYPE_IPV4 or VIRTIO_NET_FF_MASK_TYPE_IPV6 respectively.

If \field{count} is more than 2, when \field{selector[1].type} is
VIRTIO_NET_FF_MASK_TYPE_IPV4 or VIRTIO_NET_FF_MASK_TYPE_IPV6, key
bytes of \field{Protocol} or \field{Next Header} is set as per
\field{Protocol Numbers} defined \hyperref[intro:IANA Protocol Numbers]{IANA Protocol Numbers}
respectively.

\field{action} is the action to take when a packet matches the
\field{key} using the \field{classifier_id}. Supported actions are described in
\ref{table:Device Types / Network Device / Device Operation / Flow filter / Device and driver capabilities / VIRTIO-NET-FF-ACTION-CAP / flow filter rule actions}.

\field{vq_index} specifies a receive virtqueue. When the \field{action} is set
to VIRTIO_NET_FF_ACTION_DIRECT_RX_VQ, and the packet matches the \field{key},
the matching packet is directed to this virtqueue.

Note that at most one action is ever taken for a given packet. If a rule is
applied and an action is taken, the action of other rules is not taken.

\devicenormative{\paragraph}{Flow filter}{Device Types / Network Device / Device Operation / Flow filter}

When the device supports flow filter operations,
\begin{itemize}
\item the device MUST set VIRTIO_NET_FF_RESOURCE_CAP, VIRTIO_NET_FF_SELECTOR_CAP
and VIRTIO_NET_FF_ACTION_CAP capability in the \field{supported_caps} in the
command VIRTIO_ADMIN_CMD_CAP_SUPPORT_QUERY.
\item the device MUST support the administration commands
VIRTIO_ADMIN_CMD_RESOURCE_OBJ_CREATE,
VIRTIO_ADMIN_CMD_RESOURCE_OBJ_MODIFY, VIRTIO_ADMIN_CMD_RESOURCE_OBJ_QUERY,
VIRTIO_ADMIN_CMD_RESOURCE_OBJ_DESTROY for the resource types
VIRTIO_NET_RESOURCE_OBJ_FF_GROUP, VIRTIO_NET_RESOURCE_OBJ_FF_CLASSIFIER and
VIRTIO_NET_RESOURCE_OBJ_FF_RULE.
\end{itemize}

When any of the VIRTIO_NET_FF_RESOURCE_CAP, VIRTIO_NET_FF_SELECTOR_CAP, or
VIRTIO_NET_FF_ACTION_CAP capability is disabled, the device SHOULD set
\field{status} to VIRTIO_ADMIN_STATUS_Q_INVALID_OPCODE for the commands
VIRTIO_ADMIN_CMD_RESOURCE_OBJ_CREATE,
VIRTIO_ADMIN_CMD_RESOURCE_OBJ_MODIFY, VIRTIO_ADMIN_CMD_RESOURCE_OBJ_QUERY,
and VIRTIO_ADMIN_CMD_RESOURCE_OBJ_DESTROY. These commands apply to the resource
\field{type} of VIRTIO_NET_RESOURCE_OBJ_FF_GROUP, VIRTIO_NET_RESOURCE_OBJ_FF_CLASSIFIER, and
VIRTIO_NET_RESOURCE_OBJ_FF_RULE.

The device SHOULD set \field{status} to VIRTIO_ADMIN_STATUS_EINVAL for the
command VIRTIO_ADMIN_CMD_RESOURCE_OBJ_CREATE when the resource \field{type}
is VIRTIO_NET_RESOURCE_OBJ_FF_GROUP, if a flow filter group already exists
with the supplied \field{group_priority}.

The device SHOULD set \field{status} to VIRTIO_ADMIN_STATUS_ENOSPC for the
command VIRTIO_ADMIN_CMD_RESOURCE_OBJ_CREATE when the resource \field{type}
is VIRTIO_NET_RESOURCE_OBJ_FF_GROUP, if the number of flow filter group
objects in the device exceeds the lower of the configured driver
capabilities \field{groups_limit} and \field{rules_per_group_limit}.

The device SHOULD set \field{status} to VIRTIO_ADMIN_STATUS_ENOSPC for the
command VIRTIO_ADMIN_CMD_RESOURCE_OBJ_CREATE when the resource \field{type} is
VIRTIO_NET_RESOURCE_OBJ_FF_CLASSIFIER, if the number of flow filter selector
objects in the device exceeds the configured driver capability
\field{selectors_limit}.

The device SHOULD set \field{status} to VIRTIO_ADMIN_STATUS_EBUSY for the
command VIRTIO_ADMIN_CMD_RESOURCE_OBJ_DESTROY for a flow filter group when
the flow filter group has one or more flow filter rules depending on it.

The device SHOULD set \field{status} to VIRTIO_ADMIN_STATUS_EBUSY for the
command VIRTIO_ADMIN_CMD_RESOURCE_OBJ_DESTROY for a flow filter classifier when
the flow filter classifier has one or more flow filter rules depending on it.

The device SHOULD fail the command VIRTIO_ADMIN_CMD_RESOURCE_OBJ_CREATE for the
flow filter rule resource object if,
\begin{itemize}
\item \field{vq_index} is not a valid receive virtqueue index for
the VIRTIO_NET_FF_ACTION_DIRECT_RX_VQ action,
\item \field{priority} is greater than or equal to
      \field{last_rule_priority},
\item \field{id} is greater than or equal to \field{rules_limit} or
      greater than or equal to \field{rules_per_group_limit}, whichever is lower,
\item the length of \field{keys} and the length of all the mask bytes of
      \field{selectors[].mask} as referred by \field{classifier_id} differs,
\item the supplied \field{action} is not supported in the capability VIRTIO_NET_FF_ACTION_CAP.
\end{itemize}

When the flow filter directs a packet to the virtqueue identified by
\field{vq_index} and if the receive virtqueue is reset, the device
MUST drop such packets.

Upon applying a flow filter rule to a packet, the device MUST STOP any further
application of rules and cease applying any other steering configurations.

For multiple flow filter groups, the device MUST apply the rules from
the group with the highest priority. If any rule from this group is applied,
the device MUST ignore the remaining groups. If none of the rules from the
highest priority group match, the device MUST apply the rules from
the group with the next highest priority, until either a rule matches or
all groups have been attempted.

The device MUST apply the rules within the group from the highest to the
lowest priority until a rule matches the packet, and the device MUST take
the action. If an action is taken, the device MUST not take any other
action for this packet.

The device MAY apply the rules with the same \field{rule_priority} in any
order within the group.

The device MUST process incoming packets in the following order:
\begin{itemize}
\item apply the steering configuration received using control virtqueue
      commands VIRTIO_NET_CTRL_RX, VIRTIO_NET_CTRL_MAC, and
      VIRTIO_NET_CTRL_VLAN.
\item apply flow filter rules if any.
\item if no filter rule is applied, apply the steering configuration
      received using the command VIRTIO_NET_CTRL_MQ_RSS_CONFIG
      or according to automatic receive steering.
\end{itemize}

When processing an incoming packet, if the packet is dropped at any stage, the device
MUST skip further processing.

When the device drops the packet due to the configuration done using the control
virtqueue commands VIRTIO_NET_CTRL_RX or VIRTIO_NET_CTRL_MAC or VIRTIO_NET_CTRL_VLAN,
the device MUST skip flow filter rules for this packet.

When the device performs flow filter match operations and if the operation
result did not have any match in all the groups, the receive packet processing
continues to next level, i.e. to apply configuration done using
VIRTIO_NET_CTRL_MQ_RSS_CONFIG command.

The device MUST support the creation of flow filter classifier objects
using the command VIRTIO_ADMIN_CMD_RESOURCE_OBJ_CREATE with \field{flags}
set to VIRTIO_NET_FF_MASK_F_PARTIAL_MASK;
this support is required even if all the bits of the masks are set for
a field in \field{selectors}, provided that partial masking is supported
for the selectors.

\drivernormative{\paragraph}{Flow filter}{Device Types / Network Device / Device Operation / Flow filter}

The driver MUST enable VIRTIO_NET_FF_RESOURCE_CAP, VIRTIO_NET_FF_SELECTOR_CAP,
and VIRTIO_NET_FF_ACTION_CAP capabilities to use flow filter.

The driver SHOULD NOT remove a flow filter group using the command
VIRTIO_ADMIN_CMD_RESOURCE_OBJ_DESTROY when one or more flow filter rules
depend on that group. The driver SHOULD only destroy the group after
all the associated rules have been destroyed.

The driver SHOULD NOT remove a flow filter classifier using the command
VIRTIO_ADMIN_CMD_RESOURCE_OBJ_DESTROY when one or more flow filter rules
depend on the classifier. The driver SHOULD only destroy the classifier
after all the associated rules have been destroyed.

The driver SHOULD NOT add multiple flow filter rules with the same
\field{rule_priority} within a flow filter group, as these rules MAY match
the same packet. The driver SHOULD assign different \field{rule_priority}
values to different flow filter rules if multiple rules may match a single
packet.

For the command VIRTIO_ADMIN_CMD_RESOURCE_OBJ_CREATE, when creating a resource
of \field{type} VIRTIO_NET_RESOURCE_OBJ_FF_CLASSIFIER, the driver MUST set:
\begin{itemize}
\item \field{selectors[0].type} to VIRTIO_NET_FF_MASK_TYPE_ETH.
\item \field{selectors[1].type} to VIRTIO_NET_FF_MASK_TYPE_IPV4 or
      VIRTIO_NET_FF_MASK_TYPE_IPV6 when \field{count} is more than 1,
\item \field{selectors[2].type} VIRTIO_NET_FF_MASK_TYPE_UDP or
      VIRTIO_NET_FF_MASK_TYPE_TCP when \field{count} is more than 2.
\end{itemize}

For the command VIRTIO_ADMIN_CMD_RESOURCE_OBJ_CREATE, when creating a resource
of \field{type} VIRTIO_NET_RESOURCE_OBJ_FF_CLASSIFIER, the driver MUST set:
\begin{itemize}
\item \field{selectors[0].mask} bytes to all 1s for the \field{EtherType}
       when \field{count} is 2 or more.
\item \field{selectors[1].mask} bytes to all 1s for \field{Protocol} or \field{Next Header}
       when \field{selector[1].type} is VIRTIO_NET_FF_MASK_TYPE_IPV4 or VIRTIO_NET_FF_MASK_TYPE_IPV6,
       and when \field{count} is more than 2.
\end{itemize}

For the command VIRTIO_ADMIN_CMD_RESOURCE_OBJ_CREATE, the resource \field{type}
VIRTIO_NET_RESOURCE_OBJ_FF_RULE, if the corresponding classifier object's
\field{count} is 2 or more, the driver MUST SET the \field{keys} bytes of
\field{EtherType} in accordance with
\hyperref[intro:IEEE 802 Ethertypes]{IEEE 802 Ethertypes}
for either VIRTIO_NET_FF_MASK_TYPE_IPV4 or VIRTIO_NET_FF_MASK_TYPE_IPV6.

For the command VIRTIO_ADMIN_CMD_RESOURCE_OBJ_CREATE, when creating a resource of
\field{type} VIRTIO_NET_RESOURCE_OBJ_FF_RULE, if the corresponding classifier
object's \field{count} is more than 2, and the \field{selector[1].type} is either
VIRTIO_NET_FF_MASK_TYPE_IPV4 or VIRTIO_NET_FF_MASK_TYPE_IPV6, the driver MUST
set the \field{keys} bytes for the \field{Protocol} or \field{Next Header}
according to \hyperref[intro:IANA Protocol Numbers]{IANA Protocol Numbers} respectively.

The driver SHOULD set all the bits for a field in the mask of a selector in both the
capability and the classifier object, unless the VIRTIO_NET_FF_MASK_F_PARTIAL_MASK
is enabled.

\subsubsection{Legacy Interface: Framing Requirements}\label{sec:Device
Types / Network Device / Legacy Interface: Framing Requirements}

When using legacy interfaces, transitional drivers which have not
negotiated VIRTIO_F_ANY_LAYOUT MUST use a single descriptor for the
\field{struct virtio_net_hdr} on both transmit and receive, with the
network data in the following descriptors.

Additionally, when using the control virtqueue (see \ref{sec:Device
Types / Network Device / Device Operation / Control Virtqueue})
, transitional drivers which have not
negotiated VIRTIO_F_ANY_LAYOUT MUST:
\begin{itemize}
\item for all commands, use a single 2-byte descriptor including the first two
fields: \field{class} and \field{command}
\item for all commands except VIRTIO_NET_CTRL_MAC_TABLE_SET
use a single descriptor including command-specific-data
with no padding.
\item for the VIRTIO_NET_CTRL_MAC_TABLE_SET command use exactly
two descriptors including command-specific-data with no padding:
the first of these descriptors MUST include the
virtio_net_ctrl_mac table structure for the unicast addresses with no padding,
the second of these descriptors MUST include the
virtio_net_ctrl_mac table structure for the multicast addresses
with no padding.
\item for all commands, use a single 1-byte descriptor for the
\field{ack} field
\end{itemize}

See \ref{sec:Basic
Facilities of a Virtio Device / Virtqueues / Message Framing}.

\section{Network Device}\label{sec:Device Types / Network Device}

The virtio network device is a virtual network interface controller.
It consists of a virtual Ethernet link which connects the device
to the Ethernet network. The device has transmit and receive
queues. The driver adds empty buffers to the receive virtqueue.
The device receives incoming packets from the link; the device
places these incoming packets in the receive virtqueue buffers.
The driver adds outgoing packets to the transmit virtqueue. The device
removes these packets from the transmit virtqueue and sends them to
the link. The device may have a control virtqueue. The driver
uses the control virtqueue to dynamically manipulate various
features of the initialized device.

\subsection{Device ID}\label{sec:Device Types / Network Device / Device ID}

 1

\subsection{Virtqueues}\label{sec:Device Types / Network Device / Virtqueues}

\begin{description}
\item[0] receiveq1
\item[1] transmitq1
\item[\ldots]
\item[2(N-1)] receiveqN
\item[2(N-1)+1] transmitqN
\item[2N] controlq
\end{description}

 N=1 if neither VIRTIO_NET_F_MQ nor VIRTIO_NET_F_RSS are negotiated, otherwise N is set by
 \field{max_virtqueue_pairs}.

controlq is optional; it only exists if VIRTIO_NET_F_CTRL_VQ is
negotiated.

\subsection{Feature bits}\label{sec:Device Types / Network Device / Feature bits}

\begin{description}
\item[VIRTIO_NET_F_CSUM (0)] Device handles packets with partial checksum offload.

\item[VIRTIO_NET_F_GUEST_CSUM (1)] Driver handles packets with partial checksum.

\item[VIRTIO_NET_F_CTRL_GUEST_OFFLOADS (2)] Control channel offloads
        reconfiguration support.

\item[VIRTIO_NET_F_MTU(3)] Device maximum MTU reporting is supported. If
    offered by the device, device advises driver about the value of
    its maximum MTU. If negotiated, the driver uses \field{mtu} as
    the maximum MTU value.

\item[VIRTIO_NET_F_MAC (5)] Device has given MAC address.

\item[VIRTIO_NET_F_GUEST_TSO4 (7)] Driver can receive TSOv4.

\item[VIRTIO_NET_F_GUEST_TSO6 (8)] Driver can receive TSOv6.

\item[VIRTIO_NET_F_GUEST_ECN (9)] Driver can receive TSO with ECN.

\item[VIRTIO_NET_F_GUEST_UFO (10)] Driver can receive UFO.

\item[VIRTIO_NET_F_HOST_TSO4 (11)] Device can receive TSOv4.

\item[VIRTIO_NET_F_HOST_TSO6 (12)] Device can receive TSOv6.

\item[VIRTIO_NET_F_HOST_ECN (13)] Device can receive TSO with ECN.

\item[VIRTIO_NET_F_HOST_UFO (14)] Device can receive UFO.

\item[VIRTIO_NET_F_MRG_RXBUF (15)] Driver can merge receive buffers.

\item[VIRTIO_NET_F_STATUS (16)] Configuration status field is
    available.

\item[VIRTIO_NET_F_CTRL_VQ (17)] Control channel is available.

\item[VIRTIO_NET_F_CTRL_RX (18)] Control channel RX mode support.

\item[VIRTIO_NET_F_CTRL_VLAN (19)] Control channel VLAN filtering.

\item[VIRTIO_NET_F_CTRL_RX_EXTRA (20)]	Control channel RX extra mode support.

\item[VIRTIO_NET_F_GUEST_ANNOUNCE(21)] Driver can send gratuitous
    packets.

\item[VIRTIO_NET_F_MQ(22)] Device supports multiqueue with automatic
    receive steering.

\item[VIRTIO_NET_F_CTRL_MAC_ADDR(23)] Set MAC address through control
    channel.

\item[VIRTIO_NET_F_DEVICE_STATS(50)] Device can provide device-level statistics
    to the driver through the control virtqueue.

\item[VIRTIO_NET_F_HASH_TUNNEL(51)] Device supports inner header hash for encapsulated packets.

\item[VIRTIO_NET_F_VQ_NOTF_COAL(52)] Device supports virtqueue notification coalescing.

\item[VIRTIO_NET_F_NOTF_COAL(53)] Device supports notifications coalescing.

\item[VIRTIO_NET_F_GUEST_USO4 (54)] Driver can receive USOv4 packets.

\item[VIRTIO_NET_F_GUEST_USO6 (55)] Driver can receive USOv6 packets.

\item[VIRTIO_NET_F_HOST_USO (56)] Device can receive USO packets. Unlike UFO
 (fragmenting the packet) the USO splits large UDP packet
 to several segments when each of these smaller packets has UDP header.

\item[VIRTIO_NET_F_HASH_REPORT(57)] Device can report per-packet hash
    value and a type of calculated hash.

\item[VIRTIO_NET_F_GUEST_HDRLEN(59)] Driver can provide the exact \field{hdr_len}
    value. Device benefits from knowing the exact header length.

\item[VIRTIO_NET_F_RSS(60)] Device supports RSS (receive-side scaling)
    with Toeplitz hash calculation and configurable hash
    parameters for receive steering.

\item[VIRTIO_NET_F_RSC_EXT(61)] Device can process duplicated ACKs
    and report number of coalesced segments and duplicated ACKs.

\item[VIRTIO_NET_F_STANDBY(62)] Device may act as a standby for a primary
    device with the same MAC address.

\item[VIRTIO_NET_F_SPEED_DUPLEX(63)] Device reports speed and duplex.

\item[VIRTIO_NET_F_RSS_CONTEXT(64)] Device supports multiple RSS contexts.

\item[VIRTIO_NET_F_GUEST_UDP_TUNNEL_GSO (65)] Driver can receive GSO packets
  carried by a UDP tunnel.

\item[VIRTIO_NET_F_GUEST_UDP_TUNNEL_GSO_CSUM (66)] Driver handles packets
  carried by a UDP tunnel with partial csum for the outer header.

\item[VIRTIO_NET_F_HOST_UDP_TUNNEL_GSO (67)] Device can receive GSO packets
  carried by a UDP tunnel.

\item[VIRTIO_NET_F_HOST_UDP_TUNNEL_GSO_CSUM (68)] Device handles packets
  carried by a UDP tunnel with partial csum for the outer header.
\end{description}

\subsubsection{Feature bit requirements}\label{sec:Device Types / Network Device / Feature bits / Feature bit requirements}

Some networking feature bits require other networking feature bits
(see \ref{drivernormative:Basic Facilities of a Virtio Device / Feature Bits}):

\begin{description}
\item[VIRTIO_NET_F_GUEST_TSO4] Requires VIRTIO_NET_F_GUEST_CSUM.
\item[VIRTIO_NET_F_GUEST_TSO6] Requires VIRTIO_NET_F_GUEST_CSUM.
\item[VIRTIO_NET_F_GUEST_ECN] Requires VIRTIO_NET_F_GUEST_TSO4 or VIRTIO_NET_F_GUEST_TSO6.
\item[VIRTIO_NET_F_GUEST_UFO] Requires VIRTIO_NET_F_GUEST_CSUM.
\item[VIRTIO_NET_F_GUEST_USO4] Requires VIRTIO_NET_F_GUEST_CSUM.
\item[VIRTIO_NET_F_GUEST_USO6] Requires VIRTIO_NET_F_GUEST_CSUM.
\item[VIRTIO_NET_F_GUEST_UDP_TUNNEL_GSO] Requires VIRTIO_NET_F_GUEST_TSO4, VIRTIO_NET_F_GUEST_TSO6,
   VIRTIO_NET_F_GUEST_USO4 and VIRTIO_NET_F_GUEST_USO6.
\item[VIRTIO_NET_F_GUEST_UDP_TUNNEL_GSO_CSUM] Requires VIRTIO_NET_F_GUEST_UDP_TUNNEL_GSO

\item[VIRTIO_NET_F_HOST_TSO4] Requires VIRTIO_NET_F_CSUM.
\item[VIRTIO_NET_F_HOST_TSO6] Requires VIRTIO_NET_F_CSUM.
\item[VIRTIO_NET_F_HOST_ECN] Requires VIRTIO_NET_F_HOST_TSO4 or VIRTIO_NET_F_HOST_TSO6.
\item[VIRTIO_NET_F_HOST_UFO] Requires VIRTIO_NET_F_CSUM.
\item[VIRTIO_NET_F_HOST_USO] Requires VIRTIO_NET_F_CSUM.
\item[VIRTIO_NET_F_HOST_UDP_TUNNEL_GSO] Requires VIRTIO_NET_F_HOST_TSO4, VIRTIO_NET_F_HOST_TSO6
   and VIRTIO_NET_F_HOST_USO.
\item[VIRTIO_NET_F_HOST_UDP_TUNNEL_GSO_CSUM] Requires VIRTIO_NET_F_HOST_UDP_TUNNEL_GSO

\item[VIRTIO_NET_F_CTRL_RX] Requires VIRTIO_NET_F_CTRL_VQ.
\item[VIRTIO_NET_F_CTRL_VLAN] Requires VIRTIO_NET_F_CTRL_VQ.
\item[VIRTIO_NET_F_GUEST_ANNOUNCE] Requires VIRTIO_NET_F_CTRL_VQ.
\item[VIRTIO_NET_F_MQ] Requires VIRTIO_NET_F_CTRL_VQ.
\item[VIRTIO_NET_F_CTRL_MAC_ADDR] Requires VIRTIO_NET_F_CTRL_VQ.
\item[VIRTIO_NET_F_NOTF_COAL] Requires VIRTIO_NET_F_CTRL_VQ.
\item[VIRTIO_NET_F_RSC_EXT] Requires VIRTIO_NET_F_HOST_TSO4 or VIRTIO_NET_F_HOST_TSO6.
\item[VIRTIO_NET_F_RSS] Requires VIRTIO_NET_F_CTRL_VQ.
\item[VIRTIO_NET_F_VQ_NOTF_COAL] Requires VIRTIO_NET_F_CTRL_VQ.
\item[VIRTIO_NET_F_HASH_TUNNEL] Requires VIRTIO_NET_F_CTRL_VQ along with VIRTIO_NET_F_RSS or VIRTIO_NET_F_HASH_REPORT.
\item[VIRTIO_NET_F_RSS_CONTEXT] Requires VIRTIO_NET_F_CTRL_VQ and VIRTIO_NET_F_RSS.
\end{description}

\begin{note}
The dependency between UDP_TUNNEL_GSO_CSUM and UDP_TUNNEL_GSO is intentionally
in the opposite direction with respect to the plain GSO features and the plain
checksum offload because UDP tunnel checksum offload gives very little gain
for non GSO packets and is quite complex to implement in H/W.
\end{note}

\subsubsection{Legacy Interface: Feature bits}\label{sec:Device Types / Network Device / Feature bits / Legacy Interface: Feature bits}
\begin{description}
\item[VIRTIO_NET_F_GSO (6)] Device handles packets with any GSO type. This was supposed to indicate segmentation offload support, but
upon further investigation it became clear that multiple bits were needed.
\item[VIRTIO_NET_F_GUEST_RSC4 (41)] Device coalesces TCPIP v4 packets. This was implemented by hypervisor patch for certification
purposes and current Windows driver depends on it. It will not function if virtio-net device reports this feature.
\item[VIRTIO_NET_F_GUEST_RSC6 (42)] Device coalesces TCPIP v6 packets. Similar to VIRTIO_NET_F_GUEST_RSC4.
\end{description}

\subsection{Device configuration layout}\label{sec:Device Types / Network Device / Device configuration layout}
\label{sec:Device Types / Block Device / Feature bits / Device configuration layout}

The network device has the following device configuration layout.
All of the device configuration fields are read-only for the driver.

\begin{lstlisting}
struct virtio_net_config {
        u8 mac[6];
        le16 status;
        le16 max_virtqueue_pairs;
        le16 mtu;
        le32 speed;
        u8 duplex;
        u8 rss_max_key_size;
        le16 rss_max_indirection_table_length;
        le32 supported_hash_types;
        le32 supported_tunnel_types;
};
\end{lstlisting}

The \field{mac} address field always exists (although it is only
valid if VIRTIO_NET_F_MAC is set).

The \field{status} only exists if VIRTIO_NET_F_STATUS is set.
Two bits are currently defined for the status field: VIRTIO_NET_S_LINK_UP
and VIRTIO_NET_S_ANNOUNCE.

\begin{lstlisting}
#define VIRTIO_NET_S_LINK_UP     1
#define VIRTIO_NET_S_ANNOUNCE    2
\end{lstlisting}

The following field, \field{max_virtqueue_pairs} only exists if
VIRTIO_NET_F_MQ or VIRTIO_NET_F_RSS is set. This field specifies the maximum number
of each of transmit and receive virtqueues (receiveq1\ldots receiveqN
and transmitq1\ldots transmitqN respectively) that can be configured once at least one of these features
is negotiated.

The following field, \field{mtu} only exists if VIRTIO_NET_F_MTU
is set. This field specifies the maximum MTU for the driver to
use.

The following two fields, \field{speed} and \field{duplex}, only
exist if VIRTIO_NET_F_SPEED_DUPLEX is set.

\field{speed} contains the device speed, in units of 1 MBit per
second, 0 to 0x7fffffff, or 0xffffffff for unknown speed.

\field{duplex} has the values of 0x01 for full duplex, 0x00 for
half duplex and 0xff for unknown duplex state.

Both \field{speed} and \field{duplex} can change, thus the driver
is expected to re-read these values after receiving a
configuration change notification.

The following field, \field{rss_max_key_size} only exists if VIRTIO_NET_F_RSS or VIRTIO_NET_F_HASH_REPORT is set.
It specifies the maximum supported length of RSS key in bytes.

The following field, \field{rss_max_indirection_table_length} only exists if VIRTIO_NET_F_RSS is set.
It specifies the maximum number of 16-bit entries in RSS indirection table.

The next field, \field{supported_hash_types} only exists if the device supports hash calculation,
i.e. if VIRTIO_NET_F_RSS or VIRTIO_NET_F_HASH_REPORT is set.

Field \field{supported_hash_types} contains the bitmask of supported hash types.
See \ref{sec:Device Types / Network Device / Device Operation / Processing of Incoming Packets / Hash calculation for incoming packets / Supported/enabled hash types} for details of supported hash types.

Field \field{supported_tunnel_types} only exists if the device supports inner header hash, i.e. if VIRTIO_NET_F_HASH_TUNNEL is set.

Field \field{supported_tunnel_types} contains the bitmask of encapsulation types supported by the device for inner header hash.
Encapsulation types are defined in \ref{sec:Device Types / Network Device / Device Operation / Processing of Incoming Packets /
Hash calculation for incoming packets / Encapsulation types supported/enabled for inner header hash}.

\devicenormative{\subsubsection}{Device configuration layout}{Device Types / Network Device / Device configuration layout}

The device MUST set \field{max_virtqueue_pairs} to between 1 and 0x8000 inclusive,
if it offers VIRTIO_NET_F_MQ.

The device MUST set \field{mtu} to between 68 and 65535 inclusive,
if it offers VIRTIO_NET_F_MTU.

The device SHOULD set \field{mtu} to at least 1280, if it offers
VIRTIO_NET_F_MTU.

The device MUST NOT modify \field{mtu} once it has been set.

The device MUST NOT pass received packets that exceed \field{mtu} (plus low
level ethernet header length) size with \field{gso_type} NONE or ECN
after VIRTIO_NET_F_MTU has been successfully negotiated.

The device MUST forward transmitted packets of up to \field{mtu} (plus low
level ethernet header length) size with \field{gso_type} NONE or ECN, and do
so without fragmentation, after VIRTIO_NET_F_MTU has been successfully
negotiated.

The device MUST set \field{rss_max_key_size} to at least 40, if it offers
VIRTIO_NET_F_RSS or VIRTIO_NET_F_HASH_REPORT.

The device MUST set \field{rss_max_indirection_table_length} to at least 128, if it offers
VIRTIO_NET_F_RSS.

If the driver negotiates the VIRTIO_NET_F_STANDBY feature, the device MAY act
as a standby device for a primary device with the same MAC address.

If VIRTIO_NET_F_SPEED_DUPLEX has been negotiated, \field{speed}
MUST contain the device speed, in units of 1 MBit per second, 0 to
0x7ffffffff, or 0xfffffffff for unknown.

If VIRTIO_NET_F_SPEED_DUPLEX has been negotiated, \field{duplex}
MUST have the values of 0x00 for full duplex, 0x01 for half
duplex, or 0xff for unknown.

If VIRTIO_NET_F_SPEED_DUPLEX and VIRTIO_NET_F_STATUS have both
been negotiated, the device SHOULD NOT change the \field{speed} and
\field{duplex} fields as long as VIRTIO_NET_S_LINK_UP is set in
the \field{status}.

The device SHOULD NOT offer VIRTIO_NET_F_HASH_REPORT if it
does not offer VIRTIO_NET_F_CTRL_VQ.

The device SHOULD NOT offer VIRTIO_NET_F_CTRL_RX_EXTRA if it
does not offer VIRTIO_NET_F_CTRL_VQ.

\drivernormative{\subsubsection}{Device configuration layout}{Device Types / Network Device / Device configuration layout}

The driver MUST NOT write to any of the device configuration fields.

A driver SHOULD negotiate VIRTIO_NET_F_MAC if the device offers it.
If the driver negotiates the VIRTIO_NET_F_MAC feature, the driver MUST set
the physical address of the NIC to \field{mac}.  Otherwise, it SHOULD
use a locally-administered MAC address (see \hyperref[intro:IEEE 802]{IEEE 802},
``9.2 48-bit universal LAN MAC addresses'').

If the driver does not negotiate the VIRTIO_NET_F_STATUS feature, it SHOULD
assume the link is active, otherwise it SHOULD read the link status from
the bottom bit of \field{status}.

A driver SHOULD negotiate VIRTIO_NET_F_MTU if the device offers it.

If the driver negotiates VIRTIO_NET_F_MTU, it MUST supply enough receive
buffers to receive at least one receive packet of size \field{mtu} (plus low
level ethernet header length) with \field{gso_type} NONE or ECN.

If the driver negotiates VIRTIO_NET_F_MTU, it MUST NOT transmit packets of
size exceeding the value of \field{mtu} (plus low level ethernet header length)
with \field{gso_type} NONE or ECN.

A driver SHOULD negotiate the VIRTIO_NET_F_STANDBY feature if the device offers it.

If VIRTIO_NET_F_SPEED_DUPLEX has been negotiated,
the driver MUST treat any value of \field{speed} above
0x7fffffff as well as any value of \field{duplex} not
matching 0x00 or 0x01 as an unknown value.

If VIRTIO_NET_F_SPEED_DUPLEX has been negotiated, the driver
SHOULD re-read \field{speed} and \field{duplex} after a
configuration change notification.

A driver SHOULD NOT negotiate VIRTIO_NET_F_HASH_REPORT if it
does not negotiate VIRTIO_NET_F_CTRL_VQ.

A driver SHOULD NOT negotiate VIRTIO_NET_F_CTRL_RX_EXTRA if it
does not negotiate VIRTIO_NET_F_CTRL_VQ.

\subsubsection{Legacy Interface: Device configuration layout}\label{sec:Device Types / Network Device / Device configuration layout / Legacy Interface: Device configuration layout}
\label{sec:Device Types / Block Device / Feature bits / Device configuration layout / Legacy Interface: Device configuration layout}
When using the legacy interface, transitional devices and drivers
MUST format \field{status} and
\field{max_virtqueue_pairs} in struct virtio_net_config
according to the native endian of the guest rather than
(necessarily when not using the legacy interface) little-endian.

When using the legacy interface, \field{mac} is driver-writable
which provided a way for drivers to update the MAC without
negotiating VIRTIO_NET_F_CTRL_MAC_ADDR.

\subsection{Device Initialization}\label{sec:Device Types / Network Device / Device Initialization}

A driver would perform a typical initialization routine like so:

\begin{enumerate}
\item Identify and initialize the receive and
  transmission virtqueues, up to N of each kind. If
  VIRTIO_NET_F_MQ feature bit is negotiated,
  N=\field{max_virtqueue_pairs}, otherwise identify N=1.

\item If the VIRTIO_NET_F_CTRL_VQ feature bit is negotiated,
  identify the control virtqueue.

\item Fill the receive queues with buffers: see \ref{sec:Device Types / Network Device / Device Operation / Setting Up Receive Buffers}.

\item Even with VIRTIO_NET_F_MQ, only receiveq1, transmitq1 and
  controlq are used by default.  The driver would send the
  VIRTIO_NET_CTRL_MQ_VQ_PAIRS_SET command specifying the
  number of the transmit and receive queues to use.

\item If the VIRTIO_NET_F_MAC feature bit is set, the configuration
  space \field{mac} entry indicates the ``physical'' address of the
  device, otherwise the driver would typically generate a random
  local MAC address.

\item If the VIRTIO_NET_F_STATUS feature bit is negotiated, the link
  status comes from the bottom bit of \field{status}.
  Otherwise, the driver assumes it's active.

\item A performant driver would indicate that it will generate checksumless
  packets by negotiating the VIRTIO_NET_F_CSUM feature.

\item If that feature is negotiated, a driver can use TCP segmentation or UDP
  segmentation/fragmentation offload by negotiating the VIRTIO_NET_F_HOST_TSO4 (IPv4
  TCP), VIRTIO_NET_F_HOST_TSO6 (IPv6 TCP), VIRTIO_NET_F_HOST_UFO
  (UDP fragmentation) and VIRTIO_NET_F_HOST_USO (UDP segmentation) features.

\item If the VIRTIO_NET_F_HOST_TSO6, VIRTIO_NET_F_HOST_TSO4 and VIRTIO_NET_F_HOST_USO
  segmentation features are negotiated, a driver can
  use TCP segmentation or UDP segmentation on top of UDP encapsulation
  offload, when the outer header does not require checksumming - e.g.
  the outer UDP checksum is zero - by negotiating the
  VIRTIO_NET_F_HOST_UDP_TUNNEL_GSO feature.
  GSO over UDP tunnels packets carry two sets of headers: the outer ones
  and the inner ones. The outer transport protocol is UDP, the inner
  could be either TCP or UDP. Only a single level of encapsulation
  offload is supported.

\item If VIRTIO_NET_F_HOST_UDP_TUNNEL_GSO is negotiated, a driver can
  additionally use TCP segmentation or UDP segmentation on top of UDP
  encapsulation with the outer header requiring checksum offload,
  negotiating the VIRTIO_NET_F_HOST_UDP_TUNNEL_GSO_CSUM feature.

\item The converse features are also available: a driver can save
  the virtual device some work by negotiating these features.\note{For example, a network packet transported between two guests on
the same system might not need checksumming at all, nor segmentation,
if both guests are amenable.}
   The VIRTIO_NET_F_GUEST_CSUM feature indicates that partially
  checksummed packets can be received, and if it can do that then
  the VIRTIO_NET_F_GUEST_TSO4, VIRTIO_NET_F_GUEST_TSO6,
  VIRTIO_NET_F_GUEST_UFO, VIRTIO_NET_F_GUEST_ECN, VIRTIO_NET_F_GUEST_USO4,
  VIRTIO_NET_F_GUEST_USO6 VIRTIO_NET_F_GUEST_UDP_TUNNEL_GSO and
  VIRTIO_NET_F_GUEST_UDP_TUNNEL_GSO_CSUM are the input equivalents of
  the features described above.
  See \ref{sec:Device Types / Network Device / Device Operation /
Setting Up Receive Buffers}~\nameref{sec:Device Types / Network
Device / Device Operation / Setting Up Receive Buffers} and
\ref{sec:Device Types / Network Device / Device Operation /
Processing of Incoming Packets}~\nameref{sec:Device Types /
Network Device / Device Operation / Processing of Incoming Packets} below.
\end{enumerate}

A truly minimal driver would only accept VIRTIO_NET_F_MAC and ignore
everything else.

\subsection{Device and driver capabilities}\label{sec:Device Types / Network Device / Device and driver capabilities}

The network device has the following capabilities.

\begin{tabularx}{\textwidth}{ |l||l|X| }
\hline
Identifier & Name & Description \\
\hline \hline
0x0800 & \hyperref[par:Device Types / Network Device / Device Operation / Flow filter / Device and driver capabilities / VIRTIO-NET-FF-RESOURCE-CAP]{VIRTIO_NET_FF_RESOURCE_CAP} & Flow filter resource capability \\
\hline
0x0801 & \hyperref[par:Device Types / Network Device / Device Operation / Flow filter / Device and driver capabilities / VIRTIO-NET-FF-SELECTOR-CAP]{VIRTIO_NET_FF_SELECTOR_CAP} & Flow filter classifier capability \\
\hline
0x0802 & \hyperref[par:Device Types / Network Device / Device Operation / Flow filter / Device and driver capabilities / VIRTIO-NET-FF-ACTION-CAP]{VIRTIO_NET_FF_ACTION_CAP} & Flow filter action capability \\
\hline
\end{tabularx}

\subsection{Device resource objects}\label{sec:Device Types / Network Device / Device resource objects}

The network device has the following resource objects.

\begin{tabularx}{\textwidth}{ |l||l|X| }
\hline
type & Name & Description \\
\hline \hline
0x0200 & \hyperref[par:Device Types / Network Device / Device Operation / Flow filter / Resource objects / VIRTIO-NET-RESOURCE-OBJ-FF-GROUP]{VIRTIO_NET_RESOURCE_OBJ_FF_GROUP} & Flow filter group resource object \\
\hline
0x0201 & \hyperref[par:Device Types / Network Device / Device Operation / Flow filter / Resource objects / VIRTIO-NET-RESOURCE-OBJ-FF-CLASSIFIER]{VIRTIO_NET_RESOURCE_OBJ_FF_CLASSIFIER} & Flow filter mask object \\
\hline
0x0202 & \hyperref[par:Device Types / Network Device / Device Operation / Flow filter / Resource objects / VIRTIO-NET-RESOURCE-OBJ-FF-RULE]{VIRTIO_NET_RESOURCE_OBJ_FF_RULE} & Flow filter rule object \\
\hline
\end{tabularx}

\subsection{Device parts}\label{sec:Device Types / Network Device / Device parts}

Network device parts represent the configuration done by the driver using control
virtqueue commands. Network device part is in the format of
\field{struct virtio_dev_part}.

\begin{tabularx}{\textwidth}{ |l||l|X| }
\hline
Type & Name & Description \\
\hline \hline
0x200 & VIRTIO_NET_DEV_PART_CVQ_CFG_PART & Represents device configuration done through a control virtqueue command, see \ref{sec:Device Types / Network Device / Device parts / VIRTIO-NET-DEV-PART-CVQ-CFG-PART} \\
\hline
0x201 - 0x5FF & - & reserved for future \\
\hline
\hline
\end{tabularx}

\subsubsection{VIRTIO_NET_DEV_PART_CVQ_CFG_PART}\label{sec:Device Types / Network Device / Device parts / VIRTIO-NET-DEV-PART-CVQ-CFG-PART}

For VIRTIO_NET_DEV_PART_CVQ_CFG_PART, \field{part_type} is set to 0x200. The
VIRTIO_NET_DEV_PART_CVQ_CFG_PART part indicates configuration performed by the
driver using a control virtqueue command.

\begin{lstlisting}
struct virtio_net_dev_part_cvq_selector {
        u8 class;
        u8 command;
        u8 reserved[6];
};
\end{lstlisting}

There is one device part of type VIRTIO_NET_DEV_PART_CVQ_CFG_PART for each
individual configuration. Each part is identified by a unique selector value.
The selector, \field{device_type_raw}, is in the format
\field{struct virtio_net_dev_part_cvq_selector}.

The selector consists of two fields: \field{class} and \field{command}. These
fields correspond to the \field{class} and \field{command} defined in
\field{struct virtio_net_ctrl}, as described in the relevant sections of
\ref{sec:Device Types / Network Device / Device Operation / Control Virtqueue}.

The value corresponding to each part’s selector follows the same format as the
respective \field{command-specific-data} described in the relevant sections of
\ref{sec:Device Types / Network Device / Device Operation / Control Virtqueue}.

For example, when the \field{class} is VIRTIO_NET_CTRL_MAC, the \field{command}
can be either VIRTIO_NET_CTRL_MAC_TABLE_SET or VIRTIO_NET_CTRL_MAC_ADDR_SET;
when \field{command} is set to VIRTIO_NET_CTRL_MAC_TABLE_SET, \field{value}
is in the format of \field{struct virtio_net_ctrl_mac}.

Supported selectors are listed in the table:

\begin{tabularx}{\textwidth}{ |l|X| }
\hline
Class selector & Command selector \\
\hline \hline
VIRTIO_NET_CTRL_RX & VIRTIO_NET_CTRL_RX_PROMISC \\
\hline
VIRTIO_NET_CTRL_RX & VIRTIO_NET_CTRL_RX_ALLMULTI \\
\hline
VIRTIO_NET_CTRL_RX & VIRTIO_NET_CTRL_RX_ALLUNI \\
\hline
VIRTIO_NET_CTRL_RX & VIRTIO_NET_CTRL_RX_NOMULTI \\
\hline
VIRTIO_NET_CTRL_RX & VIRTIO_NET_CTRL_RX_NOUNI \\
\hline
VIRTIO_NET_CTRL_RX & VIRTIO_NET_CTRL_RX_NOBCAST \\
\hline
VIRTIO_NET_CTRL_MAC & VIRTIO_NET_CTRL_MAC_TABLE_SET \\
\hline
VIRTIO_NET_CTRL_MAC & VIRTIO_NET_CTRL_MAC_ADDR_SET \\
\hline
VIRTIO_NET_CTRL_VLAN & VIRTIO_NET_CTRL_VLAN_ADD \\
\hline
VIRTIO_NET_CTRL_ANNOUNCE & VIRTIO_NET_CTRL_ANNOUNCE_ACK \\
\hline
VIRTIO_NET_CTRL_MQ & VIRTIO_NET_CTRL_MQ_VQ_PAIRS_SET \\
\hline
VIRTIO_NET_CTRL_MQ & VIRTIO_NET_CTRL_MQ_RSS_CONFIG \\
\hline
VIRTIO_NET_CTRL_MQ & VIRTIO_NET_CTRL_MQ_HASH_CONFIG \\
\hline
\hline
\end{tabularx}

For command selector VIRTIO_NET_CTRL_VLAN_ADD, device part consists of a whole
VLAN table.

\field{reserved} is reserved and set to zero.

\subsection{Device Operation}\label{sec:Device Types / Network Device / Device Operation}

Packets are transmitted by placing them in the
transmitq1\ldots transmitqN, and buffers for incoming packets are
placed in the receiveq1\ldots receiveqN. In each case, the packet
itself is preceded by a header:

\begin{lstlisting}
struct virtio_net_hdr {
#define VIRTIO_NET_HDR_F_NEEDS_CSUM    1
#define VIRTIO_NET_HDR_F_DATA_VALID    2
#define VIRTIO_NET_HDR_F_RSC_INFO      4
#define VIRTIO_NET_HDR_F_UDP_TUNNEL_CSUM 8
        u8 flags;
#define VIRTIO_NET_HDR_GSO_NONE        0
#define VIRTIO_NET_HDR_GSO_TCPV4       1
#define VIRTIO_NET_HDR_GSO_UDP         3
#define VIRTIO_NET_HDR_GSO_TCPV6       4
#define VIRTIO_NET_HDR_GSO_UDP_L4      5
#define VIRTIO_NET_HDR_GSO_UDP_TUNNEL_IPV4 0x20
#define VIRTIO_NET_HDR_GSO_UDP_TUNNEL_IPV6 0x40
#define VIRTIO_NET_HDR_GSO_ECN      0x80
        u8 gso_type;
        le16 hdr_len;
        le16 gso_size;
        le16 csum_start;
        le16 csum_offset;
        le16 num_buffers;
        le32 hash_value;        (Only if VIRTIO_NET_F_HASH_REPORT negotiated)
        le16 hash_report;       (Only if VIRTIO_NET_F_HASH_REPORT negotiated)
        le16 padding_reserved;  (Only if VIRTIO_NET_F_HASH_REPORT negotiated)
        le16 outer_th_offset    (Only if VIRTIO_NET_F_HOST_UDP_TUNNEL_GSO or VIRTIO_NET_F_GUEST_UDP_TUNNEL_GSO negotiated)
        le16 inner_nh_offset;   (Only if VIRTIO_NET_F_HOST_UDP_TUNNEL_GSO or VIRTIO_NET_F_GUEST_UDP_TUNNEL_GSO negotiated)
};
\end{lstlisting}

The controlq is used to control various device features described further in
section \ref{sec:Device Types / Network Device / Device Operation / Control Virtqueue}.

\subsubsection{Legacy Interface: Device Operation}\label{sec:Device Types / Network Device / Device Operation / Legacy Interface: Device Operation}
When using the legacy interface, transitional devices and drivers
MUST format the fields in \field{struct virtio_net_hdr}
according to the native endian of the guest rather than
(necessarily when not using the legacy interface) little-endian.

The legacy driver only presented \field{num_buffers} in the \field{struct virtio_net_hdr}
when VIRTIO_NET_F_MRG_RXBUF was negotiated; without that feature the
structure was 2 bytes shorter.

When using the legacy interface, the driver SHOULD ignore the
used length for the transmit queues
and the controlq queue.
\begin{note}
Historically, some devices put
the total descriptor length there, even though no data was
actually written.
\end{note}

\subsubsection{Packet Transmission}\label{sec:Device Types / Network Device / Device Operation / Packet Transmission}

Transmitting a single packet is simple, but varies depending on
the different features the driver negotiated.

\begin{enumerate}
\item The driver can send a completely checksummed packet.  In this case,
  \field{flags} will be zero, and \field{gso_type} will be VIRTIO_NET_HDR_GSO_NONE.

\item If the driver negotiated VIRTIO_NET_F_CSUM, it can skip
  checksumming the packet:
  \begin{itemize}
  \item \field{flags} has the VIRTIO_NET_HDR_F_NEEDS_CSUM set,

  \item \field{csum_start} is set to the offset within the packet to begin checksumming,
    and

  \item \field{csum_offset} indicates how many bytes after the csum_start the
    new (16 bit ones' complement) checksum is placed by the device.

  \item The TCP checksum field in the packet is set to the sum
    of the TCP pseudo header, so that replacing it by the ones'
    complement checksum of the TCP header and body will give the
    correct result.
  \end{itemize}

\begin{note}
For example, consider a partially checksummed TCP (IPv4) packet.
It will have a 14 byte ethernet header and 20 byte IP header
followed by the TCP header (with the TCP checksum field 16 bytes
into that header). \field{csum_start} will be 14+20 = 34 (the TCP
checksum includes the header), and \field{csum_offset} will be 16.
If the given packet has the VIRTIO_NET_HDR_GSO_UDP_TUNNEL_IPV4 bit or the
VIRTIO_NET_HDR_GSO_UDP_TUNNEL_IPV6 bit set,
the above checksum fields refer to the inner header checksum, see
the example below.
\end{note}

\item If the driver negotiated
  VIRTIO_NET_F_HOST_TSO4, TSO6, USO or UFO, and the packet requires
  TCP segmentation, UDP segmentation or fragmentation, then \field{gso_type}
  is set to VIRTIO_NET_HDR_GSO_TCPV4, TCPV6, UDP_L4 or UDP.
  (Otherwise, it is set to VIRTIO_NET_HDR_GSO_NONE). In this
  case, packets larger than 1514 bytes can be transmitted: the
  metadata indicates how to replicate the packet header to cut it
  into smaller packets. The other gso fields are set:

  \begin{itemize}
  \item If the VIRTIO_NET_F_GUEST_HDRLEN feature has been negotiated,
    \field{hdr_len} indicates the header length that needs to be replicated
    for each packet. It's the number of bytes from the beginning of the packet
    to the beginning of the transport payload.
    If the \field{gso_type} has the VIRTIO_NET_HDR_GSO_UDP_TUNNEL_IPV4 bit or
    VIRTIO_NET_HDR_GSO_UDP_TUNNEL_IPV6 bit set, \field{hdr_len} accounts for
    all the headers up to and including the inner transport.
    Otherwise, if the VIRTIO_NET_F_GUEST_HDRLEN feature has not been negotiated,
    \field{hdr_len} is a hint to the device as to how much of the header
    needs to be kept to copy into each packet, usually set to the
    length of the headers, including the transport header\footnote{Due to various bugs in implementations, this field is not useful
as a guarantee of the transport header size.
}.

  \begin{note}
  Some devices benefit from knowledge of the exact header length.
  \end{note}

  \item \field{gso_size} is the maximum size of each packet beyond that
    header (ie. MSS).

  \item If the driver negotiated the VIRTIO_NET_F_HOST_ECN feature,
    the VIRTIO_NET_HDR_GSO_ECN bit in \field{gso_type}
    indicates that the TCP packet has the ECN bit set\footnote{This case is not handled by some older hardware, so is called out
specifically in the protocol.}.
   \end{itemize}

\item If the driver negotiated the VIRTIO_NET_F_HOST_UDP_TUNNEL_GSO feature and the
  \field{gso_type} has the VIRTIO_NET_HDR_GSO_UDP_TUNNEL_IPV4 bit or
  VIRTIO_NET_HDR_GSO_UDP_TUNNEL_IPV6 bit set, the GSO protocol is encapsulated
  in a UDP tunnel.
  If the outer UDP header requires checksumming, the driver must have
  additionally negotiated the VIRTIO_NET_F_HOST_UDP_TUNNEL_GSO_CSUM feature
  and offloaded the outer checksum accordingly, otherwise
  the outer UDP header must not require checksum validation, i.e. the outer
  UDP checksum must be positive zero (0x0) as defined in UDP RFC 768.
  The other tunnel-related fields indicate how to replicate the packet
  headers to cut it into smaller packets:

  \begin{itemize}
  \item \field{outer_th_offset} field indicates the outer transport header within
      the packet. This field differs from \field{csum_start} as the latter
      points to the inner transport header within the packet.

  \item \field{inner_nh_offset} field indicates the inner network header within
      the packet.
  \end{itemize}

\begin{note}
For example, consider a partially checksummed TCP (IPv4) packet carried over a
Geneve UDP tunnel (again IPv4) with no tunnel options. The
only relevant variable related to the tunnel type is the tunnel header length.
The packet will have a 14 byte outer ethernet header, 20 byte outer IP header
followed by the 8 byte UDP header (with a 0 checksum value), 8 byte Geneve header,
14 byte inner ethernet header, 20 byte inner IP header
and the TCP header (with the TCP checksum field 16 bytes
into that header). \field{csum_start} will be 14+20+8+8+14+20 = 84 (the TCP
checksum includes the header), \field{csum_offset} will be 16.
\field{inner_nh_offset} will be 14+20+8+8+14 = 62, \field{outer_th_offset} will be
14+20+8 = 42 and \field{gso_type} will be
VIRTIO_NET_HDR_GSO_TCPV4 | VIRTIO_NET_HDR_GSO_UDP_TUNNEL_IPV4 = 0x21
\end{note}

\item If the driver negotiated the VIRTIO_NET_F_HOST_UDP_TUNNEL_GSO_CSUM feature,
  the transmitted packet is a GSO one encapsulated in a UDP tunnel, and
  the outer UDP header requires checksumming, the driver can skip checksumming
  the outer header:

  \begin{itemize}
  \item \field{flags} has the VIRTIO_NET_HDR_F_UDP_TUNNEL_CSUM set,

  \item The outer UDP checksum field in the packet is set to the sum
    of the UDP pseudo header, so that replacing it by the ones'
    complement checksum of the outer UDP header and payload will give the
    correct result.
  \end{itemize}

\item \field{num_buffers} is set to zero.  This field is unused on transmitted packets.

\item The header and packet are added as one output descriptor to the
  transmitq, and the device is notified of the new entry
  (see \ref{sec:Device Types / Network Device / Device Initialization}~\nameref{sec:Device Types / Network Device / Device Initialization}).
\end{enumerate}

\drivernormative{\paragraph}{Packet Transmission}{Device Types / Network Device / Device Operation / Packet Transmission}

For the transmit packet buffer, the driver MUST use the size of the
structure \field{struct virtio_net_hdr} same as the receive packet buffer.

The driver MUST set \field{num_buffers} to zero.

If VIRTIO_NET_F_CSUM is not negotiated, the driver MUST set
\field{flags} to zero and SHOULD supply a fully checksummed
packet to the device.

If VIRTIO_NET_F_HOST_TSO4 is negotiated, the driver MAY set
\field{gso_type} to VIRTIO_NET_HDR_GSO_TCPV4 to request TCPv4
segmentation, otherwise the driver MUST NOT set
\field{gso_type} to VIRTIO_NET_HDR_GSO_TCPV4.

If VIRTIO_NET_F_HOST_TSO6 is negotiated, the driver MAY set
\field{gso_type} to VIRTIO_NET_HDR_GSO_TCPV6 to request TCPv6
segmentation, otherwise the driver MUST NOT set
\field{gso_type} to VIRTIO_NET_HDR_GSO_TCPV6.

If VIRTIO_NET_F_HOST_UFO is negotiated, the driver MAY set
\field{gso_type} to VIRTIO_NET_HDR_GSO_UDP to request UDP
fragmentation, otherwise the driver MUST NOT set
\field{gso_type} to VIRTIO_NET_HDR_GSO_UDP.

If VIRTIO_NET_F_HOST_USO is negotiated, the driver MAY set
\field{gso_type} to VIRTIO_NET_HDR_GSO_UDP_L4 to request UDP
segmentation, otherwise the driver MUST NOT set
\field{gso_type} to VIRTIO_NET_HDR_GSO_UDP_L4.

The driver SHOULD NOT send to the device TCP packets requiring segmentation offload
which have the Explicit Congestion Notification bit set, unless the
VIRTIO_NET_F_HOST_ECN feature is negotiated, in which case the
driver MUST set the VIRTIO_NET_HDR_GSO_ECN bit in
\field{gso_type}.

If VIRTIO_NET_F_HOST_UDP_TUNNEL_GSO is negotiated, the driver MAY set
VIRTIO_NET_HDR_GSO_UDP_TUNNEL_IPV4 bit or the VIRTIO_NET_HDR_GSO_UDP_TUNNEL_IPV6 bit
in \field{gso_type} according to the inner network header protocol type
to request GSO packets over UDPv4 or UDPv6 tunnel segmentation,
otherwise the driver MUST NOT set either the
VIRTIO_NET_HDR_GSO_UDP_TUNNEL_IPV4 bit or the VIRTIO_NET_HDR_GSO_UDP_TUNNEL_IPV6 bit
in \field{gso_type}.

When requesting GSO segmentation over UDP tunnel, the driver MUST SET the
VIRTIO_NET_HDR_GSO_UDP_TUNNEL_IPV4 bit if the inner network header is IPv4, i.e. the
packet is a TCPv4 GSO one, otherwise, if the inner network header is IPv6, the driver
MUST SET the VIRTIO_NET_HDR_GSO_UDP_TUNNEL_IPV6 bit.

The driver MUST NOT send to the device GSO packets over UDP tunnel
requiring segmentation and outer UDP checksum offload, unless both the
VIRTIO_NET_F_HOST_UDP_TUNNEL_GSO and VIRTIO_NET_F_HOST_UDP_TUNNEL_GSO_CSUM features
are negotiated, in which case the driver MUST set either the
VIRTIO_NET_HDR_GSO_UDP_TUNNEL_IPV4 bit or the VIRTIO_NET_HDR_GSO_UDP_TUNNEL_IPV6
bit in the \field{gso_type} and the VIRTIO_NET_HDR_F_UDP_TUNNEL_CSUM bit in
the \field{flags}.

If VIRTIO_NET_F_HOST_UDP_TUNNEL_GSO_CSUM is not negotiated, the driver MUST not set
the VIRTIO_NET_HDR_F_UDP_TUNNEL_CSUM bit in the \field{flags} and
MUST NOT send to the device GSO packets over UDP tunnel
requiring segmentation and outer UDP checksum offload.

The driver MUST NOT set the VIRTIO_NET_HDR_GSO_UDP_TUNNEL_IPV4 bit or the
VIRTIO_NET_HDR_GSO_UDP_TUNNEL_IPV6 bit together with VIRTIO_NET_HDR_GSO_UDP, as the
latter is deprecated in favor of UDP_L4 and no new feature will support it.

The driver MUST NOT set the VIRTIO_NET_HDR_GSO_UDP_TUNNEL_IPV4 bit and the
VIRTIO_NET_HDR_GSO_UDP_TUNNEL_IPV6 bit together.

The driver MUST NOT set the VIRTIO_NET_HDR_F_UDP_TUNNEL_CSUM bit \field{flags}
without setting either the VIRTIO_NET_HDR_GSO_UDP_TUNNEL_IPV4 bit or
the VIRTIO_NET_HDR_GSO_UDP_TUNNEL_IPV6 bit in \field{gso_type}.

If the VIRTIO_NET_F_CSUM feature has been negotiated, the
driver MAY set the VIRTIO_NET_HDR_F_NEEDS_CSUM bit in
\field{flags}, if so:
\begin{enumerate}
\item the driver MUST validate the packet checksum at
	offset \field{csum_offset} from \field{csum_start} as well as all
	preceding offsets;
\begin{note}
If \field{gso_type} differs from VIRTIO_NET_HDR_GSO_NONE and the
VIRTIO_NET_HDR_GSO_UDP_TUNNEL_IPV4 bit or the VIRTIO_NET_HDR_GSO_UDP_TUNNEL_IPV6
bit are not set in \field{gso_type}, \field{csum_offset}
points to the only transport header present in the packet, and there are no
additional preceding checksums validated by VIRTIO_NET_HDR_F_NEEDS_CSUM.
\end{note}
\item the driver MUST set the packet checksum stored in the
	buffer to the TCP/UDP pseudo header;
\item the driver MUST set \field{csum_start} and
	\field{csum_offset} such that calculating a ones'
	complement checksum from \field{csum_start} up until the end of
	the packet and storing the result at offset \field{csum_offset}
	from  \field{csum_start} will result in a fully checksummed
	packet;
\end{enumerate}

If none of the VIRTIO_NET_F_HOST_TSO4, TSO6, USO or UFO options have
been negotiated, the driver MUST set \field{gso_type} to
VIRTIO_NET_HDR_GSO_NONE.

If \field{gso_type} differs from VIRTIO_NET_HDR_GSO_NONE, then
the driver MUST also set the VIRTIO_NET_HDR_F_NEEDS_CSUM bit in
\field{flags} and MUST set \field{gso_size} to indicate the
desired MSS.

If one of the VIRTIO_NET_F_HOST_TSO4, TSO6, USO or UFO options have
been negotiated:
\begin{itemize}
\item If the VIRTIO_NET_F_GUEST_HDRLEN feature has been negotiated,
	and \field{gso_type} differs from VIRTIO_NET_HDR_GSO_NONE,
	the driver MUST set \field{hdr_len} to a value equal to the length
	of the headers, including the transport header. If \field{gso_type}
	has the VIRTIO_NET_HDR_GSO_UDP_TUNNEL_IPV4 bit or the
	VIRTIO_NET_HDR_GSO_UDP_TUNNEL_IPV6 bit set, \field{hdr_len} includes
	the inner transport header.

\item If the VIRTIO_NET_F_GUEST_HDRLEN feature has not been negotiated,
	or \field{gso_type} is VIRTIO_NET_HDR_GSO_NONE,
	the driver SHOULD set \field{hdr_len} to a value
	not less than the length of the headers, including the transport
	header.
\end{itemize}

If the VIRTIO_NET_F_HOST_UDP_TUNNEL_GSO option has been negotiated, the
driver MAY set the VIRTIO_NET_HDR_GSO_UDP_TUNNEL_IPV4 bit or the
VIRTIO_NET_HDR_GSO_UDP_TUNNEL_IPV6 bit in \field{gso_type}, if so:
\begin{itemize}
\item the driver MUST set \field{outer_th_offset} to the outer UDP header
  offset and \field{inner_nh_offset} to the inner network header offset.
  The \field{csum_start} and \field{csum_offset} fields point respectively
  to the inner transport header and inner transport checksum field.
\end{itemize}

If the VIRTIO_NET_F_HOST_UDP_TUNNEL_GSO_CSUM feature has been negotiated,
and the VIRTIO_NET_HDR_GSO_UDP_TUNNEL_IPV4 bit or
VIRTIO_NET_HDR_GSO_UDP_TUNNEL_IPV6 bit in \field{gso_type} are set,
the driver MAY set the VIRTIO_NET_HDR_F_UDP_TUNNEL_CSUM bit in
\field{flags}, if so the driver MUST set the packet outer UDP header checksum
to the outer UDP pseudo header checksum.

\begin{note}
calculating a ones' complement checksum from \field{outer_th_offset}
up until the end of the packet and storing the result at offset 6
from \field{outer_th_offset} will result in a fully checksummed outer UDP packet;
\end{note}

If the VIRTIO_NET_HDR_GSO_UDP_TUNNEL_IPV4 bit or the
VIRTIO_NET_HDR_GSO_UDP_TUNNEL_IPV6 bit in \field{gso_type} are set
and the VIRTIO_NET_F_HOST_UDP_TUNNEL_GSO_CSUM feature has not
been negotiated, the
outer UDP header MUST NOT require checksum validation. That is, the
outer UDP checksum value MUST be 0 or the validated complete checksum
for such header.

\begin{note}
The valid complete checksum of the outer UDP header of individual segments
can be computed by the driver prior to segmentation only if the GSO packet
size is a multiple of \field{gso_size}, because then all segments
have the same size and thus all data included in the outer UDP
checksum is the same for every segment. These pre-computed segment
length and checksum fields are different from those of the GSO
packet.
In this scenario the outer UDP header of the GSO packet must carry the
segmented UDP packet length.
\end{note}

If the VIRTIO_NET_F_HOST_UDP_TUNNEL_GSO option has not
been negotiated, the driver MUST NOT set either the VIRTIO_NET_HDR_F_GSO_UDP_TUNNEL_IPV4
bit or the VIRTIO_NET_HDR_F_GSO_UDP_TUNNEL_IPV6 in \field{gso_type}.

If the VIRTIO_NET_F_HOST_UDP_TUNNEL_GSO_CSUM option has not been negotiated,
the driver MUST NOT set the VIRTIO_NET_HDR_F_UDP_TUNNEL_CSUM bit
in \field{flags}.

The driver SHOULD accept the VIRTIO_NET_F_GUEST_HDRLEN feature if it has
been offered, and if it's able to provide the exact header length.

The driver MUST NOT set the VIRTIO_NET_HDR_F_DATA_VALID and
VIRTIO_NET_HDR_F_RSC_INFO bits in \field{flags}.

The driver MUST NOT set the VIRTIO_NET_HDR_F_DATA_VALID bit in \field{flags}
together with the VIRTIO_NET_HDR_F_GSO_UDP_TUNNEL_IPV4 bit or the
VIRTIO_NET_HDR_F_GSO_UDP_TUNNEL_IPV6 bit in \field{gso_type}.

\devicenormative{\paragraph}{Packet Transmission}{Device Types / Network Device / Device Operation / Packet Transmission}
The device MUST ignore \field{flag} bits that it does not recognize.

If VIRTIO_NET_HDR_F_NEEDS_CSUM bit in \field{flags} is not set, the
device MUST NOT use the \field{csum_start} and \field{csum_offset}.

If one of the VIRTIO_NET_F_HOST_TSO4, TSO6, USO or UFO options have
been negotiated:
\begin{itemize}
\item If the VIRTIO_NET_F_GUEST_HDRLEN feature has been negotiated,
	and \field{gso_type} differs from VIRTIO_NET_HDR_GSO_NONE,
	the device MAY use \field{hdr_len} as the transport header size.

	\begin{note}
	Caution should be taken by the implementation so as to prevent
	a malicious driver from attacking the device by setting an incorrect hdr_len.
	\end{note}

\item If the VIRTIO_NET_F_GUEST_HDRLEN feature has not been negotiated,
	or \field{gso_type} is VIRTIO_NET_HDR_GSO_NONE,
	the device MAY use \field{hdr_len} only as a hint about the
	transport header size.
	The device MUST NOT rely on \field{hdr_len} to be correct.

	\begin{note}
	This is due to various bugs in implementations.
	\end{note}
\end{itemize}

If both the VIRTIO_NET_HDR_GSO_UDP_TUNNEL_IPV4 bit and
the VIRTIO_NET_HDR_GSO_UDP_TUNNEL_IPV6 bit in in \field{gso_type} are set,
the device MUST NOT accept the packet.

If the VIRTIO_NET_HDR_GSO_UDP_TUNNEL_IPV4 bit and the VIRTIO_NET_HDR_GSO_UDP_TUNNEL_IPV6
bit in \field{gso_type} are not set, the device MUST NOT use the
\field{outer_th_offset} and \field{inner_nh_offset}.

If either the VIRTIO_NET_HDR_GSO_UDP_TUNNEL_IPV4 bit or
the VIRTIO_NET_HDR_GSO_UDP_TUNNEL_IPV6 bit in \field{gso_type} are set, and any of
the following is true:
\begin{itemize}
\item the VIRTIO_NET_HDR_F_NEEDS_CSUM is not set in \field{flags}
\item the VIRTIO_NET_HDR_F_DATA_VALID is set in \field{flags}
\item the \field{gso_type} excluding the VIRTIO_NET_HDR_GSO_UDP_TUNNEL_IPV4
bit and the VIRTIO_NET_HDR_GSO_UDP_TUNNEL_IPV6 bit is VIRTIO_NET_HDR_GSO_NONE
\end{itemize}
the device MUST NOT accept the packet.

If the VIRTIO_NET_HDR_F_UDP_TUNNEL_CSUM bit in \field{flags} is set,
and both the bits VIRTIO_NET_HDR_GSO_UDP_TUNNEL_IPV4 and
VIRTIO_NET_HDR_GSO_UDP_TUNNEL_IPV6 in \field{gso_type} are not set,
the device MOST NOT accept the packet.

If VIRTIO_NET_HDR_F_NEEDS_CSUM is not set, the device MUST NOT
rely on the packet checksum being correct.
\paragraph{Packet Transmission Interrupt}\label{sec:Device Types / Network Device / Device Operation / Packet Transmission / Packet Transmission Interrupt}

Often a driver will suppress transmission virtqueue interrupts
and check for used packets in the transmit path of following
packets.

The normal behavior in this interrupt handler is to retrieve
used buffers from the virtqueue and free the corresponding
headers and packets.

\subsubsection{Setting Up Receive Buffers}\label{sec:Device Types / Network Device / Device Operation / Setting Up Receive Buffers}

It is generally a good idea to keep the receive virtqueue as
fully populated as possible: if it runs out, network performance
will suffer.

If the VIRTIO_NET_F_GUEST_TSO4, VIRTIO_NET_F_GUEST_TSO6,
VIRTIO_NET_F_GUEST_UFO, VIRTIO_NET_F_GUEST_USO4 or VIRTIO_NET_F_GUEST_USO6
features are used, the maximum incoming packet
will be 65589 bytes long (14 bytes of Ethernet header, plus 40 bytes of
the IPv6 header, plus 65535 bytes of maximum IPv6 payload including any
extension header), otherwise 1514 bytes.
When VIRTIO_NET_F_HASH_REPORT is not negotiated, the required receive buffer
size is either 65601 or 1526 bytes accounting for 20 bytes of
\field{struct virtio_net_hdr} followed by receive packet.
When VIRTIO_NET_F_HASH_REPORT is negotiated, the required receive buffer
size is either 65609 or 1534 bytes accounting for 12 bytes of
\field{struct virtio_net_hdr} followed by receive packet.

\drivernormative{\paragraph}{Setting Up Receive Buffers}{Device Types / Network Device / Device Operation / Setting Up Receive Buffers}

\begin{itemize}
\item If VIRTIO_NET_F_MRG_RXBUF is not negotiated:
  \begin{itemize}
    \item If VIRTIO_NET_F_GUEST_TSO4, VIRTIO_NET_F_GUEST_TSO6, VIRTIO_NET_F_GUEST_UFO,
	VIRTIO_NET_F_GUEST_USO4 or VIRTIO_NET_F_GUEST_USO6 are negotiated, the driver SHOULD populate
      the receive queue(s) with buffers of at least 65609 bytes if
      VIRTIO_NET_F_HASH_REPORT is negotiated, and of at least 65601 bytes if not.
    \item Otherwise, the driver SHOULD populate the receive queue(s)
      with buffers of at least 1534 bytes if VIRTIO_NET_F_HASH_REPORT
      is negotiated, and of at least 1526 bytes if not.
  \end{itemize}
\item If VIRTIO_NET_F_MRG_RXBUF is negotiated, each buffer MUST be at
least size of \field{struct virtio_net_hdr},
i.e. 20 bytes if VIRTIO_NET_F_HASH_REPORT is negotiated, and 12 bytes if not.
\end{itemize}

\begin{note}
Obviously each buffer can be split across multiple descriptor elements.
\end{note}

When calculating the size of \field{struct virtio_net_hdr}, the driver
MUST consider all the fields inclusive up to \field{padding_reserved},
i.e. 20 bytes if VIRTIO_NET_F_HASH_REPORT is negotiated, and 12 bytes if not.

If VIRTIO_NET_F_MQ is negotiated, each of receiveq1\ldots receiveqN
that will be used SHOULD be populated with receive buffers.

\devicenormative{\paragraph}{Setting Up Receive Buffers}{Device Types / Network Device / Device Operation / Setting Up Receive Buffers}

The device MUST set \field{num_buffers} to the number of descriptors used to
hold the incoming packet.

The device MUST use only a single descriptor if VIRTIO_NET_F_MRG_RXBUF
was not negotiated.
\begin{note}
{This means that \field{num_buffers} will always be 1
if VIRTIO_NET_F_MRG_RXBUF is not negotiated.}
\end{note}

\subsubsection{Processing of Incoming Packets}\label{sec:Device Types / Network Device / Device Operation / Processing of Incoming Packets}
\label{sec:Device Types / Network Device / Device Operation / Processing of Packets}%old label for latexdiff

When a packet is copied into a buffer in the receiveq, the
optimal path is to disable further used buffer notifications for the
receiveq and process packets until no more are found, then re-enable
them.

Processing incoming packets involves:

\begin{enumerate}
\item \field{num_buffers} indicates how many descriptors
  this packet is spread over (including this one): this will
  always be 1 if VIRTIO_NET_F_MRG_RXBUF was not negotiated.
  This allows receipt of large packets without having to allocate large
  buffers: a packet that does not fit in a single buffer can flow
  over to the next buffer, and so on. In this case, there will be
  at least \field{num_buffers} used buffers in the virtqueue, and the device
  chains them together to form a single packet in a way similar to
  how it would store it in a single buffer spread over multiple
  descriptors.
  The other buffers will not begin with a \field{struct virtio_net_hdr}.

\item If
  \field{num_buffers} is one, then the entire packet will be
  contained within this buffer, immediately following the struct
  virtio_net_hdr.
\item If the VIRTIO_NET_F_GUEST_CSUM feature was negotiated, the
  VIRTIO_NET_HDR_F_DATA_VALID bit in \field{flags} can be
  set: if so, device has validated the packet checksum.
  If the VIRTIO_NET_F_GUEST_UDP_TUNNEL_GSO_CSUM feature has been negotiated,
  and the VIRTIO_NET_HDR_F_UDP_TUNNEL_CSUM bit is set in \field{flags},
  both the outer UDP checksum and the inner transport checksum
  have been validated, otherwise only one level of checksums (the outer one
  in case of tunnels) has been validated.
\end{enumerate}

Additionally, VIRTIO_NET_F_GUEST_CSUM, TSO4, TSO6, UDP, UDP_TUNNEL
and ECN features enable receive checksum, large receive offload and ECN
support which are the input equivalents of the transmit checksum,
transmit segmentation offloading and ECN features, as described
in \ref{sec:Device Types / Network Device / Device Operation /
Packet Transmission}:
\begin{enumerate}
\item If the VIRTIO_NET_F_GUEST_TSO4, TSO6, UFO, USO4 or USO6 options were
  negotiated, then \field{gso_type} MAY be something other than
  VIRTIO_NET_HDR_GSO_NONE, and \field{gso_size} field indicates the
  desired MSS (see Packet Transmission point 2).
\item If the VIRTIO_NET_F_RSC_EXT option was negotiated (this
  implies one of VIRTIO_NET_F_GUEST_TSO4, TSO6), the
  device processes also duplicated ACK segments, reports
  number of coalesced TCP segments in \field{csum_start} field and
  number of duplicated ACK segments in \field{csum_offset} field
  and sets bit VIRTIO_NET_HDR_F_RSC_INFO in \field{flags}.
\item If the VIRTIO_NET_F_GUEST_CSUM feature was negotiated, the
  VIRTIO_NET_HDR_F_NEEDS_CSUM bit in \field{flags} can be
  set: if so, the packet checksum at offset \field{csum_offset}
  from \field{csum_start} and any preceding checksums
  have been validated.  The checksum on the packet is incomplete and
  if bit VIRTIO_NET_HDR_F_RSC_INFO is not set in \field{flags},
  then \field{csum_start} and \field{csum_offset} indicate how to calculate it
  (see Packet Transmission point 1).
\begin{note}
If \field{gso_type} differs from VIRTIO_NET_HDR_GSO_NONE and the
VIRTIO_NET_HDR_GSO_UDP_TUNNEL_IPV4 bit or the VIRTIO_NET_HDR_GSO_UDP_TUNNEL_IPV6
bit are not set, \field{csum_offset}
points to the only transport header present in the packet, and there are no
additional preceding checksums validated by VIRTIO_NET_HDR_F_NEEDS_CSUM.
\end{note}
\item If the VIRTIO_NET_F_GUEST_UDP_TUNNEL_GSO option was negotiated and
  \field{gso_type} is not VIRTIO_NET_HDR_GSO_NONE, the
  VIRTIO_NET_HDR_GSO_UDP_TUNNEL_IPV4 bit or the VIRTIO_NET_HDR_GSO_UDP_TUNNEL_IPV6
  bit MAY be set. In such case the \field{outer_th_offset} and
  \field{inner_nh_offset} fields indicate the corresponding
  headers information.
\item If the VIRTIO_NET_F_GUEST_UDP_TUNNEL_GSO_CSUM feature was
negotiated, and
  the VIRTIO_NET_HDR_GSO_UDP_TUNNEL_IPV4 bit or the VIRTIO_NET_HDR_GSO_UDP_TUNNEL_IPV6
  are set in \field{gso_type}, the VIRTIO_NET_HDR_F_UDP_TUNNEL_CSUM bit in the
  \field{flags} can be set: if so, the outer UDP checksum has been validated
  and the UDP header checksum at offset 6 from from \field{outer_th_offset}
  is set to the outer UDP pseudo header checksum.

\begin{note}
If the VIRTIO_NET_HDR_GSO_UDP_TUNNEL_IPV4 bit or VIRTIO_NET_HDR_GSO_UDP_TUNNEL_IPV6
bit are set in \field{gso_type}, the \field{csum_start} field refers to
the inner transport header offset (see Packet Transmission point 1).
If the VIRTIO_NET_HDR_F_UDP_TUNNEL_CSUM bit in \field{flags} is set both
the inner and the outer header checksums have been validated by
VIRTIO_NET_HDR_F_NEEDS_CSUM, otherwise only the inner transport header
checksum has been validated.
\end{note}
\end{enumerate}

If applicable, the device calculates per-packet hash for incoming packets as
defined in \ref{sec:Device Types / Network Device / Device Operation / Processing of Incoming Packets / Hash calculation for incoming packets}.

If applicable, the device reports hash information for incoming packets as
defined in \ref{sec:Device Types / Network Device / Device Operation / Processing of Incoming Packets / Hash reporting for incoming packets}.

\devicenormative{\paragraph}{Processing of Incoming Packets}{Device Types / Network Device / Device Operation / Processing of Incoming Packets}
\label{devicenormative:Device Types / Network Device / Device Operation / Processing of Packets}%old label for latexdiff

If VIRTIO_NET_F_MRG_RXBUF has not been negotiated, the device MUST set
\field{num_buffers} to 1.

If VIRTIO_NET_F_MRG_RXBUF has been negotiated, the device MUST set
\field{num_buffers} to indicate the number of buffers
the packet (including the header) is spread over.

If a receive packet is spread over multiple buffers, the device
MUST use all buffers but the last (i.e. the first \field{num_buffers} -
1 buffers) completely up to the full length of each buffer
supplied by the driver.

The device MUST use all buffers used by a single receive
packet together, such that at least \field{num_buffers} are
observed by driver as used.

If VIRTIO_NET_F_GUEST_CSUM is not negotiated, the device MUST set
\field{flags} to zero and SHOULD supply a fully checksummed
packet to the driver.

If VIRTIO_NET_F_GUEST_TSO4 is not negotiated, the device MUST NOT set
\field{gso_type} to VIRTIO_NET_HDR_GSO_TCPV4.

If VIRTIO_NET_F_GUEST_UDP is not negotiated, the device MUST NOT set
\field{gso_type} to VIRTIO_NET_HDR_GSO_UDP.

If VIRTIO_NET_F_GUEST_TSO6 is not negotiated, the device MUST NOT set
\field{gso_type} to VIRTIO_NET_HDR_GSO_TCPV6.

If none of VIRTIO_NET_F_GUEST_USO4 or VIRTIO_NET_F_GUEST_USO6 have been negotiated,
the device MUST NOT set \field{gso_type} to VIRTIO_NET_HDR_GSO_UDP_L4.

If VIRTIO_NET_F_GUEST_UDP_TUNNEL_GSO is not negotiated, the device MUST NOT set
either the VIRTIO_NET_HDR_GSO_UDP_TUNNEL_IPV4 bit or the
VIRTIO_NET_HDR_GSO_UDP_TUNNEL_IPV6 bit in \field{gso_type}.

If VIRTIO_NET_F_GUEST_UDP_TUNNEL_GSO_CSUM is not negotiated the device MUST NOT set
the VIRTIO_NET_HDR_F_UDP_TUNNEL_CSUM bit in \field{flags}.

The device SHOULD NOT send to the driver TCP packets requiring segmentation offload
which have the Explicit Congestion Notification bit set, unless the
VIRTIO_NET_F_GUEST_ECN feature is negotiated, in which case the
device MUST set the VIRTIO_NET_HDR_GSO_ECN bit in
\field{gso_type}.

If the VIRTIO_NET_F_GUEST_CSUM feature has been negotiated, the
device MAY set the VIRTIO_NET_HDR_F_NEEDS_CSUM bit in
\field{flags}, if so:
\begin{enumerate}
\item the device MUST validate the packet checksum at
	offset \field{csum_offset} from \field{csum_start} as well as all
	preceding offsets;
\item the device MUST set the packet checksum stored in the
	receive buffer to the TCP/UDP pseudo header;
\item the device MUST set \field{csum_start} and
	\field{csum_offset} such that calculating a ones'
	complement checksum from \field{csum_start} up until the
	end of the packet and storing the result at offset
	\field{csum_offset} from  \field{csum_start} will result in a
	fully checksummed packet;
\end{enumerate}

The device MUST NOT send to the driver GSO packets encapsulated in UDP
tunnel and requiring segmentation offload, unless the
VIRTIO_NET_F_GUEST_UDP_TUNNEL_GSO is negotiated, in which case the device MUST set
the VIRTIO_NET_HDR_GSO_UDP_TUNNEL_IPV4 bit or the VIRTIO_NET_HDR_GSO_UDP_TUNNEL_IPV6
bit in \field{gso_type} according to the inner network header protocol type,
MUST set the \field{outer_th_offset} and \field{inner_nh_offset} fields
to the corresponding header information, and the outer UDP header MUST NOT
require checksum offload.

If the VIRTIO_NET_F_GUEST_UDP_TUNNEL_GSO_CSUM feature has not been negotiated,
the device MUST NOT send the driver GSO packets encapsulated in UDP
tunnel and requiring segmentation and outer checksum offload.

If none of the VIRTIO_NET_F_GUEST_TSO4, TSO6, UFO, USO4 or USO6 options have
been negotiated, the device MUST set \field{gso_type} to
VIRTIO_NET_HDR_GSO_NONE.

If \field{gso_type} differs from VIRTIO_NET_HDR_GSO_NONE, then
the device MUST also set the VIRTIO_NET_HDR_F_NEEDS_CSUM bit in
\field{flags} MUST set \field{gso_size} to indicate the desired MSS.
If VIRTIO_NET_F_RSC_EXT was negotiated, the device MUST also
set VIRTIO_NET_HDR_F_RSC_INFO bit in \field{flags},
set \field{csum_start} to number of coalesced TCP segments and
set \field{csum_offset} to number of received duplicated ACK segments.

If VIRTIO_NET_F_RSC_EXT was not negotiated, the device MUST
not set VIRTIO_NET_HDR_F_RSC_INFO bit in \field{flags}.

If one of the VIRTIO_NET_F_GUEST_TSO4, TSO6, UFO, USO4 or USO6 options have
been negotiated, the device SHOULD set \field{hdr_len} to a value
not less than the length of the headers, including the transport
header. If \field{gso_type} has the VIRTIO_NET_HDR_GSO_UDP_TUNNEL_IPV4 bit
or the VIRTIO_NET_HDR_GSO_UDP_TUNNEL_IPV6 bit set, the referenced transport
header is the inner one.

If the VIRTIO_NET_F_GUEST_CSUM feature has been negotiated, the
device MAY set the VIRTIO_NET_HDR_F_DATA_VALID bit in
\field{flags}, if so, the device MUST validate the packet
checksum. If the VIRTIO_NET_F_GUEST_UDP_TUNNEL_GSO_CSUM feature has
been negotiated, and the VIRTIO_NET_HDR_F_UDP_TUNNEL_CSUM bit set in
\field{flags}, both the outer UDP checksum and the inner transport
checksum have been validated.
Otherwise level of checksum is validated: in case of multiple
encapsulated protocols the outermost one.

If either the VIRTIO_NET_HDR_GSO_UDP_TUNNEL_IPV4 bit or the
VIRTIO_NET_HDR_GSO_UDP_TUNNEL_IPV6 bit in \field{gso_type} are set,
the device MUST NOT set the VIRTIO_NET_HDR_F_DATA_VALID bit in
\field{flags}.

If the VIRTIO_NET_F_GUEST_UDP_TUNNEL_GSO_CSUM feature has been negotiated
and either the VIRTIO_NET_HDR_GSO_UDP_TUNNEL_IPV4 bit is set or the
VIRTIO_NET_HDR_GSO_UDP_TUNNEL_IPV6 bit is set in \field{gso_type}, the
device MAY set the VIRTIO_NET_HDR_F_UDP_TUNNEL_CSUM bit in
\field{flags}, if so the device MUST set the packet outer UDP checksum
stored in the receive buffer to the outer UDP pseudo header.

Otherwise, the VIRTIO_NET_F_GUEST_UDP_TUNNEL_GSO_CSUM feature has been
negotiated, either the VIRTIO_NET_HDR_GSO_UDP_TUNNEL_IPV4 bit is set or the
VIRTIO_NET_HDR_GSO_UDP_TUNNEL_IPV6 bit is set in \field{gso_type},
and the bit VIRTIO_NET_HDR_F_UDP_TUNNEL_CSUM is not set in
\field{flags}, the device MUST either provide a zero outer UDP header
checksum or a fully checksummed outer UDP header.

\drivernormative{\paragraph}{Processing of Incoming
Packets}{Device Types / Network Device / Device Operation /
Processing of Incoming Packets}

The driver MUST ignore \field{flag} bits that it does not recognize.

If VIRTIO_NET_HDR_F_NEEDS_CSUM bit in \field{flags} is not set or
if VIRTIO_NET_HDR_F_RSC_INFO bit \field{flags} is set, the
driver MUST NOT use the \field{csum_start} and \field{csum_offset}.

If one of the VIRTIO_NET_F_GUEST_TSO4, TSO6, UFO, USO4 or USO6 options have
been negotiated, the driver MAY use \field{hdr_len} only as a hint about the
transport header size.
The driver MUST NOT rely on \field{hdr_len} to be correct.
\begin{note}
This is due to various bugs in implementations.
\end{note}

If neither VIRTIO_NET_HDR_F_NEEDS_CSUM nor
VIRTIO_NET_HDR_F_DATA_VALID is set, the driver MUST NOT
rely on the packet checksum being correct.

If both the VIRTIO_NET_HDR_GSO_UDP_TUNNEL_IPV4 bit and
the VIRTIO_NET_HDR_GSO_UDP_TUNNEL_IPV6 bit in in \field{gso_type} are set,
the driver MUST NOT accept the packet.

If the VIRTIO_NET_HDR_GSO_UDP_TUNNEL_IPV4 bit or the VIRTIO_NET_HDR_GSO_UDP_TUNNEL_IPV6
bit in \field{gso_type} are not set, the driver MUST NOT use the
\field{outer_th_offset} and \field{inner_nh_offset}.

If either the VIRTIO_NET_HDR_GSO_UDP_TUNNEL_IPV4 bit or
the VIRTIO_NET_HDR_GSO_UDP_TUNNEL_IPV6 bit in \field{gso_type} are set, and any of
the following is true:
\begin{itemize}
\item the VIRTIO_NET_HDR_F_NEEDS_CSUM bit is not set in \field{flags}
\item the VIRTIO_NET_HDR_F_DATA_VALID bit is set in \field{flags}
\item the \field{gso_type} excluding the VIRTIO_NET_HDR_GSO_UDP_TUNNEL_IPV4
bit and the VIRTIO_NET_HDR_GSO_UDP_TUNNEL_IPV6 bit is VIRTIO_NET_HDR_GSO_NONE
\end{itemize}
the driver MUST NOT accept the packet.

If the VIRTIO_NET_HDR_F_UDP_TUNNEL_CSUM bit and the VIRTIO_NET_HDR_F_NEEDS_CSUM
bit in \field{flags} are set,
and both the bits VIRTIO_NET_HDR_GSO_UDP_TUNNEL_IPV4 and
VIRTIO_NET_HDR_GSO_UDP_TUNNEL_IPV6 in \field{gso_type} are not set,
the driver MOST NOT accept the packet.

\paragraph{Hash calculation for incoming packets}
\label{sec:Device Types / Network Device / Device Operation / Processing of Incoming Packets / Hash calculation for incoming packets}

A device attempts to calculate a per-packet hash in the following cases:
\begin{itemize}
\item The feature VIRTIO_NET_F_RSS was negotiated. The device uses the hash to determine the receive virtqueue to place incoming packets.
\item The feature VIRTIO_NET_F_HASH_REPORT was negotiated. The device reports the hash value and the hash type with the packet.
\end{itemize}

If the feature VIRTIO_NET_F_RSS was negotiated:
\begin{itemize}
\item The device uses \field{hash_types} of the virtio_net_rss_config structure as 'Enabled hash types' bitmask.
\item If additionally the feature VIRTIO_NET_F_HASH_TUNNEL was negotiated, the device uses \field{enabled_tunnel_types} of the
      virtnet_hash_tunnel structure as 'Encapsulation types enabled for inner header hash' bitmask.
\item The device uses a key as defined in \field{hash_key_data} and \field{hash_key_length} of the virtio_net_rss_config structure (see
\ref{sec:Device Types / Network Device / Device Operation / Control Virtqueue / Receive-side scaling (RSS) / Setting RSS parameters}).
\end{itemize}

If the feature VIRTIO_NET_F_RSS was not negotiated:
\begin{itemize}
\item The device uses \field{hash_types} of the virtio_net_hash_config structure as 'Enabled hash types' bitmask.
\item If additionally the feature VIRTIO_NET_F_HASH_TUNNEL was negotiated, the device uses \field{enabled_tunnel_types} of the
      virtnet_hash_tunnel structure as 'Encapsulation types enabled for inner header hash' bitmask.
\item The device uses a key as defined in \field{hash_key_data} and \field{hash_key_length} of the virtio_net_hash_config structure (see
\ref{sec:Device Types / Network Device / Device Operation / Control Virtqueue / Automatic receive steering in multiqueue mode / Hash calculation}).
\end{itemize}

Note that if the device offers VIRTIO_NET_F_HASH_REPORT, even if it supports only one pair of virtqueues, it MUST support
at least one of commands of VIRTIO_NET_CTRL_MQ class to configure reported hash parameters:
\begin{itemize}
\item If the device offers VIRTIO_NET_F_RSS, it MUST support VIRTIO_NET_CTRL_MQ_RSS_CONFIG command per
 \ref{sec:Device Types / Network Device / Device Operation / Control Virtqueue / Receive-side scaling (RSS) / Setting RSS parameters}.
\item Otherwise the device MUST support VIRTIO_NET_CTRL_MQ_HASH_CONFIG command per
 \ref{sec:Device Types / Network Device / Device Operation / Control Virtqueue / Automatic receive steering in multiqueue mode / Hash calculation}.
\end{itemize}

The per-packet hash calculation can depend on the IP packet type. See
\hyperref[intro:IP]{[IP]}, \hyperref[intro:UDP]{[UDP]} and \hyperref[intro:TCP]{[TCP]}.

\subparagraph{Supported/enabled hash types}
\label{sec:Device Types / Network Device / Device Operation / Processing of Incoming Packets / Hash calculation for incoming packets / Supported/enabled hash types}
Hash types applicable for IPv4 packets:
\begin{lstlisting}
#define VIRTIO_NET_HASH_TYPE_IPv4              (1 << 0)
#define VIRTIO_NET_HASH_TYPE_TCPv4             (1 << 1)
#define VIRTIO_NET_HASH_TYPE_UDPv4             (1 << 2)
\end{lstlisting}
Hash types applicable for IPv6 packets without extension headers
\begin{lstlisting}
#define VIRTIO_NET_HASH_TYPE_IPv6              (1 << 3)
#define VIRTIO_NET_HASH_TYPE_TCPv6             (1 << 4)
#define VIRTIO_NET_HASH_TYPE_UDPv6             (1 << 5)
\end{lstlisting}
Hash types applicable for IPv6 packets with extension headers
\begin{lstlisting}
#define VIRTIO_NET_HASH_TYPE_IP_EX             (1 << 6)
#define VIRTIO_NET_HASH_TYPE_TCP_EX            (1 << 7)
#define VIRTIO_NET_HASH_TYPE_UDP_EX            (1 << 8)
\end{lstlisting}

\subparagraph{IPv4 packets}
\label{sec:Device Types / Network Device / Device Operation / Processing of Incoming Packets / Hash calculation for incoming packets / IPv4 packets}
The device calculates the hash on IPv4 packets according to 'Enabled hash types' bitmask as follows:
\begin{itemize}
\item If VIRTIO_NET_HASH_TYPE_TCPv4 is set and the packet has
a TCP header, the hash is calculated over the following fields:
\begin{itemize}
\item Source IP address
\item Destination IP address
\item Source TCP port
\item Destination TCP port
\end{itemize}
\item Else if VIRTIO_NET_HASH_TYPE_UDPv4 is set and the
packet has a UDP header, the hash is calculated over the following fields:
\begin{itemize}
\item Source IP address
\item Destination IP address
\item Source UDP port
\item Destination UDP port
\end{itemize}
\item Else if VIRTIO_NET_HASH_TYPE_IPv4 is set, the hash is
calculated over the following fields:
\begin{itemize}
\item Source IP address
\item Destination IP address
\end{itemize}
\item Else the device does not calculate the hash
\end{itemize}

\subparagraph{IPv6 packets without extension header}
\label{sec:Device Types / Network Device / Device Operation / Processing of Incoming Packets / Hash calculation for incoming packets / IPv6 packets without extension header}
The device calculates the hash on IPv6 packets without extension
headers according to 'Enabled hash types' bitmask as follows:
\begin{itemize}
\item If VIRTIO_NET_HASH_TYPE_TCPv6 is set and the packet has
a TCPv6 header, the hash is calculated over the following fields:
\begin{itemize}
\item Source IPv6 address
\item Destination IPv6 address
\item Source TCP port
\item Destination TCP port
\end{itemize}
\item Else if VIRTIO_NET_HASH_TYPE_UDPv6 is set and the
packet has a UDPv6 header, the hash is calculated over the following fields:
\begin{itemize}
\item Source IPv6 address
\item Destination IPv6 address
\item Source UDP port
\item Destination UDP port
\end{itemize}
\item Else if VIRTIO_NET_HASH_TYPE_IPv6 is set, the hash is
calculated over the following fields:
\begin{itemize}
\item Source IPv6 address
\item Destination IPv6 address
\end{itemize}
\item Else the device does not calculate the hash
\end{itemize}

\subparagraph{IPv6 packets with extension header}
\label{sec:Device Types / Network Device / Device Operation / Processing of Incoming Packets / Hash calculation for incoming packets / IPv6 packets with extension header}
The device calculates the hash on IPv6 packets with extension
headers according to 'Enabled hash types' bitmask as follows:
\begin{itemize}
\item If VIRTIO_NET_HASH_TYPE_TCP_EX is set and the packet
has a TCPv6 header, the hash is calculated over the following fields:
\begin{itemize}
\item Home address from the home address option in the IPv6 destination options header. If the extension header is not present, use the Source IPv6 address.
\item IPv6 address that is contained in the Routing-Header-Type-2 from the associated extension header. If the extension header is not present, use the Destination IPv6 address.
\item Source TCP port
\item Destination TCP port
\end{itemize}
\item Else if VIRTIO_NET_HASH_TYPE_UDP_EX is set and the
packet has a UDPv6 header, the hash is calculated over the following fields:
\begin{itemize}
\item Home address from the home address option in the IPv6 destination options header. If the extension header is not present, use the Source IPv6 address.
\item IPv6 address that is contained in the Routing-Header-Type-2 from the associated extension header. If the extension header is not present, use the Destination IPv6 address.
\item Source UDP port
\item Destination UDP port
\end{itemize}
\item Else if VIRTIO_NET_HASH_TYPE_IP_EX is set, the hash is
calculated over the following fields:
\begin{itemize}
\item Home address from the home address option in the IPv6 destination options header. If the extension header is not present, use the Source IPv6 address.
\item IPv6 address that is contained in the Routing-Header-Type-2 from the associated extension header. If the extension header is not present, use the Destination IPv6 address.
\end{itemize}
\item Else skip IPv6 extension headers and calculate the hash as
defined for an IPv6 packet without extension headers
(see \ref{sec:Device Types / Network Device / Device Operation / Processing of Incoming Packets / Hash calculation for incoming packets / IPv6 packets without extension header}).
\end{itemize}

\paragraph{Inner Header Hash}
\label{sec:Device Types / Network Device / Device Operation / Processing of Incoming Packets / Inner Header Hash}

If VIRTIO_NET_F_HASH_TUNNEL has been negotiated, the driver can send the command
VIRTIO_NET_CTRL_HASH_TUNNEL_SET to configure the calculation of the inner header hash.

\begin{lstlisting}
struct virtnet_hash_tunnel {
    le32 enabled_tunnel_types;
};

#define VIRTIO_NET_CTRL_HASH_TUNNEL 7
 #define VIRTIO_NET_CTRL_HASH_TUNNEL_SET 0
\end{lstlisting}

Field \field{enabled_tunnel_types} contains the bitmask of encapsulation types enabled for inner header hash.
See \ref{sec:Device Types / Network Device / Device Operation / Processing of Incoming Packets /
Hash calculation for incoming packets / Encapsulation types supported/enabled for inner header hash}.

The class VIRTIO_NET_CTRL_HASH_TUNNEL has one command:
VIRTIO_NET_CTRL_HASH_TUNNEL_SET sets \field{enabled_tunnel_types} for the device using the
virtnet_hash_tunnel structure, which is read-only for the device.

Inner header hash is disabled by VIRTIO_NET_CTRL_HASH_TUNNEL_SET with \field{enabled_tunnel_types} set to 0.

Initially (before the driver sends any VIRTIO_NET_CTRL_HASH_TUNNEL_SET command) all
encapsulation types are disabled for inner header hash.

\subparagraph{Encapsulated packet}
\label{sec:Device Types / Network Device / Device Operation / Processing of Incoming Packets / Hash calculation for incoming packets / Encapsulated packet}

Multiple tunneling protocols allow encapsulating an inner, payload packet in an outer, encapsulated packet.
The encapsulated packet thus contains an outer header and an inner header, and the device calculates the
hash over either the inner header or the outer header.

If VIRTIO_NET_F_HASH_TUNNEL is negotiated and a received encapsulated packet's outer header matches one of the
encapsulation types enabled in \field{enabled_tunnel_types}, then the device uses the inner header for hash
calculations (only a single level of encapsulation is currently supported).

If VIRTIO_NET_F_HASH_TUNNEL is negotiated and a received packet's (outer) header does not match any encapsulation
types enabled in \field{enabled_tunnel_types}, then the device uses the outer header for hash calculations.

\subparagraph{Encapsulation types supported/enabled for inner header hash}
\label{sec:Device Types / Network Device / Device Operation / Processing of Incoming Packets /
Hash calculation for incoming packets / Encapsulation types supported/enabled for inner header hash}

Encapsulation types applicable for inner header hash:
\begin{lstlisting}[escapechar=|]
#define VIRTIO_NET_HASH_TUNNEL_TYPE_GRE_2784    (1 << 0) /* |\hyperref[intro:rfc2784]{[RFC2784]}| */
#define VIRTIO_NET_HASH_TUNNEL_TYPE_GRE_2890    (1 << 1) /* |\hyperref[intro:rfc2890]{[RFC2890]}| */
#define VIRTIO_NET_HASH_TUNNEL_TYPE_GRE_7676    (1 << 2) /* |\hyperref[intro:rfc7676]{[RFC7676]}| */
#define VIRTIO_NET_HASH_TUNNEL_TYPE_GRE_UDP     (1 << 3) /* |\hyperref[intro:rfc8086]{[GRE-in-UDP]}| */
#define VIRTIO_NET_HASH_TUNNEL_TYPE_VXLAN       (1 << 4) /* |\hyperref[intro:vxlan]{[VXLAN]}| */
#define VIRTIO_NET_HASH_TUNNEL_TYPE_VXLAN_GPE   (1 << 5) /* |\hyperref[intro:vxlan-gpe]{[VXLAN-GPE]}| */
#define VIRTIO_NET_HASH_TUNNEL_TYPE_GENEVE      (1 << 6) /* |\hyperref[intro:geneve]{[GENEVE]}| */
#define VIRTIO_NET_HASH_TUNNEL_TYPE_IPIP        (1 << 7) /* |\hyperref[intro:ipip]{[IPIP]}| */
#define VIRTIO_NET_HASH_TUNNEL_TYPE_NVGRE       (1 << 8) /* |\hyperref[intro:nvgre]{[NVGRE]}| */
\end{lstlisting}

\subparagraph{Advice}
Example uses of the inner header hash:
\begin{itemize}
\item Legacy tunneling protocols, lacking the outer header entropy, can use RSS with the inner header hash to
      distribute flows with identical outer but different inner headers across various queues, improving performance.
\item Identify an inner flow distributed across multiple outer tunnels.
\end{itemize}

As using the inner header hash completely discards the outer header entropy, care must be taken
if the inner header is controlled by an adversary, as the adversary can then intentionally create
configurations with insufficient entropy.

Besides disabling the inner header hash, mitigations would depend on how the hash is used. When the hash
use is limited to the RSS queue selection, the inner header hash may have quality of service (QoS) limitations.

\devicenormative{\subparagraph}{Inner Header Hash}{Device Types / Network Device / Device Operation / Control Virtqueue / Inner Header Hash}

If the (outer) header of the received packet does not match any encapsulation types enabled
in \field{enabled_tunnel_types}, the device MUST calculate the hash on the outer header.

If the device receives any bits in \field{enabled_tunnel_types} which are not set in \field{supported_tunnel_types},
it SHOULD respond to the VIRTIO_NET_CTRL_HASH_TUNNEL_SET command with VIRTIO_NET_ERR.

If the driver sets \field{enabled_tunnel_types} to 0 through VIRTIO_NET_CTRL_HASH_TUNNEL_SET or upon the device reset,
the device MUST disable the inner header hash for all encapsulation types.

\drivernormative{\subparagraph}{Inner Header Hash}{Device Types / Network Device / Device Operation / Control Virtqueue / Inner Header Hash}

The driver MUST have negotiated the VIRTIO_NET_F_HASH_TUNNEL feature when issuing the VIRTIO_NET_CTRL_HASH_TUNNEL_SET command.

The driver MUST NOT set any bits in \field{enabled_tunnel_types} which are not set in \field{supported_tunnel_types}.

The driver MUST ignore bits in \field{supported_tunnel_types} which are not documented in this specification.

\paragraph{Hash reporting for incoming packets}
\label{sec:Device Types / Network Device / Device Operation / Processing of Incoming Packets / Hash reporting for incoming packets}

If VIRTIO_NET_F_HASH_REPORT was negotiated and
 the device has calculated the hash for the packet, the device fills \field{hash_report} with the report type of calculated hash
and \field{hash_value} with the value of calculated hash.

If VIRTIO_NET_F_HASH_REPORT was negotiated but due to any reason the
hash was not calculated, the device sets \field{hash_report} to VIRTIO_NET_HASH_REPORT_NONE.

Possible values that the device can report in \field{hash_report} are defined below.
They correspond to supported hash types defined in
\ref{sec:Device Types / Network Device / Device Operation / Processing of Incoming Packets / Hash calculation for incoming packets / Supported/enabled hash types}
as follows:

VIRTIO_NET_HASH_TYPE_XXX = 1 << (VIRTIO_NET_HASH_REPORT_XXX - 1)

\begin{lstlisting}
#define VIRTIO_NET_HASH_REPORT_NONE            0
#define VIRTIO_NET_HASH_REPORT_IPv4            1
#define VIRTIO_NET_HASH_REPORT_TCPv4           2
#define VIRTIO_NET_HASH_REPORT_UDPv4           3
#define VIRTIO_NET_HASH_REPORT_IPv6            4
#define VIRTIO_NET_HASH_REPORT_TCPv6           5
#define VIRTIO_NET_HASH_REPORT_UDPv6           6
#define VIRTIO_NET_HASH_REPORT_IPv6_EX         7
#define VIRTIO_NET_HASH_REPORT_TCPv6_EX        8
#define VIRTIO_NET_HASH_REPORT_UDPv6_EX        9
\end{lstlisting}

\subsubsection{Control Virtqueue}\label{sec:Device Types / Network Device / Device Operation / Control Virtqueue}

The driver uses the control virtqueue (if VIRTIO_NET_F_CTRL_VQ is
negotiated) to send commands to manipulate various features of
the device which would not easily map into the configuration
space.

All commands are of the following form:

\begin{lstlisting}
struct virtio_net_ctrl {
        u8 class;
        u8 command;
        u8 command-specific-data[];
        u8 ack;
        u8 command-specific-result[];
};

/* ack values */
#define VIRTIO_NET_OK     0
#define VIRTIO_NET_ERR    1
\end{lstlisting}

The \field{class}, \field{command} and command-specific-data are set by the
driver, and the device sets the \field{ack} byte and optionally
\field{command-specific-result}. There is little the driver can
do except issue a diagnostic if \field{ack} is not VIRTIO_NET_OK.

The command VIRTIO_NET_CTRL_STATS_QUERY and VIRTIO_NET_CTRL_STATS_GET contain
\field{command-specific-result}.

\paragraph{Packet Receive Filtering}\label{sec:Device Types / Network Device / Device Operation / Control Virtqueue / Packet Receive Filtering}
\label{sec:Device Types / Network Device / Device Operation / Control Virtqueue / Setting Promiscuous Mode}%old label for latexdiff

If the VIRTIO_NET_F_CTRL_RX and VIRTIO_NET_F_CTRL_RX_EXTRA
features are negotiated, the driver can send control commands for
promiscuous mode, multicast, unicast and broadcast receiving.

\begin{note}
In general, these commands are best-effort: unwanted
packets could still arrive.
\end{note}

\begin{lstlisting}
#define VIRTIO_NET_CTRL_RX    0
 #define VIRTIO_NET_CTRL_RX_PROMISC      0
 #define VIRTIO_NET_CTRL_RX_ALLMULTI     1
 #define VIRTIO_NET_CTRL_RX_ALLUNI       2
 #define VIRTIO_NET_CTRL_RX_NOMULTI      3
 #define VIRTIO_NET_CTRL_RX_NOUNI        4
 #define VIRTIO_NET_CTRL_RX_NOBCAST      5
\end{lstlisting}


\devicenormative{\subparagraph}{Packet Receive Filtering}{Device Types / Network Device / Device Operation / Control Virtqueue / Packet Receive Filtering}

If the VIRTIO_NET_F_CTRL_RX feature has been negotiated,
the device MUST support the following VIRTIO_NET_CTRL_RX class
commands:
\begin{itemize}
\item VIRTIO_NET_CTRL_RX_PROMISC turns promiscuous mode on and
off. The command-specific-data is one byte containing 0 (off) or
1 (on). If promiscuous mode is on, the device SHOULD receive all
incoming packets.
This SHOULD take effect even if one of the other modes set by
a VIRTIO_NET_CTRL_RX class command is on.
\item VIRTIO_NET_CTRL_RX_ALLMULTI turns all-multicast receive on and
off. The command-specific-data is one byte containing 0 (off) or
1 (on). When all-multicast receive is on the device SHOULD allow
all incoming multicast packets.
\end{itemize}

If the VIRTIO_NET_F_CTRL_RX_EXTRA feature has been negotiated,
the device MUST support the following VIRTIO_NET_CTRL_RX class
commands:
\begin{itemize}
\item VIRTIO_NET_CTRL_RX_ALLUNI turns all-unicast receive on and
off. The command-specific-data is one byte containing 0 (off) or
1 (on). When all-unicast receive is on the device SHOULD allow
all incoming unicast packets.
\item VIRTIO_NET_CTRL_RX_NOMULTI suppresses multicast receive.
The command-specific-data is one byte containing 0 (multicast
receive allowed) or 1 (multicast receive suppressed).
When multicast receive is suppressed, the device SHOULD NOT
send multicast packets to the driver.
This SHOULD take effect even if VIRTIO_NET_CTRL_RX_ALLMULTI is on.
This filter SHOULD NOT apply to broadcast packets.
\item VIRTIO_NET_CTRL_RX_NOUNI suppresses unicast receive.
The command-specific-data is one byte containing 0 (unicast
receive allowed) or 1 (unicast receive suppressed).
When unicast receive is suppressed, the device SHOULD NOT
send unicast packets to the driver.
This SHOULD take effect even if VIRTIO_NET_CTRL_RX_ALLUNI is on.
\item VIRTIO_NET_CTRL_RX_NOBCAST suppresses broadcast receive.
The command-specific-data is one byte containing 0 (broadcast
receive allowed) or 1 (broadcast receive suppressed).
When broadcast receive is suppressed, the device SHOULD NOT
send broadcast packets to the driver.
This SHOULD take effect even if VIRTIO_NET_CTRL_RX_ALLMULTI is on.
\end{itemize}

\drivernormative{\subparagraph}{Packet Receive Filtering}{Device Types / Network Device / Device Operation / Control Virtqueue / Packet Receive Filtering}

If the VIRTIO_NET_F_CTRL_RX feature has not been negotiated,
the driver MUST NOT issue commands VIRTIO_NET_CTRL_RX_PROMISC or
VIRTIO_NET_CTRL_RX_ALLMULTI.

If the VIRTIO_NET_F_CTRL_RX_EXTRA feature has not been negotiated,
the driver MUST NOT issue commands
 VIRTIO_NET_CTRL_RX_ALLUNI,
 VIRTIO_NET_CTRL_RX_NOMULTI,
 VIRTIO_NET_CTRL_RX_NOUNI or
 VIRTIO_NET_CTRL_RX_NOBCAST.

\paragraph{Setting MAC Address Filtering}\label{sec:Device Types / Network Device / Device Operation / Control Virtqueue / Setting MAC Address Filtering}

If the VIRTIO_NET_F_CTRL_RX feature is negotiated, the driver can
send control commands for MAC address filtering.

\begin{lstlisting}
struct virtio_net_ctrl_mac {
        le32 entries;
        u8 macs[entries][6];
};

#define VIRTIO_NET_CTRL_MAC    1
 #define VIRTIO_NET_CTRL_MAC_TABLE_SET        0
 #define VIRTIO_NET_CTRL_MAC_ADDR_SET         1
\end{lstlisting}

The device can filter incoming packets by any number of destination
MAC addresses\footnote{Since there are no guarantees, it can use a hash filter or
silently switch to allmulti or promiscuous mode if it is given too
many addresses.
}. This table is set using the class
VIRTIO_NET_CTRL_MAC and the command VIRTIO_NET_CTRL_MAC_TABLE_SET. The
command-specific-data is two variable length tables of 6-byte MAC
addresses (as described in struct virtio_net_ctrl_mac). The first table contains unicast addresses, and the second
contains multicast addresses.

The VIRTIO_NET_CTRL_MAC_ADDR_SET command is used to set the
default MAC address which rx filtering
accepts (and if VIRTIO_NET_F_MAC has been negotiated,
this will be reflected in \field{mac} in config space).

The command-specific-data for VIRTIO_NET_CTRL_MAC_ADDR_SET is
the 6-byte MAC address.

\devicenormative{\subparagraph}{Setting MAC Address Filtering}{Device Types / Network Device / Device Operation / Control Virtqueue / Setting MAC Address Filtering}

The device MUST have an empty MAC filtering table on reset.

The device MUST update the MAC filtering table before it consumes
the VIRTIO_NET_CTRL_MAC_TABLE_SET command.

The device MUST update \field{mac} in config space before it consumes
the VIRTIO_NET_CTRL_MAC_ADDR_SET command, if VIRTIO_NET_F_MAC has
been negotiated.

The device SHOULD drop incoming packets which have a destination MAC which
matches neither the \field{mac} (or that set with VIRTIO_NET_CTRL_MAC_ADDR_SET)
nor the MAC filtering table.

\drivernormative{\subparagraph}{Setting MAC Address Filtering}{Device Types / Network Device / Device Operation / Control Virtqueue / Setting MAC Address Filtering}

If VIRTIO_NET_F_CTRL_RX has not been negotiated,
the driver MUST NOT issue VIRTIO_NET_CTRL_MAC class commands.

If VIRTIO_NET_F_CTRL_RX has been negotiated,
the driver SHOULD issue VIRTIO_NET_CTRL_MAC_ADDR_SET
to set the default mac if it is different from \field{mac}.

The driver MUST follow the VIRTIO_NET_CTRL_MAC_TABLE_SET command
by a le32 number, followed by that number of non-multicast
MAC addresses, followed by another le32 number, followed by
that number of multicast addresses.  Either number MAY be 0.

\subparagraph{Legacy Interface: Setting MAC Address Filtering}\label{sec:Device Types / Network Device / Device Operation / Control Virtqueue / Setting MAC Address Filtering / Legacy Interface: Setting MAC Address Filtering}
When using the legacy interface, transitional devices and drivers
MUST format \field{entries} in struct virtio_net_ctrl_mac
according to the native endian of the guest rather than
(necessarily when not using the legacy interface) little-endian.

Legacy drivers that didn't negotiate VIRTIO_NET_F_CTRL_MAC_ADDR
changed \field{mac} in config space when NIC is accepting
incoming packets. These drivers always wrote the mac value from
first to last byte, therefore after detecting such drivers,
a transitional device MAY defer MAC update, or MAY defer
processing incoming packets until driver writes the last byte
of \field{mac} in the config space.

\paragraph{VLAN Filtering}\label{sec:Device Types / Network Device / Device Operation / Control Virtqueue / VLAN Filtering}

If the driver negotiates the VIRTIO_NET_F_CTRL_VLAN feature, it
can control a VLAN filter table in the device. The VLAN filter
table applies only to VLAN tagged packets.

When VIRTIO_NET_F_CTRL_VLAN is negotiated, the device starts with
an empty VLAN filter table.

\begin{note}
Similar to the MAC address based filtering, the VLAN filtering
is also best-effort: unwanted packets could still arrive.
\end{note}

\begin{lstlisting}
#define VIRTIO_NET_CTRL_VLAN       2
 #define VIRTIO_NET_CTRL_VLAN_ADD             0
 #define VIRTIO_NET_CTRL_VLAN_DEL             1
\end{lstlisting}

Both the VIRTIO_NET_CTRL_VLAN_ADD and VIRTIO_NET_CTRL_VLAN_DEL
command take a little-endian 16-bit VLAN id as the command-specific-data.

VIRTIO_NET_CTRL_VLAN_ADD command adds the specified VLAN to the
VLAN filter table.

VIRTIO_NET_CTRL_VLAN_DEL command removes the specified VLAN from
the VLAN filter table.

\devicenormative{\subparagraph}{VLAN Filtering}{Device Types / Network Device / Device Operation / Control Virtqueue / VLAN Filtering}

When VIRTIO_NET_F_CTRL_VLAN is not negotiated, the device MUST
accept all VLAN tagged packets.

When VIRTIO_NET_F_CTRL_VLAN is negotiated, the device MUST
accept all VLAN tagged packets whose VLAN tag is present in
the VLAN filter table and SHOULD drop all VLAN tagged packets
whose VLAN tag is absent in the VLAN filter table.

\subparagraph{Legacy Interface: VLAN Filtering}\label{sec:Device Types / Network Device / Device Operation / Control Virtqueue / VLAN Filtering / Legacy Interface: VLAN Filtering}
When using the legacy interface, transitional devices and drivers
MUST format the VLAN id
according to the native endian of the guest rather than
(necessarily when not using the legacy interface) little-endian.

\paragraph{Gratuitous Packet Sending}\label{sec:Device Types / Network Device / Device Operation / Control Virtqueue / Gratuitous Packet Sending}

If the driver negotiates the VIRTIO_NET_F_GUEST_ANNOUNCE (depends
on VIRTIO_NET_F_CTRL_VQ), the device can ask the driver to send gratuitous
packets; this is usually done after the guest has been physically
migrated, and needs to announce its presence on the new network
links. (As hypervisor does not have the knowledge of guest
network configuration (eg. tagged vlan) it is simplest to prod
the guest in this way).

\begin{lstlisting}
#define VIRTIO_NET_CTRL_ANNOUNCE       3
 #define VIRTIO_NET_CTRL_ANNOUNCE_ACK             0
\end{lstlisting}

The driver checks VIRTIO_NET_S_ANNOUNCE bit in the device configuration \field{status} field
when it notices the changes of device configuration. The
command VIRTIO_NET_CTRL_ANNOUNCE_ACK is used to indicate that
driver has received the notification and device clears the
VIRTIO_NET_S_ANNOUNCE bit in \field{status}.

Processing this notification involves:

\begin{enumerate}
\item Sending the gratuitous packets (eg. ARP) or marking there are pending
  gratuitous packets to be sent and letting deferred routine to
  send them.

\item Sending VIRTIO_NET_CTRL_ANNOUNCE_ACK command through control
  vq.
\end{enumerate}

\drivernormative{\subparagraph}{Gratuitous Packet Sending}{Device Types / Network Device / Device Operation / Control Virtqueue / Gratuitous Packet Sending}

If the driver negotiates VIRTIO_NET_F_GUEST_ANNOUNCE, it SHOULD notify
network peers of its new location after it sees the VIRTIO_NET_S_ANNOUNCE bit
in \field{status}.  The driver MUST send a command on the command queue
with class VIRTIO_NET_CTRL_ANNOUNCE and command VIRTIO_NET_CTRL_ANNOUNCE_ACK.

\devicenormative{\subparagraph}{Gratuitous Packet Sending}{Device Types / Network Device / Device Operation / Control Virtqueue / Gratuitous Packet Sending}

If VIRTIO_NET_F_GUEST_ANNOUNCE is negotiated, the device MUST clear the
VIRTIO_NET_S_ANNOUNCE bit in \field{status} upon receipt of a command buffer
with class VIRTIO_NET_CTRL_ANNOUNCE and command VIRTIO_NET_CTRL_ANNOUNCE_ACK
before marking the buffer as used.

\paragraph{Device operation in multiqueue mode}\label{sec:Device Types / Network Device / Device Operation / Control Virtqueue / Device operation in multiqueue mode}

This specification defines the following modes that a device MAY implement for operation with multiple transmit/receive virtqueues:
\begin{itemize}
\item Automatic receive steering as defined in \ref{sec:Device Types / Network Device / Device Operation / Control Virtqueue / Automatic receive steering in multiqueue mode}.
 If a device supports this mode, it offers the VIRTIO_NET_F_MQ feature bit.
\item Receive-side scaling as defined in \ref{devicenormative:Device Types / Network Device / Device Operation / Control Virtqueue / Receive-side scaling (RSS) / RSS processing}.
 If a device supports this mode, it offers the VIRTIO_NET_F_RSS feature bit.
\end{itemize}

A device MAY support one of these features or both. The driver MAY negotiate any set of these features that the device supports.

Multiqueue is disabled by default.

The driver enables multiqueue by sending a command using \field{class} VIRTIO_NET_CTRL_MQ. The \field{command} selects the mode of multiqueue operation, as follows:
\begin{lstlisting}
#define VIRTIO_NET_CTRL_MQ    4
 #define VIRTIO_NET_CTRL_MQ_VQ_PAIRS_SET        0 (for automatic receive steering)
 #define VIRTIO_NET_CTRL_MQ_RSS_CONFIG          1 (for configurable receive steering)
 #define VIRTIO_NET_CTRL_MQ_HASH_CONFIG         2 (for configurable hash calculation)
\end{lstlisting}

If more than one multiqueue mode is negotiated, the resulting device configuration is defined by the last command sent by the driver.

\paragraph{Automatic receive steering in multiqueue mode}\label{sec:Device Types / Network Device / Device Operation / Control Virtqueue / Automatic receive steering in multiqueue mode}

If the driver negotiates the VIRTIO_NET_F_MQ feature bit (depends on VIRTIO_NET_F_CTRL_VQ), it MAY transmit outgoing packets on one
of the multiple transmitq1\ldots transmitqN and ask the device to
queue incoming packets into one of the multiple receiveq1\ldots receiveqN
depending on the packet flow.

The driver enables multiqueue by
sending the VIRTIO_NET_CTRL_MQ_VQ_PAIRS_SET command, specifying
the number of the transmit and receive queues to be used up to
\field{max_virtqueue_pairs}; subsequently,
transmitq1\ldots transmitqn and receiveq1\ldots receiveqn where
n=\field{virtqueue_pairs} MAY be used.
\begin{lstlisting}
struct virtio_net_ctrl_mq_pairs_set {
       le16 virtqueue_pairs;
};
#define VIRTIO_NET_CTRL_MQ_VQ_PAIRS_MIN        1
#define VIRTIO_NET_CTRL_MQ_VQ_PAIRS_MAX        0x8000

\end{lstlisting}

When multiqueue is enabled by VIRTIO_NET_CTRL_MQ_VQ_PAIRS_SET command, the device MUST use automatic receive steering
based on packet flow. Programming of the receive steering
classificator is implicit. After the driver transmitted a packet of a
flow on transmitqX, the device SHOULD cause incoming packets for that flow to
be steered to receiveqX. For uni-directional protocols, or where
no packets have been transmitted yet, the device MAY steer a packet
to a random queue out of the specified receiveq1\ldots receiveqn.

Multiqueue is disabled by VIRTIO_NET_CTRL_MQ_VQ_PAIRS_SET with \field{virtqueue_pairs} to 1 (this is
the default) and waiting for the device to use the command buffer.

\drivernormative{\subparagraph}{Automatic receive steering in multiqueue mode}{Device Types / Network Device / Device Operation / Control Virtqueue / Automatic receive steering in multiqueue mode}

The driver MUST configure the virtqueues before enabling them with the
VIRTIO_NET_CTRL_MQ_VQ_PAIRS_SET command.

The driver MUST NOT request a \field{virtqueue_pairs} of 0 or
greater than \field{max_virtqueue_pairs} in the device configuration space.

The driver MUST queue packets only on any transmitq1 before the
VIRTIO_NET_CTRL_MQ_VQ_PAIRS_SET command.

The driver MUST NOT queue packets on transmit queues greater than
\field{virtqueue_pairs} once it has placed the VIRTIO_NET_CTRL_MQ_VQ_PAIRS_SET command in the available ring.

\devicenormative{\subparagraph}{Automatic receive steering in multiqueue mode}{Device Types / Network Device / Device Operation / Control Virtqueue / Automatic receive steering in multiqueue mode}

After initialization of reset, the device MUST queue packets only on receiveq1.

The device MUST NOT queue packets on receive queues greater than
\field{virtqueue_pairs} once it has placed the
VIRTIO_NET_CTRL_MQ_VQ_PAIRS_SET command in a used buffer.

If the destination receive queue is being reset (See \ref{sec:Basic Facilities of a Virtio Device / Virtqueues / Virtqueue Reset}),
the device SHOULD re-select another random queue. If all receive queues are
being reset, the device MUST drop the packet.

\subparagraph{Legacy Interface: Automatic receive steering in multiqueue mode}\label{sec:Device Types / Network Device / Device Operation / Control Virtqueue / Automatic receive steering in multiqueue mode / Legacy Interface: Automatic receive steering in multiqueue mode}
When using the legacy interface, transitional devices and drivers
MUST format \field{virtqueue_pairs}
according to the native endian of the guest rather than
(necessarily when not using the legacy interface) little-endian.

\subparagraph{Hash calculation}\label{sec:Device Types / Network Device / Device Operation / Control Virtqueue / Automatic receive steering in multiqueue mode / Hash calculation}
If VIRTIO_NET_F_HASH_REPORT was negotiated and the device uses automatic receive steering,
the device MUST support a command to configure hash calculation parameters.

The driver provides parameters for hash calculation as follows:

\field{class} VIRTIO_NET_CTRL_MQ, \field{command} VIRTIO_NET_CTRL_MQ_HASH_CONFIG.

The \field{command-specific-data} has following format:
\begin{lstlisting}
struct virtio_net_hash_config {
    le32 hash_types;
    le16 reserved[4];
    u8 hash_key_length;
    u8 hash_key_data[hash_key_length];
};
\end{lstlisting}
Field \field{hash_types} contains a bitmask of allowed hash types as
defined in
\ref{sec:Device Types / Network Device / Device Operation / Processing of Incoming Packets / Hash calculation for incoming packets / Supported/enabled hash types}.
Initially the device has all hash types disabled and reports only VIRTIO_NET_HASH_REPORT_NONE.

Field \field{reserved} MUST contain zeroes. It is defined to make the structure to match the layout of virtio_net_rss_config structure,
defined in \ref{sec:Device Types / Network Device / Device Operation / Control Virtqueue / Receive-side scaling (RSS)}.

Fields \field{hash_key_length} and \field{hash_key_data} define the key to be used in hash calculation.

\paragraph{Receive-side scaling (RSS)}\label{sec:Device Types / Network Device / Device Operation / Control Virtqueue / Receive-side scaling (RSS)}
A device offers the feature VIRTIO_NET_F_RSS if it supports RSS receive steering with Toeplitz hash calculation and configurable parameters.

A driver queries RSS capabilities of the device by reading device configuration as defined in \ref{sec:Device Types / Network Device / Device configuration layout}

\subparagraph{Setting RSS parameters}\label{sec:Device Types / Network Device / Device Operation / Control Virtqueue / Receive-side scaling (RSS) / Setting RSS parameters}

Driver sends a VIRTIO_NET_CTRL_MQ_RSS_CONFIG command using the following format for \field{command-specific-data}:
\begin{lstlisting}
struct rss_rq_id {
   le16 vq_index_1_16: 15; /* Bits 1 to 16 of the virtqueue index */
   le16 reserved: 1; /* Set to zero */
};

struct virtio_net_rss_config {
    le32 hash_types;
    le16 indirection_table_mask;
    struct rss_rq_id unclassified_queue;
    struct rss_rq_id indirection_table[indirection_table_length];
    le16 max_tx_vq;
    u8 hash_key_length;
    u8 hash_key_data[hash_key_length];
};
\end{lstlisting}
Field \field{hash_types} contains a bitmask of allowed hash types as
defined in
\ref{sec:Device Types / Network Device / Device Operation / Processing of Incoming Packets / Hash calculation for incoming packets / Supported/enabled hash types}.

Field \field{indirection_table_mask} is a mask to be applied to
the calculated hash to produce an index in the
\field{indirection_table} array.
Number of entries in \field{indirection_table} is (\field{indirection_table_mask} + 1).

\field{rss_rq_id} is a receive virtqueue id. \field{vq_index_1_16}
consists of bits 1 to 16 of a virtqueue index. For example, a
\field{vq_index_1_16} value of 3 corresponds to virtqueue index 6,
which maps to receiveq4.

Field \field{unclassified_queue} specifies the receive virtqueue id in which to
place unclassified packets.

Field \field{indirection_table} is an array of receive virtqueues ids.

A driver sets \field{max_tx_vq} to inform a device how many transmit virtqueues it may use (transmitq1\ldots transmitq \field{max_tx_vq}).

Fields \field{hash_key_length} and \field{hash_key_data} define the key to be used in hash calculation.

\drivernormative{\subparagraph}{Setting RSS parameters}{Device Types / Network Device / Device Operation / Control Virtqueue / Receive-side scaling (RSS) }

A driver MUST NOT send the VIRTIO_NET_CTRL_MQ_RSS_CONFIG command if the feature VIRTIO_NET_F_RSS has not been negotiated.

A driver MUST fill the \field{indirection_table} array only with
enabled receive virtqueues ids.

The number of entries in \field{indirection_table} (\field{indirection_table_mask} + 1) MUST be a power of two.

A driver MUST use \field{indirection_table_mask} values that are less than \field{rss_max_indirection_table_length} reported by a device.

A driver MUST NOT set any VIRTIO_NET_HASH_TYPE_ flags that are not supported by a device.

\devicenormative{\subparagraph}{RSS processing}{Device Types / Network Device / Device Operation / Control Virtqueue / Receive-side scaling (RSS) / RSS processing}
The device MUST determine the destination queue for a network packet as follows:
\begin{itemize}
\item Calculate the hash of the packet as defined in \ref{sec:Device Types / Network Device / Device Operation / Processing of Incoming Packets / Hash calculation for incoming packets}.
\item If the device did not calculate the hash for the specific packet, the device directs the packet to the receiveq specified by \field{unclassified_queue} of virtio_net_rss_config structure.
\item Apply \field{indirection_table_mask} to the calculated hash
and use the result as the index in the indirection table to get
the destination receive virtqueue id.
\item If the destination receive queue is being reset (See \ref{sec:Basic Facilities of a Virtio Device / Virtqueues / Virtqueue Reset}), the device MUST drop the packet.
\end{itemize}

\paragraph{RSS Context}\label{sec:Device Types / Network Device / Device Operation / Control Virtqueue / RSS Context}

An RSS context consists of configurable parameters specified by \ref{sec:Device Types / Network Device
/ Device Operation / Control Virtqueue / Receive-side scaling (RSS)}.

The RSS configuration supported by VIRTIO_NET_F_RSS is considered the default RSS configuration.

The device offers the feature VIRTIO_NET_F_RSS_CONTEXT if it supports one or multiple RSS contexts
(excluding the default RSS configuration) and configurable parameters.

\subparagraph{Querying RSS Context Capability}\label{sec:Device Types / Network Device / Device Operation / Control Virtqueue / RSS Context / Querying RSS Context Capability}

\begin{lstlisting}
#define VIRTNET_RSS_CTX_CTRL 9
 #define VIRTNET_RSS_CTX_CTRL_CAP_GET  0
 #define VIRTNET_RSS_CTX_CTRL_ADD      1
 #define VIRTNET_RSS_CTX_CTRL_MOD      2
 #define VIRTNET_RSS_CTX_CTRL_DEL      3

struct virtnet_rss_ctx_cap {
    le16 max_rss_contexts;
}
\end{lstlisting}

Field \field{max_rss_contexts} specifies the maximum number of RSS contexts \ref{sec:Device Types / Network Device /
Device Operation / Control Virtqueue / RSS Context} supported by the device.

The driver queries the RSS context capability of the device by sending the command VIRTNET_RSS_CTX_CTRL_CAP_GET
with the structure virtnet_rss_ctx_cap.

For the command VIRTNET_RSS_CTX_CTRL_CAP_GET, the structure virtnet_rss_ctx_cap is write-only for the device.

\subparagraph{Setting RSS Context Parameters}\label{sec:Device Types / Network Device / Device Operation / Control Virtqueue / RSS Context / Setting RSS Context Parameters}

\begin{lstlisting}
struct virtnet_rss_ctx_add_modify {
    le16 rss_ctx_id;
    u8 reserved[6];
    struct virtio_net_rss_config rss;
};

struct virtnet_rss_ctx_del {
    le16 rss_ctx_id;
};
\end{lstlisting}

RSS context parameters:
\begin{itemize}
\item  \field{rss_ctx_id}: ID of the specific RSS context.
\item  \field{rss}: RSS context parameters of the specific RSS context whose id is \field{rss_ctx_id}.
\end{itemize}

\field{reserved} is reserved and it is ignored by the device.

If the feature VIRTIO_NET_F_RSS_CONTEXT has been negotiated, the driver can send the following
VIRTNET_RSS_CTX_CTRL class commands:
\begin{enumerate}
\item VIRTNET_RSS_CTX_CTRL_ADD: use the structure virtnet_rss_ctx_add_modify to
       add an RSS context configured as \field{rss} and id as \field{rss_ctx_id} for the device.
\item VIRTNET_RSS_CTX_CTRL_MOD: use the structure virtnet_rss_ctx_add_modify to
       configure parameters of the RSS context whose id is \field{rss_ctx_id} as \field{rss} for the device.
\item VIRTNET_RSS_CTX_CTRL_DEL: use the structure virtnet_rss_ctx_del to delete
       the RSS context whose id is \field{rss_ctx_id} for the device.
\end{enumerate}

For commands VIRTNET_RSS_CTX_CTRL_ADD and VIRTNET_RSS_CTX_CTRL_MOD, the structure virtnet_rss_ctx_add_modify is read-only for the device.
For the command VIRTNET_RSS_CTX_CTRL_DEL, the structure virtnet_rss_ctx_del is read-only for the device.

\devicenormative{\subparagraph}{RSS Context}{Device Types / Network Device / Device Operation / Control Virtqueue / RSS Context}

The device MUST set \field{max_rss_contexts} to at least 1 if it offers VIRTIO_NET_F_RSS_CONTEXT.

Upon reset, the device MUST clear all previously configured RSS contexts.

\drivernormative{\subparagraph}{RSS Context}{Device Types / Network Device / Device Operation / Control Virtqueue / RSS Context}

The driver MUST have negotiated the VIRTIO_NET_F_RSS_CONTEXT feature when issuing the VIRTNET_RSS_CTX_CTRL class commands.

The driver MUST set \field{rss_ctx_id} to between 1 and \field{max_rss_contexts} inclusive.

The driver MUST NOT send the command VIRTIO_NET_CTRL_MQ_VQ_PAIRS_SET when the device has successfully configured at least one RSS context.

\paragraph{Offloads State Configuration}\label{sec:Device Types / Network Device / Device Operation / Control Virtqueue / Offloads State Configuration}

If the VIRTIO_NET_F_CTRL_GUEST_OFFLOADS feature is negotiated, the driver can
send control commands for dynamic offloads state configuration.

\subparagraph{Setting Offloads State}\label{sec:Device Types / Network Device / Device Operation / Control Virtqueue / Offloads State Configuration / Setting Offloads State}

To configure the offloads, the following layout structure and
definitions are used:

\begin{lstlisting}
le64 offloads;

#define VIRTIO_NET_F_GUEST_CSUM       1
#define VIRTIO_NET_F_GUEST_TSO4       7
#define VIRTIO_NET_F_GUEST_TSO6       8
#define VIRTIO_NET_F_GUEST_ECN        9
#define VIRTIO_NET_F_GUEST_UFO        10
#define VIRTIO_NET_F_GUEST_UDP_TUNNEL_GSO  46
#define VIRTIO_NET_F_GUEST_UDP_TUNNEL_GSO_CSUM 47
#define VIRTIO_NET_F_GUEST_USO4       54
#define VIRTIO_NET_F_GUEST_USO6       55

#define VIRTIO_NET_CTRL_GUEST_OFFLOADS       5
 #define VIRTIO_NET_CTRL_GUEST_OFFLOADS_SET   0
\end{lstlisting}

The class VIRTIO_NET_CTRL_GUEST_OFFLOADS has one command:
VIRTIO_NET_CTRL_GUEST_OFFLOADS_SET applies the new offloads configuration.

le64 value passed as command data is a bitmask, bits set define
offloads to be enabled, bits cleared - offloads to be disabled.

There is a corresponding device feature for each offload. Upon feature
negotiation corresponding offload gets enabled to preserve backward
compatibility.

\drivernormative{\subparagraph}{Setting Offloads State}{Device Types / Network Device / Device Operation / Control Virtqueue / Offloads State Configuration / Setting Offloads State}

A driver MUST NOT enable an offload for which the appropriate feature
has not been negotiated.

\subparagraph{Legacy Interface: Setting Offloads State}\label{sec:Device Types / Network Device / Device Operation / Control Virtqueue / Offloads State Configuration / Setting Offloads State / Legacy Interface: Setting Offloads State}
When using the legacy interface, transitional devices and drivers
MUST format \field{offloads}
according to the native endian of the guest rather than
(necessarily when not using the legacy interface) little-endian.


\paragraph{Notifications Coalescing}\label{sec:Device Types / Network Device / Device Operation / Control Virtqueue / Notifications Coalescing}

If the VIRTIO_NET_F_NOTF_COAL feature is negotiated, the driver can
send commands VIRTIO_NET_CTRL_NOTF_COAL_TX_SET and VIRTIO_NET_CTRL_NOTF_COAL_RX_SET
for notification coalescing.

If the VIRTIO_NET_F_VQ_NOTF_COAL feature is negotiated, the driver can
send commands VIRTIO_NET_CTRL_NOTF_COAL_VQ_SET and VIRTIO_NET_CTRL_NOTF_COAL_VQ_GET
for virtqueue notification coalescing.

\begin{lstlisting}
struct virtio_net_ctrl_coal {
    le32 max_packets;
    le32 max_usecs;
};

struct virtio_net_ctrl_coal_vq {
    le16 vq_index;
    le16 reserved;
    struct virtio_net_ctrl_coal coal;
};

#define VIRTIO_NET_CTRL_NOTF_COAL 6
 #define VIRTIO_NET_CTRL_NOTF_COAL_TX_SET  0
 #define VIRTIO_NET_CTRL_NOTF_COAL_RX_SET 1
 #define VIRTIO_NET_CTRL_NOTF_COAL_VQ_SET 2
 #define VIRTIO_NET_CTRL_NOTF_COAL_VQ_GET 3
\end{lstlisting}

Coalescing parameters:
\begin{itemize}
\item \field{vq_index}: The virtqueue index of an enabled transmit or receive virtqueue.
\item \field{max_usecs} for RX: Maximum number of microseconds to delay a RX notification.
\item \field{max_usecs} for TX: Maximum number of microseconds to delay a TX notification.
\item \field{max_packets} for RX: Maximum number of packets to receive before a RX notification.
\item \field{max_packets} for TX: Maximum number of packets to send before a TX notification.
\end{itemize}

\field{reserved} is reserved and it is ignored by the device.

Read/Write attributes for coalescing parameters:
\begin{itemize}
\item For commands VIRTIO_NET_CTRL_NOTF_COAL_TX_SET and VIRTIO_NET_CTRL_NOTF_COAL_RX_SET, the structure virtio_net_ctrl_coal is write-only for the driver.
\item For the command VIRTIO_NET_CTRL_NOTF_COAL_VQ_SET, the structure virtio_net_ctrl_coal_vq is write-only for the driver.
\item For the command VIRTIO_NET_CTRL_NOTF_COAL_VQ_GET, \field{vq_index} and \field{reserved} are write-only
      for the driver, and the structure virtio_net_ctrl_coal is read-only for the driver.
\end{itemize}

The class VIRTIO_NET_CTRL_NOTF_COAL has the following commands:
\begin{enumerate}
\item VIRTIO_NET_CTRL_NOTF_COAL_TX_SET: use the structure virtio_net_ctrl_coal to set the \field{max_usecs} and \field{max_packets} parameters for all transmit virtqueues.
\item VIRTIO_NET_CTRL_NOTF_COAL_RX_SET: use the structure virtio_net_ctrl_coal to set the \field{max_usecs} and \field{max_packets} parameters for all receive virtqueues.
\item VIRTIO_NET_CTRL_NOTF_COAL_VQ_SET: use the structure virtio_net_ctrl_coal_vq to set the \field{max_usecs} and \field{max_packets} parameters
                                        for an enabled transmit/receive virtqueue whose index is \field{vq_index}.
\item VIRTIO_NET_CTRL_NOTF_COAL_VQ_GET: use the structure virtio_net_ctrl_coal_vq to get the \field{max_usecs} and \field{max_packets} parameters
                                        for an enabled transmit/receive virtqueue whose index is \field{vq_index}.
\end{enumerate}

The device may generate notifications more or less frequently than specified by set commands of the VIRTIO_NET_CTRL_NOTF_COAL class.

If coalescing parameters are being set, the device applies the last coalescing parameters set for a
virtqueue, regardless of the command used to set the parameters. Use the following command sequence
with two pairs of virtqueues as an example:
Each of the following commands sets \field{max_usecs} and \field{max_packets} parameters for virtqueues.
\begin{itemize}
\item Command1: VIRTIO_NET_CTRL_NOTF_COAL_RX_SET sets coalescing parameters for virtqueues having index 0 and index 2. Virtqueues having index 1 and index 3 retain their previous parameters.
\item Command2: VIRTIO_NET_CTRL_NOTF_COAL_VQ_SET with \field{vq_index} = 0 sets coalescing parameters for virtqueue having index 0. Virtqueue having index 2 retains the parameters from command1.
\item Command3: VIRTIO_NET_CTRL_NOTF_COAL_VQ_GET with \field{vq_index} = 0, the device responds with coalescing parameters of vq_index 0 set by command2.
\item Command4: VIRTIO_NET_CTRL_NOTF_COAL_VQ_SET with \field{vq_index} = 1 sets coalescing parameters for virtqueue having index 1. Virtqueue having index 3 retains its previous parameters.
\item Command5: VIRTIO_NET_CTRL_NOTF_COAL_TX_SET sets coalescing parameters for virtqueues having index 1 and index 3, and overrides the parameters set by command4.
\item Command6: VIRTIO_NET_CTRL_NOTF_COAL_VQ_GET with \field{vq_index} = 1, the device responds with coalescing parameters of index 1 set by command5.
\end{itemize}

\subparagraph{Operation}\label{sec:Device Types / Network Device / Device Operation / Control Virtqueue / Notifications Coalescing / Operation}

The device sends a used buffer notification once the notification conditions are met and if the notifications are not suppressed as explained in \ref{sec:Basic Facilities of a Virtio Device / Virtqueues / Used Buffer Notification Suppression}.

When the device has non-zero \field{max_usecs} and non-zero \field{max_packets}, it starts counting microseconds and packets upon receiving/sending a packet.
The device counts packets and microseconds for each receive virtqueue and transmit virtqueue separately.
In this case, the notification conditions are met when \field{max_usecs} microseconds elapse, or upon sending/receiving \field{max_packets} packets, whichever happens first.
Afterwards, the device waits for the next packet and starts counting packets and microseconds again.

When the device has \field{max_usecs} = 0 or \field{max_packets} = 0, the notification conditions are met after every packet received/sent.

\subparagraph{RX Example}\label{sec:Device Types / Network Device / Device Operation / Control Virtqueue / Notifications Coalescing / RX Example}

If, for example:
\begin{itemize}
\item \field{max_usecs} = 10.
\item \field{max_packets} = 15.
\end{itemize}
then each receive virtqueue of a device will operate as follows:
\begin{itemize}
\item The device will count packets received on each virtqueue until it accumulates 15, or until 10 microseconds elapsed since the first one was received.
\item If the notifications are not suppressed by the driver, the device will send an used buffer notification, otherwise, the device will not send an used buffer notification as long as the notifications are suppressed.
\end{itemize}

\subparagraph{TX Example}\label{sec:Device Types / Network Device / Device Operation / Control Virtqueue / Notifications Coalescing / TX Example}

If, for example:
\begin{itemize}
\item \field{max_usecs} = 10.
\item \field{max_packets} = 15.
\end{itemize}
then each transmit virtqueue of a device will operate as follows:
\begin{itemize}
\item The device will count packets sent on each virtqueue until it accumulates 15, or until 10 microseconds elapsed since the first one was sent.
\item If the notifications are not suppressed by the driver, the device will send an used buffer notification, otherwise, the device will not send an used buffer notification as long as the notifications are suppressed.
\end{itemize}

\subparagraph{Notifications When Coalescing Parameters Change}\label{sec:Device Types / Network Device / Device Operation / Control Virtqueue / Notifications Coalescing / Notifications When Coalescing Parameters Change}

When the coalescing parameters of a device change, the device needs to check if the new notification conditions are met and send a used buffer notification if so.

For example, \field{max_packets} = 15 for a device with a single transmit virtqueue: if the device sends 10 packets and afterwards receives a
VIRTIO_NET_CTRL_NOTF_COAL_TX_SET command with \field{max_packets} = 8, then the notification condition is immediately considered to be met;
the device needs to immediately send a used buffer notification, if the notifications are not suppressed by the driver.

\drivernormative{\subparagraph}{Notifications Coalescing}{Device Types / Network Device / Device Operation / Control Virtqueue / Notifications Coalescing}

The driver MUST set \field{vq_index} to the virtqueue index of an enabled transmit or receive virtqueue.

The driver MUST have negotiated the VIRTIO_NET_F_NOTF_COAL feature when issuing commands VIRTIO_NET_CTRL_NOTF_COAL_TX_SET and VIRTIO_NET_CTRL_NOTF_COAL_RX_SET.

The driver MUST have negotiated the VIRTIO_NET_F_VQ_NOTF_COAL feature when issuing commands VIRTIO_NET_CTRL_NOTF_COAL_VQ_SET and VIRTIO_NET_CTRL_NOTF_COAL_VQ_GET.

The driver MUST ignore the values of coalescing parameters received from the VIRTIO_NET_CTRL_NOTF_COAL_VQ_GET command if the device responds with VIRTIO_NET_ERR.

\devicenormative{\subparagraph}{Notifications Coalescing}{Device Types / Network Device / Device Operation / Control Virtqueue / Notifications Coalescing}

The device MUST ignore \field{reserved}.

The device SHOULD respond to VIRTIO_NET_CTRL_NOTF_COAL_TX_SET and VIRTIO_NET_CTRL_NOTF_COAL_RX_SET commands with VIRTIO_NET_ERR if it was not able to change the parameters.

The device MUST respond to the VIRTIO_NET_CTRL_NOTF_COAL_VQ_SET command with VIRTIO_NET_ERR if it was not able to change the parameters.

The device MUST respond to VIRTIO_NET_CTRL_NOTF_COAL_VQ_SET and VIRTIO_NET_CTRL_NOTF_COAL_VQ_GET commands with
VIRTIO_NET_ERR if the designated virtqueue is not an enabled transmit or receive virtqueue.

Upon disabling and re-enabling a transmit virtqueue, the device MUST set the coalescing parameters of the virtqueue
to those configured through the VIRTIO_NET_CTRL_NOTF_COAL_TX_SET command, or, if the driver did not set any TX coalescing parameters, to 0.

Upon disabling and re-enabling a receive virtqueue, the device MUST set the coalescing parameters of the virtqueue
to those configured through the VIRTIO_NET_CTRL_NOTF_COAL_RX_SET command, or, if the driver did not set any RX coalescing parameters, to 0.

The behavior of the device in response to set commands of the VIRTIO_NET_CTRL_NOTF_COAL class is best-effort:
the device MAY generate notifications more or less frequently than specified.

A device SHOULD NOT send used buffer notifications to the driver if the notifications are suppressed, even if the notification conditions are met.

Upon reset, a device MUST initialize all coalescing parameters to 0.

\paragraph{Device Statistics}\label{sec:Device Types / Network Device / Device Operation / Control Virtqueue / Device Statistics}

If the VIRTIO_NET_F_DEVICE_STATS feature is negotiated, the driver can obtain
device statistics from the device by using the following command.

Different types of virtqueues have different statistics. The statistics of the
receiveq are different from those of the transmitq.

The statistics of a certain type of virtqueue are also divided into multiple types
because different types require different features. This enables the expansion
of new statistics.

In one command, the driver can obtain the statistics of one or multiple virtqueues.
Additionally, the driver can obtain multiple type statistics of each virtqueue.

\subparagraph{Query Statistic Capabilities}\label{sec:Device Types / Network Device / Device Operation / Control Virtqueue / Device Statistics / Query Statistic Capabilities}

\begin{lstlisting}
#define VIRTIO_NET_CTRL_STATS         8
#define VIRTIO_NET_CTRL_STATS_QUERY   0
#define VIRTIO_NET_CTRL_STATS_GET     1

struct virtio_net_stats_capabilities {

#define VIRTIO_NET_STATS_TYPE_CVQ       (1 << 32)

#define VIRTIO_NET_STATS_TYPE_RX_BASIC  (1 << 0)
#define VIRTIO_NET_STATS_TYPE_RX_CSUM   (1 << 1)
#define VIRTIO_NET_STATS_TYPE_RX_GSO    (1 << 2)
#define VIRTIO_NET_STATS_TYPE_RX_SPEED  (1 << 3)

#define VIRTIO_NET_STATS_TYPE_TX_BASIC  (1 << 16)
#define VIRTIO_NET_STATS_TYPE_TX_CSUM   (1 << 17)
#define VIRTIO_NET_STATS_TYPE_TX_GSO    (1 << 18)
#define VIRTIO_NET_STATS_TYPE_TX_SPEED  (1 << 19)

    le64 supported_stats_types[1];
}
\end{lstlisting}

To obtain device statistic capability, use the VIRTIO_NET_CTRL_STATS_QUERY
command. When the command completes successfully, \field{command-specific-result}
is in the format of \field{struct virtio_net_stats_capabilities}.

\subparagraph{Get Statistics}\label{sec:Device Types / Network Device / Device Operation / Control Virtqueue / Device Statistics / Get Statistics}

\begin{lstlisting}
struct virtio_net_ctrl_queue_stats {
       struct {
           le16 vq_index;
           le16 reserved[3];
           le64 types_bitmap[1];
       } stats[];
};

struct virtio_net_stats_reply_hdr {
#define VIRTIO_NET_STATS_TYPE_REPLY_CVQ       32

#define VIRTIO_NET_STATS_TYPE_REPLY_RX_BASIC  0
#define VIRTIO_NET_STATS_TYPE_REPLY_RX_CSUM   1
#define VIRTIO_NET_STATS_TYPE_REPLY_RX_GSO    2
#define VIRTIO_NET_STATS_TYPE_REPLY_RX_SPEED  3

#define VIRTIO_NET_STATS_TYPE_REPLY_TX_BASIC  16
#define VIRTIO_NET_STATS_TYPE_REPLY_TX_CSUM   17
#define VIRTIO_NET_STATS_TYPE_REPLY_TX_GSO    18
#define VIRTIO_NET_STATS_TYPE_REPLY_TX_SPEED  19
    u8 type;
    u8 reserved;
    le16 vq_index;
    le16 reserved1;
    le16 size;
}
\end{lstlisting}

To obtain device statistics, use the VIRTIO_NET_CTRL_STATS_GET command with the
\field{command-specific-data} which is in the format of
\field{struct virtio_net_ctrl_queue_stats}. When the command completes
successfully, \field{command-specific-result} contains multiple statistic
results, each statistic result has the \field{struct virtio_net_stats_reply_hdr}
as the header.

The fields of the \field{struct virtio_net_ctrl_queue_stats}:
\begin{description}
    \item [vq_index]
        The index of the virtqueue to obtain the statistics.

    \item [types_bitmap]
        This is a bitmask of the types of statistics to be obtained. Therefore, a
        \field{stats} inside \field{struct virtio_net_ctrl_queue_stats} may
        indicate multiple statistic replies for the virtqueue.
\end{description}

The fields of the \field{struct virtio_net_stats_reply_hdr}:
\begin{description}
    \item [type]
        The type of the reply statistic.

    \item [vq_index]
        The virtqueue index of the reply statistic.

    \item [size]
        The number of bytes for the statistics entry including size of \field{struct virtio_net_stats_reply_hdr}.

\end{description}

\subparagraph{Controlq Statistics}\label{sec:Device Types / Network Device / Device Operation / Control Virtqueue / Device Statistics / Controlq Statistics}

The structure corresponding to the controlq statistics is
\field{struct virtio_net_stats_cvq}. The corresponding type is
VIRTIO_NET_STATS_TYPE_CVQ. This is for the controlq.

\begin{lstlisting}
struct virtio_net_stats_cvq {
    struct virtio_net_stats_reply_hdr hdr;

    le64 command_num;
    le64 ok_num;
};
\end{lstlisting}

\begin{description}
    \item [command_num]
        The number of commands received by the device including the current command.

    \item [ok_num]
        The number of commands completed successfully by the device including the current command.
\end{description}


\subparagraph{Receiveq Basic Statistics}\label{sec:Device Types / Network Device / Device Operation / Control Virtqueue / Device Statistics / Receiveq Basic Statistics}

The structure corresponding to the receiveq basic statistics is
\field{struct virtio_net_stats_rx_basic}. The corresponding type is
VIRTIO_NET_STATS_TYPE_RX_BASIC. This is for the receiveq.

Receiveq basic statistics do not require any feature. As long as the device supports
VIRTIO_NET_F_DEVICE_STATS, the following are the receiveq basic statistics.

\begin{lstlisting}
struct virtio_net_stats_rx_basic {
    struct virtio_net_stats_reply_hdr hdr;

    le64 rx_notifications;

    le64 rx_packets;
    le64 rx_bytes;

    le64 rx_interrupts;

    le64 rx_drops;
    le64 rx_drop_overruns;
};
\end{lstlisting}

The packets described below were all presented on the specified virtqueue.
\begin{description}
    \item [rx_notifications]
        The number of driver notifications received by the device for this
        receiveq.

    \item [rx_packets]
        This is the number of packets passed to the driver by the device.

    \item [rx_bytes]
        This is the bytes of packets passed to the driver by the device.

    \item [rx_interrupts]
        The number of interrupts generated by the device for this receiveq.

    \item [rx_drops]
        This is the number of packets dropped by the device. The count includes
        all types of packets dropped by the device.

    \item [rx_drop_overruns]
        This is the number of packets dropped by the device when no more
        descriptors were available.

\end{description}

\subparagraph{Transmitq Basic Statistics}\label{sec:Device Types / Network Device / Device Operation / Control Virtqueue / Device Statistics / Transmitq Basic Statistics}

The structure corresponding to the transmitq basic statistics is
\field{struct virtio_net_stats_tx_basic}. The corresponding type is
VIRTIO_NET_STATS_TYPE_TX_BASIC. This is for the transmitq.

Transmitq basic statistics do not require any feature. As long as the device supports
VIRTIO_NET_F_DEVICE_STATS, the following are the transmitq basic statistics.

\begin{lstlisting}
struct virtio_net_stats_tx_basic {
    struct virtio_net_stats_reply_hdr hdr;

    le64 tx_notifications;

    le64 tx_packets;
    le64 tx_bytes;

    le64 tx_interrupts;

    le64 tx_drops;
    le64 tx_drop_malformed;
};
\end{lstlisting}

The packets described below are all for a specific virtqueue.
\begin{description}
    \item [tx_notifications]
        The number of driver notifications received by the device for this
        transmitq.

    \item [tx_packets]
        This is the number of packets sent by the device (not the packets
        got from the driver).

    \item [tx_bytes]
        This is the number of bytes sent by the device for all the sent packets
        (not the bytes sent got from the driver).

    \item [tx_interrupts]
        The number of interrupts generated by the device for this transmitq.

    \item [tx_drops]
        The number of packets dropped by the device. The count includes all
        types of packets dropped by the device.

    \item [tx_drop_malformed]
        The number of packets dropped by the device, when the descriptors are
        malformed. For example, the buffer is too short.
\end{description}

\subparagraph{Receiveq CSUM Statistics}\label{sec:Device Types / Network Device / Device Operation / Control Virtqueue / Device Statistics / Receiveq CSUM Statistics}

The structure corresponding to the receiveq checksum statistics is
\field{struct virtio_net_stats_rx_csum}. The corresponding type is
VIRTIO_NET_STATS_TYPE_RX_CSUM. This is for the receiveq.

Only after the VIRTIO_NET_F_GUEST_CSUM is negotiated, the receiveq checksum
statistics can be obtained.

\begin{lstlisting}
struct virtio_net_stats_rx_csum {
    struct virtio_net_stats_reply_hdr hdr;

    le64 rx_csum_valid;
    le64 rx_needs_csum;
    le64 rx_csum_none;
    le64 rx_csum_bad;
};
\end{lstlisting}

The packets described below were all presented on the specified virtqueue.
\begin{description}
    \item [rx_csum_valid]
        The number of packets with VIRTIO_NET_HDR_F_DATA_VALID.

    \item [rx_needs_csum]
        The number of packets with VIRTIO_NET_HDR_F_NEEDS_CSUM.

    \item [rx_csum_none]
        The number of packets without hardware checksum. The packet here refers
        to the non-TCP/UDP packet that the device cannot recognize.

    \item [rx_csum_bad]
        The number of packets with checksum mismatch.

\end{description}

\subparagraph{Transmitq CSUM Statistics}\label{sec:Device Types / Network Device / Device Operation / Control Virtqueue / Device Statistics / Transmitq CSUM Statistics}

The structure corresponding to the transmitq checksum statistics is
\field{struct virtio_net_stats_tx_csum}. The corresponding type is
VIRTIO_NET_STATS_TYPE_TX_CSUM. This is for the transmitq.

Only after the VIRTIO_NET_F_CSUM is negotiated, the transmitq checksum
statistics can be obtained.

The following are the transmitq checksum statistics:

\begin{lstlisting}
struct virtio_net_stats_tx_csum {
    struct virtio_net_stats_reply_hdr hdr;

    le64 tx_csum_none;
    le64 tx_needs_csum;
};
\end{lstlisting}

The packets described below are all for a specific virtqueue.
\begin{description}
    \item [tx_csum_none]
        The number of packets which do not require hardware checksum.

    \item [tx_needs_csum]
        The number of packets which require checksum calculation by the device.

\end{description}

\subparagraph{Receiveq GSO Statistics}\label{sec:Device Types / Network Device / Device Operation / Control Virtqueue / Device Statistics / Receiveq GSO Statistics}

The structure corresponding to the receivq GSO statistics is
\field{struct virtio_net_stats_rx_gso}. The corresponding type is
VIRTIO_NET_STATS_TYPE_RX_GSO. This is for the receiveq.

If one or more of the VIRTIO_NET_F_GUEST_TSO4, VIRTIO_NET_F_GUEST_TSO6
have been negotiated, the receiveq GSO statistics can be obtained.

GSO packets refer to packets passed by the device to the driver where
\field{gso_type} is not VIRTIO_NET_HDR_GSO_NONE.

\begin{lstlisting}
struct virtio_net_stats_rx_gso {
    struct virtio_net_stats_reply_hdr hdr;

    le64 rx_gso_packets;
    le64 rx_gso_bytes;
    le64 rx_gso_packets_coalesced;
    le64 rx_gso_bytes_coalesced;
};
\end{lstlisting}

The packets described below were all presented on the specified virtqueue.
\begin{description}
    \item [rx_gso_packets]
        The number of the GSO packets received by the device.

    \item [rx_gso_bytes]
        The bytes of the GSO packets received by the device.
        This includes the header size of the GSO packet.

    \item [rx_gso_packets_coalesced]
        The number of the GSO packets coalesced by the device.

    \item [rx_gso_bytes_coalesced]
        The bytes of the GSO packets coalesced by the device.
        This includes the header size of the GSO packet.
\end{description}

\subparagraph{Transmitq GSO Statistics}\label{sec:Device Types / Network Device / Device Operation / Control Virtqueue / Device Statistics / Transmitq GSO Statistics}

The structure corresponding to the transmitq GSO statistics is
\field{struct virtio_net_stats_tx_gso}. The corresponding type is
VIRTIO_NET_STATS_TYPE_TX_GSO. This is for the transmitq.

If one or more of the VIRTIO_NET_F_HOST_TSO4, VIRTIO_NET_F_HOST_TSO6,
VIRTIO_NET_F_HOST_USO options have been negotiated, the transmitq GSO statistics
can be obtained.

GSO packets refer to packets passed by the driver to the device where
\field{gso_type} is not VIRTIO_NET_HDR_GSO_NONE.
See more \ref{sec:Device Types / Network Device / Device Operation / Packet
Transmission}.

\begin{lstlisting}
struct virtio_net_stats_tx_gso {
    struct virtio_net_stats_reply_hdr hdr;

    le64 tx_gso_packets;
    le64 tx_gso_bytes;
    le64 tx_gso_segments;
    le64 tx_gso_segments_bytes;
    le64 tx_gso_packets_noseg;
    le64 tx_gso_bytes_noseg;
};
\end{lstlisting}

The packets described below are all for a specific virtqueue.
\begin{description}
    \item [tx_gso_packets]
        The number of the GSO packets sent by the device.

    \item [tx_gso_bytes]
        The bytes of the GSO packets sent by the device.

    \item [tx_gso_segments]
        The number of segments prepared from GSO packets.

    \item [tx_gso_segments_bytes]
        The bytes of segments prepared from GSO packets.

    \item [tx_gso_packets_noseg]
        The number of the GSO packets without segmentation.

    \item [tx_gso_bytes_noseg]
        The bytes of the GSO packets without segmentation.

\end{description}

\subparagraph{Receiveq Speed Statistics}\label{sec:Device Types / Network Device / Device Operation / Control Virtqueue / Device Statistics / Receiveq Speed Statistics}

The structure corresponding to the receiveq speed statistics is
\field{struct virtio_net_stats_rx_speed}. The corresponding type is
VIRTIO_NET_STATS_TYPE_RX_SPEED. This is for the receiveq.

The device has the allowance for the speed. If VIRTIO_NET_F_SPEED_DUPLEX has
been negotiated, the driver can get this by \field{speed}. When the received
packets bitrate exceeds the \field{speed}, some packets may be dropped by the
device.

\begin{lstlisting}
struct virtio_net_stats_rx_speed {
    struct virtio_net_stats_reply_hdr hdr;

    le64 rx_packets_allowance_exceeded;
    le64 rx_bytes_allowance_exceeded;
};
\end{lstlisting}

The packets described below were all presented on the specified virtqueue.
\begin{description}
    \item [rx_packets_allowance_exceeded]
        The number of the packets dropped by the device due to the received
        packets bitrate exceeding the \field{speed}.

    \item [rx_bytes_allowance_exceeded]
        The bytes of the packets dropped by the device due to the received
        packets bitrate exceeding the \field{speed}.

\end{description}

\subparagraph{Transmitq Speed Statistics}\label{sec:Device Types / Network Device / Device Operation / Control Virtqueue / Device Statistics / Transmitq Speed Statistics}

The structure corresponding to the transmitq speed statistics is
\field{struct virtio_net_stats_tx_speed}. The corresponding type is
VIRTIO_NET_STATS_TYPE_TX_SPEED. This is for the transmitq.

The device has the allowance for the speed. If VIRTIO_NET_F_SPEED_DUPLEX has
been negotiated, the driver can get this by \field{speed}. When the transmit
packets bitrate exceeds the \field{speed}, some packets may be dropped by the
device.

\begin{lstlisting}
struct virtio_net_stats_tx_speed {
    struct virtio_net_stats_reply_hdr hdr;

    le64 tx_packets_allowance_exceeded;
    le64 tx_bytes_allowance_exceeded;
};
\end{lstlisting}

The packets described below were all presented on the specified virtqueue.
\begin{description}
    \item [tx_packets_allowance_exceeded]
        The number of the packets dropped by the device due to the transmit packets
        bitrate exceeding the \field{speed}.

    \item [tx_bytes_allowance_exceeded]
        The bytes of the packets dropped by the device due to the transmit packets
        bitrate exceeding the \field{speed}.

\end{description}

\devicenormative{\subparagraph}{Device Statistics}{Device Types / Network Device / Device Operation / Control Virtqueue / Device Statistics}

When the VIRTIO_NET_F_DEVICE_STATS feature is negotiated, the device MUST reply
to the command VIRTIO_NET_CTRL_STATS_QUERY with the
\field{struct virtio_net_stats_capabilities}. \field{supported_stats_types}
includes all the statistic types supported by the device.

If \field{struct virtio_net_ctrl_queue_stats} is incorrect (such as the
following), the device MUST set \field{ack} to VIRTIO_NET_ERR. Even if there is
only one error, the device MUST fail the entire command.
\begin{itemize}
    \item \field{vq_index} exceeds the queue range.
    \item \field{types_bitmap} contains unknown types.
    \item One or more of the bits present in \field{types_bitmap} is not valid
        for the specified virtqueue.
    \item The feature corresponding to the specified \field{types_bitmap} was
        not negotiated.
\end{itemize}

The device MUST set the actual size of the bytes occupied by the reply to the
\field{size} of the \field{hdr}. And the device MUST set the \field{type} and
the \field{vq_index} of the statistic header.

The \field{command-specific-result} buffer allocated by the driver may be
smaller or bigger than all the statistics specified by
\field{struct virtio_net_ctrl_queue_stats}. The device MUST fill up only upto
the valid bytes.

The statistics counter replied by the device MUST wrap around to zero by the
device on the overflow.

\drivernormative{\subparagraph}{Device Statistics}{Device Types / Network Device / Device Operation / Control Virtqueue / Device Statistics}

The types contained in the \field{types_bitmap} MUST be queried from the device
via command VIRTIO_NET_CTRL_STATS_QUERY.

\field{types_bitmap} in \field{struct virtio_net_ctrl_queue_stats} MUST be valid to the
vq specified by \field{vq_index}.

The \field{command-specific-result} buffer allocated by the driver MUST have
enough capacity to store all the statistics reply headers defined in
\field{struct virtio_net_ctrl_queue_stats}. If the
\field{command-specific-result} buffer is fully utilized by the device but some
replies are missed, it is possible that some statistics may exceed the capacity
of the driver's records. In such cases, the driver should allocate additional
space for the \field{command-specific-result} buffer.

\subsubsection{Flow filter}\label{sec:Device Types / Network Device / Device Operation / Flow filter}

A network device can support one or more flow filter rules. Each flow filter rule
is applied by matching a packet and then taking an action, such as directing the packet
to a specific receiveq or dropping the packet. An example of a match is
matching on specific source and destination IP addresses.

A flow filter rule is a device resource object that consists of a key,
a processing priority, and an action to either direct a packet to a
receive queue or drop the packet.

Each rule uses a classifier. The key is matched against the packet using
a classifier, defining which fields in the packet are matched.
A classifier resource object consists of one or more field selectors, each with
a type that specifies the header fields to be matched against, and a mask.
The mask can match whole fields or parts of a field in a header. Each
rule resource object depends on the classifier resource object.

When a packet is received, relevant fields are extracted
(in the same way) from both the packet and the key according to the
classifier. The resulting field contents are then compared -
if they are identical the rule action is taken, if they are not, the rule is ignored.

Multiple flow filter rules are part of a group. The rule resource object
depends on the group. Each rule within a
group has a rule priority, and each group also has a group priority. For a
packet, a group with the highest priority is selected first. Within a group,
rules are applied from highest to lowest priority, until one of the rules
matches the packet and an action is taken. If all the rules within a group
are ignored, the group with the next highest priority is selected, and so on.

The device and the driver indicates flow filter resource limits using the capability
\ref{par:Device Types / Network Device / Device Operation / Flow filter / Device and driver capabilities / VIRTIO-NET-FF-RESOURCE-CAP} specifying the limits on the number of flow filter rule,
group and classifier resource objects. The capability \ref{par:Device Types / Network Device / Device Operation / Flow filter / Device and driver capabilities / VIRTIO-NET-FF-SELECTOR-CAP} specifies which selectors the device supports.
The driver indicates the selectors it is using by setting the flow
filter selector capability, prior to adding any resource objects.

The capability \ref{par:Device Types / Network Device / Device Operation / Flow filter / Device and driver capabilities / VIRTIO-NET-FF-ACTION-CAP} specifies which actions the device supports.

The driver controls the flow filter rule, classifier and group resource objects using
administration commands described in
\ref{sec:Basic Facilities of a Virtio Device / Device groups / Group administration commands / Device resource objects}.

\paragraph{Packet processing order}\label{sec:sec:Device Types / Network Device / Device Operation / Flow filter / Packet processing order}

Note that flow filter rules are applied after MAC/VLAN filtering. Flow filter
rules take precedence over steering: if a flow filter rule results in an action,
the steering configuration does not apply. The steering configuration only applies
to packets for which no flow filter rule action was performed. For example,
incoming packets can be processed in the following order:

\begin{itemize}
\item apply steering configuration received using control virtqueue commands
      VIRTIO_NET_CTRL_RX, VIRTIO_NET_CTRL_MAC and VIRTIO_NET_CTRL_VLAN.
\item apply flow filter rules if any.
\item if no filter rule applied, apply steering configuration received using command
      VIRTIO_NET_CTRL_MQ_RSS_CONFIG or as per automatic receive steering.
\end{itemize}

Some incoming packet processing examples:
\begin{itemize}
\item If the packet is dropped by the flow filter rule, RSS
      steering is ignored for the packet.
\item If the packet is directed to a specific receiveq using flow filter rule,
      the RSS steering is ignored for the packet.
\item If a packet is dropped due to the VIRTIO_NET_CTRL_MAC configuration,
      both flow filter rules and the RSS steering are ignored for the packet.
\item If a packet does not match any flow filter rules,
      the RSS steering is used to select the receiveq for the packet (if enabled).
\item If there are two flow filter groups configured as group_A and group_B
      with respective group priorities as 4, and 5; flow filter rules of
      group_B are applied first having highest group priority, if there is a match,
      the flow filter rules of group_A are ignored; if there is no match for
      the flow filter rules in group_B, the flow filter rules of next level group_A are applied.
\end{itemize}

\paragraph{Device and driver capabilities}
\label{par:Device Types / Network Device / Device Operation / Flow filter / Device and driver capabilities}

\subparagraph{VIRTIO_NET_FF_RESOURCE_CAP}
\label{par:Device Types / Network Device / Device Operation / Flow filter / Device and driver capabilities / VIRTIO-NET-FF-RESOURCE-CAP}

The capability VIRTIO_NET_FF_RESOURCE_CAP indicates the flow filter resource limits.
\field{cap_specific_data} is in the format
\field{struct virtio_net_ff_cap_data}.

\begin{lstlisting}
struct virtio_net_ff_cap_data {
        le32 groups_limit;
        le32 selectors_limit;
        le32 rules_limit;
        le32 rules_per_group_limit;
        u8 last_rule_priority;
        u8 selectors_per_classifier_limit;
};
\end{lstlisting}

\field{groups_limit}, and \field{selectors_limit} represent the maximum
number of flow filter groups and selectors, respectively, that the driver can create.
 \field{rules_limit} is the maximum number of
flow fiilter rules that the driver can create across all the groups.
\field{rules_per_group_limit} is the maximum number of flow filter rules that the driver
can create for each flow filter group.

\field{last_rule_priority} is the highest priority that can be assigned to a
flow filter rule.

\field{selectors_per_classifier_limit} is the maximum number of selectors
that a classifier can have.

\subparagraph{VIRTIO_NET_FF_SELECTOR_CAP}
\label{par:Device Types / Network Device / Device Operation / Flow filter / Device and driver capabilities / VIRTIO-NET-FF-SELECTOR-CAP}

The capability VIRTIO_NET_FF_SELECTOR_CAP lists the supported selectors and the
supported packet header fields for each selector.
\field{cap_specific_data} is in the format \field{struct virtio_net_ff_cap_mask_data}.

\begin{lstlisting}[label={lst:Device Types / Network Device / Device Operation / Flow filter / Device and driver capabilities / VIRTIO-NET-FF-SELECTOR-CAP / virtio-net-ff-selector}]
struct virtio_net_ff_selector {
        u8 type;
        u8 flags;
        u8 reserved[2];
        u8 length;
        u8 reserved1[3];
        u8 mask[];
};

struct virtio_net_ff_cap_mask_data {
        u8 count;
        u8 reserved[7];
        struct virtio_net_ff_selector selectors[];
};

#define VIRTIO_NET_FF_MASK_F_PARTIAL_MASK (1 << 0)
\end{lstlisting}

\field{count} indicates number of valid entries in the \field{selectors} array.
\field{selectors[]} is an array of supported selectors. Within each array entry:
\field{type} specifies the type of the packet header, as defined in table
\ref{table:Device Types / Network Device / Device Operation / Flow filter / Device and driver capabilities / VIRTIO-NET-FF-SELECTOR-CAP / flow filter selector types}. \field{mask} specifies which fields of the
packet header can be matched in a flow filter rule.

Each \field{type} is also listed in table
\ref{table:Device Types / Network Device / Device Operation / Flow filter / Device and driver capabilities / VIRTIO-NET-FF-SELECTOR-CAP / flow filter selector types}. \field{mask} is a byte array
in network byte order. For example, when \field{type} is VIRTIO_NET_FF_MASK_TYPE_IPV6,
the \field{mask} is in the format \hyperref[intro:IPv6-Header-Format]{IPv6 Header Format}.

If partial masking is not set, then all bits in each field have to be either all 0s
to ignore this field or all 1s to match on this field. If partial masking is set,
then any combination of bits can bit set to match on these bits.
For example, when a selector \field{type} is VIRTIO_NET_FF_MASK_TYPE_ETH, if
\field{mask[0-12]} are zero and \field{mask[13-14]} are 0xff (all 1s), it
indicates that matching is only supported for \field{EtherType} of
\field{Ethernet MAC frame}, matching is not supported for
\field{Destination Address} and \field{Source Address}.

The entries in the array \field{selectors} are ordered by
\field{type}, with each \field{type} value only appearing once.

\field{length} is the length of a dynamic array \field{mask} in bytes.
\field{reserved} and \field{reserved1} are reserved and set to zero.

\begin{table}[H]
\caption{Flow filter selector types}
\label{table:Device Types / Network Device / Device Operation / Flow filter / Device and driver capabilities / VIRTIO-NET-FF-SELECTOR-CAP / flow filter selector types}
\begin{tabularx}{\textwidth}{ |l|X|X| }
\hline
Type & Name & Description \\
\hline \hline
0x0 & - & Reserved \\
\hline
0x1 & VIRTIO_NET_FF_MASK_TYPE_ETH & 14 bytes of frame header starting from destination address described in \hyperref[intro:IEEE 802.3-2022]{IEEE 802.3-2022} \\
\hline
0x2 & VIRTIO_NET_FF_MASK_TYPE_IPV4 & 20 bytes of \hyperref[intro:Internet-Header-Format]{IPv4: Internet Header Format} \\
\hline
0x3 & VIRTIO_NET_FF_MASK_TYPE_IPV6 & 40 bytes of \hyperref[intro:IPv6-Header-Format]{IPv6 Header Format} \\
\hline
0x4 & VIRTIO_NET_FF_MASK_TYPE_TCP & 20 bytes of \hyperref[intro:TCP-Header-Format]{TCP Header Format} \\
\hline
0x5 & VIRTIO_NET_FF_MASK_TYPE_UDP & 8 bytes of UDP header described in \hyperref[intro:UDP]{UDP} \\
\hline
0x6 - 0xFF & & Reserved for future \\
\hline
\end{tabularx}
\end{table}

When VIRTIO_NET_FF_MASK_F_PARTIAL_MASK (bit 0) is set, it indicates that
partial masking is supported for all the fields of the selector identified by \field{type}.

For the selector \field{type} VIRTIO_NET_FF_MASK_TYPE_IPV4, if a partial mask is unsupported,
then matching on an individual bit of \field{Flags} in the
\field{IPv4: Internet Header Format} is unsupported. \field{Flags} has to match as a whole
if it is supported.

For the selector \field{type} VIRTIO_NET_FF_MASK_TYPE_IPV4, \field{mask} includes fields
up to the \field{Destination Address}; that is, \field{Options} and
\field{Padding} are excluded.

For the selector \field{type} VIRTIO_NET_FF_MASK_TYPE_IPV6, the \field{Next Header} field
of the \field{mask} corresponds to the \field{Next Header} in the packet
when \field{IPv6 Extension Headers} are not present. When the packet includes
one or more \field{IPv6 Extension Headers}, the \field{Next Header} field of
the \field{mask} corresponds to the \field{Next Header} of the last
\field{IPv6 Extension Header} in the packet.

For the selector \field{type} VIRTIO_NET_FF_MASK_TYPE_TCP, \field{Control bits}
are treated as individual fields for matching; that is, matching individual
\field{Control bits} does not depend on the partial mask support.

\subparagraph{VIRTIO_NET_FF_ACTION_CAP}
\label{par:Device Types / Network Device / Device Operation / Flow filter / Device and driver capabilities / VIRTIO-NET-FF-ACTION-CAP}

The capability VIRTIO_NET_FF_ACTION_CAP lists the supported actions in a rule.
\field{cap_specific_data} is in the format \field{struct virtio_net_ff_cap_actions}.

\begin{lstlisting}
struct virtio_net_ff_actions {
        u8 count;
        u8 reserved[7];
        u8 actions[];
};
\end{lstlisting}

\field{actions} is an array listing all possible actions.
The entries in the array are ordered from the smallest to the largest,
with each supported value appearing exactly once. Each entry can have the
following values:

\begin{table}[H]
\caption{Flow filter rule actions}
\label{table:Device Types / Network Device / Device Operation / Flow filter / Device and driver capabilities / VIRTIO-NET-FF-ACTION-CAP / flow filter rule actions}
\begin{tabularx}{\textwidth}{ |l|X|X| }
\hline
Action & Name & Description \\
\hline \hline
0x0 & - & reserved \\
\hline
0x1 & VIRTIO_NET_FF_ACTION_DROP & Matching packet will be dropped by the device \\
\hline
0x2 & VIRTIO_NET_FF_ACTION_DIRECT_RX_VQ & Matching packet will be directed to a receive queue \\
\hline
0x3 - 0xFF & & Reserved for future \\
\hline
\end{tabularx}
\end{table}

\paragraph{Resource objects}
\label{par:Device Types / Network Device / Device Operation / Flow filter / Resource objects}

\subparagraph{VIRTIO_NET_RESOURCE_OBJ_FF_GROUP}\label{par:Device Types / Network Device / Device Operation / Flow filter / Resource objects / VIRTIO-NET-RESOURCE-OBJ-FF-GROUP}

A flow filter group contains between 0 and \field{rules_limit} rules, as specified by the
capability VIRTIO_NET_FF_RESOURCE_CAP. For the flow filter group object both
\field{resource_obj_specific_data} and
\field{resource_obj_specific_result} are in the format
\field{struct virtio_net_resource_obj_ff_group}.

\begin{lstlisting}
struct virtio_net_resource_obj_ff_group {
        le16 group_priority;
};
\end{lstlisting}

\field{group_priority} specifies the priority for the group. Each group has a
distinct priority. For each incoming packet, the device tries to apply rules
from groups from higher \field{group_priority} value to lower, until either a
rule matches the packet or all groups have been tried.

\subparagraph{VIRTIO_NET_RESOURCE_OBJ_FF_CLASSIFIER}\label{par:Device Types / Network Device / Device Operation / Flow filter / Resource objects / VIRTIO-NET-RESOURCE-OBJ-FF-CLASSIFIER}

A classifier is used to match a flow filter key against a packet. The
classifier defines the desired packet fields to match, and is represented by
the VIRTIO_NET_RESOURCE_OBJ_FF_CLASSIFIER device resource object.

For the flow filter classifier object both \field{resource_obj_specific_data} and
\field{resource_obj_specific_result} are in the format
\field{struct virtio_net_resource_obj_ff_classifier}.

\begin{lstlisting}
struct virtio_net_resource_obj_ff_classifier {
        u8 count;
        u8 reserved[7];
        struct virtio_net_ff_selector selectors[];
};
\end{lstlisting}

A classifier is an array of \field{selectors}. The number of selectors in the
array is indicated by \field{count}. The selector has a type that specifies
the header fields to be matched against, and a mask.
See \ref{lst:Device Types / Network Device / Device Operation / Flow filter / Device and driver capabilities / VIRTIO-NET-FF-SELECTOR-CAP / virtio-net-ff-selector}
for details about selectors.

The first selector is always VIRTIO_NET_FF_MASK_TYPE_ETH. When there are multiple
selectors, a second selector can be either VIRTIO_NET_FF_MASK_TYPE_IPV4
or VIRTIO_NET_FF_MASK_TYPE_IPV6. If the third selector exists, the third
selector can be either VIRTIO_NET_FF_MASK_TYPE_UDP or VIRTIO_NET_FF_MASK_TYPE_TCP.
For example, to match a Ethernet IPv6 UDP packet,
\field{selectors[0].type} is set to VIRTIO_NET_FF_MASK_TYPE_ETH, \field{selectors[1].type}
is set to VIRTIO_NET_FF_MASK_TYPE_IPV6 and \field{selectors[2].type} is
set to VIRTIO_NET_FF_MASK_TYPE_UDP; accordingly, \field{selectors[0].mask[0-13]} is
for Ethernet header fields, \field{selectors[1].mask[0-39]} is set for IPV6 header
and \field{selectors[2].mask[0-7]} is set for UDP header.

When there are multiple selectors, the type of the (N+1)\textsuperscript{th} selector
affects the mask of the (N)\textsuperscript{th} selector. If
\field{count} is 2 or more, all the mask bits within \field{selectors[0]}
corresponding to \field{EtherType} of an Ethernet header are set.

If \field{count} is more than 2:
\begin{itemize}
\item if \field{selector[1].type} is, VIRTIO_NET_FF_MASK_TYPE_IPV4, then, all the mask bits within
\field{selector[1]} for \field{Protocol} is set.
\item if \field{selector[1].type} is, VIRTIO_NET_FF_MASK_TYPE_IPV6, then, all the mask bits within
\field{selector[1]} for \field{Next Header} is set.
\end{itemize}

If for a given packet header field, a subset of bits of a field is to be matched,
and if the partial mask is supported, the flow filter
mask object can specify a mask which has fewer bits set than the packet header
field size. For example, a partial mask for the Ethernet header source mac
address can be of 1-bit for multicast detection instead of 48-bits.

\subparagraph{VIRTIO_NET_RESOURCE_OBJ_FF_RULE}\label{par:Device Types / Network Device / Device Operation / Flow filter / Resource objects / VIRTIO-NET-RESOURCE-OBJ-FF-RULE}

Each flow filter rule resource object comprises a key, a priority, and an action.
For the flow filter rule object,
\field{resource_obj_specific_data} and
\field{resource_obj_specific_result} are in the format
\field{struct virtio_net_resource_obj_ff_rule}.

\begin{lstlisting}
struct virtio_net_resource_obj_ff_rule {
        le32 group_id;
        le32 classifier_id;
        u8 rule_priority;
        u8 key_length; /* length of key in bytes */
        u8 action;
        u8 reserved;
        le16 vq_index;
        u8 reserved1[2];
        u8 keys[][];
};
\end{lstlisting}

\field{group_id} is the resource object ID of the flow filter group to which
this rule belongs. \field{classifier_id} is the resource object ID of the
classifier used to match a packet against the \field{key}.

\field{rule_priority} denotes the priority of the rule within the group
specified by the \field{group_id}.
Rules within the group are applied from the highest to the lowest priority
until a rule matches the packet and an
action is taken. Rules with the same priority can be applied in any order.

\field{reserved} and \field{reserved1} are reserved and set to 0.

\field{keys[][]} is an array of keys to match against packets, using
the classifier specified by \field{classifier_id}. Each entry (key) comprises
a byte array, and they are located one immediately after another.
The size (number of entries) of the array is exactly the same as that of
\field{selectors} in the classifier, or in other words, \field{count}
in the classifier.

\field{key_length} specifies the total length of \field{keys} in bytes.
In other words, it equals the sum total of \field{length} of all
selectors in \field{selectors} in the classifier specified by
\field{classifier_id}.

For example, if a classifier object's \field{selectors[0].type} is
VIRTIO_NET_FF_MASK_TYPE_ETH and \field{selectors[1].type} is
VIRTIO_NET_FF_MASK_TYPE_IPV6,
then selectors[0].length is 14 and selectors[1].length is 40.
Accordingly, the \field{key_length} is set to 54.
This setting indicates that the \field{key} array's length is 54 bytes
comprising a first byte array of 14 bytes for the
Ethernet MAC header in bytes 0-13, immediately followed by 40 bytes for the
IPv6 header in bytes 14-53.

When there are multiple selectors in the classifier object, the key bytes
for (N)\textsuperscript{th} selector are set so that
(N+1)\textsuperscript{th} selector can be matched.

If \field{count} is 2 or more, key bytes of \field{EtherType}
are set according to \hyperref[intro:IEEE 802 Ethertypes]{IEEE 802 Ethertypes}
for VIRTIO_NET_FF_MASK_TYPE_IPV4 or VIRTIO_NET_FF_MASK_TYPE_IPV6 respectively.

If \field{count} is more than 2, when \field{selector[1].type} is
VIRTIO_NET_FF_MASK_TYPE_IPV4 or VIRTIO_NET_FF_MASK_TYPE_IPV6, key
bytes of \field{Protocol} or \field{Next Header} is set as per
\field{Protocol Numbers} defined \hyperref[intro:IANA Protocol Numbers]{IANA Protocol Numbers}
respectively.

\field{action} is the action to take when a packet matches the
\field{key} using the \field{classifier_id}. Supported actions are described in
\ref{table:Device Types / Network Device / Device Operation / Flow filter / Device and driver capabilities / VIRTIO-NET-FF-ACTION-CAP / flow filter rule actions}.

\field{vq_index} specifies a receive virtqueue. When the \field{action} is set
to VIRTIO_NET_FF_ACTION_DIRECT_RX_VQ, and the packet matches the \field{key},
the matching packet is directed to this virtqueue.

Note that at most one action is ever taken for a given packet. If a rule is
applied and an action is taken, the action of other rules is not taken.

\devicenormative{\paragraph}{Flow filter}{Device Types / Network Device / Device Operation / Flow filter}

When the device supports flow filter operations,
\begin{itemize}
\item the device MUST set VIRTIO_NET_FF_RESOURCE_CAP, VIRTIO_NET_FF_SELECTOR_CAP
and VIRTIO_NET_FF_ACTION_CAP capability in the \field{supported_caps} in the
command VIRTIO_ADMIN_CMD_CAP_SUPPORT_QUERY.
\item the device MUST support the administration commands
VIRTIO_ADMIN_CMD_RESOURCE_OBJ_CREATE,
VIRTIO_ADMIN_CMD_RESOURCE_OBJ_MODIFY, VIRTIO_ADMIN_CMD_RESOURCE_OBJ_QUERY,
VIRTIO_ADMIN_CMD_RESOURCE_OBJ_DESTROY for the resource types
VIRTIO_NET_RESOURCE_OBJ_FF_GROUP, VIRTIO_NET_RESOURCE_OBJ_FF_CLASSIFIER and
VIRTIO_NET_RESOURCE_OBJ_FF_RULE.
\end{itemize}

When any of the VIRTIO_NET_FF_RESOURCE_CAP, VIRTIO_NET_FF_SELECTOR_CAP, or
VIRTIO_NET_FF_ACTION_CAP capability is disabled, the device SHOULD set
\field{status} to VIRTIO_ADMIN_STATUS_Q_INVALID_OPCODE for the commands
VIRTIO_ADMIN_CMD_RESOURCE_OBJ_CREATE,
VIRTIO_ADMIN_CMD_RESOURCE_OBJ_MODIFY, VIRTIO_ADMIN_CMD_RESOURCE_OBJ_QUERY,
and VIRTIO_ADMIN_CMD_RESOURCE_OBJ_DESTROY. These commands apply to the resource
\field{type} of VIRTIO_NET_RESOURCE_OBJ_FF_GROUP, VIRTIO_NET_RESOURCE_OBJ_FF_CLASSIFIER, and
VIRTIO_NET_RESOURCE_OBJ_FF_RULE.

The device SHOULD set \field{status} to VIRTIO_ADMIN_STATUS_EINVAL for the
command VIRTIO_ADMIN_CMD_RESOURCE_OBJ_CREATE when the resource \field{type}
is VIRTIO_NET_RESOURCE_OBJ_FF_GROUP, if a flow filter group already exists
with the supplied \field{group_priority}.

The device SHOULD set \field{status} to VIRTIO_ADMIN_STATUS_ENOSPC for the
command VIRTIO_ADMIN_CMD_RESOURCE_OBJ_CREATE when the resource \field{type}
is VIRTIO_NET_RESOURCE_OBJ_FF_GROUP, if the number of flow filter group
objects in the device exceeds the lower of the configured driver
capabilities \field{groups_limit} and \field{rules_per_group_limit}.

The device SHOULD set \field{status} to VIRTIO_ADMIN_STATUS_ENOSPC for the
command VIRTIO_ADMIN_CMD_RESOURCE_OBJ_CREATE when the resource \field{type} is
VIRTIO_NET_RESOURCE_OBJ_FF_CLASSIFIER, if the number of flow filter selector
objects in the device exceeds the configured driver capability
\field{selectors_limit}.

The device SHOULD set \field{status} to VIRTIO_ADMIN_STATUS_EBUSY for the
command VIRTIO_ADMIN_CMD_RESOURCE_OBJ_DESTROY for a flow filter group when
the flow filter group has one or more flow filter rules depending on it.

The device SHOULD set \field{status} to VIRTIO_ADMIN_STATUS_EBUSY for the
command VIRTIO_ADMIN_CMD_RESOURCE_OBJ_DESTROY for a flow filter classifier when
the flow filter classifier has one or more flow filter rules depending on it.

The device SHOULD fail the command VIRTIO_ADMIN_CMD_RESOURCE_OBJ_CREATE for the
flow filter rule resource object if,
\begin{itemize}
\item \field{vq_index} is not a valid receive virtqueue index for
the VIRTIO_NET_FF_ACTION_DIRECT_RX_VQ action,
\item \field{priority} is greater than or equal to
      \field{last_rule_priority},
\item \field{id} is greater than or equal to \field{rules_limit} or
      greater than or equal to \field{rules_per_group_limit}, whichever is lower,
\item the length of \field{keys} and the length of all the mask bytes of
      \field{selectors[].mask} as referred by \field{classifier_id} differs,
\item the supplied \field{action} is not supported in the capability VIRTIO_NET_FF_ACTION_CAP.
\end{itemize}

When the flow filter directs a packet to the virtqueue identified by
\field{vq_index} and if the receive virtqueue is reset, the device
MUST drop such packets.

Upon applying a flow filter rule to a packet, the device MUST STOP any further
application of rules and cease applying any other steering configurations.

For multiple flow filter groups, the device MUST apply the rules from
the group with the highest priority. If any rule from this group is applied,
the device MUST ignore the remaining groups. If none of the rules from the
highest priority group match, the device MUST apply the rules from
the group with the next highest priority, until either a rule matches or
all groups have been attempted.

The device MUST apply the rules within the group from the highest to the
lowest priority until a rule matches the packet, and the device MUST take
the action. If an action is taken, the device MUST not take any other
action for this packet.

The device MAY apply the rules with the same \field{rule_priority} in any
order within the group.

The device MUST process incoming packets in the following order:
\begin{itemize}
\item apply the steering configuration received using control virtqueue
      commands VIRTIO_NET_CTRL_RX, VIRTIO_NET_CTRL_MAC, and
      VIRTIO_NET_CTRL_VLAN.
\item apply flow filter rules if any.
\item if no filter rule is applied, apply the steering configuration
      received using the command VIRTIO_NET_CTRL_MQ_RSS_CONFIG
      or according to automatic receive steering.
\end{itemize}

When processing an incoming packet, if the packet is dropped at any stage, the device
MUST skip further processing.

When the device drops the packet due to the configuration done using the control
virtqueue commands VIRTIO_NET_CTRL_RX or VIRTIO_NET_CTRL_MAC or VIRTIO_NET_CTRL_VLAN,
the device MUST skip flow filter rules for this packet.

When the device performs flow filter match operations and if the operation
result did not have any match in all the groups, the receive packet processing
continues to next level, i.e. to apply configuration done using
VIRTIO_NET_CTRL_MQ_RSS_CONFIG command.

The device MUST support the creation of flow filter classifier objects
using the command VIRTIO_ADMIN_CMD_RESOURCE_OBJ_CREATE with \field{flags}
set to VIRTIO_NET_FF_MASK_F_PARTIAL_MASK;
this support is required even if all the bits of the masks are set for
a field in \field{selectors}, provided that partial masking is supported
for the selectors.

\drivernormative{\paragraph}{Flow filter}{Device Types / Network Device / Device Operation / Flow filter}

The driver MUST enable VIRTIO_NET_FF_RESOURCE_CAP, VIRTIO_NET_FF_SELECTOR_CAP,
and VIRTIO_NET_FF_ACTION_CAP capabilities to use flow filter.

The driver SHOULD NOT remove a flow filter group using the command
VIRTIO_ADMIN_CMD_RESOURCE_OBJ_DESTROY when one or more flow filter rules
depend on that group. The driver SHOULD only destroy the group after
all the associated rules have been destroyed.

The driver SHOULD NOT remove a flow filter classifier using the command
VIRTIO_ADMIN_CMD_RESOURCE_OBJ_DESTROY when one or more flow filter rules
depend on the classifier. The driver SHOULD only destroy the classifier
after all the associated rules have been destroyed.

The driver SHOULD NOT add multiple flow filter rules with the same
\field{rule_priority} within a flow filter group, as these rules MAY match
the same packet. The driver SHOULD assign different \field{rule_priority}
values to different flow filter rules if multiple rules may match a single
packet.

For the command VIRTIO_ADMIN_CMD_RESOURCE_OBJ_CREATE, when creating a resource
of \field{type} VIRTIO_NET_RESOURCE_OBJ_FF_CLASSIFIER, the driver MUST set:
\begin{itemize}
\item \field{selectors[0].type} to VIRTIO_NET_FF_MASK_TYPE_ETH.
\item \field{selectors[1].type} to VIRTIO_NET_FF_MASK_TYPE_IPV4 or
      VIRTIO_NET_FF_MASK_TYPE_IPV6 when \field{count} is more than 1,
\item \field{selectors[2].type} VIRTIO_NET_FF_MASK_TYPE_UDP or
      VIRTIO_NET_FF_MASK_TYPE_TCP when \field{count} is more than 2.
\end{itemize}

For the command VIRTIO_ADMIN_CMD_RESOURCE_OBJ_CREATE, when creating a resource
of \field{type} VIRTIO_NET_RESOURCE_OBJ_FF_CLASSIFIER, the driver MUST set:
\begin{itemize}
\item \field{selectors[0].mask} bytes to all 1s for the \field{EtherType}
       when \field{count} is 2 or more.
\item \field{selectors[1].mask} bytes to all 1s for \field{Protocol} or \field{Next Header}
       when \field{selector[1].type} is VIRTIO_NET_FF_MASK_TYPE_IPV4 or VIRTIO_NET_FF_MASK_TYPE_IPV6,
       and when \field{count} is more than 2.
\end{itemize}

For the command VIRTIO_ADMIN_CMD_RESOURCE_OBJ_CREATE, the resource \field{type}
VIRTIO_NET_RESOURCE_OBJ_FF_RULE, if the corresponding classifier object's
\field{count} is 2 or more, the driver MUST SET the \field{keys} bytes of
\field{EtherType} in accordance with
\hyperref[intro:IEEE 802 Ethertypes]{IEEE 802 Ethertypes}
for either VIRTIO_NET_FF_MASK_TYPE_IPV4 or VIRTIO_NET_FF_MASK_TYPE_IPV6.

For the command VIRTIO_ADMIN_CMD_RESOURCE_OBJ_CREATE, when creating a resource of
\field{type} VIRTIO_NET_RESOURCE_OBJ_FF_RULE, if the corresponding classifier
object's \field{count} is more than 2, and the \field{selector[1].type} is either
VIRTIO_NET_FF_MASK_TYPE_IPV4 or VIRTIO_NET_FF_MASK_TYPE_IPV6, the driver MUST
set the \field{keys} bytes for the \field{Protocol} or \field{Next Header}
according to \hyperref[intro:IANA Protocol Numbers]{IANA Protocol Numbers} respectively.

The driver SHOULD set all the bits for a field in the mask of a selector in both the
capability and the classifier object, unless the VIRTIO_NET_FF_MASK_F_PARTIAL_MASK
is enabled.

\subsubsection{Legacy Interface: Framing Requirements}\label{sec:Device
Types / Network Device / Legacy Interface: Framing Requirements}

When using legacy interfaces, transitional drivers which have not
negotiated VIRTIO_F_ANY_LAYOUT MUST use a single descriptor for the
\field{struct virtio_net_hdr} on both transmit and receive, with the
network data in the following descriptors.

Additionally, when using the control virtqueue (see \ref{sec:Device
Types / Network Device / Device Operation / Control Virtqueue})
, transitional drivers which have not
negotiated VIRTIO_F_ANY_LAYOUT MUST:
\begin{itemize}
\item for all commands, use a single 2-byte descriptor including the first two
fields: \field{class} and \field{command}
\item for all commands except VIRTIO_NET_CTRL_MAC_TABLE_SET
use a single descriptor including command-specific-data
with no padding.
\item for the VIRTIO_NET_CTRL_MAC_TABLE_SET command use exactly
two descriptors including command-specific-data with no padding:
the first of these descriptors MUST include the
virtio_net_ctrl_mac table structure for the unicast addresses with no padding,
the second of these descriptors MUST include the
virtio_net_ctrl_mac table structure for the multicast addresses
with no padding.
\item for all commands, use a single 1-byte descriptor for the
\field{ack} field
\end{itemize}

See \ref{sec:Basic
Facilities of a Virtio Device / Virtqueues / Message Framing}.

\section{Network Device}\label{sec:Device Types / Network Device}

The virtio network device is a virtual network interface controller.
It consists of a virtual Ethernet link which connects the device
to the Ethernet network. The device has transmit and receive
queues. The driver adds empty buffers to the receive virtqueue.
The device receives incoming packets from the link; the device
places these incoming packets in the receive virtqueue buffers.
The driver adds outgoing packets to the transmit virtqueue. The device
removes these packets from the transmit virtqueue and sends them to
the link. The device may have a control virtqueue. The driver
uses the control virtqueue to dynamically manipulate various
features of the initialized device.

\subsection{Device ID}\label{sec:Device Types / Network Device / Device ID}

 1

\subsection{Virtqueues}\label{sec:Device Types / Network Device / Virtqueues}

\begin{description}
\item[0] receiveq1
\item[1] transmitq1
\item[\ldots]
\item[2(N-1)] receiveqN
\item[2(N-1)+1] transmitqN
\item[2N] controlq
\end{description}

 N=1 if neither VIRTIO_NET_F_MQ nor VIRTIO_NET_F_RSS are negotiated, otherwise N is set by
 \field{max_virtqueue_pairs}.

controlq is optional; it only exists if VIRTIO_NET_F_CTRL_VQ is
negotiated.

\subsection{Feature bits}\label{sec:Device Types / Network Device / Feature bits}

\begin{description}
\item[VIRTIO_NET_F_CSUM (0)] Device handles packets with partial checksum offload.

\item[VIRTIO_NET_F_GUEST_CSUM (1)] Driver handles packets with partial checksum.

\item[VIRTIO_NET_F_CTRL_GUEST_OFFLOADS (2)] Control channel offloads
        reconfiguration support.

\item[VIRTIO_NET_F_MTU(3)] Device maximum MTU reporting is supported. If
    offered by the device, device advises driver about the value of
    its maximum MTU. If negotiated, the driver uses \field{mtu} as
    the maximum MTU value.

\item[VIRTIO_NET_F_MAC (5)] Device has given MAC address.

\item[VIRTIO_NET_F_GUEST_TSO4 (7)] Driver can receive TSOv4.

\item[VIRTIO_NET_F_GUEST_TSO6 (8)] Driver can receive TSOv6.

\item[VIRTIO_NET_F_GUEST_ECN (9)] Driver can receive TSO with ECN.

\item[VIRTIO_NET_F_GUEST_UFO (10)] Driver can receive UFO.

\item[VIRTIO_NET_F_HOST_TSO4 (11)] Device can receive TSOv4.

\item[VIRTIO_NET_F_HOST_TSO6 (12)] Device can receive TSOv6.

\item[VIRTIO_NET_F_HOST_ECN (13)] Device can receive TSO with ECN.

\item[VIRTIO_NET_F_HOST_UFO (14)] Device can receive UFO.

\item[VIRTIO_NET_F_MRG_RXBUF (15)] Driver can merge receive buffers.

\item[VIRTIO_NET_F_STATUS (16)] Configuration status field is
    available.

\item[VIRTIO_NET_F_CTRL_VQ (17)] Control channel is available.

\item[VIRTIO_NET_F_CTRL_RX (18)] Control channel RX mode support.

\item[VIRTIO_NET_F_CTRL_VLAN (19)] Control channel VLAN filtering.

\item[VIRTIO_NET_F_CTRL_RX_EXTRA (20)]	Control channel RX extra mode support.

\item[VIRTIO_NET_F_GUEST_ANNOUNCE(21)] Driver can send gratuitous
    packets.

\item[VIRTIO_NET_F_MQ(22)] Device supports multiqueue with automatic
    receive steering.

\item[VIRTIO_NET_F_CTRL_MAC_ADDR(23)] Set MAC address through control
    channel.

\item[VIRTIO_NET_F_DEVICE_STATS(50)] Device can provide device-level statistics
    to the driver through the control virtqueue.

\item[VIRTIO_NET_F_HASH_TUNNEL(51)] Device supports inner header hash for encapsulated packets.

\item[VIRTIO_NET_F_VQ_NOTF_COAL(52)] Device supports virtqueue notification coalescing.

\item[VIRTIO_NET_F_NOTF_COAL(53)] Device supports notifications coalescing.

\item[VIRTIO_NET_F_GUEST_USO4 (54)] Driver can receive USOv4 packets.

\item[VIRTIO_NET_F_GUEST_USO6 (55)] Driver can receive USOv6 packets.

\item[VIRTIO_NET_F_HOST_USO (56)] Device can receive USO packets. Unlike UFO
 (fragmenting the packet) the USO splits large UDP packet
 to several segments when each of these smaller packets has UDP header.

\item[VIRTIO_NET_F_HASH_REPORT(57)] Device can report per-packet hash
    value and a type of calculated hash.

\item[VIRTIO_NET_F_GUEST_HDRLEN(59)] Driver can provide the exact \field{hdr_len}
    value. Device benefits from knowing the exact header length.

\item[VIRTIO_NET_F_RSS(60)] Device supports RSS (receive-side scaling)
    with Toeplitz hash calculation and configurable hash
    parameters for receive steering.

\item[VIRTIO_NET_F_RSC_EXT(61)] Device can process duplicated ACKs
    and report number of coalesced segments and duplicated ACKs.

\item[VIRTIO_NET_F_STANDBY(62)] Device may act as a standby for a primary
    device with the same MAC address.

\item[VIRTIO_NET_F_SPEED_DUPLEX(63)] Device reports speed and duplex.

\item[VIRTIO_NET_F_RSS_CONTEXT(64)] Device supports multiple RSS contexts.

\item[VIRTIO_NET_F_GUEST_UDP_TUNNEL_GSO (65)] Driver can receive GSO packets
  carried by a UDP tunnel.

\item[VIRTIO_NET_F_GUEST_UDP_TUNNEL_GSO_CSUM (66)] Driver handles packets
  carried by a UDP tunnel with partial csum for the outer header.

\item[VIRTIO_NET_F_HOST_UDP_TUNNEL_GSO (67)] Device can receive GSO packets
  carried by a UDP tunnel.

\item[VIRTIO_NET_F_HOST_UDP_TUNNEL_GSO_CSUM (68)] Device handles packets
  carried by a UDP tunnel with partial csum for the outer header.
\end{description}

\subsubsection{Feature bit requirements}\label{sec:Device Types / Network Device / Feature bits / Feature bit requirements}

Some networking feature bits require other networking feature bits
(see \ref{drivernormative:Basic Facilities of a Virtio Device / Feature Bits}):

\begin{description}
\item[VIRTIO_NET_F_GUEST_TSO4] Requires VIRTIO_NET_F_GUEST_CSUM.
\item[VIRTIO_NET_F_GUEST_TSO6] Requires VIRTIO_NET_F_GUEST_CSUM.
\item[VIRTIO_NET_F_GUEST_ECN] Requires VIRTIO_NET_F_GUEST_TSO4 or VIRTIO_NET_F_GUEST_TSO6.
\item[VIRTIO_NET_F_GUEST_UFO] Requires VIRTIO_NET_F_GUEST_CSUM.
\item[VIRTIO_NET_F_GUEST_USO4] Requires VIRTIO_NET_F_GUEST_CSUM.
\item[VIRTIO_NET_F_GUEST_USO6] Requires VIRTIO_NET_F_GUEST_CSUM.
\item[VIRTIO_NET_F_GUEST_UDP_TUNNEL_GSO] Requires VIRTIO_NET_F_GUEST_TSO4, VIRTIO_NET_F_GUEST_TSO6,
   VIRTIO_NET_F_GUEST_USO4 and VIRTIO_NET_F_GUEST_USO6.
\item[VIRTIO_NET_F_GUEST_UDP_TUNNEL_GSO_CSUM] Requires VIRTIO_NET_F_GUEST_UDP_TUNNEL_GSO

\item[VIRTIO_NET_F_HOST_TSO4] Requires VIRTIO_NET_F_CSUM.
\item[VIRTIO_NET_F_HOST_TSO6] Requires VIRTIO_NET_F_CSUM.
\item[VIRTIO_NET_F_HOST_ECN] Requires VIRTIO_NET_F_HOST_TSO4 or VIRTIO_NET_F_HOST_TSO6.
\item[VIRTIO_NET_F_HOST_UFO] Requires VIRTIO_NET_F_CSUM.
\item[VIRTIO_NET_F_HOST_USO] Requires VIRTIO_NET_F_CSUM.
\item[VIRTIO_NET_F_HOST_UDP_TUNNEL_GSO] Requires VIRTIO_NET_F_HOST_TSO4, VIRTIO_NET_F_HOST_TSO6
   and VIRTIO_NET_F_HOST_USO.
\item[VIRTIO_NET_F_HOST_UDP_TUNNEL_GSO_CSUM] Requires VIRTIO_NET_F_HOST_UDP_TUNNEL_GSO

\item[VIRTIO_NET_F_CTRL_RX] Requires VIRTIO_NET_F_CTRL_VQ.
\item[VIRTIO_NET_F_CTRL_VLAN] Requires VIRTIO_NET_F_CTRL_VQ.
\item[VIRTIO_NET_F_GUEST_ANNOUNCE] Requires VIRTIO_NET_F_CTRL_VQ.
\item[VIRTIO_NET_F_MQ] Requires VIRTIO_NET_F_CTRL_VQ.
\item[VIRTIO_NET_F_CTRL_MAC_ADDR] Requires VIRTIO_NET_F_CTRL_VQ.
\item[VIRTIO_NET_F_NOTF_COAL] Requires VIRTIO_NET_F_CTRL_VQ.
\item[VIRTIO_NET_F_RSC_EXT] Requires VIRTIO_NET_F_HOST_TSO4 or VIRTIO_NET_F_HOST_TSO6.
\item[VIRTIO_NET_F_RSS] Requires VIRTIO_NET_F_CTRL_VQ.
\item[VIRTIO_NET_F_VQ_NOTF_COAL] Requires VIRTIO_NET_F_CTRL_VQ.
\item[VIRTIO_NET_F_HASH_TUNNEL] Requires VIRTIO_NET_F_CTRL_VQ along with VIRTIO_NET_F_RSS or VIRTIO_NET_F_HASH_REPORT.
\item[VIRTIO_NET_F_RSS_CONTEXT] Requires VIRTIO_NET_F_CTRL_VQ and VIRTIO_NET_F_RSS.
\end{description}

\begin{note}
The dependency between UDP_TUNNEL_GSO_CSUM and UDP_TUNNEL_GSO is intentionally
in the opposite direction with respect to the plain GSO features and the plain
checksum offload because UDP tunnel checksum offload gives very little gain
for non GSO packets and is quite complex to implement in H/W.
\end{note}

\subsubsection{Legacy Interface: Feature bits}\label{sec:Device Types / Network Device / Feature bits / Legacy Interface: Feature bits}
\begin{description}
\item[VIRTIO_NET_F_GSO (6)] Device handles packets with any GSO type. This was supposed to indicate segmentation offload support, but
upon further investigation it became clear that multiple bits were needed.
\item[VIRTIO_NET_F_GUEST_RSC4 (41)] Device coalesces TCPIP v4 packets. This was implemented by hypervisor patch for certification
purposes and current Windows driver depends on it. It will not function if virtio-net device reports this feature.
\item[VIRTIO_NET_F_GUEST_RSC6 (42)] Device coalesces TCPIP v6 packets. Similar to VIRTIO_NET_F_GUEST_RSC4.
\end{description}

\subsection{Device configuration layout}\label{sec:Device Types / Network Device / Device configuration layout}
\label{sec:Device Types / Block Device / Feature bits / Device configuration layout}

The network device has the following device configuration layout.
All of the device configuration fields are read-only for the driver.

\begin{lstlisting}
struct virtio_net_config {
        u8 mac[6];
        le16 status;
        le16 max_virtqueue_pairs;
        le16 mtu;
        le32 speed;
        u8 duplex;
        u8 rss_max_key_size;
        le16 rss_max_indirection_table_length;
        le32 supported_hash_types;
        le32 supported_tunnel_types;
};
\end{lstlisting}

The \field{mac} address field always exists (although it is only
valid if VIRTIO_NET_F_MAC is set).

The \field{status} only exists if VIRTIO_NET_F_STATUS is set.
Two bits are currently defined for the status field: VIRTIO_NET_S_LINK_UP
and VIRTIO_NET_S_ANNOUNCE.

\begin{lstlisting}
#define VIRTIO_NET_S_LINK_UP     1
#define VIRTIO_NET_S_ANNOUNCE    2
\end{lstlisting}

The following field, \field{max_virtqueue_pairs} only exists if
VIRTIO_NET_F_MQ or VIRTIO_NET_F_RSS is set. This field specifies the maximum number
of each of transmit and receive virtqueues (receiveq1\ldots receiveqN
and transmitq1\ldots transmitqN respectively) that can be configured once at least one of these features
is negotiated.

The following field, \field{mtu} only exists if VIRTIO_NET_F_MTU
is set. This field specifies the maximum MTU for the driver to
use.

The following two fields, \field{speed} and \field{duplex}, only
exist if VIRTIO_NET_F_SPEED_DUPLEX is set.

\field{speed} contains the device speed, in units of 1 MBit per
second, 0 to 0x7fffffff, or 0xffffffff for unknown speed.

\field{duplex} has the values of 0x01 for full duplex, 0x00 for
half duplex and 0xff for unknown duplex state.

Both \field{speed} and \field{duplex} can change, thus the driver
is expected to re-read these values after receiving a
configuration change notification.

The following field, \field{rss_max_key_size} only exists if VIRTIO_NET_F_RSS or VIRTIO_NET_F_HASH_REPORT is set.
It specifies the maximum supported length of RSS key in bytes.

The following field, \field{rss_max_indirection_table_length} only exists if VIRTIO_NET_F_RSS is set.
It specifies the maximum number of 16-bit entries in RSS indirection table.

The next field, \field{supported_hash_types} only exists if the device supports hash calculation,
i.e. if VIRTIO_NET_F_RSS or VIRTIO_NET_F_HASH_REPORT is set.

Field \field{supported_hash_types} contains the bitmask of supported hash types.
See \ref{sec:Device Types / Network Device / Device Operation / Processing of Incoming Packets / Hash calculation for incoming packets / Supported/enabled hash types} for details of supported hash types.

Field \field{supported_tunnel_types} only exists if the device supports inner header hash, i.e. if VIRTIO_NET_F_HASH_TUNNEL is set.

Field \field{supported_tunnel_types} contains the bitmask of encapsulation types supported by the device for inner header hash.
Encapsulation types are defined in \ref{sec:Device Types / Network Device / Device Operation / Processing of Incoming Packets /
Hash calculation for incoming packets / Encapsulation types supported/enabled for inner header hash}.

\devicenormative{\subsubsection}{Device configuration layout}{Device Types / Network Device / Device configuration layout}

The device MUST set \field{max_virtqueue_pairs} to between 1 and 0x8000 inclusive,
if it offers VIRTIO_NET_F_MQ.

The device MUST set \field{mtu} to between 68 and 65535 inclusive,
if it offers VIRTIO_NET_F_MTU.

The device SHOULD set \field{mtu} to at least 1280, if it offers
VIRTIO_NET_F_MTU.

The device MUST NOT modify \field{mtu} once it has been set.

The device MUST NOT pass received packets that exceed \field{mtu} (plus low
level ethernet header length) size with \field{gso_type} NONE or ECN
after VIRTIO_NET_F_MTU has been successfully negotiated.

The device MUST forward transmitted packets of up to \field{mtu} (plus low
level ethernet header length) size with \field{gso_type} NONE or ECN, and do
so without fragmentation, after VIRTIO_NET_F_MTU has been successfully
negotiated.

The device MUST set \field{rss_max_key_size} to at least 40, if it offers
VIRTIO_NET_F_RSS or VIRTIO_NET_F_HASH_REPORT.

The device MUST set \field{rss_max_indirection_table_length} to at least 128, if it offers
VIRTIO_NET_F_RSS.

If the driver negotiates the VIRTIO_NET_F_STANDBY feature, the device MAY act
as a standby device for a primary device with the same MAC address.

If VIRTIO_NET_F_SPEED_DUPLEX has been negotiated, \field{speed}
MUST contain the device speed, in units of 1 MBit per second, 0 to
0x7ffffffff, or 0xfffffffff for unknown.

If VIRTIO_NET_F_SPEED_DUPLEX has been negotiated, \field{duplex}
MUST have the values of 0x00 for full duplex, 0x01 for half
duplex, or 0xff for unknown.

If VIRTIO_NET_F_SPEED_DUPLEX and VIRTIO_NET_F_STATUS have both
been negotiated, the device SHOULD NOT change the \field{speed} and
\field{duplex} fields as long as VIRTIO_NET_S_LINK_UP is set in
the \field{status}.

The device SHOULD NOT offer VIRTIO_NET_F_HASH_REPORT if it
does not offer VIRTIO_NET_F_CTRL_VQ.

The device SHOULD NOT offer VIRTIO_NET_F_CTRL_RX_EXTRA if it
does not offer VIRTIO_NET_F_CTRL_VQ.

\drivernormative{\subsubsection}{Device configuration layout}{Device Types / Network Device / Device configuration layout}

The driver MUST NOT write to any of the device configuration fields.

A driver SHOULD negotiate VIRTIO_NET_F_MAC if the device offers it.
If the driver negotiates the VIRTIO_NET_F_MAC feature, the driver MUST set
the physical address of the NIC to \field{mac}.  Otherwise, it SHOULD
use a locally-administered MAC address (see \hyperref[intro:IEEE 802]{IEEE 802},
``9.2 48-bit universal LAN MAC addresses'').

If the driver does not negotiate the VIRTIO_NET_F_STATUS feature, it SHOULD
assume the link is active, otherwise it SHOULD read the link status from
the bottom bit of \field{status}.

A driver SHOULD negotiate VIRTIO_NET_F_MTU if the device offers it.

If the driver negotiates VIRTIO_NET_F_MTU, it MUST supply enough receive
buffers to receive at least one receive packet of size \field{mtu} (plus low
level ethernet header length) with \field{gso_type} NONE or ECN.

If the driver negotiates VIRTIO_NET_F_MTU, it MUST NOT transmit packets of
size exceeding the value of \field{mtu} (plus low level ethernet header length)
with \field{gso_type} NONE or ECN.

A driver SHOULD negotiate the VIRTIO_NET_F_STANDBY feature if the device offers it.

If VIRTIO_NET_F_SPEED_DUPLEX has been negotiated,
the driver MUST treat any value of \field{speed} above
0x7fffffff as well as any value of \field{duplex} not
matching 0x00 or 0x01 as an unknown value.

If VIRTIO_NET_F_SPEED_DUPLEX has been negotiated, the driver
SHOULD re-read \field{speed} and \field{duplex} after a
configuration change notification.

A driver SHOULD NOT negotiate VIRTIO_NET_F_HASH_REPORT if it
does not negotiate VIRTIO_NET_F_CTRL_VQ.

A driver SHOULD NOT negotiate VIRTIO_NET_F_CTRL_RX_EXTRA if it
does not negotiate VIRTIO_NET_F_CTRL_VQ.

\subsubsection{Legacy Interface: Device configuration layout}\label{sec:Device Types / Network Device / Device configuration layout / Legacy Interface: Device configuration layout}
\label{sec:Device Types / Block Device / Feature bits / Device configuration layout / Legacy Interface: Device configuration layout}
When using the legacy interface, transitional devices and drivers
MUST format \field{status} and
\field{max_virtqueue_pairs} in struct virtio_net_config
according to the native endian of the guest rather than
(necessarily when not using the legacy interface) little-endian.

When using the legacy interface, \field{mac} is driver-writable
which provided a way for drivers to update the MAC without
negotiating VIRTIO_NET_F_CTRL_MAC_ADDR.

\subsection{Device Initialization}\label{sec:Device Types / Network Device / Device Initialization}

A driver would perform a typical initialization routine like so:

\begin{enumerate}
\item Identify and initialize the receive and
  transmission virtqueues, up to N of each kind. If
  VIRTIO_NET_F_MQ feature bit is negotiated,
  N=\field{max_virtqueue_pairs}, otherwise identify N=1.

\item If the VIRTIO_NET_F_CTRL_VQ feature bit is negotiated,
  identify the control virtqueue.

\item Fill the receive queues with buffers: see \ref{sec:Device Types / Network Device / Device Operation / Setting Up Receive Buffers}.

\item Even with VIRTIO_NET_F_MQ, only receiveq1, transmitq1 and
  controlq are used by default.  The driver would send the
  VIRTIO_NET_CTRL_MQ_VQ_PAIRS_SET command specifying the
  number of the transmit and receive queues to use.

\item If the VIRTIO_NET_F_MAC feature bit is set, the configuration
  space \field{mac} entry indicates the ``physical'' address of the
  device, otherwise the driver would typically generate a random
  local MAC address.

\item If the VIRTIO_NET_F_STATUS feature bit is negotiated, the link
  status comes from the bottom bit of \field{status}.
  Otherwise, the driver assumes it's active.

\item A performant driver would indicate that it will generate checksumless
  packets by negotiating the VIRTIO_NET_F_CSUM feature.

\item If that feature is negotiated, a driver can use TCP segmentation or UDP
  segmentation/fragmentation offload by negotiating the VIRTIO_NET_F_HOST_TSO4 (IPv4
  TCP), VIRTIO_NET_F_HOST_TSO6 (IPv6 TCP), VIRTIO_NET_F_HOST_UFO
  (UDP fragmentation) and VIRTIO_NET_F_HOST_USO (UDP segmentation) features.

\item If the VIRTIO_NET_F_HOST_TSO6, VIRTIO_NET_F_HOST_TSO4 and VIRTIO_NET_F_HOST_USO
  segmentation features are negotiated, a driver can
  use TCP segmentation or UDP segmentation on top of UDP encapsulation
  offload, when the outer header does not require checksumming - e.g.
  the outer UDP checksum is zero - by negotiating the
  VIRTIO_NET_F_HOST_UDP_TUNNEL_GSO feature.
  GSO over UDP tunnels packets carry two sets of headers: the outer ones
  and the inner ones. The outer transport protocol is UDP, the inner
  could be either TCP or UDP. Only a single level of encapsulation
  offload is supported.

\item If VIRTIO_NET_F_HOST_UDP_TUNNEL_GSO is negotiated, a driver can
  additionally use TCP segmentation or UDP segmentation on top of UDP
  encapsulation with the outer header requiring checksum offload,
  negotiating the VIRTIO_NET_F_HOST_UDP_TUNNEL_GSO_CSUM feature.

\item The converse features are also available: a driver can save
  the virtual device some work by negotiating these features.\note{For example, a network packet transported between two guests on
the same system might not need checksumming at all, nor segmentation,
if both guests are amenable.}
   The VIRTIO_NET_F_GUEST_CSUM feature indicates that partially
  checksummed packets can be received, and if it can do that then
  the VIRTIO_NET_F_GUEST_TSO4, VIRTIO_NET_F_GUEST_TSO6,
  VIRTIO_NET_F_GUEST_UFO, VIRTIO_NET_F_GUEST_ECN, VIRTIO_NET_F_GUEST_USO4,
  VIRTIO_NET_F_GUEST_USO6 VIRTIO_NET_F_GUEST_UDP_TUNNEL_GSO and
  VIRTIO_NET_F_GUEST_UDP_TUNNEL_GSO_CSUM are the input equivalents of
  the features described above.
  See \ref{sec:Device Types / Network Device / Device Operation /
Setting Up Receive Buffers}~\nameref{sec:Device Types / Network
Device / Device Operation / Setting Up Receive Buffers} and
\ref{sec:Device Types / Network Device / Device Operation /
Processing of Incoming Packets}~\nameref{sec:Device Types /
Network Device / Device Operation / Processing of Incoming Packets} below.
\end{enumerate}

A truly minimal driver would only accept VIRTIO_NET_F_MAC and ignore
everything else.

\subsection{Device and driver capabilities}\label{sec:Device Types / Network Device / Device and driver capabilities}

The network device has the following capabilities.

\begin{tabularx}{\textwidth}{ |l||l|X| }
\hline
Identifier & Name & Description \\
\hline \hline
0x0800 & \hyperref[par:Device Types / Network Device / Device Operation / Flow filter / Device and driver capabilities / VIRTIO-NET-FF-RESOURCE-CAP]{VIRTIO_NET_FF_RESOURCE_CAP} & Flow filter resource capability \\
\hline
0x0801 & \hyperref[par:Device Types / Network Device / Device Operation / Flow filter / Device and driver capabilities / VIRTIO-NET-FF-SELECTOR-CAP]{VIRTIO_NET_FF_SELECTOR_CAP} & Flow filter classifier capability \\
\hline
0x0802 & \hyperref[par:Device Types / Network Device / Device Operation / Flow filter / Device and driver capabilities / VIRTIO-NET-FF-ACTION-CAP]{VIRTIO_NET_FF_ACTION_CAP} & Flow filter action capability \\
\hline
\end{tabularx}

\subsection{Device resource objects}\label{sec:Device Types / Network Device / Device resource objects}

The network device has the following resource objects.

\begin{tabularx}{\textwidth}{ |l||l|X| }
\hline
type & Name & Description \\
\hline \hline
0x0200 & \hyperref[par:Device Types / Network Device / Device Operation / Flow filter / Resource objects / VIRTIO-NET-RESOURCE-OBJ-FF-GROUP]{VIRTIO_NET_RESOURCE_OBJ_FF_GROUP} & Flow filter group resource object \\
\hline
0x0201 & \hyperref[par:Device Types / Network Device / Device Operation / Flow filter / Resource objects / VIRTIO-NET-RESOURCE-OBJ-FF-CLASSIFIER]{VIRTIO_NET_RESOURCE_OBJ_FF_CLASSIFIER} & Flow filter mask object \\
\hline
0x0202 & \hyperref[par:Device Types / Network Device / Device Operation / Flow filter / Resource objects / VIRTIO-NET-RESOURCE-OBJ-FF-RULE]{VIRTIO_NET_RESOURCE_OBJ_FF_RULE} & Flow filter rule object \\
\hline
\end{tabularx}

\subsection{Device parts}\label{sec:Device Types / Network Device / Device parts}

Network device parts represent the configuration done by the driver using control
virtqueue commands. Network device part is in the format of
\field{struct virtio_dev_part}.

\begin{tabularx}{\textwidth}{ |l||l|X| }
\hline
Type & Name & Description \\
\hline \hline
0x200 & VIRTIO_NET_DEV_PART_CVQ_CFG_PART & Represents device configuration done through a control virtqueue command, see \ref{sec:Device Types / Network Device / Device parts / VIRTIO-NET-DEV-PART-CVQ-CFG-PART} \\
\hline
0x201 - 0x5FF & - & reserved for future \\
\hline
\hline
\end{tabularx}

\subsubsection{VIRTIO_NET_DEV_PART_CVQ_CFG_PART}\label{sec:Device Types / Network Device / Device parts / VIRTIO-NET-DEV-PART-CVQ-CFG-PART}

For VIRTIO_NET_DEV_PART_CVQ_CFG_PART, \field{part_type} is set to 0x200. The
VIRTIO_NET_DEV_PART_CVQ_CFG_PART part indicates configuration performed by the
driver using a control virtqueue command.

\begin{lstlisting}
struct virtio_net_dev_part_cvq_selector {
        u8 class;
        u8 command;
        u8 reserved[6];
};
\end{lstlisting}

There is one device part of type VIRTIO_NET_DEV_PART_CVQ_CFG_PART for each
individual configuration. Each part is identified by a unique selector value.
The selector, \field{device_type_raw}, is in the format
\field{struct virtio_net_dev_part_cvq_selector}.

The selector consists of two fields: \field{class} and \field{command}. These
fields correspond to the \field{class} and \field{command} defined in
\field{struct virtio_net_ctrl}, as described in the relevant sections of
\ref{sec:Device Types / Network Device / Device Operation / Control Virtqueue}.

The value corresponding to each part’s selector follows the same format as the
respective \field{command-specific-data} described in the relevant sections of
\ref{sec:Device Types / Network Device / Device Operation / Control Virtqueue}.

For example, when the \field{class} is VIRTIO_NET_CTRL_MAC, the \field{command}
can be either VIRTIO_NET_CTRL_MAC_TABLE_SET or VIRTIO_NET_CTRL_MAC_ADDR_SET;
when \field{command} is set to VIRTIO_NET_CTRL_MAC_TABLE_SET, \field{value}
is in the format of \field{struct virtio_net_ctrl_mac}.

Supported selectors are listed in the table:

\begin{tabularx}{\textwidth}{ |l|X| }
\hline
Class selector & Command selector \\
\hline \hline
VIRTIO_NET_CTRL_RX & VIRTIO_NET_CTRL_RX_PROMISC \\
\hline
VIRTIO_NET_CTRL_RX & VIRTIO_NET_CTRL_RX_ALLMULTI \\
\hline
VIRTIO_NET_CTRL_RX & VIRTIO_NET_CTRL_RX_ALLUNI \\
\hline
VIRTIO_NET_CTRL_RX & VIRTIO_NET_CTRL_RX_NOMULTI \\
\hline
VIRTIO_NET_CTRL_RX & VIRTIO_NET_CTRL_RX_NOUNI \\
\hline
VIRTIO_NET_CTRL_RX & VIRTIO_NET_CTRL_RX_NOBCAST \\
\hline
VIRTIO_NET_CTRL_MAC & VIRTIO_NET_CTRL_MAC_TABLE_SET \\
\hline
VIRTIO_NET_CTRL_MAC & VIRTIO_NET_CTRL_MAC_ADDR_SET \\
\hline
VIRTIO_NET_CTRL_VLAN & VIRTIO_NET_CTRL_VLAN_ADD \\
\hline
VIRTIO_NET_CTRL_ANNOUNCE & VIRTIO_NET_CTRL_ANNOUNCE_ACK \\
\hline
VIRTIO_NET_CTRL_MQ & VIRTIO_NET_CTRL_MQ_VQ_PAIRS_SET \\
\hline
VIRTIO_NET_CTRL_MQ & VIRTIO_NET_CTRL_MQ_RSS_CONFIG \\
\hline
VIRTIO_NET_CTRL_MQ & VIRTIO_NET_CTRL_MQ_HASH_CONFIG \\
\hline
\hline
\end{tabularx}

For command selector VIRTIO_NET_CTRL_VLAN_ADD, device part consists of a whole
VLAN table.

\field{reserved} is reserved and set to zero.

\subsection{Device Operation}\label{sec:Device Types / Network Device / Device Operation}

Packets are transmitted by placing them in the
transmitq1\ldots transmitqN, and buffers for incoming packets are
placed in the receiveq1\ldots receiveqN. In each case, the packet
itself is preceded by a header:

\begin{lstlisting}
struct virtio_net_hdr {
#define VIRTIO_NET_HDR_F_NEEDS_CSUM    1
#define VIRTIO_NET_HDR_F_DATA_VALID    2
#define VIRTIO_NET_HDR_F_RSC_INFO      4
#define VIRTIO_NET_HDR_F_UDP_TUNNEL_CSUM 8
        u8 flags;
#define VIRTIO_NET_HDR_GSO_NONE        0
#define VIRTIO_NET_HDR_GSO_TCPV4       1
#define VIRTIO_NET_HDR_GSO_UDP         3
#define VIRTIO_NET_HDR_GSO_TCPV6       4
#define VIRTIO_NET_HDR_GSO_UDP_L4      5
#define VIRTIO_NET_HDR_GSO_UDP_TUNNEL_IPV4 0x20
#define VIRTIO_NET_HDR_GSO_UDP_TUNNEL_IPV6 0x40
#define VIRTIO_NET_HDR_GSO_ECN      0x80
        u8 gso_type;
        le16 hdr_len;
        le16 gso_size;
        le16 csum_start;
        le16 csum_offset;
        le16 num_buffers;
        le32 hash_value;        (Only if VIRTIO_NET_F_HASH_REPORT negotiated)
        le16 hash_report;       (Only if VIRTIO_NET_F_HASH_REPORT negotiated)
        le16 padding_reserved;  (Only if VIRTIO_NET_F_HASH_REPORT negotiated)
        le16 outer_th_offset    (Only if VIRTIO_NET_F_HOST_UDP_TUNNEL_GSO or VIRTIO_NET_F_GUEST_UDP_TUNNEL_GSO negotiated)
        le16 inner_nh_offset;   (Only if VIRTIO_NET_F_HOST_UDP_TUNNEL_GSO or VIRTIO_NET_F_GUEST_UDP_TUNNEL_GSO negotiated)
};
\end{lstlisting}

The controlq is used to control various device features described further in
section \ref{sec:Device Types / Network Device / Device Operation / Control Virtqueue}.

\subsubsection{Legacy Interface: Device Operation}\label{sec:Device Types / Network Device / Device Operation / Legacy Interface: Device Operation}
When using the legacy interface, transitional devices and drivers
MUST format the fields in \field{struct virtio_net_hdr}
according to the native endian of the guest rather than
(necessarily when not using the legacy interface) little-endian.

The legacy driver only presented \field{num_buffers} in the \field{struct virtio_net_hdr}
when VIRTIO_NET_F_MRG_RXBUF was negotiated; without that feature the
structure was 2 bytes shorter.

When using the legacy interface, the driver SHOULD ignore the
used length for the transmit queues
and the controlq queue.
\begin{note}
Historically, some devices put
the total descriptor length there, even though no data was
actually written.
\end{note}

\subsubsection{Packet Transmission}\label{sec:Device Types / Network Device / Device Operation / Packet Transmission}

Transmitting a single packet is simple, but varies depending on
the different features the driver negotiated.

\begin{enumerate}
\item The driver can send a completely checksummed packet.  In this case,
  \field{flags} will be zero, and \field{gso_type} will be VIRTIO_NET_HDR_GSO_NONE.

\item If the driver negotiated VIRTIO_NET_F_CSUM, it can skip
  checksumming the packet:
  \begin{itemize}
  \item \field{flags} has the VIRTIO_NET_HDR_F_NEEDS_CSUM set,

  \item \field{csum_start} is set to the offset within the packet to begin checksumming,
    and

  \item \field{csum_offset} indicates how many bytes after the csum_start the
    new (16 bit ones' complement) checksum is placed by the device.

  \item The TCP checksum field in the packet is set to the sum
    of the TCP pseudo header, so that replacing it by the ones'
    complement checksum of the TCP header and body will give the
    correct result.
  \end{itemize}

\begin{note}
For example, consider a partially checksummed TCP (IPv4) packet.
It will have a 14 byte ethernet header and 20 byte IP header
followed by the TCP header (with the TCP checksum field 16 bytes
into that header). \field{csum_start} will be 14+20 = 34 (the TCP
checksum includes the header), and \field{csum_offset} will be 16.
If the given packet has the VIRTIO_NET_HDR_GSO_UDP_TUNNEL_IPV4 bit or the
VIRTIO_NET_HDR_GSO_UDP_TUNNEL_IPV6 bit set,
the above checksum fields refer to the inner header checksum, see
the example below.
\end{note}

\item If the driver negotiated
  VIRTIO_NET_F_HOST_TSO4, TSO6, USO or UFO, and the packet requires
  TCP segmentation, UDP segmentation or fragmentation, then \field{gso_type}
  is set to VIRTIO_NET_HDR_GSO_TCPV4, TCPV6, UDP_L4 or UDP.
  (Otherwise, it is set to VIRTIO_NET_HDR_GSO_NONE). In this
  case, packets larger than 1514 bytes can be transmitted: the
  metadata indicates how to replicate the packet header to cut it
  into smaller packets. The other gso fields are set:

  \begin{itemize}
  \item If the VIRTIO_NET_F_GUEST_HDRLEN feature has been negotiated,
    \field{hdr_len} indicates the header length that needs to be replicated
    for each packet. It's the number of bytes from the beginning of the packet
    to the beginning of the transport payload.
    If the \field{gso_type} has the VIRTIO_NET_HDR_GSO_UDP_TUNNEL_IPV4 bit or
    VIRTIO_NET_HDR_GSO_UDP_TUNNEL_IPV6 bit set, \field{hdr_len} accounts for
    all the headers up to and including the inner transport.
    Otherwise, if the VIRTIO_NET_F_GUEST_HDRLEN feature has not been negotiated,
    \field{hdr_len} is a hint to the device as to how much of the header
    needs to be kept to copy into each packet, usually set to the
    length of the headers, including the transport header\footnote{Due to various bugs in implementations, this field is not useful
as a guarantee of the transport header size.
}.

  \begin{note}
  Some devices benefit from knowledge of the exact header length.
  \end{note}

  \item \field{gso_size} is the maximum size of each packet beyond that
    header (ie. MSS).

  \item If the driver negotiated the VIRTIO_NET_F_HOST_ECN feature,
    the VIRTIO_NET_HDR_GSO_ECN bit in \field{gso_type}
    indicates that the TCP packet has the ECN bit set\footnote{This case is not handled by some older hardware, so is called out
specifically in the protocol.}.
   \end{itemize}

\item If the driver negotiated the VIRTIO_NET_F_HOST_UDP_TUNNEL_GSO feature and the
  \field{gso_type} has the VIRTIO_NET_HDR_GSO_UDP_TUNNEL_IPV4 bit or
  VIRTIO_NET_HDR_GSO_UDP_TUNNEL_IPV6 bit set, the GSO protocol is encapsulated
  in a UDP tunnel.
  If the outer UDP header requires checksumming, the driver must have
  additionally negotiated the VIRTIO_NET_F_HOST_UDP_TUNNEL_GSO_CSUM feature
  and offloaded the outer checksum accordingly, otherwise
  the outer UDP header must not require checksum validation, i.e. the outer
  UDP checksum must be positive zero (0x0) as defined in UDP RFC 768.
  The other tunnel-related fields indicate how to replicate the packet
  headers to cut it into smaller packets:

  \begin{itemize}
  \item \field{outer_th_offset} field indicates the outer transport header within
      the packet. This field differs from \field{csum_start} as the latter
      points to the inner transport header within the packet.

  \item \field{inner_nh_offset} field indicates the inner network header within
      the packet.
  \end{itemize}

\begin{note}
For example, consider a partially checksummed TCP (IPv4) packet carried over a
Geneve UDP tunnel (again IPv4) with no tunnel options. The
only relevant variable related to the tunnel type is the tunnel header length.
The packet will have a 14 byte outer ethernet header, 20 byte outer IP header
followed by the 8 byte UDP header (with a 0 checksum value), 8 byte Geneve header,
14 byte inner ethernet header, 20 byte inner IP header
and the TCP header (with the TCP checksum field 16 bytes
into that header). \field{csum_start} will be 14+20+8+8+14+20 = 84 (the TCP
checksum includes the header), \field{csum_offset} will be 16.
\field{inner_nh_offset} will be 14+20+8+8+14 = 62, \field{outer_th_offset} will be
14+20+8 = 42 and \field{gso_type} will be
VIRTIO_NET_HDR_GSO_TCPV4 | VIRTIO_NET_HDR_GSO_UDP_TUNNEL_IPV4 = 0x21
\end{note}

\item If the driver negotiated the VIRTIO_NET_F_HOST_UDP_TUNNEL_GSO_CSUM feature,
  the transmitted packet is a GSO one encapsulated in a UDP tunnel, and
  the outer UDP header requires checksumming, the driver can skip checksumming
  the outer header:

  \begin{itemize}
  \item \field{flags} has the VIRTIO_NET_HDR_F_UDP_TUNNEL_CSUM set,

  \item The outer UDP checksum field in the packet is set to the sum
    of the UDP pseudo header, so that replacing it by the ones'
    complement checksum of the outer UDP header and payload will give the
    correct result.
  \end{itemize}

\item \field{num_buffers} is set to zero.  This field is unused on transmitted packets.

\item The header and packet are added as one output descriptor to the
  transmitq, and the device is notified of the new entry
  (see \ref{sec:Device Types / Network Device / Device Initialization}~\nameref{sec:Device Types / Network Device / Device Initialization}).
\end{enumerate}

\drivernormative{\paragraph}{Packet Transmission}{Device Types / Network Device / Device Operation / Packet Transmission}

For the transmit packet buffer, the driver MUST use the size of the
structure \field{struct virtio_net_hdr} same as the receive packet buffer.

The driver MUST set \field{num_buffers} to zero.

If VIRTIO_NET_F_CSUM is not negotiated, the driver MUST set
\field{flags} to zero and SHOULD supply a fully checksummed
packet to the device.

If VIRTIO_NET_F_HOST_TSO4 is negotiated, the driver MAY set
\field{gso_type} to VIRTIO_NET_HDR_GSO_TCPV4 to request TCPv4
segmentation, otherwise the driver MUST NOT set
\field{gso_type} to VIRTIO_NET_HDR_GSO_TCPV4.

If VIRTIO_NET_F_HOST_TSO6 is negotiated, the driver MAY set
\field{gso_type} to VIRTIO_NET_HDR_GSO_TCPV6 to request TCPv6
segmentation, otherwise the driver MUST NOT set
\field{gso_type} to VIRTIO_NET_HDR_GSO_TCPV6.

If VIRTIO_NET_F_HOST_UFO is negotiated, the driver MAY set
\field{gso_type} to VIRTIO_NET_HDR_GSO_UDP to request UDP
fragmentation, otherwise the driver MUST NOT set
\field{gso_type} to VIRTIO_NET_HDR_GSO_UDP.

If VIRTIO_NET_F_HOST_USO is negotiated, the driver MAY set
\field{gso_type} to VIRTIO_NET_HDR_GSO_UDP_L4 to request UDP
segmentation, otherwise the driver MUST NOT set
\field{gso_type} to VIRTIO_NET_HDR_GSO_UDP_L4.

The driver SHOULD NOT send to the device TCP packets requiring segmentation offload
which have the Explicit Congestion Notification bit set, unless the
VIRTIO_NET_F_HOST_ECN feature is negotiated, in which case the
driver MUST set the VIRTIO_NET_HDR_GSO_ECN bit in
\field{gso_type}.

If VIRTIO_NET_F_HOST_UDP_TUNNEL_GSO is negotiated, the driver MAY set
VIRTIO_NET_HDR_GSO_UDP_TUNNEL_IPV4 bit or the VIRTIO_NET_HDR_GSO_UDP_TUNNEL_IPV6 bit
in \field{gso_type} according to the inner network header protocol type
to request GSO packets over UDPv4 or UDPv6 tunnel segmentation,
otherwise the driver MUST NOT set either the
VIRTIO_NET_HDR_GSO_UDP_TUNNEL_IPV4 bit or the VIRTIO_NET_HDR_GSO_UDP_TUNNEL_IPV6 bit
in \field{gso_type}.

When requesting GSO segmentation over UDP tunnel, the driver MUST SET the
VIRTIO_NET_HDR_GSO_UDP_TUNNEL_IPV4 bit if the inner network header is IPv4, i.e. the
packet is a TCPv4 GSO one, otherwise, if the inner network header is IPv6, the driver
MUST SET the VIRTIO_NET_HDR_GSO_UDP_TUNNEL_IPV6 bit.

The driver MUST NOT send to the device GSO packets over UDP tunnel
requiring segmentation and outer UDP checksum offload, unless both the
VIRTIO_NET_F_HOST_UDP_TUNNEL_GSO and VIRTIO_NET_F_HOST_UDP_TUNNEL_GSO_CSUM features
are negotiated, in which case the driver MUST set either the
VIRTIO_NET_HDR_GSO_UDP_TUNNEL_IPV4 bit or the VIRTIO_NET_HDR_GSO_UDP_TUNNEL_IPV6
bit in the \field{gso_type} and the VIRTIO_NET_HDR_F_UDP_TUNNEL_CSUM bit in
the \field{flags}.

If VIRTIO_NET_F_HOST_UDP_TUNNEL_GSO_CSUM is not negotiated, the driver MUST not set
the VIRTIO_NET_HDR_F_UDP_TUNNEL_CSUM bit in the \field{flags} and
MUST NOT send to the device GSO packets over UDP tunnel
requiring segmentation and outer UDP checksum offload.

The driver MUST NOT set the VIRTIO_NET_HDR_GSO_UDP_TUNNEL_IPV4 bit or the
VIRTIO_NET_HDR_GSO_UDP_TUNNEL_IPV6 bit together with VIRTIO_NET_HDR_GSO_UDP, as the
latter is deprecated in favor of UDP_L4 and no new feature will support it.

The driver MUST NOT set the VIRTIO_NET_HDR_GSO_UDP_TUNNEL_IPV4 bit and the
VIRTIO_NET_HDR_GSO_UDP_TUNNEL_IPV6 bit together.

The driver MUST NOT set the VIRTIO_NET_HDR_F_UDP_TUNNEL_CSUM bit \field{flags}
without setting either the VIRTIO_NET_HDR_GSO_UDP_TUNNEL_IPV4 bit or
the VIRTIO_NET_HDR_GSO_UDP_TUNNEL_IPV6 bit in \field{gso_type}.

If the VIRTIO_NET_F_CSUM feature has been negotiated, the
driver MAY set the VIRTIO_NET_HDR_F_NEEDS_CSUM bit in
\field{flags}, if so:
\begin{enumerate}
\item the driver MUST validate the packet checksum at
	offset \field{csum_offset} from \field{csum_start} as well as all
	preceding offsets;
\begin{note}
If \field{gso_type} differs from VIRTIO_NET_HDR_GSO_NONE and the
VIRTIO_NET_HDR_GSO_UDP_TUNNEL_IPV4 bit or the VIRTIO_NET_HDR_GSO_UDP_TUNNEL_IPV6
bit are not set in \field{gso_type}, \field{csum_offset}
points to the only transport header present in the packet, and there are no
additional preceding checksums validated by VIRTIO_NET_HDR_F_NEEDS_CSUM.
\end{note}
\item the driver MUST set the packet checksum stored in the
	buffer to the TCP/UDP pseudo header;
\item the driver MUST set \field{csum_start} and
	\field{csum_offset} such that calculating a ones'
	complement checksum from \field{csum_start} up until the end of
	the packet and storing the result at offset \field{csum_offset}
	from  \field{csum_start} will result in a fully checksummed
	packet;
\end{enumerate}

If none of the VIRTIO_NET_F_HOST_TSO4, TSO6, USO or UFO options have
been negotiated, the driver MUST set \field{gso_type} to
VIRTIO_NET_HDR_GSO_NONE.

If \field{gso_type} differs from VIRTIO_NET_HDR_GSO_NONE, then
the driver MUST also set the VIRTIO_NET_HDR_F_NEEDS_CSUM bit in
\field{flags} and MUST set \field{gso_size} to indicate the
desired MSS.

If one of the VIRTIO_NET_F_HOST_TSO4, TSO6, USO or UFO options have
been negotiated:
\begin{itemize}
\item If the VIRTIO_NET_F_GUEST_HDRLEN feature has been negotiated,
	and \field{gso_type} differs from VIRTIO_NET_HDR_GSO_NONE,
	the driver MUST set \field{hdr_len} to a value equal to the length
	of the headers, including the transport header. If \field{gso_type}
	has the VIRTIO_NET_HDR_GSO_UDP_TUNNEL_IPV4 bit or the
	VIRTIO_NET_HDR_GSO_UDP_TUNNEL_IPV6 bit set, \field{hdr_len} includes
	the inner transport header.

\item If the VIRTIO_NET_F_GUEST_HDRLEN feature has not been negotiated,
	or \field{gso_type} is VIRTIO_NET_HDR_GSO_NONE,
	the driver SHOULD set \field{hdr_len} to a value
	not less than the length of the headers, including the transport
	header.
\end{itemize}

If the VIRTIO_NET_F_HOST_UDP_TUNNEL_GSO option has been negotiated, the
driver MAY set the VIRTIO_NET_HDR_GSO_UDP_TUNNEL_IPV4 bit or the
VIRTIO_NET_HDR_GSO_UDP_TUNNEL_IPV6 bit in \field{gso_type}, if so:
\begin{itemize}
\item the driver MUST set \field{outer_th_offset} to the outer UDP header
  offset and \field{inner_nh_offset} to the inner network header offset.
  The \field{csum_start} and \field{csum_offset} fields point respectively
  to the inner transport header and inner transport checksum field.
\end{itemize}

If the VIRTIO_NET_F_HOST_UDP_TUNNEL_GSO_CSUM feature has been negotiated,
and the VIRTIO_NET_HDR_GSO_UDP_TUNNEL_IPV4 bit or
VIRTIO_NET_HDR_GSO_UDP_TUNNEL_IPV6 bit in \field{gso_type} are set,
the driver MAY set the VIRTIO_NET_HDR_F_UDP_TUNNEL_CSUM bit in
\field{flags}, if so the driver MUST set the packet outer UDP header checksum
to the outer UDP pseudo header checksum.

\begin{note}
calculating a ones' complement checksum from \field{outer_th_offset}
up until the end of the packet and storing the result at offset 6
from \field{outer_th_offset} will result in a fully checksummed outer UDP packet;
\end{note}

If the VIRTIO_NET_HDR_GSO_UDP_TUNNEL_IPV4 bit or the
VIRTIO_NET_HDR_GSO_UDP_TUNNEL_IPV6 bit in \field{gso_type} are set
and the VIRTIO_NET_F_HOST_UDP_TUNNEL_GSO_CSUM feature has not
been negotiated, the
outer UDP header MUST NOT require checksum validation. That is, the
outer UDP checksum value MUST be 0 or the validated complete checksum
for such header.

\begin{note}
The valid complete checksum of the outer UDP header of individual segments
can be computed by the driver prior to segmentation only if the GSO packet
size is a multiple of \field{gso_size}, because then all segments
have the same size and thus all data included in the outer UDP
checksum is the same for every segment. These pre-computed segment
length and checksum fields are different from those of the GSO
packet.
In this scenario the outer UDP header of the GSO packet must carry the
segmented UDP packet length.
\end{note}

If the VIRTIO_NET_F_HOST_UDP_TUNNEL_GSO option has not
been negotiated, the driver MUST NOT set either the VIRTIO_NET_HDR_F_GSO_UDP_TUNNEL_IPV4
bit or the VIRTIO_NET_HDR_F_GSO_UDP_TUNNEL_IPV6 in \field{gso_type}.

If the VIRTIO_NET_F_HOST_UDP_TUNNEL_GSO_CSUM option has not been negotiated,
the driver MUST NOT set the VIRTIO_NET_HDR_F_UDP_TUNNEL_CSUM bit
in \field{flags}.

The driver SHOULD accept the VIRTIO_NET_F_GUEST_HDRLEN feature if it has
been offered, and if it's able to provide the exact header length.

The driver MUST NOT set the VIRTIO_NET_HDR_F_DATA_VALID and
VIRTIO_NET_HDR_F_RSC_INFO bits in \field{flags}.

The driver MUST NOT set the VIRTIO_NET_HDR_F_DATA_VALID bit in \field{flags}
together with the VIRTIO_NET_HDR_F_GSO_UDP_TUNNEL_IPV4 bit or the
VIRTIO_NET_HDR_F_GSO_UDP_TUNNEL_IPV6 bit in \field{gso_type}.

\devicenormative{\paragraph}{Packet Transmission}{Device Types / Network Device / Device Operation / Packet Transmission}
The device MUST ignore \field{flag} bits that it does not recognize.

If VIRTIO_NET_HDR_F_NEEDS_CSUM bit in \field{flags} is not set, the
device MUST NOT use the \field{csum_start} and \field{csum_offset}.

If one of the VIRTIO_NET_F_HOST_TSO4, TSO6, USO or UFO options have
been negotiated:
\begin{itemize}
\item If the VIRTIO_NET_F_GUEST_HDRLEN feature has been negotiated,
	and \field{gso_type} differs from VIRTIO_NET_HDR_GSO_NONE,
	the device MAY use \field{hdr_len} as the transport header size.

	\begin{note}
	Caution should be taken by the implementation so as to prevent
	a malicious driver from attacking the device by setting an incorrect hdr_len.
	\end{note}

\item If the VIRTIO_NET_F_GUEST_HDRLEN feature has not been negotiated,
	or \field{gso_type} is VIRTIO_NET_HDR_GSO_NONE,
	the device MAY use \field{hdr_len} only as a hint about the
	transport header size.
	The device MUST NOT rely on \field{hdr_len} to be correct.

	\begin{note}
	This is due to various bugs in implementations.
	\end{note}
\end{itemize}

If both the VIRTIO_NET_HDR_GSO_UDP_TUNNEL_IPV4 bit and
the VIRTIO_NET_HDR_GSO_UDP_TUNNEL_IPV6 bit in in \field{gso_type} are set,
the device MUST NOT accept the packet.

If the VIRTIO_NET_HDR_GSO_UDP_TUNNEL_IPV4 bit and the VIRTIO_NET_HDR_GSO_UDP_TUNNEL_IPV6
bit in \field{gso_type} are not set, the device MUST NOT use the
\field{outer_th_offset} and \field{inner_nh_offset}.

If either the VIRTIO_NET_HDR_GSO_UDP_TUNNEL_IPV4 bit or
the VIRTIO_NET_HDR_GSO_UDP_TUNNEL_IPV6 bit in \field{gso_type} are set, and any of
the following is true:
\begin{itemize}
\item the VIRTIO_NET_HDR_F_NEEDS_CSUM is not set in \field{flags}
\item the VIRTIO_NET_HDR_F_DATA_VALID is set in \field{flags}
\item the \field{gso_type} excluding the VIRTIO_NET_HDR_GSO_UDP_TUNNEL_IPV4
bit and the VIRTIO_NET_HDR_GSO_UDP_TUNNEL_IPV6 bit is VIRTIO_NET_HDR_GSO_NONE
\end{itemize}
the device MUST NOT accept the packet.

If the VIRTIO_NET_HDR_F_UDP_TUNNEL_CSUM bit in \field{flags} is set,
and both the bits VIRTIO_NET_HDR_GSO_UDP_TUNNEL_IPV4 and
VIRTIO_NET_HDR_GSO_UDP_TUNNEL_IPV6 in \field{gso_type} are not set,
the device MOST NOT accept the packet.

If VIRTIO_NET_HDR_F_NEEDS_CSUM is not set, the device MUST NOT
rely on the packet checksum being correct.
\paragraph{Packet Transmission Interrupt}\label{sec:Device Types / Network Device / Device Operation / Packet Transmission / Packet Transmission Interrupt}

Often a driver will suppress transmission virtqueue interrupts
and check for used packets in the transmit path of following
packets.

The normal behavior in this interrupt handler is to retrieve
used buffers from the virtqueue and free the corresponding
headers and packets.

\subsubsection{Setting Up Receive Buffers}\label{sec:Device Types / Network Device / Device Operation / Setting Up Receive Buffers}

It is generally a good idea to keep the receive virtqueue as
fully populated as possible: if it runs out, network performance
will suffer.

If the VIRTIO_NET_F_GUEST_TSO4, VIRTIO_NET_F_GUEST_TSO6,
VIRTIO_NET_F_GUEST_UFO, VIRTIO_NET_F_GUEST_USO4 or VIRTIO_NET_F_GUEST_USO6
features are used, the maximum incoming packet
will be 65589 bytes long (14 bytes of Ethernet header, plus 40 bytes of
the IPv6 header, plus 65535 bytes of maximum IPv6 payload including any
extension header), otherwise 1514 bytes.
When VIRTIO_NET_F_HASH_REPORT is not negotiated, the required receive buffer
size is either 65601 or 1526 bytes accounting for 20 bytes of
\field{struct virtio_net_hdr} followed by receive packet.
When VIRTIO_NET_F_HASH_REPORT is negotiated, the required receive buffer
size is either 65609 or 1534 bytes accounting for 12 bytes of
\field{struct virtio_net_hdr} followed by receive packet.

\drivernormative{\paragraph}{Setting Up Receive Buffers}{Device Types / Network Device / Device Operation / Setting Up Receive Buffers}

\begin{itemize}
\item If VIRTIO_NET_F_MRG_RXBUF is not negotiated:
  \begin{itemize}
    \item If VIRTIO_NET_F_GUEST_TSO4, VIRTIO_NET_F_GUEST_TSO6, VIRTIO_NET_F_GUEST_UFO,
	VIRTIO_NET_F_GUEST_USO4 or VIRTIO_NET_F_GUEST_USO6 are negotiated, the driver SHOULD populate
      the receive queue(s) with buffers of at least 65609 bytes if
      VIRTIO_NET_F_HASH_REPORT is negotiated, and of at least 65601 bytes if not.
    \item Otherwise, the driver SHOULD populate the receive queue(s)
      with buffers of at least 1534 bytes if VIRTIO_NET_F_HASH_REPORT
      is negotiated, and of at least 1526 bytes if not.
  \end{itemize}
\item If VIRTIO_NET_F_MRG_RXBUF is negotiated, each buffer MUST be at
least size of \field{struct virtio_net_hdr},
i.e. 20 bytes if VIRTIO_NET_F_HASH_REPORT is negotiated, and 12 bytes if not.
\end{itemize}

\begin{note}
Obviously each buffer can be split across multiple descriptor elements.
\end{note}

When calculating the size of \field{struct virtio_net_hdr}, the driver
MUST consider all the fields inclusive up to \field{padding_reserved},
i.e. 20 bytes if VIRTIO_NET_F_HASH_REPORT is negotiated, and 12 bytes if not.

If VIRTIO_NET_F_MQ is negotiated, each of receiveq1\ldots receiveqN
that will be used SHOULD be populated with receive buffers.

\devicenormative{\paragraph}{Setting Up Receive Buffers}{Device Types / Network Device / Device Operation / Setting Up Receive Buffers}

The device MUST set \field{num_buffers} to the number of descriptors used to
hold the incoming packet.

The device MUST use only a single descriptor if VIRTIO_NET_F_MRG_RXBUF
was not negotiated.
\begin{note}
{This means that \field{num_buffers} will always be 1
if VIRTIO_NET_F_MRG_RXBUF is not negotiated.}
\end{note}

\subsubsection{Processing of Incoming Packets}\label{sec:Device Types / Network Device / Device Operation / Processing of Incoming Packets}
\label{sec:Device Types / Network Device / Device Operation / Processing of Packets}%old label for latexdiff

When a packet is copied into a buffer in the receiveq, the
optimal path is to disable further used buffer notifications for the
receiveq and process packets until no more are found, then re-enable
them.

Processing incoming packets involves:

\begin{enumerate}
\item \field{num_buffers} indicates how many descriptors
  this packet is spread over (including this one): this will
  always be 1 if VIRTIO_NET_F_MRG_RXBUF was not negotiated.
  This allows receipt of large packets without having to allocate large
  buffers: a packet that does not fit in a single buffer can flow
  over to the next buffer, and so on. In this case, there will be
  at least \field{num_buffers} used buffers in the virtqueue, and the device
  chains them together to form a single packet in a way similar to
  how it would store it in a single buffer spread over multiple
  descriptors.
  The other buffers will not begin with a \field{struct virtio_net_hdr}.

\item If
  \field{num_buffers} is one, then the entire packet will be
  contained within this buffer, immediately following the struct
  virtio_net_hdr.
\item If the VIRTIO_NET_F_GUEST_CSUM feature was negotiated, the
  VIRTIO_NET_HDR_F_DATA_VALID bit in \field{flags} can be
  set: if so, device has validated the packet checksum.
  If the VIRTIO_NET_F_GUEST_UDP_TUNNEL_GSO_CSUM feature has been negotiated,
  and the VIRTIO_NET_HDR_F_UDP_TUNNEL_CSUM bit is set in \field{flags},
  both the outer UDP checksum and the inner transport checksum
  have been validated, otherwise only one level of checksums (the outer one
  in case of tunnels) has been validated.
\end{enumerate}

Additionally, VIRTIO_NET_F_GUEST_CSUM, TSO4, TSO6, UDP, UDP_TUNNEL
and ECN features enable receive checksum, large receive offload and ECN
support which are the input equivalents of the transmit checksum,
transmit segmentation offloading and ECN features, as described
in \ref{sec:Device Types / Network Device / Device Operation /
Packet Transmission}:
\begin{enumerate}
\item If the VIRTIO_NET_F_GUEST_TSO4, TSO6, UFO, USO4 or USO6 options were
  negotiated, then \field{gso_type} MAY be something other than
  VIRTIO_NET_HDR_GSO_NONE, and \field{gso_size} field indicates the
  desired MSS (see Packet Transmission point 2).
\item If the VIRTIO_NET_F_RSC_EXT option was negotiated (this
  implies one of VIRTIO_NET_F_GUEST_TSO4, TSO6), the
  device processes also duplicated ACK segments, reports
  number of coalesced TCP segments in \field{csum_start} field and
  number of duplicated ACK segments in \field{csum_offset} field
  and sets bit VIRTIO_NET_HDR_F_RSC_INFO in \field{flags}.
\item If the VIRTIO_NET_F_GUEST_CSUM feature was negotiated, the
  VIRTIO_NET_HDR_F_NEEDS_CSUM bit in \field{flags} can be
  set: if so, the packet checksum at offset \field{csum_offset}
  from \field{csum_start} and any preceding checksums
  have been validated.  The checksum on the packet is incomplete and
  if bit VIRTIO_NET_HDR_F_RSC_INFO is not set in \field{flags},
  then \field{csum_start} and \field{csum_offset} indicate how to calculate it
  (see Packet Transmission point 1).
\begin{note}
If \field{gso_type} differs from VIRTIO_NET_HDR_GSO_NONE and the
VIRTIO_NET_HDR_GSO_UDP_TUNNEL_IPV4 bit or the VIRTIO_NET_HDR_GSO_UDP_TUNNEL_IPV6
bit are not set, \field{csum_offset}
points to the only transport header present in the packet, and there are no
additional preceding checksums validated by VIRTIO_NET_HDR_F_NEEDS_CSUM.
\end{note}
\item If the VIRTIO_NET_F_GUEST_UDP_TUNNEL_GSO option was negotiated and
  \field{gso_type} is not VIRTIO_NET_HDR_GSO_NONE, the
  VIRTIO_NET_HDR_GSO_UDP_TUNNEL_IPV4 bit or the VIRTIO_NET_HDR_GSO_UDP_TUNNEL_IPV6
  bit MAY be set. In such case the \field{outer_th_offset} and
  \field{inner_nh_offset} fields indicate the corresponding
  headers information.
\item If the VIRTIO_NET_F_GUEST_UDP_TUNNEL_GSO_CSUM feature was
negotiated, and
  the VIRTIO_NET_HDR_GSO_UDP_TUNNEL_IPV4 bit or the VIRTIO_NET_HDR_GSO_UDP_TUNNEL_IPV6
  are set in \field{gso_type}, the VIRTIO_NET_HDR_F_UDP_TUNNEL_CSUM bit in the
  \field{flags} can be set: if so, the outer UDP checksum has been validated
  and the UDP header checksum at offset 6 from from \field{outer_th_offset}
  is set to the outer UDP pseudo header checksum.

\begin{note}
If the VIRTIO_NET_HDR_GSO_UDP_TUNNEL_IPV4 bit or VIRTIO_NET_HDR_GSO_UDP_TUNNEL_IPV6
bit are set in \field{gso_type}, the \field{csum_start} field refers to
the inner transport header offset (see Packet Transmission point 1).
If the VIRTIO_NET_HDR_F_UDP_TUNNEL_CSUM bit in \field{flags} is set both
the inner and the outer header checksums have been validated by
VIRTIO_NET_HDR_F_NEEDS_CSUM, otherwise only the inner transport header
checksum has been validated.
\end{note}
\end{enumerate}

If applicable, the device calculates per-packet hash for incoming packets as
defined in \ref{sec:Device Types / Network Device / Device Operation / Processing of Incoming Packets / Hash calculation for incoming packets}.

If applicable, the device reports hash information for incoming packets as
defined in \ref{sec:Device Types / Network Device / Device Operation / Processing of Incoming Packets / Hash reporting for incoming packets}.

\devicenormative{\paragraph}{Processing of Incoming Packets}{Device Types / Network Device / Device Operation / Processing of Incoming Packets}
\label{devicenormative:Device Types / Network Device / Device Operation / Processing of Packets}%old label for latexdiff

If VIRTIO_NET_F_MRG_RXBUF has not been negotiated, the device MUST set
\field{num_buffers} to 1.

If VIRTIO_NET_F_MRG_RXBUF has been negotiated, the device MUST set
\field{num_buffers} to indicate the number of buffers
the packet (including the header) is spread over.

If a receive packet is spread over multiple buffers, the device
MUST use all buffers but the last (i.e. the first \field{num_buffers} -
1 buffers) completely up to the full length of each buffer
supplied by the driver.

The device MUST use all buffers used by a single receive
packet together, such that at least \field{num_buffers} are
observed by driver as used.

If VIRTIO_NET_F_GUEST_CSUM is not negotiated, the device MUST set
\field{flags} to zero and SHOULD supply a fully checksummed
packet to the driver.

If VIRTIO_NET_F_GUEST_TSO4 is not negotiated, the device MUST NOT set
\field{gso_type} to VIRTIO_NET_HDR_GSO_TCPV4.

If VIRTIO_NET_F_GUEST_UDP is not negotiated, the device MUST NOT set
\field{gso_type} to VIRTIO_NET_HDR_GSO_UDP.

If VIRTIO_NET_F_GUEST_TSO6 is not negotiated, the device MUST NOT set
\field{gso_type} to VIRTIO_NET_HDR_GSO_TCPV6.

If none of VIRTIO_NET_F_GUEST_USO4 or VIRTIO_NET_F_GUEST_USO6 have been negotiated,
the device MUST NOT set \field{gso_type} to VIRTIO_NET_HDR_GSO_UDP_L4.

If VIRTIO_NET_F_GUEST_UDP_TUNNEL_GSO is not negotiated, the device MUST NOT set
either the VIRTIO_NET_HDR_GSO_UDP_TUNNEL_IPV4 bit or the
VIRTIO_NET_HDR_GSO_UDP_TUNNEL_IPV6 bit in \field{gso_type}.

If VIRTIO_NET_F_GUEST_UDP_TUNNEL_GSO_CSUM is not negotiated the device MUST NOT set
the VIRTIO_NET_HDR_F_UDP_TUNNEL_CSUM bit in \field{flags}.

The device SHOULD NOT send to the driver TCP packets requiring segmentation offload
which have the Explicit Congestion Notification bit set, unless the
VIRTIO_NET_F_GUEST_ECN feature is negotiated, in which case the
device MUST set the VIRTIO_NET_HDR_GSO_ECN bit in
\field{gso_type}.

If the VIRTIO_NET_F_GUEST_CSUM feature has been negotiated, the
device MAY set the VIRTIO_NET_HDR_F_NEEDS_CSUM bit in
\field{flags}, if so:
\begin{enumerate}
\item the device MUST validate the packet checksum at
	offset \field{csum_offset} from \field{csum_start} as well as all
	preceding offsets;
\item the device MUST set the packet checksum stored in the
	receive buffer to the TCP/UDP pseudo header;
\item the device MUST set \field{csum_start} and
	\field{csum_offset} such that calculating a ones'
	complement checksum from \field{csum_start} up until the
	end of the packet and storing the result at offset
	\field{csum_offset} from  \field{csum_start} will result in a
	fully checksummed packet;
\end{enumerate}

The device MUST NOT send to the driver GSO packets encapsulated in UDP
tunnel and requiring segmentation offload, unless the
VIRTIO_NET_F_GUEST_UDP_TUNNEL_GSO is negotiated, in which case the device MUST set
the VIRTIO_NET_HDR_GSO_UDP_TUNNEL_IPV4 bit or the VIRTIO_NET_HDR_GSO_UDP_TUNNEL_IPV6
bit in \field{gso_type} according to the inner network header protocol type,
MUST set the \field{outer_th_offset} and \field{inner_nh_offset} fields
to the corresponding header information, and the outer UDP header MUST NOT
require checksum offload.

If the VIRTIO_NET_F_GUEST_UDP_TUNNEL_GSO_CSUM feature has not been negotiated,
the device MUST NOT send the driver GSO packets encapsulated in UDP
tunnel and requiring segmentation and outer checksum offload.

If none of the VIRTIO_NET_F_GUEST_TSO4, TSO6, UFO, USO4 or USO6 options have
been negotiated, the device MUST set \field{gso_type} to
VIRTIO_NET_HDR_GSO_NONE.

If \field{gso_type} differs from VIRTIO_NET_HDR_GSO_NONE, then
the device MUST also set the VIRTIO_NET_HDR_F_NEEDS_CSUM bit in
\field{flags} MUST set \field{gso_size} to indicate the desired MSS.
If VIRTIO_NET_F_RSC_EXT was negotiated, the device MUST also
set VIRTIO_NET_HDR_F_RSC_INFO bit in \field{flags},
set \field{csum_start} to number of coalesced TCP segments and
set \field{csum_offset} to number of received duplicated ACK segments.

If VIRTIO_NET_F_RSC_EXT was not negotiated, the device MUST
not set VIRTIO_NET_HDR_F_RSC_INFO bit in \field{flags}.

If one of the VIRTIO_NET_F_GUEST_TSO4, TSO6, UFO, USO4 or USO6 options have
been negotiated, the device SHOULD set \field{hdr_len} to a value
not less than the length of the headers, including the transport
header. If \field{gso_type} has the VIRTIO_NET_HDR_GSO_UDP_TUNNEL_IPV4 bit
or the VIRTIO_NET_HDR_GSO_UDP_TUNNEL_IPV6 bit set, the referenced transport
header is the inner one.

If the VIRTIO_NET_F_GUEST_CSUM feature has been negotiated, the
device MAY set the VIRTIO_NET_HDR_F_DATA_VALID bit in
\field{flags}, if so, the device MUST validate the packet
checksum. If the VIRTIO_NET_F_GUEST_UDP_TUNNEL_GSO_CSUM feature has
been negotiated, and the VIRTIO_NET_HDR_F_UDP_TUNNEL_CSUM bit set in
\field{flags}, both the outer UDP checksum and the inner transport
checksum have been validated.
Otherwise level of checksum is validated: in case of multiple
encapsulated protocols the outermost one.

If either the VIRTIO_NET_HDR_GSO_UDP_TUNNEL_IPV4 bit or the
VIRTIO_NET_HDR_GSO_UDP_TUNNEL_IPV6 bit in \field{gso_type} are set,
the device MUST NOT set the VIRTIO_NET_HDR_F_DATA_VALID bit in
\field{flags}.

If the VIRTIO_NET_F_GUEST_UDP_TUNNEL_GSO_CSUM feature has been negotiated
and either the VIRTIO_NET_HDR_GSO_UDP_TUNNEL_IPV4 bit is set or the
VIRTIO_NET_HDR_GSO_UDP_TUNNEL_IPV6 bit is set in \field{gso_type}, the
device MAY set the VIRTIO_NET_HDR_F_UDP_TUNNEL_CSUM bit in
\field{flags}, if so the device MUST set the packet outer UDP checksum
stored in the receive buffer to the outer UDP pseudo header.

Otherwise, the VIRTIO_NET_F_GUEST_UDP_TUNNEL_GSO_CSUM feature has been
negotiated, either the VIRTIO_NET_HDR_GSO_UDP_TUNNEL_IPV4 bit is set or the
VIRTIO_NET_HDR_GSO_UDP_TUNNEL_IPV6 bit is set in \field{gso_type},
and the bit VIRTIO_NET_HDR_F_UDP_TUNNEL_CSUM is not set in
\field{flags}, the device MUST either provide a zero outer UDP header
checksum or a fully checksummed outer UDP header.

\drivernormative{\paragraph}{Processing of Incoming
Packets}{Device Types / Network Device / Device Operation /
Processing of Incoming Packets}

The driver MUST ignore \field{flag} bits that it does not recognize.

If VIRTIO_NET_HDR_F_NEEDS_CSUM bit in \field{flags} is not set or
if VIRTIO_NET_HDR_F_RSC_INFO bit \field{flags} is set, the
driver MUST NOT use the \field{csum_start} and \field{csum_offset}.

If one of the VIRTIO_NET_F_GUEST_TSO4, TSO6, UFO, USO4 or USO6 options have
been negotiated, the driver MAY use \field{hdr_len} only as a hint about the
transport header size.
The driver MUST NOT rely on \field{hdr_len} to be correct.
\begin{note}
This is due to various bugs in implementations.
\end{note}

If neither VIRTIO_NET_HDR_F_NEEDS_CSUM nor
VIRTIO_NET_HDR_F_DATA_VALID is set, the driver MUST NOT
rely on the packet checksum being correct.

If both the VIRTIO_NET_HDR_GSO_UDP_TUNNEL_IPV4 bit and
the VIRTIO_NET_HDR_GSO_UDP_TUNNEL_IPV6 bit in in \field{gso_type} are set,
the driver MUST NOT accept the packet.

If the VIRTIO_NET_HDR_GSO_UDP_TUNNEL_IPV4 bit or the VIRTIO_NET_HDR_GSO_UDP_TUNNEL_IPV6
bit in \field{gso_type} are not set, the driver MUST NOT use the
\field{outer_th_offset} and \field{inner_nh_offset}.

If either the VIRTIO_NET_HDR_GSO_UDP_TUNNEL_IPV4 bit or
the VIRTIO_NET_HDR_GSO_UDP_TUNNEL_IPV6 bit in \field{gso_type} are set, and any of
the following is true:
\begin{itemize}
\item the VIRTIO_NET_HDR_F_NEEDS_CSUM bit is not set in \field{flags}
\item the VIRTIO_NET_HDR_F_DATA_VALID bit is set in \field{flags}
\item the \field{gso_type} excluding the VIRTIO_NET_HDR_GSO_UDP_TUNNEL_IPV4
bit and the VIRTIO_NET_HDR_GSO_UDP_TUNNEL_IPV6 bit is VIRTIO_NET_HDR_GSO_NONE
\end{itemize}
the driver MUST NOT accept the packet.

If the VIRTIO_NET_HDR_F_UDP_TUNNEL_CSUM bit and the VIRTIO_NET_HDR_F_NEEDS_CSUM
bit in \field{flags} are set,
and both the bits VIRTIO_NET_HDR_GSO_UDP_TUNNEL_IPV4 and
VIRTIO_NET_HDR_GSO_UDP_TUNNEL_IPV6 in \field{gso_type} are not set,
the driver MOST NOT accept the packet.

\paragraph{Hash calculation for incoming packets}
\label{sec:Device Types / Network Device / Device Operation / Processing of Incoming Packets / Hash calculation for incoming packets}

A device attempts to calculate a per-packet hash in the following cases:
\begin{itemize}
\item The feature VIRTIO_NET_F_RSS was negotiated. The device uses the hash to determine the receive virtqueue to place incoming packets.
\item The feature VIRTIO_NET_F_HASH_REPORT was negotiated. The device reports the hash value and the hash type with the packet.
\end{itemize}

If the feature VIRTIO_NET_F_RSS was negotiated:
\begin{itemize}
\item The device uses \field{hash_types} of the virtio_net_rss_config structure as 'Enabled hash types' bitmask.
\item If additionally the feature VIRTIO_NET_F_HASH_TUNNEL was negotiated, the device uses \field{enabled_tunnel_types} of the
      virtnet_hash_tunnel structure as 'Encapsulation types enabled for inner header hash' bitmask.
\item The device uses a key as defined in \field{hash_key_data} and \field{hash_key_length} of the virtio_net_rss_config structure (see
\ref{sec:Device Types / Network Device / Device Operation / Control Virtqueue / Receive-side scaling (RSS) / Setting RSS parameters}).
\end{itemize}

If the feature VIRTIO_NET_F_RSS was not negotiated:
\begin{itemize}
\item The device uses \field{hash_types} of the virtio_net_hash_config structure as 'Enabled hash types' bitmask.
\item If additionally the feature VIRTIO_NET_F_HASH_TUNNEL was negotiated, the device uses \field{enabled_tunnel_types} of the
      virtnet_hash_tunnel structure as 'Encapsulation types enabled for inner header hash' bitmask.
\item The device uses a key as defined in \field{hash_key_data} and \field{hash_key_length} of the virtio_net_hash_config structure (see
\ref{sec:Device Types / Network Device / Device Operation / Control Virtqueue / Automatic receive steering in multiqueue mode / Hash calculation}).
\end{itemize}

Note that if the device offers VIRTIO_NET_F_HASH_REPORT, even if it supports only one pair of virtqueues, it MUST support
at least one of commands of VIRTIO_NET_CTRL_MQ class to configure reported hash parameters:
\begin{itemize}
\item If the device offers VIRTIO_NET_F_RSS, it MUST support VIRTIO_NET_CTRL_MQ_RSS_CONFIG command per
 \ref{sec:Device Types / Network Device / Device Operation / Control Virtqueue / Receive-side scaling (RSS) / Setting RSS parameters}.
\item Otherwise the device MUST support VIRTIO_NET_CTRL_MQ_HASH_CONFIG command per
 \ref{sec:Device Types / Network Device / Device Operation / Control Virtqueue / Automatic receive steering in multiqueue mode / Hash calculation}.
\end{itemize}

The per-packet hash calculation can depend on the IP packet type. See
\hyperref[intro:IP]{[IP]}, \hyperref[intro:UDP]{[UDP]} and \hyperref[intro:TCP]{[TCP]}.

\subparagraph{Supported/enabled hash types}
\label{sec:Device Types / Network Device / Device Operation / Processing of Incoming Packets / Hash calculation for incoming packets / Supported/enabled hash types}
Hash types applicable for IPv4 packets:
\begin{lstlisting}
#define VIRTIO_NET_HASH_TYPE_IPv4              (1 << 0)
#define VIRTIO_NET_HASH_TYPE_TCPv4             (1 << 1)
#define VIRTIO_NET_HASH_TYPE_UDPv4             (1 << 2)
\end{lstlisting}
Hash types applicable for IPv6 packets without extension headers
\begin{lstlisting}
#define VIRTIO_NET_HASH_TYPE_IPv6              (1 << 3)
#define VIRTIO_NET_HASH_TYPE_TCPv6             (1 << 4)
#define VIRTIO_NET_HASH_TYPE_UDPv6             (1 << 5)
\end{lstlisting}
Hash types applicable for IPv6 packets with extension headers
\begin{lstlisting}
#define VIRTIO_NET_HASH_TYPE_IP_EX             (1 << 6)
#define VIRTIO_NET_HASH_TYPE_TCP_EX            (1 << 7)
#define VIRTIO_NET_HASH_TYPE_UDP_EX            (1 << 8)
\end{lstlisting}

\subparagraph{IPv4 packets}
\label{sec:Device Types / Network Device / Device Operation / Processing of Incoming Packets / Hash calculation for incoming packets / IPv4 packets}
The device calculates the hash on IPv4 packets according to 'Enabled hash types' bitmask as follows:
\begin{itemize}
\item If VIRTIO_NET_HASH_TYPE_TCPv4 is set and the packet has
a TCP header, the hash is calculated over the following fields:
\begin{itemize}
\item Source IP address
\item Destination IP address
\item Source TCP port
\item Destination TCP port
\end{itemize}
\item Else if VIRTIO_NET_HASH_TYPE_UDPv4 is set and the
packet has a UDP header, the hash is calculated over the following fields:
\begin{itemize}
\item Source IP address
\item Destination IP address
\item Source UDP port
\item Destination UDP port
\end{itemize}
\item Else if VIRTIO_NET_HASH_TYPE_IPv4 is set, the hash is
calculated over the following fields:
\begin{itemize}
\item Source IP address
\item Destination IP address
\end{itemize}
\item Else the device does not calculate the hash
\end{itemize}

\subparagraph{IPv6 packets without extension header}
\label{sec:Device Types / Network Device / Device Operation / Processing of Incoming Packets / Hash calculation for incoming packets / IPv6 packets without extension header}
The device calculates the hash on IPv6 packets without extension
headers according to 'Enabled hash types' bitmask as follows:
\begin{itemize}
\item If VIRTIO_NET_HASH_TYPE_TCPv6 is set and the packet has
a TCPv6 header, the hash is calculated over the following fields:
\begin{itemize}
\item Source IPv6 address
\item Destination IPv6 address
\item Source TCP port
\item Destination TCP port
\end{itemize}
\item Else if VIRTIO_NET_HASH_TYPE_UDPv6 is set and the
packet has a UDPv6 header, the hash is calculated over the following fields:
\begin{itemize}
\item Source IPv6 address
\item Destination IPv6 address
\item Source UDP port
\item Destination UDP port
\end{itemize}
\item Else if VIRTIO_NET_HASH_TYPE_IPv6 is set, the hash is
calculated over the following fields:
\begin{itemize}
\item Source IPv6 address
\item Destination IPv6 address
\end{itemize}
\item Else the device does not calculate the hash
\end{itemize}

\subparagraph{IPv6 packets with extension header}
\label{sec:Device Types / Network Device / Device Operation / Processing of Incoming Packets / Hash calculation for incoming packets / IPv6 packets with extension header}
The device calculates the hash on IPv6 packets with extension
headers according to 'Enabled hash types' bitmask as follows:
\begin{itemize}
\item If VIRTIO_NET_HASH_TYPE_TCP_EX is set and the packet
has a TCPv6 header, the hash is calculated over the following fields:
\begin{itemize}
\item Home address from the home address option in the IPv6 destination options header. If the extension header is not present, use the Source IPv6 address.
\item IPv6 address that is contained in the Routing-Header-Type-2 from the associated extension header. If the extension header is not present, use the Destination IPv6 address.
\item Source TCP port
\item Destination TCP port
\end{itemize}
\item Else if VIRTIO_NET_HASH_TYPE_UDP_EX is set and the
packet has a UDPv6 header, the hash is calculated over the following fields:
\begin{itemize}
\item Home address from the home address option in the IPv6 destination options header. If the extension header is not present, use the Source IPv6 address.
\item IPv6 address that is contained in the Routing-Header-Type-2 from the associated extension header. If the extension header is not present, use the Destination IPv6 address.
\item Source UDP port
\item Destination UDP port
\end{itemize}
\item Else if VIRTIO_NET_HASH_TYPE_IP_EX is set, the hash is
calculated over the following fields:
\begin{itemize}
\item Home address from the home address option in the IPv6 destination options header. If the extension header is not present, use the Source IPv6 address.
\item IPv6 address that is contained in the Routing-Header-Type-2 from the associated extension header. If the extension header is not present, use the Destination IPv6 address.
\end{itemize}
\item Else skip IPv6 extension headers and calculate the hash as
defined for an IPv6 packet without extension headers
(see \ref{sec:Device Types / Network Device / Device Operation / Processing of Incoming Packets / Hash calculation for incoming packets / IPv6 packets without extension header}).
\end{itemize}

\paragraph{Inner Header Hash}
\label{sec:Device Types / Network Device / Device Operation / Processing of Incoming Packets / Inner Header Hash}

If VIRTIO_NET_F_HASH_TUNNEL has been negotiated, the driver can send the command
VIRTIO_NET_CTRL_HASH_TUNNEL_SET to configure the calculation of the inner header hash.

\begin{lstlisting}
struct virtnet_hash_tunnel {
    le32 enabled_tunnel_types;
};

#define VIRTIO_NET_CTRL_HASH_TUNNEL 7
 #define VIRTIO_NET_CTRL_HASH_TUNNEL_SET 0
\end{lstlisting}

Field \field{enabled_tunnel_types} contains the bitmask of encapsulation types enabled for inner header hash.
See \ref{sec:Device Types / Network Device / Device Operation / Processing of Incoming Packets /
Hash calculation for incoming packets / Encapsulation types supported/enabled for inner header hash}.

The class VIRTIO_NET_CTRL_HASH_TUNNEL has one command:
VIRTIO_NET_CTRL_HASH_TUNNEL_SET sets \field{enabled_tunnel_types} for the device using the
virtnet_hash_tunnel structure, which is read-only for the device.

Inner header hash is disabled by VIRTIO_NET_CTRL_HASH_TUNNEL_SET with \field{enabled_tunnel_types} set to 0.

Initially (before the driver sends any VIRTIO_NET_CTRL_HASH_TUNNEL_SET command) all
encapsulation types are disabled for inner header hash.

\subparagraph{Encapsulated packet}
\label{sec:Device Types / Network Device / Device Operation / Processing of Incoming Packets / Hash calculation for incoming packets / Encapsulated packet}

Multiple tunneling protocols allow encapsulating an inner, payload packet in an outer, encapsulated packet.
The encapsulated packet thus contains an outer header and an inner header, and the device calculates the
hash over either the inner header or the outer header.

If VIRTIO_NET_F_HASH_TUNNEL is negotiated and a received encapsulated packet's outer header matches one of the
encapsulation types enabled in \field{enabled_tunnel_types}, then the device uses the inner header for hash
calculations (only a single level of encapsulation is currently supported).

If VIRTIO_NET_F_HASH_TUNNEL is negotiated and a received packet's (outer) header does not match any encapsulation
types enabled in \field{enabled_tunnel_types}, then the device uses the outer header for hash calculations.

\subparagraph{Encapsulation types supported/enabled for inner header hash}
\label{sec:Device Types / Network Device / Device Operation / Processing of Incoming Packets /
Hash calculation for incoming packets / Encapsulation types supported/enabled for inner header hash}

Encapsulation types applicable for inner header hash:
\begin{lstlisting}[escapechar=|]
#define VIRTIO_NET_HASH_TUNNEL_TYPE_GRE_2784    (1 << 0) /* |\hyperref[intro:rfc2784]{[RFC2784]}| */
#define VIRTIO_NET_HASH_TUNNEL_TYPE_GRE_2890    (1 << 1) /* |\hyperref[intro:rfc2890]{[RFC2890]}| */
#define VIRTIO_NET_HASH_TUNNEL_TYPE_GRE_7676    (1 << 2) /* |\hyperref[intro:rfc7676]{[RFC7676]}| */
#define VIRTIO_NET_HASH_TUNNEL_TYPE_GRE_UDP     (1 << 3) /* |\hyperref[intro:rfc8086]{[GRE-in-UDP]}| */
#define VIRTIO_NET_HASH_TUNNEL_TYPE_VXLAN       (1 << 4) /* |\hyperref[intro:vxlan]{[VXLAN]}| */
#define VIRTIO_NET_HASH_TUNNEL_TYPE_VXLAN_GPE   (1 << 5) /* |\hyperref[intro:vxlan-gpe]{[VXLAN-GPE]}| */
#define VIRTIO_NET_HASH_TUNNEL_TYPE_GENEVE      (1 << 6) /* |\hyperref[intro:geneve]{[GENEVE]}| */
#define VIRTIO_NET_HASH_TUNNEL_TYPE_IPIP        (1 << 7) /* |\hyperref[intro:ipip]{[IPIP]}| */
#define VIRTIO_NET_HASH_TUNNEL_TYPE_NVGRE       (1 << 8) /* |\hyperref[intro:nvgre]{[NVGRE]}| */
\end{lstlisting}

\subparagraph{Advice}
Example uses of the inner header hash:
\begin{itemize}
\item Legacy tunneling protocols, lacking the outer header entropy, can use RSS with the inner header hash to
      distribute flows with identical outer but different inner headers across various queues, improving performance.
\item Identify an inner flow distributed across multiple outer tunnels.
\end{itemize}

As using the inner header hash completely discards the outer header entropy, care must be taken
if the inner header is controlled by an adversary, as the adversary can then intentionally create
configurations with insufficient entropy.

Besides disabling the inner header hash, mitigations would depend on how the hash is used. When the hash
use is limited to the RSS queue selection, the inner header hash may have quality of service (QoS) limitations.

\devicenormative{\subparagraph}{Inner Header Hash}{Device Types / Network Device / Device Operation / Control Virtqueue / Inner Header Hash}

If the (outer) header of the received packet does not match any encapsulation types enabled
in \field{enabled_tunnel_types}, the device MUST calculate the hash on the outer header.

If the device receives any bits in \field{enabled_tunnel_types} which are not set in \field{supported_tunnel_types},
it SHOULD respond to the VIRTIO_NET_CTRL_HASH_TUNNEL_SET command with VIRTIO_NET_ERR.

If the driver sets \field{enabled_tunnel_types} to 0 through VIRTIO_NET_CTRL_HASH_TUNNEL_SET or upon the device reset,
the device MUST disable the inner header hash for all encapsulation types.

\drivernormative{\subparagraph}{Inner Header Hash}{Device Types / Network Device / Device Operation / Control Virtqueue / Inner Header Hash}

The driver MUST have negotiated the VIRTIO_NET_F_HASH_TUNNEL feature when issuing the VIRTIO_NET_CTRL_HASH_TUNNEL_SET command.

The driver MUST NOT set any bits in \field{enabled_tunnel_types} which are not set in \field{supported_tunnel_types}.

The driver MUST ignore bits in \field{supported_tunnel_types} which are not documented in this specification.

\paragraph{Hash reporting for incoming packets}
\label{sec:Device Types / Network Device / Device Operation / Processing of Incoming Packets / Hash reporting for incoming packets}

If VIRTIO_NET_F_HASH_REPORT was negotiated and
 the device has calculated the hash for the packet, the device fills \field{hash_report} with the report type of calculated hash
and \field{hash_value} with the value of calculated hash.

If VIRTIO_NET_F_HASH_REPORT was negotiated but due to any reason the
hash was not calculated, the device sets \field{hash_report} to VIRTIO_NET_HASH_REPORT_NONE.

Possible values that the device can report in \field{hash_report} are defined below.
They correspond to supported hash types defined in
\ref{sec:Device Types / Network Device / Device Operation / Processing of Incoming Packets / Hash calculation for incoming packets / Supported/enabled hash types}
as follows:

VIRTIO_NET_HASH_TYPE_XXX = 1 << (VIRTIO_NET_HASH_REPORT_XXX - 1)

\begin{lstlisting}
#define VIRTIO_NET_HASH_REPORT_NONE            0
#define VIRTIO_NET_HASH_REPORT_IPv4            1
#define VIRTIO_NET_HASH_REPORT_TCPv4           2
#define VIRTIO_NET_HASH_REPORT_UDPv4           3
#define VIRTIO_NET_HASH_REPORT_IPv6            4
#define VIRTIO_NET_HASH_REPORT_TCPv6           5
#define VIRTIO_NET_HASH_REPORT_UDPv6           6
#define VIRTIO_NET_HASH_REPORT_IPv6_EX         7
#define VIRTIO_NET_HASH_REPORT_TCPv6_EX        8
#define VIRTIO_NET_HASH_REPORT_UDPv6_EX        9
\end{lstlisting}

\subsubsection{Control Virtqueue}\label{sec:Device Types / Network Device / Device Operation / Control Virtqueue}

The driver uses the control virtqueue (if VIRTIO_NET_F_CTRL_VQ is
negotiated) to send commands to manipulate various features of
the device which would not easily map into the configuration
space.

All commands are of the following form:

\begin{lstlisting}
struct virtio_net_ctrl {
        u8 class;
        u8 command;
        u8 command-specific-data[];
        u8 ack;
        u8 command-specific-result[];
};

/* ack values */
#define VIRTIO_NET_OK     0
#define VIRTIO_NET_ERR    1
\end{lstlisting}

The \field{class}, \field{command} and command-specific-data are set by the
driver, and the device sets the \field{ack} byte and optionally
\field{command-specific-result}. There is little the driver can
do except issue a diagnostic if \field{ack} is not VIRTIO_NET_OK.

The command VIRTIO_NET_CTRL_STATS_QUERY and VIRTIO_NET_CTRL_STATS_GET contain
\field{command-specific-result}.

\paragraph{Packet Receive Filtering}\label{sec:Device Types / Network Device / Device Operation / Control Virtqueue / Packet Receive Filtering}
\label{sec:Device Types / Network Device / Device Operation / Control Virtqueue / Setting Promiscuous Mode}%old label for latexdiff

If the VIRTIO_NET_F_CTRL_RX and VIRTIO_NET_F_CTRL_RX_EXTRA
features are negotiated, the driver can send control commands for
promiscuous mode, multicast, unicast and broadcast receiving.

\begin{note}
In general, these commands are best-effort: unwanted
packets could still arrive.
\end{note}

\begin{lstlisting}
#define VIRTIO_NET_CTRL_RX    0
 #define VIRTIO_NET_CTRL_RX_PROMISC      0
 #define VIRTIO_NET_CTRL_RX_ALLMULTI     1
 #define VIRTIO_NET_CTRL_RX_ALLUNI       2
 #define VIRTIO_NET_CTRL_RX_NOMULTI      3
 #define VIRTIO_NET_CTRL_RX_NOUNI        4
 #define VIRTIO_NET_CTRL_RX_NOBCAST      5
\end{lstlisting}


\devicenormative{\subparagraph}{Packet Receive Filtering}{Device Types / Network Device / Device Operation / Control Virtqueue / Packet Receive Filtering}

If the VIRTIO_NET_F_CTRL_RX feature has been negotiated,
the device MUST support the following VIRTIO_NET_CTRL_RX class
commands:
\begin{itemize}
\item VIRTIO_NET_CTRL_RX_PROMISC turns promiscuous mode on and
off. The command-specific-data is one byte containing 0 (off) or
1 (on). If promiscuous mode is on, the device SHOULD receive all
incoming packets.
This SHOULD take effect even if one of the other modes set by
a VIRTIO_NET_CTRL_RX class command is on.
\item VIRTIO_NET_CTRL_RX_ALLMULTI turns all-multicast receive on and
off. The command-specific-data is one byte containing 0 (off) or
1 (on). When all-multicast receive is on the device SHOULD allow
all incoming multicast packets.
\end{itemize}

If the VIRTIO_NET_F_CTRL_RX_EXTRA feature has been negotiated,
the device MUST support the following VIRTIO_NET_CTRL_RX class
commands:
\begin{itemize}
\item VIRTIO_NET_CTRL_RX_ALLUNI turns all-unicast receive on and
off. The command-specific-data is one byte containing 0 (off) or
1 (on). When all-unicast receive is on the device SHOULD allow
all incoming unicast packets.
\item VIRTIO_NET_CTRL_RX_NOMULTI suppresses multicast receive.
The command-specific-data is one byte containing 0 (multicast
receive allowed) or 1 (multicast receive suppressed).
When multicast receive is suppressed, the device SHOULD NOT
send multicast packets to the driver.
This SHOULD take effect even if VIRTIO_NET_CTRL_RX_ALLMULTI is on.
This filter SHOULD NOT apply to broadcast packets.
\item VIRTIO_NET_CTRL_RX_NOUNI suppresses unicast receive.
The command-specific-data is one byte containing 0 (unicast
receive allowed) or 1 (unicast receive suppressed).
When unicast receive is suppressed, the device SHOULD NOT
send unicast packets to the driver.
This SHOULD take effect even if VIRTIO_NET_CTRL_RX_ALLUNI is on.
\item VIRTIO_NET_CTRL_RX_NOBCAST suppresses broadcast receive.
The command-specific-data is one byte containing 0 (broadcast
receive allowed) or 1 (broadcast receive suppressed).
When broadcast receive is suppressed, the device SHOULD NOT
send broadcast packets to the driver.
This SHOULD take effect even if VIRTIO_NET_CTRL_RX_ALLMULTI is on.
\end{itemize}

\drivernormative{\subparagraph}{Packet Receive Filtering}{Device Types / Network Device / Device Operation / Control Virtqueue / Packet Receive Filtering}

If the VIRTIO_NET_F_CTRL_RX feature has not been negotiated,
the driver MUST NOT issue commands VIRTIO_NET_CTRL_RX_PROMISC or
VIRTIO_NET_CTRL_RX_ALLMULTI.

If the VIRTIO_NET_F_CTRL_RX_EXTRA feature has not been negotiated,
the driver MUST NOT issue commands
 VIRTIO_NET_CTRL_RX_ALLUNI,
 VIRTIO_NET_CTRL_RX_NOMULTI,
 VIRTIO_NET_CTRL_RX_NOUNI or
 VIRTIO_NET_CTRL_RX_NOBCAST.

\paragraph{Setting MAC Address Filtering}\label{sec:Device Types / Network Device / Device Operation / Control Virtqueue / Setting MAC Address Filtering}

If the VIRTIO_NET_F_CTRL_RX feature is negotiated, the driver can
send control commands for MAC address filtering.

\begin{lstlisting}
struct virtio_net_ctrl_mac {
        le32 entries;
        u8 macs[entries][6];
};

#define VIRTIO_NET_CTRL_MAC    1
 #define VIRTIO_NET_CTRL_MAC_TABLE_SET        0
 #define VIRTIO_NET_CTRL_MAC_ADDR_SET         1
\end{lstlisting}

The device can filter incoming packets by any number of destination
MAC addresses\footnote{Since there are no guarantees, it can use a hash filter or
silently switch to allmulti or promiscuous mode if it is given too
many addresses.
}. This table is set using the class
VIRTIO_NET_CTRL_MAC and the command VIRTIO_NET_CTRL_MAC_TABLE_SET. The
command-specific-data is two variable length tables of 6-byte MAC
addresses (as described in struct virtio_net_ctrl_mac). The first table contains unicast addresses, and the second
contains multicast addresses.

The VIRTIO_NET_CTRL_MAC_ADDR_SET command is used to set the
default MAC address which rx filtering
accepts (and if VIRTIO_NET_F_MAC has been negotiated,
this will be reflected in \field{mac} in config space).

The command-specific-data for VIRTIO_NET_CTRL_MAC_ADDR_SET is
the 6-byte MAC address.

\devicenormative{\subparagraph}{Setting MAC Address Filtering}{Device Types / Network Device / Device Operation / Control Virtqueue / Setting MAC Address Filtering}

The device MUST have an empty MAC filtering table on reset.

The device MUST update the MAC filtering table before it consumes
the VIRTIO_NET_CTRL_MAC_TABLE_SET command.

The device MUST update \field{mac} in config space before it consumes
the VIRTIO_NET_CTRL_MAC_ADDR_SET command, if VIRTIO_NET_F_MAC has
been negotiated.

The device SHOULD drop incoming packets which have a destination MAC which
matches neither the \field{mac} (or that set with VIRTIO_NET_CTRL_MAC_ADDR_SET)
nor the MAC filtering table.

\drivernormative{\subparagraph}{Setting MAC Address Filtering}{Device Types / Network Device / Device Operation / Control Virtqueue / Setting MAC Address Filtering}

If VIRTIO_NET_F_CTRL_RX has not been negotiated,
the driver MUST NOT issue VIRTIO_NET_CTRL_MAC class commands.

If VIRTIO_NET_F_CTRL_RX has been negotiated,
the driver SHOULD issue VIRTIO_NET_CTRL_MAC_ADDR_SET
to set the default mac if it is different from \field{mac}.

The driver MUST follow the VIRTIO_NET_CTRL_MAC_TABLE_SET command
by a le32 number, followed by that number of non-multicast
MAC addresses, followed by another le32 number, followed by
that number of multicast addresses.  Either number MAY be 0.

\subparagraph{Legacy Interface: Setting MAC Address Filtering}\label{sec:Device Types / Network Device / Device Operation / Control Virtqueue / Setting MAC Address Filtering / Legacy Interface: Setting MAC Address Filtering}
When using the legacy interface, transitional devices and drivers
MUST format \field{entries} in struct virtio_net_ctrl_mac
according to the native endian of the guest rather than
(necessarily when not using the legacy interface) little-endian.

Legacy drivers that didn't negotiate VIRTIO_NET_F_CTRL_MAC_ADDR
changed \field{mac} in config space when NIC is accepting
incoming packets. These drivers always wrote the mac value from
first to last byte, therefore after detecting such drivers,
a transitional device MAY defer MAC update, or MAY defer
processing incoming packets until driver writes the last byte
of \field{mac} in the config space.

\paragraph{VLAN Filtering}\label{sec:Device Types / Network Device / Device Operation / Control Virtqueue / VLAN Filtering}

If the driver negotiates the VIRTIO_NET_F_CTRL_VLAN feature, it
can control a VLAN filter table in the device. The VLAN filter
table applies only to VLAN tagged packets.

When VIRTIO_NET_F_CTRL_VLAN is negotiated, the device starts with
an empty VLAN filter table.

\begin{note}
Similar to the MAC address based filtering, the VLAN filtering
is also best-effort: unwanted packets could still arrive.
\end{note}

\begin{lstlisting}
#define VIRTIO_NET_CTRL_VLAN       2
 #define VIRTIO_NET_CTRL_VLAN_ADD             0
 #define VIRTIO_NET_CTRL_VLAN_DEL             1
\end{lstlisting}

Both the VIRTIO_NET_CTRL_VLAN_ADD and VIRTIO_NET_CTRL_VLAN_DEL
command take a little-endian 16-bit VLAN id as the command-specific-data.

VIRTIO_NET_CTRL_VLAN_ADD command adds the specified VLAN to the
VLAN filter table.

VIRTIO_NET_CTRL_VLAN_DEL command removes the specified VLAN from
the VLAN filter table.

\devicenormative{\subparagraph}{VLAN Filtering}{Device Types / Network Device / Device Operation / Control Virtqueue / VLAN Filtering}

When VIRTIO_NET_F_CTRL_VLAN is not negotiated, the device MUST
accept all VLAN tagged packets.

When VIRTIO_NET_F_CTRL_VLAN is negotiated, the device MUST
accept all VLAN tagged packets whose VLAN tag is present in
the VLAN filter table and SHOULD drop all VLAN tagged packets
whose VLAN tag is absent in the VLAN filter table.

\subparagraph{Legacy Interface: VLAN Filtering}\label{sec:Device Types / Network Device / Device Operation / Control Virtqueue / VLAN Filtering / Legacy Interface: VLAN Filtering}
When using the legacy interface, transitional devices and drivers
MUST format the VLAN id
according to the native endian of the guest rather than
(necessarily when not using the legacy interface) little-endian.

\paragraph{Gratuitous Packet Sending}\label{sec:Device Types / Network Device / Device Operation / Control Virtqueue / Gratuitous Packet Sending}

If the driver negotiates the VIRTIO_NET_F_GUEST_ANNOUNCE (depends
on VIRTIO_NET_F_CTRL_VQ), the device can ask the driver to send gratuitous
packets; this is usually done after the guest has been physically
migrated, and needs to announce its presence on the new network
links. (As hypervisor does not have the knowledge of guest
network configuration (eg. tagged vlan) it is simplest to prod
the guest in this way).

\begin{lstlisting}
#define VIRTIO_NET_CTRL_ANNOUNCE       3
 #define VIRTIO_NET_CTRL_ANNOUNCE_ACK             0
\end{lstlisting}

The driver checks VIRTIO_NET_S_ANNOUNCE bit in the device configuration \field{status} field
when it notices the changes of device configuration. The
command VIRTIO_NET_CTRL_ANNOUNCE_ACK is used to indicate that
driver has received the notification and device clears the
VIRTIO_NET_S_ANNOUNCE bit in \field{status}.

Processing this notification involves:

\begin{enumerate}
\item Sending the gratuitous packets (eg. ARP) or marking there are pending
  gratuitous packets to be sent and letting deferred routine to
  send them.

\item Sending VIRTIO_NET_CTRL_ANNOUNCE_ACK command through control
  vq.
\end{enumerate}

\drivernormative{\subparagraph}{Gratuitous Packet Sending}{Device Types / Network Device / Device Operation / Control Virtqueue / Gratuitous Packet Sending}

If the driver negotiates VIRTIO_NET_F_GUEST_ANNOUNCE, it SHOULD notify
network peers of its new location after it sees the VIRTIO_NET_S_ANNOUNCE bit
in \field{status}.  The driver MUST send a command on the command queue
with class VIRTIO_NET_CTRL_ANNOUNCE and command VIRTIO_NET_CTRL_ANNOUNCE_ACK.

\devicenormative{\subparagraph}{Gratuitous Packet Sending}{Device Types / Network Device / Device Operation / Control Virtqueue / Gratuitous Packet Sending}

If VIRTIO_NET_F_GUEST_ANNOUNCE is negotiated, the device MUST clear the
VIRTIO_NET_S_ANNOUNCE bit in \field{status} upon receipt of a command buffer
with class VIRTIO_NET_CTRL_ANNOUNCE and command VIRTIO_NET_CTRL_ANNOUNCE_ACK
before marking the buffer as used.

\paragraph{Device operation in multiqueue mode}\label{sec:Device Types / Network Device / Device Operation / Control Virtqueue / Device operation in multiqueue mode}

This specification defines the following modes that a device MAY implement for operation with multiple transmit/receive virtqueues:
\begin{itemize}
\item Automatic receive steering as defined in \ref{sec:Device Types / Network Device / Device Operation / Control Virtqueue / Automatic receive steering in multiqueue mode}.
 If a device supports this mode, it offers the VIRTIO_NET_F_MQ feature bit.
\item Receive-side scaling as defined in \ref{devicenormative:Device Types / Network Device / Device Operation / Control Virtqueue / Receive-side scaling (RSS) / RSS processing}.
 If a device supports this mode, it offers the VIRTIO_NET_F_RSS feature bit.
\end{itemize}

A device MAY support one of these features or both. The driver MAY negotiate any set of these features that the device supports.

Multiqueue is disabled by default.

The driver enables multiqueue by sending a command using \field{class} VIRTIO_NET_CTRL_MQ. The \field{command} selects the mode of multiqueue operation, as follows:
\begin{lstlisting}
#define VIRTIO_NET_CTRL_MQ    4
 #define VIRTIO_NET_CTRL_MQ_VQ_PAIRS_SET        0 (for automatic receive steering)
 #define VIRTIO_NET_CTRL_MQ_RSS_CONFIG          1 (for configurable receive steering)
 #define VIRTIO_NET_CTRL_MQ_HASH_CONFIG         2 (for configurable hash calculation)
\end{lstlisting}

If more than one multiqueue mode is negotiated, the resulting device configuration is defined by the last command sent by the driver.

\paragraph{Automatic receive steering in multiqueue mode}\label{sec:Device Types / Network Device / Device Operation / Control Virtqueue / Automatic receive steering in multiqueue mode}

If the driver negotiates the VIRTIO_NET_F_MQ feature bit (depends on VIRTIO_NET_F_CTRL_VQ), it MAY transmit outgoing packets on one
of the multiple transmitq1\ldots transmitqN and ask the device to
queue incoming packets into one of the multiple receiveq1\ldots receiveqN
depending on the packet flow.

The driver enables multiqueue by
sending the VIRTIO_NET_CTRL_MQ_VQ_PAIRS_SET command, specifying
the number of the transmit and receive queues to be used up to
\field{max_virtqueue_pairs}; subsequently,
transmitq1\ldots transmitqn and receiveq1\ldots receiveqn where
n=\field{virtqueue_pairs} MAY be used.
\begin{lstlisting}
struct virtio_net_ctrl_mq_pairs_set {
       le16 virtqueue_pairs;
};
#define VIRTIO_NET_CTRL_MQ_VQ_PAIRS_MIN        1
#define VIRTIO_NET_CTRL_MQ_VQ_PAIRS_MAX        0x8000

\end{lstlisting}

When multiqueue is enabled by VIRTIO_NET_CTRL_MQ_VQ_PAIRS_SET command, the device MUST use automatic receive steering
based on packet flow. Programming of the receive steering
classificator is implicit. After the driver transmitted a packet of a
flow on transmitqX, the device SHOULD cause incoming packets for that flow to
be steered to receiveqX. For uni-directional protocols, or where
no packets have been transmitted yet, the device MAY steer a packet
to a random queue out of the specified receiveq1\ldots receiveqn.

Multiqueue is disabled by VIRTIO_NET_CTRL_MQ_VQ_PAIRS_SET with \field{virtqueue_pairs} to 1 (this is
the default) and waiting for the device to use the command buffer.

\drivernormative{\subparagraph}{Automatic receive steering in multiqueue mode}{Device Types / Network Device / Device Operation / Control Virtqueue / Automatic receive steering in multiqueue mode}

The driver MUST configure the virtqueues before enabling them with the
VIRTIO_NET_CTRL_MQ_VQ_PAIRS_SET command.

The driver MUST NOT request a \field{virtqueue_pairs} of 0 or
greater than \field{max_virtqueue_pairs} in the device configuration space.

The driver MUST queue packets only on any transmitq1 before the
VIRTIO_NET_CTRL_MQ_VQ_PAIRS_SET command.

The driver MUST NOT queue packets on transmit queues greater than
\field{virtqueue_pairs} once it has placed the VIRTIO_NET_CTRL_MQ_VQ_PAIRS_SET command in the available ring.

\devicenormative{\subparagraph}{Automatic receive steering in multiqueue mode}{Device Types / Network Device / Device Operation / Control Virtqueue / Automatic receive steering in multiqueue mode}

After initialization of reset, the device MUST queue packets only on receiveq1.

The device MUST NOT queue packets on receive queues greater than
\field{virtqueue_pairs} once it has placed the
VIRTIO_NET_CTRL_MQ_VQ_PAIRS_SET command in a used buffer.

If the destination receive queue is being reset (See \ref{sec:Basic Facilities of a Virtio Device / Virtqueues / Virtqueue Reset}),
the device SHOULD re-select another random queue. If all receive queues are
being reset, the device MUST drop the packet.

\subparagraph{Legacy Interface: Automatic receive steering in multiqueue mode}\label{sec:Device Types / Network Device / Device Operation / Control Virtqueue / Automatic receive steering in multiqueue mode / Legacy Interface: Automatic receive steering in multiqueue mode}
When using the legacy interface, transitional devices and drivers
MUST format \field{virtqueue_pairs}
according to the native endian of the guest rather than
(necessarily when not using the legacy interface) little-endian.

\subparagraph{Hash calculation}\label{sec:Device Types / Network Device / Device Operation / Control Virtqueue / Automatic receive steering in multiqueue mode / Hash calculation}
If VIRTIO_NET_F_HASH_REPORT was negotiated and the device uses automatic receive steering,
the device MUST support a command to configure hash calculation parameters.

The driver provides parameters for hash calculation as follows:

\field{class} VIRTIO_NET_CTRL_MQ, \field{command} VIRTIO_NET_CTRL_MQ_HASH_CONFIG.

The \field{command-specific-data} has following format:
\begin{lstlisting}
struct virtio_net_hash_config {
    le32 hash_types;
    le16 reserved[4];
    u8 hash_key_length;
    u8 hash_key_data[hash_key_length];
};
\end{lstlisting}
Field \field{hash_types} contains a bitmask of allowed hash types as
defined in
\ref{sec:Device Types / Network Device / Device Operation / Processing of Incoming Packets / Hash calculation for incoming packets / Supported/enabled hash types}.
Initially the device has all hash types disabled and reports only VIRTIO_NET_HASH_REPORT_NONE.

Field \field{reserved} MUST contain zeroes. It is defined to make the structure to match the layout of virtio_net_rss_config structure,
defined in \ref{sec:Device Types / Network Device / Device Operation / Control Virtqueue / Receive-side scaling (RSS)}.

Fields \field{hash_key_length} and \field{hash_key_data} define the key to be used in hash calculation.

\paragraph{Receive-side scaling (RSS)}\label{sec:Device Types / Network Device / Device Operation / Control Virtqueue / Receive-side scaling (RSS)}
A device offers the feature VIRTIO_NET_F_RSS if it supports RSS receive steering with Toeplitz hash calculation and configurable parameters.

A driver queries RSS capabilities of the device by reading device configuration as defined in \ref{sec:Device Types / Network Device / Device configuration layout}

\subparagraph{Setting RSS parameters}\label{sec:Device Types / Network Device / Device Operation / Control Virtqueue / Receive-side scaling (RSS) / Setting RSS parameters}

Driver sends a VIRTIO_NET_CTRL_MQ_RSS_CONFIG command using the following format for \field{command-specific-data}:
\begin{lstlisting}
struct rss_rq_id {
   le16 vq_index_1_16: 15; /* Bits 1 to 16 of the virtqueue index */
   le16 reserved: 1; /* Set to zero */
};

struct virtio_net_rss_config {
    le32 hash_types;
    le16 indirection_table_mask;
    struct rss_rq_id unclassified_queue;
    struct rss_rq_id indirection_table[indirection_table_length];
    le16 max_tx_vq;
    u8 hash_key_length;
    u8 hash_key_data[hash_key_length];
};
\end{lstlisting}
Field \field{hash_types} contains a bitmask of allowed hash types as
defined in
\ref{sec:Device Types / Network Device / Device Operation / Processing of Incoming Packets / Hash calculation for incoming packets / Supported/enabled hash types}.

Field \field{indirection_table_mask} is a mask to be applied to
the calculated hash to produce an index in the
\field{indirection_table} array.
Number of entries in \field{indirection_table} is (\field{indirection_table_mask} + 1).

\field{rss_rq_id} is a receive virtqueue id. \field{vq_index_1_16}
consists of bits 1 to 16 of a virtqueue index. For example, a
\field{vq_index_1_16} value of 3 corresponds to virtqueue index 6,
which maps to receiveq4.

Field \field{unclassified_queue} specifies the receive virtqueue id in which to
place unclassified packets.

Field \field{indirection_table} is an array of receive virtqueues ids.

A driver sets \field{max_tx_vq} to inform a device how many transmit virtqueues it may use (transmitq1\ldots transmitq \field{max_tx_vq}).

Fields \field{hash_key_length} and \field{hash_key_data} define the key to be used in hash calculation.

\drivernormative{\subparagraph}{Setting RSS parameters}{Device Types / Network Device / Device Operation / Control Virtqueue / Receive-side scaling (RSS) }

A driver MUST NOT send the VIRTIO_NET_CTRL_MQ_RSS_CONFIG command if the feature VIRTIO_NET_F_RSS has not been negotiated.

A driver MUST fill the \field{indirection_table} array only with
enabled receive virtqueues ids.

The number of entries in \field{indirection_table} (\field{indirection_table_mask} + 1) MUST be a power of two.

A driver MUST use \field{indirection_table_mask} values that are less than \field{rss_max_indirection_table_length} reported by a device.

A driver MUST NOT set any VIRTIO_NET_HASH_TYPE_ flags that are not supported by a device.

\devicenormative{\subparagraph}{RSS processing}{Device Types / Network Device / Device Operation / Control Virtqueue / Receive-side scaling (RSS) / RSS processing}
The device MUST determine the destination queue for a network packet as follows:
\begin{itemize}
\item Calculate the hash of the packet as defined in \ref{sec:Device Types / Network Device / Device Operation / Processing of Incoming Packets / Hash calculation for incoming packets}.
\item If the device did not calculate the hash for the specific packet, the device directs the packet to the receiveq specified by \field{unclassified_queue} of virtio_net_rss_config structure.
\item Apply \field{indirection_table_mask} to the calculated hash
and use the result as the index in the indirection table to get
the destination receive virtqueue id.
\item If the destination receive queue is being reset (See \ref{sec:Basic Facilities of a Virtio Device / Virtqueues / Virtqueue Reset}), the device MUST drop the packet.
\end{itemize}

\paragraph{RSS Context}\label{sec:Device Types / Network Device / Device Operation / Control Virtqueue / RSS Context}

An RSS context consists of configurable parameters specified by \ref{sec:Device Types / Network Device
/ Device Operation / Control Virtqueue / Receive-side scaling (RSS)}.

The RSS configuration supported by VIRTIO_NET_F_RSS is considered the default RSS configuration.

The device offers the feature VIRTIO_NET_F_RSS_CONTEXT if it supports one or multiple RSS contexts
(excluding the default RSS configuration) and configurable parameters.

\subparagraph{Querying RSS Context Capability}\label{sec:Device Types / Network Device / Device Operation / Control Virtqueue / RSS Context / Querying RSS Context Capability}

\begin{lstlisting}
#define VIRTNET_RSS_CTX_CTRL 9
 #define VIRTNET_RSS_CTX_CTRL_CAP_GET  0
 #define VIRTNET_RSS_CTX_CTRL_ADD      1
 #define VIRTNET_RSS_CTX_CTRL_MOD      2
 #define VIRTNET_RSS_CTX_CTRL_DEL      3

struct virtnet_rss_ctx_cap {
    le16 max_rss_contexts;
}
\end{lstlisting}

Field \field{max_rss_contexts} specifies the maximum number of RSS contexts \ref{sec:Device Types / Network Device /
Device Operation / Control Virtqueue / RSS Context} supported by the device.

The driver queries the RSS context capability of the device by sending the command VIRTNET_RSS_CTX_CTRL_CAP_GET
with the structure virtnet_rss_ctx_cap.

For the command VIRTNET_RSS_CTX_CTRL_CAP_GET, the structure virtnet_rss_ctx_cap is write-only for the device.

\subparagraph{Setting RSS Context Parameters}\label{sec:Device Types / Network Device / Device Operation / Control Virtqueue / RSS Context / Setting RSS Context Parameters}

\begin{lstlisting}
struct virtnet_rss_ctx_add_modify {
    le16 rss_ctx_id;
    u8 reserved[6];
    struct virtio_net_rss_config rss;
};

struct virtnet_rss_ctx_del {
    le16 rss_ctx_id;
};
\end{lstlisting}

RSS context parameters:
\begin{itemize}
\item  \field{rss_ctx_id}: ID of the specific RSS context.
\item  \field{rss}: RSS context parameters of the specific RSS context whose id is \field{rss_ctx_id}.
\end{itemize}

\field{reserved} is reserved and it is ignored by the device.

If the feature VIRTIO_NET_F_RSS_CONTEXT has been negotiated, the driver can send the following
VIRTNET_RSS_CTX_CTRL class commands:
\begin{enumerate}
\item VIRTNET_RSS_CTX_CTRL_ADD: use the structure virtnet_rss_ctx_add_modify to
       add an RSS context configured as \field{rss} and id as \field{rss_ctx_id} for the device.
\item VIRTNET_RSS_CTX_CTRL_MOD: use the structure virtnet_rss_ctx_add_modify to
       configure parameters of the RSS context whose id is \field{rss_ctx_id} as \field{rss} for the device.
\item VIRTNET_RSS_CTX_CTRL_DEL: use the structure virtnet_rss_ctx_del to delete
       the RSS context whose id is \field{rss_ctx_id} for the device.
\end{enumerate}

For commands VIRTNET_RSS_CTX_CTRL_ADD and VIRTNET_RSS_CTX_CTRL_MOD, the structure virtnet_rss_ctx_add_modify is read-only for the device.
For the command VIRTNET_RSS_CTX_CTRL_DEL, the structure virtnet_rss_ctx_del is read-only for the device.

\devicenormative{\subparagraph}{RSS Context}{Device Types / Network Device / Device Operation / Control Virtqueue / RSS Context}

The device MUST set \field{max_rss_contexts} to at least 1 if it offers VIRTIO_NET_F_RSS_CONTEXT.

Upon reset, the device MUST clear all previously configured RSS contexts.

\drivernormative{\subparagraph}{RSS Context}{Device Types / Network Device / Device Operation / Control Virtqueue / RSS Context}

The driver MUST have negotiated the VIRTIO_NET_F_RSS_CONTEXT feature when issuing the VIRTNET_RSS_CTX_CTRL class commands.

The driver MUST set \field{rss_ctx_id} to between 1 and \field{max_rss_contexts} inclusive.

The driver MUST NOT send the command VIRTIO_NET_CTRL_MQ_VQ_PAIRS_SET when the device has successfully configured at least one RSS context.

\paragraph{Offloads State Configuration}\label{sec:Device Types / Network Device / Device Operation / Control Virtqueue / Offloads State Configuration}

If the VIRTIO_NET_F_CTRL_GUEST_OFFLOADS feature is negotiated, the driver can
send control commands for dynamic offloads state configuration.

\subparagraph{Setting Offloads State}\label{sec:Device Types / Network Device / Device Operation / Control Virtqueue / Offloads State Configuration / Setting Offloads State}

To configure the offloads, the following layout structure and
definitions are used:

\begin{lstlisting}
le64 offloads;

#define VIRTIO_NET_F_GUEST_CSUM       1
#define VIRTIO_NET_F_GUEST_TSO4       7
#define VIRTIO_NET_F_GUEST_TSO6       8
#define VIRTIO_NET_F_GUEST_ECN        9
#define VIRTIO_NET_F_GUEST_UFO        10
#define VIRTIO_NET_F_GUEST_UDP_TUNNEL_GSO  46
#define VIRTIO_NET_F_GUEST_UDP_TUNNEL_GSO_CSUM 47
#define VIRTIO_NET_F_GUEST_USO4       54
#define VIRTIO_NET_F_GUEST_USO6       55

#define VIRTIO_NET_CTRL_GUEST_OFFLOADS       5
 #define VIRTIO_NET_CTRL_GUEST_OFFLOADS_SET   0
\end{lstlisting}

The class VIRTIO_NET_CTRL_GUEST_OFFLOADS has one command:
VIRTIO_NET_CTRL_GUEST_OFFLOADS_SET applies the new offloads configuration.

le64 value passed as command data is a bitmask, bits set define
offloads to be enabled, bits cleared - offloads to be disabled.

There is a corresponding device feature for each offload. Upon feature
negotiation corresponding offload gets enabled to preserve backward
compatibility.

\drivernormative{\subparagraph}{Setting Offloads State}{Device Types / Network Device / Device Operation / Control Virtqueue / Offloads State Configuration / Setting Offloads State}

A driver MUST NOT enable an offload for which the appropriate feature
has not been negotiated.

\subparagraph{Legacy Interface: Setting Offloads State}\label{sec:Device Types / Network Device / Device Operation / Control Virtqueue / Offloads State Configuration / Setting Offloads State / Legacy Interface: Setting Offloads State}
When using the legacy interface, transitional devices and drivers
MUST format \field{offloads}
according to the native endian of the guest rather than
(necessarily when not using the legacy interface) little-endian.


\paragraph{Notifications Coalescing}\label{sec:Device Types / Network Device / Device Operation / Control Virtqueue / Notifications Coalescing}

If the VIRTIO_NET_F_NOTF_COAL feature is negotiated, the driver can
send commands VIRTIO_NET_CTRL_NOTF_COAL_TX_SET and VIRTIO_NET_CTRL_NOTF_COAL_RX_SET
for notification coalescing.

If the VIRTIO_NET_F_VQ_NOTF_COAL feature is negotiated, the driver can
send commands VIRTIO_NET_CTRL_NOTF_COAL_VQ_SET and VIRTIO_NET_CTRL_NOTF_COAL_VQ_GET
for virtqueue notification coalescing.

\begin{lstlisting}
struct virtio_net_ctrl_coal {
    le32 max_packets;
    le32 max_usecs;
};

struct virtio_net_ctrl_coal_vq {
    le16 vq_index;
    le16 reserved;
    struct virtio_net_ctrl_coal coal;
};

#define VIRTIO_NET_CTRL_NOTF_COAL 6
 #define VIRTIO_NET_CTRL_NOTF_COAL_TX_SET  0
 #define VIRTIO_NET_CTRL_NOTF_COAL_RX_SET 1
 #define VIRTIO_NET_CTRL_NOTF_COAL_VQ_SET 2
 #define VIRTIO_NET_CTRL_NOTF_COAL_VQ_GET 3
\end{lstlisting}

Coalescing parameters:
\begin{itemize}
\item \field{vq_index}: The virtqueue index of an enabled transmit or receive virtqueue.
\item \field{max_usecs} for RX: Maximum number of microseconds to delay a RX notification.
\item \field{max_usecs} for TX: Maximum number of microseconds to delay a TX notification.
\item \field{max_packets} for RX: Maximum number of packets to receive before a RX notification.
\item \field{max_packets} for TX: Maximum number of packets to send before a TX notification.
\end{itemize}

\field{reserved} is reserved and it is ignored by the device.

Read/Write attributes for coalescing parameters:
\begin{itemize}
\item For commands VIRTIO_NET_CTRL_NOTF_COAL_TX_SET and VIRTIO_NET_CTRL_NOTF_COAL_RX_SET, the structure virtio_net_ctrl_coal is write-only for the driver.
\item For the command VIRTIO_NET_CTRL_NOTF_COAL_VQ_SET, the structure virtio_net_ctrl_coal_vq is write-only for the driver.
\item For the command VIRTIO_NET_CTRL_NOTF_COAL_VQ_GET, \field{vq_index} and \field{reserved} are write-only
      for the driver, and the structure virtio_net_ctrl_coal is read-only for the driver.
\end{itemize}

The class VIRTIO_NET_CTRL_NOTF_COAL has the following commands:
\begin{enumerate}
\item VIRTIO_NET_CTRL_NOTF_COAL_TX_SET: use the structure virtio_net_ctrl_coal to set the \field{max_usecs} and \field{max_packets} parameters for all transmit virtqueues.
\item VIRTIO_NET_CTRL_NOTF_COAL_RX_SET: use the structure virtio_net_ctrl_coal to set the \field{max_usecs} and \field{max_packets} parameters for all receive virtqueues.
\item VIRTIO_NET_CTRL_NOTF_COAL_VQ_SET: use the structure virtio_net_ctrl_coal_vq to set the \field{max_usecs} and \field{max_packets} parameters
                                        for an enabled transmit/receive virtqueue whose index is \field{vq_index}.
\item VIRTIO_NET_CTRL_NOTF_COAL_VQ_GET: use the structure virtio_net_ctrl_coal_vq to get the \field{max_usecs} and \field{max_packets} parameters
                                        for an enabled transmit/receive virtqueue whose index is \field{vq_index}.
\end{enumerate}

The device may generate notifications more or less frequently than specified by set commands of the VIRTIO_NET_CTRL_NOTF_COAL class.

If coalescing parameters are being set, the device applies the last coalescing parameters set for a
virtqueue, regardless of the command used to set the parameters. Use the following command sequence
with two pairs of virtqueues as an example:
Each of the following commands sets \field{max_usecs} and \field{max_packets} parameters for virtqueues.
\begin{itemize}
\item Command1: VIRTIO_NET_CTRL_NOTF_COAL_RX_SET sets coalescing parameters for virtqueues having index 0 and index 2. Virtqueues having index 1 and index 3 retain their previous parameters.
\item Command2: VIRTIO_NET_CTRL_NOTF_COAL_VQ_SET with \field{vq_index} = 0 sets coalescing parameters for virtqueue having index 0. Virtqueue having index 2 retains the parameters from command1.
\item Command3: VIRTIO_NET_CTRL_NOTF_COAL_VQ_GET with \field{vq_index} = 0, the device responds with coalescing parameters of vq_index 0 set by command2.
\item Command4: VIRTIO_NET_CTRL_NOTF_COAL_VQ_SET with \field{vq_index} = 1 sets coalescing parameters for virtqueue having index 1. Virtqueue having index 3 retains its previous parameters.
\item Command5: VIRTIO_NET_CTRL_NOTF_COAL_TX_SET sets coalescing parameters for virtqueues having index 1 and index 3, and overrides the parameters set by command4.
\item Command6: VIRTIO_NET_CTRL_NOTF_COAL_VQ_GET with \field{vq_index} = 1, the device responds with coalescing parameters of index 1 set by command5.
\end{itemize}

\subparagraph{Operation}\label{sec:Device Types / Network Device / Device Operation / Control Virtqueue / Notifications Coalescing / Operation}

The device sends a used buffer notification once the notification conditions are met and if the notifications are not suppressed as explained in \ref{sec:Basic Facilities of a Virtio Device / Virtqueues / Used Buffer Notification Suppression}.

When the device has non-zero \field{max_usecs} and non-zero \field{max_packets}, it starts counting microseconds and packets upon receiving/sending a packet.
The device counts packets and microseconds for each receive virtqueue and transmit virtqueue separately.
In this case, the notification conditions are met when \field{max_usecs} microseconds elapse, or upon sending/receiving \field{max_packets} packets, whichever happens first.
Afterwards, the device waits for the next packet and starts counting packets and microseconds again.

When the device has \field{max_usecs} = 0 or \field{max_packets} = 0, the notification conditions are met after every packet received/sent.

\subparagraph{RX Example}\label{sec:Device Types / Network Device / Device Operation / Control Virtqueue / Notifications Coalescing / RX Example}

If, for example:
\begin{itemize}
\item \field{max_usecs} = 10.
\item \field{max_packets} = 15.
\end{itemize}
then each receive virtqueue of a device will operate as follows:
\begin{itemize}
\item The device will count packets received on each virtqueue until it accumulates 15, or until 10 microseconds elapsed since the first one was received.
\item If the notifications are not suppressed by the driver, the device will send an used buffer notification, otherwise, the device will not send an used buffer notification as long as the notifications are suppressed.
\end{itemize}

\subparagraph{TX Example}\label{sec:Device Types / Network Device / Device Operation / Control Virtqueue / Notifications Coalescing / TX Example}

If, for example:
\begin{itemize}
\item \field{max_usecs} = 10.
\item \field{max_packets} = 15.
\end{itemize}
then each transmit virtqueue of a device will operate as follows:
\begin{itemize}
\item The device will count packets sent on each virtqueue until it accumulates 15, or until 10 microseconds elapsed since the first one was sent.
\item If the notifications are not suppressed by the driver, the device will send an used buffer notification, otherwise, the device will not send an used buffer notification as long as the notifications are suppressed.
\end{itemize}

\subparagraph{Notifications When Coalescing Parameters Change}\label{sec:Device Types / Network Device / Device Operation / Control Virtqueue / Notifications Coalescing / Notifications When Coalescing Parameters Change}

When the coalescing parameters of a device change, the device needs to check if the new notification conditions are met and send a used buffer notification if so.

For example, \field{max_packets} = 15 for a device with a single transmit virtqueue: if the device sends 10 packets and afterwards receives a
VIRTIO_NET_CTRL_NOTF_COAL_TX_SET command with \field{max_packets} = 8, then the notification condition is immediately considered to be met;
the device needs to immediately send a used buffer notification, if the notifications are not suppressed by the driver.

\drivernormative{\subparagraph}{Notifications Coalescing}{Device Types / Network Device / Device Operation / Control Virtqueue / Notifications Coalescing}

The driver MUST set \field{vq_index} to the virtqueue index of an enabled transmit or receive virtqueue.

The driver MUST have negotiated the VIRTIO_NET_F_NOTF_COAL feature when issuing commands VIRTIO_NET_CTRL_NOTF_COAL_TX_SET and VIRTIO_NET_CTRL_NOTF_COAL_RX_SET.

The driver MUST have negotiated the VIRTIO_NET_F_VQ_NOTF_COAL feature when issuing commands VIRTIO_NET_CTRL_NOTF_COAL_VQ_SET and VIRTIO_NET_CTRL_NOTF_COAL_VQ_GET.

The driver MUST ignore the values of coalescing parameters received from the VIRTIO_NET_CTRL_NOTF_COAL_VQ_GET command if the device responds with VIRTIO_NET_ERR.

\devicenormative{\subparagraph}{Notifications Coalescing}{Device Types / Network Device / Device Operation / Control Virtqueue / Notifications Coalescing}

The device MUST ignore \field{reserved}.

The device SHOULD respond to VIRTIO_NET_CTRL_NOTF_COAL_TX_SET and VIRTIO_NET_CTRL_NOTF_COAL_RX_SET commands with VIRTIO_NET_ERR if it was not able to change the parameters.

The device MUST respond to the VIRTIO_NET_CTRL_NOTF_COAL_VQ_SET command with VIRTIO_NET_ERR if it was not able to change the parameters.

The device MUST respond to VIRTIO_NET_CTRL_NOTF_COAL_VQ_SET and VIRTIO_NET_CTRL_NOTF_COAL_VQ_GET commands with
VIRTIO_NET_ERR if the designated virtqueue is not an enabled transmit or receive virtqueue.

Upon disabling and re-enabling a transmit virtqueue, the device MUST set the coalescing parameters of the virtqueue
to those configured through the VIRTIO_NET_CTRL_NOTF_COAL_TX_SET command, or, if the driver did not set any TX coalescing parameters, to 0.

Upon disabling and re-enabling a receive virtqueue, the device MUST set the coalescing parameters of the virtqueue
to those configured through the VIRTIO_NET_CTRL_NOTF_COAL_RX_SET command, or, if the driver did not set any RX coalescing parameters, to 0.

The behavior of the device in response to set commands of the VIRTIO_NET_CTRL_NOTF_COAL class is best-effort:
the device MAY generate notifications more or less frequently than specified.

A device SHOULD NOT send used buffer notifications to the driver if the notifications are suppressed, even if the notification conditions are met.

Upon reset, a device MUST initialize all coalescing parameters to 0.

\paragraph{Device Statistics}\label{sec:Device Types / Network Device / Device Operation / Control Virtqueue / Device Statistics}

If the VIRTIO_NET_F_DEVICE_STATS feature is negotiated, the driver can obtain
device statistics from the device by using the following command.

Different types of virtqueues have different statistics. The statistics of the
receiveq are different from those of the transmitq.

The statistics of a certain type of virtqueue are also divided into multiple types
because different types require different features. This enables the expansion
of new statistics.

In one command, the driver can obtain the statistics of one or multiple virtqueues.
Additionally, the driver can obtain multiple type statistics of each virtqueue.

\subparagraph{Query Statistic Capabilities}\label{sec:Device Types / Network Device / Device Operation / Control Virtqueue / Device Statistics / Query Statistic Capabilities}

\begin{lstlisting}
#define VIRTIO_NET_CTRL_STATS         8
#define VIRTIO_NET_CTRL_STATS_QUERY   0
#define VIRTIO_NET_CTRL_STATS_GET     1

struct virtio_net_stats_capabilities {

#define VIRTIO_NET_STATS_TYPE_CVQ       (1 << 32)

#define VIRTIO_NET_STATS_TYPE_RX_BASIC  (1 << 0)
#define VIRTIO_NET_STATS_TYPE_RX_CSUM   (1 << 1)
#define VIRTIO_NET_STATS_TYPE_RX_GSO    (1 << 2)
#define VIRTIO_NET_STATS_TYPE_RX_SPEED  (1 << 3)

#define VIRTIO_NET_STATS_TYPE_TX_BASIC  (1 << 16)
#define VIRTIO_NET_STATS_TYPE_TX_CSUM   (1 << 17)
#define VIRTIO_NET_STATS_TYPE_TX_GSO    (1 << 18)
#define VIRTIO_NET_STATS_TYPE_TX_SPEED  (1 << 19)

    le64 supported_stats_types[1];
}
\end{lstlisting}

To obtain device statistic capability, use the VIRTIO_NET_CTRL_STATS_QUERY
command. When the command completes successfully, \field{command-specific-result}
is in the format of \field{struct virtio_net_stats_capabilities}.

\subparagraph{Get Statistics}\label{sec:Device Types / Network Device / Device Operation / Control Virtqueue / Device Statistics / Get Statistics}

\begin{lstlisting}
struct virtio_net_ctrl_queue_stats {
       struct {
           le16 vq_index;
           le16 reserved[3];
           le64 types_bitmap[1];
       } stats[];
};

struct virtio_net_stats_reply_hdr {
#define VIRTIO_NET_STATS_TYPE_REPLY_CVQ       32

#define VIRTIO_NET_STATS_TYPE_REPLY_RX_BASIC  0
#define VIRTIO_NET_STATS_TYPE_REPLY_RX_CSUM   1
#define VIRTIO_NET_STATS_TYPE_REPLY_RX_GSO    2
#define VIRTIO_NET_STATS_TYPE_REPLY_RX_SPEED  3

#define VIRTIO_NET_STATS_TYPE_REPLY_TX_BASIC  16
#define VIRTIO_NET_STATS_TYPE_REPLY_TX_CSUM   17
#define VIRTIO_NET_STATS_TYPE_REPLY_TX_GSO    18
#define VIRTIO_NET_STATS_TYPE_REPLY_TX_SPEED  19
    u8 type;
    u8 reserved;
    le16 vq_index;
    le16 reserved1;
    le16 size;
}
\end{lstlisting}

To obtain device statistics, use the VIRTIO_NET_CTRL_STATS_GET command with the
\field{command-specific-data} which is in the format of
\field{struct virtio_net_ctrl_queue_stats}. When the command completes
successfully, \field{command-specific-result} contains multiple statistic
results, each statistic result has the \field{struct virtio_net_stats_reply_hdr}
as the header.

The fields of the \field{struct virtio_net_ctrl_queue_stats}:
\begin{description}
    \item [vq_index]
        The index of the virtqueue to obtain the statistics.

    \item [types_bitmap]
        This is a bitmask of the types of statistics to be obtained. Therefore, a
        \field{stats} inside \field{struct virtio_net_ctrl_queue_stats} may
        indicate multiple statistic replies for the virtqueue.
\end{description}

The fields of the \field{struct virtio_net_stats_reply_hdr}:
\begin{description}
    \item [type]
        The type of the reply statistic.

    \item [vq_index]
        The virtqueue index of the reply statistic.

    \item [size]
        The number of bytes for the statistics entry including size of \field{struct virtio_net_stats_reply_hdr}.

\end{description}

\subparagraph{Controlq Statistics}\label{sec:Device Types / Network Device / Device Operation / Control Virtqueue / Device Statistics / Controlq Statistics}

The structure corresponding to the controlq statistics is
\field{struct virtio_net_stats_cvq}. The corresponding type is
VIRTIO_NET_STATS_TYPE_CVQ. This is for the controlq.

\begin{lstlisting}
struct virtio_net_stats_cvq {
    struct virtio_net_stats_reply_hdr hdr;

    le64 command_num;
    le64 ok_num;
};
\end{lstlisting}

\begin{description}
    \item [command_num]
        The number of commands received by the device including the current command.

    \item [ok_num]
        The number of commands completed successfully by the device including the current command.
\end{description}


\subparagraph{Receiveq Basic Statistics}\label{sec:Device Types / Network Device / Device Operation / Control Virtqueue / Device Statistics / Receiveq Basic Statistics}

The structure corresponding to the receiveq basic statistics is
\field{struct virtio_net_stats_rx_basic}. The corresponding type is
VIRTIO_NET_STATS_TYPE_RX_BASIC. This is for the receiveq.

Receiveq basic statistics do not require any feature. As long as the device supports
VIRTIO_NET_F_DEVICE_STATS, the following are the receiveq basic statistics.

\begin{lstlisting}
struct virtio_net_stats_rx_basic {
    struct virtio_net_stats_reply_hdr hdr;

    le64 rx_notifications;

    le64 rx_packets;
    le64 rx_bytes;

    le64 rx_interrupts;

    le64 rx_drops;
    le64 rx_drop_overruns;
};
\end{lstlisting}

The packets described below were all presented on the specified virtqueue.
\begin{description}
    \item [rx_notifications]
        The number of driver notifications received by the device for this
        receiveq.

    \item [rx_packets]
        This is the number of packets passed to the driver by the device.

    \item [rx_bytes]
        This is the bytes of packets passed to the driver by the device.

    \item [rx_interrupts]
        The number of interrupts generated by the device for this receiveq.

    \item [rx_drops]
        This is the number of packets dropped by the device. The count includes
        all types of packets dropped by the device.

    \item [rx_drop_overruns]
        This is the number of packets dropped by the device when no more
        descriptors were available.

\end{description}

\subparagraph{Transmitq Basic Statistics}\label{sec:Device Types / Network Device / Device Operation / Control Virtqueue / Device Statistics / Transmitq Basic Statistics}

The structure corresponding to the transmitq basic statistics is
\field{struct virtio_net_stats_tx_basic}. The corresponding type is
VIRTIO_NET_STATS_TYPE_TX_BASIC. This is for the transmitq.

Transmitq basic statistics do not require any feature. As long as the device supports
VIRTIO_NET_F_DEVICE_STATS, the following are the transmitq basic statistics.

\begin{lstlisting}
struct virtio_net_stats_tx_basic {
    struct virtio_net_stats_reply_hdr hdr;

    le64 tx_notifications;

    le64 tx_packets;
    le64 tx_bytes;

    le64 tx_interrupts;

    le64 tx_drops;
    le64 tx_drop_malformed;
};
\end{lstlisting}

The packets described below are all for a specific virtqueue.
\begin{description}
    \item [tx_notifications]
        The number of driver notifications received by the device for this
        transmitq.

    \item [tx_packets]
        This is the number of packets sent by the device (not the packets
        got from the driver).

    \item [tx_bytes]
        This is the number of bytes sent by the device for all the sent packets
        (not the bytes sent got from the driver).

    \item [tx_interrupts]
        The number of interrupts generated by the device for this transmitq.

    \item [tx_drops]
        The number of packets dropped by the device. The count includes all
        types of packets dropped by the device.

    \item [tx_drop_malformed]
        The number of packets dropped by the device, when the descriptors are
        malformed. For example, the buffer is too short.
\end{description}

\subparagraph{Receiveq CSUM Statistics}\label{sec:Device Types / Network Device / Device Operation / Control Virtqueue / Device Statistics / Receiveq CSUM Statistics}

The structure corresponding to the receiveq checksum statistics is
\field{struct virtio_net_stats_rx_csum}. The corresponding type is
VIRTIO_NET_STATS_TYPE_RX_CSUM. This is for the receiveq.

Only after the VIRTIO_NET_F_GUEST_CSUM is negotiated, the receiveq checksum
statistics can be obtained.

\begin{lstlisting}
struct virtio_net_stats_rx_csum {
    struct virtio_net_stats_reply_hdr hdr;

    le64 rx_csum_valid;
    le64 rx_needs_csum;
    le64 rx_csum_none;
    le64 rx_csum_bad;
};
\end{lstlisting}

The packets described below were all presented on the specified virtqueue.
\begin{description}
    \item [rx_csum_valid]
        The number of packets with VIRTIO_NET_HDR_F_DATA_VALID.

    \item [rx_needs_csum]
        The number of packets with VIRTIO_NET_HDR_F_NEEDS_CSUM.

    \item [rx_csum_none]
        The number of packets without hardware checksum. The packet here refers
        to the non-TCP/UDP packet that the device cannot recognize.

    \item [rx_csum_bad]
        The number of packets with checksum mismatch.

\end{description}

\subparagraph{Transmitq CSUM Statistics}\label{sec:Device Types / Network Device / Device Operation / Control Virtqueue / Device Statistics / Transmitq CSUM Statistics}

The structure corresponding to the transmitq checksum statistics is
\field{struct virtio_net_stats_tx_csum}. The corresponding type is
VIRTIO_NET_STATS_TYPE_TX_CSUM. This is for the transmitq.

Only after the VIRTIO_NET_F_CSUM is negotiated, the transmitq checksum
statistics can be obtained.

The following are the transmitq checksum statistics:

\begin{lstlisting}
struct virtio_net_stats_tx_csum {
    struct virtio_net_stats_reply_hdr hdr;

    le64 tx_csum_none;
    le64 tx_needs_csum;
};
\end{lstlisting}

The packets described below are all for a specific virtqueue.
\begin{description}
    \item [tx_csum_none]
        The number of packets which do not require hardware checksum.

    \item [tx_needs_csum]
        The number of packets which require checksum calculation by the device.

\end{description}

\subparagraph{Receiveq GSO Statistics}\label{sec:Device Types / Network Device / Device Operation / Control Virtqueue / Device Statistics / Receiveq GSO Statistics}

The structure corresponding to the receivq GSO statistics is
\field{struct virtio_net_stats_rx_gso}. The corresponding type is
VIRTIO_NET_STATS_TYPE_RX_GSO. This is for the receiveq.

If one or more of the VIRTIO_NET_F_GUEST_TSO4, VIRTIO_NET_F_GUEST_TSO6
have been negotiated, the receiveq GSO statistics can be obtained.

GSO packets refer to packets passed by the device to the driver where
\field{gso_type} is not VIRTIO_NET_HDR_GSO_NONE.

\begin{lstlisting}
struct virtio_net_stats_rx_gso {
    struct virtio_net_stats_reply_hdr hdr;

    le64 rx_gso_packets;
    le64 rx_gso_bytes;
    le64 rx_gso_packets_coalesced;
    le64 rx_gso_bytes_coalesced;
};
\end{lstlisting}

The packets described below were all presented on the specified virtqueue.
\begin{description}
    \item [rx_gso_packets]
        The number of the GSO packets received by the device.

    \item [rx_gso_bytes]
        The bytes of the GSO packets received by the device.
        This includes the header size of the GSO packet.

    \item [rx_gso_packets_coalesced]
        The number of the GSO packets coalesced by the device.

    \item [rx_gso_bytes_coalesced]
        The bytes of the GSO packets coalesced by the device.
        This includes the header size of the GSO packet.
\end{description}

\subparagraph{Transmitq GSO Statistics}\label{sec:Device Types / Network Device / Device Operation / Control Virtqueue / Device Statistics / Transmitq GSO Statistics}

The structure corresponding to the transmitq GSO statistics is
\field{struct virtio_net_stats_tx_gso}. The corresponding type is
VIRTIO_NET_STATS_TYPE_TX_GSO. This is for the transmitq.

If one or more of the VIRTIO_NET_F_HOST_TSO4, VIRTIO_NET_F_HOST_TSO6,
VIRTIO_NET_F_HOST_USO options have been negotiated, the transmitq GSO statistics
can be obtained.

GSO packets refer to packets passed by the driver to the device where
\field{gso_type} is not VIRTIO_NET_HDR_GSO_NONE.
See more \ref{sec:Device Types / Network Device / Device Operation / Packet
Transmission}.

\begin{lstlisting}
struct virtio_net_stats_tx_gso {
    struct virtio_net_stats_reply_hdr hdr;

    le64 tx_gso_packets;
    le64 tx_gso_bytes;
    le64 tx_gso_segments;
    le64 tx_gso_segments_bytes;
    le64 tx_gso_packets_noseg;
    le64 tx_gso_bytes_noseg;
};
\end{lstlisting}

The packets described below are all for a specific virtqueue.
\begin{description}
    \item [tx_gso_packets]
        The number of the GSO packets sent by the device.

    \item [tx_gso_bytes]
        The bytes of the GSO packets sent by the device.

    \item [tx_gso_segments]
        The number of segments prepared from GSO packets.

    \item [tx_gso_segments_bytes]
        The bytes of segments prepared from GSO packets.

    \item [tx_gso_packets_noseg]
        The number of the GSO packets without segmentation.

    \item [tx_gso_bytes_noseg]
        The bytes of the GSO packets without segmentation.

\end{description}

\subparagraph{Receiveq Speed Statistics}\label{sec:Device Types / Network Device / Device Operation / Control Virtqueue / Device Statistics / Receiveq Speed Statistics}

The structure corresponding to the receiveq speed statistics is
\field{struct virtio_net_stats_rx_speed}. The corresponding type is
VIRTIO_NET_STATS_TYPE_RX_SPEED. This is for the receiveq.

The device has the allowance for the speed. If VIRTIO_NET_F_SPEED_DUPLEX has
been negotiated, the driver can get this by \field{speed}. When the received
packets bitrate exceeds the \field{speed}, some packets may be dropped by the
device.

\begin{lstlisting}
struct virtio_net_stats_rx_speed {
    struct virtio_net_stats_reply_hdr hdr;

    le64 rx_packets_allowance_exceeded;
    le64 rx_bytes_allowance_exceeded;
};
\end{lstlisting}

The packets described below were all presented on the specified virtqueue.
\begin{description}
    \item [rx_packets_allowance_exceeded]
        The number of the packets dropped by the device due to the received
        packets bitrate exceeding the \field{speed}.

    \item [rx_bytes_allowance_exceeded]
        The bytes of the packets dropped by the device due to the received
        packets bitrate exceeding the \field{speed}.

\end{description}

\subparagraph{Transmitq Speed Statistics}\label{sec:Device Types / Network Device / Device Operation / Control Virtqueue / Device Statistics / Transmitq Speed Statistics}

The structure corresponding to the transmitq speed statistics is
\field{struct virtio_net_stats_tx_speed}. The corresponding type is
VIRTIO_NET_STATS_TYPE_TX_SPEED. This is for the transmitq.

The device has the allowance for the speed. If VIRTIO_NET_F_SPEED_DUPLEX has
been negotiated, the driver can get this by \field{speed}. When the transmit
packets bitrate exceeds the \field{speed}, some packets may be dropped by the
device.

\begin{lstlisting}
struct virtio_net_stats_tx_speed {
    struct virtio_net_stats_reply_hdr hdr;

    le64 tx_packets_allowance_exceeded;
    le64 tx_bytes_allowance_exceeded;
};
\end{lstlisting}

The packets described below were all presented on the specified virtqueue.
\begin{description}
    \item [tx_packets_allowance_exceeded]
        The number of the packets dropped by the device due to the transmit packets
        bitrate exceeding the \field{speed}.

    \item [tx_bytes_allowance_exceeded]
        The bytes of the packets dropped by the device due to the transmit packets
        bitrate exceeding the \field{speed}.

\end{description}

\devicenormative{\subparagraph}{Device Statistics}{Device Types / Network Device / Device Operation / Control Virtqueue / Device Statistics}

When the VIRTIO_NET_F_DEVICE_STATS feature is negotiated, the device MUST reply
to the command VIRTIO_NET_CTRL_STATS_QUERY with the
\field{struct virtio_net_stats_capabilities}. \field{supported_stats_types}
includes all the statistic types supported by the device.

If \field{struct virtio_net_ctrl_queue_stats} is incorrect (such as the
following), the device MUST set \field{ack} to VIRTIO_NET_ERR. Even if there is
only one error, the device MUST fail the entire command.
\begin{itemize}
    \item \field{vq_index} exceeds the queue range.
    \item \field{types_bitmap} contains unknown types.
    \item One or more of the bits present in \field{types_bitmap} is not valid
        for the specified virtqueue.
    \item The feature corresponding to the specified \field{types_bitmap} was
        not negotiated.
\end{itemize}

The device MUST set the actual size of the bytes occupied by the reply to the
\field{size} of the \field{hdr}. And the device MUST set the \field{type} and
the \field{vq_index} of the statistic header.

The \field{command-specific-result} buffer allocated by the driver may be
smaller or bigger than all the statistics specified by
\field{struct virtio_net_ctrl_queue_stats}. The device MUST fill up only upto
the valid bytes.

The statistics counter replied by the device MUST wrap around to zero by the
device on the overflow.

\drivernormative{\subparagraph}{Device Statistics}{Device Types / Network Device / Device Operation / Control Virtqueue / Device Statistics}

The types contained in the \field{types_bitmap} MUST be queried from the device
via command VIRTIO_NET_CTRL_STATS_QUERY.

\field{types_bitmap} in \field{struct virtio_net_ctrl_queue_stats} MUST be valid to the
vq specified by \field{vq_index}.

The \field{command-specific-result} buffer allocated by the driver MUST have
enough capacity to store all the statistics reply headers defined in
\field{struct virtio_net_ctrl_queue_stats}. If the
\field{command-specific-result} buffer is fully utilized by the device but some
replies are missed, it is possible that some statistics may exceed the capacity
of the driver's records. In such cases, the driver should allocate additional
space for the \field{command-specific-result} buffer.

\subsubsection{Flow filter}\label{sec:Device Types / Network Device / Device Operation / Flow filter}

A network device can support one or more flow filter rules. Each flow filter rule
is applied by matching a packet and then taking an action, such as directing the packet
to a specific receiveq or dropping the packet. An example of a match is
matching on specific source and destination IP addresses.

A flow filter rule is a device resource object that consists of a key,
a processing priority, and an action to either direct a packet to a
receive queue or drop the packet.

Each rule uses a classifier. The key is matched against the packet using
a classifier, defining which fields in the packet are matched.
A classifier resource object consists of one or more field selectors, each with
a type that specifies the header fields to be matched against, and a mask.
The mask can match whole fields or parts of a field in a header. Each
rule resource object depends on the classifier resource object.

When a packet is received, relevant fields are extracted
(in the same way) from both the packet and the key according to the
classifier. The resulting field contents are then compared -
if they are identical the rule action is taken, if they are not, the rule is ignored.

Multiple flow filter rules are part of a group. The rule resource object
depends on the group. Each rule within a
group has a rule priority, and each group also has a group priority. For a
packet, a group with the highest priority is selected first. Within a group,
rules are applied from highest to lowest priority, until one of the rules
matches the packet and an action is taken. If all the rules within a group
are ignored, the group with the next highest priority is selected, and so on.

The device and the driver indicates flow filter resource limits using the capability
\ref{par:Device Types / Network Device / Device Operation / Flow filter / Device and driver capabilities / VIRTIO-NET-FF-RESOURCE-CAP} specifying the limits on the number of flow filter rule,
group and classifier resource objects. The capability \ref{par:Device Types / Network Device / Device Operation / Flow filter / Device and driver capabilities / VIRTIO-NET-FF-SELECTOR-CAP} specifies which selectors the device supports.
The driver indicates the selectors it is using by setting the flow
filter selector capability, prior to adding any resource objects.

The capability \ref{par:Device Types / Network Device / Device Operation / Flow filter / Device and driver capabilities / VIRTIO-NET-FF-ACTION-CAP} specifies which actions the device supports.

The driver controls the flow filter rule, classifier and group resource objects using
administration commands described in
\ref{sec:Basic Facilities of a Virtio Device / Device groups / Group administration commands / Device resource objects}.

\paragraph{Packet processing order}\label{sec:sec:Device Types / Network Device / Device Operation / Flow filter / Packet processing order}

Note that flow filter rules are applied after MAC/VLAN filtering. Flow filter
rules take precedence over steering: if a flow filter rule results in an action,
the steering configuration does not apply. The steering configuration only applies
to packets for which no flow filter rule action was performed. For example,
incoming packets can be processed in the following order:

\begin{itemize}
\item apply steering configuration received using control virtqueue commands
      VIRTIO_NET_CTRL_RX, VIRTIO_NET_CTRL_MAC and VIRTIO_NET_CTRL_VLAN.
\item apply flow filter rules if any.
\item if no filter rule applied, apply steering configuration received using command
      VIRTIO_NET_CTRL_MQ_RSS_CONFIG or as per automatic receive steering.
\end{itemize}

Some incoming packet processing examples:
\begin{itemize}
\item If the packet is dropped by the flow filter rule, RSS
      steering is ignored for the packet.
\item If the packet is directed to a specific receiveq using flow filter rule,
      the RSS steering is ignored for the packet.
\item If a packet is dropped due to the VIRTIO_NET_CTRL_MAC configuration,
      both flow filter rules and the RSS steering are ignored for the packet.
\item If a packet does not match any flow filter rules,
      the RSS steering is used to select the receiveq for the packet (if enabled).
\item If there are two flow filter groups configured as group_A and group_B
      with respective group priorities as 4, and 5; flow filter rules of
      group_B are applied first having highest group priority, if there is a match,
      the flow filter rules of group_A are ignored; if there is no match for
      the flow filter rules in group_B, the flow filter rules of next level group_A are applied.
\end{itemize}

\paragraph{Device and driver capabilities}
\label{par:Device Types / Network Device / Device Operation / Flow filter / Device and driver capabilities}

\subparagraph{VIRTIO_NET_FF_RESOURCE_CAP}
\label{par:Device Types / Network Device / Device Operation / Flow filter / Device and driver capabilities / VIRTIO-NET-FF-RESOURCE-CAP}

The capability VIRTIO_NET_FF_RESOURCE_CAP indicates the flow filter resource limits.
\field{cap_specific_data} is in the format
\field{struct virtio_net_ff_cap_data}.

\begin{lstlisting}
struct virtio_net_ff_cap_data {
        le32 groups_limit;
        le32 selectors_limit;
        le32 rules_limit;
        le32 rules_per_group_limit;
        u8 last_rule_priority;
        u8 selectors_per_classifier_limit;
};
\end{lstlisting}

\field{groups_limit}, and \field{selectors_limit} represent the maximum
number of flow filter groups and selectors, respectively, that the driver can create.
 \field{rules_limit} is the maximum number of
flow fiilter rules that the driver can create across all the groups.
\field{rules_per_group_limit} is the maximum number of flow filter rules that the driver
can create for each flow filter group.

\field{last_rule_priority} is the highest priority that can be assigned to a
flow filter rule.

\field{selectors_per_classifier_limit} is the maximum number of selectors
that a classifier can have.

\subparagraph{VIRTIO_NET_FF_SELECTOR_CAP}
\label{par:Device Types / Network Device / Device Operation / Flow filter / Device and driver capabilities / VIRTIO-NET-FF-SELECTOR-CAP}

The capability VIRTIO_NET_FF_SELECTOR_CAP lists the supported selectors and the
supported packet header fields for each selector.
\field{cap_specific_data} is in the format \field{struct virtio_net_ff_cap_mask_data}.

\begin{lstlisting}[label={lst:Device Types / Network Device / Device Operation / Flow filter / Device and driver capabilities / VIRTIO-NET-FF-SELECTOR-CAP / virtio-net-ff-selector}]
struct virtio_net_ff_selector {
        u8 type;
        u8 flags;
        u8 reserved[2];
        u8 length;
        u8 reserved1[3];
        u8 mask[];
};

struct virtio_net_ff_cap_mask_data {
        u8 count;
        u8 reserved[7];
        struct virtio_net_ff_selector selectors[];
};

#define VIRTIO_NET_FF_MASK_F_PARTIAL_MASK (1 << 0)
\end{lstlisting}

\field{count} indicates number of valid entries in the \field{selectors} array.
\field{selectors[]} is an array of supported selectors. Within each array entry:
\field{type} specifies the type of the packet header, as defined in table
\ref{table:Device Types / Network Device / Device Operation / Flow filter / Device and driver capabilities / VIRTIO-NET-FF-SELECTOR-CAP / flow filter selector types}. \field{mask} specifies which fields of the
packet header can be matched in a flow filter rule.

Each \field{type} is also listed in table
\ref{table:Device Types / Network Device / Device Operation / Flow filter / Device and driver capabilities / VIRTIO-NET-FF-SELECTOR-CAP / flow filter selector types}. \field{mask} is a byte array
in network byte order. For example, when \field{type} is VIRTIO_NET_FF_MASK_TYPE_IPV6,
the \field{mask} is in the format \hyperref[intro:IPv6-Header-Format]{IPv6 Header Format}.

If partial masking is not set, then all bits in each field have to be either all 0s
to ignore this field or all 1s to match on this field. If partial masking is set,
then any combination of bits can bit set to match on these bits.
For example, when a selector \field{type} is VIRTIO_NET_FF_MASK_TYPE_ETH, if
\field{mask[0-12]} are zero and \field{mask[13-14]} are 0xff (all 1s), it
indicates that matching is only supported for \field{EtherType} of
\field{Ethernet MAC frame}, matching is not supported for
\field{Destination Address} and \field{Source Address}.

The entries in the array \field{selectors} are ordered by
\field{type}, with each \field{type} value only appearing once.

\field{length} is the length of a dynamic array \field{mask} in bytes.
\field{reserved} and \field{reserved1} are reserved and set to zero.

\begin{table}[H]
\caption{Flow filter selector types}
\label{table:Device Types / Network Device / Device Operation / Flow filter / Device and driver capabilities / VIRTIO-NET-FF-SELECTOR-CAP / flow filter selector types}
\begin{tabularx}{\textwidth}{ |l|X|X| }
\hline
Type & Name & Description \\
\hline \hline
0x0 & - & Reserved \\
\hline
0x1 & VIRTIO_NET_FF_MASK_TYPE_ETH & 14 bytes of frame header starting from destination address described in \hyperref[intro:IEEE 802.3-2022]{IEEE 802.3-2022} \\
\hline
0x2 & VIRTIO_NET_FF_MASK_TYPE_IPV4 & 20 bytes of \hyperref[intro:Internet-Header-Format]{IPv4: Internet Header Format} \\
\hline
0x3 & VIRTIO_NET_FF_MASK_TYPE_IPV6 & 40 bytes of \hyperref[intro:IPv6-Header-Format]{IPv6 Header Format} \\
\hline
0x4 & VIRTIO_NET_FF_MASK_TYPE_TCP & 20 bytes of \hyperref[intro:TCP-Header-Format]{TCP Header Format} \\
\hline
0x5 & VIRTIO_NET_FF_MASK_TYPE_UDP & 8 bytes of UDP header described in \hyperref[intro:UDP]{UDP} \\
\hline
0x6 - 0xFF & & Reserved for future \\
\hline
\end{tabularx}
\end{table}

When VIRTIO_NET_FF_MASK_F_PARTIAL_MASK (bit 0) is set, it indicates that
partial masking is supported for all the fields of the selector identified by \field{type}.

For the selector \field{type} VIRTIO_NET_FF_MASK_TYPE_IPV4, if a partial mask is unsupported,
then matching on an individual bit of \field{Flags} in the
\field{IPv4: Internet Header Format} is unsupported. \field{Flags} has to match as a whole
if it is supported.

For the selector \field{type} VIRTIO_NET_FF_MASK_TYPE_IPV4, \field{mask} includes fields
up to the \field{Destination Address}; that is, \field{Options} and
\field{Padding} are excluded.

For the selector \field{type} VIRTIO_NET_FF_MASK_TYPE_IPV6, the \field{Next Header} field
of the \field{mask} corresponds to the \field{Next Header} in the packet
when \field{IPv6 Extension Headers} are not present. When the packet includes
one or more \field{IPv6 Extension Headers}, the \field{Next Header} field of
the \field{mask} corresponds to the \field{Next Header} of the last
\field{IPv6 Extension Header} in the packet.

For the selector \field{type} VIRTIO_NET_FF_MASK_TYPE_TCP, \field{Control bits}
are treated as individual fields for matching; that is, matching individual
\field{Control bits} does not depend on the partial mask support.

\subparagraph{VIRTIO_NET_FF_ACTION_CAP}
\label{par:Device Types / Network Device / Device Operation / Flow filter / Device and driver capabilities / VIRTIO-NET-FF-ACTION-CAP}

The capability VIRTIO_NET_FF_ACTION_CAP lists the supported actions in a rule.
\field{cap_specific_data} is in the format \field{struct virtio_net_ff_cap_actions}.

\begin{lstlisting}
struct virtio_net_ff_actions {
        u8 count;
        u8 reserved[7];
        u8 actions[];
};
\end{lstlisting}

\field{actions} is an array listing all possible actions.
The entries in the array are ordered from the smallest to the largest,
with each supported value appearing exactly once. Each entry can have the
following values:

\begin{table}[H]
\caption{Flow filter rule actions}
\label{table:Device Types / Network Device / Device Operation / Flow filter / Device and driver capabilities / VIRTIO-NET-FF-ACTION-CAP / flow filter rule actions}
\begin{tabularx}{\textwidth}{ |l|X|X| }
\hline
Action & Name & Description \\
\hline \hline
0x0 & - & reserved \\
\hline
0x1 & VIRTIO_NET_FF_ACTION_DROP & Matching packet will be dropped by the device \\
\hline
0x2 & VIRTIO_NET_FF_ACTION_DIRECT_RX_VQ & Matching packet will be directed to a receive queue \\
\hline
0x3 - 0xFF & & Reserved for future \\
\hline
\end{tabularx}
\end{table}

\paragraph{Resource objects}
\label{par:Device Types / Network Device / Device Operation / Flow filter / Resource objects}

\subparagraph{VIRTIO_NET_RESOURCE_OBJ_FF_GROUP}\label{par:Device Types / Network Device / Device Operation / Flow filter / Resource objects / VIRTIO-NET-RESOURCE-OBJ-FF-GROUP}

A flow filter group contains between 0 and \field{rules_limit} rules, as specified by the
capability VIRTIO_NET_FF_RESOURCE_CAP. For the flow filter group object both
\field{resource_obj_specific_data} and
\field{resource_obj_specific_result} are in the format
\field{struct virtio_net_resource_obj_ff_group}.

\begin{lstlisting}
struct virtio_net_resource_obj_ff_group {
        le16 group_priority;
};
\end{lstlisting}

\field{group_priority} specifies the priority for the group. Each group has a
distinct priority. For each incoming packet, the device tries to apply rules
from groups from higher \field{group_priority} value to lower, until either a
rule matches the packet or all groups have been tried.

\subparagraph{VIRTIO_NET_RESOURCE_OBJ_FF_CLASSIFIER}\label{par:Device Types / Network Device / Device Operation / Flow filter / Resource objects / VIRTIO-NET-RESOURCE-OBJ-FF-CLASSIFIER}

A classifier is used to match a flow filter key against a packet. The
classifier defines the desired packet fields to match, and is represented by
the VIRTIO_NET_RESOURCE_OBJ_FF_CLASSIFIER device resource object.

For the flow filter classifier object both \field{resource_obj_specific_data} and
\field{resource_obj_specific_result} are in the format
\field{struct virtio_net_resource_obj_ff_classifier}.

\begin{lstlisting}
struct virtio_net_resource_obj_ff_classifier {
        u8 count;
        u8 reserved[7];
        struct virtio_net_ff_selector selectors[];
};
\end{lstlisting}

A classifier is an array of \field{selectors}. The number of selectors in the
array is indicated by \field{count}. The selector has a type that specifies
the header fields to be matched against, and a mask.
See \ref{lst:Device Types / Network Device / Device Operation / Flow filter / Device and driver capabilities / VIRTIO-NET-FF-SELECTOR-CAP / virtio-net-ff-selector}
for details about selectors.

The first selector is always VIRTIO_NET_FF_MASK_TYPE_ETH. When there are multiple
selectors, a second selector can be either VIRTIO_NET_FF_MASK_TYPE_IPV4
or VIRTIO_NET_FF_MASK_TYPE_IPV6. If the third selector exists, the third
selector can be either VIRTIO_NET_FF_MASK_TYPE_UDP or VIRTIO_NET_FF_MASK_TYPE_TCP.
For example, to match a Ethernet IPv6 UDP packet,
\field{selectors[0].type} is set to VIRTIO_NET_FF_MASK_TYPE_ETH, \field{selectors[1].type}
is set to VIRTIO_NET_FF_MASK_TYPE_IPV6 and \field{selectors[2].type} is
set to VIRTIO_NET_FF_MASK_TYPE_UDP; accordingly, \field{selectors[0].mask[0-13]} is
for Ethernet header fields, \field{selectors[1].mask[0-39]} is set for IPV6 header
and \field{selectors[2].mask[0-7]} is set for UDP header.

When there are multiple selectors, the type of the (N+1)\textsuperscript{th} selector
affects the mask of the (N)\textsuperscript{th} selector. If
\field{count} is 2 or more, all the mask bits within \field{selectors[0]}
corresponding to \field{EtherType} of an Ethernet header are set.

If \field{count} is more than 2:
\begin{itemize}
\item if \field{selector[1].type} is, VIRTIO_NET_FF_MASK_TYPE_IPV4, then, all the mask bits within
\field{selector[1]} for \field{Protocol} is set.
\item if \field{selector[1].type} is, VIRTIO_NET_FF_MASK_TYPE_IPV6, then, all the mask bits within
\field{selector[1]} for \field{Next Header} is set.
\end{itemize}

If for a given packet header field, a subset of bits of a field is to be matched,
and if the partial mask is supported, the flow filter
mask object can specify a mask which has fewer bits set than the packet header
field size. For example, a partial mask for the Ethernet header source mac
address can be of 1-bit for multicast detection instead of 48-bits.

\subparagraph{VIRTIO_NET_RESOURCE_OBJ_FF_RULE}\label{par:Device Types / Network Device / Device Operation / Flow filter / Resource objects / VIRTIO-NET-RESOURCE-OBJ-FF-RULE}

Each flow filter rule resource object comprises a key, a priority, and an action.
For the flow filter rule object,
\field{resource_obj_specific_data} and
\field{resource_obj_specific_result} are in the format
\field{struct virtio_net_resource_obj_ff_rule}.

\begin{lstlisting}
struct virtio_net_resource_obj_ff_rule {
        le32 group_id;
        le32 classifier_id;
        u8 rule_priority;
        u8 key_length; /* length of key in bytes */
        u8 action;
        u8 reserved;
        le16 vq_index;
        u8 reserved1[2];
        u8 keys[][];
};
\end{lstlisting}

\field{group_id} is the resource object ID of the flow filter group to which
this rule belongs. \field{classifier_id} is the resource object ID of the
classifier used to match a packet against the \field{key}.

\field{rule_priority} denotes the priority of the rule within the group
specified by the \field{group_id}.
Rules within the group are applied from the highest to the lowest priority
until a rule matches the packet and an
action is taken. Rules with the same priority can be applied in any order.

\field{reserved} and \field{reserved1} are reserved and set to 0.

\field{keys[][]} is an array of keys to match against packets, using
the classifier specified by \field{classifier_id}. Each entry (key) comprises
a byte array, and they are located one immediately after another.
The size (number of entries) of the array is exactly the same as that of
\field{selectors} in the classifier, or in other words, \field{count}
in the classifier.

\field{key_length} specifies the total length of \field{keys} in bytes.
In other words, it equals the sum total of \field{length} of all
selectors in \field{selectors} in the classifier specified by
\field{classifier_id}.

For example, if a classifier object's \field{selectors[0].type} is
VIRTIO_NET_FF_MASK_TYPE_ETH and \field{selectors[1].type} is
VIRTIO_NET_FF_MASK_TYPE_IPV6,
then selectors[0].length is 14 and selectors[1].length is 40.
Accordingly, the \field{key_length} is set to 54.
This setting indicates that the \field{key} array's length is 54 bytes
comprising a first byte array of 14 bytes for the
Ethernet MAC header in bytes 0-13, immediately followed by 40 bytes for the
IPv6 header in bytes 14-53.

When there are multiple selectors in the classifier object, the key bytes
for (N)\textsuperscript{th} selector are set so that
(N+1)\textsuperscript{th} selector can be matched.

If \field{count} is 2 or more, key bytes of \field{EtherType}
are set according to \hyperref[intro:IEEE 802 Ethertypes]{IEEE 802 Ethertypes}
for VIRTIO_NET_FF_MASK_TYPE_IPV4 or VIRTIO_NET_FF_MASK_TYPE_IPV6 respectively.

If \field{count} is more than 2, when \field{selector[1].type} is
VIRTIO_NET_FF_MASK_TYPE_IPV4 or VIRTIO_NET_FF_MASK_TYPE_IPV6, key
bytes of \field{Protocol} or \field{Next Header} is set as per
\field{Protocol Numbers} defined \hyperref[intro:IANA Protocol Numbers]{IANA Protocol Numbers}
respectively.

\field{action} is the action to take when a packet matches the
\field{key} using the \field{classifier_id}. Supported actions are described in
\ref{table:Device Types / Network Device / Device Operation / Flow filter / Device and driver capabilities / VIRTIO-NET-FF-ACTION-CAP / flow filter rule actions}.

\field{vq_index} specifies a receive virtqueue. When the \field{action} is set
to VIRTIO_NET_FF_ACTION_DIRECT_RX_VQ, and the packet matches the \field{key},
the matching packet is directed to this virtqueue.

Note that at most one action is ever taken for a given packet. If a rule is
applied and an action is taken, the action of other rules is not taken.

\devicenormative{\paragraph}{Flow filter}{Device Types / Network Device / Device Operation / Flow filter}

When the device supports flow filter operations,
\begin{itemize}
\item the device MUST set VIRTIO_NET_FF_RESOURCE_CAP, VIRTIO_NET_FF_SELECTOR_CAP
and VIRTIO_NET_FF_ACTION_CAP capability in the \field{supported_caps} in the
command VIRTIO_ADMIN_CMD_CAP_SUPPORT_QUERY.
\item the device MUST support the administration commands
VIRTIO_ADMIN_CMD_RESOURCE_OBJ_CREATE,
VIRTIO_ADMIN_CMD_RESOURCE_OBJ_MODIFY, VIRTIO_ADMIN_CMD_RESOURCE_OBJ_QUERY,
VIRTIO_ADMIN_CMD_RESOURCE_OBJ_DESTROY for the resource types
VIRTIO_NET_RESOURCE_OBJ_FF_GROUP, VIRTIO_NET_RESOURCE_OBJ_FF_CLASSIFIER and
VIRTIO_NET_RESOURCE_OBJ_FF_RULE.
\end{itemize}

When any of the VIRTIO_NET_FF_RESOURCE_CAP, VIRTIO_NET_FF_SELECTOR_CAP, or
VIRTIO_NET_FF_ACTION_CAP capability is disabled, the device SHOULD set
\field{status} to VIRTIO_ADMIN_STATUS_Q_INVALID_OPCODE for the commands
VIRTIO_ADMIN_CMD_RESOURCE_OBJ_CREATE,
VIRTIO_ADMIN_CMD_RESOURCE_OBJ_MODIFY, VIRTIO_ADMIN_CMD_RESOURCE_OBJ_QUERY,
and VIRTIO_ADMIN_CMD_RESOURCE_OBJ_DESTROY. These commands apply to the resource
\field{type} of VIRTIO_NET_RESOURCE_OBJ_FF_GROUP, VIRTIO_NET_RESOURCE_OBJ_FF_CLASSIFIER, and
VIRTIO_NET_RESOURCE_OBJ_FF_RULE.

The device SHOULD set \field{status} to VIRTIO_ADMIN_STATUS_EINVAL for the
command VIRTIO_ADMIN_CMD_RESOURCE_OBJ_CREATE when the resource \field{type}
is VIRTIO_NET_RESOURCE_OBJ_FF_GROUP, if a flow filter group already exists
with the supplied \field{group_priority}.

The device SHOULD set \field{status} to VIRTIO_ADMIN_STATUS_ENOSPC for the
command VIRTIO_ADMIN_CMD_RESOURCE_OBJ_CREATE when the resource \field{type}
is VIRTIO_NET_RESOURCE_OBJ_FF_GROUP, if the number of flow filter group
objects in the device exceeds the lower of the configured driver
capabilities \field{groups_limit} and \field{rules_per_group_limit}.

The device SHOULD set \field{status} to VIRTIO_ADMIN_STATUS_ENOSPC for the
command VIRTIO_ADMIN_CMD_RESOURCE_OBJ_CREATE when the resource \field{type} is
VIRTIO_NET_RESOURCE_OBJ_FF_CLASSIFIER, if the number of flow filter selector
objects in the device exceeds the configured driver capability
\field{selectors_limit}.

The device SHOULD set \field{status} to VIRTIO_ADMIN_STATUS_EBUSY for the
command VIRTIO_ADMIN_CMD_RESOURCE_OBJ_DESTROY for a flow filter group when
the flow filter group has one or more flow filter rules depending on it.

The device SHOULD set \field{status} to VIRTIO_ADMIN_STATUS_EBUSY for the
command VIRTIO_ADMIN_CMD_RESOURCE_OBJ_DESTROY for a flow filter classifier when
the flow filter classifier has one or more flow filter rules depending on it.

The device SHOULD fail the command VIRTIO_ADMIN_CMD_RESOURCE_OBJ_CREATE for the
flow filter rule resource object if,
\begin{itemize}
\item \field{vq_index} is not a valid receive virtqueue index for
the VIRTIO_NET_FF_ACTION_DIRECT_RX_VQ action,
\item \field{priority} is greater than or equal to
      \field{last_rule_priority},
\item \field{id} is greater than or equal to \field{rules_limit} or
      greater than or equal to \field{rules_per_group_limit}, whichever is lower,
\item the length of \field{keys} and the length of all the mask bytes of
      \field{selectors[].mask} as referred by \field{classifier_id} differs,
\item the supplied \field{action} is not supported in the capability VIRTIO_NET_FF_ACTION_CAP.
\end{itemize}

When the flow filter directs a packet to the virtqueue identified by
\field{vq_index} and if the receive virtqueue is reset, the device
MUST drop such packets.

Upon applying a flow filter rule to a packet, the device MUST STOP any further
application of rules and cease applying any other steering configurations.

For multiple flow filter groups, the device MUST apply the rules from
the group with the highest priority. If any rule from this group is applied,
the device MUST ignore the remaining groups. If none of the rules from the
highest priority group match, the device MUST apply the rules from
the group with the next highest priority, until either a rule matches or
all groups have been attempted.

The device MUST apply the rules within the group from the highest to the
lowest priority until a rule matches the packet, and the device MUST take
the action. If an action is taken, the device MUST not take any other
action for this packet.

The device MAY apply the rules with the same \field{rule_priority} in any
order within the group.

The device MUST process incoming packets in the following order:
\begin{itemize}
\item apply the steering configuration received using control virtqueue
      commands VIRTIO_NET_CTRL_RX, VIRTIO_NET_CTRL_MAC, and
      VIRTIO_NET_CTRL_VLAN.
\item apply flow filter rules if any.
\item if no filter rule is applied, apply the steering configuration
      received using the command VIRTIO_NET_CTRL_MQ_RSS_CONFIG
      or according to automatic receive steering.
\end{itemize}

When processing an incoming packet, if the packet is dropped at any stage, the device
MUST skip further processing.

When the device drops the packet due to the configuration done using the control
virtqueue commands VIRTIO_NET_CTRL_RX or VIRTIO_NET_CTRL_MAC or VIRTIO_NET_CTRL_VLAN,
the device MUST skip flow filter rules for this packet.

When the device performs flow filter match operations and if the operation
result did not have any match in all the groups, the receive packet processing
continues to next level, i.e. to apply configuration done using
VIRTIO_NET_CTRL_MQ_RSS_CONFIG command.

The device MUST support the creation of flow filter classifier objects
using the command VIRTIO_ADMIN_CMD_RESOURCE_OBJ_CREATE with \field{flags}
set to VIRTIO_NET_FF_MASK_F_PARTIAL_MASK;
this support is required even if all the bits of the masks are set for
a field in \field{selectors}, provided that partial masking is supported
for the selectors.

\drivernormative{\paragraph}{Flow filter}{Device Types / Network Device / Device Operation / Flow filter}

The driver MUST enable VIRTIO_NET_FF_RESOURCE_CAP, VIRTIO_NET_FF_SELECTOR_CAP,
and VIRTIO_NET_FF_ACTION_CAP capabilities to use flow filter.

The driver SHOULD NOT remove a flow filter group using the command
VIRTIO_ADMIN_CMD_RESOURCE_OBJ_DESTROY when one or more flow filter rules
depend on that group. The driver SHOULD only destroy the group after
all the associated rules have been destroyed.

The driver SHOULD NOT remove a flow filter classifier using the command
VIRTIO_ADMIN_CMD_RESOURCE_OBJ_DESTROY when one or more flow filter rules
depend on the classifier. The driver SHOULD only destroy the classifier
after all the associated rules have been destroyed.

The driver SHOULD NOT add multiple flow filter rules with the same
\field{rule_priority} within a flow filter group, as these rules MAY match
the same packet. The driver SHOULD assign different \field{rule_priority}
values to different flow filter rules if multiple rules may match a single
packet.

For the command VIRTIO_ADMIN_CMD_RESOURCE_OBJ_CREATE, when creating a resource
of \field{type} VIRTIO_NET_RESOURCE_OBJ_FF_CLASSIFIER, the driver MUST set:
\begin{itemize}
\item \field{selectors[0].type} to VIRTIO_NET_FF_MASK_TYPE_ETH.
\item \field{selectors[1].type} to VIRTIO_NET_FF_MASK_TYPE_IPV4 or
      VIRTIO_NET_FF_MASK_TYPE_IPV6 when \field{count} is more than 1,
\item \field{selectors[2].type} VIRTIO_NET_FF_MASK_TYPE_UDP or
      VIRTIO_NET_FF_MASK_TYPE_TCP when \field{count} is more than 2.
\end{itemize}

For the command VIRTIO_ADMIN_CMD_RESOURCE_OBJ_CREATE, when creating a resource
of \field{type} VIRTIO_NET_RESOURCE_OBJ_FF_CLASSIFIER, the driver MUST set:
\begin{itemize}
\item \field{selectors[0].mask} bytes to all 1s for the \field{EtherType}
       when \field{count} is 2 or more.
\item \field{selectors[1].mask} bytes to all 1s for \field{Protocol} or \field{Next Header}
       when \field{selector[1].type} is VIRTIO_NET_FF_MASK_TYPE_IPV4 or VIRTIO_NET_FF_MASK_TYPE_IPV6,
       and when \field{count} is more than 2.
\end{itemize}

For the command VIRTIO_ADMIN_CMD_RESOURCE_OBJ_CREATE, the resource \field{type}
VIRTIO_NET_RESOURCE_OBJ_FF_RULE, if the corresponding classifier object's
\field{count} is 2 or more, the driver MUST SET the \field{keys} bytes of
\field{EtherType} in accordance with
\hyperref[intro:IEEE 802 Ethertypes]{IEEE 802 Ethertypes}
for either VIRTIO_NET_FF_MASK_TYPE_IPV4 or VIRTIO_NET_FF_MASK_TYPE_IPV6.

For the command VIRTIO_ADMIN_CMD_RESOURCE_OBJ_CREATE, when creating a resource of
\field{type} VIRTIO_NET_RESOURCE_OBJ_FF_RULE, if the corresponding classifier
object's \field{count} is more than 2, and the \field{selector[1].type} is either
VIRTIO_NET_FF_MASK_TYPE_IPV4 or VIRTIO_NET_FF_MASK_TYPE_IPV6, the driver MUST
set the \field{keys} bytes for the \field{Protocol} or \field{Next Header}
according to \hyperref[intro:IANA Protocol Numbers]{IANA Protocol Numbers} respectively.

The driver SHOULD set all the bits for a field in the mask of a selector in both the
capability and the classifier object, unless the VIRTIO_NET_FF_MASK_F_PARTIAL_MASK
is enabled.

\subsubsection{Legacy Interface: Framing Requirements}\label{sec:Device
Types / Network Device / Legacy Interface: Framing Requirements}

When using legacy interfaces, transitional drivers which have not
negotiated VIRTIO_F_ANY_LAYOUT MUST use a single descriptor for the
\field{struct virtio_net_hdr} on both transmit and receive, with the
network data in the following descriptors.

Additionally, when using the control virtqueue (see \ref{sec:Device
Types / Network Device / Device Operation / Control Virtqueue})
, transitional drivers which have not
negotiated VIRTIO_F_ANY_LAYOUT MUST:
\begin{itemize}
\item for all commands, use a single 2-byte descriptor including the first two
fields: \field{class} and \field{command}
\item for all commands except VIRTIO_NET_CTRL_MAC_TABLE_SET
use a single descriptor including command-specific-data
with no padding.
\item for the VIRTIO_NET_CTRL_MAC_TABLE_SET command use exactly
two descriptors including command-specific-data with no padding:
the first of these descriptors MUST include the
virtio_net_ctrl_mac table structure for the unicast addresses with no padding,
the second of these descriptors MUST include the
virtio_net_ctrl_mac table structure for the multicast addresses
with no padding.
\item for all commands, use a single 1-byte descriptor for the
\field{ack} field
\end{itemize}

See \ref{sec:Basic
Facilities of a Virtio Device / Virtqueues / Message Framing}.

\section{Network Device}\label{sec:Device Types / Network Device}

The virtio network device is a virtual network interface controller.
It consists of a virtual Ethernet link which connects the device
to the Ethernet network. The device has transmit and receive
queues. The driver adds empty buffers to the receive virtqueue.
The device receives incoming packets from the link; the device
places these incoming packets in the receive virtqueue buffers.
The driver adds outgoing packets to the transmit virtqueue. The device
removes these packets from the transmit virtqueue and sends them to
the link. The device may have a control virtqueue. The driver
uses the control virtqueue to dynamically manipulate various
features of the initialized device.

\subsection{Device ID}\label{sec:Device Types / Network Device / Device ID}

 1

\subsection{Virtqueues}\label{sec:Device Types / Network Device / Virtqueues}

\begin{description}
\item[0] receiveq1
\item[1] transmitq1
\item[\ldots]
\item[2(N-1)] receiveqN
\item[2(N-1)+1] transmitqN
\item[2N] controlq
\end{description}

 N=1 if neither VIRTIO_NET_F_MQ nor VIRTIO_NET_F_RSS are negotiated, otherwise N is set by
 \field{max_virtqueue_pairs}.

controlq is optional; it only exists if VIRTIO_NET_F_CTRL_VQ is
negotiated.

\subsection{Feature bits}\label{sec:Device Types / Network Device / Feature bits}

\begin{description}
\item[VIRTIO_NET_F_CSUM (0)] Device handles packets with partial checksum offload.

\item[VIRTIO_NET_F_GUEST_CSUM (1)] Driver handles packets with partial checksum.

\item[VIRTIO_NET_F_CTRL_GUEST_OFFLOADS (2)] Control channel offloads
        reconfiguration support.

\item[VIRTIO_NET_F_MTU(3)] Device maximum MTU reporting is supported. If
    offered by the device, device advises driver about the value of
    its maximum MTU. If negotiated, the driver uses \field{mtu} as
    the maximum MTU value.

\item[VIRTIO_NET_F_MAC (5)] Device has given MAC address.

\item[VIRTIO_NET_F_GUEST_TSO4 (7)] Driver can receive TSOv4.

\item[VIRTIO_NET_F_GUEST_TSO6 (8)] Driver can receive TSOv6.

\item[VIRTIO_NET_F_GUEST_ECN (9)] Driver can receive TSO with ECN.

\item[VIRTIO_NET_F_GUEST_UFO (10)] Driver can receive UFO.

\item[VIRTIO_NET_F_HOST_TSO4 (11)] Device can receive TSOv4.

\item[VIRTIO_NET_F_HOST_TSO6 (12)] Device can receive TSOv6.

\item[VIRTIO_NET_F_HOST_ECN (13)] Device can receive TSO with ECN.

\item[VIRTIO_NET_F_HOST_UFO (14)] Device can receive UFO.

\item[VIRTIO_NET_F_MRG_RXBUF (15)] Driver can merge receive buffers.

\item[VIRTIO_NET_F_STATUS (16)] Configuration status field is
    available.

\item[VIRTIO_NET_F_CTRL_VQ (17)] Control channel is available.

\item[VIRTIO_NET_F_CTRL_RX (18)] Control channel RX mode support.

\item[VIRTIO_NET_F_CTRL_VLAN (19)] Control channel VLAN filtering.

\item[VIRTIO_NET_F_CTRL_RX_EXTRA (20)]	Control channel RX extra mode support.

\item[VIRTIO_NET_F_GUEST_ANNOUNCE(21)] Driver can send gratuitous
    packets.

\item[VIRTIO_NET_F_MQ(22)] Device supports multiqueue with automatic
    receive steering.

\item[VIRTIO_NET_F_CTRL_MAC_ADDR(23)] Set MAC address through control
    channel.

\item[VIRTIO_NET_F_DEVICE_STATS(50)] Device can provide device-level statistics
    to the driver through the control virtqueue.

\item[VIRTIO_NET_F_HASH_TUNNEL(51)] Device supports inner header hash for encapsulated packets.

\item[VIRTIO_NET_F_VQ_NOTF_COAL(52)] Device supports virtqueue notification coalescing.

\item[VIRTIO_NET_F_NOTF_COAL(53)] Device supports notifications coalescing.

\item[VIRTIO_NET_F_GUEST_USO4 (54)] Driver can receive USOv4 packets.

\item[VIRTIO_NET_F_GUEST_USO6 (55)] Driver can receive USOv6 packets.

\item[VIRTIO_NET_F_HOST_USO (56)] Device can receive USO packets. Unlike UFO
 (fragmenting the packet) the USO splits large UDP packet
 to several segments when each of these smaller packets has UDP header.

\item[VIRTIO_NET_F_HASH_REPORT(57)] Device can report per-packet hash
    value and a type of calculated hash.

\item[VIRTIO_NET_F_GUEST_HDRLEN(59)] Driver can provide the exact \field{hdr_len}
    value. Device benefits from knowing the exact header length.

\item[VIRTIO_NET_F_RSS(60)] Device supports RSS (receive-side scaling)
    with Toeplitz hash calculation and configurable hash
    parameters for receive steering.

\item[VIRTIO_NET_F_RSC_EXT(61)] Device can process duplicated ACKs
    and report number of coalesced segments and duplicated ACKs.

\item[VIRTIO_NET_F_STANDBY(62)] Device may act as a standby for a primary
    device with the same MAC address.

\item[VIRTIO_NET_F_SPEED_DUPLEX(63)] Device reports speed and duplex.

\item[VIRTIO_NET_F_RSS_CONTEXT(64)] Device supports multiple RSS contexts.

\item[VIRTIO_NET_F_GUEST_UDP_TUNNEL_GSO (65)] Driver can receive GSO packets
  carried by a UDP tunnel.

\item[VIRTIO_NET_F_GUEST_UDP_TUNNEL_GSO_CSUM (66)] Driver handles packets
  carried by a UDP tunnel with partial csum for the outer header.

\item[VIRTIO_NET_F_HOST_UDP_TUNNEL_GSO (67)] Device can receive GSO packets
  carried by a UDP tunnel.

\item[VIRTIO_NET_F_HOST_UDP_TUNNEL_GSO_CSUM (68)] Device handles packets
  carried by a UDP tunnel with partial csum for the outer header.
\end{description}

\subsubsection{Feature bit requirements}\label{sec:Device Types / Network Device / Feature bits / Feature bit requirements}

Some networking feature bits require other networking feature bits
(see \ref{drivernormative:Basic Facilities of a Virtio Device / Feature Bits}):

\begin{description}
\item[VIRTIO_NET_F_GUEST_TSO4] Requires VIRTIO_NET_F_GUEST_CSUM.
\item[VIRTIO_NET_F_GUEST_TSO6] Requires VIRTIO_NET_F_GUEST_CSUM.
\item[VIRTIO_NET_F_GUEST_ECN] Requires VIRTIO_NET_F_GUEST_TSO4 or VIRTIO_NET_F_GUEST_TSO6.
\item[VIRTIO_NET_F_GUEST_UFO] Requires VIRTIO_NET_F_GUEST_CSUM.
\item[VIRTIO_NET_F_GUEST_USO4] Requires VIRTIO_NET_F_GUEST_CSUM.
\item[VIRTIO_NET_F_GUEST_USO6] Requires VIRTIO_NET_F_GUEST_CSUM.
\item[VIRTIO_NET_F_GUEST_UDP_TUNNEL_GSO] Requires VIRTIO_NET_F_GUEST_TSO4, VIRTIO_NET_F_GUEST_TSO6,
   VIRTIO_NET_F_GUEST_USO4 and VIRTIO_NET_F_GUEST_USO6.
\item[VIRTIO_NET_F_GUEST_UDP_TUNNEL_GSO_CSUM] Requires VIRTIO_NET_F_GUEST_UDP_TUNNEL_GSO

\item[VIRTIO_NET_F_HOST_TSO4] Requires VIRTIO_NET_F_CSUM.
\item[VIRTIO_NET_F_HOST_TSO6] Requires VIRTIO_NET_F_CSUM.
\item[VIRTIO_NET_F_HOST_ECN] Requires VIRTIO_NET_F_HOST_TSO4 or VIRTIO_NET_F_HOST_TSO6.
\item[VIRTIO_NET_F_HOST_UFO] Requires VIRTIO_NET_F_CSUM.
\item[VIRTIO_NET_F_HOST_USO] Requires VIRTIO_NET_F_CSUM.
\item[VIRTIO_NET_F_HOST_UDP_TUNNEL_GSO] Requires VIRTIO_NET_F_HOST_TSO4, VIRTIO_NET_F_HOST_TSO6
   and VIRTIO_NET_F_HOST_USO.
\item[VIRTIO_NET_F_HOST_UDP_TUNNEL_GSO_CSUM] Requires VIRTIO_NET_F_HOST_UDP_TUNNEL_GSO

\item[VIRTIO_NET_F_CTRL_RX] Requires VIRTIO_NET_F_CTRL_VQ.
\item[VIRTIO_NET_F_CTRL_VLAN] Requires VIRTIO_NET_F_CTRL_VQ.
\item[VIRTIO_NET_F_GUEST_ANNOUNCE] Requires VIRTIO_NET_F_CTRL_VQ.
\item[VIRTIO_NET_F_MQ] Requires VIRTIO_NET_F_CTRL_VQ.
\item[VIRTIO_NET_F_CTRL_MAC_ADDR] Requires VIRTIO_NET_F_CTRL_VQ.
\item[VIRTIO_NET_F_NOTF_COAL] Requires VIRTIO_NET_F_CTRL_VQ.
\item[VIRTIO_NET_F_RSC_EXT] Requires VIRTIO_NET_F_HOST_TSO4 or VIRTIO_NET_F_HOST_TSO6.
\item[VIRTIO_NET_F_RSS] Requires VIRTIO_NET_F_CTRL_VQ.
\item[VIRTIO_NET_F_VQ_NOTF_COAL] Requires VIRTIO_NET_F_CTRL_VQ.
\item[VIRTIO_NET_F_HASH_TUNNEL] Requires VIRTIO_NET_F_CTRL_VQ along with VIRTIO_NET_F_RSS or VIRTIO_NET_F_HASH_REPORT.
\item[VIRTIO_NET_F_RSS_CONTEXT] Requires VIRTIO_NET_F_CTRL_VQ and VIRTIO_NET_F_RSS.
\end{description}

\begin{note}
The dependency between UDP_TUNNEL_GSO_CSUM and UDP_TUNNEL_GSO is intentionally
in the opposite direction with respect to the plain GSO features and the plain
checksum offload because UDP tunnel checksum offload gives very little gain
for non GSO packets and is quite complex to implement in H/W.
\end{note}

\subsubsection{Legacy Interface: Feature bits}\label{sec:Device Types / Network Device / Feature bits / Legacy Interface: Feature bits}
\begin{description}
\item[VIRTIO_NET_F_GSO (6)] Device handles packets with any GSO type. This was supposed to indicate segmentation offload support, but
upon further investigation it became clear that multiple bits were needed.
\item[VIRTIO_NET_F_GUEST_RSC4 (41)] Device coalesces TCPIP v4 packets. This was implemented by hypervisor patch for certification
purposes and current Windows driver depends on it. It will not function if virtio-net device reports this feature.
\item[VIRTIO_NET_F_GUEST_RSC6 (42)] Device coalesces TCPIP v6 packets. Similar to VIRTIO_NET_F_GUEST_RSC4.
\end{description}

\subsection{Device configuration layout}\label{sec:Device Types / Network Device / Device configuration layout}
\label{sec:Device Types / Block Device / Feature bits / Device configuration layout}

The network device has the following device configuration layout.
All of the device configuration fields are read-only for the driver.

\begin{lstlisting}
struct virtio_net_config {
        u8 mac[6];
        le16 status;
        le16 max_virtqueue_pairs;
        le16 mtu;
        le32 speed;
        u8 duplex;
        u8 rss_max_key_size;
        le16 rss_max_indirection_table_length;
        le32 supported_hash_types;
        le32 supported_tunnel_types;
};
\end{lstlisting}

The \field{mac} address field always exists (although it is only
valid if VIRTIO_NET_F_MAC is set).

The \field{status} only exists if VIRTIO_NET_F_STATUS is set.
Two bits are currently defined for the status field: VIRTIO_NET_S_LINK_UP
and VIRTIO_NET_S_ANNOUNCE.

\begin{lstlisting}
#define VIRTIO_NET_S_LINK_UP     1
#define VIRTIO_NET_S_ANNOUNCE    2
\end{lstlisting}

The following field, \field{max_virtqueue_pairs} only exists if
VIRTIO_NET_F_MQ or VIRTIO_NET_F_RSS is set. This field specifies the maximum number
of each of transmit and receive virtqueues (receiveq1\ldots receiveqN
and transmitq1\ldots transmitqN respectively) that can be configured once at least one of these features
is negotiated.

The following field, \field{mtu} only exists if VIRTIO_NET_F_MTU
is set. This field specifies the maximum MTU for the driver to
use.

The following two fields, \field{speed} and \field{duplex}, only
exist if VIRTIO_NET_F_SPEED_DUPLEX is set.

\field{speed} contains the device speed, in units of 1 MBit per
second, 0 to 0x7fffffff, or 0xffffffff for unknown speed.

\field{duplex} has the values of 0x01 for full duplex, 0x00 for
half duplex and 0xff for unknown duplex state.

Both \field{speed} and \field{duplex} can change, thus the driver
is expected to re-read these values after receiving a
configuration change notification.

The following field, \field{rss_max_key_size} only exists if VIRTIO_NET_F_RSS or VIRTIO_NET_F_HASH_REPORT is set.
It specifies the maximum supported length of RSS key in bytes.

The following field, \field{rss_max_indirection_table_length} only exists if VIRTIO_NET_F_RSS is set.
It specifies the maximum number of 16-bit entries in RSS indirection table.

The next field, \field{supported_hash_types} only exists if the device supports hash calculation,
i.e. if VIRTIO_NET_F_RSS or VIRTIO_NET_F_HASH_REPORT is set.

Field \field{supported_hash_types} contains the bitmask of supported hash types.
See \ref{sec:Device Types / Network Device / Device Operation / Processing of Incoming Packets / Hash calculation for incoming packets / Supported/enabled hash types} for details of supported hash types.

Field \field{supported_tunnel_types} only exists if the device supports inner header hash, i.e. if VIRTIO_NET_F_HASH_TUNNEL is set.

Field \field{supported_tunnel_types} contains the bitmask of encapsulation types supported by the device for inner header hash.
Encapsulation types are defined in \ref{sec:Device Types / Network Device / Device Operation / Processing of Incoming Packets /
Hash calculation for incoming packets / Encapsulation types supported/enabled for inner header hash}.

\devicenormative{\subsubsection}{Device configuration layout}{Device Types / Network Device / Device configuration layout}

The device MUST set \field{max_virtqueue_pairs} to between 1 and 0x8000 inclusive,
if it offers VIRTIO_NET_F_MQ.

The device MUST set \field{mtu} to between 68 and 65535 inclusive,
if it offers VIRTIO_NET_F_MTU.

The device SHOULD set \field{mtu} to at least 1280, if it offers
VIRTIO_NET_F_MTU.

The device MUST NOT modify \field{mtu} once it has been set.

The device MUST NOT pass received packets that exceed \field{mtu} (plus low
level ethernet header length) size with \field{gso_type} NONE or ECN
after VIRTIO_NET_F_MTU has been successfully negotiated.

The device MUST forward transmitted packets of up to \field{mtu} (plus low
level ethernet header length) size with \field{gso_type} NONE or ECN, and do
so without fragmentation, after VIRTIO_NET_F_MTU has been successfully
negotiated.

The device MUST set \field{rss_max_key_size} to at least 40, if it offers
VIRTIO_NET_F_RSS or VIRTIO_NET_F_HASH_REPORT.

The device MUST set \field{rss_max_indirection_table_length} to at least 128, if it offers
VIRTIO_NET_F_RSS.

If the driver negotiates the VIRTIO_NET_F_STANDBY feature, the device MAY act
as a standby device for a primary device with the same MAC address.

If VIRTIO_NET_F_SPEED_DUPLEX has been negotiated, \field{speed}
MUST contain the device speed, in units of 1 MBit per second, 0 to
0x7ffffffff, or 0xfffffffff for unknown.

If VIRTIO_NET_F_SPEED_DUPLEX has been negotiated, \field{duplex}
MUST have the values of 0x00 for full duplex, 0x01 for half
duplex, or 0xff for unknown.

If VIRTIO_NET_F_SPEED_DUPLEX and VIRTIO_NET_F_STATUS have both
been negotiated, the device SHOULD NOT change the \field{speed} and
\field{duplex} fields as long as VIRTIO_NET_S_LINK_UP is set in
the \field{status}.

The device SHOULD NOT offer VIRTIO_NET_F_HASH_REPORT if it
does not offer VIRTIO_NET_F_CTRL_VQ.

The device SHOULD NOT offer VIRTIO_NET_F_CTRL_RX_EXTRA if it
does not offer VIRTIO_NET_F_CTRL_VQ.

\drivernormative{\subsubsection}{Device configuration layout}{Device Types / Network Device / Device configuration layout}

The driver MUST NOT write to any of the device configuration fields.

A driver SHOULD negotiate VIRTIO_NET_F_MAC if the device offers it.
If the driver negotiates the VIRTIO_NET_F_MAC feature, the driver MUST set
the physical address of the NIC to \field{mac}.  Otherwise, it SHOULD
use a locally-administered MAC address (see \hyperref[intro:IEEE 802]{IEEE 802},
``9.2 48-bit universal LAN MAC addresses'').

If the driver does not negotiate the VIRTIO_NET_F_STATUS feature, it SHOULD
assume the link is active, otherwise it SHOULD read the link status from
the bottom bit of \field{status}.

A driver SHOULD negotiate VIRTIO_NET_F_MTU if the device offers it.

If the driver negotiates VIRTIO_NET_F_MTU, it MUST supply enough receive
buffers to receive at least one receive packet of size \field{mtu} (plus low
level ethernet header length) with \field{gso_type} NONE or ECN.

If the driver negotiates VIRTIO_NET_F_MTU, it MUST NOT transmit packets of
size exceeding the value of \field{mtu} (plus low level ethernet header length)
with \field{gso_type} NONE or ECN.

A driver SHOULD negotiate the VIRTIO_NET_F_STANDBY feature if the device offers it.

If VIRTIO_NET_F_SPEED_DUPLEX has been negotiated,
the driver MUST treat any value of \field{speed} above
0x7fffffff as well as any value of \field{duplex} not
matching 0x00 or 0x01 as an unknown value.

If VIRTIO_NET_F_SPEED_DUPLEX has been negotiated, the driver
SHOULD re-read \field{speed} and \field{duplex} after a
configuration change notification.

A driver SHOULD NOT negotiate VIRTIO_NET_F_HASH_REPORT if it
does not negotiate VIRTIO_NET_F_CTRL_VQ.

A driver SHOULD NOT negotiate VIRTIO_NET_F_CTRL_RX_EXTRA if it
does not negotiate VIRTIO_NET_F_CTRL_VQ.

\subsubsection{Legacy Interface: Device configuration layout}\label{sec:Device Types / Network Device / Device configuration layout / Legacy Interface: Device configuration layout}
\label{sec:Device Types / Block Device / Feature bits / Device configuration layout / Legacy Interface: Device configuration layout}
When using the legacy interface, transitional devices and drivers
MUST format \field{status} and
\field{max_virtqueue_pairs} in struct virtio_net_config
according to the native endian of the guest rather than
(necessarily when not using the legacy interface) little-endian.

When using the legacy interface, \field{mac} is driver-writable
which provided a way for drivers to update the MAC without
negotiating VIRTIO_NET_F_CTRL_MAC_ADDR.

\subsection{Device Initialization}\label{sec:Device Types / Network Device / Device Initialization}

A driver would perform a typical initialization routine like so:

\begin{enumerate}
\item Identify and initialize the receive and
  transmission virtqueues, up to N of each kind. If
  VIRTIO_NET_F_MQ feature bit is negotiated,
  N=\field{max_virtqueue_pairs}, otherwise identify N=1.

\item If the VIRTIO_NET_F_CTRL_VQ feature bit is negotiated,
  identify the control virtqueue.

\item Fill the receive queues with buffers: see \ref{sec:Device Types / Network Device / Device Operation / Setting Up Receive Buffers}.

\item Even with VIRTIO_NET_F_MQ, only receiveq1, transmitq1 and
  controlq are used by default.  The driver would send the
  VIRTIO_NET_CTRL_MQ_VQ_PAIRS_SET command specifying the
  number of the transmit and receive queues to use.

\item If the VIRTIO_NET_F_MAC feature bit is set, the configuration
  space \field{mac} entry indicates the ``physical'' address of the
  device, otherwise the driver would typically generate a random
  local MAC address.

\item If the VIRTIO_NET_F_STATUS feature bit is negotiated, the link
  status comes from the bottom bit of \field{status}.
  Otherwise, the driver assumes it's active.

\item A performant driver would indicate that it will generate checksumless
  packets by negotiating the VIRTIO_NET_F_CSUM feature.

\item If that feature is negotiated, a driver can use TCP segmentation or UDP
  segmentation/fragmentation offload by negotiating the VIRTIO_NET_F_HOST_TSO4 (IPv4
  TCP), VIRTIO_NET_F_HOST_TSO6 (IPv6 TCP), VIRTIO_NET_F_HOST_UFO
  (UDP fragmentation) and VIRTIO_NET_F_HOST_USO (UDP segmentation) features.

\item If the VIRTIO_NET_F_HOST_TSO6, VIRTIO_NET_F_HOST_TSO4 and VIRTIO_NET_F_HOST_USO
  segmentation features are negotiated, a driver can
  use TCP segmentation or UDP segmentation on top of UDP encapsulation
  offload, when the outer header does not require checksumming - e.g.
  the outer UDP checksum is zero - by negotiating the
  VIRTIO_NET_F_HOST_UDP_TUNNEL_GSO feature.
  GSO over UDP tunnels packets carry two sets of headers: the outer ones
  and the inner ones. The outer transport protocol is UDP, the inner
  could be either TCP or UDP. Only a single level of encapsulation
  offload is supported.

\item If VIRTIO_NET_F_HOST_UDP_TUNNEL_GSO is negotiated, a driver can
  additionally use TCP segmentation or UDP segmentation on top of UDP
  encapsulation with the outer header requiring checksum offload,
  negotiating the VIRTIO_NET_F_HOST_UDP_TUNNEL_GSO_CSUM feature.

\item The converse features are also available: a driver can save
  the virtual device some work by negotiating these features.\note{For example, a network packet transported between two guests on
the same system might not need checksumming at all, nor segmentation,
if both guests are amenable.}
   The VIRTIO_NET_F_GUEST_CSUM feature indicates that partially
  checksummed packets can be received, and if it can do that then
  the VIRTIO_NET_F_GUEST_TSO4, VIRTIO_NET_F_GUEST_TSO6,
  VIRTIO_NET_F_GUEST_UFO, VIRTIO_NET_F_GUEST_ECN, VIRTIO_NET_F_GUEST_USO4,
  VIRTIO_NET_F_GUEST_USO6 VIRTIO_NET_F_GUEST_UDP_TUNNEL_GSO and
  VIRTIO_NET_F_GUEST_UDP_TUNNEL_GSO_CSUM are the input equivalents of
  the features described above.
  See \ref{sec:Device Types / Network Device / Device Operation /
Setting Up Receive Buffers}~\nameref{sec:Device Types / Network
Device / Device Operation / Setting Up Receive Buffers} and
\ref{sec:Device Types / Network Device / Device Operation /
Processing of Incoming Packets}~\nameref{sec:Device Types /
Network Device / Device Operation / Processing of Incoming Packets} below.
\end{enumerate}

A truly minimal driver would only accept VIRTIO_NET_F_MAC and ignore
everything else.

\subsection{Device and driver capabilities}\label{sec:Device Types / Network Device / Device and driver capabilities}

The network device has the following capabilities.

\begin{tabularx}{\textwidth}{ |l||l|X| }
\hline
Identifier & Name & Description \\
\hline \hline
0x0800 & \hyperref[par:Device Types / Network Device / Device Operation / Flow filter / Device and driver capabilities / VIRTIO-NET-FF-RESOURCE-CAP]{VIRTIO_NET_FF_RESOURCE_CAP} & Flow filter resource capability \\
\hline
0x0801 & \hyperref[par:Device Types / Network Device / Device Operation / Flow filter / Device and driver capabilities / VIRTIO-NET-FF-SELECTOR-CAP]{VIRTIO_NET_FF_SELECTOR_CAP} & Flow filter classifier capability \\
\hline
0x0802 & \hyperref[par:Device Types / Network Device / Device Operation / Flow filter / Device and driver capabilities / VIRTIO-NET-FF-ACTION-CAP]{VIRTIO_NET_FF_ACTION_CAP} & Flow filter action capability \\
\hline
\end{tabularx}

\subsection{Device resource objects}\label{sec:Device Types / Network Device / Device resource objects}

The network device has the following resource objects.

\begin{tabularx}{\textwidth}{ |l||l|X| }
\hline
type & Name & Description \\
\hline \hline
0x0200 & \hyperref[par:Device Types / Network Device / Device Operation / Flow filter / Resource objects / VIRTIO-NET-RESOURCE-OBJ-FF-GROUP]{VIRTIO_NET_RESOURCE_OBJ_FF_GROUP} & Flow filter group resource object \\
\hline
0x0201 & \hyperref[par:Device Types / Network Device / Device Operation / Flow filter / Resource objects / VIRTIO-NET-RESOURCE-OBJ-FF-CLASSIFIER]{VIRTIO_NET_RESOURCE_OBJ_FF_CLASSIFIER} & Flow filter mask object \\
\hline
0x0202 & \hyperref[par:Device Types / Network Device / Device Operation / Flow filter / Resource objects / VIRTIO-NET-RESOURCE-OBJ-FF-RULE]{VIRTIO_NET_RESOURCE_OBJ_FF_RULE} & Flow filter rule object \\
\hline
\end{tabularx}

\subsection{Device parts}\label{sec:Device Types / Network Device / Device parts}

Network device parts represent the configuration done by the driver using control
virtqueue commands. Network device part is in the format of
\field{struct virtio_dev_part}.

\begin{tabularx}{\textwidth}{ |l||l|X| }
\hline
Type & Name & Description \\
\hline \hline
0x200 & VIRTIO_NET_DEV_PART_CVQ_CFG_PART & Represents device configuration done through a control virtqueue command, see \ref{sec:Device Types / Network Device / Device parts / VIRTIO-NET-DEV-PART-CVQ-CFG-PART} \\
\hline
0x201 - 0x5FF & - & reserved for future \\
\hline
\hline
\end{tabularx}

\subsubsection{VIRTIO_NET_DEV_PART_CVQ_CFG_PART}\label{sec:Device Types / Network Device / Device parts / VIRTIO-NET-DEV-PART-CVQ-CFG-PART}

For VIRTIO_NET_DEV_PART_CVQ_CFG_PART, \field{part_type} is set to 0x200. The
VIRTIO_NET_DEV_PART_CVQ_CFG_PART part indicates configuration performed by the
driver using a control virtqueue command.

\begin{lstlisting}
struct virtio_net_dev_part_cvq_selector {
        u8 class;
        u8 command;
        u8 reserved[6];
};
\end{lstlisting}

There is one device part of type VIRTIO_NET_DEV_PART_CVQ_CFG_PART for each
individual configuration. Each part is identified by a unique selector value.
The selector, \field{device_type_raw}, is in the format
\field{struct virtio_net_dev_part_cvq_selector}.

The selector consists of two fields: \field{class} and \field{command}. These
fields correspond to the \field{class} and \field{command} defined in
\field{struct virtio_net_ctrl}, as described in the relevant sections of
\ref{sec:Device Types / Network Device / Device Operation / Control Virtqueue}.

The value corresponding to each part’s selector follows the same format as the
respective \field{command-specific-data} described in the relevant sections of
\ref{sec:Device Types / Network Device / Device Operation / Control Virtqueue}.

For example, when the \field{class} is VIRTIO_NET_CTRL_MAC, the \field{command}
can be either VIRTIO_NET_CTRL_MAC_TABLE_SET or VIRTIO_NET_CTRL_MAC_ADDR_SET;
when \field{command} is set to VIRTIO_NET_CTRL_MAC_TABLE_SET, \field{value}
is in the format of \field{struct virtio_net_ctrl_mac}.

Supported selectors are listed in the table:

\begin{tabularx}{\textwidth}{ |l|X| }
\hline
Class selector & Command selector \\
\hline \hline
VIRTIO_NET_CTRL_RX & VIRTIO_NET_CTRL_RX_PROMISC \\
\hline
VIRTIO_NET_CTRL_RX & VIRTIO_NET_CTRL_RX_ALLMULTI \\
\hline
VIRTIO_NET_CTRL_RX & VIRTIO_NET_CTRL_RX_ALLUNI \\
\hline
VIRTIO_NET_CTRL_RX & VIRTIO_NET_CTRL_RX_NOMULTI \\
\hline
VIRTIO_NET_CTRL_RX & VIRTIO_NET_CTRL_RX_NOUNI \\
\hline
VIRTIO_NET_CTRL_RX & VIRTIO_NET_CTRL_RX_NOBCAST \\
\hline
VIRTIO_NET_CTRL_MAC & VIRTIO_NET_CTRL_MAC_TABLE_SET \\
\hline
VIRTIO_NET_CTRL_MAC & VIRTIO_NET_CTRL_MAC_ADDR_SET \\
\hline
VIRTIO_NET_CTRL_VLAN & VIRTIO_NET_CTRL_VLAN_ADD \\
\hline
VIRTIO_NET_CTRL_ANNOUNCE & VIRTIO_NET_CTRL_ANNOUNCE_ACK \\
\hline
VIRTIO_NET_CTRL_MQ & VIRTIO_NET_CTRL_MQ_VQ_PAIRS_SET \\
\hline
VIRTIO_NET_CTRL_MQ & VIRTIO_NET_CTRL_MQ_RSS_CONFIG \\
\hline
VIRTIO_NET_CTRL_MQ & VIRTIO_NET_CTRL_MQ_HASH_CONFIG \\
\hline
\hline
\end{tabularx}

For command selector VIRTIO_NET_CTRL_VLAN_ADD, device part consists of a whole
VLAN table.

\field{reserved} is reserved and set to zero.

\subsection{Device Operation}\label{sec:Device Types / Network Device / Device Operation}

Packets are transmitted by placing them in the
transmitq1\ldots transmitqN, and buffers for incoming packets are
placed in the receiveq1\ldots receiveqN. In each case, the packet
itself is preceded by a header:

\begin{lstlisting}
struct virtio_net_hdr {
#define VIRTIO_NET_HDR_F_NEEDS_CSUM    1
#define VIRTIO_NET_HDR_F_DATA_VALID    2
#define VIRTIO_NET_HDR_F_RSC_INFO      4
#define VIRTIO_NET_HDR_F_UDP_TUNNEL_CSUM 8
        u8 flags;
#define VIRTIO_NET_HDR_GSO_NONE        0
#define VIRTIO_NET_HDR_GSO_TCPV4       1
#define VIRTIO_NET_HDR_GSO_UDP         3
#define VIRTIO_NET_HDR_GSO_TCPV6       4
#define VIRTIO_NET_HDR_GSO_UDP_L4      5
#define VIRTIO_NET_HDR_GSO_UDP_TUNNEL_IPV4 0x20
#define VIRTIO_NET_HDR_GSO_UDP_TUNNEL_IPV6 0x40
#define VIRTIO_NET_HDR_GSO_ECN      0x80
        u8 gso_type;
        le16 hdr_len;
        le16 gso_size;
        le16 csum_start;
        le16 csum_offset;
        le16 num_buffers;
        le32 hash_value;        (Only if VIRTIO_NET_F_HASH_REPORT negotiated)
        le16 hash_report;       (Only if VIRTIO_NET_F_HASH_REPORT negotiated)
        le16 padding_reserved;  (Only if VIRTIO_NET_F_HASH_REPORT negotiated)
        le16 outer_th_offset    (Only if VIRTIO_NET_F_HOST_UDP_TUNNEL_GSO or VIRTIO_NET_F_GUEST_UDP_TUNNEL_GSO negotiated)
        le16 inner_nh_offset;   (Only if VIRTIO_NET_F_HOST_UDP_TUNNEL_GSO or VIRTIO_NET_F_GUEST_UDP_TUNNEL_GSO negotiated)
};
\end{lstlisting}

The controlq is used to control various device features described further in
section \ref{sec:Device Types / Network Device / Device Operation / Control Virtqueue}.

\subsubsection{Legacy Interface: Device Operation}\label{sec:Device Types / Network Device / Device Operation / Legacy Interface: Device Operation}
When using the legacy interface, transitional devices and drivers
MUST format the fields in \field{struct virtio_net_hdr}
according to the native endian of the guest rather than
(necessarily when not using the legacy interface) little-endian.

The legacy driver only presented \field{num_buffers} in the \field{struct virtio_net_hdr}
when VIRTIO_NET_F_MRG_RXBUF was negotiated; without that feature the
structure was 2 bytes shorter.

When using the legacy interface, the driver SHOULD ignore the
used length for the transmit queues
and the controlq queue.
\begin{note}
Historically, some devices put
the total descriptor length there, even though no data was
actually written.
\end{note}

\subsubsection{Packet Transmission}\label{sec:Device Types / Network Device / Device Operation / Packet Transmission}

Transmitting a single packet is simple, but varies depending on
the different features the driver negotiated.

\begin{enumerate}
\item The driver can send a completely checksummed packet.  In this case,
  \field{flags} will be zero, and \field{gso_type} will be VIRTIO_NET_HDR_GSO_NONE.

\item If the driver negotiated VIRTIO_NET_F_CSUM, it can skip
  checksumming the packet:
  \begin{itemize}
  \item \field{flags} has the VIRTIO_NET_HDR_F_NEEDS_CSUM set,

  \item \field{csum_start} is set to the offset within the packet to begin checksumming,
    and

  \item \field{csum_offset} indicates how many bytes after the csum_start the
    new (16 bit ones' complement) checksum is placed by the device.

  \item The TCP checksum field in the packet is set to the sum
    of the TCP pseudo header, so that replacing it by the ones'
    complement checksum of the TCP header and body will give the
    correct result.
  \end{itemize}

\begin{note}
For example, consider a partially checksummed TCP (IPv4) packet.
It will have a 14 byte ethernet header and 20 byte IP header
followed by the TCP header (with the TCP checksum field 16 bytes
into that header). \field{csum_start} will be 14+20 = 34 (the TCP
checksum includes the header), and \field{csum_offset} will be 16.
If the given packet has the VIRTIO_NET_HDR_GSO_UDP_TUNNEL_IPV4 bit or the
VIRTIO_NET_HDR_GSO_UDP_TUNNEL_IPV6 bit set,
the above checksum fields refer to the inner header checksum, see
the example below.
\end{note}

\item If the driver negotiated
  VIRTIO_NET_F_HOST_TSO4, TSO6, USO or UFO, and the packet requires
  TCP segmentation, UDP segmentation or fragmentation, then \field{gso_type}
  is set to VIRTIO_NET_HDR_GSO_TCPV4, TCPV6, UDP_L4 or UDP.
  (Otherwise, it is set to VIRTIO_NET_HDR_GSO_NONE). In this
  case, packets larger than 1514 bytes can be transmitted: the
  metadata indicates how to replicate the packet header to cut it
  into smaller packets. The other gso fields are set:

  \begin{itemize}
  \item If the VIRTIO_NET_F_GUEST_HDRLEN feature has been negotiated,
    \field{hdr_len} indicates the header length that needs to be replicated
    for each packet. It's the number of bytes from the beginning of the packet
    to the beginning of the transport payload.
    If the \field{gso_type} has the VIRTIO_NET_HDR_GSO_UDP_TUNNEL_IPV4 bit or
    VIRTIO_NET_HDR_GSO_UDP_TUNNEL_IPV6 bit set, \field{hdr_len} accounts for
    all the headers up to and including the inner transport.
    Otherwise, if the VIRTIO_NET_F_GUEST_HDRLEN feature has not been negotiated,
    \field{hdr_len} is a hint to the device as to how much of the header
    needs to be kept to copy into each packet, usually set to the
    length of the headers, including the transport header\footnote{Due to various bugs in implementations, this field is not useful
as a guarantee of the transport header size.
}.

  \begin{note}
  Some devices benefit from knowledge of the exact header length.
  \end{note}

  \item \field{gso_size} is the maximum size of each packet beyond that
    header (ie. MSS).

  \item If the driver negotiated the VIRTIO_NET_F_HOST_ECN feature,
    the VIRTIO_NET_HDR_GSO_ECN bit in \field{gso_type}
    indicates that the TCP packet has the ECN bit set\footnote{This case is not handled by some older hardware, so is called out
specifically in the protocol.}.
   \end{itemize}

\item If the driver negotiated the VIRTIO_NET_F_HOST_UDP_TUNNEL_GSO feature and the
  \field{gso_type} has the VIRTIO_NET_HDR_GSO_UDP_TUNNEL_IPV4 bit or
  VIRTIO_NET_HDR_GSO_UDP_TUNNEL_IPV6 bit set, the GSO protocol is encapsulated
  in a UDP tunnel.
  If the outer UDP header requires checksumming, the driver must have
  additionally negotiated the VIRTIO_NET_F_HOST_UDP_TUNNEL_GSO_CSUM feature
  and offloaded the outer checksum accordingly, otherwise
  the outer UDP header must not require checksum validation, i.e. the outer
  UDP checksum must be positive zero (0x0) as defined in UDP RFC 768.
  The other tunnel-related fields indicate how to replicate the packet
  headers to cut it into smaller packets:

  \begin{itemize}
  \item \field{outer_th_offset} field indicates the outer transport header within
      the packet. This field differs from \field{csum_start} as the latter
      points to the inner transport header within the packet.

  \item \field{inner_nh_offset} field indicates the inner network header within
      the packet.
  \end{itemize}

\begin{note}
For example, consider a partially checksummed TCP (IPv4) packet carried over a
Geneve UDP tunnel (again IPv4) with no tunnel options. The
only relevant variable related to the tunnel type is the tunnel header length.
The packet will have a 14 byte outer ethernet header, 20 byte outer IP header
followed by the 8 byte UDP header (with a 0 checksum value), 8 byte Geneve header,
14 byte inner ethernet header, 20 byte inner IP header
and the TCP header (with the TCP checksum field 16 bytes
into that header). \field{csum_start} will be 14+20+8+8+14+20 = 84 (the TCP
checksum includes the header), \field{csum_offset} will be 16.
\field{inner_nh_offset} will be 14+20+8+8+14 = 62, \field{outer_th_offset} will be
14+20+8 = 42 and \field{gso_type} will be
VIRTIO_NET_HDR_GSO_TCPV4 | VIRTIO_NET_HDR_GSO_UDP_TUNNEL_IPV4 = 0x21
\end{note}

\item If the driver negotiated the VIRTIO_NET_F_HOST_UDP_TUNNEL_GSO_CSUM feature,
  the transmitted packet is a GSO one encapsulated in a UDP tunnel, and
  the outer UDP header requires checksumming, the driver can skip checksumming
  the outer header:

  \begin{itemize}
  \item \field{flags} has the VIRTIO_NET_HDR_F_UDP_TUNNEL_CSUM set,

  \item The outer UDP checksum field in the packet is set to the sum
    of the UDP pseudo header, so that replacing it by the ones'
    complement checksum of the outer UDP header and payload will give the
    correct result.
  \end{itemize}

\item \field{num_buffers} is set to zero.  This field is unused on transmitted packets.

\item The header and packet are added as one output descriptor to the
  transmitq, and the device is notified of the new entry
  (see \ref{sec:Device Types / Network Device / Device Initialization}~\nameref{sec:Device Types / Network Device / Device Initialization}).
\end{enumerate}

\drivernormative{\paragraph}{Packet Transmission}{Device Types / Network Device / Device Operation / Packet Transmission}

For the transmit packet buffer, the driver MUST use the size of the
structure \field{struct virtio_net_hdr} same as the receive packet buffer.

The driver MUST set \field{num_buffers} to zero.

If VIRTIO_NET_F_CSUM is not negotiated, the driver MUST set
\field{flags} to zero and SHOULD supply a fully checksummed
packet to the device.

If VIRTIO_NET_F_HOST_TSO4 is negotiated, the driver MAY set
\field{gso_type} to VIRTIO_NET_HDR_GSO_TCPV4 to request TCPv4
segmentation, otherwise the driver MUST NOT set
\field{gso_type} to VIRTIO_NET_HDR_GSO_TCPV4.

If VIRTIO_NET_F_HOST_TSO6 is negotiated, the driver MAY set
\field{gso_type} to VIRTIO_NET_HDR_GSO_TCPV6 to request TCPv6
segmentation, otherwise the driver MUST NOT set
\field{gso_type} to VIRTIO_NET_HDR_GSO_TCPV6.

If VIRTIO_NET_F_HOST_UFO is negotiated, the driver MAY set
\field{gso_type} to VIRTIO_NET_HDR_GSO_UDP to request UDP
fragmentation, otherwise the driver MUST NOT set
\field{gso_type} to VIRTIO_NET_HDR_GSO_UDP.

If VIRTIO_NET_F_HOST_USO is negotiated, the driver MAY set
\field{gso_type} to VIRTIO_NET_HDR_GSO_UDP_L4 to request UDP
segmentation, otherwise the driver MUST NOT set
\field{gso_type} to VIRTIO_NET_HDR_GSO_UDP_L4.

The driver SHOULD NOT send to the device TCP packets requiring segmentation offload
which have the Explicit Congestion Notification bit set, unless the
VIRTIO_NET_F_HOST_ECN feature is negotiated, in which case the
driver MUST set the VIRTIO_NET_HDR_GSO_ECN bit in
\field{gso_type}.

If VIRTIO_NET_F_HOST_UDP_TUNNEL_GSO is negotiated, the driver MAY set
VIRTIO_NET_HDR_GSO_UDP_TUNNEL_IPV4 bit or the VIRTIO_NET_HDR_GSO_UDP_TUNNEL_IPV6 bit
in \field{gso_type} according to the inner network header protocol type
to request GSO packets over UDPv4 or UDPv6 tunnel segmentation,
otherwise the driver MUST NOT set either the
VIRTIO_NET_HDR_GSO_UDP_TUNNEL_IPV4 bit or the VIRTIO_NET_HDR_GSO_UDP_TUNNEL_IPV6 bit
in \field{gso_type}.

When requesting GSO segmentation over UDP tunnel, the driver MUST SET the
VIRTIO_NET_HDR_GSO_UDP_TUNNEL_IPV4 bit if the inner network header is IPv4, i.e. the
packet is a TCPv4 GSO one, otherwise, if the inner network header is IPv6, the driver
MUST SET the VIRTIO_NET_HDR_GSO_UDP_TUNNEL_IPV6 bit.

The driver MUST NOT send to the device GSO packets over UDP tunnel
requiring segmentation and outer UDP checksum offload, unless both the
VIRTIO_NET_F_HOST_UDP_TUNNEL_GSO and VIRTIO_NET_F_HOST_UDP_TUNNEL_GSO_CSUM features
are negotiated, in which case the driver MUST set either the
VIRTIO_NET_HDR_GSO_UDP_TUNNEL_IPV4 bit or the VIRTIO_NET_HDR_GSO_UDP_TUNNEL_IPV6
bit in the \field{gso_type} and the VIRTIO_NET_HDR_F_UDP_TUNNEL_CSUM bit in
the \field{flags}.

If VIRTIO_NET_F_HOST_UDP_TUNNEL_GSO_CSUM is not negotiated, the driver MUST not set
the VIRTIO_NET_HDR_F_UDP_TUNNEL_CSUM bit in the \field{flags} and
MUST NOT send to the device GSO packets over UDP tunnel
requiring segmentation and outer UDP checksum offload.

The driver MUST NOT set the VIRTIO_NET_HDR_GSO_UDP_TUNNEL_IPV4 bit or the
VIRTIO_NET_HDR_GSO_UDP_TUNNEL_IPV6 bit together with VIRTIO_NET_HDR_GSO_UDP, as the
latter is deprecated in favor of UDP_L4 and no new feature will support it.

The driver MUST NOT set the VIRTIO_NET_HDR_GSO_UDP_TUNNEL_IPV4 bit and the
VIRTIO_NET_HDR_GSO_UDP_TUNNEL_IPV6 bit together.

The driver MUST NOT set the VIRTIO_NET_HDR_F_UDP_TUNNEL_CSUM bit \field{flags}
without setting either the VIRTIO_NET_HDR_GSO_UDP_TUNNEL_IPV4 bit or
the VIRTIO_NET_HDR_GSO_UDP_TUNNEL_IPV6 bit in \field{gso_type}.

If the VIRTIO_NET_F_CSUM feature has been negotiated, the
driver MAY set the VIRTIO_NET_HDR_F_NEEDS_CSUM bit in
\field{flags}, if so:
\begin{enumerate}
\item the driver MUST validate the packet checksum at
	offset \field{csum_offset} from \field{csum_start} as well as all
	preceding offsets;
\begin{note}
If \field{gso_type} differs from VIRTIO_NET_HDR_GSO_NONE and the
VIRTIO_NET_HDR_GSO_UDP_TUNNEL_IPV4 bit or the VIRTIO_NET_HDR_GSO_UDP_TUNNEL_IPV6
bit are not set in \field{gso_type}, \field{csum_offset}
points to the only transport header present in the packet, and there are no
additional preceding checksums validated by VIRTIO_NET_HDR_F_NEEDS_CSUM.
\end{note}
\item the driver MUST set the packet checksum stored in the
	buffer to the TCP/UDP pseudo header;
\item the driver MUST set \field{csum_start} and
	\field{csum_offset} such that calculating a ones'
	complement checksum from \field{csum_start} up until the end of
	the packet and storing the result at offset \field{csum_offset}
	from  \field{csum_start} will result in a fully checksummed
	packet;
\end{enumerate}

If none of the VIRTIO_NET_F_HOST_TSO4, TSO6, USO or UFO options have
been negotiated, the driver MUST set \field{gso_type} to
VIRTIO_NET_HDR_GSO_NONE.

If \field{gso_type} differs from VIRTIO_NET_HDR_GSO_NONE, then
the driver MUST also set the VIRTIO_NET_HDR_F_NEEDS_CSUM bit in
\field{flags} and MUST set \field{gso_size} to indicate the
desired MSS.

If one of the VIRTIO_NET_F_HOST_TSO4, TSO6, USO or UFO options have
been negotiated:
\begin{itemize}
\item If the VIRTIO_NET_F_GUEST_HDRLEN feature has been negotiated,
	and \field{gso_type} differs from VIRTIO_NET_HDR_GSO_NONE,
	the driver MUST set \field{hdr_len} to a value equal to the length
	of the headers, including the transport header. If \field{gso_type}
	has the VIRTIO_NET_HDR_GSO_UDP_TUNNEL_IPV4 bit or the
	VIRTIO_NET_HDR_GSO_UDP_TUNNEL_IPV6 bit set, \field{hdr_len} includes
	the inner transport header.

\item If the VIRTIO_NET_F_GUEST_HDRLEN feature has not been negotiated,
	or \field{gso_type} is VIRTIO_NET_HDR_GSO_NONE,
	the driver SHOULD set \field{hdr_len} to a value
	not less than the length of the headers, including the transport
	header.
\end{itemize}

If the VIRTIO_NET_F_HOST_UDP_TUNNEL_GSO option has been negotiated, the
driver MAY set the VIRTIO_NET_HDR_GSO_UDP_TUNNEL_IPV4 bit or the
VIRTIO_NET_HDR_GSO_UDP_TUNNEL_IPV6 bit in \field{gso_type}, if so:
\begin{itemize}
\item the driver MUST set \field{outer_th_offset} to the outer UDP header
  offset and \field{inner_nh_offset} to the inner network header offset.
  The \field{csum_start} and \field{csum_offset} fields point respectively
  to the inner transport header and inner transport checksum field.
\end{itemize}

If the VIRTIO_NET_F_HOST_UDP_TUNNEL_GSO_CSUM feature has been negotiated,
and the VIRTIO_NET_HDR_GSO_UDP_TUNNEL_IPV4 bit or
VIRTIO_NET_HDR_GSO_UDP_TUNNEL_IPV6 bit in \field{gso_type} are set,
the driver MAY set the VIRTIO_NET_HDR_F_UDP_TUNNEL_CSUM bit in
\field{flags}, if so the driver MUST set the packet outer UDP header checksum
to the outer UDP pseudo header checksum.

\begin{note}
calculating a ones' complement checksum from \field{outer_th_offset}
up until the end of the packet and storing the result at offset 6
from \field{outer_th_offset} will result in a fully checksummed outer UDP packet;
\end{note}

If the VIRTIO_NET_HDR_GSO_UDP_TUNNEL_IPV4 bit or the
VIRTIO_NET_HDR_GSO_UDP_TUNNEL_IPV6 bit in \field{gso_type} are set
and the VIRTIO_NET_F_HOST_UDP_TUNNEL_GSO_CSUM feature has not
been negotiated, the
outer UDP header MUST NOT require checksum validation. That is, the
outer UDP checksum value MUST be 0 or the validated complete checksum
for such header.

\begin{note}
The valid complete checksum of the outer UDP header of individual segments
can be computed by the driver prior to segmentation only if the GSO packet
size is a multiple of \field{gso_size}, because then all segments
have the same size and thus all data included in the outer UDP
checksum is the same for every segment. These pre-computed segment
length and checksum fields are different from those of the GSO
packet.
In this scenario the outer UDP header of the GSO packet must carry the
segmented UDP packet length.
\end{note}

If the VIRTIO_NET_F_HOST_UDP_TUNNEL_GSO option has not
been negotiated, the driver MUST NOT set either the VIRTIO_NET_HDR_F_GSO_UDP_TUNNEL_IPV4
bit or the VIRTIO_NET_HDR_F_GSO_UDP_TUNNEL_IPV6 in \field{gso_type}.

If the VIRTIO_NET_F_HOST_UDP_TUNNEL_GSO_CSUM option has not been negotiated,
the driver MUST NOT set the VIRTIO_NET_HDR_F_UDP_TUNNEL_CSUM bit
in \field{flags}.

The driver SHOULD accept the VIRTIO_NET_F_GUEST_HDRLEN feature if it has
been offered, and if it's able to provide the exact header length.

The driver MUST NOT set the VIRTIO_NET_HDR_F_DATA_VALID and
VIRTIO_NET_HDR_F_RSC_INFO bits in \field{flags}.

The driver MUST NOT set the VIRTIO_NET_HDR_F_DATA_VALID bit in \field{flags}
together with the VIRTIO_NET_HDR_F_GSO_UDP_TUNNEL_IPV4 bit or the
VIRTIO_NET_HDR_F_GSO_UDP_TUNNEL_IPV6 bit in \field{gso_type}.

\devicenormative{\paragraph}{Packet Transmission}{Device Types / Network Device / Device Operation / Packet Transmission}
The device MUST ignore \field{flag} bits that it does not recognize.

If VIRTIO_NET_HDR_F_NEEDS_CSUM bit in \field{flags} is not set, the
device MUST NOT use the \field{csum_start} and \field{csum_offset}.

If one of the VIRTIO_NET_F_HOST_TSO4, TSO6, USO or UFO options have
been negotiated:
\begin{itemize}
\item If the VIRTIO_NET_F_GUEST_HDRLEN feature has been negotiated,
	and \field{gso_type} differs from VIRTIO_NET_HDR_GSO_NONE,
	the device MAY use \field{hdr_len} as the transport header size.

	\begin{note}
	Caution should be taken by the implementation so as to prevent
	a malicious driver from attacking the device by setting an incorrect hdr_len.
	\end{note}

\item If the VIRTIO_NET_F_GUEST_HDRLEN feature has not been negotiated,
	or \field{gso_type} is VIRTIO_NET_HDR_GSO_NONE,
	the device MAY use \field{hdr_len} only as a hint about the
	transport header size.
	The device MUST NOT rely on \field{hdr_len} to be correct.

	\begin{note}
	This is due to various bugs in implementations.
	\end{note}
\end{itemize}

If both the VIRTIO_NET_HDR_GSO_UDP_TUNNEL_IPV4 bit and
the VIRTIO_NET_HDR_GSO_UDP_TUNNEL_IPV6 bit in in \field{gso_type} are set,
the device MUST NOT accept the packet.

If the VIRTIO_NET_HDR_GSO_UDP_TUNNEL_IPV4 bit and the VIRTIO_NET_HDR_GSO_UDP_TUNNEL_IPV6
bit in \field{gso_type} are not set, the device MUST NOT use the
\field{outer_th_offset} and \field{inner_nh_offset}.

If either the VIRTIO_NET_HDR_GSO_UDP_TUNNEL_IPV4 bit or
the VIRTIO_NET_HDR_GSO_UDP_TUNNEL_IPV6 bit in \field{gso_type} are set, and any of
the following is true:
\begin{itemize}
\item the VIRTIO_NET_HDR_F_NEEDS_CSUM is not set in \field{flags}
\item the VIRTIO_NET_HDR_F_DATA_VALID is set in \field{flags}
\item the \field{gso_type} excluding the VIRTIO_NET_HDR_GSO_UDP_TUNNEL_IPV4
bit and the VIRTIO_NET_HDR_GSO_UDP_TUNNEL_IPV6 bit is VIRTIO_NET_HDR_GSO_NONE
\end{itemize}
the device MUST NOT accept the packet.

If the VIRTIO_NET_HDR_F_UDP_TUNNEL_CSUM bit in \field{flags} is set,
and both the bits VIRTIO_NET_HDR_GSO_UDP_TUNNEL_IPV4 and
VIRTIO_NET_HDR_GSO_UDP_TUNNEL_IPV6 in \field{gso_type} are not set,
the device MOST NOT accept the packet.

If VIRTIO_NET_HDR_F_NEEDS_CSUM is not set, the device MUST NOT
rely on the packet checksum being correct.
\paragraph{Packet Transmission Interrupt}\label{sec:Device Types / Network Device / Device Operation / Packet Transmission / Packet Transmission Interrupt}

Often a driver will suppress transmission virtqueue interrupts
and check for used packets in the transmit path of following
packets.

The normal behavior in this interrupt handler is to retrieve
used buffers from the virtqueue and free the corresponding
headers and packets.

\subsubsection{Setting Up Receive Buffers}\label{sec:Device Types / Network Device / Device Operation / Setting Up Receive Buffers}

It is generally a good idea to keep the receive virtqueue as
fully populated as possible: if it runs out, network performance
will suffer.

If the VIRTIO_NET_F_GUEST_TSO4, VIRTIO_NET_F_GUEST_TSO6,
VIRTIO_NET_F_GUEST_UFO, VIRTIO_NET_F_GUEST_USO4 or VIRTIO_NET_F_GUEST_USO6
features are used, the maximum incoming packet
will be 65589 bytes long (14 bytes of Ethernet header, plus 40 bytes of
the IPv6 header, plus 65535 bytes of maximum IPv6 payload including any
extension header), otherwise 1514 bytes.
When VIRTIO_NET_F_HASH_REPORT is not negotiated, the required receive buffer
size is either 65601 or 1526 bytes accounting for 20 bytes of
\field{struct virtio_net_hdr} followed by receive packet.
When VIRTIO_NET_F_HASH_REPORT is negotiated, the required receive buffer
size is either 65609 or 1534 bytes accounting for 12 bytes of
\field{struct virtio_net_hdr} followed by receive packet.

\drivernormative{\paragraph}{Setting Up Receive Buffers}{Device Types / Network Device / Device Operation / Setting Up Receive Buffers}

\begin{itemize}
\item If VIRTIO_NET_F_MRG_RXBUF is not negotiated:
  \begin{itemize}
    \item If VIRTIO_NET_F_GUEST_TSO4, VIRTIO_NET_F_GUEST_TSO6, VIRTIO_NET_F_GUEST_UFO,
	VIRTIO_NET_F_GUEST_USO4 or VIRTIO_NET_F_GUEST_USO6 are negotiated, the driver SHOULD populate
      the receive queue(s) with buffers of at least 65609 bytes if
      VIRTIO_NET_F_HASH_REPORT is negotiated, and of at least 65601 bytes if not.
    \item Otherwise, the driver SHOULD populate the receive queue(s)
      with buffers of at least 1534 bytes if VIRTIO_NET_F_HASH_REPORT
      is negotiated, and of at least 1526 bytes if not.
  \end{itemize}
\item If VIRTIO_NET_F_MRG_RXBUF is negotiated, each buffer MUST be at
least size of \field{struct virtio_net_hdr},
i.e. 20 bytes if VIRTIO_NET_F_HASH_REPORT is negotiated, and 12 bytes if not.
\end{itemize}

\begin{note}
Obviously each buffer can be split across multiple descriptor elements.
\end{note}

When calculating the size of \field{struct virtio_net_hdr}, the driver
MUST consider all the fields inclusive up to \field{padding_reserved},
i.e. 20 bytes if VIRTIO_NET_F_HASH_REPORT is negotiated, and 12 bytes if not.

If VIRTIO_NET_F_MQ is negotiated, each of receiveq1\ldots receiveqN
that will be used SHOULD be populated with receive buffers.

\devicenormative{\paragraph}{Setting Up Receive Buffers}{Device Types / Network Device / Device Operation / Setting Up Receive Buffers}

The device MUST set \field{num_buffers} to the number of descriptors used to
hold the incoming packet.

The device MUST use only a single descriptor if VIRTIO_NET_F_MRG_RXBUF
was not negotiated.
\begin{note}
{This means that \field{num_buffers} will always be 1
if VIRTIO_NET_F_MRG_RXBUF is not negotiated.}
\end{note}

\subsubsection{Processing of Incoming Packets}\label{sec:Device Types / Network Device / Device Operation / Processing of Incoming Packets}
\label{sec:Device Types / Network Device / Device Operation / Processing of Packets}%old label for latexdiff

When a packet is copied into a buffer in the receiveq, the
optimal path is to disable further used buffer notifications for the
receiveq and process packets until no more are found, then re-enable
them.

Processing incoming packets involves:

\begin{enumerate}
\item \field{num_buffers} indicates how many descriptors
  this packet is spread over (including this one): this will
  always be 1 if VIRTIO_NET_F_MRG_RXBUF was not negotiated.
  This allows receipt of large packets without having to allocate large
  buffers: a packet that does not fit in a single buffer can flow
  over to the next buffer, and so on. In this case, there will be
  at least \field{num_buffers} used buffers in the virtqueue, and the device
  chains them together to form a single packet in a way similar to
  how it would store it in a single buffer spread over multiple
  descriptors.
  The other buffers will not begin with a \field{struct virtio_net_hdr}.

\item If
  \field{num_buffers} is one, then the entire packet will be
  contained within this buffer, immediately following the struct
  virtio_net_hdr.
\item If the VIRTIO_NET_F_GUEST_CSUM feature was negotiated, the
  VIRTIO_NET_HDR_F_DATA_VALID bit in \field{flags} can be
  set: if so, device has validated the packet checksum.
  If the VIRTIO_NET_F_GUEST_UDP_TUNNEL_GSO_CSUM feature has been negotiated,
  and the VIRTIO_NET_HDR_F_UDP_TUNNEL_CSUM bit is set in \field{flags},
  both the outer UDP checksum and the inner transport checksum
  have been validated, otherwise only one level of checksums (the outer one
  in case of tunnels) has been validated.
\end{enumerate}

Additionally, VIRTIO_NET_F_GUEST_CSUM, TSO4, TSO6, UDP, UDP_TUNNEL
and ECN features enable receive checksum, large receive offload and ECN
support which are the input equivalents of the transmit checksum,
transmit segmentation offloading and ECN features, as described
in \ref{sec:Device Types / Network Device / Device Operation /
Packet Transmission}:
\begin{enumerate}
\item If the VIRTIO_NET_F_GUEST_TSO4, TSO6, UFO, USO4 or USO6 options were
  negotiated, then \field{gso_type} MAY be something other than
  VIRTIO_NET_HDR_GSO_NONE, and \field{gso_size} field indicates the
  desired MSS (see Packet Transmission point 2).
\item If the VIRTIO_NET_F_RSC_EXT option was negotiated (this
  implies one of VIRTIO_NET_F_GUEST_TSO4, TSO6), the
  device processes also duplicated ACK segments, reports
  number of coalesced TCP segments in \field{csum_start} field and
  number of duplicated ACK segments in \field{csum_offset} field
  and sets bit VIRTIO_NET_HDR_F_RSC_INFO in \field{flags}.
\item If the VIRTIO_NET_F_GUEST_CSUM feature was negotiated, the
  VIRTIO_NET_HDR_F_NEEDS_CSUM bit in \field{flags} can be
  set: if so, the packet checksum at offset \field{csum_offset}
  from \field{csum_start} and any preceding checksums
  have been validated.  The checksum on the packet is incomplete and
  if bit VIRTIO_NET_HDR_F_RSC_INFO is not set in \field{flags},
  then \field{csum_start} and \field{csum_offset} indicate how to calculate it
  (see Packet Transmission point 1).
\begin{note}
If \field{gso_type} differs from VIRTIO_NET_HDR_GSO_NONE and the
VIRTIO_NET_HDR_GSO_UDP_TUNNEL_IPV4 bit or the VIRTIO_NET_HDR_GSO_UDP_TUNNEL_IPV6
bit are not set, \field{csum_offset}
points to the only transport header present in the packet, and there are no
additional preceding checksums validated by VIRTIO_NET_HDR_F_NEEDS_CSUM.
\end{note}
\item If the VIRTIO_NET_F_GUEST_UDP_TUNNEL_GSO option was negotiated and
  \field{gso_type} is not VIRTIO_NET_HDR_GSO_NONE, the
  VIRTIO_NET_HDR_GSO_UDP_TUNNEL_IPV4 bit or the VIRTIO_NET_HDR_GSO_UDP_TUNNEL_IPV6
  bit MAY be set. In such case the \field{outer_th_offset} and
  \field{inner_nh_offset} fields indicate the corresponding
  headers information.
\item If the VIRTIO_NET_F_GUEST_UDP_TUNNEL_GSO_CSUM feature was
negotiated, and
  the VIRTIO_NET_HDR_GSO_UDP_TUNNEL_IPV4 bit or the VIRTIO_NET_HDR_GSO_UDP_TUNNEL_IPV6
  are set in \field{gso_type}, the VIRTIO_NET_HDR_F_UDP_TUNNEL_CSUM bit in the
  \field{flags} can be set: if so, the outer UDP checksum has been validated
  and the UDP header checksum at offset 6 from from \field{outer_th_offset}
  is set to the outer UDP pseudo header checksum.

\begin{note}
If the VIRTIO_NET_HDR_GSO_UDP_TUNNEL_IPV4 bit or VIRTIO_NET_HDR_GSO_UDP_TUNNEL_IPV6
bit are set in \field{gso_type}, the \field{csum_start} field refers to
the inner transport header offset (see Packet Transmission point 1).
If the VIRTIO_NET_HDR_F_UDP_TUNNEL_CSUM bit in \field{flags} is set both
the inner and the outer header checksums have been validated by
VIRTIO_NET_HDR_F_NEEDS_CSUM, otherwise only the inner transport header
checksum has been validated.
\end{note}
\end{enumerate}

If applicable, the device calculates per-packet hash for incoming packets as
defined in \ref{sec:Device Types / Network Device / Device Operation / Processing of Incoming Packets / Hash calculation for incoming packets}.

If applicable, the device reports hash information for incoming packets as
defined in \ref{sec:Device Types / Network Device / Device Operation / Processing of Incoming Packets / Hash reporting for incoming packets}.

\devicenormative{\paragraph}{Processing of Incoming Packets}{Device Types / Network Device / Device Operation / Processing of Incoming Packets}
\label{devicenormative:Device Types / Network Device / Device Operation / Processing of Packets}%old label for latexdiff

If VIRTIO_NET_F_MRG_RXBUF has not been negotiated, the device MUST set
\field{num_buffers} to 1.

If VIRTIO_NET_F_MRG_RXBUF has been negotiated, the device MUST set
\field{num_buffers} to indicate the number of buffers
the packet (including the header) is spread over.

If a receive packet is spread over multiple buffers, the device
MUST use all buffers but the last (i.e. the first \field{num_buffers} -
1 buffers) completely up to the full length of each buffer
supplied by the driver.

The device MUST use all buffers used by a single receive
packet together, such that at least \field{num_buffers} are
observed by driver as used.

If VIRTIO_NET_F_GUEST_CSUM is not negotiated, the device MUST set
\field{flags} to zero and SHOULD supply a fully checksummed
packet to the driver.

If VIRTIO_NET_F_GUEST_TSO4 is not negotiated, the device MUST NOT set
\field{gso_type} to VIRTIO_NET_HDR_GSO_TCPV4.

If VIRTIO_NET_F_GUEST_UDP is not negotiated, the device MUST NOT set
\field{gso_type} to VIRTIO_NET_HDR_GSO_UDP.

If VIRTIO_NET_F_GUEST_TSO6 is not negotiated, the device MUST NOT set
\field{gso_type} to VIRTIO_NET_HDR_GSO_TCPV6.

If none of VIRTIO_NET_F_GUEST_USO4 or VIRTIO_NET_F_GUEST_USO6 have been negotiated,
the device MUST NOT set \field{gso_type} to VIRTIO_NET_HDR_GSO_UDP_L4.

If VIRTIO_NET_F_GUEST_UDP_TUNNEL_GSO is not negotiated, the device MUST NOT set
either the VIRTIO_NET_HDR_GSO_UDP_TUNNEL_IPV4 bit or the
VIRTIO_NET_HDR_GSO_UDP_TUNNEL_IPV6 bit in \field{gso_type}.

If VIRTIO_NET_F_GUEST_UDP_TUNNEL_GSO_CSUM is not negotiated the device MUST NOT set
the VIRTIO_NET_HDR_F_UDP_TUNNEL_CSUM bit in \field{flags}.

The device SHOULD NOT send to the driver TCP packets requiring segmentation offload
which have the Explicit Congestion Notification bit set, unless the
VIRTIO_NET_F_GUEST_ECN feature is negotiated, in which case the
device MUST set the VIRTIO_NET_HDR_GSO_ECN bit in
\field{gso_type}.

If the VIRTIO_NET_F_GUEST_CSUM feature has been negotiated, the
device MAY set the VIRTIO_NET_HDR_F_NEEDS_CSUM bit in
\field{flags}, if so:
\begin{enumerate}
\item the device MUST validate the packet checksum at
	offset \field{csum_offset} from \field{csum_start} as well as all
	preceding offsets;
\item the device MUST set the packet checksum stored in the
	receive buffer to the TCP/UDP pseudo header;
\item the device MUST set \field{csum_start} and
	\field{csum_offset} such that calculating a ones'
	complement checksum from \field{csum_start} up until the
	end of the packet and storing the result at offset
	\field{csum_offset} from  \field{csum_start} will result in a
	fully checksummed packet;
\end{enumerate}

The device MUST NOT send to the driver GSO packets encapsulated in UDP
tunnel and requiring segmentation offload, unless the
VIRTIO_NET_F_GUEST_UDP_TUNNEL_GSO is negotiated, in which case the device MUST set
the VIRTIO_NET_HDR_GSO_UDP_TUNNEL_IPV4 bit or the VIRTIO_NET_HDR_GSO_UDP_TUNNEL_IPV6
bit in \field{gso_type} according to the inner network header protocol type,
MUST set the \field{outer_th_offset} and \field{inner_nh_offset} fields
to the corresponding header information, and the outer UDP header MUST NOT
require checksum offload.

If the VIRTIO_NET_F_GUEST_UDP_TUNNEL_GSO_CSUM feature has not been negotiated,
the device MUST NOT send the driver GSO packets encapsulated in UDP
tunnel and requiring segmentation and outer checksum offload.

If none of the VIRTIO_NET_F_GUEST_TSO4, TSO6, UFO, USO4 or USO6 options have
been negotiated, the device MUST set \field{gso_type} to
VIRTIO_NET_HDR_GSO_NONE.

If \field{gso_type} differs from VIRTIO_NET_HDR_GSO_NONE, then
the device MUST also set the VIRTIO_NET_HDR_F_NEEDS_CSUM bit in
\field{flags} MUST set \field{gso_size} to indicate the desired MSS.
If VIRTIO_NET_F_RSC_EXT was negotiated, the device MUST also
set VIRTIO_NET_HDR_F_RSC_INFO bit in \field{flags},
set \field{csum_start} to number of coalesced TCP segments and
set \field{csum_offset} to number of received duplicated ACK segments.

If VIRTIO_NET_F_RSC_EXT was not negotiated, the device MUST
not set VIRTIO_NET_HDR_F_RSC_INFO bit in \field{flags}.

If one of the VIRTIO_NET_F_GUEST_TSO4, TSO6, UFO, USO4 or USO6 options have
been negotiated, the device SHOULD set \field{hdr_len} to a value
not less than the length of the headers, including the transport
header. If \field{gso_type} has the VIRTIO_NET_HDR_GSO_UDP_TUNNEL_IPV4 bit
or the VIRTIO_NET_HDR_GSO_UDP_TUNNEL_IPV6 bit set, the referenced transport
header is the inner one.

If the VIRTIO_NET_F_GUEST_CSUM feature has been negotiated, the
device MAY set the VIRTIO_NET_HDR_F_DATA_VALID bit in
\field{flags}, if so, the device MUST validate the packet
checksum. If the VIRTIO_NET_F_GUEST_UDP_TUNNEL_GSO_CSUM feature has
been negotiated, and the VIRTIO_NET_HDR_F_UDP_TUNNEL_CSUM bit set in
\field{flags}, both the outer UDP checksum and the inner transport
checksum have been validated.
Otherwise level of checksum is validated: in case of multiple
encapsulated protocols the outermost one.

If either the VIRTIO_NET_HDR_GSO_UDP_TUNNEL_IPV4 bit or the
VIRTIO_NET_HDR_GSO_UDP_TUNNEL_IPV6 bit in \field{gso_type} are set,
the device MUST NOT set the VIRTIO_NET_HDR_F_DATA_VALID bit in
\field{flags}.

If the VIRTIO_NET_F_GUEST_UDP_TUNNEL_GSO_CSUM feature has been negotiated
and either the VIRTIO_NET_HDR_GSO_UDP_TUNNEL_IPV4 bit is set or the
VIRTIO_NET_HDR_GSO_UDP_TUNNEL_IPV6 bit is set in \field{gso_type}, the
device MAY set the VIRTIO_NET_HDR_F_UDP_TUNNEL_CSUM bit in
\field{flags}, if so the device MUST set the packet outer UDP checksum
stored in the receive buffer to the outer UDP pseudo header.

Otherwise, the VIRTIO_NET_F_GUEST_UDP_TUNNEL_GSO_CSUM feature has been
negotiated, either the VIRTIO_NET_HDR_GSO_UDP_TUNNEL_IPV4 bit is set or the
VIRTIO_NET_HDR_GSO_UDP_TUNNEL_IPV6 bit is set in \field{gso_type},
and the bit VIRTIO_NET_HDR_F_UDP_TUNNEL_CSUM is not set in
\field{flags}, the device MUST either provide a zero outer UDP header
checksum or a fully checksummed outer UDP header.

\drivernormative{\paragraph}{Processing of Incoming
Packets}{Device Types / Network Device / Device Operation /
Processing of Incoming Packets}

The driver MUST ignore \field{flag} bits that it does not recognize.

If VIRTIO_NET_HDR_F_NEEDS_CSUM bit in \field{flags} is not set or
if VIRTIO_NET_HDR_F_RSC_INFO bit \field{flags} is set, the
driver MUST NOT use the \field{csum_start} and \field{csum_offset}.

If one of the VIRTIO_NET_F_GUEST_TSO4, TSO6, UFO, USO4 or USO6 options have
been negotiated, the driver MAY use \field{hdr_len} only as a hint about the
transport header size.
The driver MUST NOT rely on \field{hdr_len} to be correct.
\begin{note}
This is due to various bugs in implementations.
\end{note}

If neither VIRTIO_NET_HDR_F_NEEDS_CSUM nor
VIRTIO_NET_HDR_F_DATA_VALID is set, the driver MUST NOT
rely on the packet checksum being correct.

If both the VIRTIO_NET_HDR_GSO_UDP_TUNNEL_IPV4 bit and
the VIRTIO_NET_HDR_GSO_UDP_TUNNEL_IPV6 bit in in \field{gso_type} are set,
the driver MUST NOT accept the packet.

If the VIRTIO_NET_HDR_GSO_UDP_TUNNEL_IPV4 bit or the VIRTIO_NET_HDR_GSO_UDP_TUNNEL_IPV6
bit in \field{gso_type} are not set, the driver MUST NOT use the
\field{outer_th_offset} and \field{inner_nh_offset}.

If either the VIRTIO_NET_HDR_GSO_UDP_TUNNEL_IPV4 bit or
the VIRTIO_NET_HDR_GSO_UDP_TUNNEL_IPV6 bit in \field{gso_type} are set, and any of
the following is true:
\begin{itemize}
\item the VIRTIO_NET_HDR_F_NEEDS_CSUM bit is not set in \field{flags}
\item the VIRTIO_NET_HDR_F_DATA_VALID bit is set in \field{flags}
\item the \field{gso_type} excluding the VIRTIO_NET_HDR_GSO_UDP_TUNNEL_IPV4
bit and the VIRTIO_NET_HDR_GSO_UDP_TUNNEL_IPV6 bit is VIRTIO_NET_HDR_GSO_NONE
\end{itemize}
the driver MUST NOT accept the packet.

If the VIRTIO_NET_HDR_F_UDP_TUNNEL_CSUM bit and the VIRTIO_NET_HDR_F_NEEDS_CSUM
bit in \field{flags} are set,
and both the bits VIRTIO_NET_HDR_GSO_UDP_TUNNEL_IPV4 and
VIRTIO_NET_HDR_GSO_UDP_TUNNEL_IPV6 in \field{gso_type} are not set,
the driver MOST NOT accept the packet.

\paragraph{Hash calculation for incoming packets}
\label{sec:Device Types / Network Device / Device Operation / Processing of Incoming Packets / Hash calculation for incoming packets}

A device attempts to calculate a per-packet hash in the following cases:
\begin{itemize}
\item The feature VIRTIO_NET_F_RSS was negotiated. The device uses the hash to determine the receive virtqueue to place incoming packets.
\item The feature VIRTIO_NET_F_HASH_REPORT was negotiated. The device reports the hash value and the hash type with the packet.
\end{itemize}

If the feature VIRTIO_NET_F_RSS was negotiated:
\begin{itemize}
\item The device uses \field{hash_types} of the virtio_net_rss_config structure as 'Enabled hash types' bitmask.
\item If additionally the feature VIRTIO_NET_F_HASH_TUNNEL was negotiated, the device uses \field{enabled_tunnel_types} of the
      virtnet_hash_tunnel structure as 'Encapsulation types enabled for inner header hash' bitmask.
\item The device uses a key as defined in \field{hash_key_data} and \field{hash_key_length} of the virtio_net_rss_config structure (see
\ref{sec:Device Types / Network Device / Device Operation / Control Virtqueue / Receive-side scaling (RSS) / Setting RSS parameters}).
\end{itemize}

If the feature VIRTIO_NET_F_RSS was not negotiated:
\begin{itemize}
\item The device uses \field{hash_types} of the virtio_net_hash_config structure as 'Enabled hash types' bitmask.
\item If additionally the feature VIRTIO_NET_F_HASH_TUNNEL was negotiated, the device uses \field{enabled_tunnel_types} of the
      virtnet_hash_tunnel structure as 'Encapsulation types enabled for inner header hash' bitmask.
\item The device uses a key as defined in \field{hash_key_data} and \field{hash_key_length} of the virtio_net_hash_config structure (see
\ref{sec:Device Types / Network Device / Device Operation / Control Virtqueue / Automatic receive steering in multiqueue mode / Hash calculation}).
\end{itemize}

Note that if the device offers VIRTIO_NET_F_HASH_REPORT, even if it supports only one pair of virtqueues, it MUST support
at least one of commands of VIRTIO_NET_CTRL_MQ class to configure reported hash parameters:
\begin{itemize}
\item If the device offers VIRTIO_NET_F_RSS, it MUST support VIRTIO_NET_CTRL_MQ_RSS_CONFIG command per
 \ref{sec:Device Types / Network Device / Device Operation / Control Virtqueue / Receive-side scaling (RSS) / Setting RSS parameters}.
\item Otherwise the device MUST support VIRTIO_NET_CTRL_MQ_HASH_CONFIG command per
 \ref{sec:Device Types / Network Device / Device Operation / Control Virtqueue / Automatic receive steering in multiqueue mode / Hash calculation}.
\end{itemize}

The per-packet hash calculation can depend on the IP packet type. See
\hyperref[intro:IP]{[IP]}, \hyperref[intro:UDP]{[UDP]} and \hyperref[intro:TCP]{[TCP]}.

\subparagraph{Supported/enabled hash types}
\label{sec:Device Types / Network Device / Device Operation / Processing of Incoming Packets / Hash calculation for incoming packets / Supported/enabled hash types}
Hash types applicable for IPv4 packets:
\begin{lstlisting}
#define VIRTIO_NET_HASH_TYPE_IPv4              (1 << 0)
#define VIRTIO_NET_HASH_TYPE_TCPv4             (1 << 1)
#define VIRTIO_NET_HASH_TYPE_UDPv4             (1 << 2)
\end{lstlisting}
Hash types applicable for IPv6 packets without extension headers
\begin{lstlisting}
#define VIRTIO_NET_HASH_TYPE_IPv6              (1 << 3)
#define VIRTIO_NET_HASH_TYPE_TCPv6             (1 << 4)
#define VIRTIO_NET_HASH_TYPE_UDPv6             (1 << 5)
\end{lstlisting}
Hash types applicable for IPv6 packets with extension headers
\begin{lstlisting}
#define VIRTIO_NET_HASH_TYPE_IP_EX             (1 << 6)
#define VIRTIO_NET_HASH_TYPE_TCP_EX            (1 << 7)
#define VIRTIO_NET_HASH_TYPE_UDP_EX            (1 << 8)
\end{lstlisting}

\subparagraph{IPv4 packets}
\label{sec:Device Types / Network Device / Device Operation / Processing of Incoming Packets / Hash calculation for incoming packets / IPv4 packets}
The device calculates the hash on IPv4 packets according to 'Enabled hash types' bitmask as follows:
\begin{itemize}
\item If VIRTIO_NET_HASH_TYPE_TCPv4 is set and the packet has
a TCP header, the hash is calculated over the following fields:
\begin{itemize}
\item Source IP address
\item Destination IP address
\item Source TCP port
\item Destination TCP port
\end{itemize}
\item Else if VIRTIO_NET_HASH_TYPE_UDPv4 is set and the
packet has a UDP header, the hash is calculated over the following fields:
\begin{itemize}
\item Source IP address
\item Destination IP address
\item Source UDP port
\item Destination UDP port
\end{itemize}
\item Else if VIRTIO_NET_HASH_TYPE_IPv4 is set, the hash is
calculated over the following fields:
\begin{itemize}
\item Source IP address
\item Destination IP address
\end{itemize}
\item Else the device does not calculate the hash
\end{itemize}

\subparagraph{IPv6 packets without extension header}
\label{sec:Device Types / Network Device / Device Operation / Processing of Incoming Packets / Hash calculation for incoming packets / IPv6 packets without extension header}
The device calculates the hash on IPv6 packets without extension
headers according to 'Enabled hash types' bitmask as follows:
\begin{itemize}
\item If VIRTIO_NET_HASH_TYPE_TCPv6 is set and the packet has
a TCPv6 header, the hash is calculated over the following fields:
\begin{itemize}
\item Source IPv6 address
\item Destination IPv6 address
\item Source TCP port
\item Destination TCP port
\end{itemize}
\item Else if VIRTIO_NET_HASH_TYPE_UDPv6 is set and the
packet has a UDPv6 header, the hash is calculated over the following fields:
\begin{itemize}
\item Source IPv6 address
\item Destination IPv6 address
\item Source UDP port
\item Destination UDP port
\end{itemize}
\item Else if VIRTIO_NET_HASH_TYPE_IPv6 is set, the hash is
calculated over the following fields:
\begin{itemize}
\item Source IPv6 address
\item Destination IPv6 address
\end{itemize}
\item Else the device does not calculate the hash
\end{itemize}

\subparagraph{IPv6 packets with extension header}
\label{sec:Device Types / Network Device / Device Operation / Processing of Incoming Packets / Hash calculation for incoming packets / IPv6 packets with extension header}
The device calculates the hash on IPv6 packets with extension
headers according to 'Enabled hash types' bitmask as follows:
\begin{itemize}
\item If VIRTIO_NET_HASH_TYPE_TCP_EX is set and the packet
has a TCPv6 header, the hash is calculated over the following fields:
\begin{itemize}
\item Home address from the home address option in the IPv6 destination options header. If the extension header is not present, use the Source IPv6 address.
\item IPv6 address that is contained in the Routing-Header-Type-2 from the associated extension header. If the extension header is not present, use the Destination IPv6 address.
\item Source TCP port
\item Destination TCP port
\end{itemize}
\item Else if VIRTIO_NET_HASH_TYPE_UDP_EX is set and the
packet has a UDPv6 header, the hash is calculated over the following fields:
\begin{itemize}
\item Home address from the home address option in the IPv6 destination options header. If the extension header is not present, use the Source IPv6 address.
\item IPv6 address that is contained in the Routing-Header-Type-2 from the associated extension header. If the extension header is not present, use the Destination IPv6 address.
\item Source UDP port
\item Destination UDP port
\end{itemize}
\item Else if VIRTIO_NET_HASH_TYPE_IP_EX is set, the hash is
calculated over the following fields:
\begin{itemize}
\item Home address from the home address option in the IPv6 destination options header. If the extension header is not present, use the Source IPv6 address.
\item IPv6 address that is contained in the Routing-Header-Type-2 from the associated extension header. If the extension header is not present, use the Destination IPv6 address.
\end{itemize}
\item Else skip IPv6 extension headers and calculate the hash as
defined for an IPv6 packet without extension headers
(see \ref{sec:Device Types / Network Device / Device Operation / Processing of Incoming Packets / Hash calculation for incoming packets / IPv6 packets without extension header}).
\end{itemize}

\paragraph{Inner Header Hash}
\label{sec:Device Types / Network Device / Device Operation / Processing of Incoming Packets / Inner Header Hash}

If VIRTIO_NET_F_HASH_TUNNEL has been negotiated, the driver can send the command
VIRTIO_NET_CTRL_HASH_TUNNEL_SET to configure the calculation of the inner header hash.

\begin{lstlisting}
struct virtnet_hash_tunnel {
    le32 enabled_tunnel_types;
};

#define VIRTIO_NET_CTRL_HASH_TUNNEL 7
 #define VIRTIO_NET_CTRL_HASH_TUNNEL_SET 0
\end{lstlisting}

Field \field{enabled_tunnel_types} contains the bitmask of encapsulation types enabled for inner header hash.
See \ref{sec:Device Types / Network Device / Device Operation / Processing of Incoming Packets /
Hash calculation for incoming packets / Encapsulation types supported/enabled for inner header hash}.

The class VIRTIO_NET_CTRL_HASH_TUNNEL has one command:
VIRTIO_NET_CTRL_HASH_TUNNEL_SET sets \field{enabled_tunnel_types} for the device using the
virtnet_hash_tunnel structure, which is read-only for the device.

Inner header hash is disabled by VIRTIO_NET_CTRL_HASH_TUNNEL_SET with \field{enabled_tunnel_types} set to 0.

Initially (before the driver sends any VIRTIO_NET_CTRL_HASH_TUNNEL_SET command) all
encapsulation types are disabled for inner header hash.

\subparagraph{Encapsulated packet}
\label{sec:Device Types / Network Device / Device Operation / Processing of Incoming Packets / Hash calculation for incoming packets / Encapsulated packet}

Multiple tunneling protocols allow encapsulating an inner, payload packet in an outer, encapsulated packet.
The encapsulated packet thus contains an outer header and an inner header, and the device calculates the
hash over either the inner header or the outer header.

If VIRTIO_NET_F_HASH_TUNNEL is negotiated and a received encapsulated packet's outer header matches one of the
encapsulation types enabled in \field{enabled_tunnel_types}, then the device uses the inner header for hash
calculations (only a single level of encapsulation is currently supported).

If VIRTIO_NET_F_HASH_TUNNEL is negotiated and a received packet's (outer) header does not match any encapsulation
types enabled in \field{enabled_tunnel_types}, then the device uses the outer header for hash calculations.

\subparagraph{Encapsulation types supported/enabled for inner header hash}
\label{sec:Device Types / Network Device / Device Operation / Processing of Incoming Packets /
Hash calculation for incoming packets / Encapsulation types supported/enabled for inner header hash}

Encapsulation types applicable for inner header hash:
\begin{lstlisting}[escapechar=|]
#define VIRTIO_NET_HASH_TUNNEL_TYPE_GRE_2784    (1 << 0) /* |\hyperref[intro:rfc2784]{[RFC2784]}| */
#define VIRTIO_NET_HASH_TUNNEL_TYPE_GRE_2890    (1 << 1) /* |\hyperref[intro:rfc2890]{[RFC2890]}| */
#define VIRTIO_NET_HASH_TUNNEL_TYPE_GRE_7676    (1 << 2) /* |\hyperref[intro:rfc7676]{[RFC7676]}| */
#define VIRTIO_NET_HASH_TUNNEL_TYPE_GRE_UDP     (1 << 3) /* |\hyperref[intro:rfc8086]{[GRE-in-UDP]}| */
#define VIRTIO_NET_HASH_TUNNEL_TYPE_VXLAN       (1 << 4) /* |\hyperref[intro:vxlan]{[VXLAN]}| */
#define VIRTIO_NET_HASH_TUNNEL_TYPE_VXLAN_GPE   (1 << 5) /* |\hyperref[intro:vxlan-gpe]{[VXLAN-GPE]}| */
#define VIRTIO_NET_HASH_TUNNEL_TYPE_GENEVE      (1 << 6) /* |\hyperref[intro:geneve]{[GENEVE]}| */
#define VIRTIO_NET_HASH_TUNNEL_TYPE_IPIP        (1 << 7) /* |\hyperref[intro:ipip]{[IPIP]}| */
#define VIRTIO_NET_HASH_TUNNEL_TYPE_NVGRE       (1 << 8) /* |\hyperref[intro:nvgre]{[NVGRE]}| */
\end{lstlisting}

\subparagraph{Advice}
Example uses of the inner header hash:
\begin{itemize}
\item Legacy tunneling protocols, lacking the outer header entropy, can use RSS with the inner header hash to
      distribute flows with identical outer but different inner headers across various queues, improving performance.
\item Identify an inner flow distributed across multiple outer tunnels.
\end{itemize}

As using the inner header hash completely discards the outer header entropy, care must be taken
if the inner header is controlled by an adversary, as the adversary can then intentionally create
configurations with insufficient entropy.

Besides disabling the inner header hash, mitigations would depend on how the hash is used. When the hash
use is limited to the RSS queue selection, the inner header hash may have quality of service (QoS) limitations.

\devicenormative{\subparagraph}{Inner Header Hash}{Device Types / Network Device / Device Operation / Control Virtqueue / Inner Header Hash}

If the (outer) header of the received packet does not match any encapsulation types enabled
in \field{enabled_tunnel_types}, the device MUST calculate the hash on the outer header.

If the device receives any bits in \field{enabled_tunnel_types} which are not set in \field{supported_tunnel_types},
it SHOULD respond to the VIRTIO_NET_CTRL_HASH_TUNNEL_SET command with VIRTIO_NET_ERR.

If the driver sets \field{enabled_tunnel_types} to 0 through VIRTIO_NET_CTRL_HASH_TUNNEL_SET or upon the device reset,
the device MUST disable the inner header hash for all encapsulation types.

\drivernormative{\subparagraph}{Inner Header Hash}{Device Types / Network Device / Device Operation / Control Virtqueue / Inner Header Hash}

The driver MUST have negotiated the VIRTIO_NET_F_HASH_TUNNEL feature when issuing the VIRTIO_NET_CTRL_HASH_TUNNEL_SET command.

The driver MUST NOT set any bits in \field{enabled_tunnel_types} which are not set in \field{supported_tunnel_types}.

The driver MUST ignore bits in \field{supported_tunnel_types} which are not documented in this specification.

\paragraph{Hash reporting for incoming packets}
\label{sec:Device Types / Network Device / Device Operation / Processing of Incoming Packets / Hash reporting for incoming packets}

If VIRTIO_NET_F_HASH_REPORT was negotiated and
 the device has calculated the hash for the packet, the device fills \field{hash_report} with the report type of calculated hash
and \field{hash_value} with the value of calculated hash.

If VIRTIO_NET_F_HASH_REPORT was negotiated but due to any reason the
hash was not calculated, the device sets \field{hash_report} to VIRTIO_NET_HASH_REPORT_NONE.

Possible values that the device can report in \field{hash_report} are defined below.
They correspond to supported hash types defined in
\ref{sec:Device Types / Network Device / Device Operation / Processing of Incoming Packets / Hash calculation for incoming packets / Supported/enabled hash types}
as follows:

VIRTIO_NET_HASH_TYPE_XXX = 1 << (VIRTIO_NET_HASH_REPORT_XXX - 1)

\begin{lstlisting}
#define VIRTIO_NET_HASH_REPORT_NONE            0
#define VIRTIO_NET_HASH_REPORT_IPv4            1
#define VIRTIO_NET_HASH_REPORT_TCPv4           2
#define VIRTIO_NET_HASH_REPORT_UDPv4           3
#define VIRTIO_NET_HASH_REPORT_IPv6            4
#define VIRTIO_NET_HASH_REPORT_TCPv6           5
#define VIRTIO_NET_HASH_REPORT_UDPv6           6
#define VIRTIO_NET_HASH_REPORT_IPv6_EX         7
#define VIRTIO_NET_HASH_REPORT_TCPv6_EX        8
#define VIRTIO_NET_HASH_REPORT_UDPv6_EX        9
\end{lstlisting}

\subsubsection{Control Virtqueue}\label{sec:Device Types / Network Device / Device Operation / Control Virtqueue}

The driver uses the control virtqueue (if VIRTIO_NET_F_CTRL_VQ is
negotiated) to send commands to manipulate various features of
the device which would not easily map into the configuration
space.

All commands are of the following form:

\begin{lstlisting}
struct virtio_net_ctrl {
        u8 class;
        u8 command;
        u8 command-specific-data[];
        u8 ack;
        u8 command-specific-result[];
};

/* ack values */
#define VIRTIO_NET_OK     0
#define VIRTIO_NET_ERR    1
\end{lstlisting}

The \field{class}, \field{command} and command-specific-data are set by the
driver, and the device sets the \field{ack} byte and optionally
\field{command-specific-result}. There is little the driver can
do except issue a diagnostic if \field{ack} is not VIRTIO_NET_OK.

The command VIRTIO_NET_CTRL_STATS_QUERY and VIRTIO_NET_CTRL_STATS_GET contain
\field{command-specific-result}.

\paragraph{Packet Receive Filtering}\label{sec:Device Types / Network Device / Device Operation / Control Virtqueue / Packet Receive Filtering}
\label{sec:Device Types / Network Device / Device Operation / Control Virtqueue / Setting Promiscuous Mode}%old label for latexdiff

If the VIRTIO_NET_F_CTRL_RX and VIRTIO_NET_F_CTRL_RX_EXTRA
features are negotiated, the driver can send control commands for
promiscuous mode, multicast, unicast and broadcast receiving.

\begin{note}
In general, these commands are best-effort: unwanted
packets could still arrive.
\end{note}

\begin{lstlisting}
#define VIRTIO_NET_CTRL_RX    0
 #define VIRTIO_NET_CTRL_RX_PROMISC      0
 #define VIRTIO_NET_CTRL_RX_ALLMULTI     1
 #define VIRTIO_NET_CTRL_RX_ALLUNI       2
 #define VIRTIO_NET_CTRL_RX_NOMULTI      3
 #define VIRTIO_NET_CTRL_RX_NOUNI        4
 #define VIRTIO_NET_CTRL_RX_NOBCAST      5
\end{lstlisting}


\devicenormative{\subparagraph}{Packet Receive Filtering}{Device Types / Network Device / Device Operation / Control Virtqueue / Packet Receive Filtering}

If the VIRTIO_NET_F_CTRL_RX feature has been negotiated,
the device MUST support the following VIRTIO_NET_CTRL_RX class
commands:
\begin{itemize}
\item VIRTIO_NET_CTRL_RX_PROMISC turns promiscuous mode on and
off. The command-specific-data is one byte containing 0 (off) or
1 (on). If promiscuous mode is on, the device SHOULD receive all
incoming packets.
This SHOULD take effect even if one of the other modes set by
a VIRTIO_NET_CTRL_RX class command is on.
\item VIRTIO_NET_CTRL_RX_ALLMULTI turns all-multicast receive on and
off. The command-specific-data is one byte containing 0 (off) or
1 (on). When all-multicast receive is on the device SHOULD allow
all incoming multicast packets.
\end{itemize}

If the VIRTIO_NET_F_CTRL_RX_EXTRA feature has been negotiated,
the device MUST support the following VIRTIO_NET_CTRL_RX class
commands:
\begin{itemize}
\item VIRTIO_NET_CTRL_RX_ALLUNI turns all-unicast receive on and
off. The command-specific-data is one byte containing 0 (off) or
1 (on). When all-unicast receive is on the device SHOULD allow
all incoming unicast packets.
\item VIRTIO_NET_CTRL_RX_NOMULTI suppresses multicast receive.
The command-specific-data is one byte containing 0 (multicast
receive allowed) or 1 (multicast receive suppressed).
When multicast receive is suppressed, the device SHOULD NOT
send multicast packets to the driver.
This SHOULD take effect even if VIRTIO_NET_CTRL_RX_ALLMULTI is on.
This filter SHOULD NOT apply to broadcast packets.
\item VIRTIO_NET_CTRL_RX_NOUNI suppresses unicast receive.
The command-specific-data is one byte containing 0 (unicast
receive allowed) or 1 (unicast receive suppressed).
When unicast receive is suppressed, the device SHOULD NOT
send unicast packets to the driver.
This SHOULD take effect even if VIRTIO_NET_CTRL_RX_ALLUNI is on.
\item VIRTIO_NET_CTRL_RX_NOBCAST suppresses broadcast receive.
The command-specific-data is one byte containing 0 (broadcast
receive allowed) or 1 (broadcast receive suppressed).
When broadcast receive is suppressed, the device SHOULD NOT
send broadcast packets to the driver.
This SHOULD take effect even if VIRTIO_NET_CTRL_RX_ALLMULTI is on.
\end{itemize}

\drivernormative{\subparagraph}{Packet Receive Filtering}{Device Types / Network Device / Device Operation / Control Virtqueue / Packet Receive Filtering}

If the VIRTIO_NET_F_CTRL_RX feature has not been negotiated,
the driver MUST NOT issue commands VIRTIO_NET_CTRL_RX_PROMISC or
VIRTIO_NET_CTRL_RX_ALLMULTI.

If the VIRTIO_NET_F_CTRL_RX_EXTRA feature has not been negotiated,
the driver MUST NOT issue commands
 VIRTIO_NET_CTRL_RX_ALLUNI,
 VIRTIO_NET_CTRL_RX_NOMULTI,
 VIRTIO_NET_CTRL_RX_NOUNI or
 VIRTIO_NET_CTRL_RX_NOBCAST.

\paragraph{Setting MAC Address Filtering}\label{sec:Device Types / Network Device / Device Operation / Control Virtqueue / Setting MAC Address Filtering}

If the VIRTIO_NET_F_CTRL_RX feature is negotiated, the driver can
send control commands for MAC address filtering.

\begin{lstlisting}
struct virtio_net_ctrl_mac {
        le32 entries;
        u8 macs[entries][6];
};

#define VIRTIO_NET_CTRL_MAC    1
 #define VIRTIO_NET_CTRL_MAC_TABLE_SET        0
 #define VIRTIO_NET_CTRL_MAC_ADDR_SET         1
\end{lstlisting}

The device can filter incoming packets by any number of destination
MAC addresses\footnote{Since there are no guarantees, it can use a hash filter or
silently switch to allmulti or promiscuous mode if it is given too
many addresses.
}. This table is set using the class
VIRTIO_NET_CTRL_MAC and the command VIRTIO_NET_CTRL_MAC_TABLE_SET. The
command-specific-data is two variable length tables of 6-byte MAC
addresses (as described in struct virtio_net_ctrl_mac). The first table contains unicast addresses, and the second
contains multicast addresses.

The VIRTIO_NET_CTRL_MAC_ADDR_SET command is used to set the
default MAC address which rx filtering
accepts (and if VIRTIO_NET_F_MAC has been negotiated,
this will be reflected in \field{mac} in config space).

The command-specific-data for VIRTIO_NET_CTRL_MAC_ADDR_SET is
the 6-byte MAC address.

\devicenormative{\subparagraph}{Setting MAC Address Filtering}{Device Types / Network Device / Device Operation / Control Virtqueue / Setting MAC Address Filtering}

The device MUST have an empty MAC filtering table on reset.

The device MUST update the MAC filtering table before it consumes
the VIRTIO_NET_CTRL_MAC_TABLE_SET command.

The device MUST update \field{mac} in config space before it consumes
the VIRTIO_NET_CTRL_MAC_ADDR_SET command, if VIRTIO_NET_F_MAC has
been negotiated.

The device SHOULD drop incoming packets which have a destination MAC which
matches neither the \field{mac} (or that set with VIRTIO_NET_CTRL_MAC_ADDR_SET)
nor the MAC filtering table.

\drivernormative{\subparagraph}{Setting MAC Address Filtering}{Device Types / Network Device / Device Operation / Control Virtqueue / Setting MAC Address Filtering}

If VIRTIO_NET_F_CTRL_RX has not been negotiated,
the driver MUST NOT issue VIRTIO_NET_CTRL_MAC class commands.

If VIRTIO_NET_F_CTRL_RX has been negotiated,
the driver SHOULD issue VIRTIO_NET_CTRL_MAC_ADDR_SET
to set the default mac if it is different from \field{mac}.

The driver MUST follow the VIRTIO_NET_CTRL_MAC_TABLE_SET command
by a le32 number, followed by that number of non-multicast
MAC addresses, followed by another le32 number, followed by
that number of multicast addresses.  Either number MAY be 0.

\subparagraph{Legacy Interface: Setting MAC Address Filtering}\label{sec:Device Types / Network Device / Device Operation / Control Virtqueue / Setting MAC Address Filtering / Legacy Interface: Setting MAC Address Filtering}
When using the legacy interface, transitional devices and drivers
MUST format \field{entries} in struct virtio_net_ctrl_mac
according to the native endian of the guest rather than
(necessarily when not using the legacy interface) little-endian.

Legacy drivers that didn't negotiate VIRTIO_NET_F_CTRL_MAC_ADDR
changed \field{mac} in config space when NIC is accepting
incoming packets. These drivers always wrote the mac value from
first to last byte, therefore after detecting such drivers,
a transitional device MAY defer MAC update, or MAY defer
processing incoming packets until driver writes the last byte
of \field{mac} in the config space.

\paragraph{VLAN Filtering}\label{sec:Device Types / Network Device / Device Operation / Control Virtqueue / VLAN Filtering}

If the driver negotiates the VIRTIO_NET_F_CTRL_VLAN feature, it
can control a VLAN filter table in the device. The VLAN filter
table applies only to VLAN tagged packets.

When VIRTIO_NET_F_CTRL_VLAN is negotiated, the device starts with
an empty VLAN filter table.

\begin{note}
Similar to the MAC address based filtering, the VLAN filtering
is also best-effort: unwanted packets could still arrive.
\end{note}

\begin{lstlisting}
#define VIRTIO_NET_CTRL_VLAN       2
 #define VIRTIO_NET_CTRL_VLAN_ADD             0
 #define VIRTIO_NET_CTRL_VLAN_DEL             1
\end{lstlisting}

Both the VIRTIO_NET_CTRL_VLAN_ADD and VIRTIO_NET_CTRL_VLAN_DEL
command take a little-endian 16-bit VLAN id as the command-specific-data.

VIRTIO_NET_CTRL_VLAN_ADD command adds the specified VLAN to the
VLAN filter table.

VIRTIO_NET_CTRL_VLAN_DEL command removes the specified VLAN from
the VLAN filter table.

\devicenormative{\subparagraph}{VLAN Filtering}{Device Types / Network Device / Device Operation / Control Virtqueue / VLAN Filtering}

When VIRTIO_NET_F_CTRL_VLAN is not negotiated, the device MUST
accept all VLAN tagged packets.

When VIRTIO_NET_F_CTRL_VLAN is negotiated, the device MUST
accept all VLAN tagged packets whose VLAN tag is present in
the VLAN filter table and SHOULD drop all VLAN tagged packets
whose VLAN tag is absent in the VLAN filter table.

\subparagraph{Legacy Interface: VLAN Filtering}\label{sec:Device Types / Network Device / Device Operation / Control Virtqueue / VLAN Filtering / Legacy Interface: VLAN Filtering}
When using the legacy interface, transitional devices and drivers
MUST format the VLAN id
according to the native endian of the guest rather than
(necessarily when not using the legacy interface) little-endian.

\paragraph{Gratuitous Packet Sending}\label{sec:Device Types / Network Device / Device Operation / Control Virtqueue / Gratuitous Packet Sending}

If the driver negotiates the VIRTIO_NET_F_GUEST_ANNOUNCE (depends
on VIRTIO_NET_F_CTRL_VQ), the device can ask the driver to send gratuitous
packets; this is usually done after the guest has been physically
migrated, and needs to announce its presence on the new network
links. (As hypervisor does not have the knowledge of guest
network configuration (eg. tagged vlan) it is simplest to prod
the guest in this way).

\begin{lstlisting}
#define VIRTIO_NET_CTRL_ANNOUNCE       3
 #define VIRTIO_NET_CTRL_ANNOUNCE_ACK             0
\end{lstlisting}

The driver checks VIRTIO_NET_S_ANNOUNCE bit in the device configuration \field{status} field
when it notices the changes of device configuration. The
command VIRTIO_NET_CTRL_ANNOUNCE_ACK is used to indicate that
driver has received the notification and device clears the
VIRTIO_NET_S_ANNOUNCE bit in \field{status}.

Processing this notification involves:

\begin{enumerate}
\item Sending the gratuitous packets (eg. ARP) or marking there are pending
  gratuitous packets to be sent and letting deferred routine to
  send them.

\item Sending VIRTIO_NET_CTRL_ANNOUNCE_ACK command through control
  vq.
\end{enumerate}

\drivernormative{\subparagraph}{Gratuitous Packet Sending}{Device Types / Network Device / Device Operation / Control Virtqueue / Gratuitous Packet Sending}

If the driver negotiates VIRTIO_NET_F_GUEST_ANNOUNCE, it SHOULD notify
network peers of its new location after it sees the VIRTIO_NET_S_ANNOUNCE bit
in \field{status}.  The driver MUST send a command on the command queue
with class VIRTIO_NET_CTRL_ANNOUNCE and command VIRTIO_NET_CTRL_ANNOUNCE_ACK.

\devicenormative{\subparagraph}{Gratuitous Packet Sending}{Device Types / Network Device / Device Operation / Control Virtqueue / Gratuitous Packet Sending}

If VIRTIO_NET_F_GUEST_ANNOUNCE is negotiated, the device MUST clear the
VIRTIO_NET_S_ANNOUNCE bit in \field{status} upon receipt of a command buffer
with class VIRTIO_NET_CTRL_ANNOUNCE and command VIRTIO_NET_CTRL_ANNOUNCE_ACK
before marking the buffer as used.

\paragraph{Device operation in multiqueue mode}\label{sec:Device Types / Network Device / Device Operation / Control Virtqueue / Device operation in multiqueue mode}

This specification defines the following modes that a device MAY implement for operation with multiple transmit/receive virtqueues:
\begin{itemize}
\item Automatic receive steering as defined in \ref{sec:Device Types / Network Device / Device Operation / Control Virtqueue / Automatic receive steering in multiqueue mode}.
 If a device supports this mode, it offers the VIRTIO_NET_F_MQ feature bit.
\item Receive-side scaling as defined in \ref{devicenormative:Device Types / Network Device / Device Operation / Control Virtqueue / Receive-side scaling (RSS) / RSS processing}.
 If a device supports this mode, it offers the VIRTIO_NET_F_RSS feature bit.
\end{itemize}

A device MAY support one of these features or both. The driver MAY negotiate any set of these features that the device supports.

Multiqueue is disabled by default.

The driver enables multiqueue by sending a command using \field{class} VIRTIO_NET_CTRL_MQ. The \field{command} selects the mode of multiqueue operation, as follows:
\begin{lstlisting}
#define VIRTIO_NET_CTRL_MQ    4
 #define VIRTIO_NET_CTRL_MQ_VQ_PAIRS_SET        0 (for automatic receive steering)
 #define VIRTIO_NET_CTRL_MQ_RSS_CONFIG          1 (for configurable receive steering)
 #define VIRTIO_NET_CTRL_MQ_HASH_CONFIG         2 (for configurable hash calculation)
\end{lstlisting}

If more than one multiqueue mode is negotiated, the resulting device configuration is defined by the last command sent by the driver.

\paragraph{Automatic receive steering in multiqueue mode}\label{sec:Device Types / Network Device / Device Operation / Control Virtqueue / Automatic receive steering in multiqueue mode}

If the driver negotiates the VIRTIO_NET_F_MQ feature bit (depends on VIRTIO_NET_F_CTRL_VQ), it MAY transmit outgoing packets on one
of the multiple transmitq1\ldots transmitqN and ask the device to
queue incoming packets into one of the multiple receiveq1\ldots receiveqN
depending on the packet flow.

The driver enables multiqueue by
sending the VIRTIO_NET_CTRL_MQ_VQ_PAIRS_SET command, specifying
the number of the transmit and receive queues to be used up to
\field{max_virtqueue_pairs}; subsequently,
transmitq1\ldots transmitqn and receiveq1\ldots receiveqn where
n=\field{virtqueue_pairs} MAY be used.
\begin{lstlisting}
struct virtio_net_ctrl_mq_pairs_set {
       le16 virtqueue_pairs;
};
#define VIRTIO_NET_CTRL_MQ_VQ_PAIRS_MIN        1
#define VIRTIO_NET_CTRL_MQ_VQ_PAIRS_MAX        0x8000

\end{lstlisting}

When multiqueue is enabled by VIRTIO_NET_CTRL_MQ_VQ_PAIRS_SET command, the device MUST use automatic receive steering
based on packet flow. Programming of the receive steering
classificator is implicit. After the driver transmitted a packet of a
flow on transmitqX, the device SHOULD cause incoming packets for that flow to
be steered to receiveqX. For uni-directional protocols, or where
no packets have been transmitted yet, the device MAY steer a packet
to a random queue out of the specified receiveq1\ldots receiveqn.

Multiqueue is disabled by VIRTIO_NET_CTRL_MQ_VQ_PAIRS_SET with \field{virtqueue_pairs} to 1 (this is
the default) and waiting for the device to use the command buffer.

\drivernormative{\subparagraph}{Automatic receive steering in multiqueue mode}{Device Types / Network Device / Device Operation / Control Virtqueue / Automatic receive steering in multiqueue mode}

The driver MUST configure the virtqueues before enabling them with the
VIRTIO_NET_CTRL_MQ_VQ_PAIRS_SET command.

The driver MUST NOT request a \field{virtqueue_pairs} of 0 or
greater than \field{max_virtqueue_pairs} in the device configuration space.

The driver MUST queue packets only on any transmitq1 before the
VIRTIO_NET_CTRL_MQ_VQ_PAIRS_SET command.

The driver MUST NOT queue packets on transmit queues greater than
\field{virtqueue_pairs} once it has placed the VIRTIO_NET_CTRL_MQ_VQ_PAIRS_SET command in the available ring.

\devicenormative{\subparagraph}{Automatic receive steering in multiqueue mode}{Device Types / Network Device / Device Operation / Control Virtqueue / Automatic receive steering in multiqueue mode}

After initialization of reset, the device MUST queue packets only on receiveq1.

The device MUST NOT queue packets on receive queues greater than
\field{virtqueue_pairs} once it has placed the
VIRTIO_NET_CTRL_MQ_VQ_PAIRS_SET command in a used buffer.

If the destination receive queue is being reset (See \ref{sec:Basic Facilities of a Virtio Device / Virtqueues / Virtqueue Reset}),
the device SHOULD re-select another random queue. If all receive queues are
being reset, the device MUST drop the packet.

\subparagraph{Legacy Interface: Automatic receive steering in multiqueue mode}\label{sec:Device Types / Network Device / Device Operation / Control Virtqueue / Automatic receive steering in multiqueue mode / Legacy Interface: Automatic receive steering in multiqueue mode}
When using the legacy interface, transitional devices and drivers
MUST format \field{virtqueue_pairs}
according to the native endian of the guest rather than
(necessarily when not using the legacy interface) little-endian.

\subparagraph{Hash calculation}\label{sec:Device Types / Network Device / Device Operation / Control Virtqueue / Automatic receive steering in multiqueue mode / Hash calculation}
If VIRTIO_NET_F_HASH_REPORT was negotiated and the device uses automatic receive steering,
the device MUST support a command to configure hash calculation parameters.

The driver provides parameters for hash calculation as follows:

\field{class} VIRTIO_NET_CTRL_MQ, \field{command} VIRTIO_NET_CTRL_MQ_HASH_CONFIG.

The \field{command-specific-data} has following format:
\begin{lstlisting}
struct virtio_net_hash_config {
    le32 hash_types;
    le16 reserved[4];
    u8 hash_key_length;
    u8 hash_key_data[hash_key_length];
};
\end{lstlisting}
Field \field{hash_types} contains a bitmask of allowed hash types as
defined in
\ref{sec:Device Types / Network Device / Device Operation / Processing of Incoming Packets / Hash calculation for incoming packets / Supported/enabled hash types}.
Initially the device has all hash types disabled and reports only VIRTIO_NET_HASH_REPORT_NONE.

Field \field{reserved} MUST contain zeroes. It is defined to make the structure to match the layout of virtio_net_rss_config structure,
defined in \ref{sec:Device Types / Network Device / Device Operation / Control Virtqueue / Receive-side scaling (RSS)}.

Fields \field{hash_key_length} and \field{hash_key_data} define the key to be used in hash calculation.

\paragraph{Receive-side scaling (RSS)}\label{sec:Device Types / Network Device / Device Operation / Control Virtqueue / Receive-side scaling (RSS)}
A device offers the feature VIRTIO_NET_F_RSS if it supports RSS receive steering with Toeplitz hash calculation and configurable parameters.

A driver queries RSS capabilities of the device by reading device configuration as defined in \ref{sec:Device Types / Network Device / Device configuration layout}

\subparagraph{Setting RSS parameters}\label{sec:Device Types / Network Device / Device Operation / Control Virtqueue / Receive-side scaling (RSS) / Setting RSS parameters}

Driver sends a VIRTIO_NET_CTRL_MQ_RSS_CONFIG command using the following format for \field{command-specific-data}:
\begin{lstlisting}
struct rss_rq_id {
   le16 vq_index_1_16: 15; /* Bits 1 to 16 of the virtqueue index */
   le16 reserved: 1; /* Set to zero */
};

struct virtio_net_rss_config {
    le32 hash_types;
    le16 indirection_table_mask;
    struct rss_rq_id unclassified_queue;
    struct rss_rq_id indirection_table[indirection_table_length];
    le16 max_tx_vq;
    u8 hash_key_length;
    u8 hash_key_data[hash_key_length];
};
\end{lstlisting}
Field \field{hash_types} contains a bitmask of allowed hash types as
defined in
\ref{sec:Device Types / Network Device / Device Operation / Processing of Incoming Packets / Hash calculation for incoming packets / Supported/enabled hash types}.

Field \field{indirection_table_mask} is a mask to be applied to
the calculated hash to produce an index in the
\field{indirection_table} array.
Number of entries in \field{indirection_table} is (\field{indirection_table_mask} + 1).

\field{rss_rq_id} is a receive virtqueue id. \field{vq_index_1_16}
consists of bits 1 to 16 of a virtqueue index. For example, a
\field{vq_index_1_16} value of 3 corresponds to virtqueue index 6,
which maps to receiveq4.

Field \field{unclassified_queue} specifies the receive virtqueue id in which to
place unclassified packets.

Field \field{indirection_table} is an array of receive virtqueues ids.

A driver sets \field{max_tx_vq} to inform a device how many transmit virtqueues it may use (transmitq1\ldots transmitq \field{max_tx_vq}).

Fields \field{hash_key_length} and \field{hash_key_data} define the key to be used in hash calculation.

\drivernormative{\subparagraph}{Setting RSS parameters}{Device Types / Network Device / Device Operation / Control Virtqueue / Receive-side scaling (RSS) }

A driver MUST NOT send the VIRTIO_NET_CTRL_MQ_RSS_CONFIG command if the feature VIRTIO_NET_F_RSS has not been negotiated.

A driver MUST fill the \field{indirection_table} array only with
enabled receive virtqueues ids.

The number of entries in \field{indirection_table} (\field{indirection_table_mask} + 1) MUST be a power of two.

A driver MUST use \field{indirection_table_mask} values that are less than \field{rss_max_indirection_table_length} reported by a device.

A driver MUST NOT set any VIRTIO_NET_HASH_TYPE_ flags that are not supported by a device.

\devicenormative{\subparagraph}{RSS processing}{Device Types / Network Device / Device Operation / Control Virtqueue / Receive-side scaling (RSS) / RSS processing}
The device MUST determine the destination queue for a network packet as follows:
\begin{itemize}
\item Calculate the hash of the packet as defined in \ref{sec:Device Types / Network Device / Device Operation / Processing of Incoming Packets / Hash calculation for incoming packets}.
\item If the device did not calculate the hash for the specific packet, the device directs the packet to the receiveq specified by \field{unclassified_queue} of virtio_net_rss_config structure.
\item Apply \field{indirection_table_mask} to the calculated hash
and use the result as the index in the indirection table to get
the destination receive virtqueue id.
\item If the destination receive queue is being reset (See \ref{sec:Basic Facilities of a Virtio Device / Virtqueues / Virtqueue Reset}), the device MUST drop the packet.
\end{itemize}

\paragraph{RSS Context}\label{sec:Device Types / Network Device / Device Operation / Control Virtqueue / RSS Context}

An RSS context consists of configurable parameters specified by \ref{sec:Device Types / Network Device
/ Device Operation / Control Virtqueue / Receive-side scaling (RSS)}.

The RSS configuration supported by VIRTIO_NET_F_RSS is considered the default RSS configuration.

The device offers the feature VIRTIO_NET_F_RSS_CONTEXT if it supports one or multiple RSS contexts
(excluding the default RSS configuration) and configurable parameters.

\subparagraph{Querying RSS Context Capability}\label{sec:Device Types / Network Device / Device Operation / Control Virtqueue / RSS Context / Querying RSS Context Capability}

\begin{lstlisting}
#define VIRTNET_RSS_CTX_CTRL 9
 #define VIRTNET_RSS_CTX_CTRL_CAP_GET  0
 #define VIRTNET_RSS_CTX_CTRL_ADD      1
 #define VIRTNET_RSS_CTX_CTRL_MOD      2
 #define VIRTNET_RSS_CTX_CTRL_DEL      3

struct virtnet_rss_ctx_cap {
    le16 max_rss_contexts;
}
\end{lstlisting}

Field \field{max_rss_contexts} specifies the maximum number of RSS contexts \ref{sec:Device Types / Network Device /
Device Operation / Control Virtqueue / RSS Context} supported by the device.

The driver queries the RSS context capability of the device by sending the command VIRTNET_RSS_CTX_CTRL_CAP_GET
with the structure virtnet_rss_ctx_cap.

For the command VIRTNET_RSS_CTX_CTRL_CAP_GET, the structure virtnet_rss_ctx_cap is write-only for the device.

\subparagraph{Setting RSS Context Parameters}\label{sec:Device Types / Network Device / Device Operation / Control Virtqueue / RSS Context / Setting RSS Context Parameters}

\begin{lstlisting}
struct virtnet_rss_ctx_add_modify {
    le16 rss_ctx_id;
    u8 reserved[6];
    struct virtio_net_rss_config rss;
};

struct virtnet_rss_ctx_del {
    le16 rss_ctx_id;
};
\end{lstlisting}

RSS context parameters:
\begin{itemize}
\item  \field{rss_ctx_id}: ID of the specific RSS context.
\item  \field{rss}: RSS context parameters of the specific RSS context whose id is \field{rss_ctx_id}.
\end{itemize}

\field{reserved} is reserved and it is ignored by the device.

If the feature VIRTIO_NET_F_RSS_CONTEXT has been negotiated, the driver can send the following
VIRTNET_RSS_CTX_CTRL class commands:
\begin{enumerate}
\item VIRTNET_RSS_CTX_CTRL_ADD: use the structure virtnet_rss_ctx_add_modify to
       add an RSS context configured as \field{rss} and id as \field{rss_ctx_id} for the device.
\item VIRTNET_RSS_CTX_CTRL_MOD: use the structure virtnet_rss_ctx_add_modify to
       configure parameters of the RSS context whose id is \field{rss_ctx_id} as \field{rss} for the device.
\item VIRTNET_RSS_CTX_CTRL_DEL: use the structure virtnet_rss_ctx_del to delete
       the RSS context whose id is \field{rss_ctx_id} for the device.
\end{enumerate}

For commands VIRTNET_RSS_CTX_CTRL_ADD and VIRTNET_RSS_CTX_CTRL_MOD, the structure virtnet_rss_ctx_add_modify is read-only for the device.
For the command VIRTNET_RSS_CTX_CTRL_DEL, the structure virtnet_rss_ctx_del is read-only for the device.

\devicenormative{\subparagraph}{RSS Context}{Device Types / Network Device / Device Operation / Control Virtqueue / RSS Context}

The device MUST set \field{max_rss_contexts} to at least 1 if it offers VIRTIO_NET_F_RSS_CONTEXT.

Upon reset, the device MUST clear all previously configured RSS contexts.

\drivernormative{\subparagraph}{RSS Context}{Device Types / Network Device / Device Operation / Control Virtqueue / RSS Context}

The driver MUST have negotiated the VIRTIO_NET_F_RSS_CONTEXT feature when issuing the VIRTNET_RSS_CTX_CTRL class commands.

The driver MUST set \field{rss_ctx_id} to between 1 and \field{max_rss_contexts} inclusive.

The driver MUST NOT send the command VIRTIO_NET_CTRL_MQ_VQ_PAIRS_SET when the device has successfully configured at least one RSS context.

\paragraph{Offloads State Configuration}\label{sec:Device Types / Network Device / Device Operation / Control Virtqueue / Offloads State Configuration}

If the VIRTIO_NET_F_CTRL_GUEST_OFFLOADS feature is negotiated, the driver can
send control commands for dynamic offloads state configuration.

\subparagraph{Setting Offloads State}\label{sec:Device Types / Network Device / Device Operation / Control Virtqueue / Offloads State Configuration / Setting Offloads State}

To configure the offloads, the following layout structure and
definitions are used:

\begin{lstlisting}
le64 offloads;

#define VIRTIO_NET_F_GUEST_CSUM       1
#define VIRTIO_NET_F_GUEST_TSO4       7
#define VIRTIO_NET_F_GUEST_TSO6       8
#define VIRTIO_NET_F_GUEST_ECN        9
#define VIRTIO_NET_F_GUEST_UFO        10
#define VIRTIO_NET_F_GUEST_UDP_TUNNEL_GSO  46
#define VIRTIO_NET_F_GUEST_UDP_TUNNEL_GSO_CSUM 47
#define VIRTIO_NET_F_GUEST_USO4       54
#define VIRTIO_NET_F_GUEST_USO6       55

#define VIRTIO_NET_CTRL_GUEST_OFFLOADS       5
 #define VIRTIO_NET_CTRL_GUEST_OFFLOADS_SET   0
\end{lstlisting}

The class VIRTIO_NET_CTRL_GUEST_OFFLOADS has one command:
VIRTIO_NET_CTRL_GUEST_OFFLOADS_SET applies the new offloads configuration.

le64 value passed as command data is a bitmask, bits set define
offloads to be enabled, bits cleared - offloads to be disabled.

There is a corresponding device feature for each offload. Upon feature
negotiation corresponding offload gets enabled to preserve backward
compatibility.

\drivernormative{\subparagraph}{Setting Offloads State}{Device Types / Network Device / Device Operation / Control Virtqueue / Offloads State Configuration / Setting Offloads State}

A driver MUST NOT enable an offload for which the appropriate feature
has not been negotiated.

\subparagraph{Legacy Interface: Setting Offloads State}\label{sec:Device Types / Network Device / Device Operation / Control Virtqueue / Offloads State Configuration / Setting Offloads State / Legacy Interface: Setting Offloads State}
When using the legacy interface, transitional devices and drivers
MUST format \field{offloads}
according to the native endian of the guest rather than
(necessarily when not using the legacy interface) little-endian.


\paragraph{Notifications Coalescing}\label{sec:Device Types / Network Device / Device Operation / Control Virtqueue / Notifications Coalescing}

If the VIRTIO_NET_F_NOTF_COAL feature is negotiated, the driver can
send commands VIRTIO_NET_CTRL_NOTF_COAL_TX_SET and VIRTIO_NET_CTRL_NOTF_COAL_RX_SET
for notification coalescing.

If the VIRTIO_NET_F_VQ_NOTF_COAL feature is negotiated, the driver can
send commands VIRTIO_NET_CTRL_NOTF_COAL_VQ_SET and VIRTIO_NET_CTRL_NOTF_COAL_VQ_GET
for virtqueue notification coalescing.

\begin{lstlisting}
struct virtio_net_ctrl_coal {
    le32 max_packets;
    le32 max_usecs;
};

struct virtio_net_ctrl_coal_vq {
    le16 vq_index;
    le16 reserved;
    struct virtio_net_ctrl_coal coal;
};

#define VIRTIO_NET_CTRL_NOTF_COAL 6
 #define VIRTIO_NET_CTRL_NOTF_COAL_TX_SET  0
 #define VIRTIO_NET_CTRL_NOTF_COAL_RX_SET 1
 #define VIRTIO_NET_CTRL_NOTF_COAL_VQ_SET 2
 #define VIRTIO_NET_CTRL_NOTF_COAL_VQ_GET 3
\end{lstlisting}

Coalescing parameters:
\begin{itemize}
\item \field{vq_index}: The virtqueue index of an enabled transmit or receive virtqueue.
\item \field{max_usecs} for RX: Maximum number of microseconds to delay a RX notification.
\item \field{max_usecs} for TX: Maximum number of microseconds to delay a TX notification.
\item \field{max_packets} for RX: Maximum number of packets to receive before a RX notification.
\item \field{max_packets} for TX: Maximum number of packets to send before a TX notification.
\end{itemize}

\field{reserved} is reserved and it is ignored by the device.

Read/Write attributes for coalescing parameters:
\begin{itemize}
\item For commands VIRTIO_NET_CTRL_NOTF_COAL_TX_SET and VIRTIO_NET_CTRL_NOTF_COAL_RX_SET, the structure virtio_net_ctrl_coal is write-only for the driver.
\item For the command VIRTIO_NET_CTRL_NOTF_COAL_VQ_SET, the structure virtio_net_ctrl_coal_vq is write-only for the driver.
\item For the command VIRTIO_NET_CTRL_NOTF_COAL_VQ_GET, \field{vq_index} and \field{reserved} are write-only
      for the driver, and the structure virtio_net_ctrl_coal is read-only for the driver.
\end{itemize}

The class VIRTIO_NET_CTRL_NOTF_COAL has the following commands:
\begin{enumerate}
\item VIRTIO_NET_CTRL_NOTF_COAL_TX_SET: use the structure virtio_net_ctrl_coal to set the \field{max_usecs} and \field{max_packets} parameters for all transmit virtqueues.
\item VIRTIO_NET_CTRL_NOTF_COAL_RX_SET: use the structure virtio_net_ctrl_coal to set the \field{max_usecs} and \field{max_packets} parameters for all receive virtqueues.
\item VIRTIO_NET_CTRL_NOTF_COAL_VQ_SET: use the structure virtio_net_ctrl_coal_vq to set the \field{max_usecs} and \field{max_packets} parameters
                                        for an enabled transmit/receive virtqueue whose index is \field{vq_index}.
\item VIRTIO_NET_CTRL_NOTF_COAL_VQ_GET: use the structure virtio_net_ctrl_coal_vq to get the \field{max_usecs} and \field{max_packets} parameters
                                        for an enabled transmit/receive virtqueue whose index is \field{vq_index}.
\end{enumerate}

The device may generate notifications more or less frequently than specified by set commands of the VIRTIO_NET_CTRL_NOTF_COAL class.

If coalescing parameters are being set, the device applies the last coalescing parameters set for a
virtqueue, regardless of the command used to set the parameters. Use the following command sequence
with two pairs of virtqueues as an example:
Each of the following commands sets \field{max_usecs} and \field{max_packets} parameters for virtqueues.
\begin{itemize}
\item Command1: VIRTIO_NET_CTRL_NOTF_COAL_RX_SET sets coalescing parameters for virtqueues having index 0 and index 2. Virtqueues having index 1 and index 3 retain their previous parameters.
\item Command2: VIRTIO_NET_CTRL_NOTF_COAL_VQ_SET with \field{vq_index} = 0 sets coalescing parameters for virtqueue having index 0. Virtqueue having index 2 retains the parameters from command1.
\item Command3: VIRTIO_NET_CTRL_NOTF_COAL_VQ_GET with \field{vq_index} = 0, the device responds with coalescing parameters of vq_index 0 set by command2.
\item Command4: VIRTIO_NET_CTRL_NOTF_COAL_VQ_SET with \field{vq_index} = 1 sets coalescing parameters for virtqueue having index 1. Virtqueue having index 3 retains its previous parameters.
\item Command5: VIRTIO_NET_CTRL_NOTF_COAL_TX_SET sets coalescing parameters for virtqueues having index 1 and index 3, and overrides the parameters set by command4.
\item Command6: VIRTIO_NET_CTRL_NOTF_COAL_VQ_GET with \field{vq_index} = 1, the device responds with coalescing parameters of index 1 set by command5.
\end{itemize}

\subparagraph{Operation}\label{sec:Device Types / Network Device / Device Operation / Control Virtqueue / Notifications Coalescing / Operation}

The device sends a used buffer notification once the notification conditions are met and if the notifications are not suppressed as explained in \ref{sec:Basic Facilities of a Virtio Device / Virtqueues / Used Buffer Notification Suppression}.

When the device has non-zero \field{max_usecs} and non-zero \field{max_packets}, it starts counting microseconds and packets upon receiving/sending a packet.
The device counts packets and microseconds for each receive virtqueue and transmit virtqueue separately.
In this case, the notification conditions are met when \field{max_usecs} microseconds elapse, or upon sending/receiving \field{max_packets} packets, whichever happens first.
Afterwards, the device waits for the next packet and starts counting packets and microseconds again.

When the device has \field{max_usecs} = 0 or \field{max_packets} = 0, the notification conditions are met after every packet received/sent.

\subparagraph{RX Example}\label{sec:Device Types / Network Device / Device Operation / Control Virtqueue / Notifications Coalescing / RX Example}

If, for example:
\begin{itemize}
\item \field{max_usecs} = 10.
\item \field{max_packets} = 15.
\end{itemize}
then each receive virtqueue of a device will operate as follows:
\begin{itemize}
\item The device will count packets received on each virtqueue until it accumulates 15, or until 10 microseconds elapsed since the first one was received.
\item If the notifications are not suppressed by the driver, the device will send an used buffer notification, otherwise, the device will not send an used buffer notification as long as the notifications are suppressed.
\end{itemize}

\subparagraph{TX Example}\label{sec:Device Types / Network Device / Device Operation / Control Virtqueue / Notifications Coalescing / TX Example}

If, for example:
\begin{itemize}
\item \field{max_usecs} = 10.
\item \field{max_packets} = 15.
\end{itemize}
then each transmit virtqueue of a device will operate as follows:
\begin{itemize}
\item The device will count packets sent on each virtqueue until it accumulates 15, or until 10 microseconds elapsed since the first one was sent.
\item If the notifications are not suppressed by the driver, the device will send an used buffer notification, otherwise, the device will not send an used buffer notification as long as the notifications are suppressed.
\end{itemize}

\subparagraph{Notifications When Coalescing Parameters Change}\label{sec:Device Types / Network Device / Device Operation / Control Virtqueue / Notifications Coalescing / Notifications When Coalescing Parameters Change}

When the coalescing parameters of a device change, the device needs to check if the new notification conditions are met and send a used buffer notification if so.

For example, \field{max_packets} = 15 for a device with a single transmit virtqueue: if the device sends 10 packets and afterwards receives a
VIRTIO_NET_CTRL_NOTF_COAL_TX_SET command with \field{max_packets} = 8, then the notification condition is immediately considered to be met;
the device needs to immediately send a used buffer notification, if the notifications are not suppressed by the driver.

\drivernormative{\subparagraph}{Notifications Coalescing}{Device Types / Network Device / Device Operation / Control Virtqueue / Notifications Coalescing}

The driver MUST set \field{vq_index} to the virtqueue index of an enabled transmit or receive virtqueue.

The driver MUST have negotiated the VIRTIO_NET_F_NOTF_COAL feature when issuing commands VIRTIO_NET_CTRL_NOTF_COAL_TX_SET and VIRTIO_NET_CTRL_NOTF_COAL_RX_SET.

The driver MUST have negotiated the VIRTIO_NET_F_VQ_NOTF_COAL feature when issuing commands VIRTIO_NET_CTRL_NOTF_COAL_VQ_SET and VIRTIO_NET_CTRL_NOTF_COAL_VQ_GET.

The driver MUST ignore the values of coalescing parameters received from the VIRTIO_NET_CTRL_NOTF_COAL_VQ_GET command if the device responds with VIRTIO_NET_ERR.

\devicenormative{\subparagraph}{Notifications Coalescing}{Device Types / Network Device / Device Operation / Control Virtqueue / Notifications Coalescing}

The device MUST ignore \field{reserved}.

The device SHOULD respond to VIRTIO_NET_CTRL_NOTF_COAL_TX_SET and VIRTIO_NET_CTRL_NOTF_COAL_RX_SET commands with VIRTIO_NET_ERR if it was not able to change the parameters.

The device MUST respond to the VIRTIO_NET_CTRL_NOTF_COAL_VQ_SET command with VIRTIO_NET_ERR if it was not able to change the parameters.

The device MUST respond to VIRTIO_NET_CTRL_NOTF_COAL_VQ_SET and VIRTIO_NET_CTRL_NOTF_COAL_VQ_GET commands with
VIRTIO_NET_ERR if the designated virtqueue is not an enabled transmit or receive virtqueue.

Upon disabling and re-enabling a transmit virtqueue, the device MUST set the coalescing parameters of the virtqueue
to those configured through the VIRTIO_NET_CTRL_NOTF_COAL_TX_SET command, or, if the driver did not set any TX coalescing parameters, to 0.

Upon disabling and re-enabling a receive virtqueue, the device MUST set the coalescing parameters of the virtqueue
to those configured through the VIRTIO_NET_CTRL_NOTF_COAL_RX_SET command, or, if the driver did not set any RX coalescing parameters, to 0.

The behavior of the device in response to set commands of the VIRTIO_NET_CTRL_NOTF_COAL class is best-effort:
the device MAY generate notifications more or less frequently than specified.

A device SHOULD NOT send used buffer notifications to the driver if the notifications are suppressed, even if the notification conditions are met.

Upon reset, a device MUST initialize all coalescing parameters to 0.

\paragraph{Device Statistics}\label{sec:Device Types / Network Device / Device Operation / Control Virtqueue / Device Statistics}

If the VIRTIO_NET_F_DEVICE_STATS feature is negotiated, the driver can obtain
device statistics from the device by using the following command.

Different types of virtqueues have different statistics. The statistics of the
receiveq are different from those of the transmitq.

The statistics of a certain type of virtqueue are also divided into multiple types
because different types require different features. This enables the expansion
of new statistics.

In one command, the driver can obtain the statistics of one or multiple virtqueues.
Additionally, the driver can obtain multiple type statistics of each virtqueue.

\subparagraph{Query Statistic Capabilities}\label{sec:Device Types / Network Device / Device Operation / Control Virtqueue / Device Statistics / Query Statistic Capabilities}

\begin{lstlisting}
#define VIRTIO_NET_CTRL_STATS         8
#define VIRTIO_NET_CTRL_STATS_QUERY   0
#define VIRTIO_NET_CTRL_STATS_GET     1

struct virtio_net_stats_capabilities {

#define VIRTIO_NET_STATS_TYPE_CVQ       (1 << 32)

#define VIRTIO_NET_STATS_TYPE_RX_BASIC  (1 << 0)
#define VIRTIO_NET_STATS_TYPE_RX_CSUM   (1 << 1)
#define VIRTIO_NET_STATS_TYPE_RX_GSO    (1 << 2)
#define VIRTIO_NET_STATS_TYPE_RX_SPEED  (1 << 3)

#define VIRTIO_NET_STATS_TYPE_TX_BASIC  (1 << 16)
#define VIRTIO_NET_STATS_TYPE_TX_CSUM   (1 << 17)
#define VIRTIO_NET_STATS_TYPE_TX_GSO    (1 << 18)
#define VIRTIO_NET_STATS_TYPE_TX_SPEED  (1 << 19)

    le64 supported_stats_types[1];
}
\end{lstlisting}

To obtain device statistic capability, use the VIRTIO_NET_CTRL_STATS_QUERY
command. When the command completes successfully, \field{command-specific-result}
is in the format of \field{struct virtio_net_stats_capabilities}.

\subparagraph{Get Statistics}\label{sec:Device Types / Network Device / Device Operation / Control Virtqueue / Device Statistics / Get Statistics}

\begin{lstlisting}
struct virtio_net_ctrl_queue_stats {
       struct {
           le16 vq_index;
           le16 reserved[3];
           le64 types_bitmap[1];
       } stats[];
};

struct virtio_net_stats_reply_hdr {
#define VIRTIO_NET_STATS_TYPE_REPLY_CVQ       32

#define VIRTIO_NET_STATS_TYPE_REPLY_RX_BASIC  0
#define VIRTIO_NET_STATS_TYPE_REPLY_RX_CSUM   1
#define VIRTIO_NET_STATS_TYPE_REPLY_RX_GSO    2
#define VIRTIO_NET_STATS_TYPE_REPLY_RX_SPEED  3

#define VIRTIO_NET_STATS_TYPE_REPLY_TX_BASIC  16
#define VIRTIO_NET_STATS_TYPE_REPLY_TX_CSUM   17
#define VIRTIO_NET_STATS_TYPE_REPLY_TX_GSO    18
#define VIRTIO_NET_STATS_TYPE_REPLY_TX_SPEED  19
    u8 type;
    u8 reserved;
    le16 vq_index;
    le16 reserved1;
    le16 size;
}
\end{lstlisting}

To obtain device statistics, use the VIRTIO_NET_CTRL_STATS_GET command with the
\field{command-specific-data} which is in the format of
\field{struct virtio_net_ctrl_queue_stats}. When the command completes
successfully, \field{command-specific-result} contains multiple statistic
results, each statistic result has the \field{struct virtio_net_stats_reply_hdr}
as the header.

The fields of the \field{struct virtio_net_ctrl_queue_stats}:
\begin{description}
    \item [vq_index]
        The index of the virtqueue to obtain the statistics.

    \item [types_bitmap]
        This is a bitmask of the types of statistics to be obtained. Therefore, a
        \field{stats} inside \field{struct virtio_net_ctrl_queue_stats} may
        indicate multiple statistic replies for the virtqueue.
\end{description}

The fields of the \field{struct virtio_net_stats_reply_hdr}:
\begin{description}
    \item [type]
        The type of the reply statistic.

    \item [vq_index]
        The virtqueue index of the reply statistic.

    \item [size]
        The number of bytes for the statistics entry including size of \field{struct virtio_net_stats_reply_hdr}.

\end{description}

\subparagraph{Controlq Statistics}\label{sec:Device Types / Network Device / Device Operation / Control Virtqueue / Device Statistics / Controlq Statistics}

The structure corresponding to the controlq statistics is
\field{struct virtio_net_stats_cvq}. The corresponding type is
VIRTIO_NET_STATS_TYPE_CVQ. This is for the controlq.

\begin{lstlisting}
struct virtio_net_stats_cvq {
    struct virtio_net_stats_reply_hdr hdr;

    le64 command_num;
    le64 ok_num;
};
\end{lstlisting}

\begin{description}
    \item [command_num]
        The number of commands received by the device including the current command.

    \item [ok_num]
        The number of commands completed successfully by the device including the current command.
\end{description}


\subparagraph{Receiveq Basic Statistics}\label{sec:Device Types / Network Device / Device Operation / Control Virtqueue / Device Statistics / Receiveq Basic Statistics}

The structure corresponding to the receiveq basic statistics is
\field{struct virtio_net_stats_rx_basic}. The corresponding type is
VIRTIO_NET_STATS_TYPE_RX_BASIC. This is for the receiveq.

Receiveq basic statistics do not require any feature. As long as the device supports
VIRTIO_NET_F_DEVICE_STATS, the following are the receiveq basic statistics.

\begin{lstlisting}
struct virtio_net_stats_rx_basic {
    struct virtio_net_stats_reply_hdr hdr;

    le64 rx_notifications;

    le64 rx_packets;
    le64 rx_bytes;

    le64 rx_interrupts;

    le64 rx_drops;
    le64 rx_drop_overruns;
};
\end{lstlisting}

The packets described below were all presented on the specified virtqueue.
\begin{description}
    \item [rx_notifications]
        The number of driver notifications received by the device for this
        receiveq.

    \item [rx_packets]
        This is the number of packets passed to the driver by the device.

    \item [rx_bytes]
        This is the bytes of packets passed to the driver by the device.

    \item [rx_interrupts]
        The number of interrupts generated by the device for this receiveq.

    \item [rx_drops]
        This is the number of packets dropped by the device. The count includes
        all types of packets dropped by the device.

    \item [rx_drop_overruns]
        This is the number of packets dropped by the device when no more
        descriptors were available.

\end{description}

\subparagraph{Transmitq Basic Statistics}\label{sec:Device Types / Network Device / Device Operation / Control Virtqueue / Device Statistics / Transmitq Basic Statistics}

The structure corresponding to the transmitq basic statistics is
\field{struct virtio_net_stats_tx_basic}. The corresponding type is
VIRTIO_NET_STATS_TYPE_TX_BASIC. This is for the transmitq.

Transmitq basic statistics do not require any feature. As long as the device supports
VIRTIO_NET_F_DEVICE_STATS, the following are the transmitq basic statistics.

\begin{lstlisting}
struct virtio_net_stats_tx_basic {
    struct virtio_net_stats_reply_hdr hdr;

    le64 tx_notifications;

    le64 tx_packets;
    le64 tx_bytes;

    le64 tx_interrupts;

    le64 tx_drops;
    le64 tx_drop_malformed;
};
\end{lstlisting}

The packets described below are all for a specific virtqueue.
\begin{description}
    \item [tx_notifications]
        The number of driver notifications received by the device for this
        transmitq.

    \item [tx_packets]
        This is the number of packets sent by the device (not the packets
        got from the driver).

    \item [tx_bytes]
        This is the number of bytes sent by the device for all the sent packets
        (not the bytes sent got from the driver).

    \item [tx_interrupts]
        The number of interrupts generated by the device for this transmitq.

    \item [tx_drops]
        The number of packets dropped by the device. The count includes all
        types of packets dropped by the device.

    \item [tx_drop_malformed]
        The number of packets dropped by the device, when the descriptors are
        malformed. For example, the buffer is too short.
\end{description}

\subparagraph{Receiveq CSUM Statistics}\label{sec:Device Types / Network Device / Device Operation / Control Virtqueue / Device Statistics / Receiveq CSUM Statistics}

The structure corresponding to the receiveq checksum statistics is
\field{struct virtio_net_stats_rx_csum}. The corresponding type is
VIRTIO_NET_STATS_TYPE_RX_CSUM. This is for the receiveq.

Only after the VIRTIO_NET_F_GUEST_CSUM is negotiated, the receiveq checksum
statistics can be obtained.

\begin{lstlisting}
struct virtio_net_stats_rx_csum {
    struct virtio_net_stats_reply_hdr hdr;

    le64 rx_csum_valid;
    le64 rx_needs_csum;
    le64 rx_csum_none;
    le64 rx_csum_bad;
};
\end{lstlisting}

The packets described below were all presented on the specified virtqueue.
\begin{description}
    \item [rx_csum_valid]
        The number of packets with VIRTIO_NET_HDR_F_DATA_VALID.

    \item [rx_needs_csum]
        The number of packets with VIRTIO_NET_HDR_F_NEEDS_CSUM.

    \item [rx_csum_none]
        The number of packets without hardware checksum. The packet here refers
        to the non-TCP/UDP packet that the device cannot recognize.

    \item [rx_csum_bad]
        The number of packets with checksum mismatch.

\end{description}

\subparagraph{Transmitq CSUM Statistics}\label{sec:Device Types / Network Device / Device Operation / Control Virtqueue / Device Statistics / Transmitq CSUM Statistics}

The structure corresponding to the transmitq checksum statistics is
\field{struct virtio_net_stats_tx_csum}. The corresponding type is
VIRTIO_NET_STATS_TYPE_TX_CSUM. This is for the transmitq.

Only after the VIRTIO_NET_F_CSUM is negotiated, the transmitq checksum
statistics can be obtained.

The following are the transmitq checksum statistics:

\begin{lstlisting}
struct virtio_net_stats_tx_csum {
    struct virtio_net_stats_reply_hdr hdr;

    le64 tx_csum_none;
    le64 tx_needs_csum;
};
\end{lstlisting}

The packets described below are all for a specific virtqueue.
\begin{description}
    \item [tx_csum_none]
        The number of packets which do not require hardware checksum.

    \item [tx_needs_csum]
        The number of packets which require checksum calculation by the device.

\end{description}

\subparagraph{Receiveq GSO Statistics}\label{sec:Device Types / Network Device / Device Operation / Control Virtqueue / Device Statistics / Receiveq GSO Statistics}

The structure corresponding to the receivq GSO statistics is
\field{struct virtio_net_stats_rx_gso}. The corresponding type is
VIRTIO_NET_STATS_TYPE_RX_GSO. This is for the receiveq.

If one or more of the VIRTIO_NET_F_GUEST_TSO4, VIRTIO_NET_F_GUEST_TSO6
have been negotiated, the receiveq GSO statistics can be obtained.

GSO packets refer to packets passed by the device to the driver where
\field{gso_type} is not VIRTIO_NET_HDR_GSO_NONE.

\begin{lstlisting}
struct virtio_net_stats_rx_gso {
    struct virtio_net_stats_reply_hdr hdr;

    le64 rx_gso_packets;
    le64 rx_gso_bytes;
    le64 rx_gso_packets_coalesced;
    le64 rx_gso_bytes_coalesced;
};
\end{lstlisting}

The packets described below were all presented on the specified virtqueue.
\begin{description}
    \item [rx_gso_packets]
        The number of the GSO packets received by the device.

    \item [rx_gso_bytes]
        The bytes of the GSO packets received by the device.
        This includes the header size of the GSO packet.

    \item [rx_gso_packets_coalesced]
        The number of the GSO packets coalesced by the device.

    \item [rx_gso_bytes_coalesced]
        The bytes of the GSO packets coalesced by the device.
        This includes the header size of the GSO packet.
\end{description}

\subparagraph{Transmitq GSO Statistics}\label{sec:Device Types / Network Device / Device Operation / Control Virtqueue / Device Statistics / Transmitq GSO Statistics}

The structure corresponding to the transmitq GSO statistics is
\field{struct virtio_net_stats_tx_gso}. The corresponding type is
VIRTIO_NET_STATS_TYPE_TX_GSO. This is for the transmitq.

If one or more of the VIRTIO_NET_F_HOST_TSO4, VIRTIO_NET_F_HOST_TSO6,
VIRTIO_NET_F_HOST_USO options have been negotiated, the transmitq GSO statistics
can be obtained.

GSO packets refer to packets passed by the driver to the device where
\field{gso_type} is not VIRTIO_NET_HDR_GSO_NONE.
See more \ref{sec:Device Types / Network Device / Device Operation / Packet
Transmission}.

\begin{lstlisting}
struct virtio_net_stats_tx_gso {
    struct virtio_net_stats_reply_hdr hdr;

    le64 tx_gso_packets;
    le64 tx_gso_bytes;
    le64 tx_gso_segments;
    le64 tx_gso_segments_bytes;
    le64 tx_gso_packets_noseg;
    le64 tx_gso_bytes_noseg;
};
\end{lstlisting}

The packets described below are all for a specific virtqueue.
\begin{description}
    \item [tx_gso_packets]
        The number of the GSO packets sent by the device.

    \item [tx_gso_bytes]
        The bytes of the GSO packets sent by the device.

    \item [tx_gso_segments]
        The number of segments prepared from GSO packets.

    \item [tx_gso_segments_bytes]
        The bytes of segments prepared from GSO packets.

    \item [tx_gso_packets_noseg]
        The number of the GSO packets without segmentation.

    \item [tx_gso_bytes_noseg]
        The bytes of the GSO packets without segmentation.

\end{description}

\subparagraph{Receiveq Speed Statistics}\label{sec:Device Types / Network Device / Device Operation / Control Virtqueue / Device Statistics / Receiveq Speed Statistics}

The structure corresponding to the receiveq speed statistics is
\field{struct virtio_net_stats_rx_speed}. The corresponding type is
VIRTIO_NET_STATS_TYPE_RX_SPEED. This is for the receiveq.

The device has the allowance for the speed. If VIRTIO_NET_F_SPEED_DUPLEX has
been negotiated, the driver can get this by \field{speed}. When the received
packets bitrate exceeds the \field{speed}, some packets may be dropped by the
device.

\begin{lstlisting}
struct virtio_net_stats_rx_speed {
    struct virtio_net_stats_reply_hdr hdr;

    le64 rx_packets_allowance_exceeded;
    le64 rx_bytes_allowance_exceeded;
};
\end{lstlisting}

The packets described below were all presented on the specified virtqueue.
\begin{description}
    \item [rx_packets_allowance_exceeded]
        The number of the packets dropped by the device due to the received
        packets bitrate exceeding the \field{speed}.

    \item [rx_bytes_allowance_exceeded]
        The bytes of the packets dropped by the device due to the received
        packets bitrate exceeding the \field{speed}.

\end{description}

\subparagraph{Transmitq Speed Statistics}\label{sec:Device Types / Network Device / Device Operation / Control Virtqueue / Device Statistics / Transmitq Speed Statistics}

The structure corresponding to the transmitq speed statistics is
\field{struct virtio_net_stats_tx_speed}. The corresponding type is
VIRTIO_NET_STATS_TYPE_TX_SPEED. This is for the transmitq.

The device has the allowance for the speed. If VIRTIO_NET_F_SPEED_DUPLEX has
been negotiated, the driver can get this by \field{speed}. When the transmit
packets bitrate exceeds the \field{speed}, some packets may be dropped by the
device.

\begin{lstlisting}
struct virtio_net_stats_tx_speed {
    struct virtio_net_stats_reply_hdr hdr;

    le64 tx_packets_allowance_exceeded;
    le64 tx_bytes_allowance_exceeded;
};
\end{lstlisting}

The packets described below were all presented on the specified virtqueue.
\begin{description}
    \item [tx_packets_allowance_exceeded]
        The number of the packets dropped by the device due to the transmit packets
        bitrate exceeding the \field{speed}.

    \item [tx_bytes_allowance_exceeded]
        The bytes of the packets dropped by the device due to the transmit packets
        bitrate exceeding the \field{speed}.

\end{description}

\devicenormative{\subparagraph}{Device Statistics}{Device Types / Network Device / Device Operation / Control Virtqueue / Device Statistics}

When the VIRTIO_NET_F_DEVICE_STATS feature is negotiated, the device MUST reply
to the command VIRTIO_NET_CTRL_STATS_QUERY with the
\field{struct virtio_net_stats_capabilities}. \field{supported_stats_types}
includes all the statistic types supported by the device.

If \field{struct virtio_net_ctrl_queue_stats} is incorrect (such as the
following), the device MUST set \field{ack} to VIRTIO_NET_ERR. Even if there is
only one error, the device MUST fail the entire command.
\begin{itemize}
    \item \field{vq_index} exceeds the queue range.
    \item \field{types_bitmap} contains unknown types.
    \item One or more of the bits present in \field{types_bitmap} is not valid
        for the specified virtqueue.
    \item The feature corresponding to the specified \field{types_bitmap} was
        not negotiated.
\end{itemize}

The device MUST set the actual size of the bytes occupied by the reply to the
\field{size} of the \field{hdr}. And the device MUST set the \field{type} and
the \field{vq_index} of the statistic header.

The \field{command-specific-result} buffer allocated by the driver may be
smaller or bigger than all the statistics specified by
\field{struct virtio_net_ctrl_queue_stats}. The device MUST fill up only upto
the valid bytes.

The statistics counter replied by the device MUST wrap around to zero by the
device on the overflow.

\drivernormative{\subparagraph}{Device Statistics}{Device Types / Network Device / Device Operation / Control Virtqueue / Device Statistics}

The types contained in the \field{types_bitmap} MUST be queried from the device
via command VIRTIO_NET_CTRL_STATS_QUERY.

\field{types_bitmap} in \field{struct virtio_net_ctrl_queue_stats} MUST be valid to the
vq specified by \field{vq_index}.

The \field{command-specific-result} buffer allocated by the driver MUST have
enough capacity to store all the statistics reply headers defined in
\field{struct virtio_net_ctrl_queue_stats}. If the
\field{command-specific-result} buffer is fully utilized by the device but some
replies are missed, it is possible that some statistics may exceed the capacity
of the driver's records. In such cases, the driver should allocate additional
space for the \field{command-specific-result} buffer.

\subsubsection{Flow filter}\label{sec:Device Types / Network Device / Device Operation / Flow filter}

A network device can support one or more flow filter rules. Each flow filter rule
is applied by matching a packet and then taking an action, such as directing the packet
to a specific receiveq or dropping the packet. An example of a match is
matching on specific source and destination IP addresses.

A flow filter rule is a device resource object that consists of a key,
a processing priority, and an action to either direct a packet to a
receive queue or drop the packet.

Each rule uses a classifier. The key is matched against the packet using
a classifier, defining which fields in the packet are matched.
A classifier resource object consists of one or more field selectors, each with
a type that specifies the header fields to be matched against, and a mask.
The mask can match whole fields or parts of a field in a header. Each
rule resource object depends on the classifier resource object.

When a packet is received, relevant fields are extracted
(in the same way) from both the packet and the key according to the
classifier. The resulting field contents are then compared -
if they are identical the rule action is taken, if they are not, the rule is ignored.

Multiple flow filter rules are part of a group. The rule resource object
depends on the group. Each rule within a
group has a rule priority, and each group also has a group priority. For a
packet, a group with the highest priority is selected first. Within a group,
rules are applied from highest to lowest priority, until one of the rules
matches the packet and an action is taken. If all the rules within a group
are ignored, the group with the next highest priority is selected, and so on.

The device and the driver indicates flow filter resource limits using the capability
\ref{par:Device Types / Network Device / Device Operation / Flow filter / Device and driver capabilities / VIRTIO-NET-FF-RESOURCE-CAP} specifying the limits on the number of flow filter rule,
group and classifier resource objects. The capability \ref{par:Device Types / Network Device / Device Operation / Flow filter / Device and driver capabilities / VIRTIO-NET-FF-SELECTOR-CAP} specifies which selectors the device supports.
The driver indicates the selectors it is using by setting the flow
filter selector capability, prior to adding any resource objects.

The capability \ref{par:Device Types / Network Device / Device Operation / Flow filter / Device and driver capabilities / VIRTIO-NET-FF-ACTION-CAP} specifies which actions the device supports.

The driver controls the flow filter rule, classifier and group resource objects using
administration commands described in
\ref{sec:Basic Facilities of a Virtio Device / Device groups / Group administration commands / Device resource objects}.

\paragraph{Packet processing order}\label{sec:sec:Device Types / Network Device / Device Operation / Flow filter / Packet processing order}

Note that flow filter rules are applied after MAC/VLAN filtering. Flow filter
rules take precedence over steering: if a flow filter rule results in an action,
the steering configuration does not apply. The steering configuration only applies
to packets for which no flow filter rule action was performed. For example,
incoming packets can be processed in the following order:

\begin{itemize}
\item apply steering configuration received using control virtqueue commands
      VIRTIO_NET_CTRL_RX, VIRTIO_NET_CTRL_MAC and VIRTIO_NET_CTRL_VLAN.
\item apply flow filter rules if any.
\item if no filter rule applied, apply steering configuration received using command
      VIRTIO_NET_CTRL_MQ_RSS_CONFIG or as per automatic receive steering.
\end{itemize}

Some incoming packet processing examples:
\begin{itemize}
\item If the packet is dropped by the flow filter rule, RSS
      steering is ignored for the packet.
\item If the packet is directed to a specific receiveq using flow filter rule,
      the RSS steering is ignored for the packet.
\item If a packet is dropped due to the VIRTIO_NET_CTRL_MAC configuration,
      both flow filter rules and the RSS steering are ignored for the packet.
\item If a packet does not match any flow filter rules,
      the RSS steering is used to select the receiveq for the packet (if enabled).
\item If there are two flow filter groups configured as group_A and group_B
      with respective group priorities as 4, and 5; flow filter rules of
      group_B are applied first having highest group priority, if there is a match,
      the flow filter rules of group_A are ignored; if there is no match for
      the flow filter rules in group_B, the flow filter rules of next level group_A are applied.
\end{itemize}

\paragraph{Device and driver capabilities}
\label{par:Device Types / Network Device / Device Operation / Flow filter / Device and driver capabilities}

\subparagraph{VIRTIO_NET_FF_RESOURCE_CAP}
\label{par:Device Types / Network Device / Device Operation / Flow filter / Device and driver capabilities / VIRTIO-NET-FF-RESOURCE-CAP}

The capability VIRTIO_NET_FF_RESOURCE_CAP indicates the flow filter resource limits.
\field{cap_specific_data} is in the format
\field{struct virtio_net_ff_cap_data}.

\begin{lstlisting}
struct virtio_net_ff_cap_data {
        le32 groups_limit;
        le32 selectors_limit;
        le32 rules_limit;
        le32 rules_per_group_limit;
        u8 last_rule_priority;
        u8 selectors_per_classifier_limit;
};
\end{lstlisting}

\field{groups_limit}, and \field{selectors_limit} represent the maximum
number of flow filter groups and selectors, respectively, that the driver can create.
 \field{rules_limit} is the maximum number of
flow fiilter rules that the driver can create across all the groups.
\field{rules_per_group_limit} is the maximum number of flow filter rules that the driver
can create for each flow filter group.

\field{last_rule_priority} is the highest priority that can be assigned to a
flow filter rule.

\field{selectors_per_classifier_limit} is the maximum number of selectors
that a classifier can have.

\subparagraph{VIRTIO_NET_FF_SELECTOR_CAP}
\label{par:Device Types / Network Device / Device Operation / Flow filter / Device and driver capabilities / VIRTIO-NET-FF-SELECTOR-CAP}

The capability VIRTIO_NET_FF_SELECTOR_CAP lists the supported selectors and the
supported packet header fields for each selector.
\field{cap_specific_data} is in the format \field{struct virtio_net_ff_cap_mask_data}.

\begin{lstlisting}[label={lst:Device Types / Network Device / Device Operation / Flow filter / Device and driver capabilities / VIRTIO-NET-FF-SELECTOR-CAP / virtio-net-ff-selector}]
struct virtio_net_ff_selector {
        u8 type;
        u8 flags;
        u8 reserved[2];
        u8 length;
        u8 reserved1[3];
        u8 mask[];
};

struct virtio_net_ff_cap_mask_data {
        u8 count;
        u8 reserved[7];
        struct virtio_net_ff_selector selectors[];
};

#define VIRTIO_NET_FF_MASK_F_PARTIAL_MASK (1 << 0)
\end{lstlisting}

\field{count} indicates number of valid entries in the \field{selectors} array.
\field{selectors[]} is an array of supported selectors. Within each array entry:
\field{type} specifies the type of the packet header, as defined in table
\ref{table:Device Types / Network Device / Device Operation / Flow filter / Device and driver capabilities / VIRTIO-NET-FF-SELECTOR-CAP / flow filter selector types}. \field{mask} specifies which fields of the
packet header can be matched in a flow filter rule.

Each \field{type} is also listed in table
\ref{table:Device Types / Network Device / Device Operation / Flow filter / Device and driver capabilities / VIRTIO-NET-FF-SELECTOR-CAP / flow filter selector types}. \field{mask} is a byte array
in network byte order. For example, when \field{type} is VIRTIO_NET_FF_MASK_TYPE_IPV6,
the \field{mask} is in the format \hyperref[intro:IPv6-Header-Format]{IPv6 Header Format}.

If partial masking is not set, then all bits in each field have to be either all 0s
to ignore this field or all 1s to match on this field. If partial masking is set,
then any combination of bits can bit set to match on these bits.
For example, when a selector \field{type} is VIRTIO_NET_FF_MASK_TYPE_ETH, if
\field{mask[0-12]} are zero and \field{mask[13-14]} are 0xff (all 1s), it
indicates that matching is only supported for \field{EtherType} of
\field{Ethernet MAC frame}, matching is not supported for
\field{Destination Address} and \field{Source Address}.

The entries in the array \field{selectors} are ordered by
\field{type}, with each \field{type} value only appearing once.

\field{length} is the length of a dynamic array \field{mask} in bytes.
\field{reserved} and \field{reserved1} are reserved and set to zero.

\begin{table}[H]
\caption{Flow filter selector types}
\label{table:Device Types / Network Device / Device Operation / Flow filter / Device and driver capabilities / VIRTIO-NET-FF-SELECTOR-CAP / flow filter selector types}
\begin{tabularx}{\textwidth}{ |l|X|X| }
\hline
Type & Name & Description \\
\hline \hline
0x0 & - & Reserved \\
\hline
0x1 & VIRTIO_NET_FF_MASK_TYPE_ETH & 14 bytes of frame header starting from destination address described in \hyperref[intro:IEEE 802.3-2022]{IEEE 802.3-2022} \\
\hline
0x2 & VIRTIO_NET_FF_MASK_TYPE_IPV4 & 20 bytes of \hyperref[intro:Internet-Header-Format]{IPv4: Internet Header Format} \\
\hline
0x3 & VIRTIO_NET_FF_MASK_TYPE_IPV6 & 40 bytes of \hyperref[intro:IPv6-Header-Format]{IPv6 Header Format} \\
\hline
0x4 & VIRTIO_NET_FF_MASK_TYPE_TCP & 20 bytes of \hyperref[intro:TCP-Header-Format]{TCP Header Format} \\
\hline
0x5 & VIRTIO_NET_FF_MASK_TYPE_UDP & 8 bytes of UDP header described in \hyperref[intro:UDP]{UDP} \\
\hline
0x6 - 0xFF & & Reserved for future \\
\hline
\end{tabularx}
\end{table}

When VIRTIO_NET_FF_MASK_F_PARTIAL_MASK (bit 0) is set, it indicates that
partial masking is supported for all the fields of the selector identified by \field{type}.

For the selector \field{type} VIRTIO_NET_FF_MASK_TYPE_IPV4, if a partial mask is unsupported,
then matching on an individual bit of \field{Flags} in the
\field{IPv4: Internet Header Format} is unsupported. \field{Flags} has to match as a whole
if it is supported.

For the selector \field{type} VIRTIO_NET_FF_MASK_TYPE_IPV4, \field{mask} includes fields
up to the \field{Destination Address}; that is, \field{Options} and
\field{Padding} are excluded.

For the selector \field{type} VIRTIO_NET_FF_MASK_TYPE_IPV6, the \field{Next Header} field
of the \field{mask} corresponds to the \field{Next Header} in the packet
when \field{IPv6 Extension Headers} are not present. When the packet includes
one or more \field{IPv6 Extension Headers}, the \field{Next Header} field of
the \field{mask} corresponds to the \field{Next Header} of the last
\field{IPv6 Extension Header} in the packet.

For the selector \field{type} VIRTIO_NET_FF_MASK_TYPE_TCP, \field{Control bits}
are treated as individual fields for matching; that is, matching individual
\field{Control bits} does not depend on the partial mask support.

\subparagraph{VIRTIO_NET_FF_ACTION_CAP}
\label{par:Device Types / Network Device / Device Operation / Flow filter / Device and driver capabilities / VIRTIO-NET-FF-ACTION-CAP}

The capability VIRTIO_NET_FF_ACTION_CAP lists the supported actions in a rule.
\field{cap_specific_data} is in the format \field{struct virtio_net_ff_cap_actions}.

\begin{lstlisting}
struct virtio_net_ff_actions {
        u8 count;
        u8 reserved[7];
        u8 actions[];
};
\end{lstlisting}

\field{actions} is an array listing all possible actions.
The entries in the array are ordered from the smallest to the largest,
with each supported value appearing exactly once. Each entry can have the
following values:

\begin{table}[H]
\caption{Flow filter rule actions}
\label{table:Device Types / Network Device / Device Operation / Flow filter / Device and driver capabilities / VIRTIO-NET-FF-ACTION-CAP / flow filter rule actions}
\begin{tabularx}{\textwidth}{ |l|X|X| }
\hline
Action & Name & Description \\
\hline \hline
0x0 & - & reserved \\
\hline
0x1 & VIRTIO_NET_FF_ACTION_DROP & Matching packet will be dropped by the device \\
\hline
0x2 & VIRTIO_NET_FF_ACTION_DIRECT_RX_VQ & Matching packet will be directed to a receive queue \\
\hline
0x3 - 0xFF & & Reserved for future \\
\hline
\end{tabularx}
\end{table}

\paragraph{Resource objects}
\label{par:Device Types / Network Device / Device Operation / Flow filter / Resource objects}

\subparagraph{VIRTIO_NET_RESOURCE_OBJ_FF_GROUP}\label{par:Device Types / Network Device / Device Operation / Flow filter / Resource objects / VIRTIO-NET-RESOURCE-OBJ-FF-GROUP}

A flow filter group contains between 0 and \field{rules_limit} rules, as specified by the
capability VIRTIO_NET_FF_RESOURCE_CAP. For the flow filter group object both
\field{resource_obj_specific_data} and
\field{resource_obj_specific_result} are in the format
\field{struct virtio_net_resource_obj_ff_group}.

\begin{lstlisting}
struct virtio_net_resource_obj_ff_group {
        le16 group_priority;
};
\end{lstlisting}

\field{group_priority} specifies the priority for the group. Each group has a
distinct priority. For each incoming packet, the device tries to apply rules
from groups from higher \field{group_priority} value to lower, until either a
rule matches the packet or all groups have been tried.

\subparagraph{VIRTIO_NET_RESOURCE_OBJ_FF_CLASSIFIER}\label{par:Device Types / Network Device / Device Operation / Flow filter / Resource objects / VIRTIO-NET-RESOURCE-OBJ-FF-CLASSIFIER}

A classifier is used to match a flow filter key against a packet. The
classifier defines the desired packet fields to match, and is represented by
the VIRTIO_NET_RESOURCE_OBJ_FF_CLASSIFIER device resource object.

For the flow filter classifier object both \field{resource_obj_specific_data} and
\field{resource_obj_specific_result} are in the format
\field{struct virtio_net_resource_obj_ff_classifier}.

\begin{lstlisting}
struct virtio_net_resource_obj_ff_classifier {
        u8 count;
        u8 reserved[7];
        struct virtio_net_ff_selector selectors[];
};
\end{lstlisting}

A classifier is an array of \field{selectors}. The number of selectors in the
array is indicated by \field{count}. The selector has a type that specifies
the header fields to be matched against, and a mask.
See \ref{lst:Device Types / Network Device / Device Operation / Flow filter / Device and driver capabilities / VIRTIO-NET-FF-SELECTOR-CAP / virtio-net-ff-selector}
for details about selectors.

The first selector is always VIRTIO_NET_FF_MASK_TYPE_ETH. When there are multiple
selectors, a second selector can be either VIRTIO_NET_FF_MASK_TYPE_IPV4
or VIRTIO_NET_FF_MASK_TYPE_IPV6. If the third selector exists, the third
selector can be either VIRTIO_NET_FF_MASK_TYPE_UDP or VIRTIO_NET_FF_MASK_TYPE_TCP.
For example, to match a Ethernet IPv6 UDP packet,
\field{selectors[0].type} is set to VIRTIO_NET_FF_MASK_TYPE_ETH, \field{selectors[1].type}
is set to VIRTIO_NET_FF_MASK_TYPE_IPV6 and \field{selectors[2].type} is
set to VIRTIO_NET_FF_MASK_TYPE_UDP; accordingly, \field{selectors[0].mask[0-13]} is
for Ethernet header fields, \field{selectors[1].mask[0-39]} is set for IPV6 header
and \field{selectors[2].mask[0-7]} is set for UDP header.

When there are multiple selectors, the type of the (N+1)\textsuperscript{th} selector
affects the mask of the (N)\textsuperscript{th} selector. If
\field{count} is 2 or more, all the mask bits within \field{selectors[0]}
corresponding to \field{EtherType} of an Ethernet header are set.

If \field{count} is more than 2:
\begin{itemize}
\item if \field{selector[1].type} is, VIRTIO_NET_FF_MASK_TYPE_IPV4, then, all the mask bits within
\field{selector[1]} for \field{Protocol} is set.
\item if \field{selector[1].type} is, VIRTIO_NET_FF_MASK_TYPE_IPV6, then, all the mask bits within
\field{selector[1]} for \field{Next Header} is set.
\end{itemize}

If for a given packet header field, a subset of bits of a field is to be matched,
and if the partial mask is supported, the flow filter
mask object can specify a mask which has fewer bits set than the packet header
field size. For example, a partial mask for the Ethernet header source mac
address can be of 1-bit for multicast detection instead of 48-bits.

\subparagraph{VIRTIO_NET_RESOURCE_OBJ_FF_RULE}\label{par:Device Types / Network Device / Device Operation / Flow filter / Resource objects / VIRTIO-NET-RESOURCE-OBJ-FF-RULE}

Each flow filter rule resource object comprises a key, a priority, and an action.
For the flow filter rule object,
\field{resource_obj_specific_data} and
\field{resource_obj_specific_result} are in the format
\field{struct virtio_net_resource_obj_ff_rule}.

\begin{lstlisting}
struct virtio_net_resource_obj_ff_rule {
        le32 group_id;
        le32 classifier_id;
        u8 rule_priority;
        u8 key_length; /* length of key in bytes */
        u8 action;
        u8 reserved;
        le16 vq_index;
        u8 reserved1[2];
        u8 keys[][];
};
\end{lstlisting}

\field{group_id} is the resource object ID of the flow filter group to which
this rule belongs. \field{classifier_id} is the resource object ID of the
classifier used to match a packet against the \field{key}.

\field{rule_priority} denotes the priority of the rule within the group
specified by the \field{group_id}.
Rules within the group are applied from the highest to the lowest priority
until a rule matches the packet and an
action is taken. Rules with the same priority can be applied in any order.

\field{reserved} and \field{reserved1} are reserved and set to 0.

\field{keys[][]} is an array of keys to match against packets, using
the classifier specified by \field{classifier_id}. Each entry (key) comprises
a byte array, and they are located one immediately after another.
The size (number of entries) of the array is exactly the same as that of
\field{selectors} in the classifier, or in other words, \field{count}
in the classifier.

\field{key_length} specifies the total length of \field{keys} in bytes.
In other words, it equals the sum total of \field{length} of all
selectors in \field{selectors} in the classifier specified by
\field{classifier_id}.

For example, if a classifier object's \field{selectors[0].type} is
VIRTIO_NET_FF_MASK_TYPE_ETH and \field{selectors[1].type} is
VIRTIO_NET_FF_MASK_TYPE_IPV6,
then selectors[0].length is 14 and selectors[1].length is 40.
Accordingly, the \field{key_length} is set to 54.
This setting indicates that the \field{key} array's length is 54 bytes
comprising a first byte array of 14 bytes for the
Ethernet MAC header in bytes 0-13, immediately followed by 40 bytes for the
IPv6 header in bytes 14-53.

When there are multiple selectors in the classifier object, the key bytes
for (N)\textsuperscript{th} selector are set so that
(N+1)\textsuperscript{th} selector can be matched.

If \field{count} is 2 or more, key bytes of \field{EtherType}
are set according to \hyperref[intro:IEEE 802 Ethertypes]{IEEE 802 Ethertypes}
for VIRTIO_NET_FF_MASK_TYPE_IPV4 or VIRTIO_NET_FF_MASK_TYPE_IPV6 respectively.

If \field{count} is more than 2, when \field{selector[1].type} is
VIRTIO_NET_FF_MASK_TYPE_IPV4 or VIRTIO_NET_FF_MASK_TYPE_IPV6, key
bytes of \field{Protocol} or \field{Next Header} is set as per
\field{Protocol Numbers} defined \hyperref[intro:IANA Protocol Numbers]{IANA Protocol Numbers}
respectively.

\field{action} is the action to take when a packet matches the
\field{key} using the \field{classifier_id}. Supported actions are described in
\ref{table:Device Types / Network Device / Device Operation / Flow filter / Device and driver capabilities / VIRTIO-NET-FF-ACTION-CAP / flow filter rule actions}.

\field{vq_index} specifies a receive virtqueue. When the \field{action} is set
to VIRTIO_NET_FF_ACTION_DIRECT_RX_VQ, and the packet matches the \field{key},
the matching packet is directed to this virtqueue.

Note that at most one action is ever taken for a given packet. If a rule is
applied and an action is taken, the action of other rules is not taken.

\devicenormative{\paragraph}{Flow filter}{Device Types / Network Device / Device Operation / Flow filter}

When the device supports flow filter operations,
\begin{itemize}
\item the device MUST set VIRTIO_NET_FF_RESOURCE_CAP, VIRTIO_NET_FF_SELECTOR_CAP
and VIRTIO_NET_FF_ACTION_CAP capability in the \field{supported_caps} in the
command VIRTIO_ADMIN_CMD_CAP_SUPPORT_QUERY.
\item the device MUST support the administration commands
VIRTIO_ADMIN_CMD_RESOURCE_OBJ_CREATE,
VIRTIO_ADMIN_CMD_RESOURCE_OBJ_MODIFY, VIRTIO_ADMIN_CMD_RESOURCE_OBJ_QUERY,
VIRTIO_ADMIN_CMD_RESOURCE_OBJ_DESTROY for the resource types
VIRTIO_NET_RESOURCE_OBJ_FF_GROUP, VIRTIO_NET_RESOURCE_OBJ_FF_CLASSIFIER and
VIRTIO_NET_RESOURCE_OBJ_FF_RULE.
\end{itemize}

When any of the VIRTIO_NET_FF_RESOURCE_CAP, VIRTIO_NET_FF_SELECTOR_CAP, or
VIRTIO_NET_FF_ACTION_CAP capability is disabled, the device SHOULD set
\field{status} to VIRTIO_ADMIN_STATUS_Q_INVALID_OPCODE for the commands
VIRTIO_ADMIN_CMD_RESOURCE_OBJ_CREATE,
VIRTIO_ADMIN_CMD_RESOURCE_OBJ_MODIFY, VIRTIO_ADMIN_CMD_RESOURCE_OBJ_QUERY,
and VIRTIO_ADMIN_CMD_RESOURCE_OBJ_DESTROY. These commands apply to the resource
\field{type} of VIRTIO_NET_RESOURCE_OBJ_FF_GROUP, VIRTIO_NET_RESOURCE_OBJ_FF_CLASSIFIER, and
VIRTIO_NET_RESOURCE_OBJ_FF_RULE.

The device SHOULD set \field{status} to VIRTIO_ADMIN_STATUS_EINVAL for the
command VIRTIO_ADMIN_CMD_RESOURCE_OBJ_CREATE when the resource \field{type}
is VIRTIO_NET_RESOURCE_OBJ_FF_GROUP, if a flow filter group already exists
with the supplied \field{group_priority}.

The device SHOULD set \field{status} to VIRTIO_ADMIN_STATUS_ENOSPC for the
command VIRTIO_ADMIN_CMD_RESOURCE_OBJ_CREATE when the resource \field{type}
is VIRTIO_NET_RESOURCE_OBJ_FF_GROUP, if the number of flow filter group
objects in the device exceeds the lower of the configured driver
capabilities \field{groups_limit} and \field{rules_per_group_limit}.

The device SHOULD set \field{status} to VIRTIO_ADMIN_STATUS_ENOSPC for the
command VIRTIO_ADMIN_CMD_RESOURCE_OBJ_CREATE when the resource \field{type} is
VIRTIO_NET_RESOURCE_OBJ_FF_CLASSIFIER, if the number of flow filter selector
objects in the device exceeds the configured driver capability
\field{selectors_limit}.

The device SHOULD set \field{status} to VIRTIO_ADMIN_STATUS_EBUSY for the
command VIRTIO_ADMIN_CMD_RESOURCE_OBJ_DESTROY for a flow filter group when
the flow filter group has one or more flow filter rules depending on it.

The device SHOULD set \field{status} to VIRTIO_ADMIN_STATUS_EBUSY for the
command VIRTIO_ADMIN_CMD_RESOURCE_OBJ_DESTROY for a flow filter classifier when
the flow filter classifier has one or more flow filter rules depending on it.

The device SHOULD fail the command VIRTIO_ADMIN_CMD_RESOURCE_OBJ_CREATE for the
flow filter rule resource object if,
\begin{itemize}
\item \field{vq_index} is not a valid receive virtqueue index for
the VIRTIO_NET_FF_ACTION_DIRECT_RX_VQ action,
\item \field{priority} is greater than or equal to
      \field{last_rule_priority},
\item \field{id} is greater than or equal to \field{rules_limit} or
      greater than or equal to \field{rules_per_group_limit}, whichever is lower,
\item the length of \field{keys} and the length of all the mask bytes of
      \field{selectors[].mask} as referred by \field{classifier_id} differs,
\item the supplied \field{action} is not supported in the capability VIRTIO_NET_FF_ACTION_CAP.
\end{itemize}

When the flow filter directs a packet to the virtqueue identified by
\field{vq_index} and if the receive virtqueue is reset, the device
MUST drop such packets.

Upon applying a flow filter rule to a packet, the device MUST STOP any further
application of rules and cease applying any other steering configurations.

For multiple flow filter groups, the device MUST apply the rules from
the group with the highest priority. If any rule from this group is applied,
the device MUST ignore the remaining groups. If none of the rules from the
highest priority group match, the device MUST apply the rules from
the group with the next highest priority, until either a rule matches or
all groups have been attempted.

The device MUST apply the rules within the group from the highest to the
lowest priority until a rule matches the packet, and the device MUST take
the action. If an action is taken, the device MUST not take any other
action for this packet.

The device MAY apply the rules with the same \field{rule_priority} in any
order within the group.

The device MUST process incoming packets in the following order:
\begin{itemize}
\item apply the steering configuration received using control virtqueue
      commands VIRTIO_NET_CTRL_RX, VIRTIO_NET_CTRL_MAC, and
      VIRTIO_NET_CTRL_VLAN.
\item apply flow filter rules if any.
\item if no filter rule is applied, apply the steering configuration
      received using the command VIRTIO_NET_CTRL_MQ_RSS_CONFIG
      or according to automatic receive steering.
\end{itemize}

When processing an incoming packet, if the packet is dropped at any stage, the device
MUST skip further processing.

When the device drops the packet due to the configuration done using the control
virtqueue commands VIRTIO_NET_CTRL_RX or VIRTIO_NET_CTRL_MAC or VIRTIO_NET_CTRL_VLAN,
the device MUST skip flow filter rules for this packet.

When the device performs flow filter match operations and if the operation
result did not have any match in all the groups, the receive packet processing
continues to next level, i.e. to apply configuration done using
VIRTIO_NET_CTRL_MQ_RSS_CONFIG command.

The device MUST support the creation of flow filter classifier objects
using the command VIRTIO_ADMIN_CMD_RESOURCE_OBJ_CREATE with \field{flags}
set to VIRTIO_NET_FF_MASK_F_PARTIAL_MASK;
this support is required even if all the bits of the masks are set for
a field in \field{selectors}, provided that partial masking is supported
for the selectors.

\drivernormative{\paragraph}{Flow filter}{Device Types / Network Device / Device Operation / Flow filter}

The driver MUST enable VIRTIO_NET_FF_RESOURCE_CAP, VIRTIO_NET_FF_SELECTOR_CAP,
and VIRTIO_NET_FF_ACTION_CAP capabilities to use flow filter.

The driver SHOULD NOT remove a flow filter group using the command
VIRTIO_ADMIN_CMD_RESOURCE_OBJ_DESTROY when one or more flow filter rules
depend on that group. The driver SHOULD only destroy the group after
all the associated rules have been destroyed.

The driver SHOULD NOT remove a flow filter classifier using the command
VIRTIO_ADMIN_CMD_RESOURCE_OBJ_DESTROY when one or more flow filter rules
depend on the classifier. The driver SHOULD only destroy the classifier
after all the associated rules have been destroyed.

The driver SHOULD NOT add multiple flow filter rules with the same
\field{rule_priority} within a flow filter group, as these rules MAY match
the same packet. The driver SHOULD assign different \field{rule_priority}
values to different flow filter rules if multiple rules may match a single
packet.

For the command VIRTIO_ADMIN_CMD_RESOURCE_OBJ_CREATE, when creating a resource
of \field{type} VIRTIO_NET_RESOURCE_OBJ_FF_CLASSIFIER, the driver MUST set:
\begin{itemize}
\item \field{selectors[0].type} to VIRTIO_NET_FF_MASK_TYPE_ETH.
\item \field{selectors[1].type} to VIRTIO_NET_FF_MASK_TYPE_IPV4 or
      VIRTIO_NET_FF_MASK_TYPE_IPV6 when \field{count} is more than 1,
\item \field{selectors[2].type} VIRTIO_NET_FF_MASK_TYPE_UDP or
      VIRTIO_NET_FF_MASK_TYPE_TCP when \field{count} is more than 2.
\end{itemize}

For the command VIRTIO_ADMIN_CMD_RESOURCE_OBJ_CREATE, when creating a resource
of \field{type} VIRTIO_NET_RESOURCE_OBJ_FF_CLASSIFIER, the driver MUST set:
\begin{itemize}
\item \field{selectors[0].mask} bytes to all 1s for the \field{EtherType}
       when \field{count} is 2 or more.
\item \field{selectors[1].mask} bytes to all 1s for \field{Protocol} or \field{Next Header}
       when \field{selector[1].type} is VIRTIO_NET_FF_MASK_TYPE_IPV4 or VIRTIO_NET_FF_MASK_TYPE_IPV6,
       and when \field{count} is more than 2.
\end{itemize}

For the command VIRTIO_ADMIN_CMD_RESOURCE_OBJ_CREATE, the resource \field{type}
VIRTIO_NET_RESOURCE_OBJ_FF_RULE, if the corresponding classifier object's
\field{count} is 2 or more, the driver MUST SET the \field{keys} bytes of
\field{EtherType} in accordance with
\hyperref[intro:IEEE 802 Ethertypes]{IEEE 802 Ethertypes}
for either VIRTIO_NET_FF_MASK_TYPE_IPV4 or VIRTIO_NET_FF_MASK_TYPE_IPV6.

For the command VIRTIO_ADMIN_CMD_RESOURCE_OBJ_CREATE, when creating a resource of
\field{type} VIRTIO_NET_RESOURCE_OBJ_FF_RULE, if the corresponding classifier
object's \field{count} is more than 2, and the \field{selector[1].type} is either
VIRTIO_NET_FF_MASK_TYPE_IPV4 or VIRTIO_NET_FF_MASK_TYPE_IPV6, the driver MUST
set the \field{keys} bytes for the \field{Protocol} or \field{Next Header}
according to \hyperref[intro:IANA Protocol Numbers]{IANA Protocol Numbers} respectively.

The driver SHOULD set all the bits for a field in the mask of a selector in both the
capability and the classifier object, unless the VIRTIO_NET_FF_MASK_F_PARTIAL_MASK
is enabled.

\subsubsection{Legacy Interface: Framing Requirements}\label{sec:Device
Types / Network Device / Legacy Interface: Framing Requirements}

When using legacy interfaces, transitional drivers which have not
negotiated VIRTIO_F_ANY_LAYOUT MUST use a single descriptor for the
\field{struct virtio_net_hdr} on both transmit and receive, with the
network data in the following descriptors.

Additionally, when using the control virtqueue (see \ref{sec:Device
Types / Network Device / Device Operation / Control Virtqueue})
, transitional drivers which have not
negotiated VIRTIO_F_ANY_LAYOUT MUST:
\begin{itemize}
\item for all commands, use a single 2-byte descriptor including the first two
fields: \field{class} and \field{command}
\item for all commands except VIRTIO_NET_CTRL_MAC_TABLE_SET
use a single descriptor including command-specific-data
with no padding.
\item for the VIRTIO_NET_CTRL_MAC_TABLE_SET command use exactly
two descriptors including command-specific-data with no padding:
the first of these descriptors MUST include the
virtio_net_ctrl_mac table structure for the unicast addresses with no padding,
the second of these descriptors MUST include the
virtio_net_ctrl_mac table structure for the multicast addresses
with no padding.
\item for all commands, use a single 1-byte descriptor for the
\field{ack} field
\end{itemize}

See \ref{sec:Basic
Facilities of a Virtio Device / Virtqueues / Message Framing}.

\section{Network Device}\label{sec:Device Types / Network Device}

The virtio network device is a virtual network interface controller.
It consists of a virtual Ethernet link which connects the device
to the Ethernet network. The device has transmit and receive
queues. The driver adds empty buffers to the receive virtqueue.
The device receives incoming packets from the link; the device
places these incoming packets in the receive virtqueue buffers.
The driver adds outgoing packets to the transmit virtqueue. The device
removes these packets from the transmit virtqueue and sends them to
the link. The device may have a control virtqueue. The driver
uses the control virtqueue to dynamically manipulate various
features of the initialized device.

\subsection{Device ID}\label{sec:Device Types / Network Device / Device ID}

 1

\subsection{Virtqueues}\label{sec:Device Types / Network Device / Virtqueues}

\begin{description}
\item[0] receiveq1
\item[1] transmitq1
\item[\ldots]
\item[2(N-1)] receiveqN
\item[2(N-1)+1] transmitqN
\item[2N] controlq
\end{description}

 N=1 if neither VIRTIO_NET_F_MQ nor VIRTIO_NET_F_RSS are negotiated, otherwise N is set by
 \field{max_virtqueue_pairs}.

controlq is optional; it only exists if VIRTIO_NET_F_CTRL_VQ is
negotiated.

\subsection{Feature bits}\label{sec:Device Types / Network Device / Feature bits}

\begin{description}
\item[VIRTIO_NET_F_CSUM (0)] Device handles packets with partial checksum offload.

\item[VIRTIO_NET_F_GUEST_CSUM (1)] Driver handles packets with partial checksum.

\item[VIRTIO_NET_F_CTRL_GUEST_OFFLOADS (2)] Control channel offloads
        reconfiguration support.

\item[VIRTIO_NET_F_MTU(3)] Device maximum MTU reporting is supported. If
    offered by the device, device advises driver about the value of
    its maximum MTU. If negotiated, the driver uses \field{mtu} as
    the maximum MTU value.

\item[VIRTIO_NET_F_MAC (5)] Device has given MAC address.

\item[VIRTIO_NET_F_GUEST_TSO4 (7)] Driver can receive TSOv4.

\item[VIRTIO_NET_F_GUEST_TSO6 (8)] Driver can receive TSOv6.

\item[VIRTIO_NET_F_GUEST_ECN (9)] Driver can receive TSO with ECN.

\item[VIRTIO_NET_F_GUEST_UFO (10)] Driver can receive UFO.

\item[VIRTIO_NET_F_HOST_TSO4 (11)] Device can receive TSOv4.

\item[VIRTIO_NET_F_HOST_TSO6 (12)] Device can receive TSOv6.

\item[VIRTIO_NET_F_HOST_ECN (13)] Device can receive TSO with ECN.

\item[VIRTIO_NET_F_HOST_UFO (14)] Device can receive UFO.

\item[VIRTIO_NET_F_MRG_RXBUF (15)] Driver can merge receive buffers.

\item[VIRTIO_NET_F_STATUS (16)] Configuration status field is
    available.

\item[VIRTIO_NET_F_CTRL_VQ (17)] Control channel is available.

\item[VIRTIO_NET_F_CTRL_RX (18)] Control channel RX mode support.

\item[VIRTIO_NET_F_CTRL_VLAN (19)] Control channel VLAN filtering.

\item[VIRTIO_NET_F_CTRL_RX_EXTRA (20)]	Control channel RX extra mode support.

\item[VIRTIO_NET_F_GUEST_ANNOUNCE(21)] Driver can send gratuitous
    packets.

\item[VIRTIO_NET_F_MQ(22)] Device supports multiqueue with automatic
    receive steering.

\item[VIRTIO_NET_F_CTRL_MAC_ADDR(23)] Set MAC address through control
    channel.

\item[VIRTIO_NET_F_DEVICE_STATS(50)] Device can provide device-level statistics
    to the driver through the control virtqueue.

\item[VIRTIO_NET_F_HASH_TUNNEL(51)] Device supports inner header hash for encapsulated packets.

\item[VIRTIO_NET_F_VQ_NOTF_COAL(52)] Device supports virtqueue notification coalescing.

\item[VIRTIO_NET_F_NOTF_COAL(53)] Device supports notifications coalescing.

\item[VIRTIO_NET_F_GUEST_USO4 (54)] Driver can receive USOv4 packets.

\item[VIRTIO_NET_F_GUEST_USO6 (55)] Driver can receive USOv6 packets.

\item[VIRTIO_NET_F_HOST_USO (56)] Device can receive USO packets. Unlike UFO
 (fragmenting the packet) the USO splits large UDP packet
 to several segments when each of these smaller packets has UDP header.

\item[VIRTIO_NET_F_HASH_REPORT(57)] Device can report per-packet hash
    value and a type of calculated hash.

\item[VIRTIO_NET_F_GUEST_HDRLEN(59)] Driver can provide the exact \field{hdr_len}
    value. Device benefits from knowing the exact header length.

\item[VIRTIO_NET_F_RSS(60)] Device supports RSS (receive-side scaling)
    with Toeplitz hash calculation and configurable hash
    parameters for receive steering.

\item[VIRTIO_NET_F_RSC_EXT(61)] Device can process duplicated ACKs
    and report number of coalesced segments and duplicated ACKs.

\item[VIRTIO_NET_F_STANDBY(62)] Device may act as a standby for a primary
    device with the same MAC address.

\item[VIRTIO_NET_F_SPEED_DUPLEX(63)] Device reports speed and duplex.

\item[VIRTIO_NET_F_RSS_CONTEXT(64)] Device supports multiple RSS contexts.

\item[VIRTIO_NET_F_GUEST_UDP_TUNNEL_GSO (65)] Driver can receive GSO packets
  carried by a UDP tunnel.

\item[VIRTIO_NET_F_GUEST_UDP_TUNNEL_GSO_CSUM (66)] Driver handles packets
  carried by a UDP tunnel with partial csum for the outer header.

\item[VIRTIO_NET_F_HOST_UDP_TUNNEL_GSO (67)] Device can receive GSO packets
  carried by a UDP tunnel.

\item[VIRTIO_NET_F_HOST_UDP_TUNNEL_GSO_CSUM (68)] Device handles packets
  carried by a UDP tunnel with partial csum for the outer header.
\end{description}

\subsubsection{Feature bit requirements}\label{sec:Device Types / Network Device / Feature bits / Feature bit requirements}

Some networking feature bits require other networking feature bits
(see \ref{drivernormative:Basic Facilities of a Virtio Device / Feature Bits}):

\begin{description}
\item[VIRTIO_NET_F_GUEST_TSO4] Requires VIRTIO_NET_F_GUEST_CSUM.
\item[VIRTIO_NET_F_GUEST_TSO6] Requires VIRTIO_NET_F_GUEST_CSUM.
\item[VIRTIO_NET_F_GUEST_ECN] Requires VIRTIO_NET_F_GUEST_TSO4 or VIRTIO_NET_F_GUEST_TSO6.
\item[VIRTIO_NET_F_GUEST_UFO] Requires VIRTIO_NET_F_GUEST_CSUM.
\item[VIRTIO_NET_F_GUEST_USO4] Requires VIRTIO_NET_F_GUEST_CSUM.
\item[VIRTIO_NET_F_GUEST_USO6] Requires VIRTIO_NET_F_GUEST_CSUM.
\item[VIRTIO_NET_F_GUEST_UDP_TUNNEL_GSO] Requires VIRTIO_NET_F_GUEST_TSO4, VIRTIO_NET_F_GUEST_TSO6,
   VIRTIO_NET_F_GUEST_USO4 and VIRTIO_NET_F_GUEST_USO6.
\item[VIRTIO_NET_F_GUEST_UDP_TUNNEL_GSO_CSUM] Requires VIRTIO_NET_F_GUEST_UDP_TUNNEL_GSO

\item[VIRTIO_NET_F_HOST_TSO4] Requires VIRTIO_NET_F_CSUM.
\item[VIRTIO_NET_F_HOST_TSO6] Requires VIRTIO_NET_F_CSUM.
\item[VIRTIO_NET_F_HOST_ECN] Requires VIRTIO_NET_F_HOST_TSO4 or VIRTIO_NET_F_HOST_TSO6.
\item[VIRTIO_NET_F_HOST_UFO] Requires VIRTIO_NET_F_CSUM.
\item[VIRTIO_NET_F_HOST_USO] Requires VIRTIO_NET_F_CSUM.
\item[VIRTIO_NET_F_HOST_UDP_TUNNEL_GSO] Requires VIRTIO_NET_F_HOST_TSO4, VIRTIO_NET_F_HOST_TSO6
   and VIRTIO_NET_F_HOST_USO.
\item[VIRTIO_NET_F_HOST_UDP_TUNNEL_GSO_CSUM] Requires VIRTIO_NET_F_HOST_UDP_TUNNEL_GSO

\item[VIRTIO_NET_F_CTRL_RX] Requires VIRTIO_NET_F_CTRL_VQ.
\item[VIRTIO_NET_F_CTRL_VLAN] Requires VIRTIO_NET_F_CTRL_VQ.
\item[VIRTIO_NET_F_GUEST_ANNOUNCE] Requires VIRTIO_NET_F_CTRL_VQ.
\item[VIRTIO_NET_F_MQ] Requires VIRTIO_NET_F_CTRL_VQ.
\item[VIRTIO_NET_F_CTRL_MAC_ADDR] Requires VIRTIO_NET_F_CTRL_VQ.
\item[VIRTIO_NET_F_NOTF_COAL] Requires VIRTIO_NET_F_CTRL_VQ.
\item[VIRTIO_NET_F_RSC_EXT] Requires VIRTIO_NET_F_HOST_TSO4 or VIRTIO_NET_F_HOST_TSO6.
\item[VIRTIO_NET_F_RSS] Requires VIRTIO_NET_F_CTRL_VQ.
\item[VIRTIO_NET_F_VQ_NOTF_COAL] Requires VIRTIO_NET_F_CTRL_VQ.
\item[VIRTIO_NET_F_HASH_TUNNEL] Requires VIRTIO_NET_F_CTRL_VQ along with VIRTIO_NET_F_RSS or VIRTIO_NET_F_HASH_REPORT.
\item[VIRTIO_NET_F_RSS_CONTEXT] Requires VIRTIO_NET_F_CTRL_VQ and VIRTIO_NET_F_RSS.
\end{description}

\begin{note}
The dependency between UDP_TUNNEL_GSO_CSUM and UDP_TUNNEL_GSO is intentionally
in the opposite direction with respect to the plain GSO features and the plain
checksum offload because UDP tunnel checksum offload gives very little gain
for non GSO packets and is quite complex to implement in H/W.
\end{note}

\subsubsection{Legacy Interface: Feature bits}\label{sec:Device Types / Network Device / Feature bits / Legacy Interface: Feature bits}
\begin{description}
\item[VIRTIO_NET_F_GSO (6)] Device handles packets with any GSO type. This was supposed to indicate segmentation offload support, but
upon further investigation it became clear that multiple bits were needed.
\item[VIRTIO_NET_F_GUEST_RSC4 (41)] Device coalesces TCPIP v4 packets. This was implemented by hypervisor patch for certification
purposes and current Windows driver depends on it. It will not function if virtio-net device reports this feature.
\item[VIRTIO_NET_F_GUEST_RSC6 (42)] Device coalesces TCPIP v6 packets. Similar to VIRTIO_NET_F_GUEST_RSC4.
\end{description}

\subsection{Device configuration layout}\label{sec:Device Types / Network Device / Device configuration layout}
\label{sec:Device Types / Block Device / Feature bits / Device configuration layout}

The network device has the following device configuration layout.
All of the device configuration fields are read-only for the driver.

\begin{lstlisting}
struct virtio_net_config {
        u8 mac[6];
        le16 status;
        le16 max_virtqueue_pairs;
        le16 mtu;
        le32 speed;
        u8 duplex;
        u8 rss_max_key_size;
        le16 rss_max_indirection_table_length;
        le32 supported_hash_types;
        le32 supported_tunnel_types;
};
\end{lstlisting}

The \field{mac} address field always exists (although it is only
valid if VIRTIO_NET_F_MAC is set).

The \field{status} only exists if VIRTIO_NET_F_STATUS is set.
Two bits are currently defined for the status field: VIRTIO_NET_S_LINK_UP
and VIRTIO_NET_S_ANNOUNCE.

\begin{lstlisting}
#define VIRTIO_NET_S_LINK_UP     1
#define VIRTIO_NET_S_ANNOUNCE    2
\end{lstlisting}

The following field, \field{max_virtqueue_pairs} only exists if
VIRTIO_NET_F_MQ or VIRTIO_NET_F_RSS is set. This field specifies the maximum number
of each of transmit and receive virtqueues (receiveq1\ldots receiveqN
and transmitq1\ldots transmitqN respectively) that can be configured once at least one of these features
is negotiated.

The following field, \field{mtu} only exists if VIRTIO_NET_F_MTU
is set. This field specifies the maximum MTU for the driver to
use.

The following two fields, \field{speed} and \field{duplex}, only
exist if VIRTIO_NET_F_SPEED_DUPLEX is set.

\field{speed} contains the device speed, in units of 1 MBit per
second, 0 to 0x7fffffff, or 0xffffffff for unknown speed.

\field{duplex} has the values of 0x01 for full duplex, 0x00 for
half duplex and 0xff for unknown duplex state.

Both \field{speed} and \field{duplex} can change, thus the driver
is expected to re-read these values after receiving a
configuration change notification.

The following field, \field{rss_max_key_size} only exists if VIRTIO_NET_F_RSS or VIRTIO_NET_F_HASH_REPORT is set.
It specifies the maximum supported length of RSS key in bytes.

The following field, \field{rss_max_indirection_table_length} only exists if VIRTIO_NET_F_RSS is set.
It specifies the maximum number of 16-bit entries in RSS indirection table.

The next field, \field{supported_hash_types} only exists if the device supports hash calculation,
i.e. if VIRTIO_NET_F_RSS or VIRTIO_NET_F_HASH_REPORT is set.

Field \field{supported_hash_types} contains the bitmask of supported hash types.
See \ref{sec:Device Types / Network Device / Device Operation / Processing of Incoming Packets / Hash calculation for incoming packets / Supported/enabled hash types} for details of supported hash types.

Field \field{supported_tunnel_types} only exists if the device supports inner header hash, i.e. if VIRTIO_NET_F_HASH_TUNNEL is set.

Field \field{supported_tunnel_types} contains the bitmask of encapsulation types supported by the device for inner header hash.
Encapsulation types are defined in \ref{sec:Device Types / Network Device / Device Operation / Processing of Incoming Packets /
Hash calculation for incoming packets / Encapsulation types supported/enabled for inner header hash}.

\devicenormative{\subsubsection}{Device configuration layout}{Device Types / Network Device / Device configuration layout}

The device MUST set \field{max_virtqueue_pairs} to between 1 and 0x8000 inclusive,
if it offers VIRTIO_NET_F_MQ.

The device MUST set \field{mtu} to between 68 and 65535 inclusive,
if it offers VIRTIO_NET_F_MTU.

The device SHOULD set \field{mtu} to at least 1280, if it offers
VIRTIO_NET_F_MTU.

The device MUST NOT modify \field{mtu} once it has been set.

The device MUST NOT pass received packets that exceed \field{mtu} (plus low
level ethernet header length) size with \field{gso_type} NONE or ECN
after VIRTIO_NET_F_MTU has been successfully negotiated.

The device MUST forward transmitted packets of up to \field{mtu} (plus low
level ethernet header length) size with \field{gso_type} NONE or ECN, and do
so without fragmentation, after VIRTIO_NET_F_MTU has been successfully
negotiated.

The device MUST set \field{rss_max_key_size} to at least 40, if it offers
VIRTIO_NET_F_RSS or VIRTIO_NET_F_HASH_REPORT.

The device MUST set \field{rss_max_indirection_table_length} to at least 128, if it offers
VIRTIO_NET_F_RSS.

If the driver negotiates the VIRTIO_NET_F_STANDBY feature, the device MAY act
as a standby device for a primary device with the same MAC address.

If VIRTIO_NET_F_SPEED_DUPLEX has been negotiated, \field{speed}
MUST contain the device speed, in units of 1 MBit per second, 0 to
0x7ffffffff, or 0xfffffffff for unknown.

If VIRTIO_NET_F_SPEED_DUPLEX has been negotiated, \field{duplex}
MUST have the values of 0x00 for full duplex, 0x01 for half
duplex, or 0xff for unknown.

If VIRTIO_NET_F_SPEED_DUPLEX and VIRTIO_NET_F_STATUS have both
been negotiated, the device SHOULD NOT change the \field{speed} and
\field{duplex} fields as long as VIRTIO_NET_S_LINK_UP is set in
the \field{status}.

The device SHOULD NOT offer VIRTIO_NET_F_HASH_REPORT if it
does not offer VIRTIO_NET_F_CTRL_VQ.

The device SHOULD NOT offer VIRTIO_NET_F_CTRL_RX_EXTRA if it
does not offer VIRTIO_NET_F_CTRL_VQ.

\drivernormative{\subsubsection}{Device configuration layout}{Device Types / Network Device / Device configuration layout}

The driver MUST NOT write to any of the device configuration fields.

A driver SHOULD negotiate VIRTIO_NET_F_MAC if the device offers it.
If the driver negotiates the VIRTIO_NET_F_MAC feature, the driver MUST set
the physical address of the NIC to \field{mac}.  Otherwise, it SHOULD
use a locally-administered MAC address (see \hyperref[intro:IEEE 802]{IEEE 802},
``9.2 48-bit universal LAN MAC addresses'').

If the driver does not negotiate the VIRTIO_NET_F_STATUS feature, it SHOULD
assume the link is active, otherwise it SHOULD read the link status from
the bottom bit of \field{status}.

A driver SHOULD negotiate VIRTIO_NET_F_MTU if the device offers it.

If the driver negotiates VIRTIO_NET_F_MTU, it MUST supply enough receive
buffers to receive at least one receive packet of size \field{mtu} (plus low
level ethernet header length) with \field{gso_type} NONE or ECN.

If the driver negotiates VIRTIO_NET_F_MTU, it MUST NOT transmit packets of
size exceeding the value of \field{mtu} (plus low level ethernet header length)
with \field{gso_type} NONE or ECN.

A driver SHOULD negotiate the VIRTIO_NET_F_STANDBY feature if the device offers it.

If VIRTIO_NET_F_SPEED_DUPLEX has been negotiated,
the driver MUST treat any value of \field{speed} above
0x7fffffff as well as any value of \field{duplex} not
matching 0x00 or 0x01 as an unknown value.

If VIRTIO_NET_F_SPEED_DUPLEX has been negotiated, the driver
SHOULD re-read \field{speed} and \field{duplex} after a
configuration change notification.

A driver SHOULD NOT negotiate VIRTIO_NET_F_HASH_REPORT if it
does not negotiate VIRTIO_NET_F_CTRL_VQ.

A driver SHOULD NOT negotiate VIRTIO_NET_F_CTRL_RX_EXTRA if it
does not negotiate VIRTIO_NET_F_CTRL_VQ.

\subsubsection{Legacy Interface: Device configuration layout}\label{sec:Device Types / Network Device / Device configuration layout / Legacy Interface: Device configuration layout}
\label{sec:Device Types / Block Device / Feature bits / Device configuration layout / Legacy Interface: Device configuration layout}
When using the legacy interface, transitional devices and drivers
MUST format \field{status} and
\field{max_virtqueue_pairs} in struct virtio_net_config
according to the native endian of the guest rather than
(necessarily when not using the legacy interface) little-endian.

When using the legacy interface, \field{mac} is driver-writable
which provided a way for drivers to update the MAC without
negotiating VIRTIO_NET_F_CTRL_MAC_ADDR.

\subsection{Device Initialization}\label{sec:Device Types / Network Device / Device Initialization}

A driver would perform a typical initialization routine like so:

\begin{enumerate}
\item Identify and initialize the receive and
  transmission virtqueues, up to N of each kind. If
  VIRTIO_NET_F_MQ feature bit is negotiated,
  N=\field{max_virtqueue_pairs}, otherwise identify N=1.

\item If the VIRTIO_NET_F_CTRL_VQ feature bit is negotiated,
  identify the control virtqueue.

\item Fill the receive queues with buffers: see \ref{sec:Device Types / Network Device / Device Operation / Setting Up Receive Buffers}.

\item Even with VIRTIO_NET_F_MQ, only receiveq1, transmitq1 and
  controlq are used by default.  The driver would send the
  VIRTIO_NET_CTRL_MQ_VQ_PAIRS_SET command specifying the
  number of the transmit and receive queues to use.

\item If the VIRTIO_NET_F_MAC feature bit is set, the configuration
  space \field{mac} entry indicates the ``physical'' address of the
  device, otherwise the driver would typically generate a random
  local MAC address.

\item If the VIRTIO_NET_F_STATUS feature bit is negotiated, the link
  status comes from the bottom bit of \field{status}.
  Otherwise, the driver assumes it's active.

\item A performant driver would indicate that it will generate checksumless
  packets by negotiating the VIRTIO_NET_F_CSUM feature.

\item If that feature is negotiated, a driver can use TCP segmentation or UDP
  segmentation/fragmentation offload by negotiating the VIRTIO_NET_F_HOST_TSO4 (IPv4
  TCP), VIRTIO_NET_F_HOST_TSO6 (IPv6 TCP), VIRTIO_NET_F_HOST_UFO
  (UDP fragmentation) and VIRTIO_NET_F_HOST_USO (UDP segmentation) features.

\item If the VIRTIO_NET_F_HOST_TSO6, VIRTIO_NET_F_HOST_TSO4 and VIRTIO_NET_F_HOST_USO
  segmentation features are negotiated, a driver can
  use TCP segmentation or UDP segmentation on top of UDP encapsulation
  offload, when the outer header does not require checksumming - e.g.
  the outer UDP checksum is zero - by negotiating the
  VIRTIO_NET_F_HOST_UDP_TUNNEL_GSO feature.
  GSO over UDP tunnels packets carry two sets of headers: the outer ones
  and the inner ones. The outer transport protocol is UDP, the inner
  could be either TCP or UDP. Only a single level of encapsulation
  offload is supported.

\item If VIRTIO_NET_F_HOST_UDP_TUNNEL_GSO is negotiated, a driver can
  additionally use TCP segmentation or UDP segmentation on top of UDP
  encapsulation with the outer header requiring checksum offload,
  negotiating the VIRTIO_NET_F_HOST_UDP_TUNNEL_GSO_CSUM feature.

\item The converse features are also available: a driver can save
  the virtual device some work by negotiating these features.\note{For example, a network packet transported between two guests on
the same system might not need checksumming at all, nor segmentation,
if both guests are amenable.}
   The VIRTIO_NET_F_GUEST_CSUM feature indicates that partially
  checksummed packets can be received, and if it can do that then
  the VIRTIO_NET_F_GUEST_TSO4, VIRTIO_NET_F_GUEST_TSO6,
  VIRTIO_NET_F_GUEST_UFO, VIRTIO_NET_F_GUEST_ECN, VIRTIO_NET_F_GUEST_USO4,
  VIRTIO_NET_F_GUEST_USO6 VIRTIO_NET_F_GUEST_UDP_TUNNEL_GSO and
  VIRTIO_NET_F_GUEST_UDP_TUNNEL_GSO_CSUM are the input equivalents of
  the features described above.
  See \ref{sec:Device Types / Network Device / Device Operation /
Setting Up Receive Buffers}~\nameref{sec:Device Types / Network
Device / Device Operation / Setting Up Receive Buffers} and
\ref{sec:Device Types / Network Device / Device Operation /
Processing of Incoming Packets}~\nameref{sec:Device Types /
Network Device / Device Operation / Processing of Incoming Packets} below.
\end{enumerate}

A truly minimal driver would only accept VIRTIO_NET_F_MAC and ignore
everything else.

\subsection{Device and driver capabilities}\label{sec:Device Types / Network Device / Device and driver capabilities}

The network device has the following capabilities.

\begin{tabularx}{\textwidth}{ |l||l|X| }
\hline
Identifier & Name & Description \\
\hline \hline
0x0800 & \hyperref[par:Device Types / Network Device / Device Operation / Flow filter / Device and driver capabilities / VIRTIO-NET-FF-RESOURCE-CAP]{VIRTIO_NET_FF_RESOURCE_CAP} & Flow filter resource capability \\
\hline
0x0801 & \hyperref[par:Device Types / Network Device / Device Operation / Flow filter / Device and driver capabilities / VIRTIO-NET-FF-SELECTOR-CAP]{VIRTIO_NET_FF_SELECTOR_CAP} & Flow filter classifier capability \\
\hline
0x0802 & \hyperref[par:Device Types / Network Device / Device Operation / Flow filter / Device and driver capabilities / VIRTIO-NET-FF-ACTION-CAP]{VIRTIO_NET_FF_ACTION_CAP} & Flow filter action capability \\
\hline
\end{tabularx}

\subsection{Device resource objects}\label{sec:Device Types / Network Device / Device resource objects}

The network device has the following resource objects.

\begin{tabularx}{\textwidth}{ |l||l|X| }
\hline
type & Name & Description \\
\hline \hline
0x0200 & \hyperref[par:Device Types / Network Device / Device Operation / Flow filter / Resource objects / VIRTIO-NET-RESOURCE-OBJ-FF-GROUP]{VIRTIO_NET_RESOURCE_OBJ_FF_GROUP} & Flow filter group resource object \\
\hline
0x0201 & \hyperref[par:Device Types / Network Device / Device Operation / Flow filter / Resource objects / VIRTIO-NET-RESOURCE-OBJ-FF-CLASSIFIER]{VIRTIO_NET_RESOURCE_OBJ_FF_CLASSIFIER} & Flow filter mask object \\
\hline
0x0202 & \hyperref[par:Device Types / Network Device / Device Operation / Flow filter / Resource objects / VIRTIO-NET-RESOURCE-OBJ-FF-RULE]{VIRTIO_NET_RESOURCE_OBJ_FF_RULE} & Flow filter rule object \\
\hline
\end{tabularx}

\subsection{Device parts}\label{sec:Device Types / Network Device / Device parts}

Network device parts represent the configuration done by the driver using control
virtqueue commands. Network device part is in the format of
\field{struct virtio_dev_part}.

\begin{tabularx}{\textwidth}{ |l||l|X| }
\hline
Type & Name & Description \\
\hline \hline
0x200 & VIRTIO_NET_DEV_PART_CVQ_CFG_PART & Represents device configuration done through a control virtqueue command, see \ref{sec:Device Types / Network Device / Device parts / VIRTIO-NET-DEV-PART-CVQ-CFG-PART} \\
\hline
0x201 - 0x5FF & - & reserved for future \\
\hline
\hline
\end{tabularx}

\subsubsection{VIRTIO_NET_DEV_PART_CVQ_CFG_PART}\label{sec:Device Types / Network Device / Device parts / VIRTIO-NET-DEV-PART-CVQ-CFG-PART}

For VIRTIO_NET_DEV_PART_CVQ_CFG_PART, \field{part_type} is set to 0x200. The
VIRTIO_NET_DEV_PART_CVQ_CFG_PART part indicates configuration performed by the
driver using a control virtqueue command.

\begin{lstlisting}
struct virtio_net_dev_part_cvq_selector {
        u8 class;
        u8 command;
        u8 reserved[6];
};
\end{lstlisting}

There is one device part of type VIRTIO_NET_DEV_PART_CVQ_CFG_PART for each
individual configuration. Each part is identified by a unique selector value.
The selector, \field{device_type_raw}, is in the format
\field{struct virtio_net_dev_part_cvq_selector}.

The selector consists of two fields: \field{class} and \field{command}. These
fields correspond to the \field{class} and \field{command} defined in
\field{struct virtio_net_ctrl}, as described in the relevant sections of
\ref{sec:Device Types / Network Device / Device Operation / Control Virtqueue}.

The value corresponding to each part’s selector follows the same format as the
respective \field{command-specific-data} described in the relevant sections of
\ref{sec:Device Types / Network Device / Device Operation / Control Virtqueue}.

For example, when the \field{class} is VIRTIO_NET_CTRL_MAC, the \field{command}
can be either VIRTIO_NET_CTRL_MAC_TABLE_SET or VIRTIO_NET_CTRL_MAC_ADDR_SET;
when \field{command} is set to VIRTIO_NET_CTRL_MAC_TABLE_SET, \field{value}
is in the format of \field{struct virtio_net_ctrl_mac}.

Supported selectors are listed in the table:

\begin{tabularx}{\textwidth}{ |l|X| }
\hline
Class selector & Command selector \\
\hline \hline
VIRTIO_NET_CTRL_RX & VIRTIO_NET_CTRL_RX_PROMISC \\
\hline
VIRTIO_NET_CTRL_RX & VIRTIO_NET_CTRL_RX_ALLMULTI \\
\hline
VIRTIO_NET_CTRL_RX & VIRTIO_NET_CTRL_RX_ALLUNI \\
\hline
VIRTIO_NET_CTRL_RX & VIRTIO_NET_CTRL_RX_NOMULTI \\
\hline
VIRTIO_NET_CTRL_RX & VIRTIO_NET_CTRL_RX_NOUNI \\
\hline
VIRTIO_NET_CTRL_RX & VIRTIO_NET_CTRL_RX_NOBCAST \\
\hline
VIRTIO_NET_CTRL_MAC & VIRTIO_NET_CTRL_MAC_TABLE_SET \\
\hline
VIRTIO_NET_CTRL_MAC & VIRTIO_NET_CTRL_MAC_ADDR_SET \\
\hline
VIRTIO_NET_CTRL_VLAN & VIRTIO_NET_CTRL_VLAN_ADD \\
\hline
VIRTIO_NET_CTRL_ANNOUNCE & VIRTIO_NET_CTRL_ANNOUNCE_ACK \\
\hline
VIRTIO_NET_CTRL_MQ & VIRTIO_NET_CTRL_MQ_VQ_PAIRS_SET \\
\hline
VIRTIO_NET_CTRL_MQ & VIRTIO_NET_CTRL_MQ_RSS_CONFIG \\
\hline
VIRTIO_NET_CTRL_MQ & VIRTIO_NET_CTRL_MQ_HASH_CONFIG \\
\hline
\hline
\end{tabularx}

For command selector VIRTIO_NET_CTRL_VLAN_ADD, device part consists of a whole
VLAN table.

\field{reserved} is reserved and set to zero.

\subsection{Device Operation}\label{sec:Device Types / Network Device / Device Operation}

Packets are transmitted by placing them in the
transmitq1\ldots transmitqN, and buffers for incoming packets are
placed in the receiveq1\ldots receiveqN. In each case, the packet
itself is preceded by a header:

\begin{lstlisting}
struct virtio_net_hdr {
#define VIRTIO_NET_HDR_F_NEEDS_CSUM    1
#define VIRTIO_NET_HDR_F_DATA_VALID    2
#define VIRTIO_NET_HDR_F_RSC_INFO      4
#define VIRTIO_NET_HDR_F_UDP_TUNNEL_CSUM 8
        u8 flags;
#define VIRTIO_NET_HDR_GSO_NONE        0
#define VIRTIO_NET_HDR_GSO_TCPV4       1
#define VIRTIO_NET_HDR_GSO_UDP         3
#define VIRTIO_NET_HDR_GSO_TCPV6       4
#define VIRTIO_NET_HDR_GSO_UDP_L4      5
#define VIRTIO_NET_HDR_GSO_UDP_TUNNEL_IPV4 0x20
#define VIRTIO_NET_HDR_GSO_UDP_TUNNEL_IPV6 0x40
#define VIRTIO_NET_HDR_GSO_ECN      0x80
        u8 gso_type;
        le16 hdr_len;
        le16 gso_size;
        le16 csum_start;
        le16 csum_offset;
        le16 num_buffers;
        le32 hash_value;        (Only if VIRTIO_NET_F_HASH_REPORT negotiated)
        le16 hash_report;       (Only if VIRTIO_NET_F_HASH_REPORT negotiated)
        le16 padding_reserved;  (Only if VIRTIO_NET_F_HASH_REPORT negotiated)
        le16 outer_th_offset    (Only if VIRTIO_NET_F_HOST_UDP_TUNNEL_GSO or VIRTIO_NET_F_GUEST_UDP_TUNNEL_GSO negotiated)
        le16 inner_nh_offset;   (Only if VIRTIO_NET_F_HOST_UDP_TUNNEL_GSO or VIRTIO_NET_F_GUEST_UDP_TUNNEL_GSO negotiated)
};
\end{lstlisting}

The controlq is used to control various device features described further in
section \ref{sec:Device Types / Network Device / Device Operation / Control Virtqueue}.

\subsubsection{Legacy Interface: Device Operation}\label{sec:Device Types / Network Device / Device Operation / Legacy Interface: Device Operation}
When using the legacy interface, transitional devices and drivers
MUST format the fields in \field{struct virtio_net_hdr}
according to the native endian of the guest rather than
(necessarily when not using the legacy interface) little-endian.

The legacy driver only presented \field{num_buffers} in the \field{struct virtio_net_hdr}
when VIRTIO_NET_F_MRG_RXBUF was negotiated; without that feature the
structure was 2 bytes shorter.

When using the legacy interface, the driver SHOULD ignore the
used length for the transmit queues
and the controlq queue.
\begin{note}
Historically, some devices put
the total descriptor length there, even though no data was
actually written.
\end{note}

\subsubsection{Packet Transmission}\label{sec:Device Types / Network Device / Device Operation / Packet Transmission}

Transmitting a single packet is simple, but varies depending on
the different features the driver negotiated.

\begin{enumerate}
\item The driver can send a completely checksummed packet.  In this case,
  \field{flags} will be zero, and \field{gso_type} will be VIRTIO_NET_HDR_GSO_NONE.

\item If the driver negotiated VIRTIO_NET_F_CSUM, it can skip
  checksumming the packet:
  \begin{itemize}
  \item \field{flags} has the VIRTIO_NET_HDR_F_NEEDS_CSUM set,

  \item \field{csum_start} is set to the offset within the packet to begin checksumming,
    and

  \item \field{csum_offset} indicates how many bytes after the csum_start the
    new (16 bit ones' complement) checksum is placed by the device.

  \item The TCP checksum field in the packet is set to the sum
    of the TCP pseudo header, so that replacing it by the ones'
    complement checksum of the TCP header and body will give the
    correct result.
  \end{itemize}

\begin{note}
For example, consider a partially checksummed TCP (IPv4) packet.
It will have a 14 byte ethernet header and 20 byte IP header
followed by the TCP header (with the TCP checksum field 16 bytes
into that header). \field{csum_start} will be 14+20 = 34 (the TCP
checksum includes the header), and \field{csum_offset} will be 16.
If the given packet has the VIRTIO_NET_HDR_GSO_UDP_TUNNEL_IPV4 bit or the
VIRTIO_NET_HDR_GSO_UDP_TUNNEL_IPV6 bit set,
the above checksum fields refer to the inner header checksum, see
the example below.
\end{note}

\item If the driver negotiated
  VIRTIO_NET_F_HOST_TSO4, TSO6, USO or UFO, and the packet requires
  TCP segmentation, UDP segmentation or fragmentation, then \field{gso_type}
  is set to VIRTIO_NET_HDR_GSO_TCPV4, TCPV6, UDP_L4 or UDP.
  (Otherwise, it is set to VIRTIO_NET_HDR_GSO_NONE). In this
  case, packets larger than 1514 bytes can be transmitted: the
  metadata indicates how to replicate the packet header to cut it
  into smaller packets. The other gso fields are set:

  \begin{itemize}
  \item If the VIRTIO_NET_F_GUEST_HDRLEN feature has been negotiated,
    \field{hdr_len} indicates the header length that needs to be replicated
    for each packet. It's the number of bytes from the beginning of the packet
    to the beginning of the transport payload.
    If the \field{gso_type} has the VIRTIO_NET_HDR_GSO_UDP_TUNNEL_IPV4 bit or
    VIRTIO_NET_HDR_GSO_UDP_TUNNEL_IPV6 bit set, \field{hdr_len} accounts for
    all the headers up to and including the inner transport.
    Otherwise, if the VIRTIO_NET_F_GUEST_HDRLEN feature has not been negotiated,
    \field{hdr_len} is a hint to the device as to how much of the header
    needs to be kept to copy into each packet, usually set to the
    length of the headers, including the transport header\footnote{Due to various bugs in implementations, this field is not useful
as a guarantee of the transport header size.
}.

  \begin{note}
  Some devices benefit from knowledge of the exact header length.
  \end{note}

  \item \field{gso_size} is the maximum size of each packet beyond that
    header (ie. MSS).

  \item If the driver negotiated the VIRTIO_NET_F_HOST_ECN feature,
    the VIRTIO_NET_HDR_GSO_ECN bit in \field{gso_type}
    indicates that the TCP packet has the ECN bit set\footnote{This case is not handled by some older hardware, so is called out
specifically in the protocol.}.
   \end{itemize}

\item If the driver negotiated the VIRTIO_NET_F_HOST_UDP_TUNNEL_GSO feature and the
  \field{gso_type} has the VIRTIO_NET_HDR_GSO_UDP_TUNNEL_IPV4 bit or
  VIRTIO_NET_HDR_GSO_UDP_TUNNEL_IPV6 bit set, the GSO protocol is encapsulated
  in a UDP tunnel.
  If the outer UDP header requires checksumming, the driver must have
  additionally negotiated the VIRTIO_NET_F_HOST_UDP_TUNNEL_GSO_CSUM feature
  and offloaded the outer checksum accordingly, otherwise
  the outer UDP header must not require checksum validation, i.e. the outer
  UDP checksum must be positive zero (0x0) as defined in UDP RFC 768.
  The other tunnel-related fields indicate how to replicate the packet
  headers to cut it into smaller packets:

  \begin{itemize}
  \item \field{outer_th_offset} field indicates the outer transport header within
      the packet. This field differs from \field{csum_start} as the latter
      points to the inner transport header within the packet.

  \item \field{inner_nh_offset} field indicates the inner network header within
      the packet.
  \end{itemize}

\begin{note}
For example, consider a partially checksummed TCP (IPv4) packet carried over a
Geneve UDP tunnel (again IPv4) with no tunnel options. The
only relevant variable related to the tunnel type is the tunnel header length.
The packet will have a 14 byte outer ethernet header, 20 byte outer IP header
followed by the 8 byte UDP header (with a 0 checksum value), 8 byte Geneve header,
14 byte inner ethernet header, 20 byte inner IP header
and the TCP header (with the TCP checksum field 16 bytes
into that header). \field{csum_start} will be 14+20+8+8+14+20 = 84 (the TCP
checksum includes the header), \field{csum_offset} will be 16.
\field{inner_nh_offset} will be 14+20+8+8+14 = 62, \field{outer_th_offset} will be
14+20+8 = 42 and \field{gso_type} will be
VIRTIO_NET_HDR_GSO_TCPV4 | VIRTIO_NET_HDR_GSO_UDP_TUNNEL_IPV4 = 0x21
\end{note}

\item If the driver negotiated the VIRTIO_NET_F_HOST_UDP_TUNNEL_GSO_CSUM feature,
  the transmitted packet is a GSO one encapsulated in a UDP tunnel, and
  the outer UDP header requires checksumming, the driver can skip checksumming
  the outer header:

  \begin{itemize}
  \item \field{flags} has the VIRTIO_NET_HDR_F_UDP_TUNNEL_CSUM set,

  \item The outer UDP checksum field in the packet is set to the sum
    of the UDP pseudo header, so that replacing it by the ones'
    complement checksum of the outer UDP header and payload will give the
    correct result.
  \end{itemize}

\item \field{num_buffers} is set to zero.  This field is unused on transmitted packets.

\item The header and packet are added as one output descriptor to the
  transmitq, and the device is notified of the new entry
  (see \ref{sec:Device Types / Network Device / Device Initialization}~\nameref{sec:Device Types / Network Device / Device Initialization}).
\end{enumerate}

\drivernormative{\paragraph}{Packet Transmission}{Device Types / Network Device / Device Operation / Packet Transmission}

For the transmit packet buffer, the driver MUST use the size of the
structure \field{struct virtio_net_hdr} same as the receive packet buffer.

The driver MUST set \field{num_buffers} to zero.

If VIRTIO_NET_F_CSUM is not negotiated, the driver MUST set
\field{flags} to zero and SHOULD supply a fully checksummed
packet to the device.

If VIRTIO_NET_F_HOST_TSO4 is negotiated, the driver MAY set
\field{gso_type} to VIRTIO_NET_HDR_GSO_TCPV4 to request TCPv4
segmentation, otherwise the driver MUST NOT set
\field{gso_type} to VIRTIO_NET_HDR_GSO_TCPV4.

If VIRTIO_NET_F_HOST_TSO6 is negotiated, the driver MAY set
\field{gso_type} to VIRTIO_NET_HDR_GSO_TCPV6 to request TCPv6
segmentation, otherwise the driver MUST NOT set
\field{gso_type} to VIRTIO_NET_HDR_GSO_TCPV6.

If VIRTIO_NET_F_HOST_UFO is negotiated, the driver MAY set
\field{gso_type} to VIRTIO_NET_HDR_GSO_UDP to request UDP
fragmentation, otherwise the driver MUST NOT set
\field{gso_type} to VIRTIO_NET_HDR_GSO_UDP.

If VIRTIO_NET_F_HOST_USO is negotiated, the driver MAY set
\field{gso_type} to VIRTIO_NET_HDR_GSO_UDP_L4 to request UDP
segmentation, otherwise the driver MUST NOT set
\field{gso_type} to VIRTIO_NET_HDR_GSO_UDP_L4.

The driver SHOULD NOT send to the device TCP packets requiring segmentation offload
which have the Explicit Congestion Notification bit set, unless the
VIRTIO_NET_F_HOST_ECN feature is negotiated, in which case the
driver MUST set the VIRTIO_NET_HDR_GSO_ECN bit in
\field{gso_type}.

If VIRTIO_NET_F_HOST_UDP_TUNNEL_GSO is negotiated, the driver MAY set
VIRTIO_NET_HDR_GSO_UDP_TUNNEL_IPV4 bit or the VIRTIO_NET_HDR_GSO_UDP_TUNNEL_IPV6 bit
in \field{gso_type} according to the inner network header protocol type
to request GSO packets over UDPv4 or UDPv6 tunnel segmentation,
otherwise the driver MUST NOT set either the
VIRTIO_NET_HDR_GSO_UDP_TUNNEL_IPV4 bit or the VIRTIO_NET_HDR_GSO_UDP_TUNNEL_IPV6 bit
in \field{gso_type}.

When requesting GSO segmentation over UDP tunnel, the driver MUST SET the
VIRTIO_NET_HDR_GSO_UDP_TUNNEL_IPV4 bit if the inner network header is IPv4, i.e. the
packet is a TCPv4 GSO one, otherwise, if the inner network header is IPv6, the driver
MUST SET the VIRTIO_NET_HDR_GSO_UDP_TUNNEL_IPV6 bit.

The driver MUST NOT send to the device GSO packets over UDP tunnel
requiring segmentation and outer UDP checksum offload, unless both the
VIRTIO_NET_F_HOST_UDP_TUNNEL_GSO and VIRTIO_NET_F_HOST_UDP_TUNNEL_GSO_CSUM features
are negotiated, in which case the driver MUST set either the
VIRTIO_NET_HDR_GSO_UDP_TUNNEL_IPV4 bit or the VIRTIO_NET_HDR_GSO_UDP_TUNNEL_IPV6
bit in the \field{gso_type} and the VIRTIO_NET_HDR_F_UDP_TUNNEL_CSUM bit in
the \field{flags}.

If VIRTIO_NET_F_HOST_UDP_TUNNEL_GSO_CSUM is not negotiated, the driver MUST not set
the VIRTIO_NET_HDR_F_UDP_TUNNEL_CSUM bit in the \field{flags} and
MUST NOT send to the device GSO packets over UDP tunnel
requiring segmentation and outer UDP checksum offload.

The driver MUST NOT set the VIRTIO_NET_HDR_GSO_UDP_TUNNEL_IPV4 bit or the
VIRTIO_NET_HDR_GSO_UDP_TUNNEL_IPV6 bit together with VIRTIO_NET_HDR_GSO_UDP, as the
latter is deprecated in favor of UDP_L4 and no new feature will support it.

The driver MUST NOT set the VIRTIO_NET_HDR_GSO_UDP_TUNNEL_IPV4 bit and the
VIRTIO_NET_HDR_GSO_UDP_TUNNEL_IPV6 bit together.

The driver MUST NOT set the VIRTIO_NET_HDR_F_UDP_TUNNEL_CSUM bit \field{flags}
without setting either the VIRTIO_NET_HDR_GSO_UDP_TUNNEL_IPV4 bit or
the VIRTIO_NET_HDR_GSO_UDP_TUNNEL_IPV6 bit in \field{gso_type}.

If the VIRTIO_NET_F_CSUM feature has been negotiated, the
driver MAY set the VIRTIO_NET_HDR_F_NEEDS_CSUM bit in
\field{flags}, if so:
\begin{enumerate}
\item the driver MUST validate the packet checksum at
	offset \field{csum_offset} from \field{csum_start} as well as all
	preceding offsets;
\begin{note}
If \field{gso_type} differs from VIRTIO_NET_HDR_GSO_NONE and the
VIRTIO_NET_HDR_GSO_UDP_TUNNEL_IPV4 bit or the VIRTIO_NET_HDR_GSO_UDP_TUNNEL_IPV6
bit are not set in \field{gso_type}, \field{csum_offset}
points to the only transport header present in the packet, and there are no
additional preceding checksums validated by VIRTIO_NET_HDR_F_NEEDS_CSUM.
\end{note}
\item the driver MUST set the packet checksum stored in the
	buffer to the TCP/UDP pseudo header;
\item the driver MUST set \field{csum_start} and
	\field{csum_offset} such that calculating a ones'
	complement checksum from \field{csum_start} up until the end of
	the packet and storing the result at offset \field{csum_offset}
	from  \field{csum_start} will result in a fully checksummed
	packet;
\end{enumerate}

If none of the VIRTIO_NET_F_HOST_TSO4, TSO6, USO or UFO options have
been negotiated, the driver MUST set \field{gso_type} to
VIRTIO_NET_HDR_GSO_NONE.

If \field{gso_type} differs from VIRTIO_NET_HDR_GSO_NONE, then
the driver MUST also set the VIRTIO_NET_HDR_F_NEEDS_CSUM bit in
\field{flags} and MUST set \field{gso_size} to indicate the
desired MSS.

If one of the VIRTIO_NET_F_HOST_TSO4, TSO6, USO or UFO options have
been negotiated:
\begin{itemize}
\item If the VIRTIO_NET_F_GUEST_HDRLEN feature has been negotiated,
	and \field{gso_type} differs from VIRTIO_NET_HDR_GSO_NONE,
	the driver MUST set \field{hdr_len} to a value equal to the length
	of the headers, including the transport header. If \field{gso_type}
	has the VIRTIO_NET_HDR_GSO_UDP_TUNNEL_IPV4 bit or the
	VIRTIO_NET_HDR_GSO_UDP_TUNNEL_IPV6 bit set, \field{hdr_len} includes
	the inner transport header.

\item If the VIRTIO_NET_F_GUEST_HDRLEN feature has not been negotiated,
	or \field{gso_type} is VIRTIO_NET_HDR_GSO_NONE,
	the driver SHOULD set \field{hdr_len} to a value
	not less than the length of the headers, including the transport
	header.
\end{itemize}

If the VIRTIO_NET_F_HOST_UDP_TUNNEL_GSO option has been negotiated, the
driver MAY set the VIRTIO_NET_HDR_GSO_UDP_TUNNEL_IPV4 bit or the
VIRTIO_NET_HDR_GSO_UDP_TUNNEL_IPV6 bit in \field{gso_type}, if so:
\begin{itemize}
\item the driver MUST set \field{outer_th_offset} to the outer UDP header
  offset and \field{inner_nh_offset} to the inner network header offset.
  The \field{csum_start} and \field{csum_offset} fields point respectively
  to the inner transport header and inner transport checksum field.
\end{itemize}

If the VIRTIO_NET_F_HOST_UDP_TUNNEL_GSO_CSUM feature has been negotiated,
and the VIRTIO_NET_HDR_GSO_UDP_TUNNEL_IPV4 bit or
VIRTIO_NET_HDR_GSO_UDP_TUNNEL_IPV6 bit in \field{gso_type} are set,
the driver MAY set the VIRTIO_NET_HDR_F_UDP_TUNNEL_CSUM bit in
\field{flags}, if so the driver MUST set the packet outer UDP header checksum
to the outer UDP pseudo header checksum.

\begin{note}
calculating a ones' complement checksum from \field{outer_th_offset}
up until the end of the packet and storing the result at offset 6
from \field{outer_th_offset} will result in a fully checksummed outer UDP packet;
\end{note}

If the VIRTIO_NET_HDR_GSO_UDP_TUNNEL_IPV4 bit or the
VIRTIO_NET_HDR_GSO_UDP_TUNNEL_IPV6 bit in \field{gso_type} are set
and the VIRTIO_NET_F_HOST_UDP_TUNNEL_GSO_CSUM feature has not
been negotiated, the
outer UDP header MUST NOT require checksum validation. That is, the
outer UDP checksum value MUST be 0 or the validated complete checksum
for such header.

\begin{note}
The valid complete checksum of the outer UDP header of individual segments
can be computed by the driver prior to segmentation only if the GSO packet
size is a multiple of \field{gso_size}, because then all segments
have the same size and thus all data included in the outer UDP
checksum is the same for every segment. These pre-computed segment
length and checksum fields are different from those of the GSO
packet.
In this scenario the outer UDP header of the GSO packet must carry the
segmented UDP packet length.
\end{note}

If the VIRTIO_NET_F_HOST_UDP_TUNNEL_GSO option has not
been negotiated, the driver MUST NOT set either the VIRTIO_NET_HDR_F_GSO_UDP_TUNNEL_IPV4
bit or the VIRTIO_NET_HDR_F_GSO_UDP_TUNNEL_IPV6 in \field{gso_type}.

If the VIRTIO_NET_F_HOST_UDP_TUNNEL_GSO_CSUM option has not been negotiated,
the driver MUST NOT set the VIRTIO_NET_HDR_F_UDP_TUNNEL_CSUM bit
in \field{flags}.

The driver SHOULD accept the VIRTIO_NET_F_GUEST_HDRLEN feature if it has
been offered, and if it's able to provide the exact header length.

The driver MUST NOT set the VIRTIO_NET_HDR_F_DATA_VALID and
VIRTIO_NET_HDR_F_RSC_INFO bits in \field{flags}.

The driver MUST NOT set the VIRTIO_NET_HDR_F_DATA_VALID bit in \field{flags}
together with the VIRTIO_NET_HDR_F_GSO_UDP_TUNNEL_IPV4 bit or the
VIRTIO_NET_HDR_F_GSO_UDP_TUNNEL_IPV6 bit in \field{gso_type}.

\devicenormative{\paragraph}{Packet Transmission}{Device Types / Network Device / Device Operation / Packet Transmission}
The device MUST ignore \field{flag} bits that it does not recognize.

If VIRTIO_NET_HDR_F_NEEDS_CSUM bit in \field{flags} is not set, the
device MUST NOT use the \field{csum_start} and \field{csum_offset}.

If one of the VIRTIO_NET_F_HOST_TSO4, TSO6, USO or UFO options have
been negotiated:
\begin{itemize}
\item If the VIRTIO_NET_F_GUEST_HDRLEN feature has been negotiated,
	and \field{gso_type} differs from VIRTIO_NET_HDR_GSO_NONE,
	the device MAY use \field{hdr_len} as the transport header size.

	\begin{note}
	Caution should be taken by the implementation so as to prevent
	a malicious driver from attacking the device by setting an incorrect hdr_len.
	\end{note}

\item If the VIRTIO_NET_F_GUEST_HDRLEN feature has not been negotiated,
	or \field{gso_type} is VIRTIO_NET_HDR_GSO_NONE,
	the device MAY use \field{hdr_len} only as a hint about the
	transport header size.
	The device MUST NOT rely on \field{hdr_len} to be correct.

	\begin{note}
	This is due to various bugs in implementations.
	\end{note}
\end{itemize}

If both the VIRTIO_NET_HDR_GSO_UDP_TUNNEL_IPV4 bit and
the VIRTIO_NET_HDR_GSO_UDP_TUNNEL_IPV6 bit in in \field{gso_type} are set,
the device MUST NOT accept the packet.

If the VIRTIO_NET_HDR_GSO_UDP_TUNNEL_IPV4 bit and the VIRTIO_NET_HDR_GSO_UDP_TUNNEL_IPV6
bit in \field{gso_type} are not set, the device MUST NOT use the
\field{outer_th_offset} and \field{inner_nh_offset}.

If either the VIRTIO_NET_HDR_GSO_UDP_TUNNEL_IPV4 bit or
the VIRTIO_NET_HDR_GSO_UDP_TUNNEL_IPV6 bit in \field{gso_type} are set, and any of
the following is true:
\begin{itemize}
\item the VIRTIO_NET_HDR_F_NEEDS_CSUM is not set in \field{flags}
\item the VIRTIO_NET_HDR_F_DATA_VALID is set in \field{flags}
\item the \field{gso_type} excluding the VIRTIO_NET_HDR_GSO_UDP_TUNNEL_IPV4
bit and the VIRTIO_NET_HDR_GSO_UDP_TUNNEL_IPV6 bit is VIRTIO_NET_HDR_GSO_NONE
\end{itemize}
the device MUST NOT accept the packet.

If the VIRTIO_NET_HDR_F_UDP_TUNNEL_CSUM bit in \field{flags} is set,
and both the bits VIRTIO_NET_HDR_GSO_UDP_TUNNEL_IPV4 and
VIRTIO_NET_HDR_GSO_UDP_TUNNEL_IPV6 in \field{gso_type} are not set,
the device MOST NOT accept the packet.

If VIRTIO_NET_HDR_F_NEEDS_CSUM is not set, the device MUST NOT
rely on the packet checksum being correct.
\paragraph{Packet Transmission Interrupt}\label{sec:Device Types / Network Device / Device Operation / Packet Transmission / Packet Transmission Interrupt}

Often a driver will suppress transmission virtqueue interrupts
and check for used packets in the transmit path of following
packets.

The normal behavior in this interrupt handler is to retrieve
used buffers from the virtqueue and free the corresponding
headers and packets.

\subsubsection{Setting Up Receive Buffers}\label{sec:Device Types / Network Device / Device Operation / Setting Up Receive Buffers}

It is generally a good idea to keep the receive virtqueue as
fully populated as possible: if it runs out, network performance
will suffer.

If the VIRTIO_NET_F_GUEST_TSO4, VIRTIO_NET_F_GUEST_TSO6,
VIRTIO_NET_F_GUEST_UFO, VIRTIO_NET_F_GUEST_USO4 or VIRTIO_NET_F_GUEST_USO6
features are used, the maximum incoming packet
will be 65589 bytes long (14 bytes of Ethernet header, plus 40 bytes of
the IPv6 header, plus 65535 bytes of maximum IPv6 payload including any
extension header), otherwise 1514 bytes.
When VIRTIO_NET_F_HASH_REPORT is not negotiated, the required receive buffer
size is either 65601 or 1526 bytes accounting for 20 bytes of
\field{struct virtio_net_hdr} followed by receive packet.
When VIRTIO_NET_F_HASH_REPORT is negotiated, the required receive buffer
size is either 65609 or 1534 bytes accounting for 12 bytes of
\field{struct virtio_net_hdr} followed by receive packet.

\drivernormative{\paragraph}{Setting Up Receive Buffers}{Device Types / Network Device / Device Operation / Setting Up Receive Buffers}

\begin{itemize}
\item If VIRTIO_NET_F_MRG_RXBUF is not negotiated:
  \begin{itemize}
    \item If VIRTIO_NET_F_GUEST_TSO4, VIRTIO_NET_F_GUEST_TSO6, VIRTIO_NET_F_GUEST_UFO,
	VIRTIO_NET_F_GUEST_USO4 or VIRTIO_NET_F_GUEST_USO6 are negotiated, the driver SHOULD populate
      the receive queue(s) with buffers of at least 65609 bytes if
      VIRTIO_NET_F_HASH_REPORT is negotiated, and of at least 65601 bytes if not.
    \item Otherwise, the driver SHOULD populate the receive queue(s)
      with buffers of at least 1534 bytes if VIRTIO_NET_F_HASH_REPORT
      is negotiated, and of at least 1526 bytes if not.
  \end{itemize}
\item If VIRTIO_NET_F_MRG_RXBUF is negotiated, each buffer MUST be at
least size of \field{struct virtio_net_hdr},
i.e. 20 bytes if VIRTIO_NET_F_HASH_REPORT is negotiated, and 12 bytes if not.
\end{itemize}

\begin{note}
Obviously each buffer can be split across multiple descriptor elements.
\end{note}

When calculating the size of \field{struct virtio_net_hdr}, the driver
MUST consider all the fields inclusive up to \field{padding_reserved},
i.e. 20 bytes if VIRTIO_NET_F_HASH_REPORT is negotiated, and 12 bytes if not.

If VIRTIO_NET_F_MQ is negotiated, each of receiveq1\ldots receiveqN
that will be used SHOULD be populated with receive buffers.

\devicenormative{\paragraph}{Setting Up Receive Buffers}{Device Types / Network Device / Device Operation / Setting Up Receive Buffers}

The device MUST set \field{num_buffers} to the number of descriptors used to
hold the incoming packet.

The device MUST use only a single descriptor if VIRTIO_NET_F_MRG_RXBUF
was not negotiated.
\begin{note}
{This means that \field{num_buffers} will always be 1
if VIRTIO_NET_F_MRG_RXBUF is not negotiated.}
\end{note}

\subsubsection{Processing of Incoming Packets}\label{sec:Device Types / Network Device / Device Operation / Processing of Incoming Packets}
\label{sec:Device Types / Network Device / Device Operation / Processing of Packets}%old label for latexdiff

When a packet is copied into a buffer in the receiveq, the
optimal path is to disable further used buffer notifications for the
receiveq and process packets until no more are found, then re-enable
them.

Processing incoming packets involves:

\begin{enumerate}
\item \field{num_buffers} indicates how many descriptors
  this packet is spread over (including this one): this will
  always be 1 if VIRTIO_NET_F_MRG_RXBUF was not negotiated.
  This allows receipt of large packets without having to allocate large
  buffers: a packet that does not fit in a single buffer can flow
  over to the next buffer, and so on. In this case, there will be
  at least \field{num_buffers} used buffers in the virtqueue, and the device
  chains them together to form a single packet in a way similar to
  how it would store it in a single buffer spread over multiple
  descriptors.
  The other buffers will not begin with a \field{struct virtio_net_hdr}.

\item If
  \field{num_buffers} is one, then the entire packet will be
  contained within this buffer, immediately following the struct
  virtio_net_hdr.
\item If the VIRTIO_NET_F_GUEST_CSUM feature was negotiated, the
  VIRTIO_NET_HDR_F_DATA_VALID bit in \field{flags} can be
  set: if so, device has validated the packet checksum.
  If the VIRTIO_NET_F_GUEST_UDP_TUNNEL_GSO_CSUM feature has been negotiated,
  and the VIRTIO_NET_HDR_F_UDP_TUNNEL_CSUM bit is set in \field{flags},
  both the outer UDP checksum and the inner transport checksum
  have been validated, otherwise only one level of checksums (the outer one
  in case of tunnels) has been validated.
\end{enumerate}

Additionally, VIRTIO_NET_F_GUEST_CSUM, TSO4, TSO6, UDP, UDP_TUNNEL
and ECN features enable receive checksum, large receive offload and ECN
support which are the input equivalents of the transmit checksum,
transmit segmentation offloading and ECN features, as described
in \ref{sec:Device Types / Network Device / Device Operation /
Packet Transmission}:
\begin{enumerate}
\item If the VIRTIO_NET_F_GUEST_TSO4, TSO6, UFO, USO4 or USO6 options were
  negotiated, then \field{gso_type} MAY be something other than
  VIRTIO_NET_HDR_GSO_NONE, and \field{gso_size} field indicates the
  desired MSS (see Packet Transmission point 2).
\item If the VIRTIO_NET_F_RSC_EXT option was negotiated (this
  implies one of VIRTIO_NET_F_GUEST_TSO4, TSO6), the
  device processes also duplicated ACK segments, reports
  number of coalesced TCP segments in \field{csum_start} field and
  number of duplicated ACK segments in \field{csum_offset} field
  and sets bit VIRTIO_NET_HDR_F_RSC_INFO in \field{flags}.
\item If the VIRTIO_NET_F_GUEST_CSUM feature was negotiated, the
  VIRTIO_NET_HDR_F_NEEDS_CSUM bit in \field{flags} can be
  set: if so, the packet checksum at offset \field{csum_offset}
  from \field{csum_start} and any preceding checksums
  have been validated.  The checksum on the packet is incomplete and
  if bit VIRTIO_NET_HDR_F_RSC_INFO is not set in \field{flags},
  then \field{csum_start} and \field{csum_offset} indicate how to calculate it
  (see Packet Transmission point 1).
\begin{note}
If \field{gso_type} differs from VIRTIO_NET_HDR_GSO_NONE and the
VIRTIO_NET_HDR_GSO_UDP_TUNNEL_IPV4 bit or the VIRTIO_NET_HDR_GSO_UDP_TUNNEL_IPV6
bit are not set, \field{csum_offset}
points to the only transport header present in the packet, and there are no
additional preceding checksums validated by VIRTIO_NET_HDR_F_NEEDS_CSUM.
\end{note}
\item If the VIRTIO_NET_F_GUEST_UDP_TUNNEL_GSO option was negotiated and
  \field{gso_type} is not VIRTIO_NET_HDR_GSO_NONE, the
  VIRTIO_NET_HDR_GSO_UDP_TUNNEL_IPV4 bit or the VIRTIO_NET_HDR_GSO_UDP_TUNNEL_IPV6
  bit MAY be set. In such case the \field{outer_th_offset} and
  \field{inner_nh_offset} fields indicate the corresponding
  headers information.
\item If the VIRTIO_NET_F_GUEST_UDP_TUNNEL_GSO_CSUM feature was
negotiated, and
  the VIRTIO_NET_HDR_GSO_UDP_TUNNEL_IPV4 bit or the VIRTIO_NET_HDR_GSO_UDP_TUNNEL_IPV6
  are set in \field{gso_type}, the VIRTIO_NET_HDR_F_UDP_TUNNEL_CSUM bit in the
  \field{flags} can be set: if so, the outer UDP checksum has been validated
  and the UDP header checksum at offset 6 from from \field{outer_th_offset}
  is set to the outer UDP pseudo header checksum.

\begin{note}
If the VIRTIO_NET_HDR_GSO_UDP_TUNNEL_IPV4 bit or VIRTIO_NET_HDR_GSO_UDP_TUNNEL_IPV6
bit are set in \field{gso_type}, the \field{csum_start} field refers to
the inner transport header offset (see Packet Transmission point 1).
If the VIRTIO_NET_HDR_F_UDP_TUNNEL_CSUM bit in \field{flags} is set both
the inner and the outer header checksums have been validated by
VIRTIO_NET_HDR_F_NEEDS_CSUM, otherwise only the inner transport header
checksum has been validated.
\end{note}
\end{enumerate}

If applicable, the device calculates per-packet hash for incoming packets as
defined in \ref{sec:Device Types / Network Device / Device Operation / Processing of Incoming Packets / Hash calculation for incoming packets}.

If applicable, the device reports hash information for incoming packets as
defined in \ref{sec:Device Types / Network Device / Device Operation / Processing of Incoming Packets / Hash reporting for incoming packets}.

\devicenormative{\paragraph}{Processing of Incoming Packets}{Device Types / Network Device / Device Operation / Processing of Incoming Packets}
\label{devicenormative:Device Types / Network Device / Device Operation / Processing of Packets}%old label for latexdiff

If VIRTIO_NET_F_MRG_RXBUF has not been negotiated, the device MUST set
\field{num_buffers} to 1.

If VIRTIO_NET_F_MRG_RXBUF has been negotiated, the device MUST set
\field{num_buffers} to indicate the number of buffers
the packet (including the header) is spread over.

If a receive packet is spread over multiple buffers, the device
MUST use all buffers but the last (i.e. the first \field{num_buffers} -
1 buffers) completely up to the full length of each buffer
supplied by the driver.

The device MUST use all buffers used by a single receive
packet together, such that at least \field{num_buffers} are
observed by driver as used.

If VIRTIO_NET_F_GUEST_CSUM is not negotiated, the device MUST set
\field{flags} to zero and SHOULD supply a fully checksummed
packet to the driver.

If VIRTIO_NET_F_GUEST_TSO4 is not negotiated, the device MUST NOT set
\field{gso_type} to VIRTIO_NET_HDR_GSO_TCPV4.

If VIRTIO_NET_F_GUEST_UDP is not negotiated, the device MUST NOT set
\field{gso_type} to VIRTIO_NET_HDR_GSO_UDP.

If VIRTIO_NET_F_GUEST_TSO6 is not negotiated, the device MUST NOT set
\field{gso_type} to VIRTIO_NET_HDR_GSO_TCPV6.

If none of VIRTIO_NET_F_GUEST_USO4 or VIRTIO_NET_F_GUEST_USO6 have been negotiated,
the device MUST NOT set \field{gso_type} to VIRTIO_NET_HDR_GSO_UDP_L4.

If VIRTIO_NET_F_GUEST_UDP_TUNNEL_GSO is not negotiated, the device MUST NOT set
either the VIRTIO_NET_HDR_GSO_UDP_TUNNEL_IPV4 bit or the
VIRTIO_NET_HDR_GSO_UDP_TUNNEL_IPV6 bit in \field{gso_type}.

If VIRTIO_NET_F_GUEST_UDP_TUNNEL_GSO_CSUM is not negotiated the device MUST NOT set
the VIRTIO_NET_HDR_F_UDP_TUNNEL_CSUM bit in \field{flags}.

The device SHOULD NOT send to the driver TCP packets requiring segmentation offload
which have the Explicit Congestion Notification bit set, unless the
VIRTIO_NET_F_GUEST_ECN feature is negotiated, in which case the
device MUST set the VIRTIO_NET_HDR_GSO_ECN bit in
\field{gso_type}.

If the VIRTIO_NET_F_GUEST_CSUM feature has been negotiated, the
device MAY set the VIRTIO_NET_HDR_F_NEEDS_CSUM bit in
\field{flags}, if so:
\begin{enumerate}
\item the device MUST validate the packet checksum at
	offset \field{csum_offset} from \field{csum_start} as well as all
	preceding offsets;
\item the device MUST set the packet checksum stored in the
	receive buffer to the TCP/UDP pseudo header;
\item the device MUST set \field{csum_start} and
	\field{csum_offset} such that calculating a ones'
	complement checksum from \field{csum_start} up until the
	end of the packet and storing the result at offset
	\field{csum_offset} from  \field{csum_start} will result in a
	fully checksummed packet;
\end{enumerate}

The device MUST NOT send to the driver GSO packets encapsulated in UDP
tunnel and requiring segmentation offload, unless the
VIRTIO_NET_F_GUEST_UDP_TUNNEL_GSO is negotiated, in which case the device MUST set
the VIRTIO_NET_HDR_GSO_UDP_TUNNEL_IPV4 bit or the VIRTIO_NET_HDR_GSO_UDP_TUNNEL_IPV6
bit in \field{gso_type} according to the inner network header protocol type,
MUST set the \field{outer_th_offset} and \field{inner_nh_offset} fields
to the corresponding header information, and the outer UDP header MUST NOT
require checksum offload.

If the VIRTIO_NET_F_GUEST_UDP_TUNNEL_GSO_CSUM feature has not been negotiated,
the device MUST NOT send the driver GSO packets encapsulated in UDP
tunnel and requiring segmentation and outer checksum offload.

If none of the VIRTIO_NET_F_GUEST_TSO4, TSO6, UFO, USO4 or USO6 options have
been negotiated, the device MUST set \field{gso_type} to
VIRTIO_NET_HDR_GSO_NONE.

If \field{gso_type} differs from VIRTIO_NET_HDR_GSO_NONE, then
the device MUST also set the VIRTIO_NET_HDR_F_NEEDS_CSUM bit in
\field{flags} MUST set \field{gso_size} to indicate the desired MSS.
If VIRTIO_NET_F_RSC_EXT was negotiated, the device MUST also
set VIRTIO_NET_HDR_F_RSC_INFO bit in \field{flags},
set \field{csum_start} to number of coalesced TCP segments and
set \field{csum_offset} to number of received duplicated ACK segments.

If VIRTIO_NET_F_RSC_EXT was not negotiated, the device MUST
not set VIRTIO_NET_HDR_F_RSC_INFO bit in \field{flags}.

If one of the VIRTIO_NET_F_GUEST_TSO4, TSO6, UFO, USO4 or USO6 options have
been negotiated, the device SHOULD set \field{hdr_len} to a value
not less than the length of the headers, including the transport
header. If \field{gso_type} has the VIRTIO_NET_HDR_GSO_UDP_TUNNEL_IPV4 bit
or the VIRTIO_NET_HDR_GSO_UDP_TUNNEL_IPV6 bit set, the referenced transport
header is the inner one.

If the VIRTIO_NET_F_GUEST_CSUM feature has been negotiated, the
device MAY set the VIRTIO_NET_HDR_F_DATA_VALID bit in
\field{flags}, if so, the device MUST validate the packet
checksum. If the VIRTIO_NET_F_GUEST_UDP_TUNNEL_GSO_CSUM feature has
been negotiated, and the VIRTIO_NET_HDR_F_UDP_TUNNEL_CSUM bit set in
\field{flags}, both the outer UDP checksum and the inner transport
checksum have been validated.
Otherwise level of checksum is validated: in case of multiple
encapsulated protocols the outermost one.

If either the VIRTIO_NET_HDR_GSO_UDP_TUNNEL_IPV4 bit or the
VIRTIO_NET_HDR_GSO_UDP_TUNNEL_IPV6 bit in \field{gso_type} are set,
the device MUST NOT set the VIRTIO_NET_HDR_F_DATA_VALID bit in
\field{flags}.

If the VIRTIO_NET_F_GUEST_UDP_TUNNEL_GSO_CSUM feature has been negotiated
and either the VIRTIO_NET_HDR_GSO_UDP_TUNNEL_IPV4 bit is set or the
VIRTIO_NET_HDR_GSO_UDP_TUNNEL_IPV6 bit is set in \field{gso_type}, the
device MAY set the VIRTIO_NET_HDR_F_UDP_TUNNEL_CSUM bit in
\field{flags}, if so the device MUST set the packet outer UDP checksum
stored in the receive buffer to the outer UDP pseudo header.

Otherwise, the VIRTIO_NET_F_GUEST_UDP_TUNNEL_GSO_CSUM feature has been
negotiated, either the VIRTIO_NET_HDR_GSO_UDP_TUNNEL_IPV4 bit is set or the
VIRTIO_NET_HDR_GSO_UDP_TUNNEL_IPV6 bit is set in \field{gso_type},
and the bit VIRTIO_NET_HDR_F_UDP_TUNNEL_CSUM is not set in
\field{flags}, the device MUST either provide a zero outer UDP header
checksum or a fully checksummed outer UDP header.

\drivernormative{\paragraph}{Processing of Incoming
Packets}{Device Types / Network Device / Device Operation /
Processing of Incoming Packets}

The driver MUST ignore \field{flag} bits that it does not recognize.

If VIRTIO_NET_HDR_F_NEEDS_CSUM bit in \field{flags} is not set or
if VIRTIO_NET_HDR_F_RSC_INFO bit \field{flags} is set, the
driver MUST NOT use the \field{csum_start} and \field{csum_offset}.

If one of the VIRTIO_NET_F_GUEST_TSO4, TSO6, UFO, USO4 or USO6 options have
been negotiated, the driver MAY use \field{hdr_len} only as a hint about the
transport header size.
The driver MUST NOT rely on \field{hdr_len} to be correct.
\begin{note}
This is due to various bugs in implementations.
\end{note}

If neither VIRTIO_NET_HDR_F_NEEDS_CSUM nor
VIRTIO_NET_HDR_F_DATA_VALID is set, the driver MUST NOT
rely on the packet checksum being correct.

If both the VIRTIO_NET_HDR_GSO_UDP_TUNNEL_IPV4 bit and
the VIRTIO_NET_HDR_GSO_UDP_TUNNEL_IPV6 bit in in \field{gso_type} are set,
the driver MUST NOT accept the packet.

If the VIRTIO_NET_HDR_GSO_UDP_TUNNEL_IPV4 bit or the VIRTIO_NET_HDR_GSO_UDP_TUNNEL_IPV6
bit in \field{gso_type} are not set, the driver MUST NOT use the
\field{outer_th_offset} and \field{inner_nh_offset}.

If either the VIRTIO_NET_HDR_GSO_UDP_TUNNEL_IPV4 bit or
the VIRTIO_NET_HDR_GSO_UDP_TUNNEL_IPV6 bit in \field{gso_type} are set, and any of
the following is true:
\begin{itemize}
\item the VIRTIO_NET_HDR_F_NEEDS_CSUM bit is not set in \field{flags}
\item the VIRTIO_NET_HDR_F_DATA_VALID bit is set in \field{flags}
\item the \field{gso_type} excluding the VIRTIO_NET_HDR_GSO_UDP_TUNNEL_IPV4
bit and the VIRTIO_NET_HDR_GSO_UDP_TUNNEL_IPV6 bit is VIRTIO_NET_HDR_GSO_NONE
\end{itemize}
the driver MUST NOT accept the packet.

If the VIRTIO_NET_HDR_F_UDP_TUNNEL_CSUM bit and the VIRTIO_NET_HDR_F_NEEDS_CSUM
bit in \field{flags} are set,
and both the bits VIRTIO_NET_HDR_GSO_UDP_TUNNEL_IPV4 and
VIRTIO_NET_HDR_GSO_UDP_TUNNEL_IPV6 in \field{gso_type} are not set,
the driver MOST NOT accept the packet.

\paragraph{Hash calculation for incoming packets}
\label{sec:Device Types / Network Device / Device Operation / Processing of Incoming Packets / Hash calculation for incoming packets}

A device attempts to calculate a per-packet hash in the following cases:
\begin{itemize}
\item The feature VIRTIO_NET_F_RSS was negotiated. The device uses the hash to determine the receive virtqueue to place incoming packets.
\item The feature VIRTIO_NET_F_HASH_REPORT was negotiated. The device reports the hash value and the hash type with the packet.
\end{itemize}

If the feature VIRTIO_NET_F_RSS was negotiated:
\begin{itemize}
\item The device uses \field{hash_types} of the virtio_net_rss_config structure as 'Enabled hash types' bitmask.
\item If additionally the feature VIRTIO_NET_F_HASH_TUNNEL was negotiated, the device uses \field{enabled_tunnel_types} of the
      virtnet_hash_tunnel structure as 'Encapsulation types enabled for inner header hash' bitmask.
\item The device uses a key as defined in \field{hash_key_data} and \field{hash_key_length} of the virtio_net_rss_config structure (see
\ref{sec:Device Types / Network Device / Device Operation / Control Virtqueue / Receive-side scaling (RSS) / Setting RSS parameters}).
\end{itemize}

If the feature VIRTIO_NET_F_RSS was not negotiated:
\begin{itemize}
\item The device uses \field{hash_types} of the virtio_net_hash_config structure as 'Enabled hash types' bitmask.
\item If additionally the feature VIRTIO_NET_F_HASH_TUNNEL was negotiated, the device uses \field{enabled_tunnel_types} of the
      virtnet_hash_tunnel structure as 'Encapsulation types enabled for inner header hash' bitmask.
\item The device uses a key as defined in \field{hash_key_data} and \field{hash_key_length} of the virtio_net_hash_config structure (see
\ref{sec:Device Types / Network Device / Device Operation / Control Virtqueue / Automatic receive steering in multiqueue mode / Hash calculation}).
\end{itemize}

Note that if the device offers VIRTIO_NET_F_HASH_REPORT, even if it supports only one pair of virtqueues, it MUST support
at least one of commands of VIRTIO_NET_CTRL_MQ class to configure reported hash parameters:
\begin{itemize}
\item If the device offers VIRTIO_NET_F_RSS, it MUST support VIRTIO_NET_CTRL_MQ_RSS_CONFIG command per
 \ref{sec:Device Types / Network Device / Device Operation / Control Virtqueue / Receive-side scaling (RSS) / Setting RSS parameters}.
\item Otherwise the device MUST support VIRTIO_NET_CTRL_MQ_HASH_CONFIG command per
 \ref{sec:Device Types / Network Device / Device Operation / Control Virtqueue / Automatic receive steering in multiqueue mode / Hash calculation}.
\end{itemize}

The per-packet hash calculation can depend on the IP packet type. See
\hyperref[intro:IP]{[IP]}, \hyperref[intro:UDP]{[UDP]} and \hyperref[intro:TCP]{[TCP]}.

\subparagraph{Supported/enabled hash types}
\label{sec:Device Types / Network Device / Device Operation / Processing of Incoming Packets / Hash calculation for incoming packets / Supported/enabled hash types}
Hash types applicable for IPv4 packets:
\begin{lstlisting}
#define VIRTIO_NET_HASH_TYPE_IPv4              (1 << 0)
#define VIRTIO_NET_HASH_TYPE_TCPv4             (1 << 1)
#define VIRTIO_NET_HASH_TYPE_UDPv4             (1 << 2)
\end{lstlisting}
Hash types applicable for IPv6 packets without extension headers
\begin{lstlisting}
#define VIRTIO_NET_HASH_TYPE_IPv6              (1 << 3)
#define VIRTIO_NET_HASH_TYPE_TCPv6             (1 << 4)
#define VIRTIO_NET_HASH_TYPE_UDPv6             (1 << 5)
\end{lstlisting}
Hash types applicable for IPv6 packets with extension headers
\begin{lstlisting}
#define VIRTIO_NET_HASH_TYPE_IP_EX             (1 << 6)
#define VIRTIO_NET_HASH_TYPE_TCP_EX            (1 << 7)
#define VIRTIO_NET_HASH_TYPE_UDP_EX            (1 << 8)
\end{lstlisting}

\subparagraph{IPv4 packets}
\label{sec:Device Types / Network Device / Device Operation / Processing of Incoming Packets / Hash calculation for incoming packets / IPv4 packets}
The device calculates the hash on IPv4 packets according to 'Enabled hash types' bitmask as follows:
\begin{itemize}
\item If VIRTIO_NET_HASH_TYPE_TCPv4 is set and the packet has
a TCP header, the hash is calculated over the following fields:
\begin{itemize}
\item Source IP address
\item Destination IP address
\item Source TCP port
\item Destination TCP port
\end{itemize}
\item Else if VIRTIO_NET_HASH_TYPE_UDPv4 is set and the
packet has a UDP header, the hash is calculated over the following fields:
\begin{itemize}
\item Source IP address
\item Destination IP address
\item Source UDP port
\item Destination UDP port
\end{itemize}
\item Else if VIRTIO_NET_HASH_TYPE_IPv4 is set, the hash is
calculated over the following fields:
\begin{itemize}
\item Source IP address
\item Destination IP address
\end{itemize}
\item Else the device does not calculate the hash
\end{itemize}

\subparagraph{IPv6 packets without extension header}
\label{sec:Device Types / Network Device / Device Operation / Processing of Incoming Packets / Hash calculation for incoming packets / IPv6 packets without extension header}
The device calculates the hash on IPv6 packets without extension
headers according to 'Enabled hash types' bitmask as follows:
\begin{itemize}
\item If VIRTIO_NET_HASH_TYPE_TCPv6 is set and the packet has
a TCPv6 header, the hash is calculated over the following fields:
\begin{itemize}
\item Source IPv6 address
\item Destination IPv6 address
\item Source TCP port
\item Destination TCP port
\end{itemize}
\item Else if VIRTIO_NET_HASH_TYPE_UDPv6 is set and the
packet has a UDPv6 header, the hash is calculated over the following fields:
\begin{itemize}
\item Source IPv6 address
\item Destination IPv6 address
\item Source UDP port
\item Destination UDP port
\end{itemize}
\item Else if VIRTIO_NET_HASH_TYPE_IPv6 is set, the hash is
calculated over the following fields:
\begin{itemize}
\item Source IPv6 address
\item Destination IPv6 address
\end{itemize}
\item Else the device does not calculate the hash
\end{itemize}

\subparagraph{IPv6 packets with extension header}
\label{sec:Device Types / Network Device / Device Operation / Processing of Incoming Packets / Hash calculation for incoming packets / IPv6 packets with extension header}
The device calculates the hash on IPv6 packets with extension
headers according to 'Enabled hash types' bitmask as follows:
\begin{itemize}
\item If VIRTIO_NET_HASH_TYPE_TCP_EX is set and the packet
has a TCPv6 header, the hash is calculated over the following fields:
\begin{itemize}
\item Home address from the home address option in the IPv6 destination options header. If the extension header is not present, use the Source IPv6 address.
\item IPv6 address that is contained in the Routing-Header-Type-2 from the associated extension header. If the extension header is not present, use the Destination IPv6 address.
\item Source TCP port
\item Destination TCP port
\end{itemize}
\item Else if VIRTIO_NET_HASH_TYPE_UDP_EX is set and the
packet has a UDPv6 header, the hash is calculated over the following fields:
\begin{itemize}
\item Home address from the home address option in the IPv6 destination options header. If the extension header is not present, use the Source IPv6 address.
\item IPv6 address that is contained in the Routing-Header-Type-2 from the associated extension header. If the extension header is not present, use the Destination IPv6 address.
\item Source UDP port
\item Destination UDP port
\end{itemize}
\item Else if VIRTIO_NET_HASH_TYPE_IP_EX is set, the hash is
calculated over the following fields:
\begin{itemize}
\item Home address from the home address option in the IPv6 destination options header. If the extension header is not present, use the Source IPv6 address.
\item IPv6 address that is contained in the Routing-Header-Type-2 from the associated extension header. If the extension header is not present, use the Destination IPv6 address.
\end{itemize}
\item Else skip IPv6 extension headers and calculate the hash as
defined for an IPv6 packet without extension headers
(see \ref{sec:Device Types / Network Device / Device Operation / Processing of Incoming Packets / Hash calculation for incoming packets / IPv6 packets without extension header}).
\end{itemize}

\paragraph{Inner Header Hash}
\label{sec:Device Types / Network Device / Device Operation / Processing of Incoming Packets / Inner Header Hash}

If VIRTIO_NET_F_HASH_TUNNEL has been negotiated, the driver can send the command
VIRTIO_NET_CTRL_HASH_TUNNEL_SET to configure the calculation of the inner header hash.

\begin{lstlisting}
struct virtnet_hash_tunnel {
    le32 enabled_tunnel_types;
};

#define VIRTIO_NET_CTRL_HASH_TUNNEL 7
 #define VIRTIO_NET_CTRL_HASH_TUNNEL_SET 0
\end{lstlisting}

Field \field{enabled_tunnel_types} contains the bitmask of encapsulation types enabled for inner header hash.
See \ref{sec:Device Types / Network Device / Device Operation / Processing of Incoming Packets /
Hash calculation for incoming packets / Encapsulation types supported/enabled for inner header hash}.

The class VIRTIO_NET_CTRL_HASH_TUNNEL has one command:
VIRTIO_NET_CTRL_HASH_TUNNEL_SET sets \field{enabled_tunnel_types} for the device using the
virtnet_hash_tunnel structure, which is read-only for the device.

Inner header hash is disabled by VIRTIO_NET_CTRL_HASH_TUNNEL_SET with \field{enabled_tunnel_types} set to 0.

Initially (before the driver sends any VIRTIO_NET_CTRL_HASH_TUNNEL_SET command) all
encapsulation types are disabled for inner header hash.

\subparagraph{Encapsulated packet}
\label{sec:Device Types / Network Device / Device Operation / Processing of Incoming Packets / Hash calculation for incoming packets / Encapsulated packet}

Multiple tunneling protocols allow encapsulating an inner, payload packet in an outer, encapsulated packet.
The encapsulated packet thus contains an outer header and an inner header, and the device calculates the
hash over either the inner header or the outer header.

If VIRTIO_NET_F_HASH_TUNNEL is negotiated and a received encapsulated packet's outer header matches one of the
encapsulation types enabled in \field{enabled_tunnel_types}, then the device uses the inner header for hash
calculations (only a single level of encapsulation is currently supported).

If VIRTIO_NET_F_HASH_TUNNEL is negotiated and a received packet's (outer) header does not match any encapsulation
types enabled in \field{enabled_tunnel_types}, then the device uses the outer header for hash calculations.

\subparagraph{Encapsulation types supported/enabled for inner header hash}
\label{sec:Device Types / Network Device / Device Operation / Processing of Incoming Packets /
Hash calculation for incoming packets / Encapsulation types supported/enabled for inner header hash}

Encapsulation types applicable for inner header hash:
\begin{lstlisting}[escapechar=|]
#define VIRTIO_NET_HASH_TUNNEL_TYPE_GRE_2784    (1 << 0) /* |\hyperref[intro:rfc2784]{[RFC2784]}| */
#define VIRTIO_NET_HASH_TUNNEL_TYPE_GRE_2890    (1 << 1) /* |\hyperref[intro:rfc2890]{[RFC2890]}| */
#define VIRTIO_NET_HASH_TUNNEL_TYPE_GRE_7676    (1 << 2) /* |\hyperref[intro:rfc7676]{[RFC7676]}| */
#define VIRTIO_NET_HASH_TUNNEL_TYPE_GRE_UDP     (1 << 3) /* |\hyperref[intro:rfc8086]{[GRE-in-UDP]}| */
#define VIRTIO_NET_HASH_TUNNEL_TYPE_VXLAN       (1 << 4) /* |\hyperref[intro:vxlan]{[VXLAN]}| */
#define VIRTIO_NET_HASH_TUNNEL_TYPE_VXLAN_GPE   (1 << 5) /* |\hyperref[intro:vxlan-gpe]{[VXLAN-GPE]}| */
#define VIRTIO_NET_HASH_TUNNEL_TYPE_GENEVE      (1 << 6) /* |\hyperref[intro:geneve]{[GENEVE]}| */
#define VIRTIO_NET_HASH_TUNNEL_TYPE_IPIP        (1 << 7) /* |\hyperref[intro:ipip]{[IPIP]}| */
#define VIRTIO_NET_HASH_TUNNEL_TYPE_NVGRE       (1 << 8) /* |\hyperref[intro:nvgre]{[NVGRE]}| */
\end{lstlisting}

\subparagraph{Advice}
Example uses of the inner header hash:
\begin{itemize}
\item Legacy tunneling protocols, lacking the outer header entropy, can use RSS with the inner header hash to
      distribute flows with identical outer but different inner headers across various queues, improving performance.
\item Identify an inner flow distributed across multiple outer tunnels.
\end{itemize}

As using the inner header hash completely discards the outer header entropy, care must be taken
if the inner header is controlled by an adversary, as the adversary can then intentionally create
configurations with insufficient entropy.

Besides disabling the inner header hash, mitigations would depend on how the hash is used. When the hash
use is limited to the RSS queue selection, the inner header hash may have quality of service (QoS) limitations.

\devicenormative{\subparagraph}{Inner Header Hash}{Device Types / Network Device / Device Operation / Control Virtqueue / Inner Header Hash}

If the (outer) header of the received packet does not match any encapsulation types enabled
in \field{enabled_tunnel_types}, the device MUST calculate the hash on the outer header.

If the device receives any bits in \field{enabled_tunnel_types} which are not set in \field{supported_tunnel_types},
it SHOULD respond to the VIRTIO_NET_CTRL_HASH_TUNNEL_SET command with VIRTIO_NET_ERR.

If the driver sets \field{enabled_tunnel_types} to 0 through VIRTIO_NET_CTRL_HASH_TUNNEL_SET or upon the device reset,
the device MUST disable the inner header hash for all encapsulation types.

\drivernormative{\subparagraph}{Inner Header Hash}{Device Types / Network Device / Device Operation / Control Virtqueue / Inner Header Hash}

The driver MUST have negotiated the VIRTIO_NET_F_HASH_TUNNEL feature when issuing the VIRTIO_NET_CTRL_HASH_TUNNEL_SET command.

The driver MUST NOT set any bits in \field{enabled_tunnel_types} which are not set in \field{supported_tunnel_types}.

The driver MUST ignore bits in \field{supported_tunnel_types} which are not documented in this specification.

\paragraph{Hash reporting for incoming packets}
\label{sec:Device Types / Network Device / Device Operation / Processing of Incoming Packets / Hash reporting for incoming packets}

If VIRTIO_NET_F_HASH_REPORT was negotiated and
 the device has calculated the hash for the packet, the device fills \field{hash_report} with the report type of calculated hash
and \field{hash_value} with the value of calculated hash.

If VIRTIO_NET_F_HASH_REPORT was negotiated but due to any reason the
hash was not calculated, the device sets \field{hash_report} to VIRTIO_NET_HASH_REPORT_NONE.

Possible values that the device can report in \field{hash_report} are defined below.
They correspond to supported hash types defined in
\ref{sec:Device Types / Network Device / Device Operation / Processing of Incoming Packets / Hash calculation for incoming packets / Supported/enabled hash types}
as follows:

VIRTIO_NET_HASH_TYPE_XXX = 1 << (VIRTIO_NET_HASH_REPORT_XXX - 1)

\begin{lstlisting}
#define VIRTIO_NET_HASH_REPORT_NONE            0
#define VIRTIO_NET_HASH_REPORT_IPv4            1
#define VIRTIO_NET_HASH_REPORT_TCPv4           2
#define VIRTIO_NET_HASH_REPORT_UDPv4           3
#define VIRTIO_NET_HASH_REPORT_IPv6            4
#define VIRTIO_NET_HASH_REPORT_TCPv6           5
#define VIRTIO_NET_HASH_REPORT_UDPv6           6
#define VIRTIO_NET_HASH_REPORT_IPv6_EX         7
#define VIRTIO_NET_HASH_REPORT_TCPv6_EX        8
#define VIRTIO_NET_HASH_REPORT_UDPv6_EX        9
\end{lstlisting}

\subsubsection{Control Virtqueue}\label{sec:Device Types / Network Device / Device Operation / Control Virtqueue}

The driver uses the control virtqueue (if VIRTIO_NET_F_CTRL_VQ is
negotiated) to send commands to manipulate various features of
the device which would not easily map into the configuration
space.

All commands are of the following form:

\begin{lstlisting}
struct virtio_net_ctrl {
        u8 class;
        u8 command;
        u8 command-specific-data[];
        u8 ack;
        u8 command-specific-result[];
};

/* ack values */
#define VIRTIO_NET_OK     0
#define VIRTIO_NET_ERR    1
\end{lstlisting}

The \field{class}, \field{command} and command-specific-data are set by the
driver, and the device sets the \field{ack} byte and optionally
\field{command-specific-result}. There is little the driver can
do except issue a diagnostic if \field{ack} is not VIRTIO_NET_OK.

The command VIRTIO_NET_CTRL_STATS_QUERY and VIRTIO_NET_CTRL_STATS_GET contain
\field{command-specific-result}.

\paragraph{Packet Receive Filtering}\label{sec:Device Types / Network Device / Device Operation / Control Virtqueue / Packet Receive Filtering}
\label{sec:Device Types / Network Device / Device Operation / Control Virtqueue / Setting Promiscuous Mode}%old label for latexdiff

If the VIRTIO_NET_F_CTRL_RX and VIRTIO_NET_F_CTRL_RX_EXTRA
features are negotiated, the driver can send control commands for
promiscuous mode, multicast, unicast and broadcast receiving.

\begin{note}
In general, these commands are best-effort: unwanted
packets could still arrive.
\end{note}

\begin{lstlisting}
#define VIRTIO_NET_CTRL_RX    0
 #define VIRTIO_NET_CTRL_RX_PROMISC      0
 #define VIRTIO_NET_CTRL_RX_ALLMULTI     1
 #define VIRTIO_NET_CTRL_RX_ALLUNI       2
 #define VIRTIO_NET_CTRL_RX_NOMULTI      3
 #define VIRTIO_NET_CTRL_RX_NOUNI        4
 #define VIRTIO_NET_CTRL_RX_NOBCAST      5
\end{lstlisting}


\devicenormative{\subparagraph}{Packet Receive Filtering}{Device Types / Network Device / Device Operation / Control Virtqueue / Packet Receive Filtering}

If the VIRTIO_NET_F_CTRL_RX feature has been negotiated,
the device MUST support the following VIRTIO_NET_CTRL_RX class
commands:
\begin{itemize}
\item VIRTIO_NET_CTRL_RX_PROMISC turns promiscuous mode on and
off. The command-specific-data is one byte containing 0 (off) or
1 (on). If promiscuous mode is on, the device SHOULD receive all
incoming packets.
This SHOULD take effect even if one of the other modes set by
a VIRTIO_NET_CTRL_RX class command is on.
\item VIRTIO_NET_CTRL_RX_ALLMULTI turns all-multicast receive on and
off. The command-specific-data is one byte containing 0 (off) or
1 (on). When all-multicast receive is on the device SHOULD allow
all incoming multicast packets.
\end{itemize}

If the VIRTIO_NET_F_CTRL_RX_EXTRA feature has been negotiated,
the device MUST support the following VIRTIO_NET_CTRL_RX class
commands:
\begin{itemize}
\item VIRTIO_NET_CTRL_RX_ALLUNI turns all-unicast receive on and
off. The command-specific-data is one byte containing 0 (off) or
1 (on). When all-unicast receive is on the device SHOULD allow
all incoming unicast packets.
\item VIRTIO_NET_CTRL_RX_NOMULTI suppresses multicast receive.
The command-specific-data is one byte containing 0 (multicast
receive allowed) or 1 (multicast receive suppressed).
When multicast receive is suppressed, the device SHOULD NOT
send multicast packets to the driver.
This SHOULD take effect even if VIRTIO_NET_CTRL_RX_ALLMULTI is on.
This filter SHOULD NOT apply to broadcast packets.
\item VIRTIO_NET_CTRL_RX_NOUNI suppresses unicast receive.
The command-specific-data is one byte containing 0 (unicast
receive allowed) or 1 (unicast receive suppressed).
When unicast receive is suppressed, the device SHOULD NOT
send unicast packets to the driver.
This SHOULD take effect even if VIRTIO_NET_CTRL_RX_ALLUNI is on.
\item VIRTIO_NET_CTRL_RX_NOBCAST suppresses broadcast receive.
The command-specific-data is one byte containing 0 (broadcast
receive allowed) or 1 (broadcast receive suppressed).
When broadcast receive is suppressed, the device SHOULD NOT
send broadcast packets to the driver.
This SHOULD take effect even if VIRTIO_NET_CTRL_RX_ALLMULTI is on.
\end{itemize}

\drivernormative{\subparagraph}{Packet Receive Filtering}{Device Types / Network Device / Device Operation / Control Virtqueue / Packet Receive Filtering}

If the VIRTIO_NET_F_CTRL_RX feature has not been negotiated,
the driver MUST NOT issue commands VIRTIO_NET_CTRL_RX_PROMISC or
VIRTIO_NET_CTRL_RX_ALLMULTI.

If the VIRTIO_NET_F_CTRL_RX_EXTRA feature has not been negotiated,
the driver MUST NOT issue commands
 VIRTIO_NET_CTRL_RX_ALLUNI,
 VIRTIO_NET_CTRL_RX_NOMULTI,
 VIRTIO_NET_CTRL_RX_NOUNI or
 VIRTIO_NET_CTRL_RX_NOBCAST.

\paragraph{Setting MAC Address Filtering}\label{sec:Device Types / Network Device / Device Operation / Control Virtqueue / Setting MAC Address Filtering}

If the VIRTIO_NET_F_CTRL_RX feature is negotiated, the driver can
send control commands for MAC address filtering.

\begin{lstlisting}
struct virtio_net_ctrl_mac {
        le32 entries;
        u8 macs[entries][6];
};

#define VIRTIO_NET_CTRL_MAC    1
 #define VIRTIO_NET_CTRL_MAC_TABLE_SET        0
 #define VIRTIO_NET_CTRL_MAC_ADDR_SET         1
\end{lstlisting}

The device can filter incoming packets by any number of destination
MAC addresses\footnote{Since there are no guarantees, it can use a hash filter or
silently switch to allmulti or promiscuous mode if it is given too
many addresses.
}. This table is set using the class
VIRTIO_NET_CTRL_MAC and the command VIRTIO_NET_CTRL_MAC_TABLE_SET. The
command-specific-data is two variable length tables of 6-byte MAC
addresses (as described in struct virtio_net_ctrl_mac). The first table contains unicast addresses, and the second
contains multicast addresses.

The VIRTIO_NET_CTRL_MAC_ADDR_SET command is used to set the
default MAC address which rx filtering
accepts (and if VIRTIO_NET_F_MAC has been negotiated,
this will be reflected in \field{mac} in config space).

The command-specific-data for VIRTIO_NET_CTRL_MAC_ADDR_SET is
the 6-byte MAC address.

\devicenormative{\subparagraph}{Setting MAC Address Filtering}{Device Types / Network Device / Device Operation / Control Virtqueue / Setting MAC Address Filtering}

The device MUST have an empty MAC filtering table on reset.

The device MUST update the MAC filtering table before it consumes
the VIRTIO_NET_CTRL_MAC_TABLE_SET command.

The device MUST update \field{mac} in config space before it consumes
the VIRTIO_NET_CTRL_MAC_ADDR_SET command, if VIRTIO_NET_F_MAC has
been negotiated.

The device SHOULD drop incoming packets which have a destination MAC which
matches neither the \field{mac} (or that set with VIRTIO_NET_CTRL_MAC_ADDR_SET)
nor the MAC filtering table.

\drivernormative{\subparagraph}{Setting MAC Address Filtering}{Device Types / Network Device / Device Operation / Control Virtqueue / Setting MAC Address Filtering}

If VIRTIO_NET_F_CTRL_RX has not been negotiated,
the driver MUST NOT issue VIRTIO_NET_CTRL_MAC class commands.

If VIRTIO_NET_F_CTRL_RX has been negotiated,
the driver SHOULD issue VIRTIO_NET_CTRL_MAC_ADDR_SET
to set the default mac if it is different from \field{mac}.

The driver MUST follow the VIRTIO_NET_CTRL_MAC_TABLE_SET command
by a le32 number, followed by that number of non-multicast
MAC addresses, followed by another le32 number, followed by
that number of multicast addresses.  Either number MAY be 0.

\subparagraph{Legacy Interface: Setting MAC Address Filtering}\label{sec:Device Types / Network Device / Device Operation / Control Virtqueue / Setting MAC Address Filtering / Legacy Interface: Setting MAC Address Filtering}
When using the legacy interface, transitional devices and drivers
MUST format \field{entries} in struct virtio_net_ctrl_mac
according to the native endian of the guest rather than
(necessarily when not using the legacy interface) little-endian.

Legacy drivers that didn't negotiate VIRTIO_NET_F_CTRL_MAC_ADDR
changed \field{mac} in config space when NIC is accepting
incoming packets. These drivers always wrote the mac value from
first to last byte, therefore after detecting such drivers,
a transitional device MAY defer MAC update, or MAY defer
processing incoming packets until driver writes the last byte
of \field{mac} in the config space.

\paragraph{VLAN Filtering}\label{sec:Device Types / Network Device / Device Operation / Control Virtqueue / VLAN Filtering}

If the driver negotiates the VIRTIO_NET_F_CTRL_VLAN feature, it
can control a VLAN filter table in the device. The VLAN filter
table applies only to VLAN tagged packets.

When VIRTIO_NET_F_CTRL_VLAN is negotiated, the device starts with
an empty VLAN filter table.

\begin{note}
Similar to the MAC address based filtering, the VLAN filtering
is also best-effort: unwanted packets could still arrive.
\end{note}

\begin{lstlisting}
#define VIRTIO_NET_CTRL_VLAN       2
 #define VIRTIO_NET_CTRL_VLAN_ADD             0
 #define VIRTIO_NET_CTRL_VLAN_DEL             1
\end{lstlisting}

Both the VIRTIO_NET_CTRL_VLAN_ADD and VIRTIO_NET_CTRL_VLAN_DEL
command take a little-endian 16-bit VLAN id as the command-specific-data.

VIRTIO_NET_CTRL_VLAN_ADD command adds the specified VLAN to the
VLAN filter table.

VIRTIO_NET_CTRL_VLAN_DEL command removes the specified VLAN from
the VLAN filter table.

\devicenormative{\subparagraph}{VLAN Filtering}{Device Types / Network Device / Device Operation / Control Virtqueue / VLAN Filtering}

When VIRTIO_NET_F_CTRL_VLAN is not negotiated, the device MUST
accept all VLAN tagged packets.

When VIRTIO_NET_F_CTRL_VLAN is negotiated, the device MUST
accept all VLAN tagged packets whose VLAN tag is present in
the VLAN filter table and SHOULD drop all VLAN tagged packets
whose VLAN tag is absent in the VLAN filter table.

\subparagraph{Legacy Interface: VLAN Filtering}\label{sec:Device Types / Network Device / Device Operation / Control Virtqueue / VLAN Filtering / Legacy Interface: VLAN Filtering}
When using the legacy interface, transitional devices and drivers
MUST format the VLAN id
according to the native endian of the guest rather than
(necessarily when not using the legacy interface) little-endian.

\paragraph{Gratuitous Packet Sending}\label{sec:Device Types / Network Device / Device Operation / Control Virtqueue / Gratuitous Packet Sending}

If the driver negotiates the VIRTIO_NET_F_GUEST_ANNOUNCE (depends
on VIRTIO_NET_F_CTRL_VQ), the device can ask the driver to send gratuitous
packets; this is usually done after the guest has been physically
migrated, and needs to announce its presence on the new network
links. (As hypervisor does not have the knowledge of guest
network configuration (eg. tagged vlan) it is simplest to prod
the guest in this way).

\begin{lstlisting}
#define VIRTIO_NET_CTRL_ANNOUNCE       3
 #define VIRTIO_NET_CTRL_ANNOUNCE_ACK             0
\end{lstlisting}

The driver checks VIRTIO_NET_S_ANNOUNCE bit in the device configuration \field{status} field
when it notices the changes of device configuration. The
command VIRTIO_NET_CTRL_ANNOUNCE_ACK is used to indicate that
driver has received the notification and device clears the
VIRTIO_NET_S_ANNOUNCE bit in \field{status}.

Processing this notification involves:

\begin{enumerate}
\item Sending the gratuitous packets (eg. ARP) or marking there are pending
  gratuitous packets to be sent and letting deferred routine to
  send them.

\item Sending VIRTIO_NET_CTRL_ANNOUNCE_ACK command through control
  vq.
\end{enumerate}

\drivernormative{\subparagraph}{Gratuitous Packet Sending}{Device Types / Network Device / Device Operation / Control Virtqueue / Gratuitous Packet Sending}

If the driver negotiates VIRTIO_NET_F_GUEST_ANNOUNCE, it SHOULD notify
network peers of its new location after it sees the VIRTIO_NET_S_ANNOUNCE bit
in \field{status}.  The driver MUST send a command on the command queue
with class VIRTIO_NET_CTRL_ANNOUNCE and command VIRTIO_NET_CTRL_ANNOUNCE_ACK.

\devicenormative{\subparagraph}{Gratuitous Packet Sending}{Device Types / Network Device / Device Operation / Control Virtqueue / Gratuitous Packet Sending}

If VIRTIO_NET_F_GUEST_ANNOUNCE is negotiated, the device MUST clear the
VIRTIO_NET_S_ANNOUNCE bit in \field{status} upon receipt of a command buffer
with class VIRTIO_NET_CTRL_ANNOUNCE and command VIRTIO_NET_CTRL_ANNOUNCE_ACK
before marking the buffer as used.

\paragraph{Device operation in multiqueue mode}\label{sec:Device Types / Network Device / Device Operation / Control Virtqueue / Device operation in multiqueue mode}

This specification defines the following modes that a device MAY implement for operation with multiple transmit/receive virtqueues:
\begin{itemize}
\item Automatic receive steering as defined in \ref{sec:Device Types / Network Device / Device Operation / Control Virtqueue / Automatic receive steering in multiqueue mode}.
 If a device supports this mode, it offers the VIRTIO_NET_F_MQ feature bit.
\item Receive-side scaling as defined in \ref{devicenormative:Device Types / Network Device / Device Operation / Control Virtqueue / Receive-side scaling (RSS) / RSS processing}.
 If a device supports this mode, it offers the VIRTIO_NET_F_RSS feature bit.
\end{itemize}

A device MAY support one of these features or both. The driver MAY negotiate any set of these features that the device supports.

Multiqueue is disabled by default.

The driver enables multiqueue by sending a command using \field{class} VIRTIO_NET_CTRL_MQ. The \field{command} selects the mode of multiqueue operation, as follows:
\begin{lstlisting}
#define VIRTIO_NET_CTRL_MQ    4
 #define VIRTIO_NET_CTRL_MQ_VQ_PAIRS_SET        0 (for automatic receive steering)
 #define VIRTIO_NET_CTRL_MQ_RSS_CONFIG          1 (for configurable receive steering)
 #define VIRTIO_NET_CTRL_MQ_HASH_CONFIG         2 (for configurable hash calculation)
\end{lstlisting}

If more than one multiqueue mode is negotiated, the resulting device configuration is defined by the last command sent by the driver.

\paragraph{Automatic receive steering in multiqueue mode}\label{sec:Device Types / Network Device / Device Operation / Control Virtqueue / Automatic receive steering in multiqueue mode}

If the driver negotiates the VIRTIO_NET_F_MQ feature bit (depends on VIRTIO_NET_F_CTRL_VQ), it MAY transmit outgoing packets on one
of the multiple transmitq1\ldots transmitqN and ask the device to
queue incoming packets into one of the multiple receiveq1\ldots receiveqN
depending on the packet flow.

The driver enables multiqueue by
sending the VIRTIO_NET_CTRL_MQ_VQ_PAIRS_SET command, specifying
the number of the transmit and receive queues to be used up to
\field{max_virtqueue_pairs}; subsequently,
transmitq1\ldots transmitqn and receiveq1\ldots receiveqn where
n=\field{virtqueue_pairs} MAY be used.
\begin{lstlisting}
struct virtio_net_ctrl_mq_pairs_set {
       le16 virtqueue_pairs;
};
#define VIRTIO_NET_CTRL_MQ_VQ_PAIRS_MIN        1
#define VIRTIO_NET_CTRL_MQ_VQ_PAIRS_MAX        0x8000

\end{lstlisting}

When multiqueue is enabled by VIRTIO_NET_CTRL_MQ_VQ_PAIRS_SET command, the device MUST use automatic receive steering
based on packet flow. Programming of the receive steering
classificator is implicit. After the driver transmitted a packet of a
flow on transmitqX, the device SHOULD cause incoming packets for that flow to
be steered to receiveqX. For uni-directional protocols, or where
no packets have been transmitted yet, the device MAY steer a packet
to a random queue out of the specified receiveq1\ldots receiveqn.

Multiqueue is disabled by VIRTIO_NET_CTRL_MQ_VQ_PAIRS_SET with \field{virtqueue_pairs} to 1 (this is
the default) and waiting for the device to use the command buffer.

\drivernormative{\subparagraph}{Automatic receive steering in multiqueue mode}{Device Types / Network Device / Device Operation / Control Virtqueue / Automatic receive steering in multiqueue mode}

The driver MUST configure the virtqueues before enabling them with the
VIRTIO_NET_CTRL_MQ_VQ_PAIRS_SET command.

The driver MUST NOT request a \field{virtqueue_pairs} of 0 or
greater than \field{max_virtqueue_pairs} in the device configuration space.

The driver MUST queue packets only on any transmitq1 before the
VIRTIO_NET_CTRL_MQ_VQ_PAIRS_SET command.

The driver MUST NOT queue packets on transmit queues greater than
\field{virtqueue_pairs} once it has placed the VIRTIO_NET_CTRL_MQ_VQ_PAIRS_SET command in the available ring.

\devicenormative{\subparagraph}{Automatic receive steering in multiqueue mode}{Device Types / Network Device / Device Operation / Control Virtqueue / Automatic receive steering in multiqueue mode}

After initialization of reset, the device MUST queue packets only on receiveq1.

The device MUST NOT queue packets on receive queues greater than
\field{virtqueue_pairs} once it has placed the
VIRTIO_NET_CTRL_MQ_VQ_PAIRS_SET command in a used buffer.

If the destination receive queue is being reset (See \ref{sec:Basic Facilities of a Virtio Device / Virtqueues / Virtqueue Reset}),
the device SHOULD re-select another random queue. If all receive queues are
being reset, the device MUST drop the packet.

\subparagraph{Legacy Interface: Automatic receive steering in multiqueue mode}\label{sec:Device Types / Network Device / Device Operation / Control Virtqueue / Automatic receive steering in multiqueue mode / Legacy Interface: Automatic receive steering in multiqueue mode}
When using the legacy interface, transitional devices and drivers
MUST format \field{virtqueue_pairs}
according to the native endian of the guest rather than
(necessarily when not using the legacy interface) little-endian.

\subparagraph{Hash calculation}\label{sec:Device Types / Network Device / Device Operation / Control Virtqueue / Automatic receive steering in multiqueue mode / Hash calculation}
If VIRTIO_NET_F_HASH_REPORT was negotiated and the device uses automatic receive steering,
the device MUST support a command to configure hash calculation parameters.

The driver provides parameters for hash calculation as follows:

\field{class} VIRTIO_NET_CTRL_MQ, \field{command} VIRTIO_NET_CTRL_MQ_HASH_CONFIG.

The \field{command-specific-data} has following format:
\begin{lstlisting}
struct virtio_net_hash_config {
    le32 hash_types;
    le16 reserved[4];
    u8 hash_key_length;
    u8 hash_key_data[hash_key_length];
};
\end{lstlisting}
Field \field{hash_types} contains a bitmask of allowed hash types as
defined in
\ref{sec:Device Types / Network Device / Device Operation / Processing of Incoming Packets / Hash calculation for incoming packets / Supported/enabled hash types}.
Initially the device has all hash types disabled and reports only VIRTIO_NET_HASH_REPORT_NONE.

Field \field{reserved} MUST contain zeroes. It is defined to make the structure to match the layout of virtio_net_rss_config structure,
defined in \ref{sec:Device Types / Network Device / Device Operation / Control Virtqueue / Receive-side scaling (RSS)}.

Fields \field{hash_key_length} and \field{hash_key_data} define the key to be used in hash calculation.

\paragraph{Receive-side scaling (RSS)}\label{sec:Device Types / Network Device / Device Operation / Control Virtqueue / Receive-side scaling (RSS)}
A device offers the feature VIRTIO_NET_F_RSS if it supports RSS receive steering with Toeplitz hash calculation and configurable parameters.

A driver queries RSS capabilities of the device by reading device configuration as defined in \ref{sec:Device Types / Network Device / Device configuration layout}

\subparagraph{Setting RSS parameters}\label{sec:Device Types / Network Device / Device Operation / Control Virtqueue / Receive-side scaling (RSS) / Setting RSS parameters}

Driver sends a VIRTIO_NET_CTRL_MQ_RSS_CONFIG command using the following format for \field{command-specific-data}:
\begin{lstlisting}
struct rss_rq_id {
   le16 vq_index_1_16: 15; /* Bits 1 to 16 of the virtqueue index */
   le16 reserved: 1; /* Set to zero */
};

struct virtio_net_rss_config {
    le32 hash_types;
    le16 indirection_table_mask;
    struct rss_rq_id unclassified_queue;
    struct rss_rq_id indirection_table[indirection_table_length];
    le16 max_tx_vq;
    u8 hash_key_length;
    u8 hash_key_data[hash_key_length];
};
\end{lstlisting}
Field \field{hash_types} contains a bitmask of allowed hash types as
defined in
\ref{sec:Device Types / Network Device / Device Operation / Processing of Incoming Packets / Hash calculation for incoming packets / Supported/enabled hash types}.

Field \field{indirection_table_mask} is a mask to be applied to
the calculated hash to produce an index in the
\field{indirection_table} array.
Number of entries in \field{indirection_table} is (\field{indirection_table_mask} + 1).

\field{rss_rq_id} is a receive virtqueue id. \field{vq_index_1_16}
consists of bits 1 to 16 of a virtqueue index. For example, a
\field{vq_index_1_16} value of 3 corresponds to virtqueue index 6,
which maps to receiveq4.

Field \field{unclassified_queue} specifies the receive virtqueue id in which to
place unclassified packets.

Field \field{indirection_table} is an array of receive virtqueues ids.

A driver sets \field{max_tx_vq} to inform a device how many transmit virtqueues it may use (transmitq1\ldots transmitq \field{max_tx_vq}).

Fields \field{hash_key_length} and \field{hash_key_data} define the key to be used in hash calculation.

\drivernormative{\subparagraph}{Setting RSS parameters}{Device Types / Network Device / Device Operation / Control Virtqueue / Receive-side scaling (RSS) }

A driver MUST NOT send the VIRTIO_NET_CTRL_MQ_RSS_CONFIG command if the feature VIRTIO_NET_F_RSS has not been negotiated.

A driver MUST fill the \field{indirection_table} array only with
enabled receive virtqueues ids.

The number of entries in \field{indirection_table} (\field{indirection_table_mask} + 1) MUST be a power of two.

A driver MUST use \field{indirection_table_mask} values that are less than \field{rss_max_indirection_table_length} reported by a device.

A driver MUST NOT set any VIRTIO_NET_HASH_TYPE_ flags that are not supported by a device.

\devicenormative{\subparagraph}{RSS processing}{Device Types / Network Device / Device Operation / Control Virtqueue / Receive-side scaling (RSS) / RSS processing}
The device MUST determine the destination queue for a network packet as follows:
\begin{itemize}
\item Calculate the hash of the packet as defined in \ref{sec:Device Types / Network Device / Device Operation / Processing of Incoming Packets / Hash calculation for incoming packets}.
\item If the device did not calculate the hash for the specific packet, the device directs the packet to the receiveq specified by \field{unclassified_queue} of virtio_net_rss_config structure.
\item Apply \field{indirection_table_mask} to the calculated hash
and use the result as the index in the indirection table to get
the destination receive virtqueue id.
\item If the destination receive queue is being reset (See \ref{sec:Basic Facilities of a Virtio Device / Virtqueues / Virtqueue Reset}), the device MUST drop the packet.
\end{itemize}

\paragraph{RSS Context}\label{sec:Device Types / Network Device / Device Operation / Control Virtqueue / RSS Context}

An RSS context consists of configurable parameters specified by \ref{sec:Device Types / Network Device
/ Device Operation / Control Virtqueue / Receive-side scaling (RSS)}.

The RSS configuration supported by VIRTIO_NET_F_RSS is considered the default RSS configuration.

The device offers the feature VIRTIO_NET_F_RSS_CONTEXT if it supports one or multiple RSS contexts
(excluding the default RSS configuration) and configurable parameters.

\subparagraph{Querying RSS Context Capability}\label{sec:Device Types / Network Device / Device Operation / Control Virtqueue / RSS Context / Querying RSS Context Capability}

\begin{lstlisting}
#define VIRTNET_RSS_CTX_CTRL 9
 #define VIRTNET_RSS_CTX_CTRL_CAP_GET  0
 #define VIRTNET_RSS_CTX_CTRL_ADD      1
 #define VIRTNET_RSS_CTX_CTRL_MOD      2
 #define VIRTNET_RSS_CTX_CTRL_DEL      3

struct virtnet_rss_ctx_cap {
    le16 max_rss_contexts;
}
\end{lstlisting}

Field \field{max_rss_contexts} specifies the maximum number of RSS contexts \ref{sec:Device Types / Network Device /
Device Operation / Control Virtqueue / RSS Context} supported by the device.

The driver queries the RSS context capability of the device by sending the command VIRTNET_RSS_CTX_CTRL_CAP_GET
with the structure virtnet_rss_ctx_cap.

For the command VIRTNET_RSS_CTX_CTRL_CAP_GET, the structure virtnet_rss_ctx_cap is write-only for the device.

\subparagraph{Setting RSS Context Parameters}\label{sec:Device Types / Network Device / Device Operation / Control Virtqueue / RSS Context / Setting RSS Context Parameters}

\begin{lstlisting}
struct virtnet_rss_ctx_add_modify {
    le16 rss_ctx_id;
    u8 reserved[6];
    struct virtio_net_rss_config rss;
};

struct virtnet_rss_ctx_del {
    le16 rss_ctx_id;
};
\end{lstlisting}

RSS context parameters:
\begin{itemize}
\item  \field{rss_ctx_id}: ID of the specific RSS context.
\item  \field{rss}: RSS context parameters of the specific RSS context whose id is \field{rss_ctx_id}.
\end{itemize}

\field{reserved} is reserved and it is ignored by the device.

If the feature VIRTIO_NET_F_RSS_CONTEXT has been negotiated, the driver can send the following
VIRTNET_RSS_CTX_CTRL class commands:
\begin{enumerate}
\item VIRTNET_RSS_CTX_CTRL_ADD: use the structure virtnet_rss_ctx_add_modify to
       add an RSS context configured as \field{rss} and id as \field{rss_ctx_id} for the device.
\item VIRTNET_RSS_CTX_CTRL_MOD: use the structure virtnet_rss_ctx_add_modify to
       configure parameters of the RSS context whose id is \field{rss_ctx_id} as \field{rss} for the device.
\item VIRTNET_RSS_CTX_CTRL_DEL: use the structure virtnet_rss_ctx_del to delete
       the RSS context whose id is \field{rss_ctx_id} for the device.
\end{enumerate}

For commands VIRTNET_RSS_CTX_CTRL_ADD and VIRTNET_RSS_CTX_CTRL_MOD, the structure virtnet_rss_ctx_add_modify is read-only for the device.
For the command VIRTNET_RSS_CTX_CTRL_DEL, the structure virtnet_rss_ctx_del is read-only for the device.

\devicenormative{\subparagraph}{RSS Context}{Device Types / Network Device / Device Operation / Control Virtqueue / RSS Context}

The device MUST set \field{max_rss_contexts} to at least 1 if it offers VIRTIO_NET_F_RSS_CONTEXT.

Upon reset, the device MUST clear all previously configured RSS contexts.

\drivernormative{\subparagraph}{RSS Context}{Device Types / Network Device / Device Operation / Control Virtqueue / RSS Context}

The driver MUST have negotiated the VIRTIO_NET_F_RSS_CONTEXT feature when issuing the VIRTNET_RSS_CTX_CTRL class commands.

The driver MUST set \field{rss_ctx_id} to between 1 and \field{max_rss_contexts} inclusive.

The driver MUST NOT send the command VIRTIO_NET_CTRL_MQ_VQ_PAIRS_SET when the device has successfully configured at least one RSS context.

\paragraph{Offloads State Configuration}\label{sec:Device Types / Network Device / Device Operation / Control Virtqueue / Offloads State Configuration}

If the VIRTIO_NET_F_CTRL_GUEST_OFFLOADS feature is negotiated, the driver can
send control commands for dynamic offloads state configuration.

\subparagraph{Setting Offloads State}\label{sec:Device Types / Network Device / Device Operation / Control Virtqueue / Offloads State Configuration / Setting Offloads State}

To configure the offloads, the following layout structure and
definitions are used:

\begin{lstlisting}
le64 offloads;

#define VIRTIO_NET_F_GUEST_CSUM       1
#define VIRTIO_NET_F_GUEST_TSO4       7
#define VIRTIO_NET_F_GUEST_TSO6       8
#define VIRTIO_NET_F_GUEST_ECN        9
#define VIRTIO_NET_F_GUEST_UFO        10
#define VIRTIO_NET_F_GUEST_UDP_TUNNEL_GSO  46
#define VIRTIO_NET_F_GUEST_UDP_TUNNEL_GSO_CSUM 47
#define VIRTIO_NET_F_GUEST_USO4       54
#define VIRTIO_NET_F_GUEST_USO6       55

#define VIRTIO_NET_CTRL_GUEST_OFFLOADS       5
 #define VIRTIO_NET_CTRL_GUEST_OFFLOADS_SET   0
\end{lstlisting}

The class VIRTIO_NET_CTRL_GUEST_OFFLOADS has one command:
VIRTIO_NET_CTRL_GUEST_OFFLOADS_SET applies the new offloads configuration.

le64 value passed as command data is a bitmask, bits set define
offloads to be enabled, bits cleared - offloads to be disabled.

There is a corresponding device feature for each offload. Upon feature
negotiation corresponding offload gets enabled to preserve backward
compatibility.

\drivernormative{\subparagraph}{Setting Offloads State}{Device Types / Network Device / Device Operation / Control Virtqueue / Offloads State Configuration / Setting Offloads State}

A driver MUST NOT enable an offload for which the appropriate feature
has not been negotiated.

\subparagraph{Legacy Interface: Setting Offloads State}\label{sec:Device Types / Network Device / Device Operation / Control Virtqueue / Offloads State Configuration / Setting Offloads State / Legacy Interface: Setting Offloads State}
When using the legacy interface, transitional devices and drivers
MUST format \field{offloads}
according to the native endian of the guest rather than
(necessarily when not using the legacy interface) little-endian.


\paragraph{Notifications Coalescing}\label{sec:Device Types / Network Device / Device Operation / Control Virtqueue / Notifications Coalescing}

If the VIRTIO_NET_F_NOTF_COAL feature is negotiated, the driver can
send commands VIRTIO_NET_CTRL_NOTF_COAL_TX_SET and VIRTIO_NET_CTRL_NOTF_COAL_RX_SET
for notification coalescing.

If the VIRTIO_NET_F_VQ_NOTF_COAL feature is negotiated, the driver can
send commands VIRTIO_NET_CTRL_NOTF_COAL_VQ_SET and VIRTIO_NET_CTRL_NOTF_COAL_VQ_GET
for virtqueue notification coalescing.

\begin{lstlisting}
struct virtio_net_ctrl_coal {
    le32 max_packets;
    le32 max_usecs;
};

struct virtio_net_ctrl_coal_vq {
    le16 vq_index;
    le16 reserved;
    struct virtio_net_ctrl_coal coal;
};

#define VIRTIO_NET_CTRL_NOTF_COAL 6
 #define VIRTIO_NET_CTRL_NOTF_COAL_TX_SET  0
 #define VIRTIO_NET_CTRL_NOTF_COAL_RX_SET 1
 #define VIRTIO_NET_CTRL_NOTF_COAL_VQ_SET 2
 #define VIRTIO_NET_CTRL_NOTF_COAL_VQ_GET 3
\end{lstlisting}

Coalescing parameters:
\begin{itemize}
\item \field{vq_index}: The virtqueue index of an enabled transmit or receive virtqueue.
\item \field{max_usecs} for RX: Maximum number of microseconds to delay a RX notification.
\item \field{max_usecs} for TX: Maximum number of microseconds to delay a TX notification.
\item \field{max_packets} for RX: Maximum number of packets to receive before a RX notification.
\item \field{max_packets} for TX: Maximum number of packets to send before a TX notification.
\end{itemize}

\field{reserved} is reserved and it is ignored by the device.

Read/Write attributes for coalescing parameters:
\begin{itemize}
\item For commands VIRTIO_NET_CTRL_NOTF_COAL_TX_SET and VIRTIO_NET_CTRL_NOTF_COAL_RX_SET, the structure virtio_net_ctrl_coal is write-only for the driver.
\item For the command VIRTIO_NET_CTRL_NOTF_COAL_VQ_SET, the structure virtio_net_ctrl_coal_vq is write-only for the driver.
\item For the command VIRTIO_NET_CTRL_NOTF_COAL_VQ_GET, \field{vq_index} and \field{reserved} are write-only
      for the driver, and the structure virtio_net_ctrl_coal is read-only for the driver.
\end{itemize}

The class VIRTIO_NET_CTRL_NOTF_COAL has the following commands:
\begin{enumerate}
\item VIRTIO_NET_CTRL_NOTF_COAL_TX_SET: use the structure virtio_net_ctrl_coal to set the \field{max_usecs} and \field{max_packets} parameters for all transmit virtqueues.
\item VIRTIO_NET_CTRL_NOTF_COAL_RX_SET: use the structure virtio_net_ctrl_coal to set the \field{max_usecs} and \field{max_packets} parameters for all receive virtqueues.
\item VIRTIO_NET_CTRL_NOTF_COAL_VQ_SET: use the structure virtio_net_ctrl_coal_vq to set the \field{max_usecs} and \field{max_packets} parameters
                                        for an enabled transmit/receive virtqueue whose index is \field{vq_index}.
\item VIRTIO_NET_CTRL_NOTF_COAL_VQ_GET: use the structure virtio_net_ctrl_coal_vq to get the \field{max_usecs} and \field{max_packets} parameters
                                        for an enabled transmit/receive virtqueue whose index is \field{vq_index}.
\end{enumerate}

The device may generate notifications more or less frequently than specified by set commands of the VIRTIO_NET_CTRL_NOTF_COAL class.

If coalescing parameters are being set, the device applies the last coalescing parameters set for a
virtqueue, regardless of the command used to set the parameters. Use the following command sequence
with two pairs of virtqueues as an example:
Each of the following commands sets \field{max_usecs} and \field{max_packets} parameters for virtqueues.
\begin{itemize}
\item Command1: VIRTIO_NET_CTRL_NOTF_COAL_RX_SET sets coalescing parameters for virtqueues having index 0 and index 2. Virtqueues having index 1 and index 3 retain their previous parameters.
\item Command2: VIRTIO_NET_CTRL_NOTF_COAL_VQ_SET with \field{vq_index} = 0 sets coalescing parameters for virtqueue having index 0. Virtqueue having index 2 retains the parameters from command1.
\item Command3: VIRTIO_NET_CTRL_NOTF_COAL_VQ_GET with \field{vq_index} = 0, the device responds with coalescing parameters of vq_index 0 set by command2.
\item Command4: VIRTIO_NET_CTRL_NOTF_COAL_VQ_SET with \field{vq_index} = 1 sets coalescing parameters for virtqueue having index 1. Virtqueue having index 3 retains its previous parameters.
\item Command5: VIRTIO_NET_CTRL_NOTF_COAL_TX_SET sets coalescing parameters for virtqueues having index 1 and index 3, and overrides the parameters set by command4.
\item Command6: VIRTIO_NET_CTRL_NOTF_COAL_VQ_GET with \field{vq_index} = 1, the device responds with coalescing parameters of index 1 set by command5.
\end{itemize}

\subparagraph{Operation}\label{sec:Device Types / Network Device / Device Operation / Control Virtqueue / Notifications Coalescing / Operation}

The device sends a used buffer notification once the notification conditions are met and if the notifications are not suppressed as explained in \ref{sec:Basic Facilities of a Virtio Device / Virtqueues / Used Buffer Notification Suppression}.

When the device has non-zero \field{max_usecs} and non-zero \field{max_packets}, it starts counting microseconds and packets upon receiving/sending a packet.
The device counts packets and microseconds for each receive virtqueue and transmit virtqueue separately.
In this case, the notification conditions are met when \field{max_usecs} microseconds elapse, or upon sending/receiving \field{max_packets} packets, whichever happens first.
Afterwards, the device waits for the next packet and starts counting packets and microseconds again.

When the device has \field{max_usecs} = 0 or \field{max_packets} = 0, the notification conditions are met after every packet received/sent.

\subparagraph{RX Example}\label{sec:Device Types / Network Device / Device Operation / Control Virtqueue / Notifications Coalescing / RX Example}

If, for example:
\begin{itemize}
\item \field{max_usecs} = 10.
\item \field{max_packets} = 15.
\end{itemize}
then each receive virtqueue of a device will operate as follows:
\begin{itemize}
\item The device will count packets received on each virtqueue until it accumulates 15, or until 10 microseconds elapsed since the first one was received.
\item If the notifications are not suppressed by the driver, the device will send an used buffer notification, otherwise, the device will not send an used buffer notification as long as the notifications are suppressed.
\end{itemize}

\subparagraph{TX Example}\label{sec:Device Types / Network Device / Device Operation / Control Virtqueue / Notifications Coalescing / TX Example}

If, for example:
\begin{itemize}
\item \field{max_usecs} = 10.
\item \field{max_packets} = 15.
\end{itemize}
then each transmit virtqueue of a device will operate as follows:
\begin{itemize}
\item The device will count packets sent on each virtqueue until it accumulates 15, or until 10 microseconds elapsed since the first one was sent.
\item If the notifications are not suppressed by the driver, the device will send an used buffer notification, otherwise, the device will not send an used buffer notification as long as the notifications are suppressed.
\end{itemize}

\subparagraph{Notifications When Coalescing Parameters Change}\label{sec:Device Types / Network Device / Device Operation / Control Virtqueue / Notifications Coalescing / Notifications When Coalescing Parameters Change}

When the coalescing parameters of a device change, the device needs to check if the new notification conditions are met and send a used buffer notification if so.

For example, \field{max_packets} = 15 for a device with a single transmit virtqueue: if the device sends 10 packets and afterwards receives a
VIRTIO_NET_CTRL_NOTF_COAL_TX_SET command with \field{max_packets} = 8, then the notification condition is immediately considered to be met;
the device needs to immediately send a used buffer notification, if the notifications are not suppressed by the driver.

\drivernormative{\subparagraph}{Notifications Coalescing}{Device Types / Network Device / Device Operation / Control Virtqueue / Notifications Coalescing}

The driver MUST set \field{vq_index} to the virtqueue index of an enabled transmit or receive virtqueue.

The driver MUST have negotiated the VIRTIO_NET_F_NOTF_COAL feature when issuing commands VIRTIO_NET_CTRL_NOTF_COAL_TX_SET and VIRTIO_NET_CTRL_NOTF_COAL_RX_SET.

The driver MUST have negotiated the VIRTIO_NET_F_VQ_NOTF_COAL feature when issuing commands VIRTIO_NET_CTRL_NOTF_COAL_VQ_SET and VIRTIO_NET_CTRL_NOTF_COAL_VQ_GET.

The driver MUST ignore the values of coalescing parameters received from the VIRTIO_NET_CTRL_NOTF_COAL_VQ_GET command if the device responds with VIRTIO_NET_ERR.

\devicenormative{\subparagraph}{Notifications Coalescing}{Device Types / Network Device / Device Operation / Control Virtqueue / Notifications Coalescing}

The device MUST ignore \field{reserved}.

The device SHOULD respond to VIRTIO_NET_CTRL_NOTF_COAL_TX_SET and VIRTIO_NET_CTRL_NOTF_COAL_RX_SET commands with VIRTIO_NET_ERR if it was not able to change the parameters.

The device MUST respond to the VIRTIO_NET_CTRL_NOTF_COAL_VQ_SET command with VIRTIO_NET_ERR if it was not able to change the parameters.

The device MUST respond to VIRTIO_NET_CTRL_NOTF_COAL_VQ_SET and VIRTIO_NET_CTRL_NOTF_COAL_VQ_GET commands with
VIRTIO_NET_ERR if the designated virtqueue is not an enabled transmit or receive virtqueue.

Upon disabling and re-enabling a transmit virtqueue, the device MUST set the coalescing parameters of the virtqueue
to those configured through the VIRTIO_NET_CTRL_NOTF_COAL_TX_SET command, or, if the driver did not set any TX coalescing parameters, to 0.

Upon disabling and re-enabling a receive virtqueue, the device MUST set the coalescing parameters of the virtqueue
to those configured through the VIRTIO_NET_CTRL_NOTF_COAL_RX_SET command, or, if the driver did not set any RX coalescing parameters, to 0.

The behavior of the device in response to set commands of the VIRTIO_NET_CTRL_NOTF_COAL class is best-effort:
the device MAY generate notifications more or less frequently than specified.

A device SHOULD NOT send used buffer notifications to the driver if the notifications are suppressed, even if the notification conditions are met.

Upon reset, a device MUST initialize all coalescing parameters to 0.

\paragraph{Device Statistics}\label{sec:Device Types / Network Device / Device Operation / Control Virtqueue / Device Statistics}

If the VIRTIO_NET_F_DEVICE_STATS feature is negotiated, the driver can obtain
device statistics from the device by using the following command.

Different types of virtqueues have different statistics. The statistics of the
receiveq are different from those of the transmitq.

The statistics of a certain type of virtqueue are also divided into multiple types
because different types require different features. This enables the expansion
of new statistics.

In one command, the driver can obtain the statistics of one or multiple virtqueues.
Additionally, the driver can obtain multiple type statistics of each virtqueue.

\subparagraph{Query Statistic Capabilities}\label{sec:Device Types / Network Device / Device Operation / Control Virtqueue / Device Statistics / Query Statistic Capabilities}

\begin{lstlisting}
#define VIRTIO_NET_CTRL_STATS         8
#define VIRTIO_NET_CTRL_STATS_QUERY   0
#define VIRTIO_NET_CTRL_STATS_GET     1

struct virtio_net_stats_capabilities {

#define VIRTIO_NET_STATS_TYPE_CVQ       (1 << 32)

#define VIRTIO_NET_STATS_TYPE_RX_BASIC  (1 << 0)
#define VIRTIO_NET_STATS_TYPE_RX_CSUM   (1 << 1)
#define VIRTIO_NET_STATS_TYPE_RX_GSO    (1 << 2)
#define VIRTIO_NET_STATS_TYPE_RX_SPEED  (1 << 3)

#define VIRTIO_NET_STATS_TYPE_TX_BASIC  (1 << 16)
#define VIRTIO_NET_STATS_TYPE_TX_CSUM   (1 << 17)
#define VIRTIO_NET_STATS_TYPE_TX_GSO    (1 << 18)
#define VIRTIO_NET_STATS_TYPE_TX_SPEED  (1 << 19)

    le64 supported_stats_types[1];
}
\end{lstlisting}

To obtain device statistic capability, use the VIRTIO_NET_CTRL_STATS_QUERY
command. When the command completes successfully, \field{command-specific-result}
is in the format of \field{struct virtio_net_stats_capabilities}.

\subparagraph{Get Statistics}\label{sec:Device Types / Network Device / Device Operation / Control Virtqueue / Device Statistics / Get Statistics}

\begin{lstlisting}
struct virtio_net_ctrl_queue_stats {
       struct {
           le16 vq_index;
           le16 reserved[3];
           le64 types_bitmap[1];
       } stats[];
};

struct virtio_net_stats_reply_hdr {
#define VIRTIO_NET_STATS_TYPE_REPLY_CVQ       32

#define VIRTIO_NET_STATS_TYPE_REPLY_RX_BASIC  0
#define VIRTIO_NET_STATS_TYPE_REPLY_RX_CSUM   1
#define VIRTIO_NET_STATS_TYPE_REPLY_RX_GSO    2
#define VIRTIO_NET_STATS_TYPE_REPLY_RX_SPEED  3

#define VIRTIO_NET_STATS_TYPE_REPLY_TX_BASIC  16
#define VIRTIO_NET_STATS_TYPE_REPLY_TX_CSUM   17
#define VIRTIO_NET_STATS_TYPE_REPLY_TX_GSO    18
#define VIRTIO_NET_STATS_TYPE_REPLY_TX_SPEED  19
    u8 type;
    u8 reserved;
    le16 vq_index;
    le16 reserved1;
    le16 size;
}
\end{lstlisting}

To obtain device statistics, use the VIRTIO_NET_CTRL_STATS_GET command with the
\field{command-specific-data} which is in the format of
\field{struct virtio_net_ctrl_queue_stats}. When the command completes
successfully, \field{command-specific-result} contains multiple statistic
results, each statistic result has the \field{struct virtio_net_stats_reply_hdr}
as the header.

The fields of the \field{struct virtio_net_ctrl_queue_stats}:
\begin{description}
    \item [vq_index]
        The index of the virtqueue to obtain the statistics.

    \item [types_bitmap]
        This is a bitmask of the types of statistics to be obtained. Therefore, a
        \field{stats} inside \field{struct virtio_net_ctrl_queue_stats} may
        indicate multiple statistic replies for the virtqueue.
\end{description}

The fields of the \field{struct virtio_net_stats_reply_hdr}:
\begin{description}
    \item [type]
        The type of the reply statistic.

    \item [vq_index]
        The virtqueue index of the reply statistic.

    \item [size]
        The number of bytes for the statistics entry including size of \field{struct virtio_net_stats_reply_hdr}.

\end{description}

\subparagraph{Controlq Statistics}\label{sec:Device Types / Network Device / Device Operation / Control Virtqueue / Device Statistics / Controlq Statistics}

The structure corresponding to the controlq statistics is
\field{struct virtio_net_stats_cvq}. The corresponding type is
VIRTIO_NET_STATS_TYPE_CVQ. This is for the controlq.

\begin{lstlisting}
struct virtio_net_stats_cvq {
    struct virtio_net_stats_reply_hdr hdr;

    le64 command_num;
    le64 ok_num;
};
\end{lstlisting}

\begin{description}
    \item [command_num]
        The number of commands received by the device including the current command.

    \item [ok_num]
        The number of commands completed successfully by the device including the current command.
\end{description}


\subparagraph{Receiveq Basic Statistics}\label{sec:Device Types / Network Device / Device Operation / Control Virtqueue / Device Statistics / Receiveq Basic Statistics}

The structure corresponding to the receiveq basic statistics is
\field{struct virtio_net_stats_rx_basic}. The corresponding type is
VIRTIO_NET_STATS_TYPE_RX_BASIC. This is for the receiveq.

Receiveq basic statistics do not require any feature. As long as the device supports
VIRTIO_NET_F_DEVICE_STATS, the following are the receiveq basic statistics.

\begin{lstlisting}
struct virtio_net_stats_rx_basic {
    struct virtio_net_stats_reply_hdr hdr;

    le64 rx_notifications;

    le64 rx_packets;
    le64 rx_bytes;

    le64 rx_interrupts;

    le64 rx_drops;
    le64 rx_drop_overruns;
};
\end{lstlisting}

The packets described below were all presented on the specified virtqueue.
\begin{description}
    \item [rx_notifications]
        The number of driver notifications received by the device for this
        receiveq.

    \item [rx_packets]
        This is the number of packets passed to the driver by the device.

    \item [rx_bytes]
        This is the bytes of packets passed to the driver by the device.

    \item [rx_interrupts]
        The number of interrupts generated by the device for this receiveq.

    \item [rx_drops]
        This is the number of packets dropped by the device. The count includes
        all types of packets dropped by the device.

    \item [rx_drop_overruns]
        This is the number of packets dropped by the device when no more
        descriptors were available.

\end{description}

\subparagraph{Transmitq Basic Statistics}\label{sec:Device Types / Network Device / Device Operation / Control Virtqueue / Device Statistics / Transmitq Basic Statistics}

The structure corresponding to the transmitq basic statistics is
\field{struct virtio_net_stats_tx_basic}. The corresponding type is
VIRTIO_NET_STATS_TYPE_TX_BASIC. This is for the transmitq.

Transmitq basic statistics do not require any feature. As long as the device supports
VIRTIO_NET_F_DEVICE_STATS, the following are the transmitq basic statistics.

\begin{lstlisting}
struct virtio_net_stats_tx_basic {
    struct virtio_net_stats_reply_hdr hdr;

    le64 tx_notifications;

    le64 tx_packets;
    le64 tx_bytes;

    le64 tx_interrupts;

    le64 tx_drops;
    le64 tx_drop_malformed;
};
\end{lstlisting}

The packets described below are all for a specific virtqueue.
\begin{description}
    \item [tx_notifications]
        The number of driver notifications received by the device for this
        transmitq.

    \item [tx_packets]
        This is the number of packets sent by the device (not the packets
        got from the driver).

    \item [tx_bytes]
        This is the number of bytes sent by the device for all the sent packets
        (not the bytes sent got from the driver).

    \item [tx_interrupts]
        The number of interrupts generated by the device for this transmitq.

    \item [tx_drops]
        The number of packets dropped by the device. The count includes all
        types of packets dropped by the device.

    \item [tx_drop_malformed]
        The number of packets dropped by the device, when the descriptors are
        malformed. For example, the buffer is too short.
\end{description}

\subparagraph{Receiveq CSUM Statistics}\label{sec:Device Types / Network Device / Device Operation / Control Virtqueue / Device Statistics / Receiveq CSUM Statistics}

The structure corresponding to the receiveq checksum statistics is
\field{struct virtio_net_stats_rx_csum}. The corresponding type is
VIRTIO_NET_STATS_TYPE_RX_CSUM. This is for the receiveq.

Only after the VIRTIO_NET_F_GUEST_CSUM is negotiated, the receiveq checksum
statistics can be obtained.

\begin{lstlisting}
struct virtio_net_stats_rx_csum {
    struct virtio_net_stats_reply_hdr hdr;

    le64 rx_csum_valid;
    le64 rx_needs_csum;
    le64 rx_csum_none;
    le64 rx_csum_bad;
};
\end{lstlisting}

The packets described below were all presented on the specified virtqueue.
\begin{description}
    \item [rx_csum_valid]
        The number of packets with VIRTIO_NET_HDR_F_DATA_VALID.

    \item [rx_needs_csum]
        The number of packets with VIRTIO_NET_HDR_F_NEEDS_CSUM.

    \item [rx_csum_none]
        The number of packets without hardware checksum. The packet here refers
        to the non-TCP/UDP packet that the device cannot recognize.

    \item [rx_csum_bad]
        The number of packets with checksum mismatch.

\end{description}

\subparagraph{Transmitq CSUM Statistics}\label{sec:Device Types / Network Device / Device Operation / Control Virtqueue / Device Statistics / Transmitq CSUM Statistics}

The structure corresponding to the transmitq checksum statistics is
\field{struct virtio_net_stats_tx_csum}. The corresponding type is
VIRTIO_NET_STATS_TYPE_TX_CSUM. This is for the transmitq.

Only after the VIRTIO_NET_F_CSUM is negotiated, the transmitq checksum
statistics can be obtained.

The following are the transmitq checksum statistics:

\begin{lstlisting}
struct virtio_net_stats_tx_csum {
    struct virtio_net_stats_reply_hdr hdr;

    le64 tx_csum_none;
    le64 tx_needs_csum;
};
\end{lstlisting}

The packets described below are all for a specific virtqueue.
\begin{description}
    \item [tx_csum_none]
        The number of packets which do not require hardware checksum.

    \item [tx_needs_csum]
        The number of packets which require checksum calculation by the device.

\end{description}

\subparagraph{Receiveq GSO Statistics}\label{sec:Device Types / Network Device / Device Operation / Control Virtqueue / Device Statistics / Receiveq GSO Statistics}

The structure corresponding to the receivq GSO statistics is
\field{struct virtio_net_stats_rx_gso}. The corresponding type is
VIRTIO_NET_STATS_TYPE_RX_GSO. This is for the receiveq.

If one or more of the VIRTIO_NET_F_GUEST_TSO4, VIRTIO_NET_F_GUEST_TSO6
have been negotiated, the receiveq GSO statistics can be obtained.

GSO packets refer to packets passed by the device to the driver where
\field{gso_type} is not VIRTIO_NET_HDR_GSO_NONE.

\begin{lstlisting}
struct virtio_net_stats_rx_gso {
    struct virtio_net_stats_reply_hdr hdr;

    le64 rx_gso_packets;
    le64 rx_gso_bytes;
    le64 rx_gso_packets_coalesced;
    le64 rx_gso_bytes_coalesced;
};
\end{lstlisting}

The packets described below were all presented on the specified virtqueue.
\begin{description}
    \item [rx_gso_packets]
        The number of the GSO packets received by the device.

    \item [rx_gso_bytes]
        The bytes of the GSO packets received by the device.
        This includes the header size of the GSO packet.

    \item [rx_gso_packets_coalesced]
        The number of the GSO packets coalesced by the device.

    \item [rx_gso_bytes_coalesced]
        The bytes of the GSO packets coalesced by the device.
        This includes the header size of the GSO packet.
\end{description}

\subparagraph{Transmitq GSO Statistics}\label{sec:Device Types / Network Device / Device Operation / Control Virtqueue / Device Statistics / Transmitq GSO Statistics}

The structure corresponding to the transmitq GSO statistics is
\field{struct virtio_net_stats_tx_gso}. The corresponding type is
VIRTIO_NET_STATS_TYPE_TX_GSO. This is for the transmitq.

If one or more of the VIRTIO_NET_F_HOST_TSO4, VIRTIO_NET_F_HOST_TSO6,
VIRTIO_NET_F_HOST_USO options have been negotiated, the transmitq GSO statistics
can be obtained.

GSO packets refer to packets passed by the driver to the device where
\field{gso_type} is not VIRTIO_NET_HDR_GSO_NONE.
See more \ref{sec:Device Types / Network Device / Device Operation / Packet
Transmission}.

\begin{lstlisting}
struct virtio_net_stats_tx_gso {
    struct virtio_net_stats_reply_hdr hdr;

    le64 tx_gso_packets;
    le64 tx_gso_bytes;
    le64 tx_gso_segments;
    le64 tx_gso_segments_bytes;
    le64 tx_gso_packets_noseg;
    le64 tx_gso_bytes_noseg;
};
\end{lstlisting}

The packets described below are all for a specific virtqueue.
\begin{description}
    \item [tx_gso_packets]
        The number of the GSO packets sent by the device.

    \item [tx_gso_bytes]
        The bytes of the GSO packets sent by the device.

    \item [tx_gso_segments]
        The number of segments prepared from GSO packets.

    \item [tx_gso_segments_bytes]
        The bytes of segments prepared from GSO packets.

    \item [tx_gso_packets_noseg]
        The number of the GSO packets without segmentation.

    \item [tx_gso_bytes_noseg]
        The bytes of the GSO packets without segmentation.

\end{description}

\subparagraph{Receiveq Speed Statistics}\label{sec:Device Types / Network Device / Device Operation / Control Virtqueue / Device Statistics / Receiveq Speed Statistics}

The structure corresponding to the receiveq speed statistics is
\field{struct virtio_net_stats_rx_speed}. The corresponding type is
VIRTIO_NET_STATS_TYPE_RX_SPEED. This is for the receiveq.

The device has the allowance for the speed. If VIRTIO_NET_F_SPEED_DUPLEX has
been negotiated, the driver can get this by \field{speed}. When the received
packets bitrate exceeds the \field{speed}, some packets may be dropped by the
device.

\begin{lstlisting}
struct virtio_net_stats_rx_speed {
    struct virtio_net_stats_reply_hdr hdr;

    le64 rx_packets_allowance_exceeded;
    le64 rx_bytes_allowance_exceeded;
};
\end{lstlisting}

The packets described below were all presented on the specified virtqueue.
\begin{description}
    \item [rx_packets_allowance_exceeded]
        The number of the packets dropped by the device due to the received
        packets bitrate exceeding the \field{speed}.

    \item [rx_bytes_allowance_exceeded]
        The bytes of the packets dropped by the device due to the received
        packets bitrate exceeding the \field{speed}.

\end{description}

\subparagraph{Transmitq Speed Statistics}\label{sec:Device Types / Network Device / Device Operation / Control Virtqueue / Device Statistics / Transmitq Speed Statistics}

The structure corresponding to the transmitq speed statistics is
\field{struct virtio_net_stats_tx_speed}. The corresponding type is
VIRTIO_NET_STATS_TYPE_TX_SPEED. This is for the transmitq.

The device has the allowance for the speed. If VIRTIO_NET_F_SPEED_DUPLEX has
been negotiated, the driver can get this by \field{speed}. When the transmit
packets bitrate exceeds the \field{speed}, some packets may be dropped by the
device.

\begin{lstlisting}
struct virtio_net_stats_tx_speed {
    struct virtio_net_stats_reply_hdr hdr;

    le64 tx_packets_allowance_exceeded;
    le64 tx_bytes_allowance_exceeded;
};
\end{lstlisting}

The packets described below were all presented on the specified virtqueue.
\begin{description}
    \item [tx_packets_allowance_exceeded]
        The number of the packets dropped by the device due to the transmit packets
        bitrate exceeding the \field{speed}.

    \item [tx_bytes_allowance_exceeded]
        The bytes of the packets dropped by the device due to the transmit packets
        bitrate exceeding the \field{speed}.

\end{description}

\devicenormative{\subparagraph}{Device Statistics}{Device Types / Network Device / Device Operation / Control Virtqueue / Device Statistics}

When the VIRTIO_NET_F_DEVICE_STATS feature is negotiated, the device MUST reply
to the command VIRTIO_NET_CTRL_STATS_QUERY with the
\field{struct virtio_net_stats_capabilities}. \field{supported_stats_types}
includes all the statistic types supported by the device.

If \field{struct virtio_net_ctrl_queue_stats} is incorrect (such as the
following), the device MUST set \field{ack} to VIRTIO_NET_ERR. Even if there is
only one error, the device MUST fail the entire command.
\begin{itemize}
    \item \field{vq_index} exceeds the queue range.
    \item \field{types_bitmap} contains unknown types.
    \item One or more of the bits present in \field{types_bitmap} is not valid
        for the specified virtqueue.
    \item The feature corresponding to the specified \field{types_bitmap} was
        not negotiated.
\end{itemize}

The device MUST set the actual size of the bytes occupied by the reply to the
\field{size} of the \field{hdr}. And the device MUST set the \field{type} and
the \field{vq_index} of the statistic header.

The \field{command-specific-result} buffer allocated by the driver may be
smaller or bigger than all the statistics specified by
\field{struct virtio_net_ctrl_queue_stats}. The device MUST fill up only upto
the valid bytes.

The statistics counter replied by the device MUST wrap around to zero by the
device on the overflow.

\drivernormative{\subparagraph}{Device Statistics}{Device Types / Network Device / Device Operation / Control Virtqueue / Device Statistics}

The types contained in the \field{types_bitmap} MUST be queried from the device
via command VIRTIO_NET_CTRL_STATS_QUERY.

\field{types_bitmap} in \field{struct virtio_net_ctrl_queue_stats} MUST be valid to the
vq specified by \field{vq_index}.

The \field{command-specific-result} buffer allocated by the driver MUST have
enough capacity to store all the statistics reply headers defined in
\field{struct virtio_net_ctrl_queue_stats}. If the
\field{command-specific-result} buffer is fully utilized by the device but some
replies are missed, it is possible that some statistics may exceed the capacity
of the driver's records. In such cases, the driver should allocate additional
space for the \field{command-specific-result} buffer.

\subsubsection{Flow filter}\label{sec:Device Types / Network Device / Device Operation / Flow filter}

A network device can support one or more flow filter rules. Each flow filter rule
is applied by matching a packet and then taking an action, such as directing the packet
to a specific receiveq or dropping the packet. An example of a match is
matching on specific source and destination IP addresses.

A flow filter rule is a device resource object that consists of a key,
a processing priority, and an action to either direct a packet to a
receive queue or drop the packet.

Each rule uses a classifier. The key is matched against the packet using
a classifier, defining which fields in the packet are matched.
A classifier resource object consists of one or more field selectors, each with
a type that specifies the header fields to be matched against, and a mask.
The mask can match whole fields or parts of a field in a header. Each
rule resource object depends on the classifier resource object.

When a packet is received, relevant fields are extracted
(in the same way) from both the packet and the key according to the
classifier. The resulting field contents are then compared -
if they are identical the rule action is taken, if they are not, the rule is ignored.

Multiple flow filter rules are part of a group. The rule resource object
depends on the group. Each rule within a
group has a rule priority, and each group also has a group priority. For a
packet, a group with the highest priority is selected first. Within a group,
rules are applied from highest to lowest priority, until one of the rules
matches the packet and an action is taken. If all the rules within a group
are ignored, the group with the next highest priority is selected, and so on.

The device and the driver indicates flow filter resource limits using the capability
\ref{par:Device Types / Network Device / Device Operation / Flow filter / Device and driver capabilities / VIRTIO-NET-FF-RESOURCE-CAP} specifying the limits on the number of flow filter rule,
group and classifier resource objects. The capability \ref{par:Device Types / Network Device / Device Operation / Flow filter / Device and driver capabilities / VIRTIO-NET-FF-SELECTOR-CAP} specifies which selectors the device supports.
The driver indicates the selectors it is using by setting the flow
filter selector capability, prior to adding any resource objects.

The capability \ref{par:Device Types / Network Device / Device Operation / Flow filter / Device and driver capabilities / VIRTIO-NET-FF-ACTION-CAP} specifies which actions the device supports.

The driver controls the flow filter rule, classifier and group resource objects using
administration commands described in
\ref{sec:Basic Facilities of a Virtio Device / Device groups / Group administration commands / Device resource objects}.

\paragraph{Packet processing order}\label{sec:sec:Device Types / Network Device / Device Operation / Flow filter / Packet processing order}

Note that flow filter rules are applied after MAC/VLAN filtering. Flow filter
rules take precedence over steering: if a flow filter rule results in an action,
the steering configuration does not apply. The steering configuration only applies
to packets for which no flow filter rule action was performed. For example,
incoming packets can be processed in the following order:

\begin{itemize}
\item apply steering configuration received using control virtqueue commands
      VIRTIO_NET_CTRL_RX, VIRTIO_NET_CTRL_MAC and VIRTIO_NET_CTRL_VLAN.
\item apply flow filter rules if any.
\item if no filter rule applied, apply steering configuration received using command
      VIRTIO_NET_CTRL_MQ_RSS_CONFIG or as per automatic receive steering.
\end{itemize}

Some incoming packet processing examples:
\begin{itemize}
\item If the packet is dropped by the flow filter rule, RSS
      steering is ignored for the packet.
\item If the packet is directed to a specific receiveq using flow filter rule,
      the RSS steering is ignored for the packet.
\item If a packet is dropped due to the VIRTIO_NET_CTRL_MAC configuration,
      both flow filter rules and the RSS steering are ignored for the packet.
\item If a packet does not match any flow filter rules,
      the RSS steering is used to select the receiveq for the packet (if enabled).
\item If there are two flow filter groups configured as group_A and group_B
      with respective group priorities as 4, and 5; flow filter rules of
      group_B are applied first having highest group priority, if there is a match,
      the flow filter rules of group_A are ignored; if there is no match for
      the flow filter rules in group_B, the flow filter rules of next level group_A are applied.
\end{itemize}

\paragraph{Device and driver capabilities}
\label{par:Device Types / Network Device / Device Operation / Flow filter / Device and driver capabilities}

\subparagraph{VIRTIO_NET_FF_RESOURCE_CAP}
\label{par:Device Types / Network Device / Device Operation / Flow filter / Device and driver capabilities / VIRTIO-NET-FF-RESOURCE-CAP}

The capability VIRTIO_NET_FF_RESOURCE_CAP indicates the flow filter resource limits.
\field{cap_specific_data} is in the format
\field{struct virtio_net_ff_cap_data}.

\begin{lstlisting}
struct virtio_net_ff_cap_data {
        le32 groups_limit;
        le32 selectors_limit;
        le32 rules_limit;
        le32 rules_per_group_limit;
        u8 last_rule_priority;
        u8 selectors_per_classifier_limit;
};
\end{lstlisting}

\field{groups_limit}, and \field{selectors_limit} represent the maximum
number of flow filter groups and selectors, respectively, that the driver can create.
 \field{rules_limit} is the maximum number of
flow fiilter rules that the driver can create across all the groups.
\field{rules_per_group_limit} is the maximum number of flow filter rules that the driver
can create for each flow filter group.

\field{last_rule_priority} is the highest priority that can be assigned to a
flow filter rule.

\field{selectors_per_classifier_limit} is the maximum number of selectors
that a classifier can have.

\subparagraph{VIRTIO_NET_FF_SELECTOR_CAP}
\label{par:Device Types / Network Device / Device Operation / Flow filter / Device and driver capabilities / VIRTIO-NET-FF-SELECTOR-CAP}

The capability VIRTIO_NET_FF_SELECTOR_CAP lists the supported selectors and the
supported packet header fields for each selector.
\field{cap_specific_data} is in the format \field{struct virtio_net_ff_cap_mask_data}.

\begin{lstlisting}[label={lst:Device Types / Network Device / Device Operation / Flow filter / Device and driver capabilities / VIRTIO-NET-FF-SELECTOR-CAP / virtio-net-ff-selector}]
struct virtio_net_ff_selector {
        u8 type;
        u8 flags;
        u8 reserved[2];
        u8 length;
        u8 reserved1[3];
        u8 mask[];
};

struct virtio_net_ff_cap_mask_data {
        u8 count;
        u8 reserved[7];
        struct virtio_net_ff_selector selectors[];
};

#define VIRTIO_NET_FF_MASK_F_PARTIAL_MASK (1 << 0)
\end{lstlisting}

\field{count} indicates number of valid entries in the \field{selectors} array.
\field{selectors[]} is an array of supported selectors. Within each array entry:
\field{type} specifies the type of the packet header, as defined in table
\ref{table:Device Types / Network Device / Device Operation / Flow filter / Device and driver capabilities / VIRTIO-NET-FF-SELECTOR-CAP / flow filter selector types}. \field{mask} specifies which fields of the
packet header can be matched in a flow filter rule.

Each \field{type} is also listed in table
\ref{table:Device Types / Network Device / Device Operation / Flow filter / Device and driver capabilities / VIRTIO-NET-FF-SELECTOR-CAP / flow filter selector types}. \field{mask} is a byte array
in network byte order. For example, when \field{type} is VIRTIO_NET_FF_MASK_TYPE_IPV6,
the \field{mask} is in the format \hyperref[intro:IPv6-Header-Format]{IPv6 Header Format}.

If partial masking is not set, then all bits in each field have to be either all 0s
to ignore this field or all 1s to match on this field. If partial masking is set,
then any combination of bits can bit set to match on these bits.
For example, when a selector \field{type} is VIRTIO_NET_FF_MASK_TYPE_ETH, if
\field{mask[0-12]} are zero and \field{mask[13-14]} are 0xff (all 1s), it
indicates that matching is only supported for \field{EtherType} of
\field{Ethernet MAC frame}, matching is not supported for
\field{Destination Address} and \field{Source Address}.

The entries in the array \field{selectors} are ordered by
\field{type}, with each \field{type} value only appearing once.

\field{length} is the length of a dynamic array \field{mask} in bytes.
\field{reserved} and \field{reserved1} are reserved and set to zero.

\begin{table}[H]
\caption{Flow filter selector types}
\label{table:Device Types / Network Device / Device Operation / Flow filter / Device and driver capabilities / VIRTIO-NET-FF-SELECTOR-CAP / flow filter selector types}
\begin{tabularx}{\textwidth}{ |l|X|X| }
\hline
Type & Name & Description \\
\hline \hline
0x0 & - & Reserved \\
\hline
0x1 & VIRTIO_NET_FF_MASK_TYPE_ETH & 14 bytes of frame header starting from destination address described in \hyperref[intro:IEEE 802.3-2022]{IEEE 802.3-2022} \\
\hline
0x2 & VIRTIO_NET_FF_MASK_TYPE_IPV4 & 20 bytes of \hyperref[intro:Internet-Header-Format]{IPv4: Internet Header Format} \\
\hline
0x3 & VIRTIO_NET_FF_MASK_TYPE_IPV6 & 40 bytes of \hyperref[intro:IPv6-Header-Format]{IPv6 Header Format} \\
\hline
0x4 & VIRTIO_NET_FF_MASK_TYPE_TCP & 20 bytes of \hyperref[intro:TCP-Header-Format]{TCP Header Format} \\
\hline
0x5 & VIRTIO_NET_FF_MASK_TYPE_UDP & 8 bytes of UDP header described in \hyperref[intro:UDP]{UDP} \\
\hline
0x6 - 0xFF & & Reserved for future \\
\hline
\end{tabularx}
\end{table}

When VIRTIO_NET_FF_MASK_F_PARTIAL_MASK (bit 0) is set, it indicates that
partial masking is supported for all the fields of the selector identified by \field{type}.

For the selector \field{type} VIRTIO_NET_FF_MASK_TYPE_IPV4, if a partial mask is unsupported,
then matching on an individual bit of \field{Flags} in the
\field{IPv4: Internet Header Format} is unsupported. \field{Flags} has to match as a whole
if it is supported.

For the selector \field{type} VIRTIO_NET_FF_MASK_TYPE_IPV4, \field{mask} includes fields
up to the \field{Destination Address}; that is, \field{Options} and
\field{Padding} are excluded.

For the selector \field{type} VIRTIO_NET_FF_MASK_TYPE_IPV6, the \field{Next Header} field
of the \field{mask} corresponds to the \field{Next Header} in the packet
when \field{IPv6 Extension Headers} are not present. When the packet includes
one or more \field{IPv6 Extension Headers}, the \field{Next Header} field of
the \field{mask} corresponds to the \field{Next Header} of the last
\field{IPv6 Extension Header} in the packet.

For the selector \field{type} VIRTIO_NET_FF_MASK_TYPE_TCP, \field{Control bits}
are treated as individual fields for matching; that is, matching individual
\field{Control bits} does not depend on the partial mask support.

\subparagraph{VIRTIO_NET_FF_ACTION_CAP}
\label{par:Device Types / Network Device / Device Operation / Flow filter / Device and driver capabilities / VIRTIO-NET-FF-ACTION-CAP}

The capability VIRTIO_NET_FF_ACTION_CAP lists the supported actions in a rule.
\field{cap_specific_data} is in the format \field{struct virtio_net_ff_cap_actions}.

\begin{lstlisting}
struct virtio_net_ff_actions {
        u8 count;
        u8 reserved[7];
        u8 actions[];
};
\end{lstlisting}

\field{actions} is an array listing all possible actions.
The entries in the array are ordered from the smallest to the largest,
with each supported value appearing exactly once. Each entry can have the
following values:

\begin{table}[H]
\caption{Flow filter rule actions}
\label{table:Device Types / Network Device / Device Operation / Flow filter / Device and driver capabilities / VIRTIO-NET-FF-ACTION-CAP / flow filter rule actions}
\begin{tabularx}{\textwidth}{ |l|X|X| }
\hline
Action & Name & Description \\
\hline \hline
0x0 & - & reserved \\
\hline
0x1 & VIRTIO_NET_FF_ACTION_DROP & Matching packet will be dropped by the device \\
\hline
0x2 & VIRTIO_NET_FF_ACTION_DIRECT_RX_VQ & Matching packet will be directed to a receive queue \\
\hline
0x3 - 0xFF & & Reserved for future \\
\hline
\end{tabularx}
\end{table}

\paragraph{Resource objects}
\label{par:Device Types / Network Device / Device Operation / Flow filter / Resource objects}

\subparagraph{VIRTIO_NET_RESOURCE_OBJ_FF_GROUP}\label{par:Device Types / Network Device / Device Operation / Flow filter / Resource objects / VIRTIO-NET-RESOURCE-OBJ-FF-GROUP}

A flow filter group contains between 0 and \field{rules_limit} rules, as specified by the
capability VIRTIO_NET_FF_RESOURCE_CAP. For the flow filter group object both
\field{resource_obj_specific_data} and
\field{resource_obj_specific_result} are in the format
\field{struct virtio_net_resource_obj_ff_group}.

\begin{lstlisting}
struct virtio_net_resource_obj_ff_group {
        le16 group_priority;
};
\end{lstlisting}

\field{group_priority} specifies the priority for the group. Each group has a
distinct priority. For each incoming packet, the device tries to apply rules
from groups from higher \field{group_priority} value to lower, until either a
rule matches the packet or all groups have been tried.

\subparagraph{VIRTIO_NET_RESOURCE_OBJ_FF_CLASSIFIER}\label{par:Device Types / Network Device / Device Operation / Flow filter / Resource objects / VIRTIO-NET-RESOURCE-OBJ-FF-CLASSIFIER}

A classifier is used to match a flow filter key against a packet. The
classifier defines the desired packet fields to match, and is represented by
the VIRTIO_NET_RESOURCE_OBJ_FF_CLASSIFIER device resource object.

For the flow filter classifier object both \field{resource_obj_specific_data} and
\field{resource_obj_specific_result} are in the format
\field{struct virtio_net_resource_obj_ff_classifier}.

\begin{lstlisting}
struct virtio_net_resource_obj_ff_classifier {
        u8 count;
        u8 reserved[7];
        struct virtio_net_ff_selector selectors[];
};
\end{lstlisting}

A classifier is an array of \field{selectors}. The number of selectors in the
array is indicated by \field{count}. The selector has a type that specifies
the header fields to be matched against, and a mask.
See \ref{lst:Device Types / Network Device / Device Operation / Flow filter / Device and driver capabilities / VIRTIO-NET-FF-SELECTOR-CAP / virtio-net-ff-selector}
for details about selectors.

The first selector is always VIRTIO_NET_FF_MASK_TYPE_ETH. When there are multiple
selectors, a second selector can be either VIRTIO_NET_FF_MASK_TYPE_IPV4
or VIRTIO_NET_FF_MASK_TYPE_IPV6. If the third selector exists, the third
selector can be either VIRTIO_NET_FF_MASK_TYPE_UDP or VIRTIO_NET_FF_MASK_TYPE_TCP.
For example, to match a Ethernet IPv6 UDP packet,
\field{selectors[0].type} is set to VIRTIO_NET_FF_MASK_TYPE_ETH, \field{selectors[1].type}
is set to VIRTIO_NET_FF_MASK_TYPE_IPV6 and \field{selectors[2].type} is
set to VIRTIO_NET_FF_MASK_TYPE_UDP; accordingly, \field{selectors[0].mask[0-13]} is
for Ethernet header fields, \field{selectors[1].mask[0-39]} is set for IPV6 header
and \field{selectors[2].mask[0-7]} is set for UDP header.

When there are multiple selectors, the type of the (N+1)\textsuperscript{th} selector
affects the mask of the (N)\textsuperscript{th} selector. If
\field{count} is 2 or more, all the mask bits within \field{selectors[0]}
corresponding to \field{EtherType} of an Ethernet header are set.

If \field{count} is more than 2:
\begin{itemize}
\item if \field{selector[1].type} is, VIRTIO_NET_FF_MASK_TYPE_IPV4, then, all the mask bits within
\field{selector[1]} for \field{Protocol} is set.
\item if \field{selector[1].type} is, VIRTIO_NET_FF_MASK_TYPE_IPV6, then, all the mask bits within
\field{selector[1]} for \field{Next Header} is set.
\end{itemize}

If for a given packet header field, a subset of bits of a field is to be matched,
and if the partial mask is supported, the flow filter
mask object can specify a mask which has fewer bits set than the packet header
field size. For example, a partial mask for the Ethernet header source mac
address can be of 1-bit for multicast detection instead of 48-bits.

\subparagraph{VIRTIO_NET_RESOURCE_OBJ_FF_RULE}\label{par:Device Types / Network Device / Device Operation / Flow filter / Resource objects / VIRTIO-NET-RESOURCE-OBJ-FF-RULE}

Each flow filter rule resource object comprises a key, a priority, and an action.
For the flow filter rule object,
\field{resource_obj_specific_data} and
\field{resource_obj_specific_result} are in the format
\field{struct virtio_net_resource_obj_ff_rule}.

\begin{lstlisting}
struct virtio_net_resource_obj_ff_rule {
        le32 group_id;
        le32 classifier_id;
        u8 rule_priority;
        u8 key_length; /* length of key in bytes */
        u8 action;
        u8 reserved;
        le16 vq_index;
        u8 reserved1[2];
        u8 keys[][];
};
\end{lstlisting}

\field{group_id} is the resource object ID of the flow filter group to which
this rule belongs. \field{classifier_id} is the resource object ID of the
classifier used to match a packet against the \field{key}.

\field{rule_priority} denotes the priority of the rule within the group
specified by the \field{group_id}.
Rules within the group are applied from the highest to the lowest priority
until a rule matches the packet and an
action is taken. Rules with the same priority can be applied in any order.

\field{reserved} and \field{reserved1} are reserved and set to 0.

\field{keys[][]} is an array of keys to match against packets, using
the classifier specified by \field{classifier_id}. Each entry (key) comprises
a byte array, and they are located one immediately after another.
The size (number of entries) of the array is exactly the same as that of
\field{selectors} in the classifier, or in other words, \field{count}
in the classifier.

\field{key_length} specifies the total length of \field{keys} in bytes.
In other words, it equals the sum total of \field{length} of all
selectors in \field{selectors} in the classifier specified by
\field{classifier_id}.

For example, if a classifier object's \field{selectors[0].type} is
VIRTIO_NET_FF_MASK_TYPE_ETH and \field{selectors[1].type} is
VIRTIO_NET_FF_MASK_TYPE_IPV6,
then selectors[0].length is 14 and selectors[1].length is 40.
Accordingly, the \field{key_length} is set to 54.
This setting indicates that the \field{key} array's length is 54 bytes
comprising a first byte array of 14 bytes for the
Ethernet MAC header in bytes 0-13, immediately followed by 40 bytes for the
IPv6 header in bytes 14-53.

When there are multiple selectors in the classifier object, the key bytes
for (N)\textsuperscript{th} selector are set so that
(N+1)\textsuperscript{th} selector can be matched.

If \field{count} is 2 or more, key bytes of \field{EtherType}
are set according to \hyperref[intro:IEEE 802 Ethertypes]{IEEE 802 Ethertypes}
for VIRTIO_NET_FF_MASK_TYPE_IPV4 or VIRTIO_NET_FF_MASK_TYPE_IPV6 respectively.

If \field{count} is more than 2, when \field{selector[1].type} is
VIRTIO_NET_FF_MASK_TYPE_IPV4 or VIRTIO_NET_FF_MASK_TYPE_IPV6, key
bytes of \field{Protocol} or \field{Next Header} is set as per
\field{Protocol Numbers} defined \hyperref[intro:IANA Protocol Numbers]{IANA Protocol Numbers}
respectively.

\field{action} is the action to take when a packet matches the
\field{key} using the \field{classifier_id}. Supported actions are described in
\ref{table:Device Types / Network Device / Device Operation / Flow filter / Device and driver capabilities / VIRTIO-NET-FF-ACTION-CAP / flow filter rule actions}.

\field{vq_index} specifies a receive virtqueue. When the \field{action} is set
to VIRTIO_NET_FF_ACTION_DIRECT_RX_VQ, and the packet matches the \field{key},
the matching packet is directed to this virtqueue.

Note that at most one action is ever taken for a given packet. If a rule is
applied and an action is taken, the action of other rules is not taken.

\devicenormative{\paragraph}{Flow filter}{Device Types / Network Device / Device Operation / Flow filter}

When the device supports flow filter operations,
\begin{itemize}
\item the device MUST set VIRTIO_NET_FF_RESOURCE_CAP, VIRTIO_NET_FF_SELECTOR_CAP
and VIRTIO_NET_FF_ACTION_CAP capability in the \field{supported_caps} in the
command VIRTIO_ADMIN_CMD_CAP_SUPPORT_QUERY.
\item the device MUST support the administration commands
VIRTIO_ADMIN_CMD_RESOURCE_OBJ_CREATE,
VIRTIO_ADMIN_CMD_RESOURCE_OBJ_MODIFY, VIRTIO_ADMIN_CMD_RESOURCE_OBJ_QUERY,
VIRTIO_ADMIN_CMD_RESOURCE_OBJ_DESTROY for the resource types
VIRTIO_NET_RESOURCE_OBJ_FF_GROUP, VIRTIO_NET_RESOURCE_OBJ_FF_CLASSIFIER and
VIRTIO_NET_RESOURCE_OBJ_FF_RULE.
\end{itemize}

When any of the VIRTIO_NET_FF_RESOURCE_CAP, VIRTIO_NET_FF_SELECTOR_CAP, or
VIRTIO_NET_FF_ACTION_CAP capability is disabled, the device SHOULD set
\field{status} to VIRTIO_ADMIN_STATUS_Q_INVALID_OPCODE for the commands
VIRTIO_ADMIN_CMD_RESOURCE_OBJ_CREATE,
VIRTIO_ADMIN_CMD_RESOURCE_OBJ_MODIFY, VIRTIO_ADMIN_CMD_RESOURCE_OBJ_QUERY,
and VIRTIO_ADMIN_CMD_RESOURCE_OBJ_DESTROY. These commands apply to the resource
\field{type} of VIRTIO_NET_RESOURCE_OBJ_FF_GROUP, VIRTIO_NET_RESOURCE_OBJ_FF_CLASSIFIER, and
VIRTIO_NET_RESOURCE_OBJ_FF_RULE.

The device SHOULD set \field{status} to VIRTIO_ADMIN_STATUS_EINVAL for the
command VIRTIO_ADMIN_CMD_RESOURCE_OBJ_CREATE when the resource \field{type}
is VIRTIO_NET_RESOURCE_OBJ_FF_GROUP, if a flow filter group already exists
with the supplied \field{group_priority}.

The device SHOULD set \field{status} to VIRTIO_ADMIN_STATUS_ENOSPC for the
command VIRTIO_ADMIN_CMD_RESOURCE_OBJ_CREATE when the resource \field{type}
is VIRTIO_NET_RESOURCE_OBJ_FF_GROUP, if the number of flow filter group
objects in the device exceeds the lower of the configured driver
capabilities \field{groups_limit} and \field{rules_per_group_limit}.

The device SHOULD set \field{status} to VIRTIO_ADMIN_STATUS_ENOSPC for the
command VIRTIO_ADMIN_CMD_RESOURCE_OBJ_CREATE when the resource \field{type} is
VIRTIO_NET_RESOURCE_OBJ_FF_CLASSIFIER, if the number of flow filter selector
objects in the device exceeds the configured driver capability
\field{selectors_limit}.

The device SHOULD set \field{status} to VIRTIO_ADMIN_STATUS_EBUSY for the
command VIRTIO_ADMIN_CMD_RESOURCE_OBJ_DESTROY for a flow filter group when
the flow filter group has one or more flow filter rules depending on it.

The device SHOULD set \field{status} to VIRTIO_ADMIN_STATUS_EBUSY for the
command VIRTIO_ADMIN_CMD_RESOURCE_OBJ_DESTROY for a flow filter classifier when
the flow filter classifier has one or more flow filter rules depending on it.

The device SHOULD fail the command VIRTIO_ADMIN_CMD_RESOURCE_OBJ_CREATE for the
flow filter rule resource object if,
\begin{itemize}
\item \field{vq_index} is not a valid receive virtqueue index for
the VIRTIO_NET_FF_ACTION_DIRECT_RX_VQ action,
\item \field{priority} is greater than or equal to
      \field{last_rule_priority},
\item \field{id} is greater than or equal to \field{rules_limit} or
      greater than or equal to \field{rules_per_group_limit}, whichever is lower,
\item the length of \field{keys} and the length of all the mask bytes of
      \field{selectors[].mask} as referred by \field{classifier_id} differs,
\item the supplied \field{action} is not supported in the capability VIRTIO_NET_FF_ACTION_CAP.
\end{itemize}

When the flow filter directs a packet to the virtqueue identified by
\field{vq_index} and if the receive virtqueue is reset, the device
MUST drop such packets.

Upon applying a flow filter rule to a packet, the device MUST STOP any further
application of rules and cease applying any other steering configurations.

For multiple flow filter groups, the device MUST apply the rules from
the group with the highest priority. If any rule from this group is applied,
the device MUST ignore the remaining groups. If none of the rules from the
highest priority group match, the device MUST apply the rules from
the group with the next highest priority, until either a rule matches or
all groups have been attempted.

The device MUST apply the rules within the group from the highest to the
lowest priority until a rule matches the packet, and the device MUST take
the action. If an action is taken, the device MUST not take any other
action for this packet.

The device MAY apply the rules with the same \field{rule_priority} in any
order within the group.

The device MUST process incoming packets in the following order:
\begin{itemize}
\item apply the steering configuration received using control virtqueue
      commands VIRTIO_NET_CTRL_RX, VIRTIO_NET_CTRL_MAC, and
      VIRTIO_NET_CTRL_VLAN.
\item apply flow filter rules if any.
\item if no filter rule is applied, apply the steering configuration
      received using the command VIRTIO_NET_CTRL_MQ_RSS_CONFIG
      or according to automatic receive steering.
\end{itemize}

When processing an incoming packet, if the packet is dropped at any stage, the device
MUST skip further processing.

When the device drops the packet due to the configuration done using the control
virtqueue commands VIRTIO_NET_CTRL_RX or VIRTIO_NET_CTRL_MAC or VIRTIO_NET_CTRL_VLAN,
the device MUST skip flow filter rules for this packet.

When the device performs flow filter match operations and if the operation
result did not have any match in all the groups, the receive packet processing
continues to next level, i.e. to apply configuration done using
VIRTIO_NET_CTRL_MQ_RSS_CONFIG command.

The device MUST support the creation of flow filter classifier objects
using the command VIRTIO_ADMIN_CMD_RESOURCE_OBJ_CREATE with \field{flags}
set to VIRTIO_NET_FF_MASK_F_PARTIAL_MASK;
this support is required even if all the bits of the masks are set for
a field in \field{selectors}, provided that partial masking is supported
for the selectors.

\drivernormative{\paragraph}{Flow filter}{Device Types / Network Device / Device Operation / Flow filter}

The driver MUST enable VIRTIO_NET_FF_RESOURCE_CAP, VIRTIO_NET_FF_SELECTOR_CAP,
and VIRTIO_NET_FF_ACTION_CAP capabilities to use flow filter.

The driver SHOULD NOT remove a flow filter group using the command
VIRTIO_ADMIN_CMD_RESOURCE_OBJ_DESTROY when one or more flow filter rules
depend on that group. The driver SHOULD only destroy the group after
all the associated rules have been destroyed.

The driver SHOULD NOT remove a flow filter classifier using the command
VIRTIO_ADMIN_CMD_RESOURCE_OBJ_DESTROY when one or more flow filter rules
depend on the classifier. The driver SHOULD only destroy the classifier
after all the associated rules have been destroyed.

The driver SHOULD NOT add multiple flow filter rules with the same
\field{rule_priority} within a flow filter group, as these rules MAY match
the same packet. The driver SHOULD assign different \field{rule_priority}
values to different flow filter rules if multiple rules may match a single
packet.

For the command VIRTIO_ADMIN_CMD_RESOURCE_OBJ_CREATE, when creating a resource
of \field{type} VIRTIO_NET_RESOURCE_OBJ_FF_CLASSIFIER, the driver MUST set:
\begin{itemize}
\item \field{selectors[0].type} to VIRTIO_NET_FF_MASK_TYPE_ETH.
\item \field{selectors[1].type} to VIRTIO_NET_FF_MASK_TYPE_IPV4 or
      VIRTIO_NET_FF_MASK_TYPE_IPV6 when \field{count} is more than 1,
\item \field{selectors[2].type} VIRTIO_NET_FF_MASK_TYPE_UDP or
      VIRTIO_NET_FF_MASK_TYPE_TCP when \field{count} is more than 2.
\end{itemize}

For the command VIRTIO_ADMIN_CMD_RESOURCE_OBJ_CREATE, when creating a resource
of \field{type} VIRTIO_NET_RESOURCE_OBJ_FF_CLASSIFIER, the driver MUST set:
\begin{itemize}
\item \field{selectors[0].mask} bytes to all 1s for the \field{EtherType}
       when \field{count} is 2 or more.
\item \field{selectors[1].mask} bytes to all 1s for \field{Protocol} or \field{Next Header}
       when \field{selector[1].type} is VIRTIO_NET_FF_MASK_TYPE_IPV4 or VIRTIO_NET_FF_MASK_TYPE_IPV6,
       and when \field{count} is more than 2.
\end{itemize}

For the command VIRTIO_ADMIN_CMD_RESOURCE_OBJ_CREATE, the resource \field{type}
VIRTIO_NET_RESOURCE_OBJ_FF_RULE, if the corresponding classifier object's
\field{count} is 2 or more, the driver MUST SET the \field{keys} bytes of
\field{EtherType} in accordance with
\hyperref[intro:IEEE 802 Ethertypes]{IEEE 802 Ethertypes}
for either VIRTIO_NET_FF_MASK_TYPE_IPV4 or VIRTIO_NET_FF_MASK_TYPE_IPV6.

For the command VIRTIO_ADMIN_CMD_RESOURCE_OBJ_CREATE, when creating a resource of
\field{type} VIRTIO_NET_RESOURCE_OBJ_FF_RULE, if the corresponding classifier
object's \field{count} is more than 2, and the \field{selector[1].type} is either
VIRTIO_NET_FF_MASK_TYPE_IPV4 or VIRTIO_NET_FF_MASK_TYPE_IPV6, the driver MUST
set the \field{keys} bytes for the \field{Protocol} or \field{Next Header}
according to \hyperref[intro:IANA Protocol Numbers]{IANA Protocol Numbers} respectively.

The driver SHOULD set all the bits for a field in the mask of a selector in both the
capability and the classifier object, unless the VIRTIO_NET_FF_MASK_F_PARTIAL_MASK
is enabled.

\subsubsection{Legacy Interface: Framing Requirements}\label{sec:Device
Types / Network Device / Legacy Interface: Framing Requirements}

When using legacy interfaces, transitional drivers which have not
negotiated VIRTIO_F_ANY_LAYOUT MUST use a single descriptor for the
\field{struct virtio_net_hdr} on both transmit and receive, with the
network data in the following descriptors.

Additionally, when using the control virtqueue (see \ref{sec:Device
Types / Network Device / Device Operation / Control Virtqueue})
, transitional drivers which have not
negotiated VIRTIO_F_ANY_LAYOUT MUST:
\begin{itemize}
\item for all commands, use a single 2-byte descriptor including the first two
fields: \field{class} and \field{command}
\item for all commands except VIRTIO_NET_CTRL_MAC_TABLE_SET
use a single descriptor including command-specific-data
with no padding.
\item for the VIRTIO_NET_CTRL_MAC_TABLE_SET command use exactly
two descriptors including command-specific-data with no padding:
the first of these descriptors MUST include the
virtio_net_ctrl_mac table structure for the unicast addresses with no padding,
the second of these descriptors MUST include the
virtio_net_ctrl_mac table structure for the multicast addresses
with no padding.
\item for all commands, use a single 1-byte descriptor for the
\field{ack} field
\end{itemize}

See \ref{sec:Basic
Facilities of a Virtio Device / Virtqueues / Message Framing}.

\section{Network Device}\label{sec:Device Types / Network Device}

The virtio network device is a virtual network interface controller.
It consists of a virtual Ethernet link which connects the device
to the Ethernet network. The device has transmit and receive
queues. The driver adds empty buffers to the receive virtqueue.
The device receives incoming packets from the link; the device
places these incoming packets in the receive virtqueue buffers.
The driver adds outgoing packets to the transmit virtqueue. The device
removes these packets from the transmit virtqueue and sends them to
the link. The device may have a control virtqueue. The driver
uses the control virtqueue to dynamically manipulate various
features of the initialized device.

\subsection{Device ID}\label{sec:Device Types / Network Device / Device ID}

 1

\subsection{Virtqueues}\label{sec:Device Types / Network Device / Virtqueues}

\begin{description}
\item[0] receiveq1
\item[1] transmitq1
\item[\ldots]
\item[2(N-1)] receiveqN
\item[2(N-1)+1] transmitqN
\item[2N] controlq
\end{description}

 N=1 if neither VIRTIO_NET_F_MQ nor VIRTIO_NET_F_RSS are negotiated, otherwise N is set by
 \field{max_virtqueue_pairs}.

controlq is optional; it only exists if VIRTIO_NET_F_CTRL_VQ is
negotiated.

\subsection{Feature bits}\label{sec:Device Types / Network Device / Feature bits}

\begin{description}
\item[VIRTIO_NET_F_CSUM (0)] Device handles packets with partial checksum offload.

\item[VIRTIO_NET_F_GUEST_CSUM (1)] Driver handles packets with partial checksum.

\item[VIRTIO_NET_F_CTRL_GUEST_OFFLOADS (2)] Control channel offloads
        reconfiguration support.

\item[VIRTIO_NET_F_MTU(3)] Device maximum MTU reporting is supported. If
    offered by the device, device advises driver about the value of
    its maximum MTU. If negotiated, the driver uses \field{mtu} as
    the maximum MTU value.

\item[VIRTIO_NET_F_MAC (5)] Device has given MAC address.

\item[VIRTIO_NET_F_GUEST_TSO4 (7)] Driver can receive TSOv4.

\item[VIRTIO_NET_F_GUEST_TSO6 (8)] Driver can receive TSOv6.

\item[VIRTIO_NET_F_GUEST_ECN (9)] Driver can receive TSO with ECN.

\item[VIRTIO_NET_F_GUEST_UFO (10)] Driver can receive UFO.

\item[VIRTIO_NET_F_HOST_TSO4 (11)] Device can receive TSOv4.

\item[VIRTIO_NET_F_HOST_TSO6 (12)] Device can receive TSOv6.

\item[VIRTIO_NET_F_HOST_ECN (13)] Device can receive TSO with ECN.

\item[VIRTIO_NET_F_HOST_UFO (14)] Device can receive UFO.

\item[VIRTIO_NET_F_MRG_RXBUF (15)] Driver can merge receive buffers.

\item[VIRTIO_NET_F_STATUS (16)] Configuration status field is
    available.

\item[VIRTIO_NET_F_CTRL_VQ (17)] Control channel is available.

\item[VIRTIO_NET_F_CTRL_RX (18)] Control channel RX mode support.

\item[VIRTIO_NET_F_CTRL_VLAN (19)] Control channel VLAN filtering.

\item[VIRTIO_NET_F_CTRL_RX_EXTRA (20)]	Control channel RX extra mode support.

\item[VIRTIO_NET_F_GUEST_ANNOUNCE(21)] Driver can send gratuitous
    packets.

\item[VIRTIO_NET_F_MQ(22)] Device supports multiqueue with automatic
    receive steering.

\item[VIRTIO_NET_F_CTRL_MAC_ADDR(23)] Set MAC address through control
    channel.

\item[VIRTIO_NET_F_DEVICE_STATS(50)] Device can provide device-level statistics
    to the driver through the control virtqueue.

\item[VIRTIO_NET_F_HASH_TUNNEL(51)] Device supports inner header hash for encapsulated packets.

\item[VIRTIO_NET_F_VQ_NOTF_COAL(52)] Device supports virtqueue notification coalescing.

\item[VIRTIO_NET_F_NOTF_COAL(53)] Device supports notifications coalescing.

\item[VIRTIO_NET_F_GUEST_USO4 (54)] Driver can receive USOv4 packets.

\item[VIRTIO_NET_F_GUEST_USO6 (55)] Driver can receive USOv6 packets.

\item[VIRTIO_NET_F_HOST_USO (56)] Device can receive USO packets. Unlike UFO
 (fragmenting the packet) the USO splits large UDP packet
 to several segments when each of these smaller packets has UDP header.

\item[VIRTIO_NET_F_HASH_REPORT(57)] Device can report per-packet hash
    value and a type of calculated hash.

\item[VIRTIO_NET_F_GUEST_HDRLEN(59)] Driver can provide the exact \field{hdr_len}
    value. Device benefits from knowing the exact header length.

\item[VIRTIO_NET_F_RSS(60)] Device supports RSS (receive-side scaling)
    with Toeplitz hash calculation and configurable hash
    parameters for receive steering.

\item[VIRTIO_NET_F_RSC_EXT(61)] Device can process duplicated ACKs
    and report number of coalesced segments and duplicated ACKs.

\item[VIRTIO_NET_F_STANDBY(62)] Device may act as a standby for a primary
    device with the same MAC address.

\item[VIRTIO_NET_F_SPEED_DUPLEX(63)] Device reports speed and duplex.

\item[VIRTIO_NET_F_RSS_CONTEXT(64)] Device supports multiple RSS contexts.

\item[VIRTIO_NET_F_GUEST_UDP_TUNNEL_GSO (65)] Driver can receive GSO packets
  carried by a UDP tunnel.

\item[VIRTIO_NET_F_GUEST_UDP_TUNNEL_GSO_CSUM (66)] Driver handles packets
  carried by a UDP tunnel with partial csum for the outer header.

\item[VIRTIO_NET_F_HOST_UDP_TUNNEL_GSO (67)] Device can receive GSO packets
  carried by a UDP tunnel.

\item[VIRTIO_NET_F_HOST_UDP_TUNNEL_GSO_CSUM (68)] Device handles packets
  carried by a UDP tunnel with partial csum for the outer header.
\end{description}

\subsubsection{Feature bit requirements}\label{sec:Device Types / Network Device / Feature bits / Feature bit requirements}

Some networking feature bits require other networking feature bits
(see \ref{drivernormative:Basic Facilities of a Virtio Device / Feature Bits}):

\begin{description}
\item[VIRTIO_NET_F_GUEST_TSO4] Requires VIRTIO_NET_F_GUEST_CSUM.
\item[VIRTIO_NET_F_GUEST_TSO6] Requires VIRTIO_NET_F_GUEST_CSUM.
\item[VIRTIO_NET_F_GUEST_ECN] Requires VIRTIO_NET_F_GUEST_TSO4 or VIRTIO_NET_F_GUEST_TSO6.
\item[VIRTIO_NET_F_GUEST_UFO] Requires VIRTIO_NET_F_GUEST_CSUM.
\item[VIRTIO_NET_F_GUEST_USO4] Requires VIRTIO_NET_F_GUEST_CSUM.
\item[VIRTIO_NET_F_GUEST_USO6] Requires VIRTIO_NET_F_GUEST_CSUM.
\item[VIRTIO_NET_F_GUEST_UDP_TUNNEL_GSO] Requires VIRTIO_NET_F_GUEST_TSO4, VIRTIO_NET_F_GUEST_TSO6,
   VIRTIO_NET_F_GUEST_USO4 and VIRTIO_NET_F_GUEST_USO6.
\item[VIRTIO_NET_F_GUEST_UDP_TUNNEL_GSO_CSUM] Requires VIRTIO_NET_F_GUEST_UDP_TUNNEL_GSO

\item[VIRTIO_NET_F_HOST_TSO4] Requires VIRTIO_NET_F_CSUM.
\item[VIRTIO_NET_F_HOST_TSO6] Requires VIRTIO_NET_F_CSUM.
\item[VIRTIO_NET_F_HOST_ECN] Requires VIRTIO_NET_F_HOST_TSO4 or VIRTIO_NET_F_HOST_TSO6.
\item[VIRTIO_NET_F_HOST_UFO] Requires VIRTIO_NET_F_CSUM.
\item[VIRTIO_NET_F_HOST_USO] Requires VIRTIO_NET_F_CSUM.
\item[VIRTIO_NET_F_HOST_UDP_TUNNEL_GSO] Requires VIRTIO_NET_F_HOST_TSO4, VIRTIO_NET_F_HOST_TSO6
   and VIRTIO_NET_F_HOST_USO.
\item[VIRTIO_NET_F_HOST_UDP_TUNNEL_GSO_CSUM] Requires VIRTIO_NET_F_HOST_UDP_TUNNEL_GSO

\item[VIRTIO_NET_F_CTRL_RX] Requires VIRTIO_NET_F_CTRL_VQ.
\item[VIRTIO_NET_F_CTRL_VLAN] Requires VIRTIO_NET_F_CTRL_VQ.
\item[VIRTIO_NET_F_GUEST_ANNOUNCE] Requires VIRTIO_NET_F_CTRL_VQ.
\item[VIRTIO_NET_F_MQ] Requires VIRTIO_NET_F_CTRL_VQ.
\item[VIRTIO_NET_F_CTRL_MAC_ADDR] Requires VIRTIO_NET_F_CTRL_VQ.
\item[VIRTIO_NET_F_NOTF_COAL] Requires VIRTIO_NET_F_CTRL_VQ.
\item[VIRTIO_NET_F_RSC_EXT] Requires VIRTIO_NET_F_HOST_TSO4 or VIRTIO_NET_F_HOST_TSO6.
\item[VIRTIO_NET_F_RSS] Requires VIRTIO_NET_F_CTRL_VQ.
\item[VIRTIO_NET_F_VQ_NOTF_COAL] Requires VIRTIO_NET_F_CTRL_VQ.
\item[VIRTIO_NET_F_HASH_TUNNEL] Requires VIRTIO_NET_F_CTRL_VQ along with VIRTIO_NET_F_RSS or VIRTIO_NET_F_HASH_REPORT.
\item[VIRTIO_NET_F_RSS_CONTEXT] Requires VIRTIO_NET_F_CTRL_VQ and VIRTIO_NET_F_RSS.
\end{description}

\begin{note}
The dependency between UDP_TUNNEL_GSO_CSUM and UDP_TUNNEL_GSO is intentionally
in the opposite direction with respect to the plain GSO features and the plain
checksum offload because UDP tunnel checksum offload gives very little gain
for non GSO packets and is quite complex to implement in H/W.
\end{note}

\subsubsection{Legacy Interface: Feature bits}\label{sec:Device Types / Network Device / Feature bits / Legacy Interface: Feature bits}
\begin{description}
\item[VIRTIO_NET_F_GSO (6)] Device handles packets with any GSO type. This was supposed to indicate segmentation offload support, but
upon further investigation it became clear that multiple bits were needed.
\item[VIRTIO_NET_F_GUEST_RSC4 (41)] Device coalesces TCPIP v4 packets. This was implemented by hypervisor patch for certification
purposes and current Windows driver depends on it. It will not function if virtio-net device reports this feature.
\item[VIRTIO_NET_F_GUEST_RSC6 (42)] Device coalesces TCPIP v6 packets. Similar to VIRTIO_NET_F_GUEST_RSC4.
\end{description}

\subsection{Device configuration layout}\label{sec:Device Types / Network Device / Device configuration layout}
\label{sec:Device Types / Block Device / Feature bits / Device configuration layout}

The network device has the following device configuration layout.
All of the device configuration fields are read-only for the driver.

\begin{lstlisting}
struct virtio_net_config {
        u8 mac[6];
        le16 status;
        le16 max_virtqueue_pairs;
        le16 mtu;
        le32 speed;
        u8 duplex;
        u8 rss_max_key_size;
        le16 rss_max_indirection_table_length;
        le32 supported_hash_types;
        le32 supported_tunnel_types;
};
\end{lstlisting}

The \field{mac} address field always exists (although it is only
valid if VIRTIO_NET_F_MAC is set).

The \field{status} only exists if VIRTIO_NET_F_STATUS is set.
Two bits are currently defined for the status field: VIRTIO_NET_S_LINK_UP
and VIRTIO_NET_S_ANNOUNCE.

\begin{lstlisting}
#define VIRTIO_NET_S_LINK_UP     1
#define VIRTIO_NET_S_ANNOUNCE    2
\end{lstlisting}

The following field, \field{max_virtqueue_pairs} only exists if
VIRTIO_NET_F_MQ or VIRTIO_NET_F_RSS is set. This field specifies the maximum number
of each of transmit and receive virtqueues (receiveq1\ldots receiveqN
and transmitq1\ldots transmitqN respectively) that can be configured once at least one of these features
is negotiated.

The following field, \field{mtu} only exists if VIRTIO_NET_F_MTU
is set. This field specifies the maximum MTU for the driver to
use.

The following two fields, \field{speed} and \field{duplex}, only
exist if VIRTIO_NET_F_SPEED_DUPLEX is set.

\field{speed} contains the device speed, in units of 1 MBit per
second, 0 to 0x7fffffff, or 0xffffffff for unknown speed.

\field{duplex} has the values of 0x01 for full duplex, 0x00 for
half duplex and 0xff for unknown duplex state.

Both \field{speed} and \field{duplex} can change, thus the driver
is expected to re-read these values after receiving a
configuration change notification.

The following field, \field{rss_max_key_size} only exists if VIRTIO_NET_F_RSS or VIRTIO_NET_F_HASH_REPORT is set.
It specifies the maximum supported length of RSS key in bytes.

The following field, \field{rss_max_indirection_table_length} only exists if VIRTIO_NET_F_RSS is set.
It specifies the maximum number of 16-bit entries in RSS indirection table.

The next field, \field{supported_hash_types} only exists if the device supports hash calculation,
i.e. if VIRTIO_NET_F_RSS or VIRTIO_NET_F_HASH_REPORT is set.

Field \field{supported_hash_types} contains the bitmask of supported hash types.
See \ref{sec:Device Types / Network Device / Device Operation / Processing of Incoming Packets / Hash calculation for incoming packets / Supported/enabled hash types} for details of supported hash types.

Field \field{supported_tunnel_types} only exists if the device supports inner header hash, i.e. if VIRTIO_NET_F_HASH_TUNNEL is set.

Field \field{supported_tunnel_types} contains the bitmask of encapsulation types supported by the device for inner header hash.
Encapsulation types are defined in \ref{sec:Device Types / Network Device / Device Operation / Processing of Incoming Packets /
Hash calculation for incoming packets / Encapsulation types supported/enabled for inner header hash}.

\devicenormative{\subsubsection}{Device configuration layout}{Device Types / Network Device / Device configuration layout}

The device MUST set \field{max_virtqueue_pairs} to between 1 and 0x8000 inclusive,
if it offers VIRTIO_NET_F_MQ.

The device MUST set \field{mtu} to between 68 and 65535 inclusive,
if it offers VIRTIO_NET_F_MTU.

The device SHOULD set \field{mtu} to at least 1280, if it offers
VIRTIO_NET_F_MTU.

The device MUST NOT modify \field{mtu} once it has been set.

The device MUST NOT pass received packets that exceed \field{mtu} (plus low
level ethernet header length) size with \field{gso_type} NONE or ECN
after VIRTIO_NET_F_MTU has been successfully negotiated.

The device MUST forward transmitted packets of up to \field{mtu} (plus low
level ethernet header length) size with \field{gso_type} NONE or ECN, and do
so without fragmentation, after VIRTIO_NET_F_MTU has been successfully
negotiated.

The device MUST set \field{rss_max_key_size} to at least 40, if it offers
VIRTIO_NET_F_RSS or VIRTIO_NET_F_HASH_REPORT.

The device MUST set \field{rss_max_indirection_table_length} to at least 128, if it offers
VIRTIO_NET_F_RSS.

If the driver negotiates the VIRTIO_NET_F_STANDBY feature, the device MAY act
as a standby device for a primary device with the same MAC address.

If VIRTIO_NET_F_SPEED_DUPLEX has been negotiated, \field{speed}
MUST contain the device speed, in units of 1 MBit per second, 0 to
0x7ffffffff, or 0xfffffffff for unknown.

If VIRTIO_NET_F_SPEED_DUPLEX has been negotiated, \field{duplex}
MUST have the values of 0x00 for full duplex, 0x01 for half
duplex, or 0xff for unknown.

If VIRTIO_NET_F_SPEED_DUPLEX and VIRTIO_NET_F_STATUS have both
been negotiated, the device SHOULD NOT change the \field{speed} and
\field{duplex} fields as long as VIRTIO_NET_S_LINK_UP is set in
the \field{status}.

The device SHOULD NOT offer VIRTIO_NET_F_HASH_REPORT if it
does not offer VIRTIO_NET_F_CTRL_VQ.

The device SHOULD NOT offer VIRTIO_NET_F_CTRL_RX_EXTRA if it
does not offer VIRTIO_NET_F_CTRL_VQ.

\drivernormative{\subsubsection}{Device configuration layout}{Device Types / Network Device / Device configuration layout}

The driver MUST NOT write to any of the device configuration fields.

A driver SHOULD negotiate VIRTIO_NET_F_MAC if the device offers it.
If the driver negotiates the VIRTIO_NET_F_MAC feature, the driver MUST set
the physical address of the NIC to \field{mac}.  Otherwise, it SHOULD
use a locally-administered MAC address (see \hyperref[intro:IEEE 802]{IEEE 802},
``9.2 48-bit universal LAN MAC addresses'').

If the driver does not negotiate the VIRTIO_NET_F_STATUS feature, it SHOULD
assume the link is active, otherwise it SHOULD read the link status from
the bottom bit of \field{status}.

A driver SHOULD negotiate VIRTIO_NET_F_MTU if the device offers it.

If the driver negotiates VIRTIO_NET_F_MTU, it MUST supply enough receive
buffers to receive at least one receive packet of size \field{mtu} (plus low
level ethernet header length) with \field{gso_type} NONE or ECN.

If the driver negotiates VIRTIO_NET_F_MTU, it MUST NOT transmit packets of
size exceeding the value of \field{mtu} (plus low level ethernet header length)
with \field{gso_type} NONE or ECN.

A driver SHOULD negotiate the VIRTIO_NET_F_STANDBY feature if the device offers it.

If VIRTIO_NET_F_SPEED_DUPLEX has been negotiated,
the driver MUST treat any value of \field{speed} above
0x7fffffff as well as any value of \field{duplex} not
matching 0x00 or 0x01 as an unknown value.

If VIRTIO_NET_F_SPEED_DUPLEX has been negotiated, the driver
SHOULD re-read \field{speed} and \field{duplex} after a
configuration change notification.

A driver SHOULD NOT negotiate VIRTIO_NET_F_HASH_REPORT if it
does not negotiate VIRTIO_NET_F_CTRL_VQ.

A driver SHOULD NOT negotiate VIRTIO_NET_F_CTRL_RX_EXTRA if it
does not negotiate VIRTIO_NET_F_CTRL_VQ.

\subsubsection{Legacy Interface: Device configuration layout}\label{sec:Device Types / Network Device / Device configuration layout / Legacy Interface: Device configuration layout}
\label{sec:Device Types / Block Device / Feature bits / Device configuration layout / Legacy Interface: Device configuration layout}
When using the legacy interface, transitional devices and drivers
MUST format \field{status} and
\field{max_virtqueue_pairs} in struct virtio_net_config
according to the native endian of the guest rather than
(necessarily when not using the legacy interface) little-endian.

When using the legacy interface, \field{mac} is driver-writable
which provided a way for drivers to update the MAC without
negotiating VIRTIO_NET_F_CTRL_MAC_ADDR.

\subsection{Device Initialization}\label{sec:Device Types / Network Device / Device Initialization}

A driver would perform a typical initialization routine like so:

\begin{enumerate}
\item Identify and initialize the receive and
  transmission virtqueues, up to N of each kind. If
  VIRTIO_NET_F_MQ feature bit is negotiated,
  N=\field{max_virtqueue_pairs}, otherwise identify N=1.

\item If the VIRTIO_NET_F_CTRL_VQ feature bit is negotiated,
  identify the control virtqueue.

\item Fill the receive queues with buffers: see \ref{sec:Device Types / Network Device / Device Operation / Setting Up Receive Buffers}.

\item Even with VIRTIO_NET_F_MQ, only receiveq1, transmitq1 and
  controlq are used by default.  The driver would send the
  VIRTIO_NET_CTRL_MQ_VQ_PAIRS_SET command specifying the
  number of the transmit and receive queues to use.

\item If the VIRTIO_NET_F_MAC feature bit is set, the configuration
  space \field{mac} entry indicates the ``physical'' address of the
  device, otherwise the driver would typically generate a random
  local MAC address.

\item If the VIRTIO_NET_F_STATUS feature bit is negotiated, the link
  status comes from the bottom bit of \field{status}.
  Otherwise, the driver assumes it's active.

\item A performant driver would indicate that it will generate checksumless
  packets by negotiating the VIRTIO_NET_F_CSUM feature.

\item If that feature is negotiated, a driver can use TCP segmentation or UDP
  segmentation/fragmentation offload by negotiating the VIRTIO_NET_F_HOST_TSO4 (IPv4
  TCP), VIRTIO_NET_F_HOST_TSO6 (IPv6 TCP), VIRTIO_NET_F_HOST_UFO
  (UDP fragmentation) and VIRTIO_NET_F_HOST_USO (UDP segmentation) features.

\item If the VIRTIO_NET_F_HOST_TSO6, VIRTIO_NET_F_HOST_TSO4 and VIRTIO_NET_F_HOST_USO
  segmentation features are negotiated, a driver can
  use TCP segmentation or UDP segmentation on top of UDP encapsulation
  offload, when the outer header does not require checksumming - e.g.
  the outer UDP checksum is zero - by negotiating the
  VIRTIO_NET_F_HOST_UDP_TUNNEL_GSO feature.
  GSO over UDP tunnels packets carry two sets of headers: the outer ones
  and the inner ones. The outer transport protocol is UDP, the inner
  could be either TCP or UDP. Only a single level of encapsulation
  offload is supported.

\item If VIRTIO_NET_F_HOST_UDP_TUNNEL_GSO is negotiated, a driver can
  additionally use TCP segmentation or UDP segmentation on top of UDP
  encapsulation with the outer header requiring checksum offload,
  negotiating the VIRTIO_NET_F_HOST_UDP_TUNNEL_GSO_CSUM feature.

\item The converse features are also available: a driver can save
  the virtual device some work by negotiating these features.\note{For example, a network packet transported between two guests on
the same system might not need checksumming at all, nor segmentation,
if both guests are amenable.}
   The VIRTIO_NET_F_GUEST_CSUM feature indicates that partially
  checksummed packets can be received, and if it can do that then
  the VIRTIO_NET_F_GUEST_TSO4, VIRTIO_NET_F_GUEST_TSO6,
  VIRTIO_NET_F_GUEST_UFO, VIRTIO_NET_F_GUEST_ECN, VIRTIO_NET_F_GUEST_USO4,
  VIRTIO_NET_F_GUEST_USO6 VIRTIO_NET_F_GUEST_UDP_TUNNEL_GSO and
  VIRTIO_NET_F_GUEST_UDP_TUNNEL_GSO_CSUM are the input equivalents of
  the features described above.
  See \ref{sec:Device Types / Network Device / Device Operation /
Setting Up Receive Buffers}~\nameref{sec:Device Types / Network
Device / Device Operation / Setting Up Receive Buffers} and
\ref{sec:Device Types / Network Device / Device Operation /
Processing of Incoming Packets}~\nameref{sec:Device Types /
Network Device / Device Operation / Processing of Incoming Packets} below.
\end{enumerate}

A truly minimal driver would only accept VIRTIO_NET_F_MAC and ignore
everything else.

\subsection{Device and driver capabilities}\label{sec:Device Types / Network Device / Device and driver capabilities}

The network device has the following capabilities.

\begin{tabularx}{\textwidth}{ |l||l|X| }
\hline
Identifier & Name & Description \\
\hline \hline
0x0800 & \hyperref[par:Device Types / Network Device / Device Operation / Flow filter / Device and driver capabilities / VIRTIO-NET-FF-RESOURCE-CAP]{VIRTIO_NET_FF_RESOURCE_CAP} & Flow filter resource capability \\
\hline
0x0801 & \hyperref[par:Device Types / Network Device / Device Operation / Flow filter / Device and driver capabilities / VIRTIO-NET-FF-SELECTOR-CAP]{VIRTIO_NET_FF_SELECTOR_CAP} & Flow filter classifier capability \\
\hline
0x0802 & \hyperref[par:Device Types / Network Device / Device Operation / Flow filter / Device and driver capabilities / VIRTIO-NET-FF-ACTION-CAP]{VIRTIO_NET_FF_ACTION_CAP} & Flow filter action capability \\
\hline
\end{tabularx}

\subsection{Device resource objects}\label{sec:Device Types / Network Device / Device resource objects}

The network device has the following resource objects.

\begin{tabularx}{\textwidth}{ |l||l|X| }
\hline
type & Name & Description \\
\hline \hline
0x0200 & \hyperref[par:Device Types / Network Device / Device Operation / Flow filter / Resource objects / VIRTIO-NET-RESOURCE-OBJ-FF-GROUP]{VIRTIO_NET_RESOURCE_OBJ_FF_GROUP} & Flow filter group resource object \\
\hline
0x0201 & \hyperref[par:Device Types / Network Device / Device Operation / Flow filter / Resource objects / VIRTIO-NET-RESOURCE-OBJ-FF-CLASSIFIER]{VIRTIO_NET_RESOURCE_OBJ_FF_CLASSIFIER} & Flow filter mask object \\
\hline
0x0202 & \hyperref[par:Device Types / Network Device / Device Operation / Flow filter / Resource objects / VIRTIO-NET-RESOURCE-OBJ-FF-RULE]{VIRTIO_NET_RESOURCE_OBJ_FF_RULE} & Flow filter rule object \\
\hline
\end{tabularx}

\subsection{Device parts}\label{sec:Device Types / Network Device / Device parts}

Network device parts represent the configuration done by the driver using control
virtqueue commands. Network device part is in the format of
\field{struct virtio_dev_part}.

\begin{tabularx}{\textwidth}{ |l||l|X| }
\hline
Type & Name & Description \\
\hline \hline
0x200 & VIRTIO_NET_DEV_PART_CVQ_CFG_PART & Represents device configuration done through a control virtqueue command, see \ref{sec:Device Types / Network Device / Device parts / VIRTIO-NET-DEV-PART-CVQ-CFG-PART} \\
\hline
0x201 - 0x5FF & - & reserved for future \\
\hline
\hline
\end{tabularx}

\subsubsection{VIRTIO_NET_DEV_PART_CVQ_CFG_PART}\label{sec:Device Types / Network Device / Device parts / VIRTIO-NET-DEV-PART-CVQ-CFG-PART}

For VIRTIO_NET_DEV_PART_CVQ_CFG_PART, \field{part_type} is set to 0x200. The
VIRTIO_NET_DEV_PART_CVQ_CFG_PART part indicates configuration performed by the
driver using a control virtqueue command.

\begin{lstlisting}
struct virtio_net_dev_part_cvq_selector {
        u8 class;
        u8 command;
        u8 reserved[6];
};
\end{lstlisting}

There is one device part of type VIRTIO_NET_DEV_PART_CVQ_CFG_PART for each
individual configuration. Each part is identified by a unique selector value.
The selector, \field{device_type_raw}, is in the format
\field{struct virtio_net_dev_part_cvq_selector}.

The selector consists of two fields: \field{class} and \field{command}. These
fields correspond to the \field{class} and \field{command} defined in
\field{struct virtio_net_ctrl}, as described in the relevant sections of
\ref{sec:Device Types / Network Device / Device Operation / Control Virtqueue}.

The value corresponding to each part’s selector follows the same format as the
respective \field{command-specific-data} described in the relevant sections of
\ref{sec:Device Types / Network Device / Device Operation / Control Virtqueue}.

For example, when the \field{class} is VIRTIO_NET_CTRL_MAC, the \field{command}
can be either VIRTIO_NET_CTRL_MAC_TABLE_SET or VIRTIO_NET_CTRL_MAC_ADDR_SET;
when \field{command} is set to VIRTIO_NET_CTRL_MAC_TABLE_SET, \field{value}
is in the format of \field{struct virtio_net_ctrl_mac}.

Supported selectors are listed in the table:

\begin{tabularx}{\textwidth}{ |l|X| }
\hline
Class selector & Command selector \\
\hline \hline
VIRTIO_NET_CTRL_RX & VIRTIO_NET_CTRL_RX_PROMISC \\
\hline
VIRTIO_NET_CTRL_RX & VIRTIO_NET_CTRL_RX_ALLMULTI \\
\hline
VIRTIO_NET_CTRL_RX & VIRTIO_NET_CTRL_RX_ALLUNI \\
\hline
VIRTIO_NET_CTRL_RX & VIRTIO_NET_CTRL_RX_NOMULTI \\
\hline
VIRTIO_NET_CTRL_RX & VIRTIO_NET_CTRL_RX_NOUNI \\
\hline
VIRTIO_NET_CTRL_RX & VIRTIO_NET_CTRL_RX_NOBCAST \\
\hline
VIRTIO_NET_CTRL_MAC & VIRTIO_NET_CTRL_MAC_TABLE_SET \\
\hline
VIRTIO_NET_CTRL_MAC & VIRTIO_NET_CTRL_MAC_ADDR_SET \\
\hline
VIRTIO_NET_CTRL_VLAN & VIRTIO_NET_CTRL_VLAN_ADD \\
\hline
VIRTIO_NET_CTRL_ANNOUNCE & VIRTIO_NET_CTRL_ANNOUNCE_ACK \\
\hline
VIRTIO_NET_CTRL_MQ & VIRTIO_NET_CTRL_MQ_VQ_PAIRS_SET \\
\hline
VIRTIO_NET_CTRL_MQ & VIRTIO_NET_CTRL_MQ_RSS_CONFIG \\
\hline
VIRTIO_NET_CTRL_MQ & VIRTIO_NET_CTRL_MQ_HASH_CONFIG \\
\hline
\hline
\end{tabularx}

For command selector VIRTIO_NET_CTRL_VLAN_ADD, device part consists of a whole
VLAN table.

\field{reserved} is reserved and set to zero.

\subsection{Device Operation}\label{sec:Device Types / Network Device / Device Operation}

Packets are transmitted by placing them in the
transmitq1\ldots transmitqN, and buffers for incoming packets are
placed in the receiveq1\ldots receiveqN. In each case, the packet
itself is preceded by a header:

\begin{lstlisting}
struct virtio_net_hdr {
#define VIRTIO_NET_HDR_F_NEEDS_CSUM    1
#define VIRTIO_NET_HDR_F_DATA_VALID    2
#define VIRTIO_NET_HDR_F_RSC_INFO      4
#define VIRTIO_NET_HDR_F_UDP_TUNNEL_CSUM 8
        u8 flags;
#define VIRTIO_NET_HDR_GSO_NONE        0
#define VIRTIO_NET_HDR_GSO_TCPV4       1
#define VIRTIO_NET_HDR_GSO_UDP         3
#define VIRTIO_NET_HDR_GSO_TCPV6       4
#define VIRTIO_NET_HDR_GSO_UDP_L4      5
#define VIRTIO_NET_HDR_GSO_UDP_TUNNEL_IPV4 0x20
#define VIRTIO_NET_HDR_GSO_UDP_TUNNEL_IPV6 0x40
#define VIRTIO_NET_HDR_GSO_ECN      0x80
        u8 gso_type;
        le16 hdr_len;
        le16 gso_size;
        le16 csum_start;
        le16 csum_offset;
        le16 num_buffers;
        le32 hash_value;        (Only if VIRTIO_NET_F_HASH_REPORT negotiated)
        le16 hash_report;       (Only if VIRTIO_NET_F_HASH_REPORT negotiated)
        le16 padding_reserved;  (Only if VIRTIO_NET_F_HASH_REPORT negotiated)
        le16 outer_th_offset    (Only if VIRTIO_NET_F_HOST_UDP_TUNNEL_GSO or VIRTIO_NET_F_GUEST_UDP_TUNNEL_GSO negotiated)
        le16 inner_nh_offset;   (Only if VIRTIO_NET_F_HOST_UDP_TUNNEL_GSO or VIRTIO_NET_F_GUEST_UDP_TUNNEL_GSO negotiated)
};
\end{lstlisting}

The controlq is used to control various device features described further in
section \ref{sec:Device Types / Network Device / Device Operation / Control Virtqueue}.

\subsubsection{Legacy Interface: Device Operation}\label{sec:Device Types / Network Device / Device Operation / Legacy Interface: Device Operation}
When using the legacy interface, transitional devices and drivers
MUST format the fields in \field{struct virtio_net_hdr}
according to the native endian of the guest rather than
(necessarily when not using the legacy interface) little-endian.

The legacy driver only presented \field{num_buffers} in the \field{struct virtio_net_hdr}
when VIRTIO_NET_F_MRG_RXBUF was negotiated; without that feature the
structure was 2 bytes shorter.

When using the legacy interface, the driver SHOULD ignore the
used length for the transmit queues
and the controlq queue.
\begin{note}
Historically, some devices put
the total descriptor length there, even though no data was
actually written.
\end{note}

\subsubsection{Packet Transmission}\label{sec:Device Types / Network Device / Device Operation / Packet Transmission}

Transmitting a single packet is simple, but varies depending on
the different features the driver negotiated.

\begin{enumerate}
\item The driver can send a completely checksummed packet.  In this case,
  \field{flags} will be zero, and \field{gso_type} will be VIRTIO_NET_HDR_GSO_NONE.

\item If the driver negotiated VIRTIO_NET_F_CSUM, it can skip
  checksumming the packet:
  \begin{itemize}
  \item \field{flags} has the VIRTIO_NET_HDR_F_NEEDS_CSUM set,

  \item \field{csum_start} is set to the offset within the packet to begin checksumming,
    and

  \item \field{csum_offset} indicates how many bytes after the csum_start the
    new (16 bit ones' complement) checksum is placed by the device.

  \item The TCP checksum field in the packet is set to the sum
    of the TCP pseudo header, so that replacing it by the ones'
    complement checksum of the TCP header and body will give the
    correct result.
  \end{itemize}

\begin{note}
For example, consider a partially checksummed TCP (IPv4) packet.
It will have a 14 byte ethernet header and 20 byte IP header
followed by the TCP header (with the TCP checksum field 16 bytes
into that header). \field{csum_start} will be 14+20 = 34 (the TCP
checksum includes the header), and \field{csum_offset} will be 16.
If the given packet has the VIRTIO_NET_HDR_GSO_UDP_TUNNEL_IPV4 bit or the
VIRTIO_NET_HDR_GSO_UDP_TUNNEL_IPV6 bit set,
the above checksum fields refer to the inner header checksum, see
the example below.
\end{note}

\item If the driver negotiated
  VIRTIO_NET_F_HOST_TSO4, TSO6, USO or UFO, and the packet requires
  TCP segmentation, UDP segmentation or fragmentation, then \field{gso_type}
  is set to VIRTIO_NET_HDR_GSO_TCPV4, TCPV6, UDP_L4 or UDP.
  (Otherwise, it is set to VIRTIO_NET_HDR_GSO_NONE). In this
  case, packets larger than 1514 bytes can be transmitted: the
  metadata indicates how to replicate the packet header to cut it
  into smaller packets. The other gso fields are set:

  \begin{itemize}
  \item If the VIRTIO_NET_F_GUEST_HDRLEN feature has been negotiated,
    \field{hdr_len} indicates the header length that needs to be replicated
    for each packet. It's the number of bytes from the beginning of the packet
    to the beginning of the transport payload.
    If the \field{gso_type} has the VIRTIO_NET_HDR_GSO_UDP_TUNNEL_IPV4 bit or
    VIRTIO_NET_HDR_GSO_UDP_TUNNEL_IPV6 bit set, \field{hdr_len} accounts for
    all the headers up to and including the inner transport.
    Otherwise, if the VIRTIO_NET_F_GUEST_HDRLEN feature has not been negotiated,
    \field{hdr_len} is a hint to the device as to how much of the header
    needs to be kept to copy into each packet, usually set to the
    length of the headers, including the transport header\footnote{Due to various bugs in implementations, this field is not useful
as a guarantee of the transport header size.
}.

  \begin{note}
  Some devices benefit from knowledge of the exact header length.
  \end{note}

  \item \field{gso_size} is the maximum size of each packet beyond that
    header (ie. MSS).

  \item If the driver negotiated the VIRTIO_NET_F_HOST_ECN feature,
    the VIRTIO_NET_HDR_GSO_ECN bit in \field{gso_type}
    indicates that the TCP packet has the ECN bit set\footnote{This case is not handled by some older hardware, so is called out
specifically in the protocol.}.
   \end{itemize}

\item If the driver negotiated the VIRTIO_NET_F_HOST_UDP_TUNNEL_GSO feature and the
  \field{gso_type} has the VIRTIO_NET_HDR_GSO_UDP_TUNNEL_IPV4 bit or
  VIRTIO_NET_HDR_GSO_UDP_TUNNEL_IPV6 bit set, the GSO protocol is encapsulated
  in a UDP tunnel.
  If the outer UDP header requires checksumming, the driver must have
  additionally negotiated the VIRTIO_NET_F_HOST_UDP_TUNNEL_GSO_CSUM feature
  and offloaded the outer checksum accordingly, otherwise
  the outer UDP header must not require checksum validation, i.e. the outer
  UDP checksum must be positive zero (0x0) as defined in UDP RFC 768.
  The other tunnel-related fields indicate how to replicate the packet
  headers to cut it into smaller packets:

  \begin{itemize}
  \item \field{outer_th_offset} field indicates the outer transport header within
      the packet. This field differs from \field{csum_start} as the latter
      points to the inner transport header within the packet.

  \item \field{inner_nh_offset} field indicates the inner network header within
      the packet.
  \end{itemize}

\begin{note}
For example, consider a partially checksummed TCP (IPv4) packet carried over a
Geneve UDP tunnel (again IPv4) with no tunnel options. The
only relevant variable related to the tunnel type is the tunnel header length.
The packet will have a 14 byte outer ethernet header, 20 byte outer IP header
followed by the 8 byte UDP header (with a 0 checksum value), 8 byte Geneve header,
14 byte inner ethernet header, 20 byte inner IP header
and the TCP header (with the TCP checksum field 16 bytes
into that header). \field{csum_start} will be 14+20+8+8+14+20 = 84 (the TCP
checksum includes the header), \field{csum_offset} will be 16.
\field{inner_nh_offset} will be 14+20+8+8+14 = 62, \field{outer_th_offset} will be
14+20+8 = 42 and \field{gso_type} will be
VIRTIO_NET_HDR_GSO_TCPV4 | VIRTIO_NET_HDR_GSO_UDP_TUNNEL_IPV4 = 0x21
\end{note}

\item If the driver negotiated the VIRTIO_NET_F_HOST_UDP_TUNNEL_GSO_CSUM feature,
  the transmitted packet is a GSO one encapsulated in a UDP tunnel, and
  the outer UDP header requires checksumming, the driver can skip checksumming
  the outer header:

  \begin{itemize}
  \item \field{flags} has the VIRTIO_NET_HDR_F_UDP_TUNNEL_CSUM set,

  \item The outer UDP checksum field in the packet is set to the sum
    of the UDP pseudo header, so that replacing it by the ones'
    complement checksum of the outer UDP header and payload will give the
    correct result.
  \end{itemize}

\item \field{num_buffers} is set to zero.  This field is unused on transmitted packets.

\item The header and packet are added as one output descriptor to the
  transmitq, and the device is notified of the new entry
  (see \ref{sec:Device Types / Network Device / Device Initialization}~\nameref{sec:Device Types / Network Device / Device Initialization}).
\end{enumerate}

\drivernormative{\paragraph}{Packet Transmission}{Device Types / Network Device / Device Operation / Packet Transmission}

For the transmit packet buffer, the driver MUST use the size of the
structure \field{struct virtio_net_hdr} same as the receive packet buffer.

The driver MUST set \field{num_buffers} to zero.

If VIRTIO_NET_F_CSUM is not negotiated, the driver MUST set
\field{flags} to zero and SHOULD supply a fully checksummed
packet to the device.

If VIRTIO_NET_F_HOST_TSO4 is negotiated, the driver MAY set
\field{gso_type} to VIRTIO_NET_HDR_GSO_TCPV4 to request TCPv4
segmentation, otherwise the driver MUST NOT set
\field{gso_type} to VIRTIO_NET_HDR_GSO_TCPV4.

If VIRTIO_NET_F_HOST_TSO6 is negotiated, the driver MAY set
\field{gso_type} to VIRTIO_NET_HDR_GSO_TCPV6 to request TCPv6
segmentation, otherwise the driver MUST NOT set
\field{gso_type} to VIRTIO_NET_HDR_GSO_TCPV6.

If VIRTIO_NET_F_HOST_UFO is negotiated, the driver MAY set
\field{gso_type} to VIRTIO_NET_HDR_GSO_UDP to request UDP
fragmentation, otherwise the driver MUST NOT set
\field{gso_type} to VIRTIO_NET_HDR_GSO_UDP.

If VIRTIO_NET_F_HOST_USO is negotiated, the driver MAY set
\field{gso_type} to VIRTIO_NET_HDR_GSO_UDP_L4 to request UDP
segmentation, otherwise the driver MUST NOT set
\field{gso_type} to VIRTIO_NET_HDR_GSO_UDP_L4.

The driver SHOULD NOT send to the device TCP packets requiring segmentation offload
which have the Explicit Congestion Notification bit set, unless the
VIRTIO_NET_F_HOST_ECN feature is negotiated, in which case the
driver MUST set the VIRTIO_NET_HDR_GSO_ECN bit in
\field{gso_type}.

If VIRTIO_NET_F_HOST_UDP_TUNNEL_GSO is negotiated, the driver MAY set
VIRTIO_NET_HDR_GSO_UDP_TUNNEL_IPV4 bit or the VIRTIO_NET_HDR_GSO_UDP_TUNNEL_IPV6 bit
in \field{gso_type} according to the inner network header protocol type
to request GSO packets over UDPv4 or UDPv6 tunnel segmentation,
otherwise the driver MUST NOT set either the
VIRTIO_NET_HDR_GSO_UDP_TUNNEL_IPV4 bit or the VIRTIO_NET_HDR_GSO_UDP_TUNNEL_IPV6 bit
in \field{gso_type}.

When requesting GSO segmentation over UDP tunnel, the driver MUST SET the
VIRTIO_NET_HDR_GSO_UDP_TUNNEL_IPV4 bit if the inner network header is IPv4, i.e. the
packet is a TCPv4 GSO one, otherwise, if the inner network header is IPv6, the driver
MUST SET the VIRTIO_NET_HDR_GSO_UDP_TUNNEL_IPV6 bit.

The driver MUST NOT send to the device GSO packets over UDP tunnel
requiring segmentation and outer UDP checksum offload, unless both the
VIRTIO_NET_F_HOST_UDP_TUNNEL_GSO and VIRTIO_NET_F_HOST_UDP_TUNNEL_GSO_CSUM features
are negotiated, in which case the driver MUST set either the
VIRTIO_NET_HDR_GSO_UDP_TUNNEL_IPV4 bit or the VIRTIO_NET_HDR_GSO_UDP_TUNNEL_IPV6
bit in the \field{gso_type} and the VIRTIO_NET_HDR_F_UDP_TUNNEL_CSUM bit in
the \field{flags}.

If VIRTIO_NET_F_HOST_UDP_TUNNEL_GSO_CSUM is not negotiated, the driver MUST not set
the VIRTIO_NET_HDR_F_UDP_TUNNEL_CSUM bit in the \field{flags} and
MUST NOT send to the device GSO packets over UDP tunnel
requiring segmentation and outer UDP checksum offload.

The driver MUST NOT set the VIRTIO_NET_HDR_GSO_UDP_TUNNEL_IPV4 bit or the
VIRTIO_NET_HDR_GSO_UDP_TUNNEL_IPV6 bit together with VIRTIO_NET_HDR_GSO_UDP, as the
latter is deprecated in favor of UDP_L4 and no new feature will support it.

The driver MUST NOT set the VIRTIO_NET_HDR_GSO_UDP_TUNNEL_IPV4 bit and the
VIRTIO_NET_HDR_GSO_UDP_TUNNEL_IPV6 bit together.

The driver MUST NOT set the VIRTIO_NET_HDR_F_UDP_TUNNEL_CSUM bit \field{flags}
without setting either the VIRTIO_NET_HDR_GSO_UDP_TUNNEL_IPV4 bit or
the VIRTIO_NET_HDR_GSO_UDP_TUNNEL_IPV6 bit in \field{gso_type}.

If the VIRTIO_NET_F_CSUM feature has been negotiated, the
driver MAY set the VIRTIO_NET_HDR_F_NEEDS_CSUM bit in
\field{flags}, if so:
\begin{enumerate}
\item the driver MUST validate the packet checksum at
	offset \field{csum_offset} from \field{csum_start} as well as all
	preceding offsets;
\begin{note}
If \field{gso_type} differs from VIRTIO_NET_HDR_GSO_NONE and the
VIRTIO_NET_HDR_GSO_UDP_TUNNEL_IPV4 bit or the VIRTIO_NET_HDR_GSO_UDP_TUNNEL_IPV6
bit are not set in \field{gso_type}, \field{csum_offset}
points to the only transport header present in the packet, and there are no
additional preceding checksums validated by VIRTIO_NET_HDR_F_NEEDS_CSUM.
\end{note}
\item the driver MUST set the packet checksum stored in the
	buffer to the TCP/UDP pseudo header;
\item the driver MUST set \field{csum_start} and
	\field{csum_offset} such that calculating a ones'
	complement checksum from \field{csum_start} up until the end of
	the packet and storing the result at offset \field{csum_offset}
	from  \field{csum_start} will result in a fully checksummed
	packet;
\end{enumerate}

If none of the VIRTIO_NET_F_HOST_TSO4, TSO6, USO or UFO options have
been negotiated, the driver MUST set \field{gso_type} to
VIRTIO_NET_HDR_GSO_NONE.

If \field{gso_type} differs from VIRTIO_NET_HDR_GSO_NONE, then
the driver MUST also set the VIRTIO_NET_HDR_F_NEEDS_CSUM bit in
\field{flags} and MUST set \field{gso_size} to indicate the
desired MSS.

If one of the VIRTIO_NET_F_HOST_TSO4, TSO6, USO or UFO options have
been negotiated:
\begin{itemize}
\item If the VIRTIO_NET_F_GUEST_HDRLEN feature has been negotiated,
	and \field{gso_type} differs from VIRTIO_NET_HDR_GSO_NONE,
	the driver MUST set \field{hdr_len} to a value equal to the length
	of the headers, including the transport header. If \field{gso_type}
	has the VIRTIO_NET_HDR_GSO_UDP_TUNNEL_IPV4 bit or the
	VIRTIO_NET_HDR_GSO_UDP_TUNNEL_IPV6 bit set, \field{hdr_len} includes
	the inner transport header.

\item If the VIRTIO_NET_F_GUEST_HDRLEN feature has not been negotiated,
	or \field{gso_type} is VIRTIO_NET_HDR_GSO_NONE,
	the driver SHOULD set \field{hdr_len} to a value
	not less than the length of the headers, including the transport
	header.
\end{itemize}

If the VIRTIO_NET_F_HOST_UDP_TUNNEL_GSO option has been negotiated, the
driver MAY set the VIRTIO_NET_HDR_GSO_UDP_TUNNEL_IPV4 bit or the
VIRTIO_NET_HDR_GSO_UDP_TUNNEL_IPV6 bit in \field{gso_type}, if so:
\begin{itemize}
\item the driver MUST set \field{outer_th_offset} to the outer UDP header
  offset and \field{inner_nh_offset} to the inner network header offset.
  The \field{csum_start} and \field{csum_offset} fields point respectively
  to the inner transport header and inner transport checksum field.
\end{itemize}

If the VIRTIO_NET_F_HOST_UDP_TUNNEL_GSO_CSUM feature has been negotiated,
and the VIRTIO_NET_HDR_GSO_UDP_TUNNEL_IPV4 bit or
VIRTIO_NET_HDR_GSO_UDP_TUNNEL_IPV6 bit in \field{gso_type} are set,
the driver MAY set the VIRTIO_NET_HDR_F_UDP_TUNNEL_CSUM bit in
\field{flags}, if so the driver MUST set the packet outer UDP header checksum
to the outer UDP pseudo header checksum.

\begin{note}
calculating a ones' complement checksum from \field{outer_th_offset}
up until the end of the packet and storing the result at offset 6
from \field{outer_th_offset} will result in a fully checksummed outer UDP packet;
\end{note}

If the VIRTIO_NET_HDR_GSO_UDP_TUNNEL_IPV4 bit or the
VIRTIO_NET_HDR_GSO_UDP_TUNNEL_IPV6 bit in \field{gso_type} are set
and the VIRTIO_NET_F_HOST_UDP_TUNNEL_GSO_CSUM feature has not
been negotiated, the
outer UDP header MUST NOT require checksum validation. That is, the
outer UDP checksum value MUST be 0 or the validated complete checksum
for such header.

\begin{note}
The valid complete checksum of the outer UDP header of individual segments
can be computed by the driver prior to segmentation only if the GSO packet
size is a multiple of \field{gso_size}, because then all segments
have the same size and thus all data included in the outer UDP
checksum is the same for every segment. These pre-computed segment
length and checksum fields are different from those of the GSO
packet.
In this scenario the outer UDP header of the GSO packet must carry the
segmented UDP packet length.
\end{note}

If the VIRTIO_NET_F_HOST_UDP_TUNNEL_GSO option has not
been negotiated, the driver MUST NOT set either the VIRTIO_NET_HDR_F_GSO_UDP_TUNNEL_IPV4
bit or the VIRTIO_NET_HDR_F_GSO_UDP_TUNNEL_IPV6 in \field{gso_type}.

If the VIRTIO_NET_F_HOST_UDP_TUNNEL_GSO_CSUM option has not been negotiated,
the driver MUST NOT set the VIRTIO_NET_HDR_F_UDP_TUNNEL_CSUM bit
in \field{flags}.

The driver SHOULD accept the VIRTIO_NET_F_GUEST_HDRLEN feature if it has
been offered, and if it's able to provide the exact header length.

The driver MUST NOT set the VIRTIO_NET_HDR_F_DATA_VALID and
VIRTIO_NET_HDR_F_RSC_INFO bits in \field{flags}.

The driver MUST NOT set the VIRTIO_NET_HDR_F_DATA_VALID bit in \field{flags}
together with the VIRTIO_NET_HDR_F_GSO_UDP_TUNNEL_IPV4 bit or the
VIRTIO_NET_HDR_F_GSO_UDP_TUNNEL_IPV6 bit in \field{gso_type}.

\devicenormative{\paragraph}{Packet Transmission}{Device Types / Network Device / Device Operation / Packet Transmission}
The device MUST ignore \field{flag} bits that it does not recognize.

If VIRTIO_NET_HDR_F_NEEDS_CSUM bit in \field{flags} is not set, the
device MUST NOT use the \field{csum_start} and \field{csum_offset}.

If one of the VIRTIO_NET_F_HOST_TSO4, TSO6, USO or UFO options have
been negotiated:
\begin{itemize}
\item If the VIRTIO_NET_F_GUEST_HDRLEN feature has been negotiated,
	and \field{gso_type} differs from VIRTIO_NET_HDR_GSO_NONE,
	the device MAY use \field{hdr_len} as the transport header size.

	\begin{note}
	Caution should be taken by the implementation so as to prevent
	a malicious driver from attacking the device by setting an incorrect hdr_len.
	\end{note}

\item If the VIRTIO_NET_F_GUEST_HDRLEN feature has not been negotiated,
	or \field{gso_type} is VIRTIO_NET_HDR_GSO_NONE,
	the device MAY use \field{hdr_len} only as a hint about the
	transport header size.
	The device MUST NOT rely on \field{hdr_len} to be correct.

	\begin{note}
	This is due to various bugs in implementations.
	\end{note}
\end{itemize}

If both the VIRTIO_NET_HDR_GSO_UDP_TUNNEL_IPV4 bit and
the VIRTIO_NET_HDR_GSO_UDP_TUNNEL_IPV6 bit in in \field{gso_type} are set,
the device MUST NOT accept the packet.

If the VIRTIO_NET_HDR_GSO_UDP_TUNNEL_IPV4 bit and the VIRTIO_NET_HDR_GSO_UDP_TUNNEL_IPV6
bit in \field{gso_type} are not set, the device MUST NOT use the
\field{outer_th_offset} and \field{inner_nh_offset}.

If either the VIRTIO_NET_HDR_GSO_UDP_TUNNEL_IPV4 bit or
the VIRTIO_NET_HDR_GSO_UDP_TUNNEL_IPV6 bit in \field{gso_type} are set, and any of
the following is true:
\begin{itemize}
\item the VIRTIO_NET_HDR_F_NEEDS_CSUM is not set in \field{flags}
\item the VIRTIO_NET_HDR_F_DATA_VALID is set in \field{flags}
\item the \field{gso_type} excluding the VIRTIO_NET_HDR_GSO_UDP_TUNNEL_IPV4
bit and the VIRTIO_NET_HDR_GSO_UDP_TUNNEL_IPV6 bit is VIRTIO_NET_HDR_GSO_NONE
\end{itemize}
the device MUST NOT accept the packet.

If the VIRTIO_NET_HDR_F_UDP_TUNNEL_CSUM bit in \field{flags} is set,
and both the bits VIRTIO_NET_HDR_GSO_UDP_TUNNEL_IPV4 and
VIRTIO_NET_HDR_GSO_UDP_TUNNEL_IPV6 in \field{gso_type} are not set,
the device MOST NOT accept the packet.

If VIRTIO_NET_HDR_F_NEEDS_CSUM is not set, the device MUST NOT
rely on the packet checksum being correct.
\paragraph{Packet Transmission Interrupt}\label{sec:Device Types / Network Device / Device Operation / Packet Transmission / Packet Transmission Interrupt}

Often a driver will suppress transmission virtqueue interrupts
and check for used packets in the transmit path of following
packets.

The normal behavior in this interrupt handler is to retrieve
used buffers from the virtqueue and free the corresponding
headers and packets.

\subsubsection{Setting Up Receive Buffers}\label{sec:Device Types / Network Device / Device Operation / Setting Up Receive Buffers}

It is generally a good idea to keep the receive virtqueue as
fully populated as possible: if it runs out, network performance
will suffer.

If the VIRTIO_NET_F_GUEST_TSO4, VIRTIO_NET_F_GUEST_TSO6,
VIRTIO_NET_F_GUEST_UFO, VIRTIO_NET_F_GUEST_USO4 or VIRTIO_NET_F_GUEST_USO6
features are used, the maximum incoming packet
will be 65589 bytes long (14 bytes of Ethernet header, plus 40 bytes of
the IPv6 header, plus 65535 bytes of maximum IPv6 payload including any
extension header), otherwise 1514 bytes.
When VIRTIO_NET_F_HASH_REPORT is not negotiated, the required receive buffer
size is either 65601 or 1526 bytes accounting for 20 bytes of
\field{struct virtio_net_hdr} followed by receive packet.
When VIRTIO_NET_F_HASH_REPORT is negotiated, the required receive buffer
size is either 65609 or 1534 bytes accounting for 12 bytes of
\field{struct virtio_net_hdr} followed by receive packet.

\drivernormative{\paragraph}{Setting Up Receive Buffers}{Device Types / Network Device / Device Operation / Setting Up Receive Buffers}

\begin{itemize}
\item If VIRTIO_NET_F_MRG_RXBUF is not negotiated:
  \begin{itemize}
    \item If VIRTIO_NET_F_GUEST_TSO4, VIRTIO_NET_F_GUEST_TSO6, VIRTIO_NET_F_GUEST_UFO,
	VIRTIO_NET_F_GUEST_USO4 or VIRTIO_NET_F_GUEST_USO6 are negotiated, the driver SHOULD populate
      the receive queue(s) with buffers of at least 65609 bytes if
      VIRTIO_NET_F_HASH_REPORT is negotiated, and of at least 65601 bytes if not.
    \item Otherwise, the driver SHOULD populate the receive queue(s)
      with buffers of at least 1534 bytes if VIRTIO_NET_F_HASH_REPORT
      is negotiated, and of at least 1526 bytes if not.
  \end{itemize}
\item If VIRTIO_NET_F_MRG_RXBUF is negotiated, each buffer MUST be at
least size of \field{struct virtio_net_hdr},
i.e. 20 bytes if VIRTIO_NET_F_HASH_REPORT is negotiated, and 12 bytes if not.
\end{itemize}

\begin{note}
Obviously each buffer can be split across multiple descriptor elements.
\end{note}

When calculating the size of \field{struct virtio_net_hdr}, the driver
MUST consider all the fields inclusive up to \field{padding_reserved},
i.e. 20 bytes if VIRTIO_NET_F_HASH_REPORT is negotiated, and 12 bytes if not.

If VIRTIO_NET_F_MQ is negotiated, each of receiveq1\ldots receiveqN
that will be used SHOULD be populated with receive buffers.

\devicenormative{\paragraph}{Setting Up Receive Buffers}{Device Types / Network Device / Device Operation / Setting Up Receive Buffers}

The device MUST set \field{num_buffers} to the number of descriptors used to
hold the incoming packet.

The device MUST use only a single descriptor if VIRTIO_NET_F_MRG_RXBUF
was not negotiated.
\begin{note}
{This means that \field{num_buffers} will always be 1
if VIRTIO_NET_F_MRG_RXBUF is not negotiated.}
\end{note}

\subsubsection{Processing of Incoming Packets}\label{sec:Device Types / Network Device / Device Operation / Processing of Incoming Packets}
\label{sec:Device Types / Network Device / Device Operation / Processing of Packets}%old label for latexdiff

When a packet is copied into a buffer in the receiveq, the
optimal path is to disable further used buffer notifications for the
receiveq and process packets until no more are found, then re-enable
them.

Processing incoming packets involves:

\begin{enumerate}
\item \field{num_buffers} indicates how many descriptors
  this packet is spread over (including this one): this will
  always be 1 if VIRTIO_NET_F_MRG_RXBUF was not negotiated.
  This allows receipt of large packets without having to allocate large
  buffers: a packet that does not fit in a single buffer can flow
  over to the next buffer, and so on. In this case, there will be
  at least \field{num_buffers} used buffers in the virtqueue, and the device
  chains them together to form a single packet in a way similar to
  how it would store it in a single buffer spread over multiple
  descriptors.
  The other buffers will not begin with a \field{struct virtio_net_hdr}.

\item If
  \field{num_buffers} is one, then the entire packet will be
  contained within this buffer, immediately following the struct
  virtio_net_hdr.
\item If the VIRTIO_NET_F_GUEST_CSUM feature was negotiated, the
  VIRTIO_NET_HDR_F_DATA_VALID bit in \field{flags} can be
  set: if so, device has validated the packet checksum.
  If the VIRTIO_NET_F_GUEST_UDP_TUNNEL_GSO_CSUM feature has been negotiated,
  and the VIRTIO_NET_HDR_F_UDP_TUNNEL_CSUM bit is set in \field{flags},
  both the outer UDP checksum and the inner transport checksum
  have been validated, otherwise only one level of checksums (the outer one
  in case of tunnels) has been validated.
\end{enumerate}

Additionally, VIRTIO_NET_F_GUEST_CSUM, TSO4, TSO6, UDP, UDP_TUNNEL
and ECN features enable receive checksum, large receive offload and ECN
support which are the input equivalents of the transmit checksum,
transmit segmentation offloading and ECN features, as described
in \ref{sec:Device Types / Network Device / Device Operation /
Packet Transmission}:
\begin{enumerate}
\item If the VIRTIO_NET_F_GUEST_TSO4, TSO6, UFO, USO4 or USO6 options were
  negotiated, then \field{gso_type} MAY be something other than
  VIRTIO_NET_HDR_GSO_NONE, and \field{gso_size} field indicates the
  desired MSS (see Packet Transmission point 2).
\item If the VIRTIO_NET_F_RSC_EXT option was negotiated (this
  implies one of VIRTIO_NET_F_GUEST_TSO4, TSO6), the
  device processes also duplicated ACK segments, reports
  number of coalesced TCP segments in \field{csum_start} field and
  number of duplicated ACK segments in \field{csum_offset} field
  and sets bit VIRTIO_NET_HDR_F_RSC_INFO in \field{flags}.
\item If the VIRTIO_NET_F_GUEST_CSUM feature was negotiated, the
  VIRTIO_NET_HDR_F_NEEDS_CSUM bit in \field{flags} can be
  set: if so, the packet checksum at offset \field{csum_offset}
  from \field{csum_start} and any preceding checksums
  have been validated.  The checksum on the packet is incomplete and
  if bit VIRTIO_NET_HDR_F_RSC_INFO is not set in \field{flags},
  then \field{csum_start} and \field{csum_offset} indicate how to calculate it
  (see Packet Transmission point 1).
\begin{note}
If \field{gso_type} differs from VIRTIO_NET_HDR_GSO_NONE and the
VIRTIO_NET_HDR_GSO_UDP_TUNNEL_IPV4 bit or the VIRTIO_NET_HDR_GSO_UDP_TUNNEL_IPV6
bit are not set, \field{csum_offset}
points to the only transport header present in the packet, and there are no
additional preceding checksums validated by VIRTIO_NET_HDR_F_NEEDS_CSUM.
\end{note}
\item If the VIRTIO_NET_F_GUEST_UDP_TUNNEL_GSO option was negotiated and
  \field{gso_type} is not VIRTIO_NET_HDR_GSO_NONE, the
  VIRTIO_NET_HDR_GSO_UDP_TUNNEL_IPV4 bit or the VIRTIO_NET_HDR_GSO_UDP_TUNNEL_IPV6
  bit MAY be set. In such case the \field{outer_th_offset} and
  \field{inner_nh_offset} fields indicate the corresponding
  headers information.
\item If the VIRTIO_NET_F_GUEST_UDP_TUNNEL_GSO_CSUM feature was
negotiated, and
  the VIRTIO_NET_HDR_GSO_UDP_TUNNEL_IPV4 bit or the VIRTIO_NET_HDR_GSO_UDP_TUNNEL_IPV6
  are set in \field{gso_type}, the VIRTIO_NET_HDR_F_UDP_TUNNEL_CSUM bit in the
  \field{flags} can be set: if so, the outer UDP checksum has been validated
  and the UDP header checksum at offset 6 from from \field{outer_th_offset}
  is set to the outer UDP pseudo header checksum.

\begin{note}
If the VIRTIO_NET_HDR_GSO_UDP_TUNNEL_IPV4 bit or VIRTIO_NET_HDR_GSO_UDP_TUNNEL_IPV6
bit are set in \field{gso_type}, the \field{csum_start} field refers to
the inner transport header offset (see Packet Transmission point 1).
If the VIRTIO_NET_HDR_F_UDP_TUNNEL_CSUM bit in \field{flags} is set both
the inner and the outer header checksums have been validated by
VIRTIO_NET_HDR_F_NEEDS_CSUM, otherwise only the inner transport header
checksum has been validated.
\end{note}
\end{enumerate}

If applicable, the device calculates per-packet hash for incoming packets as
defined in \ref{sec:Device Types / Network Device / Device Operation / Processing of Incoming Packets / Hash calculation for incoming packets}.

If applicable, the device reports hash information for incoming packets as
defined in \ref{sec:Device Types / Network Device / Device Operation / Processing of Incoming Packets / Hash reporting for incoming packets}.

\devicenormative{\paragraph}{Processing of Incoming Packets}{Device Types / Network Device / Device Operation / Processing of Incoming Packets}
\label{devicenormative:Device Types / Network Device / Device Operation / Processing of Packets}%old label for latexdiff

If VIRTIO_NET_F_MRG_RXBUF has not been negotiated, the device MUST set
\field{num_buffers} to 1.

If VIRTIO_NET_F_MRG_RXBUF has been negotiated, the device MUST set
\field{num_buffers} to indicate the number of buffers
the packet (including the header) is spread over.

If a receive packet is spread over multiple buffers, the device
MUST use all buffers but the last (i.e. the first \field{num_buffers} -
1 buffers) completely up to the full length of each buffer
supplied by the driver.

The device MUST use all buffers used by a single receive
packet together, such that at least \field{num_buffers} are
observed by driver as used.

If VIRTIO_NET_F_GUEST_CSUM is not negotiated, the device MUST set
\field{flags} to zero and SHOULD supply a fully checksummed
packet to the driver.

If VIRTIO_NET_F_GUEST_TSO4 is not negotiated, the device MUST NOT set
\field{gso_type} to VIRTIO_NET_HDR_GSO_TCPV4.

If VIRTIO_NET_F_GUEST_UDP is not negotiated, the device MUST NOT set
\field{gso_type} to VIRTIO_NET_HDR_GSO_UDP.

If VIRTIO_NET_F_GUEST_TSO6 is not negotiated, the device MUST NOT set
\field{gso_type} to VIRTIO_NET_HDR_GSO_TCPV6.

If none of VIRTIO_NET_F_GUEST_USO4 or VIRTIO_NET_F_GUEST_USO6 have been negotiated,
the device MUST NOT set \field{gso_type} to VIRTIO_NET_HDR_GSO_UDP_L4.

If VIRTIO_NET_F_GUEST_UDP_TUNNEL_GSO is not negotiated, the device MUST NOT set
either the VIRTIO_NET_HDR_GSO_UDP_TUNNEL_IPV4 bit or the
VIRTIO_NET_HDR_GSO_UDP_TUNNEL_IPV6 bit in \field{gso_type}.

If VIRTIO_NET_F_GUEST_UDP_TUNNEL_GSO_CSUM is not negotiated the device MUST NOT set
the VIRTIO_NET_HDR_F_UDP_TUNNEL_CSUM bit in \field{flags}.

The device SHOULD NOT send to the driver TCP packets requiring segmentation offload
which have the Explicit Congestion Notification bit set, unless the
VIRTIO_NET_F_GUEST_ECN feature is negotiated, in which case the
device MUST set the VIRTIO_NET_HDR_GSO_ECN bit in
\field{gso_type}.

If the VIRTIO_NET_F_GUEST_CSUM feature has been negotiated, the
device MAY set the VIRTIO_NET_HDR_F_NEEDS_CSUM bit in
\field{flags}, if so:
\begin{enumerate}
\item the device MUST validate the packet checksum at
	offset \field{csum_offset} from \field{csum_start} as well as all
	preceding offsets;
\item the device MUST set the packet checksum stored in the
	receive buffer to the TCP/UDP pseudo header;
\item the device MUST set \field{csum_start} and
	\field{csum_offset} such that calculating a ones'
	complement checksum from \field{csum_start} up until the
	end of the packet and storing the result at offset
	\field{csum_offset} from  \field{csum_start} will result in a
	fully checksummed packet;
\end{enumerate}

The device MUST NOT send to the driver GSO packets encapsulated in UDP
tunnel and requiring segmentation offload, unless the
VIRTIO_NET_F_GUEST_UDP_TUNNEL_GSO is negotiated, in which case the device MUST set
the VIRTIO_NET_HDR_GSO_UDP_TUNNEL_IPV4 bit or the VIRTIO_NET_HDR_GSO_UDP_TUNNEL_IPV6
bit in \field{gso_type} according to the inner network header protocol type,
MUST set the \field{outer_th_offset} and \field{inner_nh_offset} fields
to the corresponding header information, and the outer UDP header MUST NOT
require checksum offload.

If the VIRTIO_NET_F_GUEST_UDP_TUNNEL_GSO_CSUM feature has not been negotiated,
the device MUST NOT send the driver GSO packets encapsulated in UDP
tunnel and requiring segmentation and outer checksum offload.

If none of the VIRTIO_NET_F_GUEST_TSO4, TSO6, UFO, USO4 or USO6 options have
been negotiated, the device MUST set \field{gso_type} to
VIRTIO_NET_HDR_GSO_NONE.

If \field{gso_type} differs from VIRTIO_NET_HDR_GSO_NONE, then
the device MUST also set the VIRTIO_NET_HDR_F_NEEDS_CSUM bit in
\field{flags} MUST set \field{gso_size} to indicate the desired MSS.
If VIRTIO_NET_F_RSC_EXT was negotiated, the device MUST also
set VIRTIO_NET_HDR_F_RSC_INFO bit in \field{flags},
set \field{csum_start} to number of coalesced TCP segments and
set \field{csum_offset} to number of received duplicated ACK segments.

If VIRTIO_NET_F_RSC_EXT was not negotiated, the device MUST
not set VIRTIO_NET_HDR_F_RSC_INFO bit in \field{flags}.

If one of the VIRTIO_NET_F_GUEST_TSO4, TSO6, UFO, USO4 or USO6 options have
been negotiated, the device SHOULD set \field{hdr_len} to a value
not less than the length of the headers, including the transport
header. If \field{gso_type} has the VIRTIO_NET_HDR_GSO_UDP_TUNNEL_IPV4 bit
or the VIRTIO_NET_HDR_GSO_UDP_TUNNEL_IPV6 bit set, the referenced transport
header is the inner one.

If the VIRTIO_NET_F_GUEST_CSUM feature has been negotiated, the
device MAY set the VIRTIO_NET_HDR_F_DATA_VALID bit in
\field{flags}, if so, the device MUST validate the packet
checksum. If the VIRTIO_NET_F_GUEST_UDP_TUNNEL_GSO_CSUM feature has
been negotiated, and the VIRTIO_NET_HDR_F_UDP_TUNNEL_CSUM bit set in
\field{flags}, both the outer UDP checksum and the inner transport
checksum have been validated.
Otherwise level of checksum is validated: in case of multiple
encapsulated protocols the outermost one.

If either the VIRTIO_NET_HDR_GSO_UDP_TUNNEL_IPV4 bit or the
VIRTIO_NET_HDR_GSO_UDP_TUNNEL_IPV6 bit in \field{gso_type} are set,
the device MUST NOT set the VIRTIO_NET_HDR_F_DATA_VALID bit in
\field{flags}.

If the VIRTIO_NET_F_GUEST_UDP_TUNNEL_GSO_CSUM feature has been negotiated
and either the VIRTIO_NET_HDR_GSO_UDP_TUNNEL_IPV4 bit is set or the
VIRTIO_NET_HDR_GSO_UDP_TUNNEL_IPV6 bit is set in \field{gso_type}, the
device MAY set the VIRTIO_NET_HDR_F_UDP_TUNNEL_CSUM bit in
\field{flags}, if so the device MUST set the packet outer UDP checksum
stored in the receive buffer to the outer UDP pseudo header.

Otherwise, the VIRTIO_NET_F_GUEST_UDP_TUNNEL_GSO_CSUM feature has been
negotiated, either the VIRTIO_NET_HDR_GSO_UDP_TUNNEL_IPV4 bit is set or the
VIRTIO_NET_HDR_GSO_UDP_TUNNEL_IPV6 bit is set in \field{gso_type},
and the bit VIRTIO_NET_HDR_F_UDP_TUNNEL_CSUM is not set in
\field{flags}, the device MUST either provide a zero outer UDP header
checksum or a fully checksummed outer UDP header.

\drivernormative{\paragraph}{Processing of Incoming
Packets}{Device Types / Network Device / Device Operation /
Processing of Incoming Packets}

The driver MUST ignore \field{flag} bits that it does not recognize.

If VIRTIO_NET_HDR_F_NEEDS_CSUM bit in \field{flags} is not set or
if VIRTIO_NET_HDR_F_RSC_INFO bit \field{flags} is set, the
driver MUST NOT use the \field{csum_start} and \field{csum_offset}.

If one of the VIRTIO_NET_F_GUEST_TSO4, TSO6, UFO, USO4 or USO6 options have
been negotiated, the driver MAY use \field{hdr_len} only as a hint about the
transport header size.
The driver MUST NOT rely on \field{hdr_len} to be correct.
\begin{note}
This is due to various bugs in implementations.
\end{note}

If neither VIRTIO_NET_HDR_F_NEEDS_CSUM nor
VIRTIO_NET_HDR_F_DATA_VALID is set, the driver MUST NOT
rely on the packet checksum being correct.

If both the VIRTIO_NET_HDR_GSO_UDP_TUNNEL_IPV4 bit and
the VIRTIO_NET_HDR_GSO_UDP_TUNNEL_IPV6 bit in in \field{gso_type} are set,
the driver MUST NOT accept the packet.

If the VIRTIO_NET_HDR_GSO_UDP_TUNNEL_IPV4 bit or the VIRTIO_NET_HDR_GSO_UDP_TUNNEL_IPV6
bit in \field{gso_type} are not set, the driver MUST NOT use the
\field{outer_th_offset} and \field{inner_nh_offset}.

If either the VIRTIO_NET_HDR_GSO_UDP_TUNNEL_IPV4 bit or
the VIRTIO_NET_HDR_GSO_UDP_TUNNEL_IPV6 bit in \field{gso_type} are set, and any of
the following is true:
\begin{itemize}
\item the VIRTIO_NET_HDR_F_NEEDS_CSUM bit is not set in \field{flags}
\item the VIRTIO_NET_HDR_F_DATA_VALID bit is set in \field{flags}
\item the \field{gso_type} excluding the VIRTIO_NET_HDR_GSO_UDP_TUNNEL_IPV4
bit and the VIRTIO_NET_HDR_GSO_UDP_TUNNEL_IPV6 bit is VIRTIO_NET_HDR_GSO_NONE
\end{itemize}
the driver MUST NOT accept the packet.

If the VIRTIO_NET_HDR_F_UDP_TUNNEL_CSUM bit and the VIRTIO_NET_HDR_F_NEEDS_CSUM
bit in \field{flags} are set,
and both the bits VIRTIO_NET_HDR_GSO_UDP_TUNNEL_IPV4 and
VIRTIO_NET_HDR_GSO_UDP_TUNNEL_IPV6 in \field{gso_type} are not set,
the driver MOST NOT accept the packet.

\paragraph{Hash calculation for incoming packets}
\label{sec:Device Types / Network Device / Device Operation / Processing of Incoming Packets / Hash calculation for incoming packets}

A device attempts to calculate a per-packet hash in the following cases:
\begin{itemize}
\item The feature VIRTIO_NET_F_RSS was negotiated. The device uses the hash to determine the receive virtqueue to place incoming packets.
\item The feature VIRTIO_NET_F_HASH_REPORT was negotiated. The device reports the hash value and the hash type with the packet.
\end{itemize}

If the feature VIRTIO_NET_F_RSS was negotiated:
\begin{itemize}
\item The device uses \field{hash_types} of the virtio_net_rss_config structure as 'Enabled hash types' bitmask.
\item If additionally the feature VIRTIO_NET_F_HASH_TUNNEL was negotiated, the device uses \field{enabled_tunnel_types} of the
      virtnet_hash_tunnel structure as 'Encapsulation types enabled for inner header hash' bitmask.
\item The device uses a key as defined in \field{hash_key_data} and \field{hash_key_length} of the virtio_net_rss_config structure (see
\ref{sec:Device Types / Network Device / Device Operation / Control Virtqueue / Receive-side scaling (RSS) / Setting RSS parameters}).
\end{itemize}

If the feature VIRTIO_NET_F_RSS was not negotiated:
\begin{itemize}
\item The device uses \field{hash_types} of the virtio_net_hash_config structure as 'Enabled hash types' bitmask.
\item If additionally the feature VIRTIO_NET_F_HASH_TUNNEL was negotiated, the device uses \field{enabled_tunnel_types} of the
      virtnet_hash_tunnel structure as 'Encapsulation types enabled for inner header hash' bitmask.
\item The device uses a key as defined in \field{hash_key_data} and \field{hash_key_length} of the virtio_net_hash_config structure (see
\ref{sec:Device Types / Network Device / Device Operation / Control Virtqueue / Automatic receive steering in multiqueue mode / Hash calculation}).
\end{itemize}

Note that if the device offers VIRTIO_NET_F_HASH_REPORT, even if it supports only one pair of virtqueues, it MUST support
at least one of commands of VIRTIO_NET_CTRL_MQ class to configure reported hash parameters:
\begin{itemize}
\item If the device offers VIRTIO_NET_F_RSS, it MUST support VIRTIO_NET_CTRL_MQ_RSS_CONFIG command per
 \ref{sec:Device Types / Network Device / Device Operation / Control Virtqueue / Receive-side scaling (RSS) / Setting RSS parameters}.
\item Otherwise the device MUST support VIRTIO_NET_CTRL_MQ_HASH_CONFIG command per
 \ref{sec:Device Types / Network Device / Device Operation / Control Virtqueue / Automatic receive steering in multiqueue mode / Hash calculation}.
\end{itemize}

The per-packet hash calculation can depend on the IP packet type. See
\hyperref[intro:IP]{[IP]}, \hyperref[intro:UDP]{[UDP]} and \hyperref[intro:TCP]{[TCP]}.

\subparagraph{Supported/enabled hash types}
\label{sec:Device Types / Network Device / Device Operation / Processing of Incoming Packets / Hash calculation for incoming packets / Supported/enabled hash types}
Hash types applicable for IPv4 packets:
\begin{lstlisting}
#define VIRTIO_NET_HASH_TYPE_IPv4              (1 << 0)
#define VIRTIO_NET_HASH_TYPE_TCPv4             (1 << 1)
#define VIRTIO_NET_HASH_TYPE_UDPv4             (1 << 2)
\end{lstlisting}
Hash types applicable for IPv6 packets without extension headers
\begin{lstlisting}
#define VIRTIO_NET_HASH_TYPE_IPv6              (1 << 3)
#define VIRTIO_NET_HASH_TYPE_TCPv6             (1 << 4)
#define VIRTIO_NET_HASH_TYPE_UDPv6             (1 << 5)
\end{lstlisting}
Hash types applicable for IPv6 packets with extension headers
\begin{lstlisting}
#define VIRTIO_NET_HASH_TYPE_IP_EX             (1 << 6)
#define VIRTIO_NET_HASH_TYPE_TCP_EX            (1 << 7)
#define VIRTIO_NET_HASH_TYPE_UDP_EX            (1 << 8)
\end{lstlisting}

\subparagraph{IPv4 packets}
\label{sec:Device Types / Network Device / Device Operation / Processing of Incoming Packets / Hash calculation for incoming packets / IPv4 packets}
The device calculates the hash on IPv4 packets according to 'Enabled hash types' bitmask as follows:
\begin{itemize}
\item If VIRTIO_NET_HASH_TYPE_TCPv4 is set and the packet has
a TCP header, the hash is calculated over the following fields:
\begin{itemize}
\item Source IP address
\item Destination IP address
\item Source TCP port
\item Destination TCP port
\end{itemize}
\item Else if VIRTIO_NET_HASH_TYPE_UDPv4 is set and the
packet has a UDP header, the hash is calculated over the following fields:
\begin{itemize}
\item Source IP address
\item Destination IP address
\item Source UDP port
\item Destination UDP port
\end{itemize}
\item Else if VIRTIO_NET_HASH_TYPE_IPv4 is set, the hash is
calculated over the following fields:
\begin{itemize}
\item Source IP address
\item Destination IP address
\end{itemize}
\item Else the device does not calculate the hash
\end{itemize}

\subparagraph{IPv6 packets without extension header}
\label{sec:Device Types / Network Device / Device Operation / Processing of Incoming Packets / Hash calculation for incoming packets / IPv6 packets without extension header}
The device calculates the hash on IPv6 packets without extension
headers according to 'Enabled hash types' bitmask as follows:
\begin{itemize}
\item If VIRTIO_NET_HASH_TYPE_TCPv6 is set and the packet has
a TCPv6 header, the hash is calculated over the following fields:
\begin{itemize}
\item Source IPv6 address
\item Destination IPv6 address
\item Source TCP port
\item Destination TCP port
\end{itemize}
\item Else if VIRTIO_NET_HASH_TYPE_UDPv6 is set and the
packet has a UDPv6 header, the hash is calculated over the following fields:
\begin{itemize}
\item Source IPv6 address
\item Destination IPv6 address
\item Source UDP port
\item Destination UDP port
\end{itemize}
\item Else if VIRTIO_NET_HASH_TYPE_IPv6 is set, the hash is
calculated over the following fields:
\begin{itemize}
\item Source IPv6 address
\item Destination IPv6 address
\end{itemize}
\item Else the device does not calculate the hash
\end{itemize}

\subparagraph{IPv6 packets with extension header}
\label{sec:Device Types / Network Device / Device Operation / Processing of Incoming Packets / Hash calculation for incoming packets / IPv6 packets with extension header}
The device calculates the hash on IPv6 packets with extension
headers according to 'Enabled hash types' bitmask as follows:
\begin{itemize}
\item If VIRTIO_NET_HASH_TYPE_TCP_EX is set and the packet
has a TCPv6 header, the hash is calculated over the following fields:
\begin{itemize}
\item Home address from the home address option in the IPv6 destination options header. If the extension header is not present, use the Source IPv6 address.
\item IPv6 address that is contained in the Routing-Header-Type-2 from the associated extension header. If the extension header is not present, use the Destination IPv6 address.
\item Source TCP port
\item Destination TCP port
\end{itemize}
\item Else if VIRTIO_NET_HASH_TYPE_UDP_EX is set and the
packet has a UDPv6 header, the hash is calculated over the following fields:
\begin{itemize}
\item Home address from the home address option in the IPv6 destination options header. If the extension header is not present, use the Source IPv6 address.
\item IPv6 address that is contained in the Routing-Header-Type-2 from the associated extension header. If the extension header is not present, use the Destination IPv6 address.
\item Source UDP port
\item Destination UDP port
\end{itemize}
\item Else if VIRTIO_NET_HASH_TYPE_IP_EX is set, the hash is
calculated over the following fields:
\begin{itemize}
\item Home address from the home address option in the IPv6 destination options header. If the extension header is not present, use the Source IPv6 address.
\item IPv6 address that is contained in the Routing-Header-Type-2 from the associated extension header. If the extension header is not present, use the Destination IPv6 address.
\end{itemize}
\item Else skip IPv6 extension headers and calculate the hash as
defined for an IPv6 packet without extension headers
(see \ref{sec:Device Types / Network Device / Device Operation / Processing of Incoming Packets / Hash calculation for incoming packets / IPv6 packets without extension header}).
\end{itemize}

\paragraph{Inner Header Hash}
\label{sec:Device Types / Network Device / Device Operation / Processing of Incoming Packets / Inner Header Hash}

If VIRTIO_NET_F_HASH_TUNNEL has been negotiated, the driver can send the command
VIRTIO_NET_CTRL_HASH_TUNNEL_SET to configure the calculation of the inner header hash.

\begin{lstlisting}
struct virtnet_hash_tunnel {
    le32 enabled_tunnel_types;
};

#define VIRTIO_NET_CTRL_HASH_TUNNEL 7
 #define VIRTIO_NET_CTRL_HASH_TUNNEL_SET 0
\end{lstlisting}

Field \field{enabled_tunnel_types} contains the bitmask of encapsulation types enabled for inner header hash.
See \ref{sec:Device Types / Network Device / Device Operation / Processing of Incoming Packets /
Hash calculation for incoming packets / Encapsulation types supported/enabled for inner header hash}.

The class VIRTIO_NET_CTRL_HASH_TUNNEL has one command:
VIRTIO_NET_CTRL_HASH_TUNNEL_SET sets \field{enabled_tunnel_types} for the device using the
virtnet_hash_tunnel structure, which is read-only for the device.

Inner header hash is disabled by VIRTIO_NET_CTRL_HASH_TUNNEL_SET with \field{enabled_tunnel_types} set to 0.

Initially (before the driver sends any VIRTIO_NET_CTRL_HASH_TUNNEL_SET command) all
encapsulation types are disabled for inner header hash.

\subparagraph{Encapsulated packet}
\label{sec:Device Types / Network Device / Device Operation / Processing of Incoming Packets / Hash calculation for incoming packets / Encapsulated packet}

Multiple tunneling protocols allow encapsulating an inner, payload packet in an outer, encapsulated packet.
The encapsulated packet thus contains an outer header and an inner header, and the device calculates the
hash over either the inner header or the outer header.

If VIRTIO_NET_F_HASH_TUNNEL is negotiated and a received encapsulated packet's outer header matches one of the
encapsulation types enabled in \field{enabled_tunnel_types}, then the device uses the inner header for hash
calculations (only a single level of encapsulation is currently supported).

If VIRTIO_NET_F_HASH_TUNNEL is negotiated and a received packet's (outer) header does not match any encapsulation
types enabled in \field{enabled_tunnel_types}, then the device uses the outer header for hash calculations.

\subparagraph{Encapsulation types supported/enabled for inner header hash}
\label{sec:Device Types / Network Device / Device Operation / Processing of Incoming Packets /
Hash calculation for incoming packets / Encapsulation types supported/enabled for inner header hash}

Encapsulation types applicable for inner header hash:
\begin{lstlisting}[escapechar=|]
#define VIRTIO_NET_HASH_TUNNEL_TYPE_GRE_2784    (1 << 0) /* |\hyperref[intro:rfc2784]{[RFC2784]}| */
#define VIRTIO_NET_HASH_TUNNEL_TYPE_GRE_2890    (1 << 1) /* |\hyperref[intro:rfc2890]{[RFC2890]}| */
#define VIRTIO_NET_HASH_TUNNEL_TYPE_GRE_7676    (1 << 2) /* |\hyperref[intro:rfc7676]{[RFC7676]}| */
#define VIRTIO_NET_HASH_TUNNEL_TYPE_GRE_UDP     (1 << 3) /* |\hyperref[intro:rfc8086]{[GRE-in-UDP]}| */
#define VIRTIO_NET_HASH_TUNNEL_TYPE_VXLAN       (1 << 4) /* |\hyperref[intro:vxlan]{[VXLAN]}| */
#define VIRTIO_NET_HASH_TUNNEL_TYPE_VXLAN_GPE   (1 << 5) /* |\hyperref[intro:vxlan-gpe]{[VXLAN-GPE]}| */
#define VIRTIO_NET_HASH_TUNNEL_TYPE_GENEVE      (1 << 6) /* |\hyperref[intro:geneve]{[GENEVE]}| */
#define VIRTIO_NET_HASH_TUNNEL_TYPE_IPIP        (1 << 7) /* |\hyperref[intro:ipip]{[IPIP]}| */
#define VIRTIO_NET_HASH_TUNNEL_TYPE_NVGRE       (1 << 8) /* |\hyperref[intro:nvgre]{[NVGRE]}| */
\end{lstlisting}

\subparagraph{Advice}
Example uses of the inner header hash:
\begin{itemize}
\item Legacy tunneling protocols, lacking the outer header entropy, can use RSS with the inner header hash to
      distribute flows with identical outer but different inner headers across various queues, improving performance.
\item Identify an inner flow distributed across multiple outer tunnels.
\end{itemize}

As using the inner header hash completely discards the outer header entropy, care must be taken
if the inner header is controlled by an adversary, as the adversary can then intentionally create
configurations with insufficient entropy.

Besides disabling the inner header hash, mitigations would depend on how the hash is used. When the hash
use is limited to the RSS queue selection, the inner header hash may have quality of service (QoS) limitations.

\devicenormative{\subparagraph}{Inner Header Hash}{Device Types / Network Device / Device Operation / Control Virtqueue / Inner Header Hash}

If the (outer) header of the received packet does not match any encapsulation types enabled
in \field{enabled_tunnel_types}, the device MUST calculate the hash on the outer header.

If the device receives any bits in \field{enabled_tunnel_types} which are not set in \field{supported_tunnel_types},
it SHOULD respond to the VIRTIO_NET_CTRL_HASH_TUNNEL_SET command with VIRTIO_NET_ERR.

If the driver sets \field{enabled_tunnel_types} to 0 through VIRTIO_NET_CTRL_HASH_TUNNEL_SET or upon the device reset,
the device MUST disable the inner header hash for all encapsulation types.

\drivernormative{\subparagraph}{Inner Header Hash}{Device Types / Network Device / Device Operation / Control Virtqueue / Inner Header Hash}

The driver MUST have negotiated the VIRTIO_NET_F_HASH_TUNNEL feature when issuing the VIRTIO_NET_CTRL_HASH_TUNNEL_SET command.

The driver MUST NOT set any bits in \field{enabled_tunnel_types} which are not set in \field{supported_tunnel_types}.

The driver MUST ignore bits in \field{supported_tunnel_types} which are not documented in this specification.

\paragraph{Hash reporting for incoming packets}
\label{sec:Device Types / Network Device / Device Operation / Processing of Incoming Packets / Hash reporting for incoming packets}

If VIRTIO_NET_F_HASH_REPORT was negotiated and
 the device has calculated the hash for the packet, the device fills \field{hash_report} with the report type of calculated hash
and \field{hash_value} with the value of calculated hash.

If VIRTIO_NET_F_HASH_REPORT was negotiated but due to any reason the
hash was not calculated, the device sets \field{hash_report} to VIRTIO_NET_HASH_REPORT_NONE.

Possible values that the device can report in \field{hash_report} are defined below.
They correspond to supported hash types defined in
\ref{sec:Device Types / Network Device / Device Operation / Processing of Incoming Packets / Hash calculation for incoming packets / Supported/enabled hash types}
as follows:

VIRTIO_NET_HASH_TYPE_XXX = 1 << (VIRTIO_NET_HASH_REPORT_XXX - 1)

\begin{lstlisting}
#define VIRTIO_NET_HASH_REPORT_NONE            0
#define VIRTIO_NET_HASH_REPORT_IPv4            1
#define VIRTIO_NET_HASH_REPORT_TCPv4           2
#define VIRTIO_NET_HASH_REPORT_UDPv4           3
#define VIRTIO_NET_HASH_REPORT_IPv6            4
#define VIRTIO_NET_HASH_REPORT_TCPv6           5
#define VIRTIO_NET_HASH_REPORT_UDPv6           6
#define VIRTIO_NET_HASH_REPORT_IPv6_EX         7
#define VIRTIO_NET_HASH_REPORT_TCPv6_EX        8
#define VIRTIO_NET_HASH_REPORT_UDPv6_EX        9
\end{lstlisting}

\subsubsection{Control Virtqueue}\label{sec:Device Types / Network Device / Device Operation / Control Virtqueue}

The driver uses the control virtqueue (if VIRTIO_NET_F_CTRL_VQ is
negotiated) to send commands to manipulate various features of
the device which would not easily map into the configuration
space.

All commands are of the following form:

\begin{lstlisting}
struct virtio_net_ctrl {
        u8 class;
        u8 command;
        u8 command-specific-data[];
        u8 ack;
        u8 command-specific-result[];
};

/* ack values */
#define VIRTIO_NET_OK     0
#define VIRTIO_NET_ERR    1
\end{lstlisting}

The \field{class}, \field{command} and command-specific-data are set by the
driver, and the device sets the \field{ack} byte and optionally
\field{command-specific-result}. There is little the driver can
do except issue a diagnostic if \field{ack} is not VIRTIO_NET_OK.

The command VIRTIO_NET_CTRL_STATS_QUERY and VIRTIO_NET_CTRL_STATS_GET contain
\field{command-specific-result}.

\paragraph{Packet Receive Filtering}\label{sec:Device Types / Network Device / Device Operation / Control Virtqueue / Packet Receive Filtering}
\label{sec:Device Types / Network Device / Device Operation / Control Virtqueue / Setting Promiscuous Mode}%old label for latexdiff

If the VIRTIO_NET_F_CTRL_RX and VIRTIO_NET_F_CTRL_RX_EXTRA
features are negotiated, the driver can send control commands for
promiscuous mode, multicast, unicast and broadcast receiving.

\begin{note}
In general, these commands are best-effort: unwanted
packets could still arrive.
\end{note}

\begin{lstlisting}
#define VIRTIO_NET_CTRL_RX    0
 #define VIRTIO_NET_CTRL_RX_PROMISC      0
 #define VIRTIO_NET_CTRL_RX_ALLMULTI     1
 #define VIRTIO_NET_CTRL_RX_ALLUNI       2
 #define VIRTIO_NET_CTRL_RX_NOMULTI      3
 #define VIRTIO_NET_CTRL_RX_NOUNI        4
 #define VIRTIO_NET_CTRL_RX_NOBCAST      5
\end{lstlisting}


\devicenormative{\subparagraph}{Packet Receive Filtering}{Device Types / Network Device / Device Operation / Control Virtqueue / Packet Receive Filtering}

If the VIRTIO_NET_F_CTRL_RX feature has been negotiated,
the device MUST support the following VIRTIO_NET_CTRL_RX class
commands:
\begin{itemize}
\item VIRTIO_NET_CTRL_RX_PROMISC turns promiscuous mode on and
off. The command-specific-data is one byte containing 0 (off) or
1 (on). If promiscuous mode is on, the device SHOULD receive all
incoming packets.
This SHOULD take effect even if one of the other modes set by
a VIRTIO_NET_CTRL_RX class command is on.
\item VIRTIO_NET_CTRL_RX_ALLMULTI turns all-multicast receive on and
off. The command-specific-data is one byte containing 0 (off) or
1 (on). When all-multicast receive is on the device SHOULD allow
all incoming multicast packets.
\end{itemize}

If the VIRTIO_NET_F_CTRL_RX_EXTRA feature has been negotiated,
the device MUST support the following VIRTIO_NET_CTRL_RX class
commands:
\begin{itemize}
\item VIRTIO_NET_CTRL_RX_ALLUNI turns all-unicast receive on and
off. The command-specific-data is one byte containing 0 (off) or
1 (on). When all-unicast receive is on the device SHOULD allow
all incoming unicast packets.
\item VIRTIO_NET_CTRL_RX_NOMULTI suppresses multicast receive.
The command-specific-data is one byte containing 0 (multicast
receive allowed) or 1 (multicast receive suppressed).
When multicast receive is suppressed, the device SHOULD NOT
send multicast packets to the driver.
This SHOULD take effect even if VIRTIO_NET_CTRL_RX_ALLMULTI is on.
This filter SHOULD NOT apply to broadcast packets.
\item VIRTIO_NET_CTRL_RX_NOUNI suppresses unicast receive.
The command-specific-data is one byte containing 0 (unicast
receive allowed) or 1 (unicast receive suppressed).
When unicast receive is suppressed, the device SHOULD NOT
send unicast packets to the driver.
This SHOULD take effect even if VIRTIO_NET_CTRL_RX_ALLUNI is on.
\item VIRTIO_NET_CTRL_RX_NOBCAST suppresses broadcast receive.
The command-specific-data is one byte containing 0 (broadcast
receive allowed) or 1 (broadcast receive suppressed).
When broadcast receive is suppressed, the device SHOULD NOT
send broadcast packets to the driver.
This SHOULD take effect even if VIRTIO_NET_CTRL_RX_ALLMULTI is on.
\end{itemize}

\drivernormative{\subparagraph}{Packet Receive Filtering}{Device Types / Network Device / Device Operation / Control Virtqueue / Packet Receive Filtering}

If the VIRTIO_NET_F_CTRL_RX feature has not been negotiated,
the driver MUST NOT issue commands VIRTIO_NET_CTRL_RX_PROMISC or
VIRTIO_NET_CTRL_RX_ALLMULTI.

If the VIRTIO_NET_F_CTRL_RX_EXTRA feature has not been negotiated,
the driver MUST NOT issue commands
 VIRTIO_NET_CTRL_RX_ALLUNI,
 VIRTIO_NET_CTRL_RX_NOMULTI,
 VIRTIO_NET_CTRL_RX_NOUNI or
 VIRTIO_NET_CTRL_RX_NOBCAST.

\paragraph{Setting MAC Address Filtering}\label{sec:Device Types / Network Device / Device Operation / Control Virtqueue / Setting MAC Address Filtering}

If the VIRTIO_NET_F_CTRL_RX feature is negotiated, the driver can
send control commands for MAC address filtering.

\begin{lstlisting}
struct virtio_net_ctrl_mac {
        le32 entries;
        u8 macs[entries][6];
};

#define VIRTIO_NET_CTRL_MAC    1
 #define VIRTIO_NET_CTRL_MAC_TABLE_SET        0
 #define VIRTIO_NET_CTRL_MAC_ADDR_SET         1
\end{lstlisting}

The device can filter incoming packets by any number of destination
MAC addresses\footnote{Since there are no guarantees, it can use a hash filter or
silently switch to allmulti or promiscuous mode if it is given too
many addresses.
}. This table is set using the class
VIRTIO_NET_CTRL_MAC and the command VIRTIO_NET_CTRL_MAC_TABLE_SET. The
command-specific-data is two variable length tables of 6-byte MAC
addresses (as described in struct virtio_net_ctrl_mac). The first table contains unicast addresses, and the second
contains multicast addresses.

The VIRTIO_NET_CTRL_MAC_ADDR_SET command is used to set the
default MAC address which rx filtering
accepts (and if VIRTIO_NET_F_MAC has been negotiated,
this will be reflected in \field{mac} in config space).

The command-specific-data for VIRTIO_NET_CTRL_MAC_ADDR_SET is
the 6-byte MAC address.

\devicenormative{\subparagraph}{Setting MAC Address Filtering}{Device Types / Network Device / Device Operation / Control Virtqueue / Setting MAC Address Filtering}

The device MUST have an empty MAC filtering table on reset.

The device MUST update the MAC filtering table before it consumes
the VIRTIO_NET_CTRL_MAC_TABLE_SET command.

The device MUST update \field{mac} in config space before it consumes
the VIRTIO_NET_CTRL_MAC_ADDR_SET command, if VIRTIO_NET_F_MAC has
been negotiated.

The device SHOULD drop incoming packets which have a destination MAC which
matches neither the \field{mac} (or that set with VIRTIO_NET_CTRL_MAC_ADDR_SET)
nor the MAC filtering table.

\drivernormative{\subparagraph}{Setting MAC Address Filtering}{Device Types / Network Device / Device Operation / Control Virtqueue / Setting MAC Address Filtering}

If VIRTIO_NET_F_CTRL_RX has not been negotiated,
the driver MUST NOT issue VIRTIO_NET_CTRL_MAC class commands.

If VIRTIO_NET_F_CTRL_RX has been negotiated,
the driver SHOULD issue VIRTIO_NET_CTRL_MAC_ADDR_SET
to set the default mac if it is different from \field{mac}.

The driver MUST follow the VIRTIO_NET_CTRL_MAC_TABLE_SET command
by a le32 number, followed by that number of non-multicast
MAC addresses, followed by another le32 number, followed by
that number of multicast addresses.  Either number MAY be 0.

\subparagraph{Legacy Interface: Setting MAC Address Filtering}\label{sec:Device Types / Network Device / Device Operation / Control Virtqueue / Setting MAC Address Filtering / Legacy Interface: Setting MAC Address Filtering}
When using the legacy interface, transitional devices and drivers
MUST format \field{entries} in struct virtio_net_ctrl_mac
according to the native endian of the guest rather than
(necessarily when not using the legacy interface) little-endian.

Legacy drivers that didn't negotiate VIRTIO_NET_F_CTRL_MAC_ADDR
changed \field{mac} in config space when NIC is accepting
incoming packets. These drivers always wrote the mac value from
first to last byte, therefore after detecting such drivers,
a transitional device MAY defer MAC update, or MAY defer
processing incoming packets until driver writes the last byte
of \field{mac} in the config space.

\paragraph{VLAN Filtering}\label{sec:Device Types / Network Device / Device Operation / Control Virtqueue / VLAN Filtering}

If the driver negotiates the VIRTIO_NET_F_CTRL_VLAN feature, it
can control a VLAN filter table in the device. The VLAN filter
table applies only to VLAN tagged packets.

When VIRTIO_NET_F_CTRL_VLAN is negotiated, the device starts with
an empty VLAN filter table.

\begin{note}
Similar to the MAC address based filtering, the VLAN filtering
is also best-effort: unwanted packets could still arrive.
\end{note}

\begin{lstlisting}
#define VIRTIO_NET_CTRL_VLAN       2
 #define VIRTIO_NET_CTRL_VLAN_ADD             0
 #define VIRTIO_NET_CTRL_VLAN_DEL             1
\end{lstlisting}

Both the VIRTIO_NET_CTRL_VLAN_ADD and VIRTIO_NET_CTRL_VLAN_DEL
command take a little-endian 16-bit VLAN id as the command-specific-data.

VIRTIO_NET_CTRL_VLAN_ADD command adds the specified VLAN to the
VLAN filter table.

VIRTIO_NET_CTRL_VLAN_DEL command removes the specified VLAN from
the VLAN filter table.

\devicenormative{\subparagraph}{VLAN Filtering}{Device Types / Network Device / Device Operation / Control Virtqueue / VLAN Filtering}

When VIRTIO_NET_F_CTRL_VLAN is not negotiated, the device MUST
accept all VLAN tagged packets.

When VIRTIO_NET_F_CTRL_VLAN is negotiated, the device MUST
accept all VLAN tagged packets whose VLAN tag is present in
the VLAN filter table and SHOULD drop all VLAN tagged packets
whose VLAN tag is absent in the VLAN filter table.

\subparagraph{Legacy Interface: VLAN Filtering}\label{sec:Device Types / Network Device / Device Operation / Control Virtqueue / VLAN Filtering / Legacy Interface: VLAN Filtering}
When using the legacy interface, transitional devices and drivers
MUST format the VLAN id
according to the native endian of the guest rather than
(necessarily when not using the legacy interface) little-endian.

\paragraph{Gratuitous Packet Sending}\label{sec:Device Types / Network Device / Device Operation / Control Virtqueue / Gratuitous Packet Sending}

If the driver negotiates the VIRTIO_NET_F_GUEST_ANNOUNCE (depends
on VIRTIO_NET_F_CTRL_VQ), the device can ask the driver to send gratuitous
packets; this is usually done after the guest has been physically
migrated, and needs to announce its presence on the new network
links. (As hypervisor does not have the knowledge of guest
network configuration (eg. tagged vlan) it is simplest to prod
the guest in this way).

\begin{lstlisting}
#define VIRTIO_NET_CTRL_ANNOUNCE       3
 #define VIRTIO_NET_CTRL_ANNOUNCE_ACK             0
\end{lstlisting}

The driver checks VIRTIO_NET_S_ANNOUNCE bit in the device configuration \field{status} field
when it notices the changes of device configuration. The
command VIRTIO_NET_CTRL_ANNOUNCE_ACK is used to indicate that
driver has received the notification and device clears the
VIRTIO_NET_S_ANNOUNCE bit in \field{status}.

Processing this notification involves:

\begin{enumerate}
\item Sending the gratuitous packets (eg. ARP) or marking there are pending
  gratuitous packets to be sent and letting deferred routine to
  send them.

\item Sending VIRTIO_NET_CTRL_ANNOUNCE_ACK command through control
  vq.
\end{enumerate}

\drivernormative{\subparagraph}{Gratuitous Packet Sending}{Device Types / Network Device / Device Operation / Control Virtqueue / Gratuitous Packet Sending}

If the driver negotiates VIRTIO_NET_F_GUEST_ANNOUNCE, it SHOULD notify
network peers of its new location after it sees the VIRTIO_NET_S_ANNOUNCE bit
in \field{status}.  The driver MUST send a command on the command queue
with class VIRTIO_NET_CTRL_ANNOUNCE and command VIRTIO_NET_CTRL_ANNOUNCE_ACK.

\devicenormative{\subparagraph}{Gratuitous Packet Sending}{Device Types / Network Device / Device Operation / Control Virtqueue / Gratuitous Packet Sending}

If VIRTIO_NET_F_GUEST_ANNOUNCE is negotiated, the device MUST clear the
VIRTIO_NET_S_ANNOUNCE bit in \field{status} upon receipt of a command buffer
with class VIRTIO_NET_CTRL_ANNOUNCE and command VIRTIO_NET_CTRL_ANNOUNCE_ACK
before marking the buffer as used.

\paragraph{Device operation in multiqueue mode}\label{sec:Device Types / Network Device / Device Operation / Control Virtqueue / Device operation in multiqueue mode}

This specification defines the following modes that a device MAY implement for operation with multiple transmit/receive virtqueues:
\begin{itemize}
\item Automatic receive steering as defined in \ref{sec:Device Types / Network Device / Device Operation / Control Virtqueue / Automatic receive steering in multiqueue mode}.
 If a device supports this mode, it offers the VIRTIO_NET_F_MQ feature bit.
\item Receive-side scaling as defined in \ref{devicenormative:Device Types / Network Device / Device Operation / Control Virtqueue / Receive-side scaling (RSS) / RSS processing}.
 If a device supports this mode, it offers the VIRTIO_NET_F_RSS feature bit.
\end{itemize}

A device MAY support one of these features or both. The driver MAY negotiate any set of these features that the device supports.

Multiqueue is disabled by default.

The driver enables multiqueue by sending a command using \field{class} VIRTIO_NET_CTRL_MQ. The \field{command} selects the mode of multiqueue operation, as follows:
\begin{lstlisting}
#define VIRTIO_NET_CTRL_MQ    4
 #define VIRTIO_NET_CTRL_MQ_VQ_PAIRS_SET        0 (for automatic receive steering)
 #define VIRTIO_NET_CTRL_MQ_RSS_CONFIG          1 (for configurable receive steering)
 #define VIRTIO_NET_CTRL_MQ_HASH_CONFIG         2 (for configurable hash calculation)
\end{lstlisting}

If more than one multiqueue mode is negotiated, the resulting device configuration is defined by the last command sent by the driver.

\paragraph{Automatic receive steering in multiqueue mode}\label{sec:Device Types / Network Device / Device Operation / Control Virtqueue / Automatic receive steering in multiqueue mode}

If the driver negotiates the VIRTIO_NET_F_MQ feature bit (depends on VIRTIO_NET_F_CTRL_VQ), it MAY transmit outgoing packets on one
of the multiple transmitq1\ldots transmitqN and ask the device to
queue incoming packets into one of the multiple receiveq1\ldots receiveqN
depending on the packet flow.

The driver enables multiqueue by
sending the VIRTIO_NET_CTRL_MQ_VQ_PAIRS_SET command, specifying
the number of the transmit and receive queues to be used up to
\field{max_virtqueue_pairs}; subsequently,
transmitq1\ldots transmitqn and receiveq1\ldots receiveqn where
n=\field{virtqueue_pairs} MAY be used.
\begin{lstlisting}
struct virtio_net_ctrl_mq_pairs_set {
       le16 virtqueue_pairs;
};
#define VIRTIO_NET_CTRL_MQ_VQ_PAIRS_MIN        1
#define VIRTIO_NET_CTRL_MQ_VQ_PAIRS_MAX        0x8000

\end{lstlisting}

When multiqueue is enabled by VIRTIO_NET_CTRL_MQ_VQ_PAIRS_SET command, the device MUST use automatic receive steering
based on packet flow. Programming of the receive steering
classificator is implicit. After the driver transmitted a packet of a
flow on transmitqX, the device SHOULD cause incoming packets for that flow to
be steered to receiveqX. For uni-directional protocols, or where
no packets have been transmitted yet, the device MAY steer a packet
to a random queue out of the specified receiveq1\ldots receiveqn.

Multiqueue is disabled by VIRTIO_NET_CTRL_MQ_VQ_PAIRS_SET with \field{virtqueue_pairs} to 1 (this is
the default) and waiting for the device to use the command buffer.

\drivernormative{\subparagraph}{Automatic receive steering in multiqueue mode}{Device Types / Network Device / Device Operation / Control Virtqueue / Automatic receive steering in multiqueue mode}

The driver MUST configure the virtqueues before enabling them with the
VIRTIO_NET_CTRL_MQ_VQ_PAIRS_SET command.

The driver MUST NOT request a \field{virtqueue_pairs} of 0 or
greater than \field{max_virtqueue_pairs} in the device configuration space.

The driver MUST queue packets only on any transmitq1 before the
VIRTIO_NET_CTRL_MQ_VQ_PAIRS_SET command.

The driver MUST NOT queue packets on transmit queues greater than
\field{virtqueue_pairs} once it has placed the VIRTIO_NET_CTRL_MQ_VQ_PAIRS_SET command in the available ring.

\devicenormative{\subparagraph}{Automatic receive steering in multiqueue mode}{Device Types / Network Device / Device Operation / Control Virtqueue / Automatic receive steering in multiqueue mode}

After initialization of reset, the device MUST queue packets only on receiveq1.

The device MUST NOT queue packets on receive queues greater than
\field{virtqueue_pairs} once it has placed the
VIRTIO_NET_CTRL_MQ_VQ_PAIRS_SET command in a used buffer.

If the destination receive queue is being reset (See \ref{sec:Basic Facilities of a Virtio Device / Virtqueues / Virtqueue Reset}),
the device SHOULD re-select another random queue. If all receive queues are
being reset, the device MUST drop the packet.

\subparagraph{Legacy Interface: Automatic receive steering in multiqueue mode}\label{sec:Device Types / Network Device / Device Operation / Control Virtqueue / Automatic receive steering in multiqueue mode / Legacy Interface: Automatic receive steering in multiqueue mode}
When using the legacy interface, transitional devices and drivers
MUST format \field{virtqueue_pairs}
according to the native endian of the guest rather than
(necessarily when not using the legacy interface) little-endian.

\subparagraph{Hash calculation}\label{sec:Device Types / Network Device / Device Operation / Control Virtqueue / Automatic receive steering in multiqueue mode / Hash calculation}
If VIRTIO_NET_F_HASH_REPORT was negotiated and the device uses automatic receive steering,
the device MUST support a command to configure hash calculation parameters.

The driver provides parameters for hash calculation as follows:

\field{class} VIRTIO_NET_CTRL_MQ, \field{command} VIRTIO_NET_CTRL_MQ_HASH_CONFIG.

The \field{command-specific-data} has following format:
\begin{lstlisting}
struct virtio_net_hash_config {
    le32 hash_types;
    le16 reserved[4];
    u8 hash_key_length;
    u8 hash_key_data[hash_key_length];
};
\end{lstlisting}
Field \field{hash_types} contains a bitmask of allowed hash types as
defined in
\ref{sec:Device Types / Network Device / Device Operation / Processing of Incoming Packets / Hash calculation for incoming packets / Supported/enabled hash types}.
Initially the device has all hash types disabled and reports only VIRTIO_NET_HASH_REPORT_NONE.

Field \field{reserved} MUST contain zeroes. It is defined to make the structure to match the layout of virtio_net_rss_config structure,
defined in \ref{sec:Device Types / Network Device / Device Operation / Control Virtqueue / Receive-side scaling (RSS)}.

Fields \field{hash_key_length} and \field{hash_key_data} define the key to be used in hash calculation.

\paragraph{Receive-side scaling (RSS)}\label{sec:Device Types / Network Device / Device Operation / Control Virtqueue / Receive-side scaling (RSS)}
A device offers the feature VIRTIO_NET_F_RSS if it supports RSS receive steering with Toeplitz hash calculation and configurable parameters.

A driver queries RSS capabilities of the device by reading device configuration as defined in \ref{sec:Device Types / Network Device / Device configuration layout}

\subparagraph{Setting RSS parameters}\label{sec:Device Types / Network Device / Device Operation / Control Virtqueue / Receive-side scaling (RSS) / Setting RSS parameters}

Driver sends a VIRTIO_NET_CTRL_MQ_RSS_CONFIG command using the following format for \field{command-specific-data}:
\begin{lstlisting}
struct rss_rq_id {
   le16 vq_index_1_16: 15; /* Bits 1 to 16 of the virtqueue index */
   le16 reserved: 1; /* Set to zero */
};

struct virtio_net_rss_config {
    le32 hash_types;
    le16 indirection_table_mask;
    struct rss_rq_id unclassified_queue;
    struct rss_rq_id indirection_table[indirection_table_length];
    le16 max_tx_vq;
    u8 hash_key_length;
    u8 hash_key_data[hash_key_length];
};
\end{lstlisting}
Field \field{hash_types} contains a bitmask of allowed hash types as
defined in
\ref{sec:Device Types / Network Device / Device Operation / Processing of Incoming Packets / Hash calculation for incoming packets / Supported/enabled hash types}.

Field \field{indirection_table_mask} is a mask to be applied to
the calculated hash to produce an index in the
\field{indirection_table} array.
Number of entries in \field{indirection_table} is (\field{indirection_table_mask} + 1).

\field{rss_rq_id} is a receive virtqueue id. \field{vq_index_1_16}
consists of bits 1 to 16 of a virtqueue index. For example, a
\field{vq_index_1_16} value of 3 corresponds to virtqueue index 6,
which maps to receiveq4.

Field \field{unclassified_queue} specifies the receive virtqueue id in which to
place unclassified packets.

Field \field{indirection_table} is an array of receive virtqueues ids.

A driver sets \field{max_tx_vq} to inform a device how many transmit virtqueues it may use (transmitq1\ldots transmitq \field{max_tx_vq}).

Fields \field{hash_key_length} and \field{hash_key_data} define the key to be used in hash calculation.

\drivernormative{\subparagraph}{Setting RSS parameters}{Device Types / Network Device / Device Operation / Control Virtqueue / Receive-side scaling (RSS) }

A driver MUST NOT send the VIRTIO_NET_CTRL_MQ_RSS_CONFIG command if the feature VIRTIO_NET_F_RSS has not been negotiated.

A driver MUST fill the \field{indirection_table} array only with
enabled receive virtqueues ids.

The number of entries in \field{indirection_table} (\field{indirection_table_mask} + 1) MUST be a power of two.

A driver MUST use \field{indirection_table_mask} values that are less than \field{rss_max_indirection_table_length} reported by a device.

A driver MUST NOT set any VIRTIO_NET_HASH_TYPE_ flags that are not supported by a device.

\devicenormative{\subparagraph}{RSS processing}{Device Types / Network Device / Device Operation / Control Virtqueue / Receive-side scaling (RSS) / RSS processing}
The device MUST determine the destination queue for a network packet as follows:
\begin{itemize}
\item Calculate the hash of the packet as defined in \ref{sec:Device Types / Network Device / Device Operation / Processing of Incoming Packets / Hash calculation for incoming packets}.
\item If the device did not calculate the hash for the specific packet, the device directs the packet to the receiveq specified by \field{unclassified_queue} of virtio_net_rss_config structure.
\item Apply \field{indirection_table_mask} to the calculated hash
and use the result as the index in the indirection table to get
the destination receive virtqueue id.
\item If the destination receive queue is being reset (See \ref{sec:Basic Facilities of a Virtio Device / Virtqueues / Virtqueue Reset}), the device MUST drop the packet.
\end{itemize}

\paragraph{RSS Context}\label{sec:Device Types / Network Device / Device Operation / Control Virtqueue / RSS Context}

An RSS context consists of configurable parameters specified by \ref{sec:Device Types / Network Device
/ Device Operation / Control Virtqueue / Receive-side scaling (RSS)}.

The RSS configuration supported by VIRTIO_NET_F_RSS is considered the default RSS configuration.

The device offers the feature VIRTIO_NET_F_RSS_CONTEXT if it supports one or multiple RSS contexts
(excluding the default RSS configuration) and configurable parameters.

\subparagraph{Querying RSS Context Capability}\label{sec:Device Types / Network Device / Device Operation / Control Virtqueue / RSS Context / Querying RSS Context Capability}

\begin{lstlisting}
#define VIRTNET_RSS_CTX_CTRL 9
 #define VIRTNET_RSS_CTX_CTRL_CAP_GET  0
 #define VIRTNET_RSS_CTX_CTRL_ADD      1
 #define VIRTNET_RSS_CTX_CTRL_MOD      2
 #define VIRTNET_RSS_CTX_CTRL_DEL      3

struct virtnet_rss_ctx_cap {
    le16 max_rss_contexts;
}
\end{lstlisting}

Field \field{max_rss_contexts} specifies the maximum number of RSS contexts \ref{sec:Device Types / Network Device /
Device Operation / Control Virtqueue / RSS Context} supported by the device.

The driver queries the RSS context capability of the device by sending the command VIRTNET_RSS_CTX_CTRL_CAP_GET
with the structure virtnet_rss_ctx_cap.

For the command VIRTNET_RSS_CTX_CTRL_CAP_GET, the structure virtnet_rss_ctx_cap is write-only for the device.

\subparagraph{Setting RSS Context Parameters}\label{sec:Device Types / Network Device / Device Operation / Control Virtqueue / RSS Context / Setting RSS Context Parameters}

\begin{lstlisting}
struct virtnet_rss_ctx_add_modify {
    le16 rss_ctx_id;
    u8 reserved[6];
    struct virtio_net_rss_config rss;
};

struct virtnet_rss_ctx_del {
    le16 rss_ctx_id;
};
\end{lstlisting}

RSS context parameters:
\begin{itemize}
\item  \field{rss_ctx_id}: ID of the specific RSS context.
\item  \field{rss}: RSS context parameters of the specific RSS context whose id is \field{rss_ctx_id}.
\end{itemize}

\field{reserved} is reserved and it is ignored by the device.

If the feature VIRTIO_NET_F_RSS_CONTEXT has been negotiated, the driver can send the following
VIRTNET_RSS_CTX_CTRL class commands:
\begin{enumerate}
\item VIRTNET_RSS_CTX_CTRL_ADD: use the structure virtnet_rss_ctx_add_modify to
       add an RSS context configured as \field{rss} and id as \field{rss_ctx_id} for the device.
\item VIRTNET_RSS_CTX_CTRL_MOD: use the structure virtnet_rss_ctx_add_modify to
       configure parameters of the RSS context whose id is \field{rss_ctx_id} as \field{rss} for the device.
\item VIRTNET_RSS_CTX_CTRL_DEL: use the structure virtnet_rss_ctx_del to delete
       the RSS context whose id is \field{rss_ctx_id} for the device.
\end{enumerate}

For commands VIRTNET_RSS_CTX_CTRL_ADD and VIRTNET_RSS_CTX_CTRL_MOD, the structure virtnet_rss_ctx_add_modify is read-only for the device.
For the command VIRTNET_RSS_CTX_CTRL_DEL, the structure virtnet_rss_ctx_del is read-only for the device.

\devicenormative{\subparagraph}{RSS Context}{Device Types / Network Device / Device Operation / Control Virtqueue / RSS Context}

The device MUST set \field{max_rss_contexts} to at least 1 if it offers VIRTIO_NET_F_RSS_CONTEXT.

Upon reset, the device MUST clear all previously configured RSS contexts.

\drivernormative{\subparagraph}{RSS Context}{Device Types / Network Device / Device Operation / Control Virtqueue / RSS Context}

The driver MUST have negotiated the VIRTIO_NET_F_RSS_CONTEXT feature when issuing the VIRTNET_RSS_CTX_CTRL class commands.

The driver MUST set \field{rss_ctx_id} to between 1 and \field{max_rss_contexts} inclusive.

The driver MUST NOT send the command VIRTIO_NET_CTRL_MQ_VQ_PAIRS_SET when the device has successfully configured at least one RSS context.

\paragraph{Offloads State Configuration}\label{sec:Device Types / Network Device / Device Operation / Control Virtqueue / Offloads State Configuration}

If the VIRTIO_NET_F_CTRL_GUEST_OFFLOADS feature is negotiated, the driver can
send control commands for dynamic offloads state configuration.

\subparagraph{Setting Offloads State}\label{sec:Device Types / Network Device / Device Operation / Control Virtqueue / Offloads State Configuration / Setting Offloads State}

To configure the offloads, the following layout structure and
definitions are used:

\begin{lstlisting}
le64 offloads;

#define VIRTIO_NET_F_GUEST_CSUM       1
#define VIRTIO_NET_F_GUEST_TSO4       7
#define VIRTIO_NET_F_GUEST_TSO6       8
#define VIRTIO_NET_F_GUEST_ECN        9
#define VIRTIO_NET_F_GUEST_UFO        10
#define VIRTIO_NET_F_GUEST_UDP_TUNNEL_GSO  46
#define VIRTIO_NET_F_GUEST_UDP_TUNNEL_GSO_CSUM 47
#define VIRTIO_NET_F_GUEST_USO4       54
#define VIRTIO_NET_F_GUEST_USO6       55

#define VIRTIO_NET_CTRL_GUEST_OFFLOADS       5
 #define VIRTIO_NET_CTRL_GUEST_OFFLOADS_SET   0
\end{lstlisting}

The class VIRTIO_NET_CTRL_GUEST_OFFLOADS has one command:
VIRTIO_NET_CTRL_GUEST_OFFLOADS_SET applies the new offloads configuration.

le64 value passed as command data is a bitmask, bits set define
offloads to be enabled, bits cleared - offloads to be disabled.

There is a corresponding device feature for each offload. Upon feature
negotiation corresponding offload gets enabled to preserve backward
compatibility.

\drivernormative{\subparagraph}{Setting Offloads State}{Device Types / Network Device / Device Operation / Control Virtqueue / Offloads State Configuration / Setting Offloads State}

A driver MUST NOT enable an offload for which the appropriate feature
has not been negotiated.

\subparagraph{Legacy Interface: Setting Offloads State}\label{sec:Device Types / Network Device / Device Operation / Control Virtqueue / Offloads State Configuration / Setting Offloads State / Legacy Interface: Setting Offloads State}
When using the legacy interface, transitional devices and drivers
MUST format \field{offloads}
according to the native endian of the guest rather than
(necessarily when not using the legacy interface) little-endian.


\paragraph{Notifications Coalescing}\label{sec:Device Types / Network Device / Device Operation / Control Virtqueue / Notifications Coalescing}

If the VIRTIO_NET_F_NOTF_COAL feature is negotiated, the driver can
send commands VIRTIO_NET_CTRL_NOTF_COAL_TX_SET and VIRTIO_NET_CTRL_NOTF_COAL_RX_SET
for notification coalescing.

If the VIRTIO_NET_F_VQ_NOTF_COAL feature is negotiated, the driver can
send commands VIRTIO_NET_CTRL_NOTF_COAL_VQ_SET and VIRTIO_NET_CTRL_NOTF_COAL_VQ_GET
for virtqueue notification coalescing.

\begin{lstlisting}
struct virtio_net_ctrl_coal {
    le32 max_packets;
    le32 max_usecs;
};

struct virtio_net_ctrl_coal_vq {
    le16 vq_index;
    le16 reserved;
    struct virtio_net_ctrl_coal coal;
};

#define VIRTIO_NET_CTRL_NOTF_COAL 6
 #define VIRTIO_NET_CTRL_NOTF_COAL_TX_SET  0
 #define VIRTIO_NET_CTRL_NOTF_COAL_RX_SET 1
 #define VIRTIO_NET_CTRL_NOTF_COAL_VQ_SET 2
 #define VIRTIO_NET_CTRL_NOTF_COAL_VQ_GET 3
\end{lstlisting}

Coalescing parameters:
\begin{itemize}
\item \field{vq_index}: The virtqueue index of an enabled transmit or receive virtqueue.
\item \field{max_usecs} for RX: Maximum number of microseconds to delay a RX notification.
\item \field{max_usecs} for TX: Maximum number of microseconds to delay a TX notification.
\item \field{max_packets} for RX: Maximum number of packets to receive before a RX notification.
\item \field{max_packets} for TX: Maximum number of packets to send before a TX notification.
\end{itemize}

\field{reserved} is reserved and it is ignored by the device.

Read/Write attributes for coalescing parameters:
\begin{itemize}
\item For commands VIRTIO_NET_CTRL_NOTF_COAL_TX_SET and VIRTIO_NET_CTRL_NOTF_COAL_RX_SET, the structure virtio_net_ctrl_coal is write-only for the driver.
\item For the command VIRTIO_NET_CTRL_NOTF_COAL_VQ_SET, the structure virtio_net_ctrl_coal_vq is write-only for the driver.
\item For the command VIRTIO_NET_CTRL_NOTF_COAL_VQ_GET, \field{vq_index} and \field{reserved} are write-only
      for the driver, and the structure virtio_net_ctrl_coal is read-only for the driver.
\end{itemize}

The class VIRTIO_NET_CTRL_NOTF_COAL has the following commands:
\begin{enumerate}
\item VIRTIO_NET_CTRL_NOTF_COAL_TX_SET: use the structure virtio_net_ctrl_coal to set the \field{max_usecs} and \field{max_packets} parameters for all transmit virtqueues.
\item VIRTIO_NET_CTRL_NOTF_COAL_RX_SET: use the structure virtio_net_ctrl_coal to set the \field{max_usecs} and \field{max_packets} parameters for all receive virtqueues.
\item VIRTIO_NET_CTRL_NOTF_COAL_VQ_SET: use the structure virtio_net_ctrl_coal_vq to set the \field{max_usecs} and \field{max_packets} parameters
                                        for an enabled transmit/receive virtqueue whose index is \field{vq_index}.
\item VIRTIO_NET_CTRL_NOTF_COAL_VQ_GET: use the structure virtio_net_ctrl_coal_vq to get the \field{max_usecs} and \field{max_packets} parameters
                                        for an enabled transmit/receive virtqueue whose index is \field{vq_index}.
\end{enumerate}

The device may generate notifications more or less frequently than specified by set commands of the VIRTIO_NET_CTRL_NOTF_COAL class.

If coalescing parameters are being set, the device applies the last coalescing parameters set for a
virtqueue, regardless of the command used to set the parameters. Use the following command sequence
with two pairs of virtqueues as an example:
Each of the following commands sets \field{max_usecs} and \field{max_packets} parameters for virtqueues.
\begin{itemize}
\item Command1: VIRTIO_NET_CTRL_NOTF_COAL_RX_SET sets coalescing parameters for virtqueues having index 0 and index 2. Virtqueues having index 1 and index 3 retain their previous parameters.
\item Command2: VIRTIO_NET_CTRL_NOTF_COAL_VQ_SET with \field{vq_index} = 0 sets coalescing parameters for virtqueue having index 0. Virtqueue having index 2 retains the parameters from command1.
\item Command3: VIRTIO_NET_CTRL_NOTF_COAL_VQ_GET with \field{vq_index} = 0, the device responds with coalescing parameters of vq_index 0 set by command2.
\item Command4: VIRTIO_NET_CTRL_NOTF_COAL_VQ_SET with \field{vq_index} = 1 sets coalescing parameters for virtqueue having index 1. Virtqueue having index 3 retains its previous parameters.
\item Command5: VIRTIO_NET_CTRL_NOTF_COAL_TX_SET sets coalescing parameters for virtqueues having index 1 and index 3, and overrides the parameters set by command4.
\item Command6: VIRTIO_NET_CTRL_NOTF_COAL_VQ_GET with \field{vq_index} = 1, the device responds with coalescing parameters of index 1 set by command5.
\end{itemize}

\subparagraph{Operation}\label{sec:Device Types / Network Device / Device Operation / Control Virtqueue / Notifications Coalescing / Operation}

The device sends a used buffer notification once the notification conditions are met and if the notifications are not suppressed as explained in \ref{sec:Basic Facilities of a Virtio Device / Virtqueues / Used Buffer Notification Suppression}.

When the device has non-zero \field{max_usecs} and non-zero \field{max_packets}, it starts counting microseconds and packets upon receiving/sending a packet.
The device counts packets and microseconds for each receive virtqueue and transmit virtqueue separately.
In this case, the notification conditions are met when \field{max_usecs} microseconds elapse, or upon sending/receiving \field{max_packets} packets, whichever happens first.
Afterwards, the device waits for the next packet and starts counting packets and microseconds again.

When the device has \field{max_usecs} = 0 or \field{max_packets} = 0, the notification conditions are met after every packet received/sent.

\subparagraph{RX Example}\label{sec:Device Types / Network Device / Device Operation / Control Virtqueue / Notifications Coalescing / RX Example}

If, for example:
\begin{itemize}
\item \field{max_usecs} = 10.
\item \field{max_packets} = 15.
\end{itemize}
then each receive virtqueue of a device will operate as follows:
\begin{itemize}
\item The device will count packets received on each virtqueue until it accumulates 15, or until 10 microseconds elapsed since the first one was received.
\item If the notifications are not suppressed by the driver, the device will send an used buffer notification, otherwise, the device will not send an used buffer notification as long as the notifications are suppressed.
\end{itemize}

\subparagraph{TX Example}\label{sec:Device Types / Network Device / Device Operation / Control Virtqueue / Notifications Coalescing / TX Example}

If, for example:
\begin{itemize}
\item \field{max_usecs} = 10.
\item \field{max_packets} = 15.
\end{itemize}
then each transmit virtqueue of a device will operate as follows:
\begin{itemize}
\item The device will count packets sent on each virtqueue until it accumulates 15, or until 10 microseconds elapsed since the first one was sent.
\item If the notifications are not suppressed by the driver, the device will send an used buffer notification, otherwise, the device will not send an used buffer notification as long as the notifications are suppressed.
\end{itemize}

\subparagraph{Notifications When Coalescing Parameters Change}\label{sec:Device Types / Network Device / Device Operation / Control Virtqueue / Notifications Coalescing / Notifications When Coalescing Parameters Change}

When the coalescing parameters of a device change, the device needs to check if the new notification conditions are met and send a used buffer notification if so.

For example, \field{max_packets} = 15 for a device with a single transmit virtqueue: if the device sends 10 packets and afterwards receives a
VIRTIO_NET_CTRL_NOTF_COAL_TX_SET command with \field{max_packets} = 8, then the notification condition is immediately considered to be met;
the device needs to immediately send a used buffer notification, if the notifications are not suppressed by the driver.

\drivernormative{\subparagraph}{Notifications Coalescing}{Device Types / Network Device / Device Operation / Control Virtqueue / Notifications Coalescing}

The driver MUST set \field{vq_index} to the virtqueue index of an enabled transmit or receive virtqueue.

The driver MUST have negotiated the VIRTIO_NET_F_NOTF_COAL feature when issuing commands VIRTIO_NET_CTRL_NOTF_COAL_TX_SET and VIRTIO_NET_CTRL_NOTF_COAL_RX_SET.

The driver MUST have negotiated the VIRTIO_NET_F_VQ_NOTF_COAL feature when issuing commands VIRTIO_NET_CTRL_NOTF_COAL_VQ_SET and VIRTIO_NET_CTRL_NOTF_COAL_VQ_GET.

The driver MUST ignore the values of coalescing parameters received from the VIRTIO_NET_CTRL_NOTF_COAL_VQ_GET command if the device responds with VIRTIO_NET_ERR.

\devicenormative{\subparagraph}{Notifications Coalescing}{Device Types / Network Device / Device Operation / Control Virtqueue / Notifications Coalescing}

The device MUST ignore \field{reserved}.

The device SHOULD respond to VIRTIO_NET_CTRL_NOTF_COAL_TX_SET and VIRTIO_NET_CTRL_NOTF_COAL_RX_SET commands with VIRTIO_NET_ERR if it was not able to change the parameters.

The device MUST respond to the VIRTIO_NET_CTRL_NOTF_COAL_VQ_SET command with VIRTIO_NET_ERR if it was not able to change the parameters.

The device MUST respond to VIRTIO_NET_CTRL_NOTF_COAL_VQ_SET and VIRTIO_NET_CTRL_NOTF_COAL_VQ_GET commands with
VIRTIO_NET_ERR if the designated virtqueue is not an enabled transmit or receive virtqueue.

Upon disabling and re-enabling a transmit virtqueue, the device MUST set the coalescing parameters of the virtqueue
to those configured through the VIRTIO_NET_CTRL_NOTF_COAL_TX_SET command, or, if the driver did not set any TX coalescing parameters, to 0.

Upon disabling and re-enabling a receive virtqueue, the device MUST set the coalescing parameters of the virtqueue
to those configured through the VIRTIO_NET_CTRL_NOTF_COAL_RX_SET command, or, if the driver did not set any RX coalescing parameters, to 0.

The behavior of the device in response to set commands of the VIRTIO_NET_CTRL_NOTF_COAL class is best-effort:
the device MAY generate notifications more or less frequently than specified.

A device SHOULD NOT send used buffer notifications to the driver if the notifications are suppressed, even if the notification conditions are met.

Upon reset, a device MUST initialize all coalescing parameters to 0.

\paragraph{Device Statistics}\label{sec:Device Types / Network Device / Device Operation / Control Virtqueue / Device Statistics}

If the VIRTIO_NET_F_DEVICE_STATS feature is negotiated, the driver can obtain
device statistics from the device by using the following command.

Different types of virtqueues have different statistics. The statistics of the
receiveq are different from those of the transmitq.

The statistics of a certain type of virtqueue are also divided into multiple types
because different types require different features. This enables the expansion
of new statistics.

In one command, the driver can obtain the statistics of one or multiple virtqueues.
Additionally, the driver can obtain multiple type statistics of each virtqueue.

\subparagraph{Query Statistic Capabilities}\label{sec:Device Types / Network Device / Device Operation / Control Virtqueue / Device Statistics / Query Statistic Capabilities}

\begin{lstlisting}
#define VIRTIO_NET_CTRL_STATS         8
#define VIRTIO_NET_CTRL_STATS_QUERY   0
#define VIRTIO_NET_CTRL_STATS_GET     1

struct virtio_net_stats_capabilities {

#define VIRTIO_NET_STATS_TYPE_CVQ       (1 << 32)

#define VIRTIO_NET_STATS_TYPE_RX_BASIC  (1 << 0)
#define VIRTIO_NET_STATS_TYPE_RX_CSUM   (1 << 1)
#define VIRTIO_NET_STATS_TYPE_RX_GSO    (1 << 2)
#define VIRTIO_NET_STATS_TYPE_RX_SPEED  (1 << 3)

#define VIRTIO_NET_STATS_TYPE_TX_BASIC  (1 << 16)
#define VIRTIO_NET_STATS_TYPE_TX_CSUM   (1 << 17)
#define VIRTIO_NET_STATS_TYPE_TX_GSO    (1 << 18)
#define VIRTIO_NET_STATS_TYPE_TX_SPEED  (1 << 19)

    le64 supported_stats_types[1];
}
\end{lstlisting}

To obtain device statistic capability, use the VIRTIO_NET_CTRL_STATS_QUERY
command. When the command completes successfully, \field{command-specific-result}
is in the format of \field{struct virtio_net_stats_capabilities}.

\subparagraph{Get Statistics}\label{sec:Device Types / Network Device / Device Operation / Control Virtqueue / Device Statistics / Get Statistics}

\begin{lstlisting}
struct virtio_net_ctrl_queue_stats {
       struct {
           le16 vq_index;
           le16 reserved[3];
           le64 types_bitmap[1];
       } stats[];
};

struct virtio_net_stats_reply_hdr {
#define VIRTIO_NET_STATS_TYPE_REPLY_CVQ       32

#define VIRTIO_NET_STATS_TYPE_REPLY_RX_BASIC  0
#define VIRTIO_NET_STATS_TYPE_REPLY_RX_CSUM   1
#define VIRTIO_NET_STATS_TYPE_REPLY_RX_GSO    2
#define VIRTIO_NET_STATS_TYPE_REPLY_RX_SPEED  3

#define VIRTIO_NET_STATS_TYPE_REPLY_TX_BASIC  16
#define VIRTIO_NET_STATS_TYPE_REPLY_TX_CSUM   17
#define VIRTIO_NET_STATS_TYPE_REPLY_TX_GSO    18
#define VIRTIO_NET_STATS_TYPE_REPLY_TX_SPEED  19
    u8 type;
    u8 reserved;
    le16 vq_index;
    le16 reserved1;
    le16 size;
}
\end{lstlisting}

To obtain device statistics, use the VIRTIO_NET_CTRL_STATS_GET command with the
\field{command-specific-data} which is in the format of
\field{struct virtio_net_ctrl_queue_stats}. When the command completes
successfully, \field{command-specific-result} contains multiple statistic
results, each statistic result has the \field{struct virtio_net_stats_reply_hdr}
as the header.

The fields of the \field{struct virtio_net_ctrl_queue_stats}:
\begin{description}
    \item [vq_index]
        The index of the virtqueue to obtain the statistics.

    \item [types_bitmap]
        This is a bitmask of the types of statistics to be obtained. Therefore, a
        \field{stats} inside \field{struct virtio_net_ctrl_queue_stats} may
        indicate multiple statistic replies for the virtqueue.
\end{description}

The fields of the \field{struct virtio_net_stats_reply_hdr}:
\begin{description}
    \item [type]
        The type of the reply statistic.

    \item [vq_index]
        The virtqueue index of the reply statistic.

    \item [size]
        The number of bytes for the statistics entry including size of \field{struct virtio_net_stats_reply_hdr}.

\end{description}

\subparagraph{Controlq Statistics}\label{sec:Device Types / Network Device / Device Operation / Control Virtqueue / Device Statistics / Controlq Statistics}

The structure corresponding to the controlq statistics is
\field{struct virtio_net_stats_cvq}. The corresponding type is
VIRTIO_NET_STATS_TYPE_CVQ. This is for the controlq.

\begin{lstlisting}
struct virtio_net_stats_cvq {
    struct virtio_net_stats_reply_hdr hdr;

    le64 command_num;
    le64 ok_num;
};
\end{lstlisting}

\begin{description}
    \item [command_num]
        The number of commands received by the device including the current command.

    \item [ok_num]
        The number of commands completed successfully by the device including the current command.
\end{description}


\subparagraph{Receiveq Basic Statistics}\label{sec:Device Types / Network Device / Device Operation / Control Virtqueue / Device Statistics / Receiveq Basic Statistics}

The structure corresponding to the receiveq basic statistics is
\field{struct virtio_net_stats_rx_basic}. The corresponding type is
VIRTIO_NET_STATS_TYPE_RX_BASIC. This is for the receiveq.

Receiveq basic statistics do not require any feature. As long as the device supports
VIRTIO_NET_F_DEVICE_STATS, the following are the receiveq basic statistics.

\begin{lstlisting}
struct virtio_net_stats_rx_basic {
    struct virtio_net_stats_reply_hdr hdr;

    le64 rx_notifications;

    le64 rx_packets;
    le64 rx_bytes;

    le64 rx_interrupts;

    le64 rx_drops;
    le64 rx_drop_overruns;
};
\end{lstlisting}

The packets described below were all presented on the specified virtqueue.
\begin{description}
    \item [rx_notifications]
        The number of driver notifications received by the device for this
        receiveq.

    \item [rx_packets]
        This is the number of packets passed to the driver by the device.

    \item [rx_bytes]
        This is the bytes of packets passed to the driver by the device.

    \item [rx_interrupts]
        The number of interrupts generated by the device for this receiveq.

    \item [rx_drops]
        This is the number of packets dropped by the device. The count includes
        all types of packets dropped by the device.

    \item [rx_drop_overruns]
        This is the number of packets dropped by the device when no more
        descriptors were available.

\end{description}

\subparagraph{Transmitq Basic Statistics}\label{sec:Device Types / Network Device / Device Operation / Control Virtqueue / Device Statistics / Transmitq Basic Statistics}

The structure corresponding to the transmitq basic statistics is
\field{struct virtio_net_stats_tx_basic}. The corresponding type is
VIRTIO_NET_STATS_TYPE_TX_BASIC. This is for the transmitq.

Transmitq basic statistics do not require any feature. As long as the device supports
VIRTIO_NET_F_DEVICE_STATS, the following are the transmitq basic statistics.

\begin{lstlisting}
struct virtio_net_stats_tx_basic {
    struct virtio_net_stats_reply_hdr hdr;

    le64 tx_notifications;

    le64 tx_packets;
    le64 tx_bytes;

    le64 tx_interrupts;

    le64 tx_drops;
    le64 tx_drop_malformed;
};
\end{lstlisting}

The packets described below are all for a specific virtqueue.
\begin{description}
    \item [tx_notifications]
        The number of driver notifications received by the device for this
        transmitq.

    \item [tx_packets]
        This is the number of packets sent by the device (not the packets
        got from the driver).

    \item [tx_bytes]
        This is the number of bytes sent by the device for all the sent packets
        (not the bytes sent got from the driver).

    \item [tx_interrupts]
        The number of interrupts generated by the device for this transmitq.

    \item [tx_drops]
        The number of packets dropped by the device. The count includes all
        types of packets dropped by the device.

    \item [tx_drop_malformed]
        The number of packets dropped by the device, when the descriptors are
        malformed. For example, the buffer is too short.
\end{description}

\subparagraph{Receiveq CSUM Statistics}\label{sec:Device Types / Network Device / Device Operation / Control Virtqueue / Device Statistics / Receiveq CSUM Statistics}

The structure corresponding to the receiveq checksum statistics is
\field{struct virtio_net_stats_rx_csum}. The corresponding type is
VIRTIO_NET_STATS_TYPE_RX_CSUM. This is for the receiveq.

Only after the VIRTIO_NET_F_GUEST_CSUM is negotiated, the receiveq checksum
statistics can be obtained.

\begin{lstlisting}
struct virtio_net_stats_rx_csum {
    struct virtio_net_stats_reply_hdr hdr;

    le64 rx_csum_valid;
    le64 rx_needs_csum;
    le64 rx_csum_none;
    le64 rx_csum_bad;
};
\end{lstlisting}

The packets described below were all presented on the specified virtqueue.
\begin{description}
    \item [rx_csum_valid]
        The number of packets with VIRTIO_NET_HDR_F_DATA_VALID.

    \item [rx_needs_csum]
        The number of packets with VIRTIO_NET_HDR_F_NEEDS_CSUM.

    \item [rx_csum_none]
        The number of packets without hardware checksum. The packet here refers
        to the non-TCP/UDP packet that the device cannot recognize.

    \item [rx_csum_bad]
        The number of packets with checksum mismatch.

\end{description}

\subparagraph{Transmitq CSUM Statistics}\label{sec:Device Types / Network Device / Device Operation / Control Virtqueue / Device Statistics / Transmitq CSUM Statistics}

The structure corresponding to the transmitq checksum statistics is
\field{struct virtio_net_stats_tx_csum}. The corresponding type is
VIRTIO_NET_STATS_TYPE_TX_CSUM. This is for the transmitq.

Only after the VIRTIO_NET_F_CSUM is negotiated, the transmitq checksum
statistics can be obtained.

The following are the transmitq checksum statistics:

\begin{lstlisting}
struct virtio_net_stats_tx_csum {
    struct virtio_net_stats_reply_hdr hdr;

    le64 tx_csum_none;
    le64 tx_needs_csum;
};
\end{lstlisting}

The packets described below are all for a specific virtqueue.
\begin{description}
    \item [tx_csum_none]
        The number of packets which do not require hardware checksum.

    \item [tx_needs_csum]
        The number of packets which require checksum calculation by the device.

\end{description}

\subparagraph{Receiveq GSO Statistics}\label{sec:Device Types / Network Device / Device Operation / Control Virtqueue / Device Statistics / Receiveq GSO Statistics}

The structure corresponding to the receivq GSO statistics is
\field{struct virtio_net_stats_rx_gso}. The corresponding type is
VIRTIO_NET_STATS_TYPE_RX_GSO. This is for the receiveq.

If one or more of the VIRTIO_NET_F_GUEST_TSO4, VIRTIO_NET_F_GUEST_TSO6
have been negotiated, the receiveq GSO statistics can be obtained.

GSO packets refer to packets passed by the device to the driver where
\field{gso_type} is not VIRTIO_NET_HDR_GSO_NONE.

\begin{lstlisting}
struct virtio_net_stats_rx_gso {
    struct virtio_net_stats_reply_hdr hdr;

    le64 rx_gso_packets;
    le64 rx_gso_bytes;
    le64 rx_gso_packets_coalesced;
    le64 rx_gso_bytes_coalesced;
};
\end{lstlisting}

The packets described below were all presented on the specified virtqueue.
\begin{description}
    \item [rx_gso_packets]
        The number of the GSO packets received by the device.

    \item [rx_gso_bytes]
        The bytes of the GSO packets received by the device.
        This includes the header size of the GSO packet.

    \item [rx_gso_packets_coalesced]
        The number of the GSO packets coalesced by the device.

    \item [rx_gso_bytes_coalesced]
        The bytes of the GSO packets coalesced by the device.
        This includes the header size of the GSO packet.
\end{description}

\subparagraph{Transmitq GSO Statistics}\label{sec:Device Types / Network Device / Device Operation / Control Virtqueue / Device Statistics / Transmitq GSO Statistics}

The structure corresponding to the transmitq GSO statistics is
\field{struct virtio_net_stats_tx_gso}. The corresponding type is
VIRTIO_NET_STATS_TYPE_TX_GSO. This is for the transmitq.

If one or more of the VIRTIO_NET_F_HOST_TSO4, VIRTIO_NET_F_HOST_TSO6,
VIRTIO_NET_F_HOST_USO options have been negotiated, the transmitq GSO statistics
can be obtained.

GSO packets refer to packets passed by the driver to the device where
\field{gso_type} is not VIRTIO_NET_HDR_GSO_NONE.
See more \ref{sec:Device Types / Network Device / Device Operation / Packet
Transmission}.

\begin{lstlisting}
struct virtio_net_stats_tx_gso {
    struct virtio_net_stats_reply_hdr hdr;

    le64 tx_gso_packets;
    le64 tx_gso_bytes;
    le64 tx_gso_segments;
    le64 tx_gso_segments_bytes;
    le64 tx_gso_packets_noseg;
    le64 tx_gso_bytes_noseg;
};
\end{lstlisting}

The packets described below are all for a specific virtqueue.
\begin{description}
    \item [tx_gso_packets]
        The number of the GSO packets sent by the device.

    \item [tx_gso_bytes]
        The bytes of the GSO packets sent by the device.

    \item [tx_gso_segments]
        The number of segments prepared from GSO packets.

    \item [tx_gso_segments_bytes]
        The bytes of segments prepared from GSO packets.

    \item [tx_gso_packets_noseg]
        The number of the GSO packets without segmentation.

    \item [tx_gso_bytes_noseg]
        The bytes of the GSO packets without segmentation.

\end{description}

\subparagraph{Receiveq Speed Statistics}\label{sec:Device Types / Network Device / Device Operation / Control Virtqueue / Device Statistics / Receiveq Speed Statistics}

The structure corresponding to the receiveq speed statistics is
\field{struct virtio_net_stats_rx_speed}. The corresponding type is
VIRTIO_NET_STATS_TYPE_RX_SPEED. This is for the receiveq.

The device has the allowance for the speed. If VIRTIO_NET_F_SPEED_DUPLEX has
been negotiated, the driver can get this by \field{speed}. When the received
packets bitrate exceeds the \field{speed}, some packets may be dropped by the
device.

\begin{lstlisting}
struct virtio_net_stats_rx_speed {
    struct virtio_net_stats_reply_hdr hdr;

    le64 rx_packets_allowance_exceeded;
    le64 rx_bytes_allowance_exceeded;
};
\end{lstlisting}

The packets described below were all presented on the specified virtqueue.
\begin{description}
    \item [rx_packets_allowance_exceeded]
        The number of the packets dropped by the device due to the received
        packets bitrate exceeding the \field{speed}.

    \item [rx_bytes_allowance_exceeded]
        The bytes of the packets dropped by the device due to the received
        packets bitrate exceeding the \field{speed}.

\end{description}

\subparagraph{Transmitq Speed Statistics}\label{sec:Device Types / Network Device / Device Operation / Control Virtqueue / Device Statistics / Transmitq Speed Statistics}

The structure corresponding to the transmitq speed statistics is
\field{struct virtio_net_stats_tx_speed}. The corresponding type is
VIRTIO_NET_STATS_TYPE_TX_SPEED. This is for the transmitq.

The device has the allowance for the speed. If VIRTIO_NET_F_SPEED_DUPLEX has
been negotiated, the driver can get this by \field{speed}. When the transmit
packets bitrate exceeds the \field{speed}, some packets may be dropped by the
device.

\begin{lstlisting}
struct virtio_net_stats_tx_speed {
    struct virtio_net_stats_reply_hdr hdr;

    le64 tx_packets_allowance_exceeded;
    le64 tx_bytes_allowance_exceeded;
};
\end{lstlisting}

The packets described below were all presented on the specified virtqueue.
\begin{description}
    \item [tx_packets_allowance_exceeded]
        The number of the packets dropped by the device due to the transmit packets
        bitrate exceeding the \field{speed}.

    \item [tx_bytes_allowance_exceeded]
        The bytes of the packets dropped by the device due to the transmit packets
        bitrate exceeding the \field{speed}.

\end{description}

\devicenormative{\subparagraph}{Device Statistics}{Device Types / Network Device / Device Operation / Control Virtqueue / Device Statistics}

When the VIRTIO_NET_F_DEVICE_STATS feature is negotiated, the device MUST reply
to the command VIRTIO_NET_CTRL_STATS_QUERY with the
\field{struct virtio_net_stats_capabilities}. \field{supported_stats_types}
includes all the statistic types supported by the device.

If \field{struct virtio_net_ctrl_queue_stats} is incorrect (such as the
following), the device MUST set \field{ack} to VIRTIO_NET_ERR. Even if there is
only one error, the device MUST fail the entire command.
\begin{itemize}
    \item \field{vq_index} exceeds the queue range.
    \item \field{types_bitmap} contains unknown types.
    \item One or more of the bits present in \field{types_bitmap} is not valid
        for the specified virtqueue.
    \item The feature corresponding to the specified \field{types_bitmap} was
        not negotiated.
\end{itemize}

The device MUST set the actual size of the bytes occupied by the reply to the
\field{size} of the \field{hdr}. And the device MUST set the \field{type} and
the \field{vq_index} of the statistic header.

The \field{command-specific-result} buffer allocated by the driver may be
smaller or bigger than all the statistics specified by
\field{struct virtio_net_ctrl_queue_stats}. The device MUST fill up only upto
the valid bytes.

The statistics counter replied by the device MUST wrap around to zero by the
device on the overflow.

\drivernormative{\subparagraph}{Device Statistics}{Device Types / Network Device / Device Operation / Control Virtqueue / Device Statistics}

The types contained in the \field{types_bitmap} MUST be queried from the device
via command VIRTIO_NET_CTRL_STATS_QUERY.

\field{types_bitmap} in \field{struct virtio_net_ctrl_queue_stats} MUST be valid to the
vq specified by \field{vq_index}.

The \field{command-specific-result} buffer allocated by the driver MUST have
enough capacity to store all the statistics reply headers defined in
\field{struct virtio_net_ctrl_queue_stats}. If the
\field{command-specific-result} buffer is fully utilized by the device but some
replies are missed, it is possible that some statistics may exceed the capacity
of the driver's records. In such cases, the driver should allocate additional
space for the \field{command-specific-result} buffer.

\subsubsection{Flow filter}\label{sec:Device Types / Network Device / Device Operation / Flow filter}

A network device can support one or more flow filter rules. Each flow filter rule
is applied by matching a packet and then taking an action, such as directing the packet
to a specific receiveq or dropping the packet. An example of a match is
matching on specific source and destination IP addresses.

A flow filter rule is a device resource object that consists of a key,
a processing priority, and an action to either direct a packet to a
receive queue or drop the packet.

Each rule uses a classifier. The key is matched against the packet using
a classifier, defining which fields in the packet are matched.
A classifier resource object consists of one or more field selectors, each with
a type that specifies the header fields to be matched against, and a mask.
The mask can match whole fields or parts of a field in a header. Each
rule resource object depends on the classifier resource object.

When a packet is received, relevant fields are extracted
(in the same way) from both the packet and the key according to the
classifier. The resulting field contents are then compared -
if they are identical the rule action is taken, if they are not, the rule is ignored.

Multiple flow filter rules are part of a group. The rule resource object
depends on the group. Each rule within a
group has a rule priority, and each group also has a group priority. For a
packet, a group with the highest priority is selected first. Within a group,
rules are applied from highest to lowest priority, until one of the rules
matches the packet and an action is taken. If all the rules within a group
are ignored, the group with the next highest priority is selected, and so on.

The device and the driver indicates flow filter resource limits using the capability
\ref{par:Device Types / Network Device / Device Operation / Flow filter / Device and driver capabilities / VIRTIO-NET-FF-RESOURCE-CAP} specifying the limits on the number of flow filter rule,
group and classifier resource objects. The capability \ref{par:Device Types / Network Device / Device Operation / Flow filter / Device and driver capabilities / VIRTIO-NET-FF-SELECTOR-CAP} specifies which selectors the device supports.
The driver indicates the selectors it is using by setting the flow
filter selector capability, prior to adding any resource objects.

The capability \ref{par:Device Types / Network Device / Device Operation / Flow filter / Device and driver capabilities / VIRTIO-NET-FF-ACTION-CAP} specifies which actions the device supports.

The driver controls the flow filter rule, classifier and group resource objects using
administration commands described in
\ref{sec:Basic Facilities of a Virtio Device / Device groups / Group administration commands / Device resource objects}.

\paragraph{Packet processing order}\label{sec:sec:Device Types / Network Device / Device Operation / Flow filter / Packet processing order}

Note that flow filter rules are applied after MAC/VLAN filtering. Flow filter
rules take precedence over steering: if a flow filter rule results in an action,
the steering configuration does not apply. The steering configuration only applies
to packets for which no flow filter rule action was performed. For example,
incoming packets can be processed in the following order:

\begin{itemize}
\item apply steering configuration received using control virtqueue commands
      VIRTIO_NET_CTRL_RX, VIRTIO_NET_CTRL_MAC and VIRTIO_NET_CTRL_VLAN.
\item apply flow filter rules if any.
\item if no filter rule applied, apply steering configuration received using command
      VIRTIO_NET_CTRL_MQ_RSS_CONFIG or as per automatic receive steering.
\end{itemize}

Some incoming packet processing examples:
\begin{itemize}
\item If the packet is dropped by the flow filter rule, RSS
      steering is ignored for the packet.
\item If the packet is directed to a specific receiveq using flow filter rule,
      the RSS steering is ignored for the packet.
\item If a packet is dropped due to the VIRTIO_NET_CTRL_MAC configuration,
      both flow filter rules and the RSS steering are ignored for the packet.
\item If a packet does not match any flow filter rules,
      the RSS steering is used to select the receiveq for the packet (if enabled).
\item If there are two flow filter groups configured as group_A and group_B
      with respective group priorities as 4, and 5; flow filter rules of
      group_B are applied first having highest group priority, if there is a match,
      the flow filter rules of group_A are ignored; if there is no match for
      the flow filter rules in group_B, the flow filter rules of next level group_A are applied.
\end{itemize}

\paragraph{Device and driver capabilities}
\label{par:Device Types / Network Device / Device Operation / Flow filter / Device and driver capabilities}

\subparagraph{VIRTIO_NET_FF_RESOURCE_CAP}
\label{par:Device Types / Network Device / Device Operation / Flow filter / Device and driver capabilities / VIRTIO-NET-FF-RESOURCE-CAP}

The capability VIRTIO_NET_FF_RESOURCE_CAP indicates the flow filter resource limits.
\field{cap_specific_data} is in the format
\field{struct virtio_net_ff_cap_data}.

\begin{lstlisting}
struct virtio_net_ff_cap_data {
        le32 groups_limit;
        le32 selectors_limit;
        le32 rules_limit;
        le32 rules_per_group_limit;
        u8 last_rule_priority;
        u8 selectors_per_classifier_limit;
};
\end{lstlisting}

\field{groups_limit}, and \field{selectors_limit} represent the maximum
number of flow filter groups and selectors, respectively, that the driver can create.
 \field{rules_limit} is the maximum number of
flow fiilter rules that the driver can create across all the groups.
\field{rules_per_group_limit} is the maximum number of flow filter rules that the driver
can create for each flow filter group.

\field{last_rule_priority} is the highest priority that can be assigned to a
flow filter rule.

\field{selectors_per_classifier_limit} is the maximum number of selectors
that a classifier can have.

\subparagraph{VIRTIO_NET_FF_SELECTOR_CAP}
\label{par:Device Types / Network Device / Device Operation / Flow filter / Device and driver capabilities / VIRTIO-NET-FF-SELECTOR-CAP}

The capability VIRTIO_NET_FF_SELECTOR_CAP lists the supported selectors and the
supported packet header fields for each selector.
\field{cap_specific_data} is in the format \field{struct virtio_net_ff_cap_mask_data}.

\begin{lstlisting}[label={lst:Device Types / Network Device / Device Operation / Flow filter / Device and driver capabilities / VIRTIO-NET-FF-SELECTOR-CAP / virtio-net-ff-selector}]
struct virtio_net_ff_selector {
        u8 type;
        u8 flags;
        u8 reserved[2];
        u8 length;
        u8 reserved1[3];
        u8 mask[];
};

struct virtio_net_ff_cap_mask_data {
        u8 count;
        u8 reserved[7];
        struct virtio_net_ff_selector selectors[];
};

#define VIRTIO_NET_FF_MASK_F_PARTIAL_MASK (1 << 0)
\end{lstlisting}

\field{count} indicates number of valid entries in the \field{selectors} array.
\field{selectors[]} is an array of supported selectors. Within each array entry:
\field{type} specifies the type of the packet header, as defined in table
\ref{table:Device Types / Network Device / Device Operation / Flow filter / Device and driver capabilities / VIRTIO-NET-FF-SELECTOR-CAP / flow filter selector types}. \field{mask} specifies which fields of the
packet header can be matched in a flow filter rule.

Each \field{type} is also listed in table
\ref{table:Device Types / Network Device / Device Operation / Flow filter / Device and driver capabilities / VIRTIO-NET-FF-SELECTOR-CAP / flow filter selector types}. \field{mask} is a byte array
in network byte order. For example, when \field{type} is VIRTIO_NET_FF_MASK_TYPE_IPV6,
the \field{mask} is in the format \hyperref[intro:IPv6-Header-Format]{IPv6 Header Format}.

If partial masking is not set, then all bits in each field have to be either all 0s
to ignore this field or all 1s to match on this field. If partial masking is set,
then any combination of bits can bit set to match on these bits.
For example, when a selector \field{type} is VIRTIO_NET_FF_MASK_TYPE_ETH, if
\field{mask[0-12]} are zero and \field{mask[13-14]} are 0xff (all 1s), it
indicates that matching is only supported for \field{EtherType} of
\field{Ethernet MAC frame}, matching is not supported for
\field{Destination Address} and \field{Source Address}.

The entries in the array \field{selectors} are ordered by
\field{type}, with each \field{type} value only appearing once.

\field{length} is the length of a dynamic array \field{mask} in bytes.
\field{reserved} and \field{reserved1} are reserved and set to zero.

\begin{table}[H]
\caption{Flow filter selector types}
\label{table:Device Types / Network Device / Device Operation / Flow filter / Device and driver capabilities / VIRTIO-NET-FF-SELECTOR-CAP / flow filter selector types}
\begin{tabularx}{\textwidth}{ |l|X|X| }
\hline
Type & Name & Description \\
\hline \hline
0x0 & - & Reserved \\
\hline
0x1 & VIRTIO_NET_FF_MASK_TYPE_ETH & 14 bytes of frame header starting from destination address described in \hyperref[intro:IEEE 802.3-2022]{IEEE 802.3-2022} \\
\hline
0x2 & VIRTIO_NET_FF_MASK_TYPE_IPV4 & 20 bytes of \hyperref[intro:Internet-Header-Format]{IPv4: Internet Header Format} \\
\hline
0x3 & VIRTIO_NET_FF_MASK_TYPE_IPV6 & 40 bytes of \hyperref[intro:IPv6-Header-Format]{IPv6 Header Format} \\
\hline
0x4 & VIRTIO_NET_FF_MASK_TYPE_TCP & 20 bytes of \hyperref[intro:TCP-Header-Format]{TCP Header Format} \\
\hline
0x5 & VIRTIO_NET_FF_MASK_TYPE_UDP & 8 bytes of UDP header described in \hyperref[intro:UDP]{UDP} \\
\hline
0x6 - 0xFF & & Reserved for future \\
\hline
\end{tabularx}
\end{table}

When VIRTIO_NET_FF_MASK_F_PARTIAL_MASK (bit 0) is set, it indicates that
partial masking is supported for all the fields of the selector identified by \field{type}.

For the selector \field{type} VIRTIO_NET_FF_MASK_TYPE_IPV4, if a partial mask is unsupported,
then matching on an individual bit of \field{Flags} in the
\field{IPv4: Internet Header Format} is unsupported. \field{Flags} has to match as a whole
if it is supported.

For the selector \field{type} VIRTIO_NET_FF_MASK_TYPE_IPV4, \field{mask} includes fields
up to the \field{Destination Address}; that is, \field{Options} and
\field{Padding} are excluded.

For the selector \field{type} VIRTIO_NET_FF_MASK_TYPE_IPV6, the \field{Next Header} field
of the \field{mask} corresponds to the \field{Next Header} in the packet
when \field{IPv6 Extension Headers} are not present. When the packet includes
one or more \field{IPv6 Extension Headers}, the \field{Next Header} field of
the \field{mask} corresponds to the \field{Next Header} of the last
\field{IPv6 Extension Header} in the packet.

For the selector \field{type} VIRTIO_NET_FF_MASK_TYPE_TCP, \field{Control bits}
are treated as individual fields for matching; that is, matching individual
\field{Control bits} does not depend on the partial mask support.

\subparagraph{VIRTIO_NET_FF_ACTION_CAP}
\label{par:Device Types / Network Device / Device Operation / Flow filter / Device and driver capabilities / VIRTIO-NET-FF-ACTION-CAP}

The capability VIRTIO_NET_FF_ACTION_CAP lists the supported actions in a rule.
\field{cap_specific_data} is in the format \field{struct virtio_net_ff_cap_actions}.

\begin{lstlisting}
struct virtio_net_ff_actions {
        u8 count;
        u8 reserved[7];
        u8 actions[];
};
\end{lstlisting}

\field{actions} is an array listing all possible actions.
The entries in the array are ordered from the smallest to the largest,
with each supported value appearing exactly once. Each entry can have the
following values:

\begin{table}[H]
\caption{Flow filter rule actions}
\label{table:Device Types / Network Device / Device Operation / Flow filter / Device and driver capabilities / VIRTIO-NET-FF-ACTION-CAP / flow filter rule actions}
\begin{tabularx}{\textwidth}{ |l|X|X| }
\hline
Action & Name & Description \\
\hline \hline
0x0 & - & reserved \\
\hline
0x1 & VIRTIO_NET_FF_ACTION_DROP & Matching packet will be dropped by the device \\
\hline
0x2 & VIRTIO_NET_FF_ACTION_DIRECT_RX_VQ & Matching packet will be directed to a receive queue \\
\hline
0x3 - 0xFF & & Reserved for future \\
\hline
\end{tabularx}
\end{table}

\paragraph{Resource objects}
\label{par:Device Types / Network Device / Device Operation / Flow filter / Resource objects}

\subparagraph{VIRTIO_NET_RESOURCE_OBJ_FF_GROUP}\label{par:Device Types / Network Device / Device Operation / Flow filter / Resource objects / VIRTIO-NET-RESOURCE-OBJ-FF-GROUP}

A flow filter group contains between 0 and \field{rules_limit} rules, as specified by the
capability VIRTIO_NET_FF_RESOURCE_CAP. For the flow filter group object both
\field{resource_obj_specific_data} and
\field{resource_obj_specific_result} are in the format
\field{struct virtio_net_resource_obj_ff_group}.

\begin{lstlisting}
struct virtio_net_resource_obj_ff_group {
        le16 group_priority;
};
\end{lstlisting}

\field{group_priority} specifies the priority for the group. Each group has a
distinct priority. For each incoming packet, the device tries to apply rules
from groups from higher \field{group_priority} value to lower, until either a
rule matches the packet or all groups have been tried.

\subparagraph{VIRTIO_NET_RESOURCE_OBJ_FF_CLASSIFIER}\label{par:Device Types / Network Device / Device Operation / Flow filter / Resource objects / VIRTIO-NET-RESOURCE-OBJ-FF-CLASSIFIER}

A classifier is used to match a flow filter key against a packet. The
classifier defines the desired packet fields to match, and is represented by
the VIRTIO_NET_RESOURCE_OBJ_FF_CLASSIFIER device resource object.

For the flow filter classifier object both \field{resource_obj_specific_data} and
\field{resource_obj_specific_result} are in the format
\field{struct virtio_net_resource_obj_ff_classifier}.

\begin{lstlisting}
struct virtio_net_resource_obj_ff_classifier {
        u8 count;
        u8 reserved[7];
        struct virtio_net_ff_selector selectors[];
};
\end{lstlisting}

A classifier is an array of \field{selectors}. The number of selectors in the
array is indicated by \field{count}. The selector has a type that specifies
the header fields to be matched against, and a mask.
See \ref{lst:Device Types / Network Device / Device Operation / Flow filter / Device and driver capabilities / VIRTIO-NET-FF-SELECTOR-CAP / virtio-net-ff-selector}
for details about selectors.

The first selector is always VIRTIO_NET_FF_MASK_TYPE_ETH. When there are multiple
selectors, a second selector can be either VIRTIO_NET_FF_MASK_TYPE_IPV4
or VIRTIO_NET_FF_MASK_TYPE_IPV6. If the third selector exists, the third
selector can be either VIRTIO_NET_FF_MASK_TYPE_UDP or VIRTIO_NET_FF_MASK_TYPE_TCP.
For example, to match a Ethernet IPv6 UDP packet,
\field{selectors[0].type} is set to VIRTIO_NET_FF_MASK_TYPE_ETH, \field{selectors[1].type}
is set to VIRTIO_NET_FF_MASK_TYPE_IPV6 and \field{selectors[2].type} is
set to VIRTIO_NET_FF_MASK_TYPE_UDP; accordingly, \field{selectors[0].mask[0-13]} is
for Ethernet header fields, \field{selectors[1].mask[0-39]} is set for IPV6 header
and \field{selectors[2].mask[0-7]} is set for UDP header.

When there are multiple selectors, the type of the (N+1)\textsuperscript{th} selector
affects the mask of the (N)\textsuperscript{th} selector. If
\field{count} is 2 or more, all the mask bits within \field{selectors[0]}
corresponding to \field{EtherType} of an Ethernet header are set.

If \field{count} is more than 2:
\begin{itemize}
\item if \field{selector[1].type} is, VIRTIO_NET_FF_MASK_TYPE_IPV4, then, all the mask bits within
\field{selector[1]} for \field{Protocol} is set.
\item if \field{selector[1].type} is, VIRTIO_NET_FF_MASK_TYPE_IPV6, then, all the mask bits within
\field{selector[1]} for \field{Next Header} is set.
\end{itemize}

If for a given packet header field, a subset of bits of a field is to be matched,
and if the partial mask is supported, the flow filter
mask object can specify a mask which has fewer bits set than the packet header
field size. For example, a partial mask for the Ethernet header source mac
address can be of 1-bit for multicast detection instead of 48-bits.

\subparagraph{VIRTIO_NET_RESOURCE_OBJ_FF_RULE}\label{par:Device Types / Network Device / Device Operation / Flow filter / Resource objects / VIRTIO-NET-RESOURCE-OBJ-FF-RULE}

Each flow filter rule resource object comprises a key, a priority, and an action.
For the flow filter rule object,
\field{resource_obj_specific_data} and
\field{resource_obj_specific_result} are in the format
\field{struct virtio_net_resource_obj_ff_rule}.

\begin{lstlisting}
struct virtio_net_resource_obj_ff_rule {
        le32 group_id;
        le32 classifier_id;
        u8 rule_priority;
        u8 key_length; /* length of key in bytes */
        u8 action;
        u8 reserved;
        le16 vq_index;
        u8 reserved1[2];
        u8 keys[][];
};
\end{lstlisting}

\field{group_id} is the resource object ID of the flow filter group to which
this rule belongs. \field{classifier_id} is the resource object ID of the
classifier used to match a packet against the \field{key}.

\field{rule_priority} denotes the priority of the rule within the group
specified by the \field{group_id}.
Rules within the group are applied from the highest to the lowest priority
until a rule matches the packet and an
action is taken. Rules with the same priority can be applied in any order.

\field{reserved} and \field{reserved1} are reserved and set to 0.

\field{keys[][]} is an array of keys to match against packets, using
the classifier specified by \field{classifier_id}. Each entry (key) comprises
a byte array, and they are located one immediately after another.
The size (number of entries) of the array is exactly the same as that of
\field{selectors} in the classifier, or in other words, \field{count}
in the classifier.

\field{key_length} specifies the total length of \field{keys} in bytes.
In other words, it equals the sum total of \field{length} of all
selectors in \field{selectors} in the classifier specified by
\field{classifier_id}.

For example, if a classifier object's \field{selectors[0].type} is
VIRTIO_NET_FF_MASK_TYPE_ETH and \field{selectors[1].type} is
VIRTIO_NET_FF_MASK_TYPE_IPV6,
then selectors[0].length is 14 and selectors[1].length is 40.
Accordingly, the \field{key_length} is set to 54.
This setting indicates that the \field{key} array's length is 54 bytes
comprising a first byte array of 14 bytes for the
Ethernet MAC header in bytes 0-13, immediately followed by 40 bytes for the
IPv6 header in bytes 14-53.

When there are multiple selectors in the classifier object, the key bytes
for (N)\textsuperscript{th} selector are set so that
(N+1)\textsuperscript{th} selector can be matched.

If \field{count} is 2 or more, key bytes of \field{EtherType}
are set according to \hyperref[intro:IEEE 802 Ethertypes]{IEEE 802 Ethertypes}
for VIRTIO_NET_FF_MASK_TYPE_IPV4 or VIRTIO_NET_FF_MASK_TYPE_IPV6 respectively.

If \field{count} is more than 2, when \field{selector[1].type} is
VIRTIO_NET_FF_MASK_TYPE_IPV4 or VIRTIO_NET_FF_MASK_TYPE_IPV6, key
bytes of \field{Protocol} or \field{Next Header} is set as per
\field{Protocol Numbers} defined \hyperref[intro:IANA Protocol Numbers]{IANA Protocol Numbers}
respectively.

\field{action} is the action to take when a packet matches the
\field{key} using the \field{classifier_id}. Supported actions are described in
\ref{table:Device Types / Network Device / Device Operation / Flow filter / Device and driver capabilities / VIRTIO-NET-FF-ACTION-CAP / flow filter rule actions}.

\field{vq_index} specifies a receive virtqueue. When the \field{action} is set
to VIRTIO_NET_FF_ACTION_DIRECT_RX_VQ, and the packet matches the \field{key},
the matching packet is directed to this virtqueue.

Note that at most one action is ever taken for a given packet. If a rule is
applied and an action is taken, the action of other rules is not taken.

\devicenormative{\paragraph}{Flow filter}{Device Types / Network Device / Device Operation / Flow filter}

When the device supports flow filter operations,
\begin{itemize}
\item the device MUST set VIRTIO_NET_FF_RESOURCE_CAP, VIRTIO_NET_FF_SELECTOR_CAP
and VIRTIO_NET_FF_ACTION_CAP capability in the \field{supported_caps} in the
command VIRTIO_ADMIN_CMD_CAP_SUPPORT_QUERY.
\item the device MUST support the administration commands
VIRTIO_ADMIN_CMD_RESOURCE_OBJ_CREATE,
VIRTIO_ADMIN_CMD_RESOURCE_OBJ_MODIFY, VIRTIO_ADMIN_CMD_RESOURCE_OBJ_QUERY,
VIRTIO_ADMIN_CMD_RESOURCE_OBJ_DESTROY for the resource types
VIRTIO_NET_RESOURCE_OBJ_FF_GROUP, VIRTIO_NET_RESOURCE_OBJ_FF_CLASSIFIER and
VIRTIO_NET_RESOURCE_OBJ_FF_RULE.
\end{itemize}

When any of the VIRTIO_NET_FF_RESOURCE_CAP, VIRTIO_NET_FF_SELECTOR_CAP, or
VIRTIO_NET_FF_ACTION_CAP capability is disabled, the device SHOULD set
\field{status} to VIRTIO_ADMIN_STATUS_Q_INVALID_OPCODE for the commands
VIRTIO_ADMIN_CMD_RESOURCE_OBJ_CREATE,
VIRTIO_ADMIN_CMD_RESOURCE_OBJ_MODIFY, VIRTIO_ADMIN_CMD_RESOURCE_OBJ_QUERY,
and VIRTIO_ADMIN_CMD_RESOURCE_OBJ_DESTROY. These commands apply to the resource
\field{type} of VIRTIO_NET_RESOURCE_OBJ_FF_GROUP, VIRTIO_NET_RESOURCE_OBJ_FF_CLASSIFIER, and
VIRTIO_NET_RESOURCE_OBJ_FF_RULE.

The device SHOULD set \field{status} to VIRTIO_ADMIN_STATUS_EINVAL for the
command VIRTIO_ADMIN_CMD_RESOURCE_OBJ_CREATE when the resource \field{type}
is VIRTIO_NET_RESOURCE_OBJ_FF_GROUP, if a flow filter group already exists
with the supplied \field{group_priority}.

The device SHOULD set \field{status} to VIRTIO_ADMIN_STATUS_ENOSPC for the
command VIRTIO_ADMIN_CMD_RESOURCE_OBJ_CREATE when the resource \field{type}
is VIRTIO_NET_RESOURCE_OBJ_FF_GROUP, if the number of flow filter group
objects in the device exceeds the lower of the configured driver
capabilities \field{groups_limit} and \field{rules_per_group_limit}.

The device SHOULD set \field{status} to VIRTIO_ADMIN_STATUS_ENOSPC for the
command VIRTIO_ADMIN_CMD_RESOURCE_OBJ_CREATE when the resource \field{type} is
VIRTIO_NET_RESOURCE_OBJ_FF_CLASSIFIER, if the number of flow filter selector
objects in the device exceeds the configured driver capability
\field{selectors_limit}.

The device SHOULD set \field{status} to VIRTIO_ADMIN_STATUS_EBUSY for the
command VIRTIO_ADMIN_CMD_RESOURCE_OBJ_DESTROY for a flow filter group when
the flow filter group has one or more flow filter rules depending on it.

The device SHOULD set \field{status} to VIRTIO_ADMIN_STATUS_EBUSY for the
command VIRTIO_ADMIN_CMD_RESOURCE_OBJ_DESTROY for a flow filter classifier when
the flow filter classifier has one or more flow filter rules depending on it.

The device SHOULD fail the command VIRTIO_ADMIN_CMD_RESOURCE_OBJ_CREATE for the
flow filter rule resource object if,
\begin{itemize}
\item \field{vq_index} is not a valid receive virtqueue index for
the VIRTIO_NET_FF_ACTION_DIRECT_RX_VQ action,
\item \field{priority} is greater than or equal to
      \field{last_rule_priority},
\item \field{id} is greater than or equal to \field{rules_limit} or
      greater than or equal to \field{rules_per_group_limit}, whichever is lower,
\item the length of \field{keys} and the length of all the mask bytes of
      \field{selectors[].mask} as referred by \field{classifier_id} differs,
\item the supplied \field{action} is not supported in the capability VIRTIO_NET_FF_ACTION_CAP.
\end{itemize}

When the flow filter directs a packet to the virtqueue identified by
\field{vq_index} and if the receive virtqueue is reset, the device
MUST drop such packets.

Upon applying a flow filter rule to a packet, the device MUST STOP any further
application of rules and cease applying any other steering configurations.

For multiple flow filter groups, the device MUST apply the rules from
the group with the highest priority. If any rule from this group is applied,
the device MUST ignore the remaining groups. If none of the rules from the
highest priority group match, the device MUST apply the rules from
the group with the next highest priority, until either a rule matches or
all groups have been attempted.

The device MUST apply the rules within the group from the highest to the
lowest priority until a rule matches the packet, and the device MUST take
the action. If an action is taken, the device MUST not take any other
action for this packet.

The device MAY apply the rules with the same \field{rule_priority} in any
order within the group.

The device MUST process incoming packets in the following order:
\begin{itemize}
\item apply the steering configuration received using control virtqueue
      commands VIRTIO_NET_CTRL_RX, VIRTIO_NET_CTRL_MAC, and
      VIRTIO_NET_CTRL_VLAN.
\item apply flow filter rules if any.
\item if no filter rule is applied, apply the steering configuration
      received using the command VIRTIO_NET_CTRL_MQ_RSS_CONFIG
      or according to automatic receive steering.
\end{itemize}

When processing an incoming packet, if the packet is dropped at any stage, the device
MUST skip further processing.

When the device drops the packet due to the configuration done using the control
virtqueue commands VIRTIO_NET_CTRL_RX or VIRTIO_NET_CTRL_MAC or VIRTIO_NET_CTRL_VLAN,
the device MUST skip flow filter rules for this packet.

When the device performs flow filter match operations and if the operation
result did not have any match in all the groups, the receive packet processing
continues to next level, i.e. to apply configuration done using
VIRTIO_NET_CTRL_MQ_RSS_CONFIG command.

The device MUST support the creation of flow filter classifier objects
using the command VIRTIO_ADMIN_CMD_RESOURCE_OBJ_CREATE with \field{flags}
set to VIRTIO_NET_FF_MASK_F_PARTIAL_MASK;
this support is required even if all the bits of the masks are set for
a field in \field{selectors}, provided that partial masking is supported
for the selectors.

\drivernormative{\paragraph}{Flow filter}{Device Types / Network Device / Device Operation / Flow filter}

The driver MUST enable VIRTIO_NET_FF_RESOURCE_CAP, VIRTIO_NET_FF_SELECTOR_CAP,
and VIRTIO_NET_FF_ACTION_CAP capabilities to use flow filter.

The driver SHOULD NOT remove a flow filter group using the command
VIRTIO_ADMIN_CMD_RESOURCE_OBJ_DESTROY when one or more flow filter rules
depend on that group. The driver SHOULD only destroy the group after
all the associated rules have been destroyed.

The driver SHOULD NOT remove a flow filter classifier using the command
VIRTIO_ADMIN_CMD_RESOURCE_OBJ_DESTROY when one or more flow filter rules
depend on the classifier. The driver SHOULD only destroy the classifier
after all the associated rules have been destroyed.

The driver SHOULD NOT add multiple flow filter rules with the same
\field{rule_priority} within a flow filter group, as these rules MAY match
the same packet. The driver SHOULD assign different \field{rule_priority}
values to different flow filter rules if multiple rules may match a single
packet.

For the command VIRTIO_ADMIN_CMD_RESOURCE_OBJ_CREATE, when creating a resource
of \field{type} VIRTIO_NET_RESOURCE_OBJ_FF_CLASSIFIER, the driver MUST set:
\begin{itemize}
\item \field{selectors[0].type} to VIRTIO_NET_FF_MASK_TYPE_ETH.
\item \field{selectors[1].type} to VIRTIO_NET_FF_MASK_TYPE_IPV4 or
      VIRTIO_NET_FF_MASK_TYPE_IPV6 when \field{count} is more than 1,
\item \field{selectors[2].type} VIRTIO_NET_FF_MASK_TYPE_UDP or
      VIRTIO_NET_FF_MASK_TYPE_TCP when \field{count} is more than 2.
\end{itemize}

For the command VIRTIO_ADMIN_CMD_RESOURCE_OBJ_CREATE, when creating a resource
of \field{type} VIRTIO_NET_RESOURCE_OBJ_FF_CLASSIFIER, the driver MUST set:
\begin{itemize}
\item \field{selectors[0].mask} bytes to all 1s for the \field{EtherType}
       when \field{count} is 2 or more.
\item \field{selectors[1].mask} bytes to all 1s for \field{Protocol} or \field{Next Header}
       when \field{selector[1].type} is VIRTIO_NET_FF_MASK_TYPE_IPV4 or VIRTIO_NET_FF_MASK_TYPE_IPV6,
       and when \field{count} is more than 2.
\end{itemize}

For the command VIRTIO_ADMIN_CMD_RESOURCE_OBJ_CREATE, the resource \field{type}
VIRTIO_NET_RESOURCE_OBJ_FF_RULE, if the corresponding classifier object's
\field{count} is 2 or more, the driver MUST SET the \field{keys} bytes of
\field{EtherType} in accordance with
\hyperref[intro:IEEE 802 Ethertypes]{IEEE 802 Ethertypes}
for either VIRTIO_NET_FF_MASK_TYPE_IPV4 or VIRTIO_NET_FF_MASK_TYPE_IPV6.

For the command VIRTIO_ADMIN_CMD_RESOURCE_OBJ_CREATE, when creating a resource of
\field{type} VIRTIO_NET_RESOURCE_OBJ_FF_RULE, if the corresponding classifier
object's \field{count} is more than 2, and the \field{selector[1].type} is either
VIRTIO_NET_FF_MASK_TYPE_IPV4 or VIRTIO_NET_FF_MASK_TYPE_IPV6, the driver MUST
set the \field{keys} bytes for the \field{Protocol} or \field{Next Header}
according to \hyperref[intro:IANA Protocol Numbers]{IANA Protocol Numbers} respectively.

The driver SHOULD set all the bits for a field in the mask of a selector in both the
capability and the classifier object, unless the VIRTIO_NET_FF_MASK_F_PARTIAL_MASK
is enabled.

\subsubsection{Legacy Interface: Framing Requirements}\label{sec:Device
Types / Network Device / Legacy Interface: Framing Requirements}

When using legacy interfaces, transitional drivers which have not
negotiated VIRTIO_F_ANY_LAYOUT MUST use a single descriptor for the
\field{struct virtio_net_hdr} on both transmit and receive, with the
network data in the following descriptors.

Additionally, when using the control virtqueue (see \ref{sec:Device
Types / Network Device / Device Operation / Control Virtqueue})
, transitional drivers which have not
negotiated VIRTIO_F_ANY_LAYOUT MUST:
\begin{itemize}
\item for all commands, use a single 2-byte descriptor including the first two
fields: \field{class} and \field{command}
\item for all commands except VIRTIO_NET_CTRL_MAC_TABLE_SET
use a single descriptor including command-specific-data
with no padding.
\item for the VIRTIO_NET_CTRL_MAC_TABLE_SET command use exactly
two descriptors including command-specific-data with no padding:
the first of these descriptors MUST include the
virtio_net_ctrl_mac table structure for the unicast addresses with no padding,
the second of these descriptors MUST include the
virtio_net_ctrl_mac table structure for the multicast addresses
with no padding.
\item for all commands, use a single 1-byte descriptor for the
\field{ack} field
\end{itemize}

See \ref{sec:Basic
Facilities of a Virtio Device / Virtqueues / Message Framing}.

\section{Network Device}\label{sec:Device Types / Network Device}

The virtio network device is a virtual network interface controller.
It consists of a virtual Ethernet link which connects the device
to the Ethernet network. The device has transmit and receive
queues. The driver adds empty buffers to the receive virtqueue.
The device receives incoming packets from the link; the device
places these incoming packets in the receive virtqueue buffers.
The driver adds outgoing packets to the transmit virtqueue. The device
removes these packets from the transmit virtqueue and sends them to
the link. The device may have a control virtqueue. The driver
uses the control virtqueue to dynamically manipulate various
features of the initialized device.

\subsection{Device ID}\label{sec:Device Types / Network Device / Device ID}

 1

\subsection{Virtqueues}\label{sec:Device Types / Network Device / Virtqueues}

\begin{description}
\item[0] receiveq1
\item[1] transmitq1
\item[\ldots]
\item[2(N-1)] receiveqN
\item[2(N-1)+1] transmitqN
\item[2N] controlq
\end{description}

 N=1 if neither VIRTIO_NET_F_MQ nor VIRTIO_NET_F_RSS are negotiated, otherwise N is set by
 \field{max_virtqueue_pairs}.

controlq is optional; it only exists if VIRTIO_NET_F_CTRL_VQ is
negotiated.

\subsection{Feature bits}\label{sec:Device Types / Network Device / Feature bits}

\begin{description}
\item[VIRTIO_NET_F_CSUM (0)] Device handles packets with partial checksum offload.

\item[VIRTIO_NET_F_GUEST_CSUM (1)] Driver handles packets with partial checksum.

\item[VIRTIO_NET_F_CTRL_GUEST_OFFLOADS (2)] Control channel offloads
        reconfiguration support.

\item[VIRTIO_NET_F_MTU(3)] Device maximum MTU reporting is supported. If
    offered by the device, device advises driver about the value of
    its maximum MTU. If negotiated, the driver uses \field{mtu} as
    the maximum MTU value.

\item[VIRTIO_NET_F_MAC (5)] Device has given MAC address.

\item[VIRTIO_NET_F_GUEST_TSO4 (7)] Driver can receive TSOv4.

\item[VIRTIO_NET_F_GUEST_TSO6 (8)] Driver can receive TSOv6.

\item[VIRTIO_NET_F_GUEST_ECN (9)] Driver can receive TSO with ECN.

\item[VIRTIO_NET_F_GUEST_UFO (10)] Driver can receive UFO.

\item[VIRTIO_NET_F_HOST_TSO4 (11)] Device can receive TSOv4.

\item[VIRTIO_NET_F_HOST_TSO6 (12)] Device can receive TSOv6.

\item[VIRTIO_NET_F_HOST_ECN (13)] Device can receive TSO with ECN.

\item[VIRTIO_NET_F_HOST_UFO (14)] Device can receive UFO.

\item[VIRTIO_NET_F_MRG_RXBUF (15)] Driver can merge receive buffers.

\item[VIRTIO_NET_F_STATUS (16)] Configuration status field is
    available.

\item[VIRTIO_NET_F_CTRL_VQ (17)] Control channel is available.

\item[VIRTIO_NET_F_CTRL_RX (18)] Control channel RX mode support.

\item[VIRTIO_NET_F_CTRL_VLAN (19)] Control channel VLAN filtering.

\item[VIRTIO_NET_F_CTRL_RX_EXTRA (20)]	Control channel RX extra mode support.

\item[VIRTIO_NET_F_GUEST_ANNOUNCE(21)] Driver can send gratuitous
    packets.

\item[VIRTIO_NET_F_MQ(22)] Device supports multiqueue with automatic
    receive steering.

\item[VIRTIO_NET_F_CTRL_MAC_ADDR(23)] Set MAC address through control
    channel.

\item[VIRTIO_NET_F_DEVICE_STATS(50)] Device can provide device-level statistics
    to the driver through the control virtqueue.

\item[VIRTIO_NET_F_HASH_TUNNEL(51)] Device supports inner header hash for encapsulated packets.

\item[VIRTIO_NET_F_VQ_NOTF_COAL(52)] Device supports virtqueue notification coalescing.

\item[VIRTIO_NET_F_NOTF_COAL(53)] Device supports notifications coalescing.

\item[VIRTIO_NET_F_GUEST_USO4 (54)] Driver can receive USOv4 packets.

\item[VIRTIO_NET_F_GUEST_USO6 (55)] Driver can receive USOv6 packets.

\item[VIRTIO_NET_F_HOST_USO (56)] Device can receive USO packets. Unlike UFO
 (fragmenting the packet) the USO splits large UDP packet
 to several segments when each of these smaller packets has UDP header.

\item[VIRTIO_NET_F_HASH_REPORT(57)] Device can report per-packet hash
    value and a type of calculated hash.

\item[VIRTIO_NET_F_GUEST_HDRLEN(59)] Driver can provide the exact \field{hdr_len}
    value. Device benefits from knowing the exact header length.

\item[VIRTIO_NET_F_RSS(60)] Device supports RSS (receive-side scaling)
    with Toeplitz hash calculation and configurable hash
    parameters for receive steering.

\item[VIRTIO_NET_F_RSC_EXT(61)] Device can process duplicated ACKs
    and report number of coalesced segments and duplicated ACKs.

\item[VIRTIO_NET_F_STANDBY(62)] Device may act as a standby for a primary
    device with the same MAC address.

\item[VIRTIO_NET_F_SPEED_DUPLEX(63)] Device reports speed and duplex.

\item[VIRTIO_NET_F_RSS_CONTEXT(64)] Device supports multiple RSS contexts.

\item[VIRTIO_NET_F_GUEST_UDP_TUNNEL_GSO (65)] Driver can receive GSO packets
  carried by a UDP tunnel.

\item[VIRTIO_NET_F_GUEST_UDP_TUNNEL_GSO_CSUM (66)] Driver handles packets
  carried by a UDP tunnel with partial csum for the outer header.

\item[VIRTIO_NET_F_HOST_UDP_TUNNEL_GSO (67)] Device can receive GSO packets
  carried by a UDP tunnel.

\item[VIRTIO_NET_F_HOST_UDP_TUNNEL_GSO_CSUM (68)] Device handles packets
  carried by a UDP tunnel with partial csum for the outer header.
\end{description}

\subsubsection{Feature bit requirements}\label{sec:Device Types / Network Device / Feature bits / Feature bit requirements}

Some networking feature bits require other networking feature bits
(see \ref{drivernormative:Basic Facilities of a Virtio Device / Feature Bits}):

\begin{description}
\item[VIRTIO_NET_F_GUEST_TSO4] Requires VIRTIO_NET_F_GUEST_CSUM.
\item[VIRTIO_NET_F_GUEST_TSO6] Requires VIRTIO_NET_F_GUEST_CSUM.
\item[VIRTIO_NET_F_GUEST_ECN] Requires VIRTIO_NET_F_GUEST_TSO4 or VIRTIO_NET_F_GUEST_TSO6.
\item[VIRTIO_NET_F_GUEST_UFO] Requires VIRTIO_NET_F_GUEST_CSUM.
\item[VIRTIO_NET_F_GUEST_USO4] Requires VIRTIO_NET_F_GUEST_CSUM.
\item[VIRTIO_NET_F_GUEST_USO6] Requires VIRTIO_NET_F_GUEST_CSUM.
\item[VIRTIO_NET_F_GUEST_UDP_TUNNEL_GSO] Requires VIRTIO_NET_F_GUEST_TSO4, VIRTIO_NET_F_GUEST_TSO6,
   VIRTIO_NET_F_GUEST_USO4 and VIRTIO_NET_F_GUEST_USO6.
\item[VIRTIO_NET_F_GUEST_UDP_TUNNEL_GSO_CSUM] Requires VIRTIO_NET_F_GUEST_UDP_TUNNEL_GSO

\item[VIRTIO_NET_F_HOST_TSO4] Requires VIRTIO_NET_F_CSUM.
\item[VIRTIO_NET_F_HOST_TSO6] Requires VIRTIO_NET_F_CSUM.
\item[VIRTIO_NET_F_HOST_ECN] Requires VIRTIO_NET_F_HOST_TSO4 or VIRTIO_NET_F_HOST_TSO6.
\item[VIRTIO_NET_F_HOST_UFO] Requires VIRTIO_NET_F_CSUM.
\item[VIRTIO_NET_F_HOST_USO] Requires VIRTIO_NET_F_CSUM.
\item[VIRTIO_NET_F_HOST_UDP_TUNNEL_GSO] Requires VIRTIO_NET_F_HOST_TSO4, VIRTIO_NET_F_HOST_TSO6
   and VIRTIO_NET_F_HOST_USO.
\item[VIRTIO_NET_F_HOST_UDP_TUNNEL_GSO_CSUM] Requires VIRTIO_NET_F_HOST_UDP_TUNNEL_GSO

\item[VIRTIO_NET_F_CTRL_RX] Requires VIRTIO_NET_F_CTRL_VQ.
\item[VIRTIO_NET_F_CTRL_VLAN] Requires VIRTIO_NET_F_CTRL_VQ.
\item[VIRTIO_NET_F_GUEST_ANNOUNCE] Requires VIRTIO_NET_F_CTRL_VQ.
\item[VIRTIO_NET_F_MQ] Requires VIRTIO_NET_F_CTRL_VQ.
\item[VIRTIO_NET_F_CTRL_MAC_ADDR] Requires VIRTIO_NET_F_CTRL_VQ.
\item[VIRTIO_NET_F_NOTF_COAL] Requires VIRTIO_NET_F_CTRL_VQ.
\item[VIRTIO_NET_F_RSC_EXT] Requires VIRTIO_NET_F_HOST_TSO4 or VIRTIO_NET_F_HOST_TSO6.
\item[VIRTIO_NET_F_RSS] Requires VIRTIO_NET_F_CTRL_VQ.
\item[VIRTIO_NET_F_VQ_NOTF_COAL] Requires VIRTIO_NET_F_CTRL_VQ.
\item[VIRTIO_NET_F_HASH_TUNNEL] Requires VIRTIO_NET_F_CTRL_VQ along with VIRTIO_NET_F_RSS or VIRTIO_NET_F_HASH_REPORT.
\item[VIRTIO_NET_F_RSS_CONTEXT] Requires VIRTIO_NET_F_CTRL_VQ and VIRTIO_NET_F_RSS.
\end{description}

\begin{note}
The dependency between UDP_TUNNEL_GSO_CSUM and UDP_TUNNEL_GSO is intentionally
in the opposite direction with respect to the plain GSO features and the plain
checksum offload because UDP tunnel checksum offload gives very little gain
for non GSO packets and is quite complex to implement in H/W.
\end{note}

\subsubsection{Legacy Interface: Feature bits}\label{sec:Device Types / Network Device / Feature bits / Legacy Interface: Feature bits}
\begin{description}
\item[VIRTIO_NET_F_GSO (6)] Device handles packets with any GSO type. This was supposed to indicate segmentation offload support, but
upon further investigation it became clear that multiple bits were needed.
\item[VIRTIO_NET_F_GUEST_RSC4 (41)] Device coalesces TCPIP v4 packets. This was implemented by hypervisor patch for certification
purposes and current Windows driver depends on it. It will not function if virtio-net device reports this feature.
\item[VIRTIO_NET_F_GUEST_RSC6 (42)] Device coalesces TCPIP v6 packets. Similar to VIRTIO_NET_F_GUEST_RSC4.
\end{description}

\subsection{Device configuration layout}\label{sec:Device Types / Network Device / Device configuration layout}
\label{sec:Device Types / Block Device / Feature bits / Device configuration layout}

The network device has the following device configuration layout.
All of the device configuration fields are read-only for the driver.

\begin{lstlisting}
struct virtio_net_config {
        u8 mac[6];
        le16 status;
        le16 max_virtqueue_pairs;
        le16 mtu;
        le32 speed;
        u8 duplex;
        u8 rss_max_key_size;
        le16 rss_max_indirection_table_length;
        le32 supported_hash_types;
        le32 supported_tunnel_types;
};
\end{lstlisting}

The \field{mac} address field always exists (although it is only
valid if VIRTIO_NET_F_MAC is set).

The \field{status} only exists if VIRTIO_NET_F_STATUS is set.
Two bits are currently defined for the status field: VIRTIO_NET_S_LINK_UP
and VIRTIO_NET_S_ANNOUNCE.

\begin{lstlisting}
#define VIRTIO_NET_S_LINK_UP     1
#define VIRTIO_NET_S_ANNOUNCE    2
\end{lstlisting}

The following field, \field{max_virtqueue_pairs} only exists if
VIRTIO_NET_F_MQ or VIRTIO_NET_F_RSS is set. This field specifies the maximum number
of each of transmit and receive virtqueues (receiveq1\ldots receiveqN
and transmitq1\ldots transmitqN respectively) that can be configured once at least one of these features
is negotiated.

The following field, \field{mtu} only exists if VIRTIO_NET_F_MTU
is set. This field specifies the maximum MTU for the driver to
use.

The following two fields, \field{speed} and \field{duplex}, only
exist if VIRTIO_NET_F_SPEED_DUPLEX is set.

\field{speed} contains the device speed, in units of 1 MBit per
second, 0 to 0x7fffffff, or 0xffffffff for unknown speed.

\field{duplex} has the values of 0x01 for full duplex, 0x00 for
half duplex and 0xff for unknown duplex state.

Both \field{speed} and \field{duplex} can change, thus the driver
is expected to re-read these values after receiving a
configuration change notification.

The following field, \field{rss_max_key_size} only exists if VIRTIO_NET_F_RSS or VIRTIO_NET_F_HASH_REPORT is set.
It specifies the maximum supported length of RSS key in bytes.

The following field, \field{rss_max_indirection_table_length} only exists if VIRTIO_NET_F_RSS is set.
It specifies the maximum number of 16-bit entries in RSS indirection table.

The next field, \field{supported_hash_types} only exists if the device supports hash calculation,
i.e. if VIRTIO_NET_F_RSS or VIRTIO_NET_F_HASH_REPORT is set.

Field \field{supported_hash_types} contains the bitmask of supported hash types.
See \ref{sec:Device Types / Network Device / Device Operation / Processing of Incoming Packets / Hash calculation for incoming packets / Supported/enabled hash types} for details of supported hash types.

Field \field{supported_tunnel_types} only exists if the device supports inner header hash, i.e. if VIRTIO_NET_F_HASH_TUNNEL is set.

Field \field{supported_tunnel_types} contains the bitmask of encapsulation types supported by the device for inner header hash.
Encapsulation types are defined in \ref{sec:Device Types / Network Device / Device Operation / Processing of Incoming Packets /
Hash calculation for incoming packets / Encapsulation types supported/enabled for inner header hash}.

\devicenormative{\subsubsection}{Device configuration layout}{Device Types / Network Device / Device configuration layout}

The device MUST set \field{max_virtqueue_pairs} to between 1 and 0x8000 inclusive,
if it offers VIRTIO_NET_F_MQ.

The device MUST set \field{mtu} to between 68 and 65535 inclusive,
if it offers VIRTIO_NET_F_MTU.

The device SHOULD set \field{mtu} to at least 1280, if it offers
VIRTIO_NET_F_MTU.

The device MUST NOT modify \field{mtu} once it has been set.

The device MUST NOT pass received packets that exceed \field{mtu} (plus low
level ethernet header length) size with \field{gso_type} NONE or ECN
after VIRTIO_NET_F_MTU has been successfully negotiated.

The device MUST forward transmitted packets of up to \field{mtu} (plus low
level ethernet header length) size with \field{gso_type} NONE or ECN, and do
so without fragmentation, after VIRTIO_NET_F_MTU has been successfully
negotiated.

The device MUST set \field{rss_max_key_size} to at least 40, if it offers
VIRTIO_NET_F_RSS or VIRTIO_NET_F_HASH_REPORT.

The device MUST set \field{rss_max_indirection_table_length} to at least 128, if it offers
VIRTIO_NET_F_RSS.

If the driver negotiates the VIRTIO_NET_F_STANDBY feature, the device MAY act
as a standby device for a primary device with the same MAC address.

If VIRTIO_NET_F_SPEED_DUPLEX has been negotiated, \field{speed}
MUST contain the device speed, in units of 1 MBit per second, 0 to
0x7ffffffff, or 0xfffffffff for unknown.

If VIRTIO_NET_F_SPEED_DUPLEX has been negotiated, \field{duplex}
MUST have the values of 0x00 for full duplex, 0x01 for half
duplex, or 0xff for unknown.

If VIRTIO_NET_F_SPEED_DUPLEX and VIRTIO_NET_F_STATUS have both
been negotiated, the device SHOULD NOT change the \field{speed} and
\field{duplex} fields as long as VIRTIO_NET_S_LINK_UP is set in
the \field{status}.

The device SHOULD NOT offer VIRTIO_NET_F_HASH_REPORT if it
does not offer VIRTIO_NET_F_CTRL_VQ.

The device SHOULD NOT offer VIRTIO_NET_F_CTRL_RX_EXTRA if it
does not offer VIRTIO_NET_F_CTRL_VQ.

\drivernormative{\subsubsection}{Device configuration layout}{Device Types / Network Device / Device configuration layout}

The driver MUST NOT write to any of the device configuration fields.

A driver SHOULD negotiate VIRTIO_NET_F_MAC if the device offers it.
If the driver negotiates the VIRTIO_NET_F_MAC feature, the driver MUST set
the physical address of the NIC to \field{mac}.  Otherwise, it SHOULD
use a locally-administered MAC address (see \hyperref[intro:IEEE 802]{IEEE 802},
``9.2 48-bit universal LAN MAC addresses'').

If the driver does not negotiate the VIRTIO_NET_F_STATUS feature, it SHOULD
assume the link is active, otherwise it SHOULD read the link status from
the bottom bit of \field{status}.

A driver SHOULD negotiate VIRTIO_NET_F_MTU if the device offers it.

If the driver negotiates VIRTIO_NET_F_MTU, it MUST supply enough receive
buffers to receive at least one receive packet of size \field{mtu} (plus low
level ethernet header length) with \field{gso_type} NONE or ECN.

If the driver negotiates VIRTIO_NET_F_MTU, it MUST NOT transmit packets of
size exceeding the value of \field{mtu} (plus low level ethernet header length)
with \field{gso_type} NONE or ECN.

A driver SHOULD negotiate the VIRTIO_NET_F_STANDBY feature if the device offers it.

If VIRTIO_NET_F_SPEED_DUPLEX has been negotiated,
the driver MUST treat any value of \field{speed} above
0x7fffffff as well as any value of \field{duplex} not
matching 0x00 or 0x01 as an unknown value.

If VIRTIO_NET_F_SPEED_DUPLEX has been negotiated, the driver
SHOULD re-read \field{speed} and \field{duplex} after a
configuration change notification.

A driver SHOULD NOT negotiate VIRTIO_NET_F_HASH_REPORT if it
does not negotiate VIRTIO_NET_F_CTRL_VQ.

A driver SHOULD NOT negotiate VIRTIO_NET_F_CTRL_RX_EXTRA if it
does not negotiate VIRTIO_NET_F_CTRL_VQ.

\subsubsection{Legacy Interface: Device configuration layout}\label{sec:Device Types / Network Device / Device configuration layout / Legacy Interface: Device configuration layout}
\label{sec:Device Types / Block Device / Feature bits / Device configuration layout / Legacy Interface: Device configuration layout}
When using the legacy interface, transitional devices and drivers
MUST format \field{status} and
\field{max_virtqueue_pairs} in struct virtio_net_config
according to the native endian of the guest rather than
(necessarily when not using the legacy interface) little-endian.

When using the legacy interface, \field{mac} is driver-writable
which provided a way for drivers to update the MAC without
negotiating VIRTIO_NET_F_CTRL_MAC_ADDR.

\subsection{Device Initialization}\label{sec:Device Types / Network Device / Device Initialization}

A driver would perform a typical initialization routine like so:

\begin{enumerate}
\item Identify and initialize the receive and
  transmission virtqueues, up to N of each kind. If
  VIRTIO_NET_F_MQ feature bit is negotiated,
  N=\field{max_virtqueue_pairs}, otherwise identify N=1.

\item If the VIRTIO_NET_F_CTRL_VQ feature bit is negotiated,
  identify the control virtqueue.

\item Fill the receive queues with buffers: see \ref{sec:Device Types / Network Device / Device Operation / Setting Up Receive Buffers}.

\item Even with VIRTIO_NET_F_MQ, only receiveq1, transmitq1 and
  controlq are used by default.  The driver would send the
  VIRTIO_NET_CTRL_MQ_VQ_PAIRS_SET command specifying the
  number of the transmit and receive queues to use.

\item If the VIRTIO_NET_F_MAC feature bit is set, the configuration
  space \field{mac} entry indicates the ``physical'' address of the
  device, otherwise the driver would typically generate a random
  local MAC address.

\item If the VIRTIO_NET_F_STATUS feature bit is negotiated, the link
  status comes from the bottom bit of \field{status}.
  Otherwise, the driver assumes it's active.

\item A performant driver would indicate that it will generate checksumless
  packets by negotiating the VIRTIO_NET_F_CSUM feature.

\item If that feature is negotiated, a driver can use TCP segmentation or UDP
  segmentation/fragmentation offload by negotiating the VIRTIO_NET_F_HOST_TSO4 (IPv4
  TCP), VIRTIO_NET_F_HOST_TSO6 (IPv6 TCP), VIRTIO_NET_F_HOST_UFO
  (UDP fragmentation) and VIRTIO_NET_F_HOST_USO (UDP segmentation) features.

\item If the VIRTIO_NET_F_HOST_TSO6, VIRTIO_NET_F_HOST_TSO4 and VIRTIO_NET_F_HOST_USO
  segmentation features are negotiated, a driver can
  use TCP segmentation or UDP segmentation on top of UDP encapsulation
  offload, when the outer header does not require checksumming - e.g.
  the outer UDP checksum is zero - by negotiating the
  VIRTIO_NET_F_HOST_UDP_TUNNEL_GSO feature.
  GSO over UDP tunnels packets carry two sets of headers: the outer ones
  and the inner ones. The outer transport protocol is UDP, the inner
  could be either TCP or UDP. Only a single level of encapsulation
  offload is supported.

\item If VIRTIO_NET_F_HOST_UDP_TUNNEL_GSO is negotiated, a driver can
  additionally use TCP segmentation or UDP segmentation on top of UDP
  encapsulation with the outer header requiring checksum offload,
  negotiating the VIRTIO_NET_F_HOST_UDP_TUNNEL_GSO_CSUM feature.

\item The converse features are also available: a driver can save
  the virtual device some work by negotiating these features.\note{For example, a network packet transported between two guests on
the same system might not need checksumming at all, nor segmentation,
if both guests are amenable.}
   The VIRTIO_NET_F_GUEST_CSUM feature indicates that partially
  checksummed packets can be received, and if it can do that then
  the VIRTIO_NET_F_GUEST_TSO4, VIRTIO_NET_F_GUEST_TSO6,
  VIRTIO_NET_F_GUEST_UFO, VIRTIO_NET_F_GUEST_ECN, VIRTIO_NET_F_GUEST_USO4,
  VIRTIO_NET_F_GUEST_USO6 VIRTIO_NET_F_GUEST_UDP_TUNNEL_GSO and
  VIRTIO_NET_F_GUEST_UDP_TUNNEL_GSO_CSUM are the input equivalents of
  the features described above.
  See \ref{sec:Device Types / Network Device / Device Operation /
Setting Up Receive Buffers}~\nameref{sec:Device Types / Network
Device / Device Operation / Setting Up Receive Buffers} and
\ref{sec:Device Types / Network Device / Device Operation /
Processing of Incoming Packets}~\nameref{sec:Device Types /
Network Device / Device Operation / Processing of Incoming Packets} below.
\end{enumerate}

A truly minimal driver would only accept VIRTIO_NET_F_MAC and ignore
everything else.

\subsection{Device and driver capabilities}\label{sec:Device Types / Network Device / Device and driver capabilities}

The network device has the following capabilities.

\begin{tabularx}{\textwidth}{ |l||l|X| }
\hline
Identifier & Name & Description \\
\hline \hline
0x0800 & \hyperref[par:Device Types / Network Device / Device Operation / Flow filter / Device and driver capabilities / VIRTIO-NET-FF-RESOURCE-CAP]{VIRTIO_NET_FF_RESOURCE_CAP} & Flow filter resource capability \\
\hline
0x0801 & \hyperref[par:Device Types / Network Device / Device Operation / Flow filter / Device and driver capabilities / VIRTIO-NET-FF-SELECTOR-CAP]{VIRTIO_NET_FF_SELECTOR_CAP} & Flow filter classifier capability \\
\hline
0x0802 & \hyperref[par:Device Types / Network Device / Device Operation / Flow filter / Device and driver capabilities / VIRTIO-NET-FF-ACTION-CAP]{VIRTIO_NET_FF_ACTION_CAP} & Flow filter action capability \\
\hline
\end{tabularx}

\subsection{Device resource objects}\label{sec:Device Types / Network Device / Device resource objects}

The network device has the following resource objects.

\begin{tabularx}{\textwidth}{ |l||l|X| }
\hline
type & Name & Description \\
\hline \hline
0x0200 & \hyperref[par:Device Types / Network Device / Device Operation / Flow filter / Resource objects / VIRTIO-NET-RESOURCE-OBJ-FF-GROUP]{VIRTIO_NET_RESOURCE_OBJ_FF_GROUP} & Flow filter group resource object \\
\hline
0x0201 & \hyperref[par:Device Types / Network Device / Device Operation / Flow filter / Resource objects / VIRTIO-NET-RESOURCE-OBJ-FF-CLASSIFIER]{VIRTIO_NET_RESOURCE_OBJ_FF_CLASSIFIER} & Flow filter mask object \\
\hline
0x0202 & \hyperref[par:Device Types / Network Device / Device Operation / Flow filter / Resource objects / VIRTIO-NET-RESOURCE-OBJ-FF-RULE]{VIRTIO_NET_RESOURCE_OBJ_FF_RULE} & Flow filter rule object \\
\hline
\end{tabularx}

\subsection{Device parts}\label{sec:Device Types / Network Device / Device parts}

Network device parts represent the configuration done by the driver using control
virtqueue commands. Network device part is in the format of
\field{struct virtio_dev_part}.

\begin{tabularx}{\textwidth}{ |l||l|X| }
\hline
Type & Name & Description \\
\hline \hline
0x200 & VIRTIO_NET_DEV_PART_CVQ_CFG_PART & Represents device configuration done through a control virtqueue command, see \ref{sec:Device Types / Network Device / Device parts / VIRTIO-NET-DEV-PART-CVQ-CFG-PART} \\
\hline
0x201 - 0x5FF & - & reserved for future \\
\hline
\hline
\end{tabularx}

\subsubsection{VIRTIO_NET_DEV_PART_CVQ_CFG_PART}\label{sec:Device Types / Network Device / Device parts / VIRTIO-NET-DEV-PART-CVQ-CFG-PART}

For VIRTIO_NET_DEV_PART_CVQ_CFG_PART, \field{part_type} is set to 0x200. The
VIRTIO_NET_DEV_PART_CVQ_CFG_PART part indicates configuration performed by the
driver using a control virtqueue command.

\begin{lstlisting}
struct virtio_net_dev_part_cvq_selector {
        u8 class;
        u8 command;
        u8 reserved[6];
};
\end{lstlisting}

There is one device part of type VIRTIO_NET_DEV_PART_CVQ_CFG_PART for each
individual configuration. Each part is identified by a unique selector value.
The selector, \field{device_type_raw}, is in the format
\field{struct virtio_net_dev_part_cvq_selector}.

The selector consists of two fields: \field{class} and \field{command}. These
fields correspond to the \field{class} and \field{command} defined in
\field{struct virtio_net_ctrl}, as described in the relevant sections of
\ref{sec:Device Types / Network Device / Device Operation / Control Virtqueue}.

The value corresponding to each part’s selector follows the same format as the
respective \field{command-specific-data} described in the relevant sections of
\ref{sec:Device Types / Network Device / Device Operation / Control Virtqueue}.

For example, when the \field{class} is VIRTIO_NET_CTRL_MAC, the \field{command}
can be either VIRTIO_NET_CTRL_MAC_TABLE_SET or VIRTIO_NET_CTRL_MAC_ADDR_SET;
when \field{command} is set to VIRTIO_NET_CTRL_MAC_TABLE_SET, \field{value}
is in the format of \field{struct virtio_net_ctrl_mac}.

Supported selectors are listed in the table:

\begin{tabularx}{\textwidth}{ |l|X| }
\hline
Class selector & Command selector \\
\hline \hline
VIRTIO_NET_CTRL_RX & VIRTIO_NET_CTRL_RX_PROMISC \\
\hline
VIRTIO_NET_CTRL_RX & VIRTIO_NET_CTRL_RX_ALLMULTI \\
\hline
VIRTIO_NET_CTRL_RX & VIRTIO_NET_CTRL_RX_ALLUNI \\
\hline
VIRTIO_NET_CTRL_RX & VIRTIO_NET_CTRL_RX_NOMULTI \\
\hline
VIRTIO_NET_CTRL_RX & VIRTIO_NET_CTRL_RX_NOUNI \\
\hline
VIRTIO_NET_CTRL_RX & VIRTIO_NET_CTRL_RX_NOBCAST \\
\hline
VIRTIO_NET_CTRL_MAC & VIRTIO_NET_CTRL_MAC_TABLE_SET \\
\hline
VIRTIO_NET_CTRL_MAC & VIRTIO_NET_CTRL_MAC_ADDR_SET \\
\hline
VIRTIO_NET_CTRL_VLAN & VIRTIO_NET_CTRL_VLAN_ADD \\
\hline
VIRTIO_NET_CTRL_ANNOUNCE & VIRTIO_NET_CTRL_ANNOUNCE_ACK \\
\hline
VIRTIO_NET_CTRL_MQ & VIRTIO_NET_CTRL_MQ_VQ_PAIRS_SET \\
\hline
VIRTIO_NET_CTRL_MQ & VIRTIO_NET_CTRL_MQ_RSS_CONFIG \\
\hline
VIRTIO_NET_CTRL_MQ & VIRTIO_NET_CTRL_MQ_HASH_CONFIG \\
\hline
\hline
\end{tabularx}

For command selector VIRTIO_NET_CTRL_VLAN_ADD, device part consists of a whole
VLAN table.

\field{reserved} is reserved and set to zero.

\subsection{Device Operation}\label{sec:Device Types / Network Device / Device Operation}

Packets are transmitted by placing them in the
transmitq1\ldots transmitqN, and buffers for incoming packets are
placed in the receiveq1\ldots receiveqN. In each case, the packet
itself is preceded by a header:

\begin{lstlisting}
struct virtio_net_hdr {
#define VIRTIO_NET_HDR_F_NEEDS_CSUM    1
#define VIRTIO_NET_HDR_F_DATA_VALID    2
#define VIRTIO_NET_HDR_F_RSC_INFO      4
#define VIRTIO_NET_HDR_F_UDP_TUNNEL_CSUM 8
        u8 flags;
#define VIRTIO_NET_HDR_GSO_NONE        0
#define VIRTIO_NET_HDR_GSO_TCPV4       1
#define VIRTIO_NET_HDR_GSO_UDP         3
#define VIRTIO_NET_HDR_GSO_TCPV6       4
#define VIRTIO_NET_HDR_GSO_UDP_L4      5
#define VIRTIO_NET_HDR_GSO_UDP_TUNNEL_IPV4 0x20
#define VIRTIO_NET_HDR_GSO_UDP_TUNNEL_IPV6 0x40
#define VIRTIO_NET_HDR_GSO_ECN      0x80
        u8 gso_type;
        le16 hdr_len;
        le16 gso_size;
        le16 csum_start;
        le16 csum_offset;
        le16 num_buffers;
        le32 hash_value;        (Only if VIRTIO_NET_F_HASH_REPORT negotiated)
        le16 hash_report;       (Only if VIRTIO_NET_F_HASH_REPORT negotiated)
        le16 padding_reserved;  (Only if VIRTIO_NET_F_HASH_REPORT negotiated)
        le16 outer_th_offset    (Only if VIRTIO_NET_F_HOST_UDP_TUNNEL_GSO or VIRTIO_NET_F_GUEST_UDP_TUNNEL_GSO negotiated)
        le16 inner_nh_offset;   (Only if VIRTIO_NET_F_HOST_UDP_TUNNEL_GSO or VIRTIO_NET_F_GUEST_UDP_TUNNEL_GSO negotiated)
};
\end{lstlisting}

The controlq is used to control various device features described further in
section \ref{sec:Device Types / Network Device / Device Operation / Control Virtqueue}.

\subsubsection{Legacy Interface: Device Operation}\label{sec:Device Types / Network Device / Device Operation / Legacy Interface: Device Operation}
When using the legacy interface, transitional devices and drivers
MUST format the fields in \field{struct virtio_net_hdr}
according to the native endian of the guest rather than
(necessarily when not using the legacy interface) little-endian.

The legacy driver only presented \field{num_buffers} in the \field{struct virtio_net_hdr}
when VIRTIO_NET_F_MRG_RXBUF was negotiated; without that feature the
structure was 2 bytes shorter.

When using the legacy interface, the driver SHOULD ignore the
used length for the transmit queues
and the controlq queue.
\begin{note}
Historically, some devices put
the total descriptor length there, even though no data was
actually written.
\end{note}

\subsubsection{Packet Transmission}\label{sec:Device Types / Network Device / Device Operation / Packet Transmission}

Transmitting a single packet is simple, but varies depending on
the different features the driver negotiated.

\begin{enumerate}
\item The driver can send a completely checksummed packet.  In this case,
  \field{flags} will be zero, and \field{gso_type} will be VIRTIO_NET_HDR_GSO_NONE.

\item If the driver negotiated VIRTIO_NET_F_CSUM, it can skip
  checksumming the packet:
  \begin{itemize}
  \item \field{flags} has the VIRTIO_NET_HDR_F_NEEDS_CSUM set,

  \item \field{csum_start} is set to the offset within the packet to begin checksumming,
    and

  \item \field{csum_offset} indicates how many bytes after the csum_start the
    new (16 bit ones' complement) checksum is placed by the device.

  \item The TCP checksum field in the packet is set to the sum
    of the TCP pseudo header, so that replacing it by the ones'
    complement checksum of the TCP header and body will give the
    correct result.
  \end{itemize}

\begin{note}
For example, consider a partially checksummed TCP (IPv4) packet.
It will have a 14 byte ethernet header and 20 byte IP header
followed by the TCP header (with the TCP checksum field 16 bytes
into that header). \field{csum_start} will be 14+20 = 34 (the TCP
checksum includes the header), and \field{csum_offset} will be 16.
If the given packet has the VIRTIO_NET_HDR_GSO_UDP_TUNNEL_IPV4 bit or the
VIRTIO_NET_HDR_GSO_UDP_TUNNEL_IPV6 bit set,
the above checksum fields refer to the inner header checksum, see
the example below.
\end{note}

\item If the driver negotiated
  VIRTIO_NET_F_HOST_TSO4, TSO6, USO or UFO, and the packet requires
  TCP segmentation, UDP segmentation or fragmentation, then \field{gso_type}
  is set to VIRTIO_NET_HDR_GSO_TCPV4, TCPV6, UDP_L4 or UDP.
  (Otherwise, it is set to VIRTIO_NET_HDR_GSO_NONE). In this
  case, packets larger than 1514 bytes can be transmitted: the
  metadata indicates how to replicate the packet header to cut it
  into smaller packets. The other gso fields are set:

  \begin{itemize}
  \item If the VIRTIO_NET_F_GUEST_HDRLEN feature has been negotiated,
    \field{hdr_len} indicates the header length that needs to be replicated
    for each packet. It's the number of bytes from the beginning of the packet
    to the beginning of the transport payload.
    If the \field{gso_type} has the VIRTIO_NET_HDR_GSO_UDP_TUNNEL_IPV4 bit or
    VIRTIO_NET_HDR_GSO_UDP_TUNNEL_IPV6 bit set, \field{hdr_len} accounts for
    all the headers up to and including the inner transport.
    Otherwise, if the VIRTIO_NET_F_GUEST_HDRLEN feature has not been negotiated,
    \field{hdr_len} is a hint to the device as to how much of the header
    needs to be kept to copy into each packet, usually set to the
    length of the headers, including the transport header\footnote{Due to various bugs in implementations, this field is not useful
as a guarantee of the transport header size.
}.

  \begin{note}
  Some devices benefit from knowledge of the exact header length.
  \end{note}

  \item \field{gso_size} is the maximum size of each packet beyond that
    header (ie. MSS).

  \item If the driver negotiated the VIRTIO_NET_F_HOST_ECN feature,
    the VIRTIO_NET_HDR_GSO_ECN bit in \field{gso_type}
    indicates that the TCP packet has the ECN bit set\footnote{This case is not handled by some older hardware, so is called out
specifically in the protocol.}.
   \end{itemize}

\item If the driver negotiated the VIRTIO_NET_F_HOST_UDP_TUNNEL_GSO feature and the
  \field{gso_type} has the VIRTIO_NET_HDR_GSO_UDP_TUNNEL_IPV4 bit or
  VIRTIO_NET_HDR_GSO_UDP_TUNNEL_IPV6 bit set, the GSO protocol is encapsulated
  in a UDP tunnel.
  If the outer UDP header requires checksumming, the driver must have
  additionally negotiated the VIRTIO_NET_F_HOST_UDP_TUNNEL_GSO_CSUM feature
  and offloaded the outer checksum accordingly, otherwise
  the outer UDP header must not require checksum validation, i.e. the outer
  UDP checksum must be positive zero (0x0) as defined in UDP RFC 768.
  The other tunnel-related fields indicate how to replicate the packet
  headers to cut it into smaller packets:

  \begin{itemize}
  \item \field{outer_th_offset} field indicates the outer transport header within
      the packet. This field differs from \field{csum_start} as the latter
      points to the inner transport header within the packet.

  \item \field{inner_nh_offset} field indicates the inner network header within
      the packet.
  \end{itemize}

\begin{note}
For example, consider a partially checksummed TCP (IPv4) packet carried over a
Geneve UDP tunnel (again IPv4) with no tunnel options. The
only relevant variable related to the tunnel type is the tunnel header length.
The packet will have a 14 byte outer ethernet header, 20 byte outer IP header
followed by the 8 byte UDP header (with a 0 checksum value), 8 byte Geneve header,
14 byte inner ethernet header, 20 byte inner IP header
and the TCP header (with the TCP checksum field 16 bytes
into that header). \field{csum_start} will be 14+20+8+8+14+20 = 84 (the TCP
checksum includes the header), \field{csum_offset} will be 16.
\field{inner_nh_offset} will be 14+20+8+8+14 = 62, \field{outer_th_offset} will be
14+20+8 = 42 and \field{gso_type} will be
VIRTIO_NET_HDR_GSO_TCPV4 | VIRTIO_NET_HDR_GSO_UDP_TUNNEL_IPV4 = 0x21
\end{note}

\item If the driver negotiated the VIRTIO_NET_F_HOST_UDP_TUNNEL_GSO_CSUM feature,
  the transmitted packet is a GSO one encapsulated in a UDP tunnel, and
  the outer UDP header requires checksumming, the driver can skip checksumming
  the outer header:

  \begin{itemize}
  \item \field{flags} has the VIRTIO_NET_HDR_F_UDP_TUNNEL_CSUM set,

  \item The outer UDP checksum field in the packet is set to the sum
    of the UDP pseudo header, so that replacing it by the ones'
    complement checksum of the outer UDP header and payload will give the
    correct result.
  \end{itemize}

\item \field{num_buffers} is set to zero.  This field is unused on transmitted packets.

\item The header and packet are added as one output descriptor to the
  transmitq, and the device is notified of the new entry
  (see \ref{sec:Device Types / Network Device / Device Initialization}~\nameref{sec:Device Types / Network Device / Device Initialization}).
\end{enumerate}

\drivernormative{\paragraph}{Packet Transmission}{Device Types / Network Device / Device Operation / Packet Transmission}

For the transmit packet buffer, the driver MUST use the size of the
structure \field{struct virtio_net_hdr} same as the receive packet buffer.

The driver MUST set \field{num_buffers} to zero.

If VIRTIO_NET_F_CSUM is not negotiated, the driver MUST set
\field{flags} to zero and SHOULD supply a fully checksummed
packet to the device.

If VIRTIO_NET_F_HOST_TSO4 is negotiated, the driver MAY set
\field{gso_type} to VIRTIO_NET_HDR_GSO_TCPV4 to request TCPv4
segmentation, otherwise the driver MUST NOT set
\field{gso_type} to VIRTIO_NET_HDR_GSO_TCPV4.

If VIRTIO_NET_F_HOST_TSO6 is negotiated, the driver MAY set
\field{gso_type} to VIRTIO_NET_HDR_GSO_TCPV6 to request TCPv6
segmentation, otherwise the driver MUST NOT set
\field{gso_type} to VIRTIO_NET_HDR_GSO_TCPV6.

If VIRTIO_NET_F_HOST_UFO is negotiated, the driver MAY set
\field{gso_type} to VIRTIO_NET_HDR_GSO_UDP to request UDP
fragmentation, otherwise the driver MUST NOT set
\field{gso_type} to VIRTIO_NET_HDR_GSO_UDP.

If VIRTIO_NET_F_HOST_USO is negotiated, the driver MAY set
\field{gso_type} to VIRTIO_NET_HDR_GSO_UDP_L4 to request UDP
segmentation, otherwise the driver MUST NOT set
\field{gso_type} to VIRTIO_NET_HDR_GSO_UDP_L4.

The driver SHOULD NOT send to the device TCP packets requiring segmentation offload
which have the Explicit Congestion Notification bit set, unless the
VIRTIO_NET_F_HOST_ECN feature is negotiated, in which case the
driver MUST set the VIRTIO_NET_HDR_GSO_ECN bit in
\field{gso_type}.

If VIRTIO_NET_F_HOST_UDP_TUNNEL_GSO is negotiated, the driver MAY set
VIRTIO_NET_HDR_GSO_UDP_TUNNEL_IPV4 bit or the VIRTIO_NET_HDR_GSO_UDP_TUNNEL_IPV6 bit
in \field{gso_type} according to the inner network header protocol type
to request GSO packets over UDPv4 or UDPv6 tunnel segmentation,
otherwise the driver MUST NOT set either the
VIRTIO_NET_HDR_GSO_UDP_TUNNEL_IPV4 bit or the VIRTIO_NET_HDR_GSO_UDP_TUNNEL_IPV6 bit
in \field{gso_type}.

When requesting GSO segmentation over UDP tunnel, the driver MUST SET the
VIRTIO_NET_HDR_GSO_UDP_TUNNEL_IPV4 bit if the inner network header is IPv4, i.e. the
packet is a TCPv4 GSO one, otherwise, if the inner network header is IPv6, the driver
MUST SET the VIRTIO_NET_HDR_GSO_UDP_TUNNEL_IPV6 bit.

The driver MUST NOT send to the device GSO packets over UDP tunnel
requiring segmentation and outer UDP checksum offload, unless both the
VIRTIO_NET_F_HOST_UDP_TUNNEL_GSO and VIRTIO_NET_F_HOST_UDP_TUNNEL_GSO_CSUM features
are negotiated, in which case the driver MUST set either the
VIRTIO_NET_HDR_GSO_UDP_TUNNEL_IPV4 bit or the VIRTIO_NET_HDR_GSO_UDP_TUNNEL_IPV6
bit in the \field{gso_type} and the VIRTIO_NET_HDR_F_UDP_TUNNEL_CSUM bit in
the \field{flags}.

If VIRTIO_NET_F_HOST_UDP_TUNNEL_GSO_CSUM is not negotiated, the driver MUST not set
the VIRTIO_NET_HDR_F_UDP_TUNNEL_CSUM bit in the \field{flags} and
MUST NOT send to the device GSO packets over UDP tunnel
requiring segmentation and outer UDP checksum offload.

The driver MUST NOT set the VIRTIO_NET_HDR_GSO_UDP_TUNNEL_IPV4 bit or the
VIRTIO_NET_HDR_GSO_UDP_TUNNEL_IPV6 bit together with VIRTIO_NET_HDR_GSO_UDP, as the
latter is deprecated in favor of UDP_L4 and no new feature will support it.

The driver MUST NOT set the VIRTIO_NET_HDR_GSO_UDP_TUNNEL_IPV4 bit and the
VIRTIO_NET_HDR_GSO_UDP_TUNNEL_IPV6 bit together.

The driver MUST NOT set the VIRTIO_NET_HDR_F_UDP_TUNNEL_CSUM bit \field{flags}
without setting either the VIRTIO_NET_HDR_GSO_UDP_TUNNEL_IPV4 bit or
the VIRTIO_NET_HDR_GSO_UDP_TUNNEL_IPV6 bit in \field{gso_type}.

If the VIRTIO_NET_F_CSUM feature has been negotiated, the
driver MAY set the VIRTIO_NET_HDR_F_NEEDS_CSUM bit in
\field{flags}, if so:
\begin{enumerate}
\item the driver MUST validate the packet checksum at
	offset \field{csum_offset} from \field{csum_start} as well as all
	preceding offsets;
\begin{note}
If \field{gso_type} differs from VIRTIO_NET_HDR_GSO_NONE and the
VIRTIO_NET_HDR_GSO_UDP_TUNNEL_IPV4 bit or the VIRTIO_NET_HDR_GSO_UDP_TUNNEL_IPV6
bit are not set in \field{gso_type}, \field{csum_offset}
points to the only transport header present in the packet, and there are no
additional preceding checksums validated by VIRTIO_NET_HDR_F_NEEDS_CSUM.
\end{note}
\item the driver MUST set the packet checksum stored in the
	buffer to the TCP/UDP pseudo header;
\item the driver MUST set \field{csum_start} and
	\field{csum_offset} such that calculating a ones'
	complement checksum from \field{csum_start} up until the end of
	the packet and storing the result at offset \field{csum_offset}
	from  \field{csum_start} will result in a fully checksummed
	packet;
\end{enumerate}

If none of the VIRTIO_NET_F_HOST_TSO4, TSO6, USO or UFO options have
been negotiated, the driver MUST set \field{gso_type} to
VIRTIO_NET_HDR_GSO_NONE.

If \field{gso_type} differs from VIRTIO_NET_HDR_GSO_NONE, then
the driver MUST also set the VIRTIO_NET_HDR_F_NEEDS_CSUM bit in
\field{flags} and MUST set \field{gso_size} to indicate the
desired MSS.

If one of the VIRTIO_NET_F_HOST_TSO4, TSO6, USO or UFO options have
been negotiated:
\begin{itemize}
\item If the VIRTIO_NET_F_GUEST_HDRLEN feature has been negotiated,
	and \field{gso_type} differs from VIRTIO_NET_HDR_GSO_NONE,
	the driver MUST set \field{hdr_len} to a value equal to the length
	of the headers, including the transport header. If \field{gso_type}
	has the VIRTIO_NET_HDR_GSO_UDP_TUNNEL_IPV4 bit or the
	VIRTIO_NET_HDR_GSO_UDP_TUNNEL_IPV6 bit set, \field{hdr_len} includes
	the inner transport header.

\item If the VIRTIO_NET_F_GUEST_HDRLEN feature has not been negotiated,
	or \field{gso_type} is VIRTIO_NET_HDR_GSO_NONE,
	the driver SHOULD set \field{hdr_len} to a value
	not less than the length of the headers, including the transport
	header.
\end{itemize}

If the VIRTIO_NET_F_HOST_UDP_TUNNEL_GSO option has been negotiated, the
driver MAY set the VIRTIO_NET_HDR_GSO_UDP_TUNNEL_IPV4 bit or the
VIRTIO_NET_HDR_GSO_UDP_TUNNEL_IPV6 bit in \field{gso_type}, if so:
\begin{itemize}
\item the driver MUST set \field{outer_th_offset} to the outer UDP header
  offset and \field{inner_nh_offset} to the inner network header offset.
  The \field{csum_start} and \field{csum_offset} fields point respectively
  to the inner transport header and inner transport checksum field.
\end{itemize}

If the VIRTIO_NET_F_HOST_UDP_TUNNEL_GSO_CSUM feature has been negotiated,
and the VIRTIO_NET_HDR_GSO_UDP_TUNNEL_IPV4 bit or
VIRTIO_NET_HDR_GSO_UDP_TUNNEL_IPV6 bit in \field{gso_type} are set,
the driver MAY set the VIRTIO_NET_HDR_F_UDP_TUNNEL_CSUM bit in
\field{flags}, if so the driver MUST set the packet outer UDP header checksum
to the outer UDP pseudo header checksum.

\begin{note}
calculating a ones' complement checksum from \field{outer_th_offset}
up until the end of the packet and storing the result at offset 6
from \field{outer_th_offset} will result in a fully checksummed outer UDP packet;
\end{note}

If the VIRTIO_NET_HDR_GSO_UDP_TUNNEL_IPV4 bit or the
VIRTIO_NET_HDR_GSO_UDP_TUNNEL_IPV6 bit in \field{gso_type} are set
and the VIRTIO_NET_F_HOST_UDP_TUNNEL_GSO_CSUM feature has not
been negotiated, the
outer UDP header MUST NOT require checksum validation. That is, the
outer UDP checksum value MUST be 0 or the validated complete checksum
for such header.

\begin{note}
The valid complete checksum of the outer UDP header of individual segments
can be computed by the driver prior to segmentation only if the GSO packet
size is a multiple of \field{gso_size}, because then all segments
have the same size and thus all data included in the outer UDP
checksum is the same for every segment. These pre-computed segment
length and checksum fields are different from those of the GSO
packet.
In this scenario the outer UDP header of the GSO packet must carry the
segmented UDP packet length.
\end{note}

If the VIRTIO_NET_F_HOST_UDP_TUNNEL_GSO option has not
been negotiated, the driver MUST NOT set either the VIRTIO_NET_HDR_F_GSO_UDP_TUNNEL_IPV4
bit or the VIRTIO_NET_HDR_F_GSO_UDP_TUNNEL_IPV6 in \field{gso_type}.

If the VIRTIO_NET_F_HOST_UDP_TUNNEL_GSO_CSUM option has not been negotiated,
the driver MUST NOT set the VIRTIO_NET_HDR_F_UDP_TUNNEL_CSUM bit
in \field{flags}.

The driver SHOULD accept the VIRTIO_NET_F_GUEST_HDRLEN feature if it has
been offered, and if it's able to provide the exact header length.

The driver MUST NOT set the VIRTIO_NET_HDR_F_DATA_VALID and
VIRTIO_NET_HDR_F_RSC_INFO bits in \field{flags}.

The driver MUST NOT set the VIRTIO_NET_HDR_F_DATA_VALID bit in \field{flags}
together with the VIRTIO_NET_HDR_F_GSO_UDP_TUNNEL_IPV4 bit or the
VIRTIO_NET_HDR_F_GSO_UDP_TUNNEL_IPV6 bit in \field{gso_type}.

\devicenormative{\paragraph}{Packet Transmission}{Device Types / Network Device / Device Operation / Packet Transmission}
The device MUST ignore \field{flag} bits that it does not recognize.

If VIRTIO_NET_HDR_F_NEEDS_CSUM bit in \field{flags} is not set, the
device MUST NOT use the \field{csum_start} and \field{csum_offset}.

If one of the VIRTIO_NET_F_HOST_TSO4, TSO6, USO or UFO options have
been negotiated:
\begin{itemize}
\item If the VIRTIO_NET_F_GUEST_HDRLEN feature has been negotiated,
	and \field{gso_type} differs from VIRTIO_NET_HDR_GSO_NONE,
	the device MAY use \field{hdr_len} as the transport header size.

	\begin{note}
	Caution should be taken by the implementation so as to prevent
	a malicious driver from attacking the device by setting an incorrect hdr_len.
	\end{note}

\item If the VIRTIO_NET_F_GUEST_HDRLEN feature has not been negotiated,
	or \field{gso_type} is VIRTIO_NET_HDR_GSO_NONE,
	the device MAY use \field{hdr_len} only as a hint about the
	transport header size.
	The device MUST NOT rely on \field{hdr_len} to be correct.

	\begin{note}
	This is due to various bugs in implementations.
	\end{note}
\end{itemize}

If both the VIRTIO_NET_HDR_GSO_UDP_TUNNEL_IPV4 bit and
the VIRTIO_NET_HDR_GSO_UDP_TUNNEL_IPV6 bit in in \field{gso_type} are set,
the device MUST NOT accept the packet.

If the VIRTIO_NET_HDR_GSO_UDP_TUNNEL_IPV4 bit and the VIRTIO_NET_HDR_GSO_UDP_TUNNEL_IPV6
bit in \field{gso_type} are not set, the device MUST NOT use the
\field{outer_th_offset} and \field{inner_nh_offset}.

If either the VIRTIO_NET_HDR_GSO_UDP_TUNNEL_IPV4 bit or
the VIRTIO_NET_HDR_GSO_UDP_TUNNEL_IPV6 bit in \field{gso_type} are set, and any of
the following is true:
\begin{itemize}
\item the VIRTIO_NET_HDR_F_NEEDS_CSUM is not set in \field{flags}
\item the VIRTIO_NET_HDR_F_DATA_VALID is set in \field{flags}
\item the \field{gso_type} excluding the VIRTIO_NET_HDR_GSO_UDP_TUNNEL_IPV4
bit and the VIRTIO_NET_HDR_GSO_UDP_TUNNEL_IPV6 bit is VIRTIO_NET_HDR_GSO_NONE
\end{itemize}
the device MUST NOT accept the packet.

If the VIRTIO_NET_HDR_F_UDP_TUNNEL_CSUM bit in \field{flags} is set,
and both the bits VIRTIO_NET_HDR_GSO_UDP_TUNNEL_IPV4 and
VIRTIO_NET_HDR_GSO_UDP_TUNNEL_IPV6 in \field{gso_type} are not set,
the device MOST NOT accept the packet.

If VIRTIO_NET_HDR_F_NEEDS_CSUM is not set, the device MUST NOT
rely on the packet checksum being correct.
\paragraph{Packet Transmission Interrupt}\label{sec:Device Types / Network Device / Device Operation / Packet Transmission / Packet Transmission Interrupt}

Often a driver will suppress transmission virtqueue interrupts
and check for used packets in the transmit path of following
packets.

The normal behavior in this interrupt handler is to retrieve
used buffers from the virtqueue and free the corresponding
headers and packets.

\subsubsection{Setting Up Receive Buffers}\label{sec:Device Types / Network Device / Device Operation / Setting Up Receive Buffers}

It is generally a good idea to keep the receive virtqueue as
fully populated as possible: if it runs out, network performance
will suffer.

If the VIRTIO_NET_F_GUEST_TSO4, VIRTIO_NET_F_GUEST_TSO6,
VIRTIO_NET_F_GUEST_UFO, VIRTIO_NET_F_GUEST_USO4 or VIRTIO_NET_F_GUEST_USO6
features are used, the maximum incoming packet
will be 65589 bytes long (14 bytes of Ethernet header, plus 40 bytes of
the IPv6 header, plus 65535 bytes of maximum IPv6 payload including any
extension header), otherwise 1514 bytes.
When VIRTIO_NET_F_HASH_REPORT is not negotiated, the required receive buffer
size is either 65601 or 1526 bytes accounting for 20 bytes of
\field{struct virtio_net_hdr} followed by receive packet.
When VIRTIO_NET_F_HASH_REPORT is negotiated, the required receive buffer
size is either 65609 or 1534 bytes accounting for 12 bytes of
\field{struct virtio_net_hdr} followed by receive packet.

\drivernormative{\paragraph}{Setting Up Receive Buffers}{Device Types / Network Device / Device Operation / Setting Up Receive Buffers}

\begin{itemize}
\item If VIRTIO_NET_F_MRG_RXBUF is not negotiated:
  \begin{itemize}
    \item If VIRTIO_NET_F_GUEST_TSO4, VIRTIO_NET_F_GUEST_TSO6, VIRTIO_NET_F_GUEST_UFO,
	VIRTIO_NET_F_GUEST_USO4 or VIRTIO_NET_F_GUEST_USO6 are negotiated, the driver SHOULD populate
      the receive queue(s) with buffers of at least 65609 bytes if
      VIRTIO_NET_F_HASH_REPORT is negotiated, and of at least 65601 bytes if not.
    \item Otherwise, the driver SHOULD populate the receive queue(s)
      with buffers of at least 1534 bytes if VIRTIO_NET_F_HASH_REPORT
      is negotiated, and of at least 1526 bytes if not.
  \end{itemize}
\item If VIRTIO_NET_F_MRG_RXBUF is negotiated, each buffer MUST be at
least size of \field{struct virtio_net_hdr},
i.e. 20 bytes if VIRTIO_NET_F_HASH_REPORT is negotiated, and 12 bytes if not.
\end{itemize}

\begin{note}
Obviously each buffer can be split across multiple descriptor elements.
\end{note}

When calculating the size of \field{struct virtio_net_hdr}, the driver
MUST consider all the fields inclusive up to \field{padding_reserved},
i.e. 20 bytes if VIRTIO_NET_F_HASH_REPORT is negotiated, and 12 bytes if not.

If VIRTIO_NET_F_MQ is negotiated, each of receiveq1\ldots receiveqN
that will be used SHOULD be populated with receive buffers.

\devicenormative{\paragraph}{Setting Up Receive Buffers}{Device Types / Network Device / Device Operation / Setting Up Receive Buffers}

The device MUST set \field{num_buffers} to the number of descriptors used to
hold the incoming packet.

The device MUST use only a single descriptor if VIRTIO_NET_F_MRG_RXBUF
was not negotiated.
\begin{note}
{This means that \field{num_buffers} will always be 1
if VIRTIO_NET_F_MRG_RXBUF is not negotiated.}
\end{note}

\subsubsection{Processing of Incoming Packets}\label{sec:Device Types / Network Device / Device Operation / Processing of Incoming Packets}
\label{sec:Device Types / Network Device / Device Operation / Processing of Packets}%old label for latexdiff

When a packet is copied into a buffer in the receiveq, the
optimal path is to disable further used buffer notifications for the
receiveq and process packets until no more are found, then re-enable
them.

Processing incoming packets involves:

\begin{enumerate}
\item \field{num_buffers} indicates how many descriptors
  this packet is spread over (including this one): this will
  always be 1 if VIRTIO_NET_F_MRG_RXBUF was not negotiated.
  This allows receipt of large packets without having to allocate large
  buffers: a packet that does not fit in a single buffer can flow
  over to the next buffer, and so on. In this case, there will be
  at least \field{num_buffers} used buffers in the virtqueue, and the device
  chains them together to form a single packet in a way similar to
  how it would store it in a single buffer spread over multiple
  descriptors.
  The other buffers will not begin with a \field{struct virtio_net_hdr}.

\item If
  \field{num_buffers} is one, then the entire packet will be
  contained within this buffer, immediately following the struct
  virtio_net_hdr.
\item If the VIRTIO_NET_F_GUEST_CSUM feature was negotiated, the
  VIRTIO_NET_HDR_F_DATA_VALID bit in \field{flags} can be
  set: if so, device has validated the packet checksum.
  If the VIRTIO_NET_F_GUEST_UDP_TUNNEL_GSO_CSUM feature has been negotiated,
  and the VIRTIO_NET_HDR_F_UDP_TUNNEL_CSUM bit is set in \field{flags},
  both the outer UDP checksum and the inner transport checksum
  have been validated, otherwise only one level of checksums (the outer one
  in case of tunnels) has been validated.
\end{enumerate}

Additionally, VIRTIO_NET_F_GUEST_CSUM, TSO4, TSO6, UDP, UDP_TUNNEL
and ECN features enable receive checksum, large receive offload and ECN
support which are the input equivalents of the transmit checksum,
transmit segmentation offloading and ECN features, as described
in \ref{sec:Device Types / Network Device / Device Operation /
Packet Transmission}:
\begin{enumerate}
\item If the VIRTIO_NET_F_GUEST_TSO4, TSO6, UFO, USO4 or USO6 options were
  negotiated, then \field{gso_type} MAY be something other than
  VIRTIO_NET_HDR_GSO_NONE, and \field{gso_size} field indicates the
  desired MSS (see Packet Transmission point 2).
\item If the VIRTIO_NET_F_RSC_EXT option was negotiated (this
  implies one of VIRTIO_NET_F_GUEST_TSO4, TSO6), the
  device processes also duplicated ACK segments, reports
  number of coalesced TCP segments in \field{csum_start} field and
  number of duplicated ACK segments in \field{csum_offset} field
  and sets bit VIRTIO_NET_HDR_F_RSC_INFO in \field{flags}.
\item If the VIRTIO_NET_F_GUEST_CSUM feature was negotiated, the
  VIRTIO_NET_HDR_F_NEEDS_CSUM bit in \field{flags} can be
  set: if so, the packet checksum at offset \field{csum_offset}
  from \field{csum_start} and any preceding checksums
  have been validated.  The checksum on the packet is incomplete and
  if bit VIRTIO_NET_HDR_F_RSC_INFO is not set in \field{flags},
  then \field{csum_start} and \field{csum_offset} indicate how to calculate it
  (see Packet Transmission point 1).
\begin{note}
If \field{gso_type} differs from VIRTIO_NET_HDR_GSO_NONE and the
VIRTIO_NET_HDR_GSO_UDP_TUNNEL_IPV4 bit or the VIRTIO_NET_HDR_GSO_UDP_TUNNEL_IPV6
bit are not set, \field{csum_offset}
points to the only transport header present in the packet, and there are no
additional preceding checksums validated by VIRTIO_NET_HDR_F_NEEDS_CSUM.
\end{note}
\item If the VIRTIO_NET_F_GUEST_UDP_TUNNEL_GSO option was negotiated and
  \field{gso_type} is not VIRTIO_NET_HDR_GSO_NONE, the
  VIRTIO_NET_HDR_GSO_UDP_TUNNEL_IPV4 bit or the VIRTIO_NET_HDR_GSO_UDP_TUNNEL_IPV6
  bit MAY be set. In such case the \field{outer_th_offset} and
  \field{inner_nh_offset} fields indicate the corresponding
  headers information.
\item If the VIRTIO_NET_F_GUEST_UDP_TUNNEL_GSO_CSUM feature was
negotiated, and
  the VIRTIO_NET_HDR_GSO_UDP_TUNNEL_IPV4 bit or the VIRTIO_NET_HDR_GSO_UDP_TUNNEL_IPV6
  are set in \field{gso_type}, the VIRTIO_NET_HDR_F_UDP_TUNNEL_CSUM bit in the
  \field{flags} can be set: if so, the outer UDP checksum has been validated
  and the UDP header checksum at offset 6 from from \field{outer_th_offset}
  is set to the outer UDP pseudo header checksum.

\begin{note}
If the VIRTIO_NET_HDR_GSO_UDP_TUNNEL_IPV4 bit or VIRTIO_NET_HDR_GSO_UDP_TUNNEL_IPV6
bit are set in \field{gso_type}, the \field{csum_start} field refers to
the inner transport header offset (see Packet Transmission point 1).
If the VIRTIO_NET_HDR_F_UDP_TUNNEL_CSUM bit in \field{flags} is set both
the inner and the outer header checksums have been validated by
VIRTIO_NET_HDR_F_NEEDS_CSUM, otherwise only the inner transport header
checksum has been validated.
\end{note}
\end{enumerate}

If applicable, the device calculates per-packet hash for incoming packets as
defined in \ref{sec:Device Types / Network Device / Device Operation / Processing of Incoming Packets / Hash calculation for incoming packets}.

If applicable, the device reports hash information for incoming packets as
defined in \ref{sec:Device Types / Network Device / Device Operation / Processing of Incoming Packets / Hash reporting for incoming packets}.

\devicenormative{\paragraph}{Processing of Incoming Packets}{Device Types / Network Device / Device Operation / Processing of Incoming Packets}
\label{devicenormative:Device Types / Network Device / Device Operation / Processing of Packets}%old label for latexdiff

If VIRTIO_NET_F_MRG_RXBUF has not been negotiated, the device MUST set
\field{num_buffers} to 1.

If VIRTIO_NET_F_MRG_RXBUF has been negotiated, the device MUST set
\field{num_buffers} to indicate the number of buffers
the packet (including the header) is spread over.

If a receive packet is spread over multiple buffers, the device
MUST use all buffers but the last (i.e. the first \field{num_buffers} -
1 buffers) completely up to the full length of each buffer
supplied by the driver.

The device MUST use all buffers used by a single receive
packet together, such that at least \field{num_buffers} are
observed by driver as used.

If VIRTIO_NET_F_GUEST_CSUM is not negotiated, the device MUST set
\field{flags} to zero and SHOULD supply a fully checksummed
packet to the driver.

If VIRTIO_NET_F_GUEST_TSO4 is not negotiated, the device MUST NOT set
\field{gso_type} to VIRTIO_NET_HDR_GSO_TCPV4.

If VIRTIO_NET_F_GUEST_UDP is not negotiated, the device MUST NOT set
\field{gso_type} to VIRTIO_NET_HDR_GSO_UDP.

If VIRTIO_NET_F_GUEST_TSO6 is not negotiated, the device MUST NOT set
\field{gso_type} to VIRTIO_NET_HDR_GSO_TCPV6.

If none of VIRTIO_NET_F_GUEST_USO4 or VIRTIO_NET_F_GUEST_USO6 have been negotiated,
the device MUST NOT set \field{gso_type} to VIRTIO_NET_HDR_GSO_UDP_L4.

If VIRTIO_NET_F_GUEST_UDP_TUNNEL_GSO is not negotiated, the device MUST NOT set
either the VIRTIO_NET_HDR_GSO_UDP_TUNNEL_IPV4 bit or the
VIRTIO_NET_HDR_GSO_UDP_TUNNEL_IPV6 bit in \field{gso_type}.

If VIRTIO_NET_F_GUEST_UDP_TUNNEL_GSO_CSUM is not negotiated the device MUST NOT set
the VIRTIO_NET_HDR_F_UDP_TUNNEL_CSUM bit in \field{flags}.

The device SHOULD NOT send to the driver TCP packets requiring segmentation offload
which have the Explicit Congestion Notification bit set, unless the
VIRTIO_NET_F_GUEST_ECN feature is negotiated, in which case the
device MUST set the VIRTIO_NET_HDR_GSO_ECN bit in
\field{gso_type}.

If the VIRTIO_NET_F_GUEST_CSUM feature has been negotiated, the
device MAY set the VIRTIO_NET_HDR_F_NEEDS_CSUM bit in
\field{flags}, if so:
\begin{enumerate}
\item the device MUST validate the packet checksum at
	offset \field{csum_offset} from \field{csum_start} as well as all
	preceding offsets;
\item the device MUST set the packet checksum stored in the
	receive buffer to the TCP/UDP pseudo header;
\item the device MUST set \field{csum_start} and
	\field{csum_offset} such that calculating a ones'
	complement checksum from \field{csum_start} up until the
	end of the packet and storing the result at offset
	\field{csum_offset} from  \field{csum_start} will result in a
	fully checksummed packet;
\end{enumerate}

The device MUST NOT send to the driver GSO packets encapsulated in UDP
tunnel and requiring segmentation offload, unless the
VIRTIO_NET_F_GUEST_UDP_TUNNEL_GSO is negotiated, in which case the device MUST set
the VIRTIO_NET_HDR_GSO_UDP_TUNNEL_IPV4 bit or the VIRTIO_NET_HDR_GSO_UDP_TUNNEL_IPV6
bit in \field{gso_type} according to the inner network header protocol type,
MUST set the \field{outer_th_offset} and \field{inner_nh_offset} fields
to the corresponding header information, and the outer UDP header MUST NOT
require checksum offload.

If the VIRTIO_NET_F_GUEST_UDP_TUNNEL_GSO_CSUM feature has not been negotiated,
the device MUST NOT send the driver GSO packets encapsulated in UDP
tunnel and requiring segmentation and outer checksum offload.

If none of the VIRTIO_NET_F_GUEST_TSO4, TSO6, UFO, USO4 or USO6 options have
been negotiated, the device MUST set \field{gso_type} to
VIRTIO_NET_HDR_GSO_NONE.

If \field{gso_type} differs from VIRTIO_NET_HDR_GSO_NONE, then
the device MUST also set the VIRTIO_NET_HDR_F_NEEDS_CSUM bit in
\field{flags} MUST set \field{gso_size} to indicate the desired MSS.
If VIRTIO_NET_F_RSC_EXT was negotiated, the device MUST also
set VIRTIO_NET_HDR_F_RSC_INFO bit in \field{flags},
set \field{csum_start} to number of coalesced TCP segments and
set \field{csum_offset} to number of received duplicated ACK segments.

If VIRTIO_NET_F_RSC_EXT was not negotiated, the device MUST
not set VIRTIO_NET_HDR_F_RSC_INFO bit in \field{flags}.

If one of the VIRTIO_NET_F_GUEST_TSO4, TSO6, UFO, USO4 or USO6 options have
been negotiated, the device SHOULD set \field{hdr_len} to a value
not less than the length of the headers, including the transport
header. If \field{gso_type} has the VIRTIO_NET_HDR_GSO_UDP_TUNNEL_IPV4 bit
or the VIRTIO_NET_HDR_GSO_UDP_TUNNEL_IPV6 bit set, the referenced transport
header is the inner one.

If the VIRTIO_NET_F_GUEST_CSUM feature has been negotiated, the
device MAY set the VIRTIO_NET_HDR_F_DATA_VALID bit in
\field{flags}, if so, the device MUST validate the packet
checksum. If the VIRTIO_NET_F_GUEST_UDP_TUNNEL_GSO_CSUM feature has
been negotiated, and the VIRTIO_NET_HDR_F_UDP_TUNNEL_CSUM bit set in
\field{flags}, both the outer UDP checksum and the inner transport
checksum have been validated.
Otherwise level of checksum is validated: in case of multiple
encapsulated protocols the outermost one.

If either the VIRTIO_NET_HDR_GSO_UDP_TUNNEL_IPV4 bit or the
VIRTIO_NET_HDR_GSO_UDP_TUNNEL_IPV6 bit in \field{gso_type} are set,
the device MUST NOT set the VIRTIO_NET_HDR_F_DATA_VALID bit in
\field{flags}.

If the VIRTIO_NET_F_GUEST_UDP_TUNNEL_GSO_CSUM feature has been negotiated
and either the VIRTIO_NET_HDR_GSO_UDP_TUNNEL_IPV4 bit is set or the
VIRTIO_NET_HDR_GSO_UDP_TUNNEL_IPV6 bit is set in \field{gso_type}, the
device MAY set the VIRTIO_NET_HDR_F_UDP_TUNNEL_CSUM bit in
\field{flags}, if so the device MUST set the packet outer UDP checksum
stored in the receive buffer to the outer UDP pseudo header.

Otherwise, the VIRTIO_NET_F_GUEST_UDP_TUNNEL_GSO_CSUM feature has been
negotiated, either the VIRTIO_NET_HDR_GSO_UDP_TUNNEL_IPV4 bit is set or the
VIRTIO_NET_HDR_GSO_UDP_TUNNEL_IPV6 bit is set in \field{gso_type},
and the bit VIRTIO_NET_HDR_F_UDP_TUNNEL_CSUM is not set in
\field{flags}, the device MUST either provide a zero outer UDP header
checksum or a fully checksummed outer UDP header.

\drivernormative{\paragraph}{Processing of Incoming
Packets}{Device Types / Network Device / Device Operation /
Processing of Incoming Packets}

The driver MUST ignore \field{flag} bits that it does not recognize.

If VIRTIO_NET_HDR_F_NEEDS_CSUM bit in \field{flags} is not set or
if VIRTIO_NET_HDR_F_RSC_INFO bit \field{flags} is set, the
driver MUST NOT use the \field{csum_start} and \field{csum_offset}.

If one of the VIRTIO_NET_F_GUEST_TSO4, TSO6, UFO, USO4 or USO6 options have
been negotiated, the driver MAY use \field{hdr_len} only as a hint about the
transport header size.
The driver MUST NOT rely on \field{hdr_len} to be correct.
\begin{note}
This is due to various bugs in implementations.
\end{note}

If neither VIRTIO_NET_HDR_F_NEEDS_CSUM nor
VIRTIO_NET_HDR_F_DATA_VALID is set, the driver MUST NOT
rely on the packet checksum being correct.

If both the VIRTIO_NET_HDR_GSO_UDP_TUNNEL_IPV4 bit and
the VIRTIO_NET_HDR_GSO_UDP_TUNNEL_IPV6 bit in in \field{gso_type} are set,
the driver MUST NOT accept the packet.

If the VIRTIO_NET_HDR_GSO_UDP_TUNNEL_IPV4 bit or the VIRTIO_NET_HDR_GSO_UDP_TUNNEL_IPV6
bit in \field{gso_type} are not set, the driver MUST NOT use the
\field{outer_th_offset} and \field{inner_nh_offset}.

If either the VIRTIO_NET_HDR_GSO_UDP_TUNNEL_IPV4 bit or
the VIRTIO_NET_HDR_GSO_UDP_TUNNEL_IPV6 bit in \field{gso_type} are set, and any of
the following is true:
\begin{itemize}
\item the VIRTIO_NET_HDR_F_NEEDS_CSUM bit is not set in \field{flags}
\item the VIRTIO_NET_HDR_F_DATA_VALID bit is set in \field{flags}
\item the \field{gso_type} excluding the VIRTIO_NET_HDR_GSO_UDP_TUNNEL_IPV4
bit and the VIRTIO_NET_HDR_GSO_UDP_TUNNEL_IPV6 bit is VIRTIO_NET_HDR_GSO_NONE
\end{itemize}
the driver MUST NOT accept the packet.

If the VIRTIO_NET_HDR_F_UDP_TUNNEL_CSUM bit and the VIRTIO_NET_HDR_F_NEEDS_CSUM
bit in \field{flags} are set,
and both the bits VIRTIO_NET_HDR_GSO_UDP_TUNNEL_IPV4 and
VIRTIO_NET_HDR_GSO_UDP_TUNNEL_IPV6 in \field{gso_type} are not set,
the driver MOST NOT accept the packet.

\paragraph{Hash calculation for incoming packets}
\label{sec:Device Types / Network Device / Device Operation / Processing of Incoming Packets / Hash calculation for incoming packets}

A device attempts to calculate a per-packet hash in the following cases:
\begin{itemize}
\item The feature VIRTIO_NET_F_RSS was negotiated. The device uses the hash to determine the receive virtqueue to place incoming packets.
\item The feature VIRTIO_NET_F_HASH_REPORT was negotiated. The device reports the hash value and the hash type with the packet.
\end{itemize}

If the feature VIRTIO_NET_F_RSS was negotiated:
\begin{itemize}
\item The device uses \field{hash_types} of the virtio_net_rss_config structure as 'Enabled hash types' bitmask.
\item If additionally the feature VIRTIO_NET_F_HASH_TUNNEL was negotiated, the device uses \field{enabled_tunnel_types} of the
      virtnet_hash_tunnel structure as 'Encapsulation types enabled for inner header hash' bitmask.
\item The device uses a key as defined in \field{hash_key_data} and \field{hash_key_length} of the virtio_net_rss_config structure (see
\ref{sec:Device Types / Network Device / Device Operation / Control Virtqueue / Receive-side scaling (RSS) / Setting RSS parameters}).
\end{itemize}

If the feature VIRTIO_NET_F_RSS was not negotiated:
\begin{itemize}
\item The device uses \field{hash_types} of the virtio_net_hash_config structure as 'Enabled hash types' bitmask.
\item If additionally the feature VIRTIO_NET_F_HASH_TUNNEL was negotiated, the device uses \field{enabled_tunnel_types} of the
      virtnet_hash_tunnel structure as 'Encapsulation types enabled for inner header hash' bitmask.
\item The device uses a key as defined in \field{hash_key_data} and \field{hash_key_length} of the virtio_net_hash_config structure (see
\ref{sec:Device Types / Network Device / Device Operation / Control Virtqueue / Automatic receive steering in multiqueue mode / Hash calculation}).
\end{itemize}

Note that if the device offers VIRTIO_NET_F_HASH_REPORT, even if it supports only one pair of virtqueues, it MUST support
at least one of commands of VIRTIO_NET_CTRL_MQ class to configure reported hash parameters:
\begin{itemize}
\item If the device offers VIRTIO_NET_F_RSS, it MUST support VIRTIO_NET_CTRL_MQ_RSS_CONFIG command per
 \ref{sec:Device Types / Network Device / Device Operation / Control Virtqueue / Receive-side scaling (RSS) / Setting RSS parameters}.
\item Otherwise the device MUST support VIRTIO_NET_CTRL_MQ_HASH_CONFIG command per
 \ref{sec:Device Types / Network Device / Device Operation / Control Virtqueue / Automatic receive steering in multiqueue mode / Hash calculation}.
\end{itemize}

The per-packet hash calculation can depend on the IP packet type. See
\hyperref[intro:IP]{[IP]}, \hyperref[intro:UDP]{[UDP]} and \hyperref[intro:TCP]{[TCP]}.

\subparagraph{Supported/enabled hash types}
\label{sec:Device Types / Network Device / Device Operation / Processing of Incoming Packets / Hash calculation for incoming packets / Supported/enabled hash types}
Hash types applicable for IPv4 packets:
\begin{lstlisting}
#define VIRTIO_NET_HASH_TYPE_IPv4              (1 << 0)
#define VIRTIO_NET_HASH_TYPE_TCPv4             (1 << 1)
#define VIRTIO_NET_HASH_TYPE_UDPv4             (1 << 2)
\end{lstlisting}
Hash types applicable for IPv6 packets without extension headers
\begin{lstlisting}
#define VIRTIO_NET_HASH_TYPE_IPv6              (1 << 3)
#define VIRTIO_NET_HASH_TYPE_TCPv6             (1 << 4)
#define VIRTIO_NET_HASH_TYPE_UDPv6             (1 << 5)
\end{lstlisting}
Hash types applicable for IPv6 packets with extension headers
\begin{lstlisting}
#define VIRTIO_NET_HASH_TYPE_IP_EX             (1 << 6)
#define VIRTIO_NET_HASH_TYPE_TCP_EX            (1 << 7)
#define VIRTIO_NET_HASH_TYPE_UDP_EX            (1 << 8)
\end{lstlisting}

\subparagraph{IPv4 packets}
\label{sec:Device Types / Network Device / Device Operation / Processing of Incoming Packets / Hash calculation for incoming packets / IPv4 packets}
The device calculates the hash on IPv4 packets according to 'Enabled hash types' bitmask as follows:
\begin{itemize}
\item If VIRTIO_NET_HASH_TYPE_TCPv4 is set and the packet has
a TCP header, the hash is calculated over the following fields:
\begin{itemize}
\item Source IP address
\item Destination IP address
\item Source TCP port
\item Destination TCP port
\end{itemize}
\item Else if VIRTIO_NET_HASH_TYPE_UDPv4 is set and the
packet has a UDP header, the hash is calculated over the following fields:
\begin{itemize}
\item Source IP address
\item Destination IP address
\item Source UDP port
\item Destination UDP port
\end{itemize}
\item Else if VIRTIO_NET_HASH_TYPE_IPv4 is set, the hash is
calculated over the following fields:
\begin{itemize}
\item Source IP address
\item Destination IP address
\end{itemize}
\item Else the device does not calculate the hash
\end{itemize}

\subparagraph{IPv6 packets without extension header}
\label{sec:Device Types / Network Device / Device Operation / Processing of Incoming Packets / Hash calculation for incoming packets / IPv6 packets without extension header}
The device calculates the hash on IPv6 packets without extension
headers according to 'Enabled hash types' bitmask as follows:
\begin{itemize}
\item If VIRTIO_NET_HASH_TYPE_TCPv6 is set and the packet has
a TCPv6 header, the hash is calculated over the following fields:
\begin{itemize}
\item Source IPv6 address
\item Destination IPv6 address
\item Source TCP port
\item Destination TCP port
\end{itemize}
\item Else if VIRTIO_NET_HASH_TYPE_UDPv6 is set and the
packet has a UDPv6 header, the hash is calculated over the following fields:
\begin{itemize}
\item Source IPv6 address
\item Destination IPv6 address
\item Source UDP port
\item Destination UDP port
\end{itemize}
\item Else if VIRTIO_NET_HASH_TYPE_IPv6 is set, the hash is
calculated over the following fields:
\begin{itemize}
\item Source IPv6 address
\item Destination IPv6 address
\end{itemize}
\item Else the device does not calculate the hash
\end{itemize}

\subparagraph{IPv6 packets with extension header}
\label{sec:Device Types / Network Device / Device Operation / Processing of Incoming Packets / Hash calculation for incoming packets / IPv6 packets with extension header}
The device calculates the hash on IPv6 packets with extension
headers according to 'Enabled hash types' bitmask as follows:
\begin{itemize}
\item If VIRTIO_NET_HASH_TYPE_TCP_EX is set and the packet
has a TCPv6 header, the hash is calculated over the following fields:
\begin{itemize}
\item Home address from the home address option in the IPv6 destination options header. If the extension header is not present, use the Source IPv6 address.
\item IPv6 address that is contained in the Routing-Header-Type-2 from the associated extension header. If the extension header is not present, use the Destination IPv6 address.
\item Source TCP port
\item Destination TCP port
\end{itemize}
\item Else if VIRTIO_NET_HASH_TYPE_UDP_EX is set and the
packet has a UDPv6 header, the hash is calculated over the following fields:
\begin{itemize}
\item Home address from the home address option in the IPv6 destination options header. If the extension header is not present, use the Source IPv6 address.
\item IPv6 address that is contained in the Routing-Header-Type-2 from the associated extension header. If the extension header is not present, use the Destination IPv6 address.
\item Source UDP port
\item Destination UDP port
\end{itemize}
\item Else if VIRTIO_NET_HASH_TYPE_IP_EX is set, the hash is
calculated over the following fields:
\begin{itemize}
\item Home address from the home address option in the IPv6 destination options header. If the extension header is not present, use the Source IPv6 address.
\item IPv6 address that is contained in the Routing-Header-Type-2 from the associated extension header. If the extension header is not present, use the Destination IPv6 address.
\end{itemize}
\item Else skip IPv6 extension headers and calculate the hash as
defined for an IPv6 packet without extension headers
(see \ref{sec:Device Types / Network Device / Device Operation / Processing of Incoming Packets / Hash calculation for incoming packets / IPv6 packets without extension header}).
\end{itemize}

\paragraph{Inner Header Hash}
\label{sec:Device Types / Network Device / Device Operation / Processing of Incoming Packets / Inner Header Hash}

If VIRTIO_NET_F_HASH_TUNNEL has been negotiated, the driver can send the command
VIRTIO_NET_CTRL_HASH_TUNNEL_SET to configure the calculation of the inner header hash.

\begin{lstlisting}
struct virtnet_hash_tunnel {
    le32 enabled_tunnel_types;
};

#define VIRTIO_NET_CTRL_HASH_TUNNEL 7
 #define VIRTIO_NET_CTRL_HASH_TUNNEL_SET 0
\end{lstlisting}

Field \field{enabled_tunnel_types} contains the bitmask of encapsulation types enabled for inner header hash.
See \ref{sec:Device Types / Network Device / Device Operation / Processing of Incoming Packets /
Hash calculation for incoming packets / Encapsulation types supported/enabled for inner header hash}.

The class VIRTIO_NET_CTRL_HASH_TUNNEL has one command:
VIRTIO_NET_CTRL_HASH_TUNNEL_SET sets \field{enabled_tunnel_types} for the device using the
virtnet_hash_tunnel structure, which is read-only for the device.

Inner header hash is disabled by VIRTIO_NET_CTRL_HASH_TUNNEL_SET with \field{enabled_tunnel_types} set to 0.

Initially (before the driver sends any VIRTIO_NET_CTRL_HASH_TUNNEL_SET command) all
encapsulation types are disabled for inner header hash.

\subparagraph{Encapsulated packet}
\label{sec:Device Types / Network Device / Device Operation / Processing of Incoming Packets / Hash calculation for incoming packets / Encapsulated packet}

Multiple tunneling protocols allow encapsulating an inner, payload packet in an outer, encapsulated packet.
The encapsulated packet thus contains an outer header and an inner header, and the device calculates the
hash over either the inner header or the outer header.

If VIRTIO_NET_F_HASH_TUNNEL is negotiated and a received encapsulated packet's outer header matches one of the
encapsulation types enabled in \field{enabled_tunnel_types}, then the device uses the inner header for hash
calculations (only a single level of encapsulation is currently supported).

If VIRTIO_NET_F_HASH_TUNNEL is negotiated and a received packet's (outer) header does not match any encapsulation
types enabled in \field{enabled_tunnel_types}, then the device uses the outer header for hash calculations.

\subparagraph{Encapsulation types supported/enabled for inner header hash}
\label{sec:Device Types / Network Device / Device Operation / Processing of Incoming Packets /
Hash calculation for incoming packets / Encapsulation types supported/enabled for inner header hash}

Encapsulation types applicable for inner header hash:
\begin{lstlisting}[escapechar=|]
#define VIRTIO_NET_HASH_TUNNEL_TYPE_GRE_2784    (1 << 0) /* |\hyperref[intro:rfc2784]{[RFC2784]}| */
#define VIRTIO_NET_HASH_TUNNEL_TYPE_GRE_2890    (1 << 1) /* |\hyperref[intro:rfc2890]{[RFC2890]}| */
#define VIRTIO_NET_HASH_TUNNEL_TYPE_GRE_7676    (1 << 2) /* |\hyperref[intro:rfc7676]{[RFC7676]}| */
#define VIRTIO_NET_HASH_TUNNEL_TYPE_GRE_UDP     (1 << 3) /* |\hyperref[intro:rfc8086]{[GRE-in-UDP]}| */
#define VIRTIO_NET_HASH_TUNNEL_TYPE_VXLAN       (1 << 4) /* |\hyperref[intro:vxlan]{[VXLAN]}| */
#define VIRTIO_NET_HASH_TUNNEL_TYPE_VXLAN_GPE   (1 << 5) /* |\hyperref[intro:vxlan-gpe]{[VXLAN-GPE]}| */
#define VIRTIO_NET_HASH_TUNNEL_TYPE_GENEVE      (1 << 6) /* |\hyperref[intro:geneve]{[GENEVE]}| */
#define VIRTIO_NET_HASH_TUNNEL_TYPE_IPIP        (1 << 7) /* |\hyperref[intro:ipip]{[IPIP]}| */
#define VIRTIO_NET_HASH_TUNNEL_TYPE_NVGRE       (1 << 8) /* |\hyperref[intro:nvgre]{[NVGRE]}| */
\end{lstlisting}

\subparagraph{Advice}
Example uses of the inner header hash:
\begin{itemize}
\item Legacy tunneling protocols, lacking the outer header entropy, can use RSS with the inner header hash to
      distribute flows with identical outer but different inner headers across various queues, improving performance.
\item Identify an inner flow distributed across multiple outer tunnels.
\end{itemize}

As using the inner header hash completely discards the outer header entropy, care must be taken
if the inner header is controlled by an adversary, as the adversary can then intentionally create
configurations with insufficient entropy.

Besides disabling the inner header hash, mitigations would depend on how the hash is used. When the hash
use is limited to the RSS queue selection, the inner header hash may have quality of service (QoS) limitations.

\devicenormative{\subparagraph}{Inner Header Hash}{Device Types / Network Device / Device Operation / Control Virtqueue / Inner Header Hash}

If the (outer) header of the received packet does not match any encapsulation types enabled
in \field{enabled_tunnel_types}, the device MUST calculate the hash on the outer header.

If the device receives any bits in \field{enabled_tunnel_types} which are not set in \field{supported_tunnel_types},
it SHOULD respond to the VIRTIO_NET_CTRL_HASH_TUNNEL_SET command with VIRTIO_NET_ERR.

If the driver sets \field{enabled_tunnel_types} to 0 through VIRTIO_NET_CTRL_HASH_TUNNEL_SET or upon the device reset,
the device MUST disable the inner header hash for all encapsulation types.

\drivernormative{\subparagraph}{Inner Header Hash}{Device Types / Network Device / Device Operation / Control Virtqueue / Inner Header Hash}

The driver MUST have negotiated the VIRTIO_NET_F_HASH_TUNNEL feature when issuing the VIRTIO_NET_CTRL_HASH_TUNNEL_SET command.

The driver MUST NOT set any bits in \field{enabled_tunnel_types} which are not set in \field{supported_tunnel_types}.

The driver MUST ignore bits in \field{supported_tunnel_types} which are not documented in this specification.

\paragraph{Hash reporting for incoming packets}
\label{sec:Device Types / Network Device / Device Operation / Processing of Incoming Packets / Hash reporting for incoming packets}

If VIRTIO_NET_F_HASH_REPORT was negotiated and
 the device has calculated the hash for the packet, the device fills \field{hash_report} with the report type of calculated hash
and \field{hash_value} with the value of calculated hash.

If VIRTIO_NET_F_HASH_REPORT was negotiated but due to any reason the
hash was not calculated, the device sets \field{hash_report} to VIRTIO_NET_HASH_REPORT_NONE.

Possible values that the device can report in \field{hash_report} are defined below.
They correspond to supported hash types defined in
\ref{sec:Device Types / Network Device / Device Operation / Processing of Incoming Packets / Hash calculation for incoming packets / Supported/enabled hash types}
as follows:

VIRTIO_NET_HASH_TYPE_XXX = 1 << (VIRTIO_NET_HASH_REPORT_XXX - 1)

\begin{lstlisting}
#define VIRTIO_NET_HASH_REPORT_NONE            0
#define VIRTIO_NET_HASH_REPORT_IPv4            1
#define VIRTIO_NET_HASH_REPORT_TCPv4           2
#define VIRTIO_NET_HASH_REPORT_UDPv4           3
#define VIRTIO_NET_HASH_REPORT_IPv6            4
#define VIRTIO_NET_HASH_REPORT_TCPv6           5
#define VIRTIO_NET_HASH_REPORT_UDPv6           6
#define VIRTIO_NET_HASH_REPORT_IPv6_EX         7
#define VIRTIO_NET_HASH_REPORT_TCPv6_EX        8
#define VIRTIO_NET_HASH_REPORT_UDPv6_EX        9
\end{lstlisting}

\subsubsection{Control Virtqueue}\label{sec:Device Types / Network Device / Device Operation / Control Virtqueue}

The driver uses the control virtqueue (if VIRTIO_NET_F_CTRL_VQ is
negotiated) to send commands to manipulate various features of
the device which would not easily map into the configuration
space.

All commands are of the following form:

\begin{lstlisting}
struct virtio_net_ctrl {
        u8 class;
        u8 command;
        u8 command-specific-data[];
        u8 ack;
        u8 command-specific-result[];
};

/* ack values */
#define VIRTIO_NET_OK     0
#define VIRTIO_NET_ERR    1
\end{lstlisting}

The \field{class}, \field{command} and command-specific-data are set by the
driver, and the device sets the \field{ack} byte and optionally
\field{command-specific-result}. There is little the driver can
do except issue a diagnostic if \field{ack} is not VIRTIO_NET_OK.

The command VIRTIO_NET_CTRL_STATS_QUERY and VIRTIO_NET_CTRL_STATS_GET contain
\field{command-specific-result}.

\paragraph{Packet Receive Filtering}\label{sec:Device Types / Network Device / Device Operation / Control Virtqueue / Packet Receive Filtering}
\label{sec:Device Types / Network Device / Device Operation / Control Virtqueue / Setting Promiscuous Mode}%old label for latexdiff

If the VIRTIO_NET_F_CTRL_RX and VIRTIO_NET_F_CTRL_RX_EXTRA
features are negotiated, the driver can send control commands for
promiscuous mode, multicast, unicast and broadcast receiving.

\begin{note}
In general, these commands are best-effort: unwanted
packets could still arrive.
\end{note}

\begin{lstlisting}
#define VIRTIO_NET_CTRL_RX    0
 #define VIRTIO_NET_CTRL_RX_PROMISC      0
 #define VIRTIO_NET_CTRL_RX_ALLMULTI     1
 #define VIRTIO_NET_CTRL_RX_ALLUNI       2
 #define VIRTIO_NET_CTRL_RX_NOMULTI      3
 #define VIRTIO_NET_CTRL_RX_NOUNI        4
 #define VIRTIO_NET_CTRL_RX_NOBCAST      5
\end{lstlisting}


\devicenormative{\subparagraph}{Packet Receive Filtering}{Device Types / Network Device / Device Operation / Control Virtqueue / Packet Receive Filtering}

If the VIRTIO_NET_F_CTRL_RX feature has been negotiated,
the device MUST support the following VIRTIO_NET_CTRL_RX class
commands:
\begin{itemize}
\item VIRTIO_NET_CTRL_RX_PROMISC turns promiscuous mode on and
off. The command-specific-data is one byte containing 0 (off) or
1 (on). If promiscuous mode is on, the device SHOULD receive all
incoming packets.
This SHOULD take effect even if one of the other modes set by
a VIRTIO_NET_CTRL_RX class command is on.
\item VIRTIO_NET_CTRL_RX_ALLMULTI turns all-multicast receive on and
off. The command-specific-data is one byte containing 0 (off) or
1 (on). When all-multicast receive is on the device SHOULD allow
all incoming multicast packets.
\end{itemize}

If the VIRTIO_NET_F_CTRL_RX_EXTRA feature has been negotiated,
the device MUST support the following VIRTIO_NET_CTRL_RX class
commands:
\begin{itemize}
\item VIRTIO_NET_CTRL_RX_ALLUNI turns all-unicast receive on and
off. The command-specific-data is one byte containing 0 (off) or
1 (on). When all-unicast receive is on the device SHOULD allow
all incoming unicast packets.
\item VIRTIO_NET_CTRL_RX_NOMULTI suppresses multicast receive.
The command-specific-data is one byte containing 0 (multicast
receive allowed) or 1 (multicast receive suppressed).
When multicast receive is suppressed, the device SHOULD NOT
send multicast packets to the driver.
This SHOULD take effect even if VIRTIO_NET_CTRL_RX_ALLMULTI is on.
This filter SHOULD NOT apply to broadcast packets.
\item VIRTIO_NET_CTRL_RX_NOUNI suppresses unicast receive.
The command-specific-data is one byte containing 0 (unicast
receive allowed) or 1 (unicast receive suppressed).
When unicast receive is suppressed, the device SHOULD NOT
send unicast packets to the driver.
This SHOULD take effect even if VIRTIO_NET_CTRL_RX_ALLUNI is on.
\item VIRTIO_NET_CTRL_RX_NOBCAST suppresses broadcast receive.
The command-specific-data is one byte containing 0 (broadcast
receive allowed) or 1 (broadcast receive suppressed).
When broadcast receive is suppressed, the device SHOULD NOT
send broadcast packets to the driver.
This SHOULD take effect even if VIRTIO_NET_CTRL_RX_ALLMULTI is on.
\end{itemize}

\drivernormative{\subparagraph}{Packet Receive Filtering}{Device Types / Network Device / Device Operation / Control Virtqueue / Packet Receive Filtering}

If the VIRTIO_NET_F_CTRL_RX feature has not been negotiated,
the driver MUST NOT issue commands VIRTIO_NET_CTRL_RX_PROMISC or
VIRTIO_NET_CTRL_RX_ALLMULTI.

If the VIRTIO_NET_F_CTRL_RX_EXTRA feature has not been negotiated,
the driver MUST NOT issue commands
 VIRTIO_NET_CTRL_RX_ALLUNI,
 VIRTIO_NET_CTRL_RX_NOMULTI,
 VIRTIO_NET_CTRL_RX_NOUNI or
 VIRTIO_NET_CTRL_RX_NOBCAST.

\paragraph{Setting MAC Address Filtering}\label{sec:Device Types / Network Device / Device Operation / Control Virtqueue / Setting MAC Address Filtering}

If the VIRTIO_NET_F_CTRL_RX feature is negotiated, the driver can
send control commands for MAC address filtering.

\begin{lstlisting}
struct virtio_net_ctrl_mac {
        le32 entries;
        u8 macs[entries][6];
};

#define VIRTIO_NET_CTRL_MAC    1
 #define VIRTIO_NET_CTRL_MAC_TABLE_SET        0
 #define VIRTIO_NET_CTRL_MAC_ADDR_SET         1
\end{lstlisting}

The device can filter incoming packets by any number of destination
MAC addresses\footnote{Since there are no guarantees, it can use a hash filter or
silently switch to allmulti or promiscuous mode if it is given too
many addresses.
}. This table is set using the class
VIRTIO_NET_CTRL_MAC and the command VIRTIO_NET_CTRL_MAC_TABLE_SET. The
command-specific-data is two variable length tables of 6-byte MAC
addresses (as described in struct virtio_net_ctrl_mac). The first table contains unicast addresses, and the second
contains multicast addresses.

The VIRTIO_NET_CTRL_MAC_ADDR_SET command is used to set the
default MAC address which rx filtering
accepts (and if VIRTIO_NET_F_MAC has been negotiated,
this will be reflected in \field{mac} in config space).

The command-specific-data for VIRTIO_NET_CTRL_MAC_ADDR_SET is
the 6-byte MAC address.

\devicenormative{\subparagraph}{Setting MAC Address Filtering}{Device Types / Network Device / Device Operation / Control Virtqueue / Setting MAC Address Filtering}

The device MUST have an empty MAC filtering table on reset.

The device MUST update the MAC filtering table before it consumes
the VIRTIO_NET_CTRL_MAC_TABLE_SET command.

The device MUST update \field{mac} in config space before it consumes
the VIRTIO_NET_CTRL_MAC_ADDR_SET command, if VIRTIO_NET_F_MAC has
been negotiated.

The device SHOULD drop incoming packets which have a destination MAC which
matches neither the \field{mac} (or that set with VIRTIO_NET_CTRL_MAC_ADDR_SET)
nor the MAC filtering table.

\drivernormative{\subparagraph}{Setting MAC Address Filtering}{Device Types / Network Device / Device Operation / Control Virtqueue / Setting MAC Address Filtering}

If VIRTIO_NET_F_CTRL_RX has not been negotiated,
the driver MUST NOT issue VIRTIO_NET_CTRL_MAC class commands.

If VIRTIO_NET_F_CTRL_RX has been negotiated,
the driver SHOULD issue VIRTIO_NET_CTRL_MAC_ADDR_SET
to set the default mac if it is different from \field{mac}.

The driver MUST follow the VIRTIO_NET_CTRL_MAC_TABLE_SET command
by a le32 number, followed by that number of non-multicast
MAC addresses, followed by another le32 number, followed by
that number of multicast addresses.  Either number MAY be 0.

\subparagraph{Legacy Interface: Setting MAC Address Filtering}\label{sec:Device Types / Network Device / Device Operation / Control Virtqueue / Setting MAC Address Filtering / Legacy Interface: Setting MAC Address Filtering}
When using the legacy interface, transitional devices and drivers
MUST format \field{entries} in struct virtio_net_ctrl_mac
according to the native endian of the guest rather than
(necessarily when not using the legacy interface) little-endian.

Legacy drivers that didn't negotiate VIRTIO_NET_F_CTRL_MAC_ADDR
changed \field{mac} in config space when NIC is accepting
incoming packets. These drivers always wrote the mac value from
first to last byte, therefore after detecting such drivers,
a transitional device MAY defer MAC update, or MAY defer
processing incoming packets until driver writes the last byte
of \field{mac} in the config space.

\paragraph{VLAN Filtering}\label{sec:Device Types / Network Device / Device Operation / Control Virtqueue / VLAN Filtering}

If the driver negotiates the VIRTIO_NET_F_CTRL_VLAN feature, it
can control a VLAN filter table in the device. The VLAN filter
table applies only to VLAN tagged packets.

When VIRTIO_NET_F_CTRL_VLAN is negotiated, the device starts with
an empty VLAN filter table.

\begin{note}
Similar to the MAC address based filtering, the VLAN filtering
is also best-effort: unwanted packets could still arrive.
\end{note}

\begin{lstlisting}
#define VIRTIO_NET_CTRL_VLAN       2
 #define VIRTIO_NET_CTRL_VLAN_ADD             0
 #define VIRTIO_NET_CTRL_VLAN_DEL             1
\end{lstlisting}

Both the VIRTIO_NET_CTRL_VLAN_ADD and VIRTIO_NET_CTRL_VLAN_DEL
command take a little-endian 16-bit VLAN id as the command-specific-data.

VIRTIO_NET_CTRL_VLAN_ADD command adds the specified VLAN to the
VLAN filter table.

VIRTIO_NET_CTRL_VLAN_DEL command removes the specified VLAN from
the VLAN filter table.

\devicenormative{\subparagraph}{VLAN Filtering}{Device Types / Network Device / Device Operation / Control Virtqueue / VLAN Filtering}

When VIRTIO_NET_F_CTRL_VLAN is not negotiated, the device MUST
accept all VLAN tagged packets.

When VIRTIO_NET_F_CTRL_VLAN is negotiated, the device MUST
accept all VLAN tagged packets whose VLAN tag is present in
the VLAN filter table and SHOULD drop all VLAN tagged packets
whose VLAN tag is absent in the VLAN filter table.

\subparagraph{Legacy Interface: VLAN Filtering}\label{sec:Device Types / Network Device / Device Operation / Control Virtqueue / VLAN Filtering / Legacy Interface: VLAN Filtering}
When using the legacy interface, transitional devices and drivers
MUST format the VLAN id
according to the native endian of the guest rather than
(necessarily when not using the legacy interface) little-endian.

\paragraph{Gratuitous Packet Sending}\label{sec:Device Types / Network Device / Device Operation / Control Virtqueue / Gratuitous Packet Sending}

If the driver negotiates the VIRTIO_NET_F_GUEST_ANNOUNCE (depends
on VIRTIO_NET_F_CTRL_VQ), the device can ask the driver to send gratuitous
packets; this is usually done after the guest has been physically
migrated, and needs to announce its presence on the new network
links. (As hypervisor does not have the knowledge of guest
network configuration (eg. tagged vlan) it is simplest to prod
the guest in this way).

\begin{lstlisting}
#define VIRTIO_NET_CTRL_ANNOUNCE       3
 #define VIRTIO_NET_CTRL_ANNOUNCE_ACK             0
\end{lstlisting}

The driver checks VIRTIO_NET_S_ANNOUNCE bit in the device configuration \field{status} field
when it notices the changes of device configuration. The
command VIRTIO_NET_CTRL_ANNOUNCE_ACK is used to indicate that
driver has received the notification and device clears the
VIRTIO_NET_S_ANNOUNCE bit in \field{status}.

Processing this notification involves:

\begin{enumerate}
\item Sending the gratuitous packets (eg. ARP) or marking there are pending
  gratuitous packets to be sent and letting deferred routine to
  send them.

\item Sending VIRTIO_NET_CTRL_ANNOUNCE_ACK command through control
  vq.
\end{enumerate}

\drivernormative{\subparagraph}{Gratuitous Packet Sending}{Device Types / Network Device / Device Operation / Control Virtqueue / Gratuitous Packet Sending}

If the driver negotiates VIRTIO_NET_F_GUEST_ANNOUNCE, it SHOULD notify
network peers of its new location after it sees the VIRTIO_NET_S_ANNOUNCE bit
in \field{status}.  The driver MUST send a command on the command queue
with class VIRTIO_NET_CTRL_ANNOUNCE and command VIRTIO_NET_CTRL_ANNOUNCE_ACK.

\devicenormative{\subparagraph}{Gratuitous Packet Sending}{Device Types / Network Device / Device Operation / Control Virtqueue / Gratuitous Packet Sending}

If VIRTIO_NET_F_GUEST_ANNOUNCE is negotiated, the device MUST clear the
VIRTIO_NET_S_ANNOUNCE bit in \field{status} upon receipt of a command buffer
with class VIRTIO_NET_CTRL_ANNOUNCE and command VIRTIO_NET_CTRL_ANNOUNCE_ACK
before marking the buffer as used.

\paragraph{Device operation in multiqueue mode}\label{sec:Device Types / Network Device / Device Operation / Control Virtqueue / Device operation in multiqueue mode}

This specification defines the following modes that a device MAY implement for operation with multiple transmit/receive virtqueues:
\begin{itemize}
\item Automatic receive steering as defined in \ref{sec:Device Types / Network Device / Device Operation / Control Virtqueue / Automatic receive steering in multiqueue mode}.
 If a device supports this mode, it offers the VIRTIO_NET_F_MQ feature bit.
\item Receive-side scaling as defined in \ref{devicenormative:Device Types / Network Device / Device Operation / Control Virtqueue / Receive-side scaling (RSS) / RSS processing}.
 If a device supports this mode, it offers the VIRTIO_NET_F_RSS feature bit.
\end{itemize}

A device MAY support one of these features or both. The driver MAY negotiate any set of these features that the device supports.

Multiqueue is disabled by default.

The driver enables multiqueue by sending a command using \field{class} VIRTIO_NET_CTRL_MQ. The \field{command} selects the mode of multiqueue operation, as follows:
\begin{lstlisting}
#define VIRTIO_NET_CTRL_MQ    4
 #define VIRTIO_NET_CTRL_MQ_VQ_PAIRS_SET        0 (for automatic receive steering)
 #define VIRTIO_NET_CTRL_MQ_RSS_CONFIG          1 (for configurable receive steering)
 #define VIRTIO_NET_CTRL_MQ_HASH_CONFIG         2 (for configurable hash calculation)
\end{lstlisting}

If more than one multiqueue mode is negotiated, the resulting device configuration is defined by the last command sent by the driver.

\paragraph{Automatic receive steering in multiqueue mode}\label{sec:Device Types / Network Device / Device Operation / Control Virtqueue / Automatic receive steering in multiqueue mode}

If the driver negotiates the VIRTIO_NET_F_MQ feature bit (depends on VIRTIO_NET_F_CTRL_VQ), it MAY transmit outgoing packets on one
of the multiple transmitq1\ldots transmitqN and ask the device to
queue incoming packets into one of the multiple receiveq1\ldots receiveqN
depending on the packet flow.

The driver enables multiqueue by
sending the VIRTIO_NET_CTRL_MQ_VQ_PAIRS_SET command, specifying
the number of the transmit and receive queues to be used up to
\field{max_virtqueue_pairs}; subsequently,
transmitq1\ldots transmitqn and receiveq1\ldots receiveqn where
n=\field{virtqueue_pairs} MAY be used.
\begin{lstlisting}
struct virtio_net_ctrl_mq_pairs_set {
       le16 virtqueue_pairs;
};
#define VIRTIO_NET_CTRL_MQ_VQ_PAIRS_MIN        1
#define VIRTIO_NET_CTRL_MQ_VQ_PAIRS_MAX        0x8000

\end{lstlisting}

When multiqueue is enabled by VIRTIO_NET_CTRL_MQ_VQ_PAIRS_SET command, the device MUST use automatic receive steering
based on packet flow. Programming of the receive steering
classificator is implicit. After the driver transmitted a packet of a
flow on transmitqX, the device SHOULD cause incoming packets for that flow to
be steered to receiveqX. For uni-directional protocols, or where
no packets have been transmitted yet, the device MAY steer a packet
to a random queue out of the specified receiveq1\ldots receiveqn.

Multiqueue is disabled by VIRTIO_NET_CTRL_MQ_VQ_PAIRS_SET with \field{virtqueue_pairs} to 1 (this is
the default) and waiting for the device to use the command buffer.

\drivernormative{\subparagraph}{Automatic receive steering in multiqueue mode}{Device Types / Network Device / Device Operation / Control Virtqueue / Automatic receive steering in multiqueue mode}

The driver MUST configure the virtqueues before enabling them with the
VIRTIO_NET_CTRL_MQ_VQ_PAIRS_SET command.

The driver MUST NOT request a \field{virtqueue_pairs} of 0 or
greater than \field{max_virtqueue_pairs} in the device configuration space.

The driver MUST queue packets only on any transmitq1 before the
VIRTIO_NET_CTRL_MQ_VQ_PAIRS_SET command.

The driver MUST NOT queue packets on transmit queues greater than
\field{virtqueue_pairs} once it has placed the VIRTIO_NET_CTRL_MQ_VQ_PAIRS_SET command in the available ring.

\devicenormative{\subparagraph}{Automatic receive steering in multiqueue mode}{Device Types / Network Device / Device Operation / Control Virtqueue / Automatic receive steering in multiqueue mode}

After initialization of reset, the device MUST queue packets only on receiveq1.

The device MUST NOT queue packets on receive queues greater than
\field{virtqueue_pairs} once it has placed the
VIRTIO_NET_CTRL_MQ_VQ_PAIRS_SET command in a used buffer.

If the destination receive queue is being reset (See \ref{sec:Basic Facilities of a Virtio Device / Virtqueues / Virtqueue Reset}),
the device SHOULD re-select another random queue. If all receive queues are
being reset, the device MUST drop the packet.

\subparagraph{Legacy Interface: Automatic receive steering in multiqueue mode}\label{sec:Device Types / Network Device / Device Operation / Control Virtqueue / Automatic receive steering in multiqueue mode / Legacy Interface: Automatic receive steering in multiqueue mode}
When using the legacy interface, transitional devices and drivers
MUST format \field{virtqueue_pairs}
according to the native endian of the guest rather than
(necessarily when not using the legacy interface) little-endian.

\subparagraph{Hash calculation}\label{sec:Device Types / Network Device / Device Operation / Control Virtqueue / Automatic receive steering in multiqueue mode / Hash calculation}
If VIRTIO_NET_F_HASH_REPORT was negotiated and the device uses automatic receive steering,
the device MUST support a command to configure hash calculation parameters.

The driver provides parameters for hash calculation as follows:

\field{class} VIRTIO_NET_CTRL_MQ, \field{command} VIRTIO_NET_CTRL_MQ_HASH_CONFIG.

The \field{command-specific-data} has following format:
\begin{lstlisting}
struct virtio_net_hash_config {
    le32 hash_types;
    le16 reserved[4];
    u8 hash_key_length;
    u8 hash_key_data[hash_key_length];
};
\end{lstlisting}
Field \field{hash_types} contains a bitmask of allowed hash types as
defined in
\ref{sec:Device Types / Network Device / Device Operation / Processing of Incoming Packets / Hash calculation for incoming packets / Supported/enabled hash types}.
Initially the device has all hash types disabled and reports only VIRTIO_NET_HASH_REPORT_NONE.

Field \field{reserved} MUST contain zeroes. It is defined to make the structure to match the layout of virtio_net_rss_config structure,
defined in \ref{sec:Device Types / Network Device / Device Operation / Control Virtqueue / Receive-side scaling (RSS)}.

Fields \field{hash_key_length} and \field{hash_key_data} define the key to be used in hash calculation.

\paragraph{Receive-side scaling (RSS)}\label{sec:Device Types / Network Device / Device Operation / Control Virtqueue / Receive-side scaling (RSS)}
A device offers the feature VIRTIO_NET_F_RSS if it supports RSS receive steering with Toeplitz hash calculation and configurable parameters.

A driver queries RSS capabilities of the device by reading device configuration as defined in \ref{sec:Device Types / Network Device / Device configuration layout}

\subparagraph{Setting RSS parameters}\label{sec:Device Types / Network Device / Device Operation / Control Virtqueue / Receive-side scaling (RSS) / Setting RSS parameters}

Driver sends a VIRTIO_NET_CTRL_MQ_RSS_CONFIG command using the following format for \field{command-specific-data}:
\begin{lstlisting}
struct rss_rq_id {
   le16 vq_index_1_16: 15; /* Bits 1 to 16 of the virtqueue index */
   le16 reserved: 1; /* Set to zero */
};

struct virtio_net_rss_config {
    le32 hash_types;
    le16 indirection_table_mask;
    struct rss_rq_id unclassified_queue;
    struct rss_rq_id indirection_table[indirection_table_length];
    le16 max_tx_vq;
    u8 hash_key_length;
    u8 hash_key_data[hash_key_length];
};
\end{lstlisting}
Field \field{hash_types} contains a bitmask of allowed hash types as
defined in
\ref{sec:Device Types / Network Device / Device Operation / Processing of Incoming Packets / Hash calculation for incoming packets / Supported/enabled hash types}.

Field \field{indirection_table_mask} is a mask to be applied to
the calculated hash to produce an index in the
\field{indirection_table} array.
Number of entries in \field{indirection_table} is (\field{indirection_table_mask} + 1).

\field{rss_rq_id} is a receive virtqueue id. \field{vq_index_1_16}
consists of bits 1 to 16 of a virtqueue index. For example, a
\field{vq_index_1_16} value of 3 corresponds to virtqueue index 6,
which maps to receiveq4.

Field \field{unclassified_queue} specifies the receive virtqueue id in which to
place unclassified packets.

Field \field{indirection_table} is an array of receive virtqueues ids.

A driver sets \field{max_tx_vq} to inform a device how many transmit virtqueues it may use (transmitq1\ldots transmitq \field{max_tx_vq}).

Fields \field{hash_key_length} and \field{hash_key_data} define the key to be used in hash calculation.

\drivernormative{\subparagraph}{Setting RSS parameters}{Device Types / Network Device / Device Operation / Control Virtqueue / Receive-side scaling (RSS) }

A driver MUST NOT send the VIRTIO_NET_CTRL_MQ_RSS_CONFIG command if the feature VIRTIO_NET_F_RSS has not been negotiated.

A driver MUST fill the \field{indirection_table} array only with
enabled receive virtqueues ids.

The number of entries in \field{indirection_table} (\field{indirection_table_mask} + 1) MUST be a power of two.

A driver MUST use \field{indirection_table_mask} values that are less than \field{rss_max_indirection_table_length} reported by a device.

A driver MUST NOT set any VIRTIO_NET_HASH_TYPE_ flags that are not supported by a device.

\devicenormative{\subparagraph}{RSS processing}{Device Types / Network Device / Device Operation / Control Virtqueue / Receive-side scaling (RSS) / RSS processing}
The device MUST determine the destination queue for a network packet as follows:
\begin{itemize}
\item Calculate the hash of the packet as defined in \ref{sec:Device Types / Network Device / Device Operation / Processing of Incoming Packets / Hash calculation for incoming packets}.
\item If the device did not calculate the hash for the specific packet, the device directs the packet to the receiveq specified by \field{unclassified_queue} of virtio_net_rss_config structure.
\item Apply \field{indirection_table_mask} to the calculated hash
and use the result as the index in the indirection table to get
the destination receive virtqueue id.
\item If the destination receive queue is being reset (See \ref{sec:Basic Facilities of a Virtio Device / Virtqueues / Virtqueue Reset}), the device MUST drop the packet.
\end{itemize}

\paragraph{RSS Context}\label{sec:Device Types / Network Device / Device Operation / Control Virtqueue / RSS Context}

An RSS context consists of configurable parameters specified by \ref{sec:Device Types / Network Device
/ Device Operation / Control Virtqueue / Receive-side scaling (RSS)}.

The RSS configuration supported by VIRTIO_NET_F_RSS is considered the default RSS configuration.

The device offers the feature VIRTIO_NET_F_RSS_CONTEXT if it supports one or multiple RSS contexts
(excluding the default RSS configuration) and configurable parameters.

\subparagraph{Querying RSS Context Capability}\label{sec:Device Types / Network Device / Device Operation / Control Virtqueue / RSS Context / Querying RSS Context Capability}

\begin{lstlisting}
#define VIRTNET_RSS_CTX_CTRL 9
 #define VIRTNET_RSS_CTX_CTRL_CAP_GET  0
 #define VIRTNET_RSS_CTX_CTRL_ADD      1
 #define VIRTNET_RSS_CTX_CTRL_MOD      2
 #define VIRTNET_RSS_CTX_CTRL_DEL      3

struct virtnet_rss_ctx_cap {
    le16 max_rss_contexts;
}
\end{lstlisting}

Field \field{max_rss_contexts} specifies the maximum number of RSS contexts \ref{sec:Device Types / Network Device /
Device Operation / Control Virtqueue / RSS Context} supported by the device.

The driver queries the RSS context capability of the device by sending the command VIRTNET_RSS_CTX_CTRL_CAP_GET
with the structure virtnet_rss_ctx_cap.

For the command VIRTNET_RSS_CTX_CTRL_CAP_GET, the structure virtnet_rss_ctx_cap is write-only for the device.

\subparagraph{Setting RSS Context Parameters}\label{sec:Device Types / Network Device / Device Operation / Control Virtqueue / RSS Context / Setting RSS Context Parameters}

\begin{lstlisting}
struct virtnet_rss_ctx_add_modify {
    le16 rss_ctx_id;
    u8 reserved[6];
    struct virtio_net_rss_config rss;
};

struct virtnet_rss_ctx_del {
    le16 rss_ctx_id;
};
\end{lstlisting}

RSS context parameters:
\begin{itemize}
\item  \field{rss_ctx_id}: ID of the specific RSS context.
\item  \field{rss}: RSS context parameters of the specific RSS context whose id is \field{rss_ctx_id}.
\end{itemize}

\field{reserved} is reserved and it is ignored by the device.

If the feature VIRTIO_NET_F_RSS_CONTEXT has been negotiated, the driver can send the following
VIRTNET_RSS_CTX_CTRL class commands:
\begin{enumerate}
\item VIRTNET_RSS_CTX_CTRL_ADD: use the structure virtnet_rss_ctx_add_modify to
       add an RSS context configured as \field{rss} and id as \field{rss_ctx_id} for the device.
\item VIRTNET_RSS_CTX_CTRL_MOD: use the structure virtnet_rss_ctx_add_modify to
       configure parameters of the RSS context whose id is \field{rss_ctx_id} as \field{rss} for the device.
\item VIRTNET_RSS_CTX_CTRL_DEL: use the structure virtnet_rss_ctx_del to delete
       the RSS context whose id is \field{rss_ctx_id} for the device.
\end{enumerate}

For commands VIRTNET_RSS_CTX_CTRL_ADD and VIRTNET_RSS_CTX_CTRL_MOD, the structure virtnet_rss_ctx_add_modify is read-only for the device.
For the command VIRTNET_RSS_CTX_CTRL_DEL, the structure virtnet_rss_ctx_del is read-only for the device.

\devicenormative{\subparagraph}{RSS Context}{Device Types / Network Device / Device Operation / Control Virtqueue / RSS Context}

The device MUST set \field{max_rss_contexts} to at least 1 if it offers VIRTIO_NET_F_RSS_CONTEXT.

Upon reset, the device MUST clear all previously configured RSS contexts.

\drivernormative{\subparagraph}{RSS Context}{Device Types / Network Device / Device Operation / Control Virtqueue / RSS Context}

The driver MUST have negotiated the VIRTIO_NET_F_RSS_CONTEXT feature when issuing the VIRTNET_RSS_CTX_CTRL class commands.

The driver MUST set \field{rss_ctx_id} to between 1 and \field{max_rss_contexts} inclusive.

The driver MUST NOT send the command VIRTIO_NET_CTRL_MQ_VQ_PAIRS_SET when the device has successfully configured at least one RSS context.

\paragraph{Offloads State Configuration}\label{sec:Device Types / Network Device / Device Operation / Control Virtqueue / Offloads State Configuration}

If the VIRTIO_NET_F_CTRL_GUEST_OFFLOADS feature is negotiated, the driver can
send control commands for dynamic offloads state configuration.

\subparagraph{Setting Offloads State}\label{sec:Device Types / Network Device / Device Operation / Control Virtqueue / Offloads State Configuration / Setting Offloads State}

To configure the offloads, the following layout structure and
definitions are used:

\begin{lstlisting}
le64 offloads;

#define VIRTIO_NET_F_GUEST_CSUM       1
#define VIRTIO_NET_F_GUEST_TSO4       7
#define VIRTIO_NET_F_GUEST_TSO6       8
#define VIRTIO_NET_F_GUEST_ECN        9
#define VIRTIO_NET_F_GUEST_UFO        10
#define VIRTIO_NET_F_GUEST_UDP_TUNNEL_GSO  46
#define VIRTIO_NET_F_GUEST_UDP_TUNNEL_GSO_CSUM 47
#define VIRTIO_NET_F_GUEST_USO4       54
#define VIRTIO_NET_F_GUEST_USO6       55

#define VIRTIO_NET_CTRL_GUEST_OFFLOADS       5
 #define VIRTIO_NET_CTRL_GUEST_OFFLOADS_SET   0
\end{lstlisting}

The class VIRTIO_NET_CTRL_GUEST_OFFLOADS has one command:
VIRTIO_NET_CTRL_GUEST_OFFLOADS_SET applies the new offloads configuration.

le64 value passed as command data is a bitmask, bits set define
offloads to be enabled, bits cleared - offloads to be disabled.

There is a corresponding device feature for each offload. Upon feature
negotiation corresponding offload gets enabled to preserve backward
compatibility.

\drivernormative{\subparagraph}{Setting Offloads State}{Device Types / Network Device / Device Operation / Control Virtqueue / Offloads State Configuration / Setting Offloads State}

A driver MUST NOT enable an offload for which the appropriate feature
has not been negotiated.

\subparagraph{Legacy Interface: Setting Offloads State}\label{sec:Device Types / Network Device / Device Operation / Control Virtqueue / Offloads State Configuration / Setting Offloads State / Legacy Interface: Setting Offloads State}
When using the legacy interface, transitional devices and drivers
MUST format \field{offloads}
according to the native endian of the guest rather than
(necessarily when not using the legacy interface) little-endian.


\paragraph{Notifications Coalescing}\label{sec:Device Types / Network Device / Device Operation / Control Virtqueue / Notifications Coalescing}

If the VIRTIO_NET_F_NOTF_COAL feature is negotiated, the driver can
send commands VIRTIO_NET_CTRL_NOTF_COAL_TX_SET and VIRTIO_NET_CTRL_NOTF_COAL_RX_SET
for notification coalescing.

If the VIRTIO_NET_F_VQ_NOTF_COAL feature is negotiated, the driver can
send commands VIRTIO_NET_CTRL_NOTF_COAL_VQ_SET and VIRTIO_NET_CTRL_NOTF_COAL_VQ_GET
for virtqueue notification coalescing.

\begin{lstlisting}
struct virtio_net_ctrl_coal {
    le32 max_packets;
    le32 max_usecs;
};

struct virtio_net_ctrl_coal_vq {
    le16 vq_index;
    le16 reserved;
    struct virtio_net_ctrl_coal coal;
};

#define VIRTIO_NET_CTRL_NOTF_COAL 6
 #define VIRTIO_NET_CTRL_NOTF_COAL_TX_SET  0
 #define VIRTIO_NET_CTRL_NOTF_COAL_RX_SET 1
 #define VIRTIO_NET_CTRL_NOTF_COAL_VQ_SET 2
 #define VIRTIO_NET_CTRL_NOTF_COAL_VQ_GET 3
\end{lstlisting}

Coalescing parameters:
\begin{itemize}
\item \field{vq_index}: The virtqueue index of an enabled transmit or receive virtqueue.
\item \field{max_usecs} for RX: Maximum number of microseconds to delay a RX notification.
\item \field{max_usecs} for TX: Maximum number of microseconds to delay a TX notification.
\item \field{max_packets} for RX: Maximum number of packets to receive before a RX notification.
\item \field{max_packets} for TX: Maximum number of packets to send before a TX notification.
\end{itemize}

\field{reserved} is reserved and it is ignored by the device.

Read/Write attributes for coalescing parameters:
\begin{itemize}
\item For commands VIRTIO_NET_CTRL_NOTF_COAL_TX_SET and VIRTIO_NET_CTRL_NOTF_COAL_RX_SET, the structure virtio_net_ctrl_coal is write-only for the driver.
\item For the command VIRTIO_NET_CTRL_NOTF_COAL_VQ_SET, the structure virtio_net_ctrl_coal_vq is write-only for the driver.
\item For the command VIRTIO_NET_CTRL_NOTF_COAL_VQ_GET, \field{vq_index} and \field{reserved} are write-only
      for the driver, and the structure virtio_net_ctrl_coal is read-only for the driver.
\end{itemize}

The class VIRTIO_NET_CTRL_NOTF_COAL has the following commands:
\begin{enumerate}
\item VIRTIO_NET_CTRL_NOTF_COAL_TX_SET: use the structure virtio_net_ctrl_coal to set the \field{max_usecs} and \field{max_packets} parameters for all transmit virtqueues.
\item VIRTIO_NET_CTRL_NOTF_COAL_RX_SET: use the structure virtio_net_ctrl_coal to set the \field{max_usecs} and \field{max_packets} parameters for all receive virtqueues.
\item VIRTIO_NET_CTRL_NOTF_COAL_VQ_SET: use the structure virtio_net_ctrl_coal_vq to set the \field{max_usecs} and \field{max_packets} parameters
                                        for an enabled transmit/receive virtqueue whose index is \field{vq_index}.
\item VIRTIO_NET_CTRL_NOTF_COAL_VQ_GET: use the structure virtio_net_ctrl_coal_vq to get the \field{max_usecs} and \field{max_packets} parameters
                                        for an enabled transmit/receive virtqueue whose index is \field{vq_index}.
\end{enumerate}

The device may generate notifications more or less frequently than specified by set commands of the VIRTIO_NET_CTRL_NOTF_COAL class.

If coalescing parameters are being set, the device applies the last coalescing parameters set for a
virtqueue, regardless of the command used to set the parameters. Use the following command sequence
with two pairs of virtqueues as an example:
Each of the following commands sets \field{max_usecs} and \field{max_packets} parameters for virtqueues.
\begin{itemize}
\item Command1: VIRTIO_NET_CTRL_NOTF_COAL_RX_SET sets coalescing parameters for virtqueues having index 0 and index 2. Virtqueues having index 1 and index 3 retain their previous parameters.
\item Command2: VIRTIO_NET_CTRL_NOTF_COAL_VQ_SET with \field{vq_index} = 0 sets coalescing parameters for virtqueue having index 0. Virtqueue having index 2 retains the parameters from command1.
\item Command3: VIRTIO_NET_CTRL_NOTF_COAL_VQ_GET with \field{vq_index} = 0, the device responds with coalescing parameters of vq_index 0 set by command2.
\item Command4: VIRTIO_NET_CTRL_NOTF_COAL_VQ_SET with \field{vq_index} = 1 sets coalescing parameters for virtqueue having index 1. Virtqueue having index 3 retains its previous parameters.
\item Command5: VIRTIO_NET_CTRL_NOTF_COAL_TX_SET sets coalescing parameters for virtqueues having index 1 and index 3, and overrides the parameters set by command4.
\item Command6: VIRTIO_NET_CTRL_NOTF_COAL_VQ_GET with \field{vq_index} = 1, the device responds with coalescing parameters of index 1 set by command5.
\end{itemize}

\subparagraph{Operation}\label{sec:Device Types / Network Device / Device Operation / Control Virtqueue / Notifications Coalescing / Operation}

The device sends a used buffer notification once the notification conditions are met and if the notifications are not suppressed as explained in \ref{sec:Basic Facilities of a Virtio Device / Virtqueues / Used Buffer Notification Suppression}.

When the device has non-zero \field{max_usecs} and non-zero \field{max_packets}, it starts counting microseconds and packets upon receiving/sending a packet.
The device counts packets and microseconds for each receive virtqueue and transmit virtqueue separately.
In this case, the notification conditions are met when \field{max_usecs} microseconds elapse, or upon sending/receiving \field{max_packets} packets, whichever happens first.
Afterwards, the device waits for the next packet and starts counting packets and microseconds again.

When the device has \field{max_usecs} = 0 or \field{max_packets} = 0, the notification conditions are met after every packet received/sent.

\subparagraph{RX Example}\label{sec:Device Types / Network Device / Device Operation / Control Virtqueue / Notifications Coalescing / RX Example}

If, for example:
\begin{itemize}
\item \field{max_usecs} = 10.
\item \field{max_packets} = 15.
\end{itemize}
then each receive virtqueue of a device will operate as follows:
\begin{itemize}
\item The device will count packets received on each virtqueue until it accumulates 15, or until 10 microseconds elapsed since the first one was received.
\item If the notifications are not suppressed by the driver, the device will send an used buffer notification, otherwise, the device will not send an used buffer notification as long as the notifications are suppressed.
\end{itemize}

\subparagraph{TX Example}\label{sec:Device Types / Network Device / Device Operation / Control Virtqueue / Notifications Coalescing / TX Example}

If, for example:
\begin{itemize}
\item \field{max_usecs} = 10.
\item \field{max_packets} = 15.
\end{itemize}
then each transmit virtqueue of a device will operate as follows:
\begin{itemize}
\item The device will count packets sent on each virtqueue until it accumulates 15, or until 10 microseconds elapsed since the first one was sent.
\item If the notifications are not suppressed by the driver, the device will send an used buffer notification, otherwise, the device will not send an used buffer notification as long as the notifications are suppressed.
\end{itemize}

\subparagraph{Notifications When Coalescing Parameters Change}\label{sec:Device Types / Network Device / Device Operation / Control Virtqueue / Notifications Coalescing / Notifications When Coalescing Parameters Change}

When the coalescing parameters of a device change, the device needs to check if the new notification conditions are met and send a used buffer notification if so.

For example, \field{max_packets} = 15 for a device with a single transmit virtqueue: if the device sends 10 packets and afterwards receives a
VIRTIO_NET_CTRL_NOTF_COAL_TX_SET command with \field{max_packets} = 8, then the notification condition is immediately considered to be met;
the device needs to immediately send a used buffer notification, if the notifications are not suppressed by the driver.

\drivernormative{\subparagraph}{Notifications Coalescing}{Device Types / Network Device / Device Operation / Control Virtqueue / Notifications Coalescing}

The driver MUST set \field{vq_index} to the virtqueue index of an enabled transmit or receive virtqueue.

The driver MUST have negotiated the VIRTIO_NET_F_NOTF_COAL feature when issuing commands VIRTIO_NET_CTRL_NOTF_COAL_TX_SET and VIRTIO_NET_CTRL_NOTF_COAL_RX_SET.

The driver MUST have negotiated the VIRTIO_NET_F_VQ_NOTF_COAL feature when issuing commands VIRTIO_NET_CTRL_NOTF_COAL_VQ_SET and VIRTIO_NET_CTRL_NOTF_COAL_VQ_GET.

The driver MUST ignore the values of coalescing parameters received from the VIRTIO_NET_CTRL_NOTF_COAL_VQ_GET command if the device responds with VIRTIO_NET_ERR.

\devicenormative{\subparagraph}{Notifications Coalescing}{Device Types / Network Device / Device Operation / Control Virtqueue / Notifications Coalescing}

The device MUST ignore \field{reserved}.

The device SHOULD respond to VIRTIO_NET_CTRL_NOTF_COAL_TX_SET and VIRTIO_NET_CTRL_NOTF_COAL_RX_SET commands with VIRTIO_NET_ERR if it was not able to change the parameters.

The device MUST respond to the VIRTIO_NET_CTRL_NOTF_COAL_VQ_SET command with VIRTIO_NET_ERR if it was not able to change the parameters.

The device MUST respond to VIRTIO_NET_CTRL_NOTF_COAL_VQ_SET and VIRTIO_NET_CTRL_NOTF_COAL_VQ_GET commands with
VIRTIO_NET_ERR if the designated virtqueue is not an enabled transmit or receive virtqueue.

Upon disabling and re-enabling a transmit virtqueue, the device MUST set the coalescing parameters of the virtqueue
to those configured through the VIRTIO_NET_CTRL_NOTF_COAL_TX_SET command, or, if the driver did not set any TX coalescing parameters, to 0.

Upon disabling and re-enabling a receive virtqueue, the device MUST set the coalescing parameters of the virtqueue
to those configured through the VIRTIO_NET_CTRL_NOTF_COAL_RX_SET command, or, if the driver did not set any RX coalescing parameters, to 0.

The behavior of the device in response to set commands of the VIRTIO_NET_CTRL_NOTF_COAL class is best-effort:
the device MAY generate notifications more or less frequently than specified.

A device SHOULD NOT send used buffer notifications to the driver if the notifications are suppressed, even if the notification conditions are met.

Upon reset, a device MUST initialize all coalescing parameters to 0.

\paragraph{Device Statistics}\label{sec:Device Types / Network Device / Device Operation / Control Virtqueue / Device Statistics}

If the VIRTIO_NET_F_DEVICE_STATS feature is negotiated, the driver can obtain
device statistics from the device by using the following command.

Different types of virtqueues have different statistics. The statistics of the
receiveq are different from those of the transmitq.

The statistics of a certain type of virtqueue are also divided into multiple types
because different types require different features. This enables the expansion
of new statistics.

In one command, the driver can obtain the statistics of one or multiple virtqueues.
Additionally, the driver can obtain multiple type statistics of each virtqueue.

\subparagraph{Query Statistic Capabilities}\label{sec:Device Types / Network Device / Device Operation / Control Virtqueue / Device Statistics / Query Statistic Capabilities}

\begin{lstlisting}
#define VIRTIO_NET_CTRL_STATS         8
#define VIRTIO_NET_CTRL_STATS_QUERY   0
#define VIRTIO_NET_CTRL_STATS_GET     1

struct virtio_net_stats_capabilities {

#define VIRTIO_NET_STATS_TYPE_CVQ       (1 << 32)

#define VIRTIO_NET_STATS_TYPE_RX_BASIC  (1 << 0)
#define VIRTIO_NET_STATS_TYPE_RX_CSUM   (1 << 1)
#define VIRTIO_NET_STATS_TYPE_RX_GSO    (1 << 2)
#define VIRTIO_NET_STATS_TYPE_RX_SPEED  (1 << 3)

#define VIRTIO_NET_STATS_TYPE_TX_BASIC  (1 << 16)
#define VIRTIO_NET_STATS_TYPE_TX_CSUM   (1 << 17)
#define VIRTIO_NET_STATS_TYPE_TX_GSO    (1 << 18)
#define VIRTIO_NET_STATS_TYPE_TX_SPEED  (1 << 19)

    le64 supported_stats_types[1];
}
\end{lstlisting}

To obtain device statistic capability, use the VIRTIO_NET_CTRL_STATS_QUERY
command. When the command completes successfully, \field{command-specific-result}
is in the format of \field{struct virtio_net_stats_capabilities}.

\subparagraph{Get Statistics}\label{sec:Device Types / Network Device / Device Operation / Control Virtqueue / Device Statistics / Get Statistics}

\begin{lstlisting}
struct virtio_net_ctrl_queue_stats {
       struct {
           le16 vq_index;
           le16 reserved[3];
           le64 types_bitmap[1];
       } stats[];
};

struct virtio_net_stats_reply_hdr {
#define VIRTIO_NET_STATS_TYPE_REPLY_CVQ       32

#define VIRTIO_NET_STATS_TYPE_REPLY_RX_BASIC  0
#define VIRTIO_NET_STATS_TYPE_REPLY_RX_CSUM   1
#define VIRTIO_NET_STATS_TYPE_REPLY_RX_GSO    2
#define VIRTIO_NET_STATS_TYPE_REPLY_RX_SPEED  3

#define VIRTIO_NET_STATS_TYPE_REPLY_TX_BASIC  16
#define VIRTIO_NET_STATS_TYPE_REPLY_TX_CSUM   17
#define VIRTIO_NET_STATS_TYPE_REPLY_TX_GSO    18
#define VIRTIO_NET_STATS_TYPE_REPLY_TX_SPEED  19
    u8 type;
    u8 reserved;
    le16 vq_index;
    le16 reserved1;
    le16 size;
}
\end{lstlisting}

To obtain device statistics, use the VIRTIO_NET_CTRL_STATS_GET command with the
\field{command-specific-data} which is in the format of
\field{struct virtio_net_ctrl_queue_stats}. When the command completes
successfully, \field{command-specific-result} contains multiple statistic
results, each statistic result has the \field{struct virtio_net_stats_reply_hdr}
as the header.

The fields of the \field{struct virtio_net_ctrl_queue_stats}:
\begin{description}
    \item [vq_index]
        The index of the virtqueue to obtain the statistics.

    \item [types_bitmap]
        This is a bitmask of the types of statistics to be obtained. Therefore, a
        \field{stats} inside \field{struct virtio_net_ctrl_queue_stats} may
        indicate multiple statistic replies for the virtqueue.
\end{description}

The fields of the \field{struct virtio_net_stats_reply_hdr}:
\begin{description}
    \item [type]
        The type of the reply statistic.

    \item [vq_index]
        The virtqueue index of the reply statistic.

    \item [size]
        The number of bytes for the statistics entry including size of \field{struct virtio_net_stats_reply_hdr}.

\end{description}

\subparagraph{Controlq Statistics}\label{sec:Device Types / Network Device / Device Operation / Control Virtqueue / Device Statistics / Controlq Statistics}

The structure corresponding to the controlq statistics is
\field{struct virtio_net_stats_cvq}. The corresponding type is
VIRTIO_NET_STATS_TYPE_CVQ. This is for the controlq.

\begin{lstlisting}
struct virtio_net_stats_cvq {
    struct virtio_net_stats_reply_hdr hdr;

    le64 command_num;
    le64 ok_num;
};
\end{lstlisting}

\begin{description}
    \item [command_num]
        The number of commands received by the device including the current command.

    \item [ok_num]
        The number of commands completed successfully by the device including the current command.
\end{description}


\subparagraph{Receiveq Basic Statistics}\label{sec:Device Types / Network Device / Device Operation / Control Virtqueue / Device Statistics / Receiveq Basic Statistics}

The structure corresponding to the receiveq basic statistics is
\field{struct virtio_net_stats_rx_basic}. The corresponding type is
VIRTIO_NET_STATS_TYPE_RX_BASIC. This is for the receiveq.

Receiveq basic statistics do not require any feature. As long as the device supports
VIRTIO_NET_F_DEVICE_STATS, the following are the receiveq basic statistics.

\begin{lstlisting}
struct virtio_net_stats_rx_basic {
    struct virtio_net_stats_reply_hdr hdr;

    le64 rx_notifications;

    le64 rx_packets;
    le64 rx_bytes;

    le64 rx_interrupts;

    le64 rx_drops;
    le64 rx_drop_overruns;
};
\end{lstlisting}

The packets described below were all presented on the specified virtqueue.
\begin{description}
    \item [rx_notifications]
        The number of driver notifications received by the device for this
        receiveq.

    \item [rx_packets]
        This is the number of packets passed to the driver by the device.

    \item [rx_bytes]
        This is the bytes of packets passed to the driver by the device.

    \item [rx_interrupts]
        The number of interrupts generated by the device for this receiveq.

    \item [rx_drops]
        This is the number of packets dropped by the device. The count includes
        all types of packets dropped by the device.

    \item [rx_drop_overruns]
        This is the number of packets dropped by the device when no more
        descriptors were available.

\end{description}

\subparagraph{Transmitq Basic Statistics}\label{sec:Device Types / Network Device / Device Operation / Control Virtqueue / Device Statistics / Transmitq Basic Statistics}

The structure corresponding to the transmitq basic statistics is
\field{struct virtio_net_stats_tx_basic}. The corresponding type is
VIRTIO_NET_STATS_TYPE_TX_BASIC. This is for the transmitq.

Transmitq basic statistics do not require any feature. As long as the device supports
VIRTIO_NET_F_DEVICE_STATS, the following are the transmitq basic statistics.

\begin{lstlisting}
struct virtio_net_stats_tx_basic {
    struct virtio_net_stats_reply_hdr hdr;

    le64 tx_notifications;

    le64 tx_packets;
    le64 tx_bytes;

    le64 tx_interrupts;

    le64 tx_drops;
    le64 tx_drop_malformed;
};
\end{lstlisting}

The packets described below are all for a specific virtqueue.
\begin{description}
    \item [tx_notifications]
        The number of driver notifications received by the device for this
        transmitq.

    \item [tx_packets]
        This is the number of packets sent by the device (not the packets
        got from the driver).

    \item [tx_bytes]
        This is the number of bytes sent by the device for all the sent packets
        (not the bytes sent got from the driver).

    \item [tx_interrupts]
        The number of interrupts generated by the device for this transmitq.

    \item [tx_drops]
        The number of packets dropped by the device. The count includes all
        types of packets dropped by the device.

    \item [tx_drop_malformed]
        The number of packets dropped by the device, when the descriptors are
        malformed. For example, the buffer is too short.
\end{description}

\subparagraph{Receiveq CSUM Statistics}\label{sec:Device Types / Network Device / Device Operation / Control Virtqueue / Device Statistics / Receiveq CSUM Statistics}

The structure corresponding to the receiveq checksum statistics is
\field{struct virtio_net_stats_rx_csum}. The corresponding type is
VIRTIO_NET_STATS_TYPE_RX_CSUM. This is for the receiveq.

Only after the VIRTIO_NET_F_GUEST_CSUM is negotiated, the receiveq checksum
statistics can be obtained.

\begin{lstlisting}
struct virtio_net_stats_rx_csum {
    struct virtio_net_stats_reply_hdr hdr;

    le64 rx_csum_valid;
    le64 rx_needs_csum;
    le64 rx_csum_none;
    le64 rx_csum_bad;
};
\end{lstlisting}

The packets described below were all presented on the specified virtqueue.
\begin{description}
    \item [rx_csum_valid]
        The number of packets with VIRTIO_NET_HDR_F_DATA_VALID.

    \item [rx_needs_csum]
        The number of packets with VIRTIO_NET_HDR_F_NEEDS_CSUM.

    \item [rx_csum_none]
        The number of packets without hardware checksum. The packet here refers
        to the non-TCP/UDP packet that the device cannot recognize.

    \item [rx_csum_bad]
        The number of packets with checksum mismatch.

\end{description}

\subparagraph{Transmitq CSUM Statistics}\label{sec:Device Types / Network Device / Device Operation / Control Virtqueue / Device Statistics / Transmitq CSUM Statistics}

The structure corresponding to the transmitq checksum statistics is
\field{struct virtio_net_stats_tx_csum}. The corresponding type is
VIRTIO_NET_STATS_TYPE_TX_CSUM. This is for the transmitq.

Only after the VIRTIO_NET_F_CSUM is negotiated, the transmitq checksum
statistics can be obtained.

The following are the transmitq checksum statistics:

\begin{lstlisting}
struct virtio_net_stats_tx_csum {
    struct virtio_net_stats_reply_hdr hdr;

    le64 tx_csum_none;
    le64 tx_needs_csum;
};
\end{lstlisting}

The packets described below are all for a specific virtqueue.
\begin{description}
    \item [tx_csum_none]
        The number of packets which do not require hardware checksum.

    \item [tx_needs_csum]
        The number of packets which require checksum calculation by the device.

\end{description}

\subparagraph{Receiveq GSO Statistics}\label{sec:Device Types / Network Device / Device Operation / Control Virtqueue / Device Statistics / Receiveq GSO Statistics}

The structure corresponding to the receivq GSO statistics is
\field{struct virtio_net_stats_rx_gso}. The corresponding type is
VIRTIO_NET_STATS_TYPE_RX_GSO. This is for the receiveq.

If one or more of the VIRTIO_NET_F_GUEST_TSO4, VIRTIO_NET_F_GUEST_TSO6
have been negotiated, the receiveq GSO statistics can be obtained.

GSO packets refer to packets passed by the device to the driver where
\field{gso_type} is not VIRTIO_NET_HDR_GSO_NONE.

\begin{lstlisting}
struct virtio_net_stats_rx_gso {
    struct virtio_net_stats_reply_hdr hdr;

    le64 rx_gso_packets;
    le64 rx_gso_bytes;
    le64 rx_gso_packets_coalesced;
    le64 rx_gso_bytes_coalesced;
};
\end{lstlisting}

The packets described below were all presented on the specified virtqueue.
\begin{description}
    \item [rx_gso_packets]
        The number of the GSO packets received by the device.

    \item [rx_gso_bytes]
        The bytes of the GSO packets received by the device.
        This includes the header size of the GSO packet.

    \item [rx_gso_packets_coalesced]
        The number of the GSO packets coalesced by the device.

    \item [rx_gso_bytes_coalesced]
        The bytes of the GSO packets coalesced by the device.
        This includes the header size of the GSO packet.
\end{description}

\subparagraph{Transmitq GSO Statistics}\label{sec:Device Types / Network Device / Device Operation / Control Virtqueue / Device Statistics / Transmitq GSO Statistics}

The structure corresponding to the transmitq GSO statistics is
\field{struct virtio_net_stats_tx_gso}. The corresponding type is
VIRTIO_NET_STATS_TYPE_TX_GSO. This is for the transmitq.

If one or more of the VIRTIO_NET_F_HOST_TSO4, VIRTIO_NET_F_HOST_TSO6,
VIRTIO_NET_F_HOST_USO options have been negotiated, the transmitq GSO statistics
can be obtained.

GSO packets refer to packets passed by the driver to the device where
\field{gso_type} is not VIRTIO_NET_HDR_GSO_NONE.
See more \ref{sec:Device Types / Network Device / Device Operation / Packet
Transmission}.

\begin{lstlisting}
struct virtio_net_stats_tx_gso {
    struct virtio_net_stats_reply_hdr hdr;

    le64 tx_gso_packets;
    le64 tx_gso_bytes;
    le64 tx_gso_segments;
    le64 tx_gso_segments_bytes;
    le64 tx_gso_packets_noseg;
    le64 tx_gso_bytes_noseg;
};
\end{lstlisting}

The packets described below are all for a specific virtqueue.
\begin{description}
    \item [tx_gso_packets]
        The number of the GSO packets sent by the device.

    \item [tx_gso_bytes]
        The bytes of the GSO packets sent by the device.

    \item [tx_gso_segments]
        The number of segments prepared from GSO packets.

    \item [tx_gso_segments_bytes]
        The bytes of segments prepared from GSO packets.

    \item [tx_gso_packets_noseg]
        The number of the GSO packets without segmentation.

    \item [tx_gso_bytes_noseg]
        The bytes of the GSO packets without segmentation.

\end{description}

\subparagraph{Receiveq Speed Statistics}\label{sec:Device Types / Network Device / Device Operation / Control Virtqueue / Device Statistics / Receiveq Speed Statistics}

The structure corresponding to the receiveq speed statistics is
\field{struct virtio_net_stats_rx_speed}. The corresponding type is
VIRTIO_NET_STATS_TYPE_RX_SPEED. This is for the receiveq.

The device has the allowance for the speed. If VIRTIO_NET_F_SPEED_DUPLEX has
been negotiated, the driver can get this by \field{speed}. When the received
packets bitrate exceeds the \field{speed}, some packets may be dropped by the
device.

\begin{lstlisting}
struct virtio_net_stats_rx_speed {
    struct virtio_net_stats_reply_hdr hdr;

    le64 rx_packets_allowance_exceeded;
    le64 rx_bytes_allowance_exceeded;
};
\end{lstlisting}

The packets described below were all presented on the specified virtqueue.
\begin{description}
    \item [rx_packets_allowance_exceeded]
        The number of the packets dropped by the device due to the received
        packets bitrate exceeding the \field{speed}.

    \item [rx_bytes_allowance_exceeded]
        The bytes of the packets dropped by the device due to the received
        packets bitrate exceeding the \field{speed}.

\end{description}

\subparagraph{Transmitq Speed Statistics}\label{sec:Device Types / Network Device / Device Operation / Control Virtqueue / Device Statistics / Transmitq Speed Statistics}

The structure corresponding to the transmitq speed statistics is
\field{struct virtio_net_stats_tx_speed}. The corresponding type is
VIRTIO_NET_STATS_TYPE_TX_SPEED. This is for the transmitq.

The device has the allowance for the speed. If VIRTIO_NET_F_SPEED_DUPLEX has
been negotiated, the driver can get this by \field{speed}. When the transmit
packets bitrate exceeds the \field{speed}, some packets may be dropped by the
device.

\begin{lstlisting}
struct virtio_net_stats_tx_speed {
    struct virtio_net_stats_reply_hdr hdr;

    le64 tx_packets_allowance_exceeded;
    le64 tx_bytes_allowance_exceeded;
};
\end{lstlisting}

The packets described below were all presented on the specified virtqueue.
\begin{description}
    \item [tx_packets_allowance_exceeded]
        The number of the packets dropped by the device due to the transmit packets
        bitrate exceeding the \field{speed}.

    \item [tx_bytes_allowance_exceeded]
        The bytes of the packets dropped by the device due to the transmit packets
        bitrate exceeding the \field{speed}.

\end{description}

\devicenormative{\subparagraph}{Device Statistics}{Device Types / Network Device / Device Operation / Control Virtqueue / Device Statistics}

When the VIRTIO_NET_F_DEVICE_STATS feature is negotiated, the device MUST reply
to the command VIRTIO_NET_CTRL_STATS_QUERY with the
\field{struct virtio_net_stats_capabilities}. \field{supported_stats_types}
includes all the statistic types supported by the device.

If \field{struct virtio_net_ctrl_queue_stats} is incorrect (such as the
following), the device MUST set \field{ack} to VIRTIO_NET_ERR. Even if there is
only one error, the device MUST fail the entire command.
\begin{itemize}
    \item \field{vq_index} exceeds the queue range.
    \item \field{types_bitmap} contains unknown types.
    \item One or more of the bits present in \field{types_bitmap} is not valid
        for the specified virtqueue.
    \item The feature corresponding to the specified \field{types_bitmap} was
        not negotiated.
\end{itemize}

The device MUST set the actual size of the bytes occupied by the reply to the
\field{size} of the \field{hdr}. And the device MUST set the \field{type} and
the \field{vq_index} of the statistic header.

The \field{command-specific-result} buffer allocated by the driver may be
smaller or bigger than all the statistics specified by
\field{struct virtio_net_ctrl_queue_stats}. The device MUST fill up only upto
the valid bytes.

The statistics counter replied by the device MUST wrap around to zero by the
device on the overflow.

\drivernormative{\subparagraph}{Device Statistics}{Device Types / Network Device / Device Operation / Control Virtqueue / Device Statistics}

The types contained in the \field{types_bitmap} MUST be queried from the device
via command VIRTIO_NET_CTRL_STATS_QUERY.

\field{types_bitmap} in \field{struct virtio_net_ctrl_queue_stats} MUST be valid to the
vq specified by \field{vq_index}.

The \field{command-specific-result} buffer allocated by the driver MUST have
enough capacity to store all the statistics reply headers defined in
\field{struct virtio_net_ctrl_queue_stats}. If the
\field{command-specific-result} buffer is fully utilized by the device but some
replies are missed, it is possible that some statistics may exceed the capacity
of the driver's records. In such cases, the driver should allocate additional
space for the \field{command-specific-result} buffer.

\subsubsection{Flow filter}\label{sec:Device Types / Network Device / Device Operation / Flow filter}

A network device can support one or more flow filter rules. Each flow filter rule
is applied by matching a packet and then taking an action, such as directing the packet
to a specific receiveq or dropping the packet. An example of a match is
matching on specific source and destination IP addresses.

A flow filter rule is a device resource object that consists of a key,
a processing priority, and an action to either direct a packet to a
receive queue or drop the packet.

Each rule uses a classifier. The key is matched against the packet using
a classifier, defining which fields in the packet are matched.
A classifier resource object consists of one or more field selectors, each with
a type that specifies the header fields to be matched against, and a mask.
The mask can match whole fields or parts of a field in a header. Each
rule resource object depends on the classifier resource object.

When a packet is received, relevant fields are extracted
(in the same way) from both the packet and the key according to the
classifier. The resulting field contents are then compared -
if they are identical the rule action is taken, if they are not, the rule is ignored.

Multiple flow filter rules are part of a group. The rule resource object
depends on the group. Each rule within a
group has a rule priority, and each group also has a group priority. For a
packet, a group with the highest priority is selected first. Within a group,
rules are applied from highest to lowest priority, until one of the rules
matches the packet and an action is taken. If all the rules within a group
are ignored, the group with the next highest priority is selected, and so on.

The device and the driver indicates flow filter resource limits using the capability
\ref{par:Device Types / Network Device / Device Operation / Flow filter / Device and driver capabilities / VIRTIO-NET-FF-RESOURCE-CAP} specifying the limits on the number of flow filter rule,
group and classifier resource objects. The capability \ref{par:Device Types / Network Device / Device Operation / Flow filter / Device and driver capabilities / VIRTIO-NET-FF-SELECTOR-CAP} specifies which selectors the device supports.
The driver indicates the selectors it is using by setting the flow
filter selector capability, prior to adding any resource objects.

The capability \ref{par:Device Types / Network Device / Device Operation / Flow filter / Device and driver capabilities / VIRTIO-NET-FF-ACTION-CAP} specifies which actions the device supports.

The driver controls the flow filter rule, classifier and group resource objects using
administration commands described in
\ref{sec:Basic Facilities of a Virtio Device / Device groups / Group administration commands / Device resource objects}.

\paragraph{Packet processing order}\label{sec:sec:Device Types / Network Device / Device Operation / Flow filter / Packet processing order}

Note that flow filter rules are applied after MAC/VLAN filtering. Flow filter
rules take precedence over steering: if a flow filter rule results in an action,
the steering configuration does not apply. The steering configuration only applies
to packets for which no flow filter rule action was performed. For example,
incoming packets can be processed in the following order:

\begin{itemize}
\item apply steering configuration received using control virtqueue commands
      VIRTIO_NET_CTRL_RX, VIRTIO_NET_CTRL_MAC and VIRTIO_NET_CTRL_VLAN.
\item apply flow filter rules if any.
\item if no filter rule applied, apply steering configuration received using command
      VIRTIO_NET_CTRL_MQ_RSS_CONFIG or as per automatic receive steering.
\end{itemize}

Some incoming packet processing examples:
\begin{itemize}
\item If the packet is dropped by the flow filter rule, RSS
      steering is ignored for the packet.
\item If the packet is directed to a specific receiveq using flow filter rule,
      the RSS steering is ignored for the packet.
\item If a packet is dropped due to the VIRTIO_NET_CTRL_MAC configuration,
      both flow filter rules and the RSS steering are ignored for the packet.
\item If a packet does not match any flow filter rules,
      the RSS steering is used to select the receiveq for the packet (if enabled).
\item If there are two flow filter groups configured as group_A and group_B
      with respective group priorities as 4, and 5; flow filter rules of
      group_B are applied first having highest group priority, if there is a match,
      the flow filter rules of group_A are ignored; if there is no match for
      the flow filter rules in group_B, the flow filter rules of next level group_A are applied.
\end{itemize}

\paragraph{Device and driver capabilities}
\label{par:Device Types / Network Device / Device Operation / Flow filter / Device and driver capabilities}

\subparagraph{VIRTIO_NET_FF_RESOURCE_CAP}
\label{par:Device Types / Network Device / Device Operation / Flow filter / Device and driver capabilities / VIRTIO-NET-FF-RESOURCE-CAP}

The capability VIRTIO_NET_FF_RESOURCE_CAP indicates the flow filter resource limits.
\field{cap_specific_data} is in the format
\field{struct virtio_net_ff_cap_data}.

\begin{lstlisting}
struct virtio_net_ff_cap_data {
        le32 groups_limit;
        le32 selectors_limit;
        le32 rules_limit;
        le32 rules_per_group_limit;
        u8 last_rule_priority;
        u8 selectors_per_classifier_limit;
};
\end{lstlisting}

\field{groups_limit}, and \field{selectors_limit} represent the maximum
number of flow filter groups and selectors, respectively, that the driver can create.
 \field{rules_limit} is the maximum number of
flow fiilter rules that the driver can create across all the groups.
\field{rules_per_group_limit} is the maximum number of flow filter rules that the driver
can create for each flow filter group.

\field{last_rule_priority} is the highest priority that can be assigned to a
flow filter rule.

\field{selectors_per_classifier_limit} is the maximum number of selectors
that a classifier can have.

\subparagraph{VIRTIO_NET_FF_SELECTOR_CAP}
\label{par:Device Types / Network Device / Device Operation / Flow filter / Device and driver capabilities / VIRTIO-NET-FF-SELECTOR-CAP}

The capability VIRTIO_NET_FF_SELECTOR_CAP lists the supported selectors and the
supported packet header fields for each selector.
\field{cap_specific_data} is in the format \field{struct virtio_net_ff_cap_mask_data}.

\begin{lstlisting}[label={lst:Device Types / Network Device / Device Operation / Flow filter / Device and driver capabilities / VIRTIO-NET-FF-SELECTOR-CAP / virtio-net-ff-selector}]
struct virtio_net_ff_selector {
        u8 type;
        u8 flags;
        u8 reserved[2];
        u8 length;
        u8 reserved1[3];
        u8 mask[];
};

struct virtio_net_ff_cap_mask_data {
        u8 count;
        u8 reserved[7];
        struct virtio_net_ff_selector selectors[];
};

#define VIRTIO_NET_FF_MASK_F_PARTIAL_MASK (1 << 0)
\end{lstlisting}

\field{count} indicates number of valid entries in the \field{selectors} array.
\field{selectors[]} is an array of supported selectors. Within each array entry:
\field{type} specifies the type of the packet header, as defined in table
\ref{table:Device Types / Network Device / Device Operation / Flow filter / Device and driver capabilities / VIRTIO-NET-FF-SELECTOR-CAP / flow filter selector types}. \field{mask} specifies which fields of the
packet header can be matched in a flow filter rule.

Each \field{type} is also listed in table
\ref{table:Device Types / Network Device / Device Operation / Flow filter / Device and driver capabilities / VIRTIO-NET-FF-SELECTOR-CAP / flow filter selector types}. \field{mask} is a byte array
in network byte order. For example, when \field{type} is VIRTIO_NET_FF_MASK_TYPE_IPV6,
the \field{mask} is in the format \hyperref[intro:IPv6-Header-Format]{IPv6 Header Format}.

If partial masking is not set, then all bits in each field have to be either all 0s
to ignore this field or all 1s to match on this field. If partial masking is set,
then any combination of bits can bit set to match on these bits.
For example, when a selector \field{type} is VIRTIO_NET_FF_MASK_TYPE_ETH, if
\field{mask[0-12]} are zero and \field{mask[13-14]} are 0xff (all 1s), it
indicates that matching is only supported for \field{EtherType} of
\field{Ethernet MAC frame}, matching is not supported for
\field{Destination Address} and \field{Source Address}.

The entries in the array \field{selectors} are ordered by
\field{type}, with each \field{type} value only appearing once.

\field{length} is the length of a dynamic array \field{mask} in bytes.
\field{reserved} and \field{reserved1} are reserved and set to zero.

\begin{table}[H]
\caption{Flow filter selector types}
\label{table:Device Types / Network Device / Device Operation / Flow filter / Device and driver capabilities / VIRTIO-NET-FF-SELECTOR-CAP / flow filter selector types}
\begin{tabularx}{\textwidth}{ |l|X|X| }
\hline
Type & Name & Description \\
\hline \hline
0x0 & - & Reserved \\
\hline
0x1 & VIRTIO_NET_FF_MASK_TYPE_ETH & 14 bytes of frame header starting from destination address described in \hyperref[intro:IEEE 802.3-2022]{IEEE 802.3-2022} \\
\hline
0x2 & VIRTIO_NET_FF_MASK_TYPE_IPV4 & 20 bytes of \hyperref[intro:Internet-Header-Format]{IPv4: Internet Header Format} \\
\hline
0x3 & VIRTIO_NET_FF_MASK_TYPE_IPV6 & 40 bytes of \hyperref[intro:IPv6-Header-Format]{IPv6 Header Format} \\
\hline
0x4 & VIRTIO_NET_FF_MASK_TYPE_TCP & 20 bytes of \hyperref[intro:TCP-Header-Format]{TCP Header Format} \\
\hline
0x5 & VIRTIO_NET_FF_MASK_TYPE_UDP & 8 bytes of UDP header described in \hyperref[intro:UDP]{UDP} \\
\hline
0x6 - 0xFF & & Reserved for future \\
\hline
\end{tabularx}
\end{table}

When VIRTIO_NET_FF_MASK_F_PARTIAL_MASK (bit 0) is set, it indicates that
partial masking is supported for all the fields of the selector identified by \field{type}.

For the selector \field{type} VIRTIO_NET_FF_MASK_TYPE_IPV4, if a partial mask is unsupported,
then matching on an individual bit of \field{Flags} in the
\field{IPv4: Internet Header Format} is unsupported. \field{Flags} has to match as a whole
if it is supported.

For the selector \field{type} VIRTIO_NET_FF_MASK_TYPE_IPV4, \field{mask} includes fields
up to the \field{Destination Address}; that is, \field{Options} and
\field{Padding} are excluded.

For the selector \field{type} VIRTIO_NET_FF_MASK_TYPE_IPV6, the \field{Next Header} field
of the \field{mask} corresponds to the \field{Next Header} in the packet
when \field{IPv6 Extension Headers} are not present. When the packet includes
one or more \field{IPv6 Extension Headers}, the \field{Next Header} field of
the \field{mask} corresponds to the \field{Next Header} of the last
\field{IPv6 Extension Header} in the packet.

For the selector \field{type} VIRTIO_NET_FF_MASK_TYPE_TCP, \field{Control bits}
are treated as individual fields for matching; that is, matching individual
\field{Control bits} does not depend on the partial mask support.

\subparagraph{VIRTIO_NET_FF_ACTION_CAP}
\label{par:Device Types / Network Device / Device Operation / Flow filter / Device and driver capabilities / VIRTIO-NET-FF-ACTION-CAP}

The capability VIRTIO_NET_FF_ACTION_CAP lists the supported actions in a rule.
\field{cap_specific_data} is in the format \field{struct virtio_net_ff_cap_actions}.

\begin{lstlisting}
struct virtio_net_ff_actions {
        u8 count;
        u8 reserved[7];
        u8 actions[];
};
\end{lstlisting}

\field{actions} is an array listing all possible actions.
The entries in the array are ordered from the smallest to the largest,
with each supported value appearing exactly once. Each entry can have the
following values:

\begin{table}[H]
\caption{Flow filter rule actions}
\label{table:Device Types / Network Device / Device Operation / Flow filter / Device and driver capabilities / VIRTIO-NET-FF-ACTION-CAP / flow filter rule actions}
\begin{tabularx}{\textwidth}{ |l|X|X| }
\hline
Action & Name & Description \\
\hline \hline
0x0 & - & reserved \\
\hline
0x1 & VIRTIO_NET_FF_ACTION_DROP & Matching packet will be dropped by the device \\
\hline
0x2 & VIRTIO_NET_FF_ACTION_DIRECT_RX_VQ & Matching packet will be directed to a receive queue \\
\hline
0x3 - 0xFF & & Reserved for future \\
\hline
\end{tabularx}
\end{table}

\paragraph{Resource objects}
\label{par:Device Types / Network Device / Device Operation / Flow filter / Resource objects}

\subparagraph{VIRTIO_NET_RESOURCE_OBJ_FF_GROUP}\label{par:Device Types / Network Device / Device Operation / Flow filter / Resource objects / VIRTIO-NET-RESOURCE-OBJ-FF-GROUP}

A flow filter group contains between 0 and \field{rules_limit} rules, as specified by the
capability VIRTIO_NET_FF_RESOURCE_CAP. For the flow filter group object both
\field{resource_obj_specific_data} and
\field{resource_obj_specific_result} are in the format
\field{struct virtio_net_resource_obj_ff_group}.

\begin{lstlisting}
struct virtio_net_resource_obj_ff_group {
        le16 group_priority;
};
\end{lstlisting}

\field{group_priority} specifies the priority for the group. Each group has a
distinct priority. For each incoming packet, the device tries to apply rules
from groups from higher \field{group_priority} value to lower, until either a
rule matches the packet or all groups have been tried.

\subparagraph{VIRTIO_NET_RESOURCE_OBJ_FF_CLASSIFIER}\label{par:Device Types / Network Device / Device Operation / Flow filter / Resource objects / VIRTIO-NET-RESOURCE-OBJ-FF-CLASSIFIER}

A classifier is used to match a flow filter key against a packet. The
classifier defines the desired packet fields to match, and is represented by
the VIRTIO_NET_RESOURCE_OBJ_FF_CLASSIFIER device resource object.

For the flow filter classifier object both \field{resource_obj_specific_data} and
\field{resource_obj_specific_result} are in the format
\field{struct virtio_net_resource_obj_ff_classifier}.

\begin{lstlisting}
struct virtio_net_resource_obj_ff_classifier {
        u8 count;
        u8 reserved[7];
        struct virtio_net_ff_selector selectors[];
};
\end{lstlisting}

A classifier is an array of \field{selectors}. The number of selectors in the
array is indicated by \field{count}. The selector has a type that specifies
the header fields to be matched against, and a mask.
See \ref{lst:Device Types / Network Device / Device Operation / Flow filter / Device and driver capabilities / VIRTIO-NET-FF-SELECTOR-CAP / virtio-net-ff-selector}
for details about selectors.

The first selector is always VIRTIO_NET_FF_MASK_TYPE_ETH. When there are multiple
selectors, a second selector can be either VIRTIO_NET_FF_MASK_TYPE_IPV4
or VIRTIO_NET_FF_MASK_TYPE_IPV6. If the third selector exists, the third
selector can be either VIRTIO_NET_FF_MASK_TYPE_UDP or VIRTIO_NET_FF_MASK_TYPE_TCP.
For example, to match a Ethernet IPv6 UDP packet,
\field{selectors[0].type} is set to VIRTIO_NET_FF_MASK_TYPE_ETH, \field{selectors[1].type}
is set to VIRTIO_NET_FF_MASK_TYPE_IPV6 and \field{selectors[2].type} is
set to VIRTIO_NET_FF_MASK_TYPE_UDP; accordingly, \field{selectors[0].mask[0-13]} is
for Ethernet header fields, \field{selectors[1].mask[0-39]} is set for IPV6 header
and \field{selectors[2].mask[0-7]} is set for UDP header.

When there are multiple selectors, the type of the (N+1)\textsuperscript{th} selector
affects the mask of the (N)\textsuperscript{th} selector. If
\field{count} is 2 or more, all the mask bits within \field{selectors[0]}
corresponding to \field{EtherType} of an Ethernet header are set.

If \field{count} is more than 2:
\begin{itemize}
\item if \field{selector[1].type} is, VIRTIO_NET_FF_MASK_TYPE_IPV4, then, all the mask bits within
\field{selector[1]} for \field{Protocol} is set.
\item if \field{selector[1].type} is, VIRTIO_NET_FF_MASK_TYPE_IPV6, then, all the mask bits within
\field{selector[1]} for \field{Next Header} is set.
\end{itemize}

If for a given packet header field, a subset of bits of a field is to be matched,
and if the partial mask is supported, the flow filter
mask object can specify a mask which has fewer bits set than the packet header
field size. For example, a partial mask for the Ethernet header source mac
address can be of 1-bit for multicast detection instead of 48-bits.

\subparagraph{VIRTIO_NET_RESOURCE_OBJ_FF_RULE}\label{par:Device Types / Network Device / Device Operation / Flow filter / Resource objects / VIRTIO-NET-RESOURCE-OBJ-FF-RULE}

Each flow filter rule resource object comprises a key, a priority, and an action.
For the flow filter rule object,
\field{resource_obj_specific_data} and
\field{resource_obj_specific_result} are in the format
\field{struct virtio_net_resource_obj_ff_rule}.

\begin{lstlisting}
struct virtio_net_resource_obj_ff_rule {
        le32 group_id;
        le32 classifier_id;
        u8 rule_priority;
        u8 key_length; /* length of key in bytes */
        u8 action;
        u8 reserved;
        le16 vq_index;
        u8 reserved1[2];
        u8 keys[][];
};
\end{lstlisting}

\field{group_id} is the resource object ID of the flow filter group to which
this rule belongs. \field{classifier_id} is the resource object ID of the
classifier used to match a packet against the \field{key}.

\field{rule_priority} denotes the priority of the rule within the group
specified by the \field{group_id}.
Rules within the group are applied from the highest to the lowest priority
until a rule matches the packet and an
action is taken. Rules with the same priority can be applied in any order.

\field{reserved} and \field{reserved1} are reserved and set to 0.

\field{keys[][]} is an array of keys to match against packets, using
the classifier specified by \field{classifier_id}. Each entry (key) comprises
a byte array, and they are located one immediately after another.
The size (number of entries) of the array is exactly the same as that of
\field{selectors} in the classifier, or in other words, \field{count}
in the classifier.

\field{key_length} specifies the total length of \field{keys} in bytes.
In other words, it equals the sum total of \field{length} of all
selectors in \field{selectors} in the classifier specified by
\field{classifier_id}.

For example, if a classifier object's \field{selectors[0].type} is
VIRTIO_NET_FF_MASK_TYPE_ETH and \field{selectors[1].type} is
VIRTIO_NET_FF_MASK_TYPE_IPV6,
then selectors[0].length is 14 and selectors[1].length is 40.
Accordingly, the \field{key_length} is set to 54.
This setting indicates that the \field{key} array's length is 54 bytes
comprising a first byte array of 14 bytes for the
Ethernet MAC header in bytes 0-13, immediately followed by 40 bytes for the
IPv6 header in bytes 14-53.

When there are multiple selectors in the classifier object, the key bytes
for (N)\textsuperscript{th} selector are set so that
(N+1)\textsuperscript{th} selector can be matched.

If \field{count} is 2 or more, key bytes of \field{EtherType}
are set according to \hyperref[intro:IEEE 802 Ethertypes]{IEEE 802 Ethertypes}
for VIRTIO_NET_FF_MASK_TYPE_IPV4 or VIRTIO_NET_FF_MASK_TYPE_IPV6 respectively.

If \field{count} is more than 2, when \field{selector[1].type} is
VIRTIO_NET_FF_MASK_TYPE_IPV4 or VIRTIO_NET_FF_MASK_TYPE_IPV6, key
bytes of \field{Protocol} or \field{Next Header} is set as per
\field{Protocol Numbers} defined \hyperref[intro:IANA Protocol Numbers]{IANA Protocol Numbers}
respectively.

\field{action} is the action to take when a packet matches the
\field{key} using the \field{classifier_id}. Supported actions are described in
\ref{table:Device Types / Network Device / Device Operation / Flow filter / Device and driver capabilities / VIRTIO-NET-FF-ACTION-CAP / flow filter rule actions}.

\field{vq_index} specifies a receive virtqueue. When the \field{action} is set
to VIRTIO_NET_FF_ACTION_DIRECT_RX_VQ, and the packet matches the \field{key},
the matching packet is directed to this virtqueue.

Note that at most one action is ever taken for a given packet. If a rule is
applied and an action is taken, the action of other rules is not taken.

\devicenormative{\paragraph}{Flow filter}{Device Types / Network Device / Device Operation / Flow filter}

When the device supports flow filter operations,
\begin{itemize}
\item the device MUST set VIRTIO_NET_FF_RESOURCE_CAP, VIRTIO_NET_FF_SELECTOR_CAP
and VIRTIO_NET_FF_ACTION_CAP capability in the \field{supported_caps} in the
command VIRTIO_ADMIN_CMD_CAP_SUPPORT_QUERY.
\item the device MUST support the administration commands
VIRTIO_ADMIN_CMD_RESOURCE_OBJ_CREATE,
VIRTIO_ADMIN_CMD_RESOURCE_OBJ_MODIFY, VIRTIO_ADMIN_CMD_RESOURCE_OBJ_QUERY,
VIRTIO_ADMIN_CMD_RESOURCE_OBJ_DESTROY for the resource types
VIRTIO_NET_RESOURCE_OBJ_FF_GROUP, VIRTIO_NET_RESOURCE_OBJ_FF_CLASSIFIER and
VIRTIO_NET_RESOURCE_OBJ_FF_RULE.
\end{itemize}

When any of the VIRTIO_NET_FF_RESOURCE_CAP, VIRTIO_NET_FF_SELECTOR_CAP, or
VIRTIO_NET_FF_ACTION_CAP capability is disabled, the device SHOULD set
\field{status} to VIRTIO_ADMIN_STATUS_Q_INVALID_OPCODE for the commands
VIRTIO_ADMIN_CMD_RESOURCE_OBJ_CREATE,
VIRTIO_ADMIN_CMD_RESOURCE_OBJ_MODIFY, VIRTIO_ADMIN_CMD_RESOURCE_OBJ_QUERY,
and VIRTIO_ADMIN_CMD_RESOURCE_OBJ_DESTROY. These commands apply to the resource
\field{type} of VIRTIO_NET_RESOURCE_OBJ_FF_GROUP, VIRTIO_NET_RESOURCE_OBJ_FF_CLASSIFIER, and
VIRTIO_NET_RESOURCE_OBJ_FF_RULE.

The device SHOULD set \field{status} to VIRTIO_ADMIN_STATUS_EINVAL for the
command VIRTIO_ADMIN_CMD_RESOURCE_OBJ_CREATE when the resource \field{type}
is VIRTIO_NET_RESOURCE_OBJ_FF_GROUP, if a flow filter group already exists
with the supplied \field{group_priority}.

The device SHOULD set \field{status} to VIRTIO_ADMIN_STATUS_ENOSPC for the
command VIRTIO_ADMIN_CMD_RESOURCE_OBJ_CREATE when the resource \field{type}
is VIRTIO_NET_RESOURCE_OBJ_FF_GROUP, if the number of flow filter group
objects in the device exceeds the lower of the configured driver
capabilities \field{groups_limit} and \field{rules_per_group_limit}.

The device SHOULD set \field{status} to VIRTIO_ADMIN_STATUS_ENOSPC for the
command VIRTIO_ADMIN_CMD_RESOURCE_OBJ_CREATE when the resource \field{type} is
VIRTIO_NET_RESOURCE_OBJ_FF_CLASSIFIER, if the number of flow filter selector
objects in the device exceeds the configured driver capability
\field{selectors_limit}.

The device SHOULD set \field{status} to VIRTIO_ADMIN_STATUS_EBUSY for the
command VIRTIO_ADMIN_CMD_RESOURCE_OBJ_DESTROY for a flow filter group when
the flow filter group has one or more flow filter rules depending on it.

The device SHOULD set \field{status} to VIRTIO_ADMIN_STATUS_EBUSY for the
command VIRTIO_ADMIN_CMD_RESOURCE_OBJ_DESTROY for a flow filter classifier when
the flow filter classifier has one or more flow filter rules depending on it.

The device SHOULD fail the command VIRTIO_ADMIN_CMD_RESOURCE_OBJ_CREATE for the
flow filter rule resource object if,
\begin{itemize}
\item \field{vq_index} is not a valid receive virtqueue index for
the VIRTIO_NET_FF_ACTION_DIRECT_RX_VQ action,
\item \field{priority} is greater than or equal to
      \field{last_rule_priority},
\item \field{id} is greater than or equal to \field{rules_limit} or
      greater than or equal to \field{rules_per_group_limit}, whichever is lower,
\item the length of \field{keys} and the length of all the mask bytes of
      \field{selectors[].mask} as referred by \field{classifier_id} differs,
\item the supplied \field{action} is not supported in the capability VIRTIO_NET_FF_ACTION_CAP.
\end{itemize}

When the flow filter directs a packet to the virtqueue identified by
\field{vq_index} and if the receive virtqueue is reset, the device
MUST drop such packets.

Upon applying a flow filter rule to a packet, the device MUST STOP any further
application of rules and cease applying any other steering configurations.

For multiple flow filter groups, the device MUST apply the rules from
the group with the highest priority. If any rule from this group is applied,
the device MUST ignore the remaining groups. If none of the rules from the
highest priority group match, the device MUST apply the rules from
the group with the next highest priority, until either a rule matches or
all groups have been attempted.

The device MUST apply the rules within the group from the highest to the
lowest priority until a rule matches the packet, and the device MUST take
the action. If an action is taken, the device MUST not take any other
action for this packet.

The device MAY apply the rules with the same \field{rule_priority} in any
order within the group.

The device MUST process incoming packets in the following order:
\begin{itemize}
\item apply the steering configuration received using control virtqueue
      commands VIRTIO_NET_CTRL_RX, VIRTIO_NET_CTRL_MAC, and
      VIRTIO_NET_CTRL_VLAN.
\item apply flow filter rules if any.
\item if no filter rule is applied, apply the steering configuration
      received using the command VIRTIO_NET_CTRL_MQ_RSS_CONFIG
      or according to automatic receive steering.
\end{itemize}

When processing an incoming packet, if the packet is dropped at any stage, the device
MUST skip further processing.

When the device drops the packet due to the configuration done using the control
virtqueue commands VIRTIO_NET_CTRL_RX or VIRTIO_NET_CTRL_MAC or VIRTIO_NET_CTRL_VLAN,
the device MUST skip flow filter rules for this packet.

When the device performs flow filter match operations and if the operation
result did not have any match in all the groups, the receive packet processing
continues to next level, i.e. to apply configuration done using
VIRTIO_NET_CTRL_MQ_RSS_CONFIG command.

The device MUST support the creation of flow filter classifier objects
using the command VIRTIO_ADMIN_CMD_RESOURCE_OBJ_CREATE with \field{flags}
set to VIRTIO_NET_FF_MASK_F_PARTIAL_MASK;
this support is required even if all the bits of the masks are set for
a field in \field{selectors}, provided that partial masking is supported
for the selectors.

\drivernormative{\paragraph}{Flow filter}{Device Types / Network Device / Device Operation / Flow filter}

The driver MUST enable VIRTIO_NET_FF_RESOURCE_CAP, VIRTIO_NET_FF_SELECTOR_CAP,
and VIRTIO_NET_FF_ACTION_CAP capabilities to use flow filter.

The driver SHOULD NOT remove a flow filter group using the command
VIRTIO_ADMIN_CMD_RESOURCE_OBJ_DESTROY when one or more flow filter rules
depend on that group. The driver SHOULD only destroy the group after
all the associated rules have been destroyed.

The driver SHOULD NOT remove a flow filter classifier using the command
VIRTIO_ADMIN_CMD_RESOURCE_OBJ_DESTROY when one or more flow filter rules
depend on the classifier. The driver SHOULD only destroy the classifier
after all the associated rules have been destroyed.

The driver SHOULD NOT add multiple flow filter rules with the same
\field{rule_priority} within a flow filter group, as these rules MAY match
the same packet. The driver SHOULD assign different \field{rule_priority}
values to different flow filter rules if multiple rules may match a single
packet.

For the command VIRTIO_ADMIN_CMD_RESOURCE_OBJ_CREATE, when creating a resource
of \field{type} VIRTIO_NET_RESOURCE_OBJ_FF_CLASSIFIER, the driver MUST set:
\begin{itemize}
\item \field{selectors[0].type} to VIRTIO_NET_FF_MASK_TYPE_ETH.
\item \field{selectors[1].type} to VIRTIO_NET_FF_MASK_TYPE_IPV4 or
      VIRTIO_NET_FF_MASK_TYPE_IPV6 when \field{count} is more than 1,
\item \field{selectors[2].type} VIRTIO_NET_FF_MASK_TYPE_UDP or
      VIRTIO_NET_FF_MASK_TYPE_TCP when \field{count} is more than 2.
\end{itemize}

For the command VIRTIO_ADMIN_CMD_RESOURCE_OBJ_CREATE, when creating a resource
of \field{type} VIRTIO_NET_RESOURCE_OBJ_FF_CLASSIFIER, the driver MUST set:
\begin{itemize}
\item \field{selectors[0].mask} bytes to all 1s for the \field{EtherType}
       when \field{count} is 2 or more.
\item \field{selectors[1].mask} bytes to all 1s for \field{Protocol} or \field{Next Header}
       when \field{selector[1].type} is VIRTIO_NET_FF_MASK_TYPE_IPV4 or VIRTIO_NET_FF_MASK_TYPE_IPV6,
       and when \field{count} is more than 2.
\end{itemize}

For the command VIRTIO_ADMIN_CMD_RESOURCE_OBJ_CREATE, the resource \field{type}
VIRTIO_NET_RESOURCE_OBJ_FF_RULE, if the corresponding classifier object's
\field{count} is 2 or more, the driver MUST SET the \field{keys} bytes of
\field{EtherType} in accordance with
\hyperref[intro:IEEE 802 Ethertypes]{IEEE 802 Ethertypes}
for either VIRTIO_NET_FF_MASK_TYPE_IPV4 or VIRTIO_NET_FF_MASK_TYPE_IPV6.

For the command VIRTIO_ADMIN_CMD_RESOURCE_OBJ_CREATE, when creating a resource of
\field{type} VIRTIO_NET_RESOURCE_OBJ_FF_RULE, if the corresponding classifier
object's \field{count} is more than 2, and the \field{selector[1].type} is either
VIRTIO_NET_FF_MASK_TYPE_IPV4 or VIRTIO_NET_FF_MASK_TYPE_IPV6, the driver MUST
set the \field{keys} bytes for the \field{Protocol} or \field{Next Header}
according to \hyperref[intro:IANA Protocol Numbers]{IANA Protocol Numbers} respectively.

The driver SHOULD set all the bits for a field in the mask of a selector in both the
capability and the classifier object, unless the VIRTIO_NET_FF_MASK_F_PARTIAL_MASK
is enabled.

\subsubsection{Legacy Interface: Framing Requirements}\label{sec:Device
Types / Network Device / Legacy Interface: Framing Requirements}

When using legacy interfaces, transitional drivers which have not
negotiated VIRTIO_F_ANY_LAYOUT MUST use a single descriptor for the
\field{struct virtio_net_hdr} on both transmit and receive, with the
network data in the following descriptors.

Additionally, when using the control virtqueue (see \ref{sec:Device
Types / Network Device / Device Operation / Control Virtqueue})
, transitional drivers which have not
negotiated VIRTIO_F_ANY_LAYOUT MUST:
\begin{itemize}
\item for all commands, use a single 2-byte descriptor including the first two
fields: \field{class} and \field{command}
\item for all commands except VIRTIO_NET_CTRL_MAC_TABLE_SET
use a single descriptor including command-specific-data
with no padding.
\item for the VIRTIO_NET_CTRL_MAC_TABLE_SET command use exactly
two descriptors including command-specific-data with no padding:
the first of these descriptors MUST include the
virtio_net_ctrl_mac table structure for the unicast addresses with no padding,
the second of these descriptors MUST include the
virtio_net_ctrl_mac table structure for the multicast addresses
with no padding.
\item for all commands, use a single 1-byte descriptor for the
\field{ack} field
\end{itemize}

See \ref{sec:Basic
Facilities of a Virtio Device / Virtqueues / Message Framing}.

\section{Network Device}\label{sec:Device Types / Network Device}

The virtio network device is a virtual network interface controller.
It consists of a virtual Ethernet link which connects the device
to the Ethernet network. The device has transmit and receive
queues. The driver adds empty buffers to the receive virtqueue.
The device receives incoming packets from the link; the device
places these incoming packets in the receive virtqueue buffers.
The driver adds outgoing packets to the transmit virtqueue. The device
removes these packets from the transmit virtqueue and sends them to
the link. The device may have a control virtqueue. The driver
uses the control virtqueue to dynamically manipulate various
features of the initialized device.

\subsection{Device ID}\label{sec:Device Types / Network Device / Device ID}

 1

\subsection{Virtqueues}\label{sec:Device Types / Network Device / Virtqueues}

\begin{description}
\item[0] receiveq1
\item[1] transmitq1
\item[\ldots]
\item[2(N-1)] receiveqN
\item[2(N-1)+1] transmitqN
\item[2N] controlq
\end{description}

 N=1 if neither VIRTIO_NET_F_MQ nor VIRTIO_NET_F_RSS are negotiated, otherwise N is set by
 \field{max_virtqueue_pairs}.

controlq is optional; it only exists if VIRTIO_NET_F_CTRL_VQ is
negotiated.

\subsection{Feature bits}\label{sec:Device Types / Network Device / Feature bits}

\begin{description}
\item[VIRTIO_NET_F_CSUM (0)] Device handles packets with partial checksum offload.

\item[VIRTIO_NET_F_GUEST_CSUM (1)] Driver handles packets with partial checksum.

\item[VIRTIO_NET_F_CTRL_GUEST_OFFLOADS (2)] Control channel offloads
        reconfiguration support.

\item[VIRTIO_NET_F_MTU(3)] Device maximum MTU reporting is supported. If
    offered by the device, device advises driver about the value of
    its maximum MTU. If negotiated, the driver uses \field{mtu} as
    the maximum MTU value.

\item[VIRTIO_NET_F_MAC (5)] Device has given MAC address.

\item[VIRTIO_NET_F_GUEST_TSO4 (7)] Driver can receive TSOv4.

\item[VIRTIO_NET_F_GUEST_TSO6 (8)] Driver can receive TSOv6.

\item[VIRTIO_NET_F_GUEST_ECN (9)] Driver can receive TSO with ECN.

\item[VIRTIO_NET_F_GUEST_UFO (10)] Driver can receive UFO.

\item[VIRTIO_NET_F_HOST_TSO4 (11)] Device can receive TSOv4.

\item[VIRTIO_NET_F_HOST_TSO6 (12)] Device can receive TSOv6.

\item[VIRTIO_NET_F_HOST_ECN (13)] Device can receive TSO with ECN.

\item[VIRTIO_NET_F_HOST_UFO (14)] Device can receive UFO.

\item[VIRTIO_NET_F_MRG_RXBUF (15)] Driver can merge receive buffers.

\item[VIRTIO_NET_F_STATUS (16)] Configuration status field is
    available.

\item[VIRTIO_NET_F_CTRL_VQ (17)] Control channel is available.

\item[VIRTIO_NET_F_CTRL_RX (18)] Control channel RX mode support.

\item[VIRTIO_NET_F_CTRL_VLAN (19)] Control channel VLAN filtering.

\item[VIRTIO_NET_F_CTRL_RX_EXTRA (20)]	Control channel RX extra mode support.

\item[VIRTIO_NET_F_GUEST_ANNOUNCE(21)] Driver can send gratuitous
    packets.

\item[VIRTIO_NET_F_MQ(22)] Device supports multiqueue with automatic
    receive steering.

\item[VIRTIO_NET_F_CTRL_MAC_ADDR(23)] Set MAC address through control
    channel.

\item[VIRTIO_NET_F_DEVICE_STATS(50)] Device can provide device-level statistics
    to the driver through the control virtqueue.

\item[VIRTIO_NET_F_HASH_TUNNEL(51)] Device supports inner header hash for encapsulated packets.

\item[VIRTIO_NET_F_VQ_NOTF_COAL(52)] Device supports virtqueue notification coalescing.

\item[VIRTIO_NET_F_NOTF_COAL(53)] Device supports notifications coalescing.

\item[VIRTIO_NET_F_GUEST_USO4 (54)] Driver can receive USOv4 packets.

\item[VIRTIO_NET_F_GUEST_USO6 (55)] Driver can receive USOv6 packets.

\item[VIRTIO_NET_F_HOST_USO (56)] Device can receive USO packets. Unlike UFO
 (fragmenting the packet) the USO splits large UDP packet
 to several segments when each of these smaller packets has UDP header.

\item[VIRTIO_NET_F_HASH_REPORT(57)] Device can report per-packet hash
    value and a type of calculated hash.

\item[VIRTIO_NET_F_GUEST_HDRLEN(59)] Driver can provide the exact \field{hdr_len}
    value. Device benefits from knowing the exact header length.

\item[VIRTIO_NET_F_RSS(60)] Device supports RSS (receive-side scaling)
    with Toeplitz hash calculation and configurable hash
    parameters for receive steering.

\item[VIRTIO_NET_F_RSC_EXT(61)] Device can process duplicated ACKs
    and report number of coalesced segments and duplicated ACKs.

\item[VIRTIO_NET_F_STANDBY(62)] Device may act as a standby for a primary
    device with the same MAC address.

\item[VIRTIO_NET_F_SPEED_DUPLEX(63)] Device reports speed and duplex.

\item[VIRTIO_NET_F_RSS_CONTEXT(64)] Device supports multiple RSS contexts.

\item[VIRTIO_NET_F_GUEST_UDP_TUNNEL_GSO (65)] Driver can receive GSO packets
  carried by a UDP tunnel.

\item[VIRTIO_NET_F_GUEST_UDP_TUNNEL_GSO_CSUM (66)] Driver handles packets
  carried by a UDP tunnel with partial csum for the outer header.

\item[VIRTIO_NET_F_HOST_UDP_TUNNEL_GSO (67)] Device can receive GSO packets
  carried by a UDP tunnel.

\item[VIRTIO_NET_F_HOST_UDP_TUNNEL_GSO_CSUM (68)] Device handles packets
  carried by a UDP tunnel with partial csum for the outer header.
\end{description}

\subsubsection{Feature bit requirements}\label{sec:Device Types / Network Device / Feature bits / Feature bit requirements}

Some networking feature bits require other networking feature bits
(see \ref{drivernormative:Basic Facilities of a Virtio Device / Feature Bits}):

\begin{description}
\item[VIRTIO_NET_F_GUEST_TSO4] Requires VIRTIO_NET_F_GUEST_CSUM.
\item[VIRTIO_NET_F_GUEST_TSO6] Requires VIRTIO_NET_F_GUEST_CSUM.
\item[VIRTIO_NET_F_GUEST_ECN] Requires VIRTIO_NET_F_GUEST_TSO4 or VIRTIO_NET_F_GUEST_TSO6.
\item[VIRTIO_NET_F_GUEST_UFO] Requires VIRTIO_NET_F_GUEST_CSUM.
\item[VIRTIO_NET_F_GUEST_USO4] Requires VIRTIO_NET_F_GUEST_CSUM.
\item[VIRTIO_NET_F_GUEST_USO6] Requires VIRTIO_NET_F_GUEST_CSUM.
\item[VIRTIO_NET_F_GUEST_UDP_TUNNEL_GSO] Requires VIRTIO_NET_F_GUEST_TSO4, VIRTIO_NET_F_GUEST_TSO6,
   VIRTIO_NET_F_GUEST_USO4 and VIRTIO_NET_F_GUEST_USO6.
\item[VIRTIO_NET_F_GUEST_UDP_TUNNEL_GSO_CSUM] Requires VIRTIO_NET_F_GUEST_UDP_TUNNEL_GSO

\item[VIRTIO_NET_F_HOST_TSO4] Requires VIRTIO_NET_F_CSUM.
\item[VIRTIO_NET_F_HOST_TSO6] Requires VIRTIO_NET_F_CSUM.
\item[VIRTIO_NET_F_HOST_ECN] Requires VIRTIO_NET_F_HOST_TSO4 or VIRTIO_NET_F_HOST_TSO6.
\item[VIRTIO_NET_F_HOST_UFO] Requires VIRTIO_NET_F_CSUM.
\item[VIRTIO_NET_F_HOST_USO] Requires VIRTIO_NET_F_CSUM.
\item[VIRTIO_NET_F_HOST_UDP_TUNNEL_GSO] Requires VIRTIO_NET_F_HOST_TSO4, VIRTIO_NET_F_HOST_TSO6
   and VIRTIO_NET_F_HOST_USO.
\item[VIRTIO_NET_F_HOST_UDP_TUNNEL_GSO_CSUM] Requires VIRTIO_NET_F_HOST_UDP_TUNNEL_GSO

\item[VIRTIO_NET_F_CTRL_RX] Requires VIRTIO_NET_F_CTRL_VQ.
\item[VIRTIO_NET_F_CTRL_VLAN] Requires VIRTIO_NET_F_CTRL_VQ.
\item[VIRTIO_NET_F_GUEST_ANNOUNCE] Requires VIRTIO_NET_F_CTRL_VQ.
\item[VIRTIO_NET_F_MQ] Requires VIRTIO_NET_F_CTRL_VQ.
\item[VIRTIO_NET_F_CTRL_MAC_ADDR] Requires VIRTIO_NET_F_CTRL_VQ.
\item[VIRTIO_NET_F_NOTF_COAL] Requires VIRTIO_NET_F_CTRL_VQ.
\item[VIRTIO_NET_F_RSC_EXT] Requires VIRTIO_NET_F_HOST_TSO4 or VIRTIO_NET_F_HOST_TSO6.
\item[VIRTIO_NET_F_RSS] Requires VIRTIO_NET_F_CTRL_VQ.
\item[VIRTIO_NET_F_VQ_NOTF_COAL] Requires VIRTIO_NET_F_CTRL_VQ.
\item[VIRTIO_NET_F_HASH_TUNNEL] Requires VIRTIO_NET_F_CTRL_VQ along with VIRTIO_NET_F_RSS or VIRTIO_NET_F_HASH_REPORT.
\item[VIRTIO_NET_F_RSS_CONTEXT] Requires VIRTIO_NET_F_CTRL_VQ and VIRTIO_NET_F_RSS.
\end{description}

\begin{note}
The dependency between UDP_TUNNEL_GSO_CSUM and UDP_TUNNEL_GSO is intentionally
in the opposite direction with respect to the plain GSO features and the plain
checksum offload because UDP tunnel checksum offload gives very little gain
for non GSO packets and is quite complex to implement in H/W.
\end{note}

\subsubsection{Legacy Interface: Feature bits}\label{sec:Device Types / Network Device / Feature bits / Legacy Interface: Feature bits}
\begin{description}
\item[VIRTIO_NET_F_GSO (6)] Device handles packets with any GSO type. This was supposed to indicate segmentation offload support, but
upon further investigation it became clear that multiple bits were needed.
\item[VIRTIO_NET_F_GUEST_RSC4 (41)] Device coalesces TCPIP v4 packets. This was implemented by hypervisor patch for certification
purposes and current Windows driver depends on it. It will not function if virtio-net device reports this feature.
\item[VIRTIO_NET_F_GUEST_RSC6 (42)] Device coalesces TCPIP v6 packets. Similar to VIRTIO_NET_F_GUEST_RSC4.
\end{description}

\subsection{Device configuration layout}\label{sec:Device Types / Network Device / Device configuration layout}
\label{sec:Device Types / Block Device / Feature bits / Device configuration layout}

The network device has the following device configuration layout.
All of the device configuration fields are read-only for the driver.

\begin{lstlisting}
struct virtio_net_config {
        u8 mac[6];
        le16 status;
        le16 max_virtqueue_pairs;
        le16 mtu;
        le32 speed;
        u8 duplex;
        u8 rss_max_key_size;
        le16 rss_max_indirection_table_length;
        le32 supported_hash_types;
        le32 supported_tunnel_types;
};
\end{lstlisting}

The \field{mac} address field always exists (although it is only
valid if VIRTIO_NET_F_MAC is set).

The \field{status} only exists if VIRTIO_NET_F_STATUS is set.
Two bits are currently defined for the status field: VIRTIO_NET_S_LINK_UP
and VIRTIO_NET_S_ANNOUNCE.

\begin{lstlisting}
#define VIRTIO_NET_S_LINK_UP     1
#define VIRTIO_NET_S_ANNOUNCE    2
\end{lstlisting}

The following field, \field{max_virtqueue_pairs} only exists if
VIRTIO_NET_F_MQ or VIRTIO_NET_F_RSS is set. This field specifies the maximum number
of each of transmit and receive virtqueues (receiveq1\ldots receiveqN
and transmitq1\ldots transmitqN respectively) that can be configured once at least one of these features
is negotiated.

The following field, \field{mtu} only exists if VIRTIO_NET_F_MTU
is set. This field specifies the maximum MTU for the driver to
use.

The following two fields, \field{speed} and \field{duplex}, only
exist if VIRTIO_NET_F_SPEED_DUPLEX is set.

\field{speed} contains the device speed, in units of 1 MBit per
second, 0 to 0x7fffffff, or 0xffffffff for unknown speed.

\field{duplex} has the values of 0x01 for full duplex, 0x00 for
half duplex and 0xff for unknown duplex state.

Both \field{speed} and \field{duplex} can change, thus the driver
is expected to re-read these values after receiving a
configuration change notification.

The following field, \field{rss_max_key_size} only exists if VIRTIO_NET_F_RSS or VIRTIO_NET_F_HASH_REPORT is set.
It specifies the maximum supported length of RSS key in bytes.

The following field, \field{rss_max_indirection_table_length} only exists if VIRTIO_NET_F_RSS is set.
It specifies the maximum number of 16-bit entries in RSS indirection table.

The next field, \field{supported_hash_types} only exists if the device supports hash calculation,
i.e. if VIRTIO_NET_F_RSS or VIRTIO_NET_F_HASH_REPORT is set.

Field \field{supported_hash_types} contains the bitmask of supported hash types.
See \ref{sec:Device Types / Network Device / Device Operation / Processing of Incoming Packets / Hash calculation for incoming packets / Supported/enabled hash types} for details of supported hash types.

Field \field{supported_tunnel_types} only exists if the device supports inner header hash, i.e. if VIRTIO_NET_F_HASH_TUNNEL is set.

Field \field{supported_tunnel_types} contains the bitmask of encapsulation types supported by the device for inner header hash.
Encapsulation types are defined in \ref{sec:Device Types / Network Device / Device Operation / Processing of Incoming Packets /
Hash calculation for incoming packets / Encapsulation types supported/enabled for inner header hash}.

\devicenormative{\subsubsection}{Device configuration layout}{Device Types / Network Device / Device configuration layout}

The device MUST set \field{max_virtqueue_pairs} to between 1 and 0x8000 inclusive,
if it offers VIRTIO_NET_F_MQ.

The device MUST set \field{mtu} to between 68 and 65535 inclusive,
if it offers VIRTIO_NET_F_MTU.

The device SHOULD set \field{mtu} to at least 1280, if it offers
VIRTIO_NET_F_MTU.

The device MUST NOT modify \field{mtu} once it has been set.

The device MUST NOT pass received packets that exceed \field{mtu} (plus low
level ethernet header length) size with \field{gso_type} NONE or ECN
after VIRTIO_NET_F_MTU has been successfully negotiated.

The device MUST forward transmitted packets of up to \field{mtu} (plus low
level ethernet header length) size with \field{gso_type} NONE or ECN, and do
so without fragmentation, after VIRTIO_NET_F_MTU has been successfully
negotiated.

The device MUST set \field{rss_max_key_size} to at least 40, if it offers
VIRTIO_NET_F_RSS or VIRTIO_NET_F_HASH_REPORT.

The device MUST set \field{rss_max_indirection_table_length} to at least 128, if it offers
VIRTIO_NET_F_RSS.

If the driver negotiates the VIRTIO_NET_F_STANDBY feature, the device MAY act
as a standby device for a primary device with the same MAC address.

If VIRTIO_NET_F_SPEED_DUPLEX has been negotiated, \field{speed}
MUST contain the device speed, in units of 1 MBit per second, 0 to
0x7ffffffff, or 0xfffffffff for unknown.

If VIRTIO_NET_F_SPEED_DUPLEX has been negotiated, \field{duplex}
MUST have the values of 0x00 for full duplex, 0x01 for half
duplex, or 0xff for unknown.

If VIRTIO_NET_F_SPEED_DUPLEX and VIRTIO_NET_F_STATUS have both
been negotiated, the device SHOULD NOT change the \field{speed} and
\field{duplex} fields as long as VIRTIO_NET_S_LINK_UP is set in
the \field{status}.

The device SHOULD NOT offer VIRTIO_NET_F_HASH_REPORT if it
does not offer VIRTIO_NET_F_CTRL_VQ.

The device SHOULD NOT offer VIRTIO_NET_F_CTRL_RX_EXTRA if it
does not offer VIRTIO_NET_F_CTRL_VQ.

\drivernormative{\subsubsection}{Device configuration layout}{Device Types / Network Device / Device configuration layout}

The driver MUST NOT write to any of the device configuration fields.

A driver SHOULD negotiate VIRTIO_NET_F_MAC if the device offers it.
If the driver negotiates the VIRTIO_NET_F_MAC feature, the driver MUST set
the physical address of the NIC to \field{mac}.  Otherwise, it SHOULD
use a locally-administered MAC address (see \hyperref[intro:IEEE 802]{IEEE 802},
``9.2 48-bit universal LAN MAC addresses'').

If the driver does not negotiate the VIRTIO_NET_F_STATUS feature, it SHOULD
assume the link is active, otherwise it SHOULD read the link status from
the bottom bit of \field{status}.

A driver SHOULD negotiate VIRTIO_NET_F_MTU if the device offers it.

If the driver negotiates VIRTIO_NET_F_MTU, it MUST supply enough receive
buffers to receive at least one receive packet of size \field{mtu} (plus low
level ethernet header length) with \field{gso_type} NONE or ECN.

If the driver negotiates VIRTIO_NET_F_MTU, it MUST NOT transmit packets of
size exceeding the value of \field{mtu} (plus low level ethernet header length)
with \field{gso_type} NONE or ECN.

A driver SHOULD negotiate the VIRTIO_NET_F_STANDBY feature if the device offers it.

If VIRTIO_NET_F_SPEED_DUPLEX has been negotiated,
the driver MUST treat any value of \field{speed} above
0x7fffffff as well as any value of \field{duplex} not
matching 0x00 or 0x01 as an unknown value.

If VIRTIO_NET_F_SPEED_DUPLEX has been negotiated, the driver
SHOULD re-read \field{speed} and \field{duplex} after a
configuration change notification.

A driver SHOULD NOT negotiate VIRTIO_NET_F_HASH_REPORT if it
does not negotiate VIRTIO_NET_F_CTRL_VQ.

A driver SHOULD NOT negotiate VIRTIO_NET_F_CTRL_RX_EXTRA if it
does not negotiate VIRTIO_NET_F_CTRL_VQ.

\subsubsection{Legacy Interface: Device configuration layout}\label{sec:Device Types / Network Device / Device configuration layout / Legacy Interface: Device configuration layout}
\label{sec:Device Types / Block Device / Feature bits / Device configuration layout / Legacy Interface: Device configuration layout}
When using the legacy interface, transitional devices and drivers
MUST format \field{status} and
\field{max_virtqueue_pairs} in struct virtio_net_config
according to the native endian of the guest rather than
(necessarily when not using the legacy interface) little-endian.

When using the legacy interface, \field{mac} is driver-writable
which provided a way for drivers to update the MAC without
negotiating VIRTIO_NET_F_CTRL_MAC_ADDR.

\subsection{Device Initialization}\label{sec:Device Types / Network Device / Device Initialization}

A driver would perform a typical initialization routine like so:

\begin{enumerate}
\item Identify and initialize the receive and
  transmission virtqueues, up to N of each kind. If
  VIRTIO_NET_F_MQ feature bit is negotiated,
  N=\field{max_virtqueue_pairs}, otherwise identify N=1.

\item If the VIRTIO_NET_F_CTRL_VQ feature bit is negotiated,
  identify the control virtqueue.

\item Fill the receive queues with buffers: see \ref{sec:Device Types / Network Device / Device Operation / Setting Up Receive Buffers}.

\item Even with VIRTIO_NET_F_MQ, only receiveq1, transmitq1 and
  controlq are used by default.  The driver would send the
  VIRTIO_NET_CTRL_MQ_VQ_PAIRS_SET command specifying the
  number of the transmit and receive queues to use.

\item If the VIRTIO_NET_F_MAC feature bit is set, the configuration
  space \field{mac} entry indicates the ``physical'' address of the
  device, otherwise the driver would typically generate a random
  local MAC address.

\item If the VIRTIO_NET_F_STATUS feature bit is negotiated, the link
  status comes from the bottom bit of \field{status}.
  Otherwise, the driver assumes it's active.

\item A performant driver would indicate that it will generate checksumless
  packets by negotiating the VIRTIO_NET_F_CSUM feature.

\item If that feature is negotiated, a driver can use TCP segmentation or UDP
  segmentation/fragmentation offload by negotiating the VIRTIO_NET_F_HOST_TSO4 (IPv4
  TCP), VIRTIO_NET_F_HOST_TSO6 (IPv6 TCP), VIRTIO_NET_F_HOST_UFO
  (UDP fragmentation) and VIRTIO_NET_F_HOST_USO (UDP segmentation) features.

\item If the VIRTIO_NET_F_HOST_TSO6, VIRTIO_NET_F_HOST_TSO4 and VIRTIO_NET_F_HOST_USO
  segmentation features are negotiated, a driver can
  use TCP segmentation or UDP segmentation on top of UDP encapsulation
  offload, when the outer header does not require checksumming - e.g.
  the outer UDP checksum is zero - by negotiating the
  VIRTIO_NET_F_HOST_UDP_TUNNEL_GSO feature.
  GSO over UDP tunnels packets carry two sets of headers: the outer ones
  and the inner ones. The outer transport protocol is UDP, the inner
  could be either TCP or UDP. Only a single level of encapsulation
  offload is supported.

\item If VIRTIO_NET_F_HOST_UDP_TUNNEL_GSO is negotiated, a driver can
  additionally use TCP segmentation or UDP segmentation on top of UDP
  encapsulation with the outer header requiring checksum offload,
  negotiating the VIRTIO_NET_F_HOST_UDP_TUNNEL_GSO_CSUM feature.

\item The converse features are also available: a driver can save
  the virtual device some work by negotiating these features.\note{For example, a network packet transported between two guests on
the same system might not need checksumming at all, nor segmentation,
if both guests are amenable.}
   The VIRTIO_NET_F_GUEST_CSUM feature indicates that partially
  checksummed packets can be received, and if it can do that then
  the VIRTIO_NET_F_GUEST_TSO4, VIRTIO_NET_F_GUEST_TSO6,
  VIRTIO_NET_F_GUEST_UFO, VIRTIO_NET_F_GUEST_ECN, VIRTIO_NET_F_GUEST_USO4,
  VIRTIO_NET_F_GUEST_USO6 VIRTIO_NET_F_GUEST_UDP_TUNNEL_GSO and
  VIRTIO_NET_F_GUEST_UDP_TUNNEL_GSO_CSUM are the input equivalents of
  the features described above.
  See \ref{sec:Device Types / Network Device / Device Operation /
Setting Up Receive Buffers}~\nameref{sec:Device Types / Network
Device / Device Operation / Setting Up Receive Buffers} and
\ref{sec:Device Types / Network Device / Device Operation /
Processing of Incoming Packets}~\nameref{sec:Device Types /
Network Device / Device Operation / Processing of Incoming Packets} below.
\end{enumerate}

A truly minimal driver would only accept VIRTIO_NET_F_MAC and ignore
everything else.

\subsection{Device and driver capabilities}\label{sec:Device Types / Network Device / Device and driver capabilities}

The network device has the following capabilities.

\begin{tabularx}{\textwidth}{ |l||l|X| }
\hline
Identifier & Name & Description \\
\hline \hline
0x0800 & \hyperref[par:Device Types / Network Device / Device Operation / Flow filter / Device and driver capabilities / VIRTIO-NET-FF-RESOURCE-CAP]{VIRTIO_NET_FF_RESOURCE_CAP} & Flow filter resource capability \\
\hline
0x0801 & \hyperref[par:Device Types / Network Device / Device Operation / Flow filter / Device and driver capabilities / VIRTIO-NET-FF-SELECTOR-CAP]{VIRTIO_NET_FF_SELECTOR_CAP} & Flow filter classifier capability \\
\hline
0x0802 & \hyperref[par:Device Types / Network Device / Device Operation / Flow filter / Device and driver capabilities / VIRTIO-NET-FF-ACTION-CAP]{VIRTIO_NET_FF_ACTION_CAP} & Flow filter action capability \\
\hline
\end{tabularx}

\subsection{Device resource objects}\label{sec:Device Types / Network Device / Device resource objects}

The network device has the following resource objects.

\begin{tabularx}{\textwidth}{ |l||l|X| }
\hline
type & Name & Description \\
\hline \hline
0x0200 & \hyperref[par:Device Types / Network Device / Device Operation / Flow filter / Resource objects / VIRTIO-NET-RESOURCE-OBJ-FF-GROUP]{VIRTIO_NET_RESOURCE_OBJ_FF_GROUP} & Flow filter group resource object \\
\hline
0x0201 & \hyperref[par:Device Types / Network Device / Device Operation / Flow filter / Resource objects / VIRTIO-NET-RESOURCE-OBJ-FF-CLASSIFIER]{VIRTIO_NET_RESOURCE_OBJ_FF_CLASSIFIER} & Flow filter mask object \\
\hline
0x0202 & \hyperref[par:Device Types / Network Device / Device Operation / Flow filter / Resource objects / VIRTIO-NET-RESOURCE-OBJ-FF-RULE]{VIRTIO_NET_RESOURCE_OBJ_FF_RULE} & Flow filter rule object \\
\hline
\end{tabularx}

\subsection{Device parts}\label{sec:Device Types / Network Device / Device parts}

Network device parts represent the configuration done by the driver using control
virtqueue commands. Network device part is in the format of
\field{struct virtio_dev_part}.

\begin{tabularx}{\textwidth}{ |l||l|X| }
\hline
Type & Name & Description \\
\hline \hline
0x200 & VIRTIO_NET_DEV_PART_CVQ_CFG_PART & Represents device configuration done through a control virtqueue command, see \ref{sec:Device Types / Network Device / Device parts / VIRTIO-NET-DEV-PART-CVQ-CFG-PART} \\
\hline
0x201 - 0x5FF & - & reserved for future \\
\hline
\hline
\end{tabularx}

\subsubsection{VIRTIO_NET_DEV_PART_CVQ_CFG_PART}\label{sec:Device Types / Network Device / Device parts / VIRTIO-NET-DEV-PART-CVQ-CFG-PART}

For VIRTIO_NET_DEV_PART_CVQ_CFG_PART, \field{part_type} is set to 0x200. The
VIRTIO_NET_DEV_PART_CVQ_CFG_PART part indicates configuration performed by the
driver using a control virtqueue command.

\begin{lstlisting}
struct virtio_net_dev_part_cvq_selector {
        u8 class;
        u8 command;
        u8 reserved[6];
};
\end{lstlisting}

There is one device part of type VIRTIO_NET_DEV_PART_CVQ_CFG_PART for each
individual configuration. Each part is identified by a unique selector value.
The selector, \field{device_type_raw}, is in the format
\field{struct virtio_net_dev_part_cvq_selector}.

The selector consists of two fields: \field{class} and \field{command}. These
fields correspond to the \field{class} and \field{command} defined in
\field{struct virtio_net_ctrl}, as described in the relevant sections of
\ref{sec:Device Types / Network Device / Device Operation / Control Virtqueue}.

The value corresponding to each part’s selector follows the same format as the
respective \field{command-specific-data} described in the relevant sections of
\ref{sec:Device Types / Network Device / Device Operation / Control Virtqueue}.

For example, when the \field{class} is VIRTIO_NET_CTRL_MAC, the \field{command}
can be either VIRTIO_NET_CTRL_MAC_TABLE_SET or VIRTIO_NET_CTRL_MAC_ADDR_SET;
when \field{command} is set to VIRTIO_NET_CTRL_MAC_TABLE_SET, \field{value}
is in the format of \field{struct virtio_net_ctrl_mac}.

Supported selectors are listed in the table:

\begin{tabularx}{\textwidth}{ |l|X| }
\hline
Class selector & Command selector \\
\hline \hline
VIRTIO_NET_CTRL_RX & VIRTIO_NET_CTRL_RX_PROMISC \\
\hline
VIRTIO_NET_CTRL_RX & VIRTIO_NET_CTRL_RX_ALLMULTI \\
\hline
VIRTIO_NET_CTRL_RX & VIRTIO_NET_CTRL_RX_ALLUNI \\
\hline
VIRTIO_NET_CTRL_RX & VIRTIO_NET_CTRL_RX_NOMULTI \\
\hline
VIRTIO_NET_CTRL_RX & VIRTIO_NET_CTRL_RX_NOUNI \\
\hline
VIRTIO_NET_CTRL_RX & VIRTIO_NET_CTRL_RX_NOBCAST \\
\hline
VIRTIO_NET_CTRL_MAC & VIRTIO_NET_CTRL_MAC_TABLE_SET \\
\hline
VIRTIO_NET_CTRL_MAC & VIRTIO_NET_CTRL_MAC_ADDR_SET \\
\hline
VIRTIO_NET_CTRL_VLAN & VIRTIO_NET_CTRL_VLAN_ADD \\
\hline
VIRTIO_NET_CTRL_ANNOUNCE & VIRTIO_NET_CTRL_ANNOUNCE_ACK \\
\hline
VIRTIO_NET_CTRL_MQ & VIRTIO_NET_CTRL_MQ_VQ_PAIRS_SET \\
\hline
VIRTIO_NET_CTRL_MQ & VIRTIO_NET_CTRL_MQ_RSS_CONFIG \\
\hline
VIRTIO_NET_CTRL_MQ & VIRTIO_NET_CTRL_MQ_HASH_CONFIG \\
\hline
\hline
\end{tabularx}

For command selector VIRTIO_NET_CTRL_VLAN_ADD, device part consists of a whole
VLAN table.

\field{reserved} is reserved and set to zero.

\subsection{Device Operation}\label{sec:Device Types / Network Device / Device Operation}

Packets are transmitted by placing them in the
transmitq1\ldots transmitqN, and buffers for incoming packets are
placed in the receiveq1\ldots receiveqN. In each case, the packet
itself is preceded by a header:

\begin{lstlisting}
struct virtio_net_hdr {
#define VIRTIO_NET_HDR_F_NEEDS_CSUM    1
#define VIRTIO_NET_HDR_F_DATA_VALID    2
#define VIRTIO_NET_HDR_F_RSC_INFO      4
#define VIRTIO_NET_HDR_F_UDP_TUNNEL_CSUM 8
        u8 flags;
#define VIRTIO_NET_HDR_GSO_NONE        0
#define VIRTIO_NET_HDR_GSO_TCPV4       1
#define VIRTIO_NET_HDR_GSO_UDP         3
#define VIRTIO_NET_HDR_GSO_TCPV6       4
#define VIRTIO_NET_HDR_GSO_UDP_L4      5
#define VIRTIO_NET_HDR_GSO_UDP_TUNNEL_IPV4 0x20
#define VIRTIO_NET_HDR_GSO_UDP_TUNNEL_IPV6 0x40
#define VIRTIO_NET_HDR_GSO_ECN      0x80
        u8 gso_type;
        le16 hdr_len;
        le16 gso_size;
        le16 csum_start;
        le16 csum_offset;
        le16 num_buffers;
        le32 hash_value;        (Only if VIRTIO_NET_F_HASH_REPORT negotiated)
        le16 hash_report;       (Only if VIRTIO_NET_F_HASH_REPORT negotiated)
        le16 padding_reserved;  (Only if VIRTIO_NET_F_HASH_REPORT negotiated)
        le16 outer_th_offset    (Only if VIRTIO_NET_F_HOST_UDP_TUNNEL_GSO or VIRTIO_NET_F_GUEST_UDP_TUNNEL_GSO negotiated)
        le16 inner_nh_offset;   (Only if VIRTIO_NET_F_HOST_UDP_TUNNEL_GSO or VIRTIO_NET_F_GUEST_UDP_TUNNEL_GSO negotiated)
};
\end{lstlisting}

The controlq is used to control various device features described further in
section \ref{sec:Device Types / Network Device / Device Operation / Control Virtqueue}.

\subsubsection{Legacy Interface: Device Operation}\label{sec:Device Types / Network Device / Device Operation / Legacy Interface: Device Operation}
When using the legacy interface, transitional devices and drivers
MUST format the fields in \field{struct virtio_net_hdr}
according to the native endian of the guest rather than
(necessarily when not using the legacy interface) little-endian.

The legacy driver only presented \field{num_buffers} in the \field{struct virtio_net_hdr}
when VIRTIO_NET_F_MRG_RXBUF was negotiated; without that feature the
structure was 2 bytes shorter.

When using the legacy interface, the driver SHOULD ignore the
used length for the transmit queues
and the controlq queue.
\begin{note}
Historically, some devices put
the total descriptor length there, even though no data was
actually written.
\end{note}

\subsubsection{Packet Transmission}\label{sec:Device Types / Network Device / Device Operation / Packet Transmission}

Transmitting a single packet is simple, but varies depending on
the different features the driver negotiated.

\begin{enumerate}
\item The driver can send a completely checksummed packet.  In this case,
  \field{flags} will be zero, and \field{gso_type} will be VIRTIO_NET_HDR_GSO_NONE.

\item If the driver negotiated VIRTIO_NET_F_CSUM, it can skip
  checksumming the packet:
  \begin{itemize}
  \item \field{flags} has the VIRTIO_NET_HDR_F_NEEDS_CSUM set,

  \item \field{csum_start} is set to the offset within the packet to begin checksumming,
    and

  \item \field{csum_offset} indicates how many bytes after the csum_start the
    new (16 bit ones' complement) checksum is placed by the device.

  \item The TCP checksum field in the packet is set to the sum
    of the TCP pseudo header, so that replacing it by the ones'
    complement checksum of the TCP header and body will give the
    correct result.
  \end{itemize}

\begin{note}
For example, consider a partially checksummed TCP (IPv4) packet.
It will have a 14 byte ethernet header and 20 byte IP header
followed by the TCP header (with the TCP checksum field 16 bytes
into that header). \field{csum_start} will be 14+20 = 34 (the TCP
checksum includes the header), and \field{csum_offset} will be 16.
If the given packet has the VIRTIO_NET_HDR_GSO_UDP_TUNNEL_IPV4 bit or the
VIRTIO_NET_HDR_GSO_UDP_TUNNEL_IPV6 bit set,
the above checksum fields refer to the inner header checksum, see
the example below.
\end{note}

\item If the driver negotiated
  VIRTIO_NET_F_HOST_TSO4, TSO6, USO or UFO, and the packet requires
  TCP segmentation, UDP segmentation or fragmentation, then \field{gso_type}
  is set to VIRTIO_NET_HDR_GSO_TCPV4, TCPV6, UDP_L4 or UDP.
  (Otherwise, it is set to VIRTIO_NET_HDR_GSO_NONE). In this
  case, packets larger than 1514 bytes can be transmitted: the
  metadata indicates how to replicate the packet header to cut it
  into smaller packets. The other gso fields are set:

  \begin{itemize}
  \item If the VIRTIO_NET_F_GUEST_HDRLEN feature has been negotiated,
    \field{hdr_len} indicates the header length that needs to be replicated
    for each packet. It's the number of bytes from the beginning of the packet
    to the beginning of the transport payload.
    If the \field{gso_type} has the VIRTIO_NET_HDR_GSO_UDP_TUNNEL_IPV4 bit or
    VIRTIO_NET_HDR_GSO_UDP_TUNNEL_IPV6 bit set, \field{hdr_len} accounts for
    all the headers up to and including the inner transport.
    Otherwise, if the VIRTIO_NET_F_GUEST_HDRLEN feature has not been negotiated,
    \field{hdr_len} is a hint to the device as to how much of the header
    needs to be kept to copy into each packet, usually set to the
    length of the headers, including the transport header\footnote{Due to various bugs in implementations, this field is not useful
as a guarantee of the transport header size.
}.

  \begin{note}
  Some devices benefit from knowledge of the exact header length.
  \end{note}

  \item \field{gso_size} is the maximum size of each packet beyond that
    header (ie. MSS).

  \item If the driver negotiated the VIRTIO_NET_F_HOST_ECN feature,
    the VIRTIO_NET_HDR_GSO_ECN bit in \field{gso_type}
    indicates that the TCP packet has the ECN bit set\footnote{This case is not handled by some older hardware, so is called out
specifically in the protocol.}.
   \end{itemize}

\item If the driver negotiated the VIRTIO_NET_F_HOST_UDP_TUNNEL_GSO feature and the
  \field{gso_type} has the VIRTIO_NET_HDR_GSO_UDP_TUNNEL_IPV4 bit or
  VIRTIO_NET_HDR_GSO_UDP_TUNNEL_IPV6 bit set, the GSO protocol is encapsulated
  in a UDP tunnel.
  If the outer UDP header requires checksumming, the driver must have
  additionally negotiated the VIRTIO_NET_F_HOST_UDP_TUNNEL_GSO_CSUM feature
  and offloaded the outer checksum accordingly, otherwise
  the outer UDP header must not require checksum validation, i.e. the outer
  UDP checksum must be positive zero (0x0) as defined in UDP RFC 768.
  The other tunnel-related fields indicate how to replicate the packet
  headers to cut it into smaller packets:

  \begin{itemize}
  \item \field{outer_th_offset} field indicates the outer transport header within
      the packet. This field differs from \field{csum_start} as the latter
      points to the inner transport header within the packet.

  \item \field{inner_nh_offset} field indicates the inner network header within
      the packet.
  \end{itemize}

\begin{note}
For example, consider a partially checksummed TCP (IPv4) packet carried over a
Geneve UDP tunnel (again IPv4) with no tunnel options. The
only relevant variable related to the tunnel type is the tunnel header length.
The packet will have a 14 byte outer ethernet header, 20 byte outer IP header
followed by the 8 byte UDP header (with a 0 checksum value), 8 byte Geneve header,
14 byte inner ethernet header, 20 byte inner IP header
and the TCP header (with the TCP checksum field 16 bytes
into that header). \field{csum_start} will be 14+20+8+8+14+20 = 84 (the TCP
checksum includes the header), \field{csum_offset} will be 16.
\field{inner_nh_offset} will be 14+20+8+8+14 = 62, \field{outer_th_offset} will be
14+20+8 = 42 and \field{gso_type} will be
VIRTIO_NET_HDR_GSO_TCPV4 | VIRTIO_NET_HDR_GSO_UDP_TUNNEL_IPV4 = 0x21
\end{note}

\item If the driver negotiated the VIRTIO_NET_F_HOST_UDP_TUNNEL_GSO_CSUM feature,
  the transmitted packet is a GSO one encapsulated in a UDP tunnel, and
  the outer UDP header requires checksumming, the driver can skip checksumming
  the outer header:

  \begin{itemize}
  \item \field{flags} has the VIRTIO_NET_HDR_F_UDP_TUNNEL_CSUM set,

  \item The outer UDP checksum field in the packet is set to the sum
    of the UDP pseudo header, so that replacing it by the ones'
    complement checksum of the outer UDP header and payload will give the
    correct result.
  \end{itemize}

\item \field{num_buffers} is set to zero.  This field is unused on transmitted packets.

\item The header and packet are added as one output descriptor to the
  transmitq, and the device is notified of the new entry
  (see \ref{sec:Device Types / Network Device / Device Initialization}~\nameref{sec:Device Types / Network Device / Device Initialization}).
\end{enumerate}

\drivernormative{\paragraph}{Packet Transmission}{Device Types / Network Device / Device Operation / Packet Transmission}

For the transmit packet buffer, the driver MUST use the size of the
structure \field{struct virtio_net_hdr} same as the receive packet buffer.

The driver MUST set \field{num_buffers} to zero.

If VIRTIO_NET_F_CSUM is not negotiated, the driver MUST set
\field{flags} to zero and SHOULD supply a fully checksummed
packet to the device.

If VIRTIO_NET_F_HOST_TSO4 is negotiated, the driver MAY set
\field{gso_type} to VIRTIO_NET_HDR_GSO_TCPV4 to request TCPv4
segmentation, otherwise the driver MUST NOT set
\field{gso_type} to VIRTIO_NET_HDR_GSO_TCPV4.

If VIRTIO_NET_F_HOST_TSO6 is negotiated, the driver MAY set
\field{gso_type} to VIRTIO_NET_HDR_GSO_TCPV6 to request TCPv6
segmentation, otherwise the driver MUST NOT set
\field{gso_type} to VIRTIO_NET_HDR_GSO_TCPV6.

If VIRTIO_NET_F_HOST_UFO is negotiated, the driver MAY set
\field{gso_type} to VIRTIO_NET_HDR_GSO_UDP to request UDP
fragmentation, otherwise the driver MUST NOT set
\field{gso_type} to VIRTIO_NET_HDR_GSO_UDP.

If VIRTIO_NET_F_HOST_USO is negotiated, the driver MAY set
\field{gso_type} to VIRTIO_NET_HDR_GSO_UDP_L4 to request UDP
segmentation, otherwise the driver MUST NOT set
\field{gso_type} to VIRTIO_NET_HDR_GSO_UDP_L4.

The driver SHOULD NOT send to the device TCP packets requiring segmentation offload
which have the Explicit Congestion Notification bit set, unless the
VIRTIO_NET_F_HOST_ECN feature is negotiated, in which case the
driver MUST set the VIRTIO_NET_HDR_GSO_ECN bit in
\field{gso_type}.

If VIRTIO_NET_F_HOST_UDP_TUNNEL_GSO is negotiated, the driver MAY set
VIRTIO_NET_HDR_GSO_UDP_TUNNEL_IPV4 bit or the VIRTIO_NET_HDR_GSO_UDP_TUNNEL_IPV6 bit
in \field{gso_type} according to the inner network header protocol type
to request GSO packets over UDPv4 or UDPv6 tunnel segmentation,
otherwise the driver MUST NOT set either the
VIRTIO_NET_HDR_GSO_UDP_TUNNEL_IPV4 bit or the VIRTIO_NET_HDR_GSO_UDP_TUNNEL_IPV6 bit
in \field{gso_type}.

When requesting GSO segmentation over UDP tunnel, the driver MUST SET the
VIRTIO_NET_HDR_GSO_UDP_TUNNEL_IPV4 bit if the inner network header is IPv4, i.e. the
packet is a TCPv4 GSO one, otherwise, if the inner network header is IPv6, the driver
MUST SET the VIRTIO_NET_HDR_GSO_UDP_TUNNEL_IPV6 bit.

The driver MUST NOT send to the device GSO packets over UDP tunnel
requiring segmentation and outer UDP checksum offload, unless both the
VIRTIO_NET_F_HOST_UDP_TUNNEL_GSO and VIRTIO_NET_F_HOST_UDP_TUNNEL_GSO_CSUM features
are negotiated, in which case the driver MUST set either the
VIRTIO_NET_HDR_GSO_UDP_TUNNEL_IPV4 bit or the VIRTIO_NET_HDR_GSO_UDP_TUNNEL_IPV6
bit in the \field{gso_type} and the VIRTIO_NET_HDR_F_UDP_TUNNEL_CSUM bit in
the \field{flags}.

If VIRTIO_NET_F_HOST_UDP_TUNNEL_GSO_CSUM is not negotiated, the driver MUST not set
the VIRTIO_NET_HDR_F_UDP_TUNNEL_CSUM bit in the \field{flags} and
MUST NOT send to the device GSO packets over UDP tunnel
requiring segmentation and outer UDP checksum offload.

The driver MUST NOT set the VIRTIO_NET_HDR_GSO_UDP_TUNNEL_IPV4 bit or the
VIRTIO_NET_HDR_GSO_UDP_TUNNEL_IPV6 bit together with VIRTIO_NET_HDR_GSO_UDP, as the
latter is deprecated in favor of UDP_L4 and no new feature will support it.

The driver MUST NOT set the VIRTIO_NET_HDR_GSO_UDP_TUNNEL_IPV4 bit and the
VIRTIO_NET_HDR_GSO_UDP_TUNNEL_IPV6 bit together.

The driver MUST NOT set the VIRTIO_NET_HDR_F_UDP_TUNNEL_CSUM bit \field{flags}
without setting either the VIRTIO_NET_HDR_GSO_UDP_TUNNEL_IPV4 bit or
the VIRTIO_NET_HDR_GSO_UDP_TUNNEL_IPV6 bit in \field{gso_type}.

If the VIRTIO_NET_F_CSUM feature has been negotiated, the
driver MAY set the VIRTIO_NET_HDR_F_NEEDS_CSUM bit in
\field{flags}, if so:
\begin{enumerate}
\item the driver MUST validate the packet checksum at
	offset \field{csum_offset} from \field{csum_start} as well as all
	preceding offsets;
\begin{note}
If \field{gso_type} differs from VIRTIO_NET_HDR_GSO_NONE and the
VIRTIO_NET_HDR_GSO_UDP_TUNNEL_IPV4 bit or the VIRTIO_NET_HDR_GSO_UDP_TUNNEL_IPV6
bit are not set in \field{gso_type}, \field{csum_offset}
points to the only transport header present in the packet, and there are no
additional preceding checksums validated by VIRTIO_NET_HDR_F_NEEDS_CSUM.
\end{note}
\item the driver MUST set the packet checksum stored in the
	buffer to the TCP/UDP pseudo header;
\item the driver MUST set \field{csum_start} and
	\field{csum_offset} such that calculating a ones'
	complement checksum from \field{csum_start} up until the end of
	the packet and storing the result at offset \field{csum_offset}
	from  \field{csum_start} will result in a fully checksummed
	packet;
\end{enumerate}

If none of the VIRTIO_NET_F_HOST_TSO4, TSO6, USO or UFO options have
been negotiated, the driver MUST set \field{gso_type} to
VIRTIO_NET_HDR_GSO_NONE.

If \field{gso_type} differs from VIRTIO_NET_HDR_GSO_NONE, then
the driver MUST also set the VIRTIO_NET_HDR_F_NEEDS_CSUM bit in
\field{flags} and MUST set \field{gso_size} to indicate the
desired MSS.

If one of the VIRTIO_NET_F_HOST_TSO4, TSO6, USO or UFO options have
been negotiated:
\begin{itemize}
\item If the VIRTIO_NET_F_GUEST_HDRLEN feature has been negotiated,
	and \field{gso_type} differs from VIRTIO_NET_HDR_GSO_NONE,
	the driver MUST set \field{hdr_len} to a value equal to the length
	of the headers, including the transport header. If \field{gso_type}
	has the VIRTIO_NET_HDR_GSO_UDP_TUNNEL_IPV4 bit or the
	VIRTIO_NET_HDR_GSO_UDP_TUNNEL_IPV6 bit set, \field{hdr_len} includes
	the inner transport header.

\item If the VIRTIO_NET_F_GUEST_HDRLEN feature has not been negotiated,
	or \field{gso_type} is VIRTIO_NET_HDR_GSO_NONE,
	the driver SHOULD set \field{hdr_len} to a value
	not less than the length of the headers, including the transport
	header.
\end{itemize}

If the VIRTIO_NET_F_HOST_UDP_TUNNEL_GSO option has been negotiated, the
driver MAY set the VIRTIO_NET_HDR_GSO_UDP_TUNNEL_IPV4 bit or the
VIRTIO_NET_HDR_GSO_UDP_TUNNEL_IPV6 bit in \field{gso_type}, if so:
\begin{itemize}
\item the driver MUST set \field{outer_th_offset} to the outer UDP header
  offset and \field{inner_nh_offset} to the inner network header offset.
  The \field{csum_start} and \field{csum_offset} fields point respectively
  to the inner transport header and inner transport checksum field.
\end{itemize}

If the VIRTIO_NET_F_HOST_UDP_TUNNEL_GSO_CSUM feature has been negotiated,
and the VIRTIO_NET_HDR_GSO_UDP_TUNNEL_IPV4 bit or
VIRTIO_NET_HDR_GSO_UDP_TUNNEL_IPV6 bit in \field{gso_type} are set,
the driver MAY set the VIRTIO_NET_HDR_F_UDP_TUNNEL_CSUM bit in
\field{flags}, if so the driver MUST set the packet outer UDP header checksum
to the outer UDP pseudo header checksum.

\begin{note}
calculating a ones' complement checksum from \field{outer_th_offset}
up until the end of the packet and storing the result at offset 6
from \field{outer_th_offset} will result in a fully checksummed outer UDP packet;
\end{note}

If the VIRTIO_NET_HDR_GSO_UDP_TUNNEL_IPV4 bit or the
VIRTIO_NET_HDR_GSO_UDP_TUNNEL_IPV6 bit in \field{gso_type} are set
and the VIRTIO_NET_F_HOST_UDP_TUNNEL_GSO_CSUM feature has not
been negotiated, the
outer UDP header MUST NOT require checksum validation. That is, the
outer UDP checksum value MUST be 0 or the validated complete checksum
for such header.

\begin{note}
The valid complete checksum of the outer UDP header of individual segments
can be computed by the driver prior to segmentation only if the GSO packet
size is a multiple of \field{gso_size}, because then all segments
have the same size and thus all data included in the outer UDP
checksum is the same for every segment. These pre-computed segment
length and checksum fields are different from those of the GSO
packet.
In this scenario the outer UDP header of the GSO packet must carry the
segmented UDP packet length.
\end{note}

If the VIRTIO_NET_F_HOST_UDP_TUNNEL_GSO option has not
been negotiated, the driver MUST NOT set either the VIRTIO_NET_HDR_F_GSO_UDP_TUNNEL_IPV4
bit or the VIRTIO_NET_HDR_F_GSO_UDP_TUNNEL_IPV6 in \field{gso_type}.

If the VIRTIO_NET_F_HOST_UDP_TUNNEL_GSO_CSUM option has not been negotiated,
the driver MUST NOT set the VIRTIO_NET_HDR_F_UDP_TUNNEL_CSUM bit
in \field{flags}.

The driver SHOULD accept the VIRTIO_NET_F_GUEST_HDRLEN feature if it has
been offered, and if it's able to provide the exact header length.

The driver MUST NOT set the VIRTIO_NET_HDR_F_DATA_VALID and
VIRTIO_NET_HDR_F_RSC_INFO bits in \field{flags}.

The driver MUST NOT set the VIRTIO_NET_HDR_F_DATA_VALID bit in \field{flags}
together with the VIRTIO_NET_HDR_F_GSO_UDP_TUNNEL_IPV4 bit or the
VIRTIO_NET_HDR_F_GSO_UDP_TUNNEL_IPV6 bit in \field{gso_type}.

\devicenormative{\paragraph}{Packet Transmission}{Device Types / Network Device / Device Operation / Packet Transmission}
The device MUST ignore \field{flag} bits that it does not recognize.

If VIRTIO_NET_HDR_F_NEEDS_CSUM bit in \field{flags} is not set, the
device MUST NOT use the \field{csum_start} and \field{csum_offset}.

If one of the VIRTIO_NET_F_HOST_TSO4, TSO6, USO or UFO options have
been negotiated:
\begin{itemize}
\item If the VIRTIO_NET_F_GUEST_HDRLEN feature has been negotiated,
	and \field{gso_type} differs from VIRTIO_NET_HDR_GSO_NONE,
	the device MAY use \field{hdr_len} as the transport header size.

	\begin{note}
	Caution should be taken by the implementation so as to prevent
	a malicious driver from attacking the device by setting an incorrect hdr_len.
	\end{note}

\item If the VIRTIO_NET_F_GUEST_HDRLEN feature has not been negotiated,
	or \field{gso_type} is VIRTIO_NET_HDR_GSO_NONE,
	the device MAY use \field{hdr_len} only as a hint about the
	transport header size.
	The device MUST NOT rely on \field{hdr_len} to be correct.

	\begin{note}
	This is due to various bugs in implementations.
	\end{note}
\end{itemize}

If both the VIRTIO_NET_HDR_GSO_UDP_TUNNEL_IPV4 bit and
the VIRTIO_NET_HDR_GSO_UDP_TUNNEL_IPV6 bit in in \field{gso_type} are set,
the device MUST NOT accept the packet.

If the VIRTIO_NET_HDR_GSO_UDP_TUNNEL_IPV4 bit and the VIRTIO_NET_HDR_GSO_UDP_TUNNEL_IPV6
bit in \field{gso_type} are not set, the device MUST NOT use the
\field{outer_th_offset} and \field{inner_nh_offset}.

If either the VIRTIO_NET_HDR_GSO_UDP_TUNNEL_IPV4 bit or
the VIRTIO_NET_HDR_GSO_UDP_TUNNEL_IPV6 bit in \field{gso_type} are set, and any of
the following is true:
\begin{itemize}
\item the VIRTIO_NET_HDR_F_NEEDS_CSUM is not set in \field{flags}
\item the VIRTIO_NET_HDR_F_DATA_VALID is set in \field{flags}
\item the \field{gso_type} excluding the VIRTIO_NET_HDR_GSO_UDP_TUNNEL_IPV4
bit and the VIRTIO_NET_HDR_GSO_UDP_TUNNEL_IPV6 bit is VIRTIO_NET_HDR_GSO_NONE
\end{itemize}
the device MUST NOT accept the packet.

If the VIRTIO_NET_HDR_F_UDP_TUNNEL_CSUM bit in \field{flags} is set,
and both the bits VIRTIO_NET_HDR_GSO_UDP_TUNNEL_IPV4 and
VIRTIO_NET_HDR_GSO_UDP_TUNNEL_IPV6 in \field{gso_type} are not set,
the device MOST NOT accept the packet.

If VIRTIO_NET_HDR_F_NEEDS_CSUM is not set, the device MUST NOT
rely on the packet checksum being correct.
\paragraph{Packet Transmission Interrupt}\label{sec:Device Types / Network Device / Device Operation / Packet Transmission / Packet Transmission Interrupt}

Often a driver will suppress transmission virtqueue interrupts
and check for used packets in the transmit path of following
packets.

The normal behavior in this interrupt handler is to retrieve
used buffers from the virtqueue and free the corresponding
headers and packets.

\subsubsection{Setting Up Receive Buffers}\label{sec:Device Types / Network Device / Device Operation / Setting Up Receive Buffers}

It is generally a good idea to keep the receive virtqueue as
fully populated as possible: if it runs out, network performance
will suffer.

If the VIRTIO_NET_F_GUEST_TSO4, VIRTIO_NET_F_GUEST_TSO6,
VIRTIO_NET_F_GUEST_UFO, VIRTIO_NET_F_GUEST_USO4 or VIRTIO_NET_F_GUEST_USO6
features are used, the maximum incoming packet
will be 65589 bytes long (14 bytes of Ethernet header, plus 40 bytes of
the IPv6 header, plus 65535 bytes of maximum IPv6 payload including any
extension header), otherwise 1514 bytes.
When VIRTIO_NET_F_HASH_REPORT is not negotiated, the required receive buffer
size is either 65601 or 1526 bytes accounting for 20 bytes of
\field{struct virtio_net_hdr} followed by receive packet.
When VIRTIO_NET_F_HASH_REPORT is negotiated, the required receive buffer
size is either 65609 or 1534 bytes accounting for 12 bytes of
\field{struct virtio_net_hdr} followed by receive packet.

\drivernormative{\paragraph}{Setting Up Receive Buffers}{Device Types / Network Device / Device Operation / Setting Up Receive Buffers}

\begin{itemize}
\item If VIRTIO_NET_F_MRG_RXBUF is not negotiated:
  \begin{itemize}
    \item If VIRTIO_NET_F_GUEST_TSO4, VIRTIO_NET_F_GUEST_TSO6, VIRTIO_NET_F_GUEST_UFO,
	VIRTIO_NET_F_GUEST_USO4 or VIRTIO_NET_F_GUEST_USO6 are negotiated, the driver SHOULD populate
      the receive queue(s) with buffers of at least 65609 bytes if
      VIRTIO_NET_F_HASH_REPORT is negotiated, and of at least 65601 bytes if not.
    \item Otherwise, the driver SHOULD populate the receive queue(s)
      with buffers of at least 1534 bytes if VIRTIO_NET_F_HASH_REPORT
      is negotiated, and of at least 1526 bytes if not.
  \end{itemize}
\item If VIRTIO_NET_F_MRG_RXBUF is negotiated, each buffer MUST be at
least size of \field{struct virtio_net_hdr},
i.e. 20 bytes if VIRTIO_NET_F_HASH_REPORT is negotiated, and 12 bytes if not.
\end{itemize}

\begin{note}
Obviously each buffer can be split across multiple descriptor elements.
\end{note}

When calculating the size of \field{struct virtio_net_hdr}, the driver
MUST consider all the fields inclusive up to \field{padding_reserved},
i.e. 20 bytes if VIRTIO_NET_F_HASH_REPORT is negotiated, and 12 bytes if not.

If VIRTIO_NET_F_MQ is negotiated, each of receiveq1\ldots receiveqN
that will be used SHOULD be populated with receive buffers.

\devicenormative{\paragraph}{Setting Up Receive Buffers}{Device Types / Network Device / Device Operation / Setting Up Receive Buffers}

The device MUST set \field{num_buffers} to the number of descriptors used to
hold the incoming packet.

The device MUST use only a single descriptor if VIRTIO_NET_F_MRG_RXBUF
was not negotiated.
\begin{note}
{This means that \field{num_buffers} will always be 1
if VIRTIO_NET_F_MRG_RXBUF is not negotiated.}
\end{note}

\subsubsection{Processing of Incoming Packets}\label{sec:Device Types / Network Device / Device Operation / Processing of Incoming Packets}
\label{sec:Device Types / Network Device / Device Operation / Processing of Packets}%old label for latexdiff

When a packet is copied into a buffer in the receiveq, the
optimal path is to disable further used buffer notifications for the
receiveq and process packets until no more are found, then re-enable
them.

Processing incoming packets involves:

\begin{enumerate}
\item \field{num_buffers} indicates how many descriptors
  this packet is spread over (including this one): this will
  always be 1 if VIRTIO_NET_F_MRG_RXBUF was not negotiated.
  This allows receipt of large packets without having to allocate large
  buffers: a packet that does not fit in a single buffer can flow
  over to the next buffer, and so on. In this case, there will be
  at least \field{num_buffers} used buffers in the virtqueue, and the device
  chains them together to form a single packet in a way similar to
  how it would store it in a single buffer spread over multiple
  descriptors.
  The other buffers will not begin with a \field{struct virtio_net_hdr}.

\item If
  \field{num_buffers} is one, then the entire packet will be
  contained within this buffer, immediately following the struct
  virtio_net_hdr.
\item If the VIRTIO_NET_F_GUEST_CSUM feature was negotiated, the
  VIRTIO_NET_HDR_F_DATA_VALID bit in \field{flags} can be
  set: if so, device has validated the packet checksum.
  If the VIRTIO_NET_F_GUEST_UDP_TUNNEL_GSO_CSUM feature has been negotiated,
  and the VIRTIO_NET_HDR_F_UDP_TUNNEL_CSUM bit is set in \field{flags},
  both the outer UDP checksum and the inner transport checksum
  have been validated, otherwise only one level of checksums (the outer one
  in case of tunnels) has been validated.
\end{enumerate}

Additionally, VIRTIO_NET_F_GUEST_CSUM, TSO4, TSO6, UDP, UDP_TUNNEL
and ECN features enable receive checksum, large receive offload and ECN
support which are the input equivalents of the transmit checksum,
transmit segmentation offloading and ECN features, as described
in \ref{sec:Device Types / Network Device / Device Operation /
Packet Transmission}:
\begin{enumerate}
\item If the VIRTIO_NET_F_GUEST_TSO4, TSO6, UFO, USO4 or USO6 options were
  negotiated, then \field{gso_type} MAY be something other than
  VIRTIO_NET_HDR_GSO_NONE, and \field{gso_size} field indicates the
  desired MSS (see Packet Transmission point 2).
\item If the VIRTIO_NET_F_RSC_EXT option was negotiated (this
  implies one of VIRTIO_NET_F_GUEST_TSO4, TSO6), the
  device processes also duplicated ACK segments, reports
  number of coalesced TCP segments in \field{csum_start} field and
  number of duplicated ACK segments in \field{csum_offset} field
  and sets bit VIRTIO_NET_HDR_F_RSC_INFO in \field{flags}.
\item If the VIRTIO_NET_F_GUEST_CSUM feature was negotiated, the
  VIRTIO_NET_HDR_F_NEEDS_CSUM bit in \field{flags} can be
  set: if so, the packet checksum at offset \field{csum_offset}
  from \field{csum_start} and any preceding checksums
  have been validated.  The checksum on the packet is incomplete and
  if bit VIRTIO_NET_HDR_F_RSC_INFO is not set in \field{flags},
  then \field{csum_start} and \field{csum_offset} indicate how to calculate it
  (see Packet Transmission point 1).
\begin{note}
If \field{gso_type} differs from VIRTIO_NET_HDR_GSO_NONE and the
VIRTIO_NET_HDR_GSO_UDP_TUNNEL_IPV4 bit or the VIRTIO_NET_HDR_GSO_UDP_TUNNEL_IPV6
bit are not set, \field{csum_offset}
points to the only transport header present in the packet, and there are no
additional preceding checksums validated by VIRTIO_NET_HDR_F_NEEDS_CSUM.
\end{note}
\item If the VIRTIO_NET_F_GUEST_UDP_TUNNEL_GSO option was negotiated and
  \field{gso_type} is not VIRTIO_NET_HDR_GSO_NONE, the
  VIRTIO_NET_HDR_GSO_UDP_TUNNEL_IPV4 bit or the VIRTIO_NET_HDR_GSO_UDP_TUNNEL_IPV6
  bit MAY be set. In such case the \field{outer_th_offset} and
  \field{inner_nh_offset} fields indicate the corresponding
  headers information.
\item If the VIRTIO_NET_F_GUEST_UDP_TUNNEL_GSO_CSUM feature was
negotiated, and
  the VIRTIO_NET_HDR_GSO_UDP_TUNNEL_IPV4 bit or the VIRTIO_NET_HDR_GSO_UDP_TUNNEL_IPV6
  are set in \field{gso_type}, the VIRTIO_NET_HDR_F_UDP_TUNNEL_CSUM bit in the
  \field{flags} can be set: if so, the outer UDP checksum has been validated
  and the UDP header checksum at offset 6 from from \field{outer_th_offset}
  is set to the outer UDP pseudo header checksum.

\begin{note}
If the VIRTIO_NET_HDR_GSO_UDP_TUNNEL_IPV4 bit or VIRTIO_NET_HDR_GSO_UDP_TUNNEL_IPV6
bit are set in \field{gso_type}, the \field{csum_start} field refers to
the inner transport header offset (see Packet Transmission point 1).
If the VIRTIO_NET_HDR_F_UDP_TUNNEL_CSUM bit in \field{flags} is set both
the inner and the outer header checksums have been validated by
VIRTIO_NET_HDR_F_NEEDS_CSUM, otherwise only the inner transport header
checksum has been validated.
\end{note}
\end{enumerate}

If applicable, the device calculates per-packet hash for incoming packets as
defined in \ref{sec:Device Types / Network Device / Device Operation / Processing of Incoming Packets / Hash calculation for incoming packets}.

If applicable, the device reports hash information for incoming packets as
defined in \ref{sec:Device Types / Network Device / Device Operation / Processing of Incoming Packets / Hash reporting for incoming packets}.

\devicenormative{\paragraph}{Processing of Incoming Packets}{Device Types / Network Device / Device Operation / Processing of Incoming Packets}
\label{devicenormative:Device Types / Network Device / Device Operation / Processing of Packets}%old label for latexdiff

If VIRTIO_NET_F_MRG_RXBUF has not been negotiated, the device MUST set
\field{num_buffers} to 1.

If VIRTIO_NET_F_MRG_RXBUF has been negotiated, the device MUST set
\field{num_buffers} to indicate the number of buffers
the packet (including the header) is spread over.

If a receive packet is spread over multiple buffers, the device
MUST use all buffers but the last (i.e. the first \field{num_buffers} -
1 buffers) completely up to the full length of each buffer
supplied by the driver.

The device MUST use all buffers used by a single receive
packet together, such that at least \field{num_buffers} are
observed by driver as used.

If VIRTIO_NET_F_GUEST_CSUM is not negotiated, the device MUST set
\field{flags} to zero and SHOULD supply a fully checksummed
packet to the driver.

If VIRTIO_NET_F_GUEST_TSO4 is not negotiated, the device MUST NOT set
\field{gso_type} to VIRTIO_NET_HDR_GSO_TCPV4.

If VIRTIO_NET_F_GUEST_UDP is not negotiated, the device MUST NOT set
\field{gso_type} to VIRTIO_NET_HDR_GSO_UDP.

If VIRTIO_NET_F_GUEST_TSO6 is not negotiated, the device MUST NOT set
\field{gso_type} to VIRTIO_NET_HDR_GSO_TCPV6.

If none of VIRTIO_NET_F_GUEST_USO4 or VIRTIO_NET_F_GUEST_USO6 have been negotiated,
the device MUST NOT set \field{gso_type} to VIRTIO_NET_HDR_GSO_UDP_L4.

If VIRTIO_NET_F_GUEST_UDP_TUNNEL_GSO is not negotiated, the device MUST NOT set
either the VIRTIO_NET_HDR_GSO_UDP_TUNNEL_IPV4 bit or the
VIRTIO_NET_HDR_GSO_UDP_TUNNEL_IPV6 bit in \field{gso_type}.

If VIRTIO_NET_F_GUEST_UDP_TUNNEL_GSO_CSUM is not negotiated the device MUST NOT set
the VIRTIO_NET_HDR_F_UDP_TUNNEL_CSUM bit in \field{flags}.

The device SHOULD NOT send to the driver TCP packets requiring segmentation offload
which have the Explicit Congestion Notification bit set, unless the
VIRTIO_NET_F_GUEST_ECN feature is negotiated, in which case the
device MUST set the VIRTIO_NET_HDR_GSO_ECN bit in
\field{gso_type}.

If the VIRTIO_NET_F_GUEST_CSUM feature has been negotiated, the
device MAY set the VIRTIO_NET_HDR_F_NEEDS_CSUM bit in
\field{flags}, if so:
\begin{enumerate}
\item the device MUST validate the packet checksum at
	offset \field{csum_offset} from \field{csum_start} as well as all
	preceding offsets;
\item the device MUST set the packet checksum stored in the
	receive buffer to the TCP/UDP pseudo header;
\item the device MUST set \field{csum_start} and
	\field{csum_offset} such that calculating a ones'
	complement checksum from \field{csum_start} up until the
	end of the packet and storing the result at offset
	\field{csum_offset} from  \field{csum_start} will result in a
	fully checksummed packet;
\end{enumerate}

The device MUST NOT send to the driver GSO packets encapsulated in UDP
tunnel and requiring segmentation offload, unless the
VIRTIO_NET_F_GUEST_UDP_TUNNEL_GSO is negotiated, in which case the device MUST set
the VIRTIO_NET_HDR_GSO_UDP_TUNNEL_IPV4 bit or the VIRTIO_NET_HDR_GSO_UDP_TUNNEL_IPV6
bit in \field{gso_type} according to the inner network header protocol type,
MUST set the \field{outer_th_offset} and \field{inner_nh_offset} fields
to the corresponding header information, and the outer UDP header MUST NOT
require checksum offload.

If the VIRTIO_NET_F_GUEST_UDP_TUNNEL_GSO_CSUM feature has not been negotiated,
the device MUST NOT send the driver GSO packets encapsulated in UDP
tunnel and requiring segmentation and outer checksum offload.

If none of the VIRTIO_NET_F_GUEST_TSO4, TSO6, UFO, USO4 or USO6 options have
been negotiated, the device MUST set \field{gso_type} to
VIRTIO_NET_HDR_GSO_NONE.

If \field{gso_type} differs from VIRTIO_NET_HDR_GSO_NONE, then
the device MUST also set the VIRTIO_NET_HDR_F_NEEDS_CSUM bit in
\field{flags} MUST set \field{gso_size} to indicate the desired MSS.
If VIRTIO_NET_F_RSC_EXT was negotiated, the device MUST also
set VIRTIO_NET_HDR_F_RSC_INFO bit in \field{flags},
set \field{csum_start} to number of coalesced TCP segments and
set \field{csum_offset} to number of received duplicated ACK segments.

If VIRTIO_NET_F_RSC_EXT was not negotiated, the device MUST
not set VIRTIO_NET_HDR_F_RSC_INFO bit in \field{flags}.

If one of the VIRTIO_NET_F_GUEST_TSO4, TSO6, UFO, USO4 or USO6 options have
been negotiated, the device SHOULD set \field{hdr_len} to a value
not less than the length of the headers, including the transport
header. If \field{gso_type} has the VIRTIO_NET_HDR_GSO_UDP_TUNNEL_IPV4 bit
or the VIRTIO_NET_HDR_GSO_UDP_TUNNEL_IPV6 bit set, the referenced transport
header is the inner one.

If the VIRTIO_NET_F_GUEST_CSUM feature has been negotiated, the
device MAY set the VIRTIO_NET_HDR_F_DATA_VALID bit in
\field{flags}, if so, the device MUST validate the packet
checksum. If the VIRTIO_NET_F_GUEST_UDP_TUNNEL_GSO_CSUM feature has
been negotiated, and the VIRTIO_NET_HDR_F_UDP_TUNNEL_CSUM bit set in
\field{flags}, both the outer UDP checksum and the inner transport
checksum have been validated.
Otherwise level of checksum is validated: in case of multiple
encapsulated protocols the outermost one.

If either the VIRTIO_NET_HDR_GSO_UDP_TUNNEL_IPV4 bit or the
VIRTIO_NET_HDR_GSO_UDP_TUNNEL_IPV6 bit in \field{gso_type} are set,
the device MUST NOT set the VIRTIO_NET_HDR_F_DATA_VALID bit in
\field{flags}.

If the VIRTIO_NET_F_GUEST_UDP_TUNNEL_GSO_CSUM feature has been negotiated
and either the VIRTIO_NET_HDR_GSO_UDP_TUNNEL_IPV4 bit is set or the
VIRTIO_NET_HDR_GSO_UDP_TUNNEL_IPV6 bit is set in \field{gso_type}, the
device MAY set the VIRTIO_NET_HDR_F_UDP_TUNNEL_CSUM bit in
\field{flags}, if so the device MUST set the packet outer UDP checksum
stored in the receive buffer to the outer UDP pseudo header.

Otherwise, the VIRTIO_NET_F_GUEST_UDP_TUNNEL_GSO_CSUM feature has been
negotiated, either the VIRTIO_NET_HDR_GSO_UDP_TUNNEL_IPV4 bit is set or the
VIRTIO_NET_HDR_GSO_UDP_TUNNEL_IPV6 bit is set in \field{gso_type},
and the bit VIRTIO_NET_HDR_F_UDP_TUNNEL_CSUM is not set in
\field{flags}, the device MUST either provide a zero outer UDP header
checksum or a fully checksummed outer UDP header.

\drivernormative{\paragraph}{Processing of Incoming
Packets}{Device Types / Network Device / Device Operation /
Processing of Incoming Packets}

The driver MUST ignore \field{flag} bits that it does not recognize.

If VIRTIO_NET_HDR_F_NEEDS_CSUM bit in \field{flags} is not set or
if VIRTIO_NET_HDR_F_RSC_INFO bit \field{flags} is set, the
driver MUST NOT use the \field{csum_start} and \field{csum_offset}.

If one of the VIRTIO_NET_F_GUEST_TSO4, TSO6, UFO, USO4 or USO6 options have
been negotiated, the driver MAY use \field{hdr_len} only as a hint about the
transport header size.
The driver MUST NOT rely on \field{hdr_len} to be correct.
\begin{note}
This is due to various bugs in implementations.
\end{note}

If neither VIRTIO_NET_HDR_F_NEEDS_CSUM nor
VIRTIO_NET_HDR_F_DATA_VALID is set, the driver MUST NOT
rely on the packet checksum being correct.

If both the VIRTIO_NET_HDR_GSO_UDP_TUNNEL_IPV4 bit and
the VIRTIO_NET_HDR_GSO_UDP_TUNNEL_IPV6 bit in in \field{gso_type} are set,
the driver MUST NOT accept the packet.

If the VIRTIO_NET_HDR_GSO_UDP_TUNNEL_IPV4 bit or the VIRTIO_NET_HDR_GSO_UDP_TUNNEL_IPV6
bit in \field{gso_type} are not set, the driver MUST NOT use the
\field{outer_th_offset} and \field{inner_nh_offset}.

If either the VIRTIO_NET_HDR_GSO_UDP_TUNNEL_IPV4 bit or
the VIRTIO_NET_HDR_GSO_UDP_TUNNEL_IPV6 bit in \field{gso_type} are set, and any of
the following is true:
\begin{itemize}
\item the VIRTIO_NET_HDR_F_NEEDS_CSUM bit is not set in \field{flags}
\item the VIRTIO_NET_HDR_F_DATA_VALID bit is set in \field{flags}
\item the \field{gso_type} excluding the VIRTIO_NET_HDR_GSO_UDP_TUNNEL_IPV4
bit and the VIRTIO_NET_HDR_GSO_UDP_TUNNEL_IPV6 bit is VIRTIO_NET_HDR_GSO_NONE
\end{itemize}
the driver MUST NOT accept the packet.

If the VIRTIO_NET_HDR_F_UDP_TUNNEL_CSUM bit and the VIRTIO_NET_HDR_F_NEEDS_CSUM
bit in \field{flags} are set,
and both the bits VIRTIO_NET_HDR_GSO_UDP_TUNNEL_IPV4 and
VIRTIO_NET_HDR_GSO_UDP_TUNNEL_IPV6 in \field{gso_type} are not set,
the driver MOST NOT accept the packet.

\paragraph{Hash calculation for incoming packets}
\label{sec:Device Types / Network Device / Device Operation / Processing of Incoming Packets / Hash calculation for incoming packets}

A device attempts to calculate a per-packet hash in the following cases:
\begin{itemize}
\item The feature VIRTIO_NET_F_RSS was negotiated. The device uses the hash to determine the receive virtqueue to place incoming packets.
\item The feature VIRTIO_NET_F_HASH_REPORT was negotiated. The device reports the hash value and the hash type with the packet.
\end{itemize}

If the feature VIRTIO_NET_F_RSS was negotiated:
\begin{itemize}
\item The device uses \field{hash_types} of the virtio_net_rss_config structure as 'Enabled hash types' bitmask.
\item If additionally the feature VIRTIO_NET_F_HASH_TUNNEL was negotiated, the device uses \field{enabled_tunnel_types} of the
      virtnet_hash_tunnel structure as 'Encapsulation types enabled for inner header hash' bitmask.
\item The device uses a key as defined in \field{hash_key_data} and \field{hash_key_length} of the virtio_net_rss_config structure (see
\ref{sec:Device Types / Network Device / Device Operation / Control Virtqueue / Receive-side scaling (RSS) / Setting RSS parameters}).
\end{itemize}

If the feature VIRTIO_NET_F_RSS was not negotiated:
\begin{itemize}
\item The device uses \field{hash_types} of the virtio_net_hash_config structure as 'Enabled hash types' bitmask.
\item If additionally the feature VIRTIO_NET_F_HASH_TUNNEL was negotiated, the device uses \field{enabled_tunnel_types} of the
      virtnet_hash_tunnel structure as 'Encapsulation types enabled for inner header hash' bitmask.
\item The device uses a key as defined in \field{hash_key_data} and \field{hash_key_length} of the virtio_net_hash_config structure (see
\ref{sec:Device Types / Network Device / Device Operation / Control Virtqueue / Automatic receive steering in multiqueue mode / Hash calculation}).
\end{itemize}

Note that if the device offers VIRTIO_NET_F_HASH_REPORT, even if it supports only one pair of virtqueues, it MUST support
at least one of commands of VIRTIO_NET_CTRL_MQ class to configure reported hash parameters:
\begin{itemize}
\item If the device offers VIRTIO_NET_F_RSS, it MUST support VIRTIO_NET_CTRL_MQ_RSS_CONFIG command per
 \ref{sec:Device Types / Network Device / Device Operation / Control Virtqueue / Receive-side scaling (RSS) / Setting RSS parameters}.
\item Otherwise the device MUST support VIRTIO_NET_CTRL_MQ_HASH_CONFIG command per
 \ref{sec:Device Types / Network Device / Device Operation / Control Virtqueue / Automatic receive steering in multiqueue mode / Hash calculation}.
\end{itemize}

The per-packet hash calculation can depend on the IP packet type. See
\hyperref[intro:IP]{[IP]}, \hyperref[intro:UDP]{[UDP]} and \hyperref[intro:TCP]{[TCP]}.

\subparagraph{Supported/enabled hash types}
\label{sec:Device Types / Network Device / Device Operation / Processing of Incoming Packets / Hash calculation for incoming packets / Supported/enabled hash types}
Hash types applicable for IPv4 packets:
\begin{lstlisting}
#define VIRTIO_NET_HASH_TYPE_IPv4              (1 << 0)
#define VIRTIO_NET_HASH_TYPE_TCPv4             (1 << 1)
#define VIRTIO_NET_HASH_TYPE_UDPv4             (1 << 2)
\end{lstlisting}
Hash types applicable for IPv6 packets without extension headers
\begin{lstlisting}
#define VIRTIO_NET_HASH_TYPE_IPv6              (1 << 3)
#define VIRTIO_NET_HASH_TYPE_TCPv6             (1 << 4)
#define VIRTIO_NET_HASH_TYPE_UDPv6             (1 << 5)
\end{lstlisting}
Hash types applicable for IPv6 packets with extension headers
\begin{lstlisting}
#define VIRTIO_NET_HASH_TYPE_IP_EX             (1 << 6)
#define VIRTIO_NET_HASH_TYPE_TCP_EX            (1 << 7)
#define VIRTIO_NET_HASH_TYPE_UDP_EX            (1 << 8)
\end{lstlisting}

\subparagraph{IPv4 packets}
\label{sec:Device Types / Network Device / Device Operation / Processing of Incoming Packets / Hash calculation for incoming packets / IPv4 packets}
The device calculates the hash on IPv4 packets according to 'Enabled hash types' bitmask as follows:
\begin{itemize}
\item If VIRTIO_NET_HASH_TYPE_TCPv4 is set and the packet has
a TCP header, the hash is calculated over the following fields:
\begin{itemize}
\item Source IP address
\item Destination IP address
\item Source TCP port
\item Destination TCP port
\end{itemize}
\item Else if VIRTIO_NET_HASH_TYPE_UDPv4 is set and the
packet has a UDP header, the hash is calculated over the following fields:
\begin{itemize}
\item Source IP address
\item Destination IP address
\item Source UDP port
\item Destination UDP port
\end{itemize}
\item Else if VIRTIO_NET_HASH_TYPE_IPv4 is set, the hash is
calculated over the following fields:
\begin{itemize}
\item Source IP address
\item Destination IP address
\end{itemize}
\item Else the device does not calculate the hash
\end{itemize}

\subparagraph{IPv6 packets without extension header}
\label{sec:Device Types / Network Device / Device Operation / Processing of Incoming Packets / Hash calculation for incoming packets / IPv6 packets without extension header}
The device calculates the hash on IPv6 packets without extension
headers according to 'Enabled hash types' bitmask as follows:
\begin{itemize}
\item If VIRTIO_NET_HASH_TYPE_TCPv6 is set and the packet has
a TCPv6 header, the hash is calculated over the following fields:
\begin{itemize}
\item Source IPv6 address
\item Destination IPv6 address
\item Source TCP port
\item Destination TCP port
\end{itemize}
\item Else if VIRTIO_NET_HASH_TYPE_UDPv6 is set and the
packet has a UDPv6 header, the hash is calculated over the following fields:
\begin{itemize}
\item Source IPv6 address
\item Destination IPv6 address
\item Source UDP port
\item Destination UDP port
\end{itemize}
\item Else if VIRTIO_NET_HASH_TYPE_IPv6 is set, the hash is
calculated over the following fields:
\begin{itemize}
\item Source IPv6 address
\item Destination IPv6 address
\end{itemize}
\item Else the device does not calculate the hash
\end{itemize}

\subparagraph{IPv6 packets with extension header}
\label{sec:Device Types / Network Device / Device Operation / Processing of Incoming Packets / Hash calculation for incoming packets / IPv6 packets with extension header}
The device calculates the hash on IPv6 packets with extension
headers according to 'Enabled hash types' bitmask as follows:
\begin{itemize}
\item If VIRTIO_NET_HASH_TYPE_TCP_EX is set and the packet
has a TCPv6 header, the hash is calculated over the following fields:
\begin{itemize}
\item Home address from the home address option in the IPv6 destination options header. If the extension header is not present, use the Source IPv6 address.
\item IPv6 address that is contained in the Routing-Header-Type-2 from the associated extension header. If the extension header is not present, use the Destination IPv6 address.
\item Source TCP port
\item Destination TCP port
\end{itemize}
\item Else if VIRTIO_NET_HASH_TYPE_UDP_EX is set and the
packet has a UDPv6 header, the hash is calculated over the following fields:
\begin{itemize}
\item Home address from the home address option in the IPv6 destination options header. If the extension header is not present, use the Source IPv6 address.
\item IPv6 address that is contained in the Routing-Header-Type-2 from the associated extension header. If the extension header is not present, use the Destination IPv6 address.
\item Source UDP port
\item Destination UDP port
\end{itemize}
\item Else if VIRTIO_NET_HASH_TYPE_IP_EX is set, the hash is
calculated over the following fields:
\begin{itemize}
\item Home address from the home address option in the IPv6 destination options header. If the extension header is not present, use the Source IPv6 address.
\item IPv6 address that is contained in the Routing-Header-Type-2 from the associated extension header. If the extension header is not present, use the Destination IPv6 address.
\end{itemize}
\item Else skip IPv6 extension headers and calculate the hash as
defined for an IPv6 packet without extension headers
(see \ref{sec:Device Types / Network Device / Device Operation / Processing of Incoming Packets / Hash calculation for incoming packets / IPv6 packets without extension header}).
\end{itemize}

\paragraph{Inner Header Hash}
\label{sec:Device Types / Network Device / Device Operation / Processing of Incoming Packets / Inner Header Hash}

If VIRTIO_NET_F_HASH_TUNNEL has been negotiated, the driver can send the command
VIRTIO_NET_CTRL_HASH_TUNNEL_SET to configure the calculation of the inner header hash.

\begin{lstlisting}
struct virtnet_hash_tunnel {
    le32 enabled_tunnel_types;
};

#define VIRTIO_NET_CTRL_HASH_TUNNEL 7
 #define VIRTIO_NET_CTRL_HASH_TUNNEL_SET 0
\end{lstlisting}

Field \field{enabled_tunnel_types} contains the bitmask of encapsulation types enabled for inner header hash.
See \ref{sec:Device Types / Network Device / Device Operation / Processing of Incoming Packets /
Hash calculation for incoming packets / Encapsulation types supported/enabled for inner header hash}.

The class VIRTIO_NET_CTRL_HASH_TUNNEL has one command:
VIRTIO_NET_CTRL_HASH_TUNNEL_SET sets \field{enabled_tunnel_types} for the device using the
virtnet_hash_tunnel structure, which is read-only for the device.

Inner header hash is disabled by VIRTIO_NET_CTRL_HASH_TUNNEL_SET with \field{enabled_tunnel_types} set to 0.

Initially (before the driver sends any VIRTIO_NET_CTRL_HASH_TUNNEL_SET command) all
encapsulation types are disabled for inner header hash.

\subparagraph{Encapsulated packet}
\label{sec:Device Types / Network Device / Device Operation / Processing of Incoming Packets / Hash calculation for incoming packets / Encapsulated packet}

Multiple tunneling protocols allow encapsulating an inner, payload packet in an outer, encapsulated packet.
The encapsulated packet thus contains an outer header and an inner header, and the device calculates the
hash over either the inner header or the outer header.

If VIRTIO_NET_F_HASH_TUNNEL is negotiated and a received encapsulated packet's outer header matches one of the
encapsulation types enabled in \field{enabled_tunnel_types}, then the device uses the inner header for hash
calculations (only a single level of encapsulation is currently supported).

If VIRTIO_NET_F_HASH_TUNNEL is negotiated and a received packet's (outer) header does not match any encapsulation
types enabled in \field{enabled_tunnel_types}, then the device uses the outer header for hash calculations.

\subparagraph{Encapsulation types supported/enabled for inner header hash}
\label{sec:Device Types / Network Device / Device Operation / Processing of Incoming Packets /
Hash calculation for incoming packets / Encapsulation types supported/enabled for inner header hash}

Encapsulation types applicable for inner header hash:
\begin{lstlisting}[escapechar=|]
#define VIRTIO_NET_HASH_TUNNEL_TYPE_GRE_2784    (1 << 0) /* |\hyperref[intro:rfc2784]{[RFC2784]}| */
#define VIRTIO_NET_HASH_TUNNEL_TYPE_GRE_2890    (1 << 1) /* |\hyperref[intro:rfc2890]{[RFC2890]}| */
#define VIRTIO_NET_HASH_TUNNEL_TYPE_GRE_7676    (1 << 2) /* |\hyperref[intro:rfc7676]{[RFC7676]}| */
#define VIRTIO_NET_HASH_TUNNEL_TYPE_GRE_UDP     (1 << 3) /* |\hyperref[intro:rfc8086]{[GRE-in-UDP]}| */
#define VIRTIO_NET_HASH_TUNNEL_TYPE_VXLAN       (1 << 4) /* |\hyperref[intro:vxlan]{[VXLAN]}| */
#define VIRTIO_NET_HASH_TUNNEL_TYPE_VXLAN_GPE   (1 << 5) /* |\hyperref[intro:vxlan-gpe]{[VXLAN-GPE]}| */
#define VIRTIO_NET_HASH_TUNNEL_TYPE_GENEVE      (1 << 6) /* |\hyperref[intro:geneve]{[GENEVE]}| */
#define VIRTIO_NET_HASH_TUNNEL_TYPE_IPIP        (1 << 7) /* |\hyperref[intro:ipip]{[IPIP]}| */
#define VIRTIO_NET_HASH_TUNNEL_TYPE_NVGRE       (1 << 8) /* |\hyperref[intro:nvgre]{[NVGRE]}| */
\end{lstlisting}

\subparagraph{Advice}
Example uses of the inner header hash:
\begin{itemize}
\item Legacy tunneling protocols, lacking the outer header entropy, can use RSS with the inner header hash to
      distribute flows with identical outer but different inner headers across various queues, improving performance.
\item Identify an inner flow distributed across multiple outer tunnels.
\end{itemize}

As using the inner header hash completely discards the outer header entropy, care must be taken
if the inner header is controlled by an adversary, as the adversary can then intentionally create
configurations with insufficient entropy.

Besides disabling the inner header hash, mitigations would depend on how the hash is used. When the hash
use is limited to the RSS queue selection, the inner header hash may have quality of service (QoS) limitations.

\devicenormative{\subparagraph}{Inner Header Hash}{Device Types / Network Device / Device Operation / Control Virtqueue / Inner Header Hash}

If the (outer) header of the received packet does not match any encapsulation types enabled
in \field{enabled_tunnel_types}, the device MUST calculate the hash on the outer header.

If the device receives any bits in \field{enabled_tunnel_types} which are not set in \field{supported_tunnel_types},
it SHOULD respond to the VIRTIO_NET_CTRL_HASH_TUNNEL_SET command with VIRTIO_NET_ERR.

If the driver sets \field{enabled_tunnel_types} to 0 through VIRTIO_NET_CTRL_HASH_TUNNEL_SET or upon the device reset,
the device MUST disable the inner header hash for all encapsulation types.

\drivernormative{\subparagraph}{Inner Header Hash}{Device Types / Network Device / Device Operation / Control Virtqueue / Inner Header Hash}

The driver MUST have negotiated the VIRTIO_NET_F_HASH_TUNNEL feature when issuing the VIRTIO_NET_CTRL_HASH_TUNNEL_SET command.

The driver MUST NOT set any bits in \field{enabled_tunnel_types} which are not set in \field{supported_tunnel_types}.

The driver MUST ignore bits in \field{supported_tunnel_types} which are not documented in this specification.

\paragraph{Hash reporting for incoming packets}
\label{sec:Device Types / Network Device / Device Operation / Processing of Incoming Packets / Hash reporting for incoming packets}

If VIRTIO_NET_F_HASH_REPORT was negotiated and
 the device has calculated the hash for the packet, the device fills \field{hash_report} with the report type of calculated hash
and \field{hash_value} with the value of calculated hash.

If VIRTIO_NET_F_HASH_REPORT was negotiated but due to any reason the
hash was not calculated, the device sets \field{hash_report} to VIRTIO_NET_HASH_REPORT_NONE.

Possible values that the device can report in \field{hash_report} are defined below.
They correspond to supported hash types defined in
\ref{sec:Device Types / Network Device / Device Operation / Processing of Incoming Packets / Hash calculation for incoming packets / Supported/enabled hash types}
as follows:

VIRTIO_NET_HASH_TYPE_XXX = 1 << (VIRTIO_NET_HASH_REPORT_XXX - 1)

\begin{lstlisting}
#define VIRTIO_NET_HASH_REPORT_NONE            0
#define VIRTIO_NET_HASH_REPORT_IPv4            1
#define VIRTIO_NET_HASH_REPORT_TCPv4           2
#define VIRTIO_NET_HASH_REPORT_UDPv4           3
#define VIRTIO_NET_HASH_REPORT_IPv6            4
#define VIRTIO_NET_HASH_REPORT_TCPv6           5
#define VIRTIO_NET_HASH_REPORT_UDPv6           6
#define VIRTIO_NET_HASH_REPORT_IPv6_EX         7
#define VIRTIO_NET_HASH_REPORT_TCPv6_EX        8
#define VIRTIO_NET_HASH_REPORT_UDPv6_EX        9
\end{lstlisting}

\subsubsection{Control Virtqueue}\label{sec:Device Types / Network Device / Device Operation / Control Virtqueue}

The driver uses the control virtqueue (if VIRTIO_NET_F_CTRL_VQ is
negotiated) to send commands to manipulate various features of
the device which would not easily map into the configuration
space.

All commands are of the following form:

\begin{lstlisting}
struct virtio_net_ctrl {
        u8 class;
        u8 command;
        u8 command-specific-data[];
        u8 ack;
        u8 command-specific-result[];
};

/* ack values */
#define VIRTIO_NET_OK     0
#define VIRTIO_NET_ERR    1
\end{lstlisting}

The \field{class}, \field{command} and command-specific-data are set by the
driver, and the device sets the \field{ack} byte and optionally
\field{command-specific-result}. There is little the driver can
do except issue a diagnostic if \field{ack} is not VIRTIO_NET_OK.

The command VIRTIO_NET_CTRL_STATS_QUERY and VIRTIO_NET_CTRL_STATS_GET contain
\field{command-specific-result}.

\paragraph{Packet Receive Filtering}\label{sec:Device Types / Network Device / Device Operation / Control Virtqueue / Packet Receive Filtering}
\label{sec:Device Types / Network Device / Device Operation / Control Virtqueue / Setting Promiscuous Mode}%old label for latexdiff

If the VIRTIO_NET_F_CTRL_RX and VIRTIO_NET_F_CTRL_RX_EXTRA
features are negotiated, the driver can send control commands for
promiscuous mode, multicast, unicast and broadcast receiving.

\begin{note}
In general, these commands are best-effort: unwanted
packets could still arrive.
\end{note}

\begin{lstlisting}
#define VIRTIO_NET_CTRL_RX    0
 #define VIRTIO_NET_CTRL_RX_PROMISC      0
 #define VIRTIO_NET_CTRL_RX_ALLMULTI     1
 #define VIRTIO_NET_CTRL_RX_ALLUNI       2
 #define VIRTIO_NET_CTRL_RX_NOMULTI      3
 #define VIRTIO_NET_CTRL_RX_NOUNI        4
 #define VIRTIO_NET_CTRL_RX_NOBCAST      5
\end{lstlisting}


\devicenormative{\subparagraph}{Packet Receive Filtering}{Device Types / Network Device / Device Operation / Control Virtqueue / Packet Receive Filtering}

If the VIRTIO_NET_F_CTRL_RX feature has been negotiated,
the device MUST support the following VIRTIO_NET_CTRL_RX class
commands:
\begin{itemize}
\item VIRTIO_NET_CTRL_RX_PROMISC turns promiscuous mode on and
off. The command-specific-data is one byte containing 0 (off) or
1 (on). If promiscuous mode is on, the device SHOULD receive all
incoming packets.
This SHOULD take effect even if one of the other modes set by
a VIRTIO_NET_CTRL_RX class command is on.
\item VIRTIO_NET_CTRL_RX_ALLMULTI turns all-multicast receive on and
off. The command-specific-data is one byte containing 0 (off) or
1 (on). When all-multicast receive is on the device SHOULD allow
all incoming multicast packets.
\end{itemize}

If the VIRTIO_NET_F_CTRL_RX_EXTRA feature has been negotiated,
the device MUST support the following VIRTIO_NET_CTRL_RX class
commands:
\begin{itemize}
\item VIRTIO_NET_CTRL_RX_ALLUNI turns all-unicast receive on and
off. The command-specific-data is one byte containing 0 (off) or
1 (on). When all-unicast receive is on the device SHOULD allow
all incoming unicast packets.
\item VIRTIO_NET_CTRL_RX_NOMULTI suppresses multicast receive.
The command-specific-data is one byte containing 0 (multicast
receive allowed) or 1 (multicast receive suppressed).
When multicast receive is suppressed, the device SHOULD NOT
send multicast packets to the driver.
This SHOULD take effect even if VIRTIO_NET_CTRL_RX_ALLMULTI is on.
This filter SHOULD NOT apply to broadcast packets.
\item VIRTIO_NET_CTRL_RX_NOUNI suppresses unicast receive.
The command-specific-data is one byte containing 0 (unicast
receive allowed) or 1 (unicast receive suppressed).
When unicast receive is suppressed, the device SHOULD NOT
send unicast packets to the driver.
This SHOULD take effect even if VIRTIO_NET_CTRL_RX_ALLUNI is on.
\item VIRTIO_NET_CTRL_RX_NOBCAST suppresses broadcast receive.
The command-specific-data is one byte containing 0 (broadcast
receive allowed) or 1 (broadcast receive suppressed).
When broadcast receive is suppressed, the device SHOULD NOT
send broadcast packets to the driver.
This SHOULD take effect even if VIRTIO_NET_CTRL_RX_ALLMULTI is on.
\end{itemize}

\drivernormative{\subparagraph}{Packet Receive Filtering}{Device Types / Network Device / Device Operation / Control Virtqueue / Packet Receive Filtering}

If the VIRTIO_NET_F_CTRL_RX feature has not been negotiated,
the driver MUST NOT issue commands VIRTIO_NET_CTRL_RX_PROMISC or
VIRTIO_NET_CTRL_RX_ALLMULTI.

If the VIRTIO_NET_F_CTRL_RX_EXTRA feature has not been negotiated,
the driver MUST NOT issue commands
 VIRTIO_NET_CTRL_RX_ALLUNI,
 VIRTIO_NET_CTRL_RX_NOMULTI,
 VIRTIO_NET_CTRL_RX_NOUNI or
 VIRTIO_NET_CTRL_RX_NOBCAST.

\paragraph{Setting MAC Address Filtering}\label{sec:Device Types / Network Device / Device Operation / Control Virtqueue / Setting MAC Address Filtering}

If the VIRTIO_NET_F_CTRL_RX feature is negotiated, the driver can
send control commands for MAC address filtering.

\begin{lstlisting}
struct virtio_net_ctrl_mac {
        le32 entries;
        u8 macs[entries][6];
};

#define VIRTIO_NET_CTRL_MAC    1
 #define VIRTIO_NET_CTRL_MAC_TABLE_SET        0
 #define VIRTIO_NET_CTRL_MAC_ADDR_SET         1
\end{lstlisting}

The device can filter incoming packets by any number of destination
MAC addresses\footnote{Since there are no guarantees, it can use a hash filter or
silently switch to allmulti or promiscuous mode if it is given too
many addresses.
}. This table is set using the class
VIRTIO_NET_CTRL_MAC and the command VIRTIO_NET_CTRL_MAC_TABLE_SET. The
command-specific-data is two variable length tables of 6-byte MAC
addresses (as described in struct virtio_net_ctrl_mac). The first table contains unicast addresses, and the second
contains multicast addresses.

The VIRTIO_NET_CTRL_MAC_ADDR_SET command is used to set the
default MAC address which rx filtering
accepts (and if VIRTIO_NET_F_MAC has been negotiated,
this will be reflected in \field{mac} in config space).

The command-specific-data for VIRTIO_NET_CTRL_MAC_ADDR_SET is
the 6-byte MAC address.

\devicenormative{\subparagraph}{Setting MAC Address Filtering}{Device Types / Network Device / Device Operation / Control Virtqueue / Setting MAC Address Filtering}

The device MUST have an empty MAC filtering table on reset.

The device MUST update the MAC filtering table before it consumes
the VIRTIO_NET_CTRL_MAC_TABLE_SET command.

The device MUST update \field{mac} in config space before it consumes
the VIRTIO_NET_CTRL_MAC_ADDR_SET command, if VIRTIO_NET_F_MAC has
been negotiated.

The device SHOULD drop incoming packets which have a destination MAC which
matches neither the \field{mac} (or that set with VIRTIO_NET_CTRL_MAC_ADDR_SET)
nor the MAC filtering table.

\drivernormative{\subparagraph}{Setting MAC Address Filtering}{Device Types / Network Device / Device Operation / Control Virtqueue / Setting MAC Address Filtering}

If VIRTIO_NET_F_CTRL_RX has not been negotiated,
the driver MUST NOT issue VIRTIO_NET_CTRL_MAC class commands.

If VIRTIO_NET_F_CTRL_RX has been negotiated,
the driver SHOULD issue VIRTIO_NET_CTRL_MAC_ADDR_SET
to set the default mac if it is different from \field{mac}.

The driver MUST follow the VIRTIO_NET_CTRL_MAC_TABLE_SET command
by a le32 number, followed by that number of non-multicast
MAC addresses, followed by another le32 number, followed by
that number of multicast addresses.  Either number MAY be 0.

\subparagraph{Legacy Interface: Setting MAC Address Filtering}\label{sec:Device Types / Network Device / Device Operation / Control Virtqueue / Setting MAC Address Filtering / Legacy Interface: Setting MAC Address Filtering}
When using the legacy interface, transitional devices and drivers
MUST format \field{entries} in struct virtio_net_ctrl_mac
according to the native endian of the guest rather than
(necessarily when not using the legacy interface) little-endian.

Legacy drivers that didn't negotiate VIRTIO_NET_F_CTRL_MAC_ADDR
changed \field{mac} in config space when NIC is accepting
incoming packets. These drivers always wrote the mac value from
first to last byte, therefore after detecting such drivers,
a transitional device MAY defer MAC update, or MAY defer
processing incoming packets until driver writes the last byte
of \field{mac} in the config space.

\paragraph{VLAN Filtering}\label{sec:Device Types / Network Device / Device Operation / Control Virtqueue / VLAN Filtering}

If the driver negotiates the VIRTIO_NET_F_CTRL_VLAN feature, it
can control a VLAN filter table in the device. The VLAN filter
table applies only to VLAN tagged packets.

When VIRTIO_NET_F_CTRL_VLAN is negotiated, the device starts with
an empty VLAN filter table.

\begin{note}
Similar to the MAC address based filtering, the VLAN filtering
is also best-effort: unwanted packets could still arrive.
\end{note}

\begin{lstlisting}
#define VIRTIO_NET_CTRL_VLAN       2
 #define VIRTIO_NET_CTRL_VLAN_ADD             0
 #define VIRTIO_NET_CTRL_VLAN_DEL             1
\end{lstlisting}

Both the VIRTIO_NET_CTRL_VLAN_ADD and VIRTIO_NET_CTRL_VLAN_DEL
command take a little-endian 16-bit VLAN id as the command-specific-data.

VIRTIO_NET_CTRL_VLAN_ADD command adds the specified VLAN to the
VLAN filter table.

VIRTIO_NET_CTRL_VLAN_DEL command removes the specified VLAN from
the VLAN filter table.

\devicenormative{\subparagraph}{VLAN Filtering}{Device Types / Network Device / Device Operation / Control Virtqueue / VLAN Filtering}

When VIRTIO_NET_F_CTRL_VLAN is not negotiated, the device MUST
accept all VLAN tagged packets.

When VIRTIO_NET_F_CTRL_VLAN is negotiated, the device MUST
accept all VLAN tagged packets whose VLAN tag is present in
the VLAN filter table and SHOULD drop all VLAN tagged packets
whose VLAN tag is absent in the VLAN filter table.

\subparagraph{Legacy Interface: VLAN Filtering}\label{sec:Device Types / Network Device / Device Operation / Control Virtqueue / VLAN Filtering / Legacy Interface: VLAN Filtering}
When using the legacy interface, transitional devices and drivers
MUST format the VLAN id
according to the native endian of the guest rather than
(necessarily when not using the legacy interface) little-endian.

\paragraph{Gratuitous Packet Sending}\label{sec:Device Types / Network Device / Device Operation / Control Virtqueue / Gratuitous Packet Sending}

If the driver negotiates the VIRTIO_NET_F_GUEST_ANNOUNCE (depends
on VIRTIO_NET_F_CTRL_VQ), the device can ask the driver to send gratuitous
packets; this is usually done after the guest has been physically
migrated, and needs to announce its presence on the new network
links. (As hypervisor does not have the knowledge of guest
network configuration (eg. tagged vlan) it is simplest to prod
the guest in this way).

\begin{lstlisting}
#define VIRTIO_NET_CTRL_ANNOUNCE       3
 #define VIRTIO_NET_CTRL_ANNOUNCE_ACK             0
\end{lstlisting}

The driver checks VIRTIO_NET_S_ANNOUNCE bit in the device configuration \field{status} field
when it notices the changes of device configuration. The
command VIRTIO_NET_CTRL_ANNOUNCE_ACK is used to indicate that
driver has received the notification and device clears the
VIRTIO_NET_S_ANNOUNCE bit in \field{status}.

Processing this notification involves:

\begin{enumerate}
\item Sending the gratuitous packets (eg. ARP) or marking there are pending
  gratuitous packets to be sent and letting deferred routine to
  send them.

\item Sending VIRTIO_NET_CTRL_ANNOUNCE_ACK command through control
  vq.
\end{enumerate}

\drivernormative{\subparagraph}{Gratuitous Packet Sending}{Device Types / Network Device / Device Operation / Control Virtqueue / Gratuitous Packet Sending}

If the driver negotiates VIRTIO_NET_F_GUEST_ANNOUNCE, it SHOULD notify
network peers of its new location after it sees the VIRTIO_NET_S_ANNOUNCE bit
in \field{status}.  The driver MUST send a command on the command queue
with class VIRTIO_NET_CTRL_ANNOUNCE and command VIRTIO_NET_CTRL_ANNOUNCE_ACK.

\devicenormative{\subparagraph}{Gratuitous Packet Sending}{Device Types / Network Device / Device Operation / Control Virtqueue / Gratuitous Packet Sending}

If VIRTIO_NET_F_GUEST_ANNOUNCE is negotiated, the device MUST clear the
VIRTIO_NET_S_ANNOUNCE bit in \field{status} upon receipt of a command buffer
with class VIRTIO_NET_CTRL_ANNOUNCE and command VIRTIO_NET_CTRL_ANNOUNCE_ACK
before marking the buffer as used.

\paragraph{Device operation in multiqueue mode}\label{sec:Device Types / Network Device / Device Operation / Control Virtqueue / Device operation in multiqueue mode}

This specification defines the following modes that a device MAY implement for operation with multiple transmit/receive virtqueues:
\begin{itemize}
\item Automatic receive steering as defined in \ref{sec:Device Types / Network Device / Device Operation / Control Virtqueue / Automatic receive steering in multiqueue mode}.
 If a device supports this mode, it offers the VIRTIO_NET_F_MQ feature bit.
\item Receive-side scaling as defined in \ref{devicenormative:Device Types / Network Device / Device Operation / Control Virtqueue / Receive-side scaling (RSS) / RSS processing}.
 If a device supports this mode, it offers the VIRTIO_NET_F_RSS feature bit.
\end{itemize}

A device MAY support one of these features or both. The driver MAY negotiate any set of these features that the device supports.

Multiqueue is disabled by default.

The driver enables multiqueue by sending a command using \field{class} VIRTIO_NET_CTRL_MQ. The \field{command} selects the mode of multiqueue operation, as follows:
\begin{lstlisting}
#define VIRTIO_NET_CTRL_MQ    4
 #define VIRTIO_NET_CTRL_MQ_VQ_PAIRS_SET        0 (for automatic receive steering)
 #define VIRTIO_NET_CTRL_MQ_RSS_CONFIG          1 (for configurable receive steering)
 #define VIRTIO_NET_CTRL_MQ_HASH_CONFIG         2 (for configurable hash calculation)
\end{lstlisting}

If more than one multiqueue mode is negotiated, the resulting device configuration is defined by the last command sent by the driver.

\paragraph{Automatic receive steering in multiqueue mode}\label{sec:Device Types / Network Device / Device Operation / Control Virtqueue / Automatic receive steering in multiqueue mode}

If the driver negotiates the VIRTIO_NET_F_MQ feature bit (depends on VIRTIO_NET_F_CTRL_VQ), it MAY transmit outgoing packets on one
of the multiple transmitq1\ldots transmitqN and ask the device to
queue incoming packets into one of the multiple receiveq1\ldots receiveqN
depending on the packet flow.

The driver enables multiqueue by
sending the VIRTIO_NET_CTRL_MQ_VQ_PAIRS_SET command, specifying
the number of the transmit and receive queues to be used up to
\field{max_virtqueue_pairs}; subsequently,
transmitq1\ldots transmitqn and receiveq1\ldots receiveqn where
n=\field{virtqueue_pairs} MAY be used.
\begin{lstlisting}
struct virtio_net_ctrl_mq_pairs_set {
       le16 virtqueue_pairs;
};
#define VIRTIO_NET_CTRL_MQ_VQ_PAIRS_MIN        1
#define VIRTIO_NET_CTRL_MQ_VQ_PAIRS_MAX        0x8000

\end{lstlisting}

When multiqueue is enabled by VIRTIO_NET_CTRL_MQ_VQ_PAIRS_SET command, the device MUST use automatic receive steering
based on packet flow. Programming of the receive steering
classificator is implicit. After the driver transmitted a packet of a
flow on transmitqX, the device SHOULD cause incoming packets for that flow to
be steered to receiveqX. For uni-directional protocols, or where
no packets have been transmitted yet, the device MAY steer a packet
to a random queue out of the specified receiveq1\ldots receiveqn.

Multiqueue is disabled by VIRTIO_NET_CTRL_MQ_VQ_PAIRS_SET with \field{virtqueue_pairs} to 1 (this is
the default) and waiting for the device to use the command buffer.

\drivernormative{\subparagraph}{Automatic receive steering in multiqueue mode}{Device Types / Network Device / Device Operation / Control Virtqueue / Automatic receive steering in multiqueue mode}

The driver MUST configure the virtqueues before enabling them with the
VIRTIO_NET_CTRL_MQ_VQ_PAIRS_SET command.

The driver MUST NOT request a \field{virtqueue_pairs} of 0 or
greater than \field{max_virtqueue_pairs} in the device configuration space.

The driver MUST queue packets only on any transmitq1 before the
VIRTIO_NET_CTRL_MQ_VQ_PAIRS_SET command.

The driver MUST NOT queue packets on transmit queues greater than
\field{virtqueue_pairs} once it has placed the VIRTIO_NET_CTRL_MQ_VQ_PAIRS_SET command in the available ring.

\devicenormative{\subparagraph}{Automatic receive steering in multiqueue mode}{Device Types / Network Device / Device Operation / Control Virtqueue / Automatic receive steering in multiqueue mode}

After initialization of reset, the device MUST queue packets only on receiveq1.

The device MUST NOT queue packets on receive queues greater than
\field{virtqueue_pairs} once it has placed the
VIRTIO_NET_CTRL_MQ_VQ_PAIRS_SET command in a used buffer.

If the destination receive queue is being reset (See \ref{sec:Basic Facilities of a Virtio Device / Virtqueues / Virtqueue Reset}),
the device SHOULD re-select another random queue. If all receive queues are
being reset, the device MUST drop the packet.

\subparagraph{Legacy Interface: Automatic receive steering in multiqueue mode}\label{sec:Device Types / Network Device / Device Operation / Control Virtqueue / Automatic receive steering in multiqueue mode / Legacy Interface: Automatic receive steering in multiqueue mode}
When using the legacy interface, transitional devices and drivers
MUST format \field{virtqueue_pairs}
according to the native endian of the guest rather than
(necessarily when not using the legacy interface) little-endian.

\subparagraph{Hash calculation}\label{sec:Device Types / Network Device / Device Operation / Control Virtqueue / Automatic receive steering in multiqueue mode / Hash calculation}
If VIRTIO_NET_F_HASH_REPORT was negotiated and the device uses automatic receive steering,
the device MUST support a command to configure hash calculation parameters.

The driver provides parameters for hash calculation as follows:

\field{class} VIRTIO_NET_CTRL_MQ, \field{command} VIRTIO_NET_CTRL_MQ_HASH_CONFIG.

The \field{command-specific-data} has following format:
\begin{lstlisting}
struct virtio_net_hash_config {
    le32 hash_types;
    le16 reserved[4];
    u8 hash_key_length;
    u8 hash_key_data[hash_key_length];
};
\end{lstlisting}
Field \field{hash_types} contains a bitmask of allowed hash types as
defined in
\ref{sec:Device Types / Network Device / Device Operation / Processing of Incoming Packets / Hash calculation for incoming packets / Supported/enabled hash types}.
Initially the device has all hash types disabled and reports only VIRTIO_NET_HASH_REPORT_NONE.

Field \field{reserved} MUST contain zeroes. It is defined to make the structure to match the layout of virtio_net_rss_config structure,
defined in \ref{sec:Device Types / Network Device / Device Operation / Control Virtqueue / Receive-side scaling (RSS)}.

Fields \field{hash_key_length} and \field{hash_key_data} define the key to be used in hash calculation.

\paragraph{Receive-side scaling (RSS)}\label{sec:Device Types / Network Device / Device Operation / Control Virtqueue / Receive-side scaling (RSS)}
A device offers the feature VIRTIO_NET_F_RSS if it supports RSS receive steering with Toeplitz hash calculation and configurable parameters.

A driver queries RSS capabilities of the device by reading device configuration as defined in \ref{sec:Device Types / Network Device / Device configuration layout}

\subparagraph{Setting RSS parameters}\label{sec:Device Types / Network Device / Device Operation / Control Virtqueue / Receive-side scaling (RSS) / Setting RSS parameters}

Driver sends a VIRTIO_NET_CTRL_MQ_RSS_CONFIG command using the following format for \field{command-specific-data}:
\begin{lstlisting}
struct rss_rq_id {
   le16 vq_index_1_16: 15; /* Bits 1 to 16 of the virtqueue index */
   le16 reserved: 1; /* Set to zero */
};

struct virtio_net_rss_config {
    le32 hash_types;
    le16 indirection_table_mask;
    struct rss_rq_id unclassified_queue;
    struct rss_rq_id indirection_table[indirection_table_length];
    le16 max_tx_vq;
    u8 hash_key_length;
    u8 hash_key_data[hash_key_length];
};
\end{lstlisting}
Field \field{hash_types} contains a bitmask of allowed hash types as
defined in
\ref{sec:Device Types / Network Device / Device Operation / Processing of Incoming Packets / Hash calculation for incoming packets / Supported/enabled hash types}.

Field \field{indirection_table_mask} is a mask to be applied to
the calculated hash to produce an index in the
\field{indirection_table} array.
Number of entries in \field{indirection_table} is (\field{indirection_table_mask} + 1).

\field{rss_rq_id} is a receive virtqueue id. \field{vq_index_1_16}
consists of bits 1 to 16 of a virtqueue index. For example, a
\field{vq_index_1_16} value of 3 corresponds to virtqueue index 6,
which maps to receiveq4.

Field \field{unclassified_queue} specifies the receive virtqueue id in which to
place unclassified packets.

Field \field{indirection_table} is an array of receive virtqueues ids.

A driver sets \field{max_tx_vq} to inform a device how many transmit virtqueues it may use (transmitq1\ldots transmitq \field{max_tx_vq}).

Fields \field{hash_key_length} and \field{hash_key_data} define the key to be used in hash calculation.

\drivernormative{\subparagraph}{Setting RSS parameters}{Device Types / Network Device / Device Operation / Control Virtqueue / Receive-side scaling (RSS) }

A driver MUST NOT send the VIRTIO_NET_CTRL_MQ_RSS_CONFIG command if the feature VIRTIO_NET_F_RSS has not been negotiated.

A driver MUST fill the \field{indirection_table} array only with
enabled receive virtqueues ids.

The number of entries in \field{indirection_table} (\field{indirection_table_mask} + 1) MUST be a power of two.

A driver MUST use \field{indirection_table_mask} values that are less than \field{rss_max_indirection_table_length} reported by a device.

A driver MUST NOT set any VIRTIO_NET_HASH_TYPE_ flags that are not supported by a device.

\devicenormative{\subparagraph}{RSS processing}{Device Types / Network Device / Device Operation / Control Virtqueue / Receive-side scaling (RSS) / RSS processing}
The device MUST determine the destination queue for a network packet as follows:
\begin{itemize}
\item Calculate the hash of the packet as defined in \ref{sec:Device Types / Network Device / Device Operation / Processing of Incoming Packets / Hash calculation for incoming packets}.
\item If the device did not calculate the hash for the specific packet, the device directs the packet to the receiveq specified by \field{unclassified_queue} of virtio_net_rss_config structure.
\item Apply \field{indirection_table_mask} to the calculated hash
and use the result as the index in the indirection table to get
the destination receive virtqueue id.
\item If the destination receive queue is being reset (See \ref{sec:Basic Facilities of a Virtio Device / Virtqueues / Virtqueue Reset}), the device MUST drop the packet.
\end{itemize}

\paragraph{RSS Context}\label{sec:Device Types / Network Device / Device Operation / Control Virtqueue / RSS Context}

An RSS context consists of configurable parameters specified by \ref{sec:Device Types / Network Device
/ Device Operation / Control Virtqueue / Receive-side scaling (RSS)}.

The RSS configuration supported by VIRTIO_NET_F_RSS is considered the default RSS configuration.

The device offers the feature VIRTIO_NET_F_RSS_CONTEXT if it supports one or multiple RSS contexts
(excluding the default RSS configuration) and configurable parameters.

\subparagraph{Querying RSS Context Capability}\label{sec:Device Types / Network Device / Device Operation / Control Virtqueue / RSS Context / Querying RSS Context Capability}

\begin{lstlisting}
#define VIRTNET_RSS_CTX_CTRL 9
 #define VIRTNET_RSS_CTX_CTRL_CAP_GET  0
 #define VIRTNET_RSS_CTX_CTRL_ADD      1
 #define VIRTNET_RSS_CTX_CTRL_MOD      2
 #define VIRTNET_RSS_CTX_CTRL_DEL      3

struct virtnet_rss_ctx_cap {
    le16 max_rss_contexts;
}
\end{lstlisting}

Field \field{max_rss_contexts} specifies the maximum number of RSS contexts \ref{sec:Device Types / Network Device /
Device Operation / Control Virtqueue / RSS Context} supported by the device.

The driver queries the RSS context capability of the device by sending the command VIRTNET_RSS_CTX_CTRL_CAP_GET
with the structure virtnet_rss_ctx_cap.

For the command VIRTNET_RSS_CTX_CTRL_CAP_GET, the structure virtnet_rss_ctx_cap is write-only for the device.

\subparagraph{Setting RSS Context Parameters}\label{sec:Device Types / Network Device / Device Operation / Control Virtqueue / RSS Context / Setting RSS Context Parameters}

\begin{lstlisting}
struct virtnet_rss_ctx_add_modify {
    le16 rss_ctx_id;
    u8 reserved[6];
    struct virtio_net_rss_config rss;
};

struct virtnet_rss_ctx_del {
    le16 rss_ctx_id;
};
\end{lstlisting}

RSS context parameters:
\begin{itemize}
\item  \field{rss_ctx_id}: ID of the specific RSS context.
\item  \field{rss}: RSS context parameters of the specific RSS context whose id is \field{rss_ctx_id}.
\end{itemize}

\field{reserved} is reserved and it is ignored by the device.

If the feature VIRTIO_NET_F_RSS_CONTEXT has been negotiated, the driver can send the following
VIRTNET_RSS_CTX_CTRL class commands:
\begin{enumerate}
\item VIRTNET_RSS_CTX_CTRL_ADD: use the structure virtnet_rss_ctx_add_modify to
       add an RSS context configured as \field{rss} and id as \field{rss_ctx_id} for the device.
\item VIRTNET_RSS_CTX_CTRL_MOD: use the structure virtnet_rss_ctx_add_modify to
       configure parameters of the RSS context whose id is \field{rss_ctx_id} as \field{rss} for the device.
\item VIRTNET_RSS_CTX_CTRL_DEL: use the structure virtnet_rss_ctx_del to delete
       the RSS context whose id is \field{rss_ctx_id} for the device.
\end{enumerate}

For commands VIRTNET_RSS_CTX_CTRL_ADD and VIRTNET_RSS_CTX_CTRL_MOD, the structure virtnet_rss_ctx_add_modify is read-only for the device.
For the command VIRTNET_RSS_CTX_CTRL_DEL, the structure virtnet_rss_ctx_del is read-only for the device.

\devicenormative{\subparagraph}{RSS Context}{Device Types / Network Device / Device Operation / Control Virtqueue / RSS Context}

The device MUST set \field{max_rss_contexts} to at least 1 if it offers VIRTIO_NET_F_RSS_CONTEXT.

Upon reset, the device MUST clear all previously configured RSS contexts.

\drivernormative{\subparagraph}{RSS Context}{Device Types / Network Device / Device Operation / Control Virtqueue / RSS Context}

The driver MUST have negotiated the VIRTIO_NET_F_RSS_CONTEXT feature when issuing the VIRTNET_RSS_CTX_CTRL class commands.

The driver MUST set \field{rss_ctx_id} to between 1 and \field{max_rss_contexts} inclusive.

The driver MUST NOT send the command VIRTIO_NET_CTRL_MQ_VQ_PAIRS_SET when the device has successfully configured at least one RSS context.

\paragraph{Offloads State Configuration}\label{sec:Device Types / Network Device / Device Operation / Control Virtqueue / Offloads State Configuration}

If the VIRTIO_NET_F_CTRL_GUEST_OFFLOADS feature is negotiated, the driver can
send control commands for dynamic offloads state configuration.

\subparagraph{Setting Offloads State}\label{sec:Device Types / Network Device / Device Operation / Control Virtqueue / Offloads State Configuration / Setting Offloads State}

To configure the offloads, the following layout structure and
definitions are used:

\begin{lstlisting}
le64 offloads;

#define VIRTIO_NET_F_GUEST_CSUM       1
#define VIRTIO_NET_F_GUEST_TSO4       7
#define VIRTIO_NET_F_GUEST_TSO6       8
#define VIRTIO_NET_F_GUEST_ECN        9
#define VIRTIO_NET_F_GUEST_UFO        10
#define VIRTIO_NET_F_GUEST_UDP_TUNNEL_GSO  46
#define VIRTIO_NET_F_GUEST_UDP_TUNNEL_GSO_CSUM 47
#define VIRTIO_NET_F_GUEST_USO4       54
#define VIRTIO_NET_F_GUEST_USO6       55

#define VIRTIO_NET_CTRL_GUEST_OFFLOADS       5
 #define VIRTIO_NET_CTRL_GUEST_OFFLOADS_SET   0
\end{lstlisting}

The class VIRTIO_NET_CTRL_GUEST_OFFLOADS has one command:
VIRTIO_NET_CTRL_GUEST_OFFLOADS_SET applies the new offloads configuration.

le64 value passed as command data is a bitmask, bits set define
offloads to be enabled, bits cleared - offloads to be disabled.

There is a corresponding device feature for each offload. Upon feature
negotiation corresponding offload gets enabled to preserve backward
compatibility.

\drivernormative{\subparagraph}{Setting Offloads State}{Device Types / Network Device / Device Operation / Control Virtqueue / Offloads State Configuration / Setting Offloads State}

A driver MUST NOT enable an offload for which the appropriate feature
has not been negotiated.

\subparagraph{Legacy Interface: Setting Offloads State}\label{sec:Device Types / Network Device / Device Operation / Control Virtqueue / Offloads State Configuration / Setting Offloads State / Legacy Interface: Setting Offloads State}
When using the legacy interface, transitional devices and drivers
MUST format \field{offloads}
according to the native endian of the guest rather than
(necessarily when not using the legacy interface) little-endian.


\paragraph{Notifications Coalescing}\label{sec:Device Types / Network Device / Device Operation / Control Virtqueue / Notifications Coalescing}

If the VIRTIO_NET_F_NOTF_COAL feature is negotiated, the driver can
send commands VIRTIO_NET_CTRL_NOTF_COAL_TX_SET and VIRTIO_NET_CTRL_NOTF_COAL_RX_SET
for notification coalescing.

If the VIRTIO_NET_F_VQ_NOTF_COAL feature is negotiated, the driver can
send commands VIRTIO_NET_CTRL_NOTF_COAL_VQ_SET and VIRTIO_NET_CTRL_NOTF_COAL_VQ_GET
for virtqueue notification coalescing.

\begin{lstlisting}
struct virtio_net_ctrl_coal {
    le32 max_packets;
    le32 max_usecs;
};

struct virtio_net_ctrl_coal_vq {
    le16 vq_index;
    le16 reserved;
    struct virtio_net_ctrl_coal coal;
};

#define VIRTIO_NET_CTRL_NOTF_COAL 6
 #define VIRTIO_NET_CTRL_NOTF_COAL_TX_SET  0
 #define VIRTIO_NET_CTRL_NOTF_COAL_RX_SET 1
 #define VIRTIO_NET_CTRL_NOTF_COAL_VQ_SET 2
 #define VIRTIO_NET_CTRL_NOTF_COAL_VQ_GET 3
\end{lstlisting}

Coalescing parameters:
\begin{itemize}
\item \field{vq_index}: The virtqueue index of an enabled transmit or receive virtqueue.
\item \field{max_usecs} for RX: Maximum number of microseconds to delay a RX notification.
\item \field{max_usecs} for TX: Maximum number of microseconds to delay a TX notification.
\item \field{max_packets} for RX: Maximum number of packets to receive before a RX notification.
\item \field{max_packets} for TX: Maximum number of packets to send before a TX notification.
\end{itemize}

\field{reserved} is reserved and it is ignored by the device.

Read/Write attributes for coalescing parameters:
\begin{itemize}
\item For commands VIRTIO_NET_CTRL_NOTF_COAL_TX_SET and VIRTIO_NET_CTRL_NOTF_COAL_RX_SET, the structure virtio_net_ctrl_coal is write-only for the driver.
\item For the command VIRTIO_NET_CTRL_NOTF_COAL_VQ_SET, the structure virtio_net_ctrl_coal_vq is write-only for the driver.
\item For the command VIRTIO_NET_CTRL_NOTF_COAL_VQ_GET, \field{vq_index} and \field{reserved} are write-only
      for the driver, and the structure virtio_net_ctrl_coal is read-only for the driver.
\end{itemize}

The class VIRTIO_NET_CTRL_NOTF_COAL has the following commands:
\begin{enumerate}
\item VIRTIO_NET_CTRL_NOTF_COAL_TX_SET: use the structure virtio_net_ctrl_coal to set the \field{max_usecs} and \field{max_packets} parameters for all transmit virtqueues.
\item VIRTIO_NET_CTRL_NOTF_COAL_RX_SET: use the structure virtio_net_ctrl_coal to set the \field{max_usecs} and \field{max_packets} parameters for all receive virtqueues.
\item VIRTIO_NET_CTRL_NOTF_COAL_VQ_SET: use the structure virtio_net_ctrl_coal_vq to set the \field{max_usecs} and \field{max_packets} parameters
                                        for an enabled transmit/receive virtqueue whose index is \field{vq_index}.
\item VIRTIO_NET_CTRL_NOTF_COAL_VQ_GET: use the structure virtio_net_ctrl_coal_vq to get the \field{max_usecs} and \field{max_packets} parameters
                                        for an enabled transmit/receive virtqueue whose index is \field{vq_index}.
\end{enumerate}

The device may generate notifications more or less frequently than specified by set commands of the VIRTIO_NET_CTRL_NOTF_COAL class.

If coalescing parameters are being set, the device applies the last coalescing parameters set for a
virtqueue, regardless of the command used to set the parameters. Use the following command sequence
with two pairs of virtqueues as an example:
Each of the following commands sets \field{max_usecs} and \field{max_packets} parameters for virtqueues.
\begin{itemize}
\item Command1: VIRTIO_NET_CTRL_NOTF_COAL_RX_SET sets coalescing parameters for virtqueues having index 0 and index 2. Virtqueues having index 1 and index 3 retain their previous parameters.
\item Command2: VIRTIO_NET_CTRL_NOTF_COAL_VQ_SET with \field{vq_index} = 0 sets coalescing parameters for virtqueue having index 0. Virtqueue having index 2 retains the parameters from command1.
\item Command3: VIRTIO_NET_CTRL_NOTF_COAL_VQ_GET with \field{vq_index} = 0, the device responds with coalescing parameters of vq_index 0 set by command2.
\item Command4: VIRTIO_NET_CTRL_NOTF_COAL_VQ_SET with \field{vq_index} = 1 sets coalescing parameters for virtqueue having index 1. Virtqueue having index 3 retains its previous parameters.
\item Command5: VIRTIO_NET_CTRL_NOTF_COAL_TX_SET sets coalescing parameters for virtqueues having index 1 and index 3, and overrides the parameters set by command4.
\item Command6: VIRTIO_NET_CTRL_NOTF_COAL_VQ_GET with \field{vq_index} = 1, the device responds with coalescing parameters of index 1 set by command5.
\end{itemize}

\subparagraph{Operation}\label{sec:Device Types / Network Device / Device Operation / Control Virtqueue / Notifications Coalescing / Operation}

The device sends a used buffer notification once the notification conditions are met and if the notifications are not suppressed as explained in \ref{sec:Basic Facilities of a Virtio Device / Virtqueues / Used Buffer Notification Suppression}.

When the device has non-zero \field{max_usecs} and non-zero \field{max_packets}, it starts counting microseconds and packets upon receiving/sending a packet.
The device counts packets and microseconds for each receive virtqueue and transmit virtqueue separately.
In this case, the notification conditions are met when \field{max_usecs} microseconds elapse, or upon sending/receiving \field{max_packets} packets, whichever happens first.
Afterwards, the device waits for the next packet and starts counting packets and microseconds again.

When the device has \field{max_usecs} = 0 or \field{max_packets} = 0, the notification conditions are met after every packet received/sent.

\subparagraph{RX Example}\label{sec:Device Types / Network Device / Device Operation / Control Virtqueue / Notifications Coalescing / RX Example}

If, for example:
\begin{itemize}
\item \field{max_usecs} = 10.
\item \field{max_packets} = 15.
\end{itemize}
then each receive virtqueue of a device will operate as follows:
\begin{itemize}
\item The device will count packets received on each virtqueue until it accumulates 15, or until 10 microseconds elapsed since the first one was received.
\item If the notifications are not suppressed by the driver, the device will send an used buffer notification, otherwise, the device will not send an used buffer notification as long as the notifications are suppressed.
\end{itemize}

\subparagraph{TX Example}\label{sec:Device Types / Network Device / Device Operation / Control Virtqueue / Notifications Coalescing / TX Example}

If, for example:
\begin{itemize}
\item \field{max_usecs} = 10.
\item \field{max_packets} = 15.
\end{itemize}
then each transmit virtqueue of a device will operate as follows:
\begin{itemize}
\item The device will count packets sent on each virtqueue until it accumulates 15, or until 10 microseconds elapsed since the first one was sent.
\item If the notifications are not suppressed by the driver, the device will send an used buffer notification, otherwise, the device will not send an used buffer notification as long as the notifications are suppressed.
\end{itemize}

\subparagraph{Notifications When Coalescing Parameters Change}\label{sec:Device Types / Network Device / Device Operation / Control Virtqueue / Notifications Coalescing / Notifications When Coalescing Parameters Change}

When the coalescing parameters of a device change, the device needs to check if the new notification conditions are met and send a used buffer notification if so.

For example, \field{max_packets} = 15 for a device with a single transmit virtqueue: if the device sends 10 packets and afterwards receives a
VIRTIO_NET_CTRL_NOTF_COAL_TX_SET command with \field{max_packets} = 8, then the notification condition is immediately considered to be met;
the device needs to immediately send a used buffer notification, if the notifications are not suppressed by the driver.

\drivernormative{\subparagraph}{Notifications Coalescing}{Device Types / Network Device / Device Operation / Control Virtqueue / Notifications Coalescing}

The driver MUST set \field{vq_index} to the virtqueue index of an enabled transmit or receive virtqueue.

The driver MUST have negotiated the VIRTIO_NET_F_NOTF_COAL feature when issuing commands VIRTIO_NET_CTRL_NOTF_COAL_TX_SET and VIRTIO_NET_CTRL_NOTF_COAL_RX_SET.

The driver MUST have negotiated the VIRTIO_NET_F_VQ_NOTF_COAL feature when issuing commands VIRTIO_NET_CTRL_NOTF_COAL_VQ_SET and VIRTIO_NET_CTRL_NOTF_COAL_VQ_GET.

The driver MUST ignore the values of coalescing parameters received from the VIRTIO_NET_CTRL_NOTF_COAL_VQ_GET command if the device responds with VIRTIO_NET_ERR.

\devicenormative{\subparagraph}{Notifications Coalescing}{Device Types / Network Device / Device Operation / Control Virtqueue / Notifications Coalescing}

The device MUST ignore \field{reserved}.

The device SHOULD respond to VIRTIO_NET_CTRL_NOTF_COAL_TX_SET and VIRTIO_NET_CTRL_NOTF_COAL_RX_SET commands with VIRTIO_NET_ERR if it was not able to change the parameters.

The device MUST respond to the VIRTIO_NET_CTRL_NOTF_COAL_VQ_SET command with VIRTIO_NET_ERR if it was not able to change the parameters.

The device MUST respond to VIRTIO_NET_CTRL_NOTF_COAL_VQ_SET and VIRTIO_NET_CTRL_NOTF_COAL_VQ_GET commands with
VIRTIO_NET_ERR if the designated virtqueue is not an enabled transmit or receive virtqueue.

Upon disabling and re-enabling a transmit virtqueue, the device MUST set the coalescing parameters of the virtqueue
to those configured through the VIRTIO_NET_CTRL_NOTF_COAL_TX_SET command, or, if the driver did not set any TX coalescing parameters, to 0.

Upon disabling and re-enabling a receive virtqueue, the device MUST set the coalescing parameters of the virtqueue
to those configured through the VIRTIO_NET_CTRL_NOTF_COAL_RX_SET command, or, if the driver did not set any RX coalescing parameters, to 0.

The behavior of the device in response to set commands of the VIRTIO_NET_CTRL_NOTF_COAL class is best-effort:
the device MAY generate notifications more or less frequently than specified.

A device SHOULD NOT send used buffer notifications to the driver if the notifications are suppressed, even if the notification conditions are met.

Upon reset, a device MUST initialize all coalescing parameters to 0.

\paragraph{Device Statistics}\label{sec:Device Types / Network Device / Device Operation / Control Virtqueue / Device Statistics}

If the VIRTIO_NET_F_DEVICE_STATS feature is negotiated, the driver can obtain
device statistics from the device by using the following command.

Different types of virtqueues have different statistics. The statistics of the
receiveq are different from those of the transmitq.

The statistics of a certain type of virtqueue are also divided into multiple types
because different types require different features. This enables the expansion
of new statistics.

In one command, the driver can obtain the statistics of one or multiple virtqueues.
Additionally, the driver can obtain multiple type statistics of each virtqueue.

\subparagraph{Query Statistic Capabilities}\label{sec:Device Types / Network Device / Device Operation / Control Virtqueue / Device Statistics / Query Statistic Capabilities}

\begin{lstlisting}
#define VIRTIO_NET_CTRL_STATS         8
#define VIRTIO_NET_CTRL_STATS_QUERY   0
#define VIRTIO_NET_CTRL_STATS_GET     1

struct virtio_net_stats_capabilities {

#define VIRTIO_NET_STATS_TYPE_CVQ       (1 << 32)

#define VIRTIO_NET_STATS_TYPE_RX_BASIC  (1 << 0)
#define VIRTIO_NET_STATS_TYPE_RX_CSUM   (1 << 1)
#define VIRTIO_NET_STATS_TYPE_RX_GSO    (1 << 2)
#define VIRTIO_NET_STATS_TYPE_RX_SPEED  (1 << 3)

#define VIRTIO_NET_STATS_TYPE_TX_BASIC  (1 << 16)
#define VIRTIO_NET_STATS_TYPE_TX_CSUM   (1 << 17)
#define VIRTIO_NET_STATS_TYPE_TX_GSO    (1 << 18)
#define VIRTIO_NET_STATS_TYPE_TX_SPEED  (1 << 19)

    le64 supported_stats_types[1];
}
\end{lstlisting}

To obtain device statistic capability, use the VIRTIO_NET_CTRL_STATS_QUERY
command. When the command completes successfully, \field{command-specific-result}
is in the format of \field{struct virtio_net_stats_capabilities}.

\subparagraph{Get Statistics}\label{sec:Device Types / Network Device / Device Operation / Control Virtqueue / Device Statistics / Get Statistics}

\begin{lstlisting}
struct virtio_net_ctrl_queue_stats {
       struct {
           le16 vq_index;
           le16 reserved[3];
           le64 types_bitmap[1];
       } stats[];
};

struct virtio_net_stats_reply_hdr {
#define VIRTIO_NET_STATS_TYPE_REPLY_CVQ       32

#define VIRTIO_NET_STATS_TYPE_REPLY_RX_BASIC  0
#define VIRTIO_NET_STATS_TYPE_REPLY_RX_CSUM   1
#define VIRTIO_NET_STATS_TYPE_REPLY_RX_GSO    2
#define VIRTIO_NET_STATS_TYPE_REPLY_RX_SPEED  3

#define VIRTIO_NET_STATS_TYPE_REPLY_TX_BASIC  16
#define VIRTIO_NET_STATS_TYPE_REPLY_TX_CSUM   17
#define VIRTIO_NET_STATS_TYPE_REPLY_TX_GSO    18
#define VIRTIO_NET_STATS_TYPE_REPLY_TX_SPEED  19
    u8 type;
    u8 reserved;
    le16 vq_index;
    le16 reserved1;
    le16 size;
}
\end{lstlisting}

To obtain device statistics, use the VIRTIO_NET_CTRL_STATS_GET command with the
\field{command-specific-data} which is in the format of
\field{struct virtio_net_ctrl_queue_stats}. When the command completes
successfully, \field{command-specific-result} contains multiple statistic
results, each statistic result has the \field{struct virtio_net_stats_reply_hdr}
as the header.

The fields of the \field{struct virtio_net_ctrl_queue_stats}:
\begin{description}
    \item [vq_index]
        The index of the virtqueue to obtain the statistics.

    \item [types_bitmap]
        This is a bitmask of the types of statistics to be obtained. Therefore, a
        \field{stats} inside \field{struct virtio_net_ctrl_queue_stats} may
        indicate multiple statistic replies for the virtqueue.
\end{description}

The fields of the \field{struct virtio_net_stats_reply_hdr}:
\begin{description}
    \item [type]
        The type of the reply statistic.

    \item [vq_index]
        The virtqueue index of the reply statistic.

    \item [size]
        The number of bytes for the statistics entry including size of \field{struct virtio_net_stats_reply_hdr}.

\end{description}

\subparagraph{Controlq Statistics}\label{sec:Device Types / Network Device / Device Operation / Control Virtqueue / Device Statistics / Controlq Statistics}

The structure corresponding to the controlq statistics is
\field{struct virtio_net_stats_cvq}. The corresponding type is
VIRTIO_NET_STATS_TYPE_CVQ. This is for the controlq.

\begin{lstlisting}
struct virtio_net_stats_cvq {
    struct virtio_net_stats_reply_hdr hdr;

    le64 command_num;
    le64 ok_num;
};
\end{lstlisting}

\begin{description}
    \item [command_num]
        The number of commands received by the device including the current command.

    \item [ok_num]
        The number of commands completed successfully by the device including the current command.
\end{description}


\subparagraph{Receiveq Basic Statistics}\label{sec:Device Types / Network Device / Device Operation / Control Virtqueue / Device Statistics / Receiveq Basic Statistics}

The structure corresponding to the receiveq basic statistics is
\field{struct virtio_net_stats_rx_basic}. The corresponding type is
VIRTIO_NET_STATS_TYPE_RX_BASIC. This is for the receiveq.

Receiveq basic statistics do not require any feature. As long as the device supports
VIRTIO_NET_F_DEVICE_STATS, the following are the receiveq basic statistics.

\begin{lstlisting}
struct virtio_net_stats_rx_basic {
    struct virtio_net_stats_reply_hdr hdr;

    le64 rx_notifications;

    le64 rx_packets;
    le64 rx_bytes;

    le64 rx_interrupts;

    le64 rx_drops;
    le64 rx_drop_overruns;
};
\end{lstlisting}

The packets described below were all presented on the specified virtqueue.
\begin{description}
    \item [rx_notifications]
        The number of driver notifications received by the device for this
        receiveq.

    \item [rx_packets]
        This is the number of packets passed to the driver by the device.

    \item [rx_bytes]
        This is the bytes of packets passed to the driver by the device.

    \item [rx_interrupts]
        The number of interrupts generated by the device for this receiveq.

    \item [rx_drops]
        This is the number of packets dropped by the device. The count includes
        all types of packets dropped by the device.

    \item [rx_drop_overruns]
        This is the number of packets dropped by the device when no more
        descriptors were available.

\end{description}

\subparagraph{Transmitq Basic Statistics}\label{sec:Device Types / Network Device / Device Operation / Control Virtqueue / Device Statistics / Transmitq Basic Statistics}

The structure corresponding to the transmitq basic statistics is
\field{struct virtio_net_stats_tx_basic}. The corresponding type is
VIRTIO_NET_STATS_TYPE_TX_BASIC. This is for the transmitq.

Transmitq basic statistics do not require any feature. As long as the device supports
VIRTIO_NET_F_DEVICE_STATS, the following are the transmitq basic statistics.

\begin{lstlisting}
struct virtio_net_stats_tx_basic {
    struct virtio_net_stats_reply_hdr hdr;

    le64 tx_notifications;

    le64 tx_packets;
    le64 tx_bytes;

    le64 tx_interrupts;

    le64 tx_drops;
    le64 tx_drop_malformed;
};
\end{lstlisting}

The packets described below are all for a specific virtqueue.
\begin{description}
    \item [tx_notifications]
        The number of driver notifications received by the device for this
        transmitq.

    \item [tx_packets]
        This is the number of packets sent by the device (not the packets
        got from the driver).

    \item [tx_bytes]
        This is the number of bytes sent by the device for all the sent packets
        (not the bytes sent got from the driver).

    \item [tx_interrupts]
        The number of interrupts generated by the device for this transmitq.

    \item [tx_drops]
        The number of packets dropped by the device. The count includes all
        types of packets dropped by the device.

    \item [tx_drop_malformed]
        The number of packets dropped by the device, when the descriptors are
        malformed. For example, the buffer is too short.
\end{description}

\subparagraph{Receiveq CSUM Statistics}\label{sec:Device Types / Network Device / Device Operation / Control Virtqueue / Device Statistics / Receiveq CSUM Statistics}

The structure corresponding to the receiveq checksum statistics is
\field{struct virtio_net_stats_rx_csum}. The corresponding type is
VIRTIO_NET_STATS_TYPE_RX_CSUM. This is for the receiveq.

Only after the VIRTIO_NET_F_GUEST_CSUM is negotiated, the receiveq checksum
statistics can be obtained.

\begin{lstlisting}
struct virtio_net_stats_rx_csum {
    struct virtio_net_stats_reply_hdr hdr;

    le64 rx_csum_valid;
    le64 rx_needs_csum;
    le64 rx_csum_none;
    le64 rx_csum_bad;
};
\end{lstlisting}

The packets described below were all presented on the specified virtqueue.
\begin{description}
    \item [rx_csum_valid]
        The number of packets with VIRTIO_NET_HDR_F_DATA_VALID.

    \item [rx_needs_csum]
        The number of packets with VIRTIO_NET_HDR_F_NEEDS_CSUM.

    \item [rx_csum_none]
        The number of packets without hardware checksum. The packet here refers
        to the non-TCP/UDP packet that the device cannot recognize.

    \item [rx_csum_bad]
        The number of packets with checksum mismatch.

\end{description}

\subparagraph{Transmitq CSUM Statistics}\label{sec:Device Types / Network Device / Device Operation / Control Virtqueue / Device Statistics / Transmitq CSUM Statistics}

The structure corresponding to the transmitq checksum statistics is
\field{struct virtio_net_stats_tx_csum}. The corresponding type is
VIRTIO_NET_STATS_TYPE_TX_CSUM. This is for the transmitq.

Only after the VIRTIO_NET_F_CSUM is negotiated, the transmitq checksum
statistics can be obtained.

The following are the transmitq checksum statistics:

\begin{lstlisting}
struct virtio_net_stats_tx_csum {
    struct virtio_net_stats_reply_hdr hdr;

    le64 tx_csum_none;
    le64 tx_needs_csum;
};
\end{lstlisting}

The packets described below are all for a specific virtqueue.
\begin{description}
    \item [tx_csum_none]
        The number of packets which do not require hardware checksum.

    \item [tx_needs_csum]
        The number of packets which require checksum calculation by the device.

\end{description}

\subparagraph{Receiveq GSO Statistics}\label{sec:Device Types / Network Device / Device Operation / Control Virtqueue / Device Statistics / Receiveq GSO Statistics}

The structure corresponding to the receivq GSO statistics is
\field{struct virtio_net_stats_rx_gso}. The corresponding type is
VIRTIO_NET_STATS_TYPE_RX_GSO. This is for the receiveq.

If one or more of the VIRTIO_NET_F_GUEST_TSO4, VIRTIO_NET_F_GUEST_TSO6
have been negotiated, the receiveq GSO statistics can be obtained.

GSO packets refer to packets passed by the device to the driver where
\field{gso_type} is not VIRTIO_NET_HDR_GSO_NONE.

\begin{lstlisting}
struct virtio_net_stats_rx_gso {
    struct virtio_net_stats_reply_hdr hdr;

    le64 rx_gso_packets;
    le64 rx_gso_bytes;
    le64 rx_gso_packets_coalesced;
    le64 rx_gso_bytes_coalesced;
};
\end{lstlisting}

The packets described below were all presented on the specified virtqueue.
\begin{description}
    \item [rx_gso_packets]
        The number of the GSO packets received by the device.

    \item [rx_gso_bytes]
        The bytes of the GSO packets received by the device.
        This includes the header size of the GSO packet.

    \item [rx_gso_packets_coalesced]
        The number of the GSO packets coalesced by the device.

    \item [rx_gso_bytes_coalesced]
        The bytes of the GSO packets coalesced by the device.
        This includes the header size of the GSO packet.
\end{description}

\subparagraph{Transmitq GSO Statistics}\label{sec:Device Types / Network Device / Device Operation / Control Virtqueue / Device Statistics / Transmitq GSO Statistics}

The structure corresponding to the transmitq GSO statistics is
\field{struct virtio_net_stats_tx_gso}. The corresponding type is
VIRTIO_NET_STATS_TYPE_TX_GSO. This is for the transmitq.

If one or more of the VIRTIO_NET_F_HOST_TSO4, VIRTIO_NET_F_HOST_TSO6,
VIRTIO_NET_F_HOST_USO options have been negotiated, the transmitq GSO statistics
can be obtained.

GSO packets refer to packets passed by the driver to the device where
\field{gso_type} is not VIRTIO_NET_HDR_GSO_NONE.
See more \ref{sec:Device Types / Network Device / Device Operation / Packet
Transmission}.

\begin{lstlisting}
struct virtio_net_stats_tx_gso {
    struct virtio_net_stats_reply_hdr hdr;

    le64 tx_gso_packets;
    le64 tx_gso_bytes;
    le64 tx_gso_segments;
    le64 tx_gso_segments_bytes;
    le64 tx_gso_packets_noseg;
    le64 tx_gso_bytes_noseg;
};
\end{lstlisting}

The packets described below are all for a specific virtqueue.
\begin{description}
    \item [tx_gso_packets]
        The number of the GSO packets sent by the device.

    \item [tx_gso_bytes]
        The bytes of the GSO packets sent by the device.

    \item [tx_gso_segments]
        The number of segments prepared from GSO packets.

    \item [tx_gso_segments_bytes]
        The bytes of segments prepared from GSO packets.

    \item [tx_gso_packets_noseg]
        The number of the GSO packets without segmentation.

    \item [tx_gso_bytes_noseg]
        The bytes of the GSO packets without segmentation.

\end{description}

\subparagraph{Receiveq Speed Statistics}\label{sec:Device Types / Network Device / Device Operation / Control Virtqueue / Device Statistics / Receiveq Speed Statistics}

The structure corresponding to the receiveq speed statistics is
\field{struct virtio_net_stats_rx_speed}. The corresponding type is
VIRTIO_NET_STATS_TYPE_RX_SPEED. This is for the receiveq.

The device has the allowance for the speed. If VIRTIO_NET_F_SPEED_DUPLEX has
been negotiated, the driver can get this by \field{speed}. When the received
packets bitrate exceeds the \field{speed}, some packets may be dropped by the
device.

\begin{lstlisting}
struct virtio_net_stats_rx_speed {
    struct virtio_net_stats_reply_hdr hdr;

    le64 rx_packets_allowance_exceeded;
    le64 rx_bytes_allowance_exceeded;
};
\end{lstlisting}

The packets described below were all presented on the specified virtqueue.
\begin{description}
    \item [rx_packets_allowance_exceeded]
        The number of the packets dropped by the device due to the received
        packets bitrate exceeding the \field{speed}.

    \item [rx_bytes_allowance_exceeded]
        The bytes of the packets dropped by the device due to the received
        packets bitrate exceeding the \field{speed}.

\end{description}

\subparagraph{Transmitq Speed Statistics}\label{sec:Device Types / Network Device / Device Operation / Control Virtqueue / Device Statistics / Transmitq Speed Statistics}

The structure corresponding to the transmitq speed statistics is
\field{struct virtio_net_stats_tx_speed}. The corresponding type is
VIRTIO_NET_STATS_TYPE_TX_SPEED. This is for the transmitq.

The device has the allowance for the speed. If VIRTIO_NET_F_SPEED_DUPLEX has
been negotiated, the driver can get this by \field{speed}. When the transmit
packets bitrate exceeds the \field{speed}, some packets may be dropped by the
device.

\begin{lstlisting}
struct virtio_net_stats_tx_speed {
    struct virtio_net_stats_reply_hdr hdr;

    le64 tx_packets_allowance_exceeded;
    le64 tx_bytes_allowance_exceeded;
};
\end{lstlisting}

The packets described below were all presented on the specified virtqueue.
\begin{description}
    \item [tx_packets_allowance_exceeded]
        The number of the packets dropped by the device due to the transmit packets
        bitrate exceeding the \field{speed}.

    \item [tx_bytes_allowance_exceeded]
        The bytes of the packets dropped by the device due to the transmit packets
        bitrate exceeding the \field{speed}.

\end{description}

\devicenormative{\subparagraph}{Device Statistics}{Device Types / Network Device / Device Operation / Control Virtqueue / Device Statistics}

When the VIRTIO_NET_F_DEVICE_STATS feature is negotiated, the device MUST reply
to the command VIRTIO_NET_CTRL_STATS_QUERY with the
\field{struct virtio_net_stats_capabilities}. \field{supported_stats_types}
includes all the statistic types supported by the device.

If \field{struct virtio_net_ctrl_queue_stats} is incorrect (such as the
following), the device MUST set \field{ack} to VIRTIO_NET_ERR. Even if there is
only one error, the device MUST fail the entire command.
\begin{itemize}
    \item \field{vq_index} exceeds the queue range.
    \item \field{types_bitmap} contains unknown types.
    \item One or more of the bits present in \field{types_bitmap} is not valid
        for the specified virtqueue.
    \item The feature corresponding to the specified \field{types_bitmap} was
        not negotiated.
\end{itemize}

The device MUST set the actual size of the bytes occupied by the reply to the
\field{size} of the \field{hdr}. And the device MUST set the \field{type} and
the \field{vq_index} of the statistic header.

The \field{command-specific-result} buffer allocated by the driver may be
smaller or bigger than all the statistics specified by
\field{struct virtio_net_ctrl_queue_stats}. The device MUST fill up only upto
the valid bytes.

The statistics counter replied by the device MUST wrap around to zero by the
device on the overflow.

\drivernormative{\subparagraph}{Device Statistics}{Device Types / Network Device / Device Operation / Control Virtqueue / Device Statistics}

The types contained in the \field{types_bitmap} MUST be queried from the device
via command VIRTIO_NET_CTRL_STATS_QUERY.

\field{types_bitmap} in \field{struct virtio_net_ctrl_queue_stats} MUST be valid to the
vq specified by \field{vq_index}.

The \field{command-specific-result} buffer allocated by the driver MUST have
enough capacity to store all the statistics reply headers defined in
\field{struct virtio_net_ctrl_queue_stats}. If the
\field{command-specific-result} buffer is fully utilized by the device but some
replies are missed, it is possible that some statistics may exceed the capacity
of the driver's records. In such cases, the driver should allocate additional
space for the \field{command-specific-result} buffer.

\subsubsection{Flow filter}\label{sec:Device Types / Network Device / Device Operation / Flow filter}

A network device can support one or more flow filter rules. Each flow filter rule
is applied by matching a packet and then taking an action, such as directing the packet
to a specific receiveq or dropping the packet. An example of a match is
matching on specific source and destination IP addresses.

A flow filter rule is a device resource object that consists of a key,
a processing priority, and an action to either direct a packet to a
receive queue or drop the packet.

Each rule uses a classifier. The key is matched against the packet using
a classifier, defining which fields in the packet are matched.
A classifier resource object consists of one or more field selectors, each with
a type that specifies the header fields to be matched against, and a mask.
The mask can match whole fields or parts of a field in a header. Each
rule resource object depends on the classifier resource object.

When a packet is received, relevant fields are extracted
(in the same way) from both the packet and the key according to the
classifier. The resulting field contents are then compared -
if they are identical the rule action is taken, if they are not, the rule is ignored.

Multiple flow filter rules are part of a group. The rule resource object
depends on the group. Each rule within a
group has a rule priority, and each group also has a group priority. For a
packet, a group with the highest priority is selected first. Within a group,
rules are applied from highest to lowest priority, until one of the rules
matches the packet and an action is taken. If all the rules within a group
are ignored, the group with the next highest priority is selected, and so on.

The device and the driver indicates flow filter resource limits using the capability
\ref{par:Device Types / Network Device / Device Operation / Flow filter / Device and driver capabilities / VIRTIO-NET-FF-RESOURCE-CAP} specifying the limits on the number of flow filter rule,
group and classifier resource objects. The capability \ref{par:Device Types / Network Device / Device Operation / Flow filter / Device and driver capabilities / VIRTIO-NET-FF-SELECTOR-CAP} specifies which selectors the device supports.
The driver indicates the selectors it is using by setting the flow
filter selector capability, prior to adding any resource objects.

The capability \ref{par:Device Types / Network Device / Device Operation / Flow filter / Device and driver capabilities / VIRTIO-NET-FF-ACTION-CAP} specifies which actions the device supports.

The driver controls the flow filter rule, classifier and group resource objects using
administration commands described in
\ref{sec:Basic Facilities of a Virtio Device / Device groups / Group administration commands / Device resource objects}.

\paragraph{Packet processing order}\label{sec:sec:Device Types / Network Device / Device Operation / Flow filter / Packet processing order}

Note that flow filter rules are applied after MAC/VLAN filtering. Flow filter
rules take precedence over steering: if a flow filter rule results in an action,
the steering configuration does not apply. The steering configuration only applies
to packets for which no flow filter rule action was performed. For example,
incoming packets can be processed in the following order:

\begin{itemize}
\item apply steering configuration received using control virtqueue commands
      VIRTIO_NET_CTRL_RX, VIRTIO_NET_CTRL_MAC and VIRTIO_NET_CTRL_VLAN.
\item apply flow filter rules if any.
\item if no filter rule applied, apply steering configuration received using command
      VIRTIO_NET_CTRL_MQ_RSS_CONFIG or as per automatic receive steering.
\end{itemize}

Some incoming packet processing examples:
\begin{itemize}
\item If the packet is dropped by the flow filter rule, RSS
      steering is ignored for the packet.
\item If the packet is directed to a specific receiveq using flow filter rule,
      the RSS steering is ignored for the packet.
\item If a packet is dropped due to the VIRTIO_NET_CTRL_MAC configuration,
      both flow filter rules and the RSS steering are ignored for the packet.
\item If a packet does not match any flow filter rules,
      the RSS steering is used to select the receiveq for the packet (if enabled).
\item If there are two flow filter groups configured as group_A and group_B
      with respective group priorities as 4, and 5; flow filter rules of
      group_B are applied first having highest group priority, if there is a match,
      the flow filter rules of group_A are ignored; if there is no match for
      the flow filter rules in group_B, the flow filter rules of next level group_A are applied.
\end{itemize}

\paragraph{Device and driver capabilities}
\label{par:Device Types / Network Device / Device Operation / Flow filter / Device and driver capabilities}

\subparagraph{VIRTIO_NET_FF_RESOURCE_CAP}
\label{par:Device Types / Network Device / Device Operation / Flow filter / Device and driver capabilities / VIRTIO-NET-FF-RESOURCE-CAP}

The capability VIRTIO_NET_FF_RESOURCE_CAP indicates the flow filter resource limits.
\field{cap_specific_data} is in the format
\field{struct virtio_net_ff_cap_data}.

\begin{lstlisting}
struct virtio_net_ff_cap_data {
        le32 groups_limit;
        le32 selectors_limit;
        le32 rules_limit;
        le32 rules_per_group_limit;
        u8 last_rule_priority;
        u8 selectors_per_classifier_limit;
};
\end{lstlisting}

\field{groups_limit}, and \field{selectors_limit} represent the maximum
number of flow filter groups and selectors, respectively, that the driver can create.
 \field{rules_limit} is the maximum number of
flow fiilter rules that the driver can create across all the groups.
\field{rules_per_group_limit} is the maximum number of flow filter rules that the driver
can create for each flow filter group.

\field{last_rule_priority} is the highest priority that can be assigned to a
flow filter rule.

\field{selectors_per_classifier_limit} is the maximum number of selectors
that a classifier can have.

\subparagraph{VIRTIO_NET_FF_SELECTOR_CAP}
\label{par:Device Types / Network Device / Device Operation / Flow filter / Device and driver capabilities / VIRTIO-NET-FF-SELECTOR-CAP}

The capability VIRTIO_NET_FF_SELECTOR_CAP lists the supported selectors and the
supported packet header fields for each selector.
\field{cap_specific_data} is in the format \field{struct virtio_net_ff_cap_mask_data}.

\begin{lstlisting}[label={lst:Device Types / Network Device / Device Operation / Flow filter / Device and driver capabilities / VIRTIO-NET-FF-SELECTOR-CAP / virtio-net-ff-selector}]
struct virtio_net_ff_selector {
        u8 type;
        u8 flags;
        u8 reserved[2];
        u8 length;
        u8 reserved1[3];
        u8 mask[];
};

struct virtio_net_ff_cap_mask_data {
        u8 count;
        u8 reserved[7];
        struct virtio_net_ff_selector selectors[];
};

#define VIRTIO_NET_FF_MASK_F_PARTIAL_MASK (1 << 0)
\end{lstlisting}

\field{count} indicates number of valid entries in the \field{selectors} array.
\field{selectors[]} is an array of supported selectors. Within each array entry:
\field{type} specifies the type of the packet header, as defined in table
\ref{table:Device Types / Network Device / Device Operation / Flow filter / Device and driver capabilities / VIRTIO-NET-FF-SELECTOR-CAP / flow filter selector types}. \field{mask} specifies which fields of the
packet header can be matched in a flow filter rule.

Each \field{type} is also listed in table
\ref{table:Device Types / Network Device / Device Operation / Flow filter / Device and driver capabilities / VIRTIO-NET-FF-SELECTOR-CAP / flow filter selector types}. \field{mask} is a byte array
in network byte order. For example, when \field{type} is VIRTIO_NET_FF_MASK_TYPE_IPV6,
the \field{mask} is in the format \hyperref[intro:IPv6-Header-Format]{IPv6 Header Format}.

If partial masking is not set, then all bits in each field have to be either all 0s
to ignore this field or all 1s to match on this field. If partial masking is set,
then any combination of bits can bit set to match on these bits.
For example, when a selector \field{type} is VIRTIO_NET_FF_MASK_TYPE_ETH, if
\field{mask[0-12]} are zero and \field{mask[13-14]} are 0xff (all 1s), it
indicates that matching is only supported for \field{EtherType} of
\field{Ethernet MAC frame}, matching is not supported for
\field{Destination Address} and \field{Source Address}.

The entries in the array \field{selectors} are ordered by
\field{type}, with each \field{type} value only appearing once.

\field{length} is the length of a dynamic array \field{mask} in bytes.
\field{reserved} and \field{reserved1} are reserved and set to zero.

\begin{table}[H]
\caption{Flow filter selector types}
\label{table:Device Types / Network Device / Device Operation / Flow filter / Device and driver capabilities / VIRTIO-NET-FF-SELECTOR-CAP / flow filter selector types}
\begin{tabularx}{\textwidth}{ |l|X|X| }
\hline
Type & Name & Description \\
\hline \hline
0x0 & - & Reserved \\
\hline
0x1 & VIRTIO_NET_FF_MASK_TYPE_ETH & 14 bytes of frame header starting from destination address described in \hyperref[intro:IEEE 802.3-2022]{IEEE 802.3-2022} \\
\hline
0x2 & VIRTIO_NET_FF_MASK_TYPE_IPV4 & 20 bytes of \hyperref[intro:Internet-Header-Format]{IPv4: Internet Header Format} \\
\hline
0x3 & VIRTIO_NET_FF_MASK_TYPE_IPV6 & 40 bytes of \hyperref[intro:IPv6-Header-Format]{IPv6 Header Format} \\
\hline
0x4 & VIRTIO_NET_FF_MASK_TYPE_TCP & 20 bytes of \hyperref[intro:TCP-Header-Format]{TCP Header Format} \\
\hline
0x5 & VIRTIO_NET_FF_MASK_TYPE_UDP & 8 bytes of UDP header described in \hyperref[intro:UDP]{UDP} \\
\hline
0x6 - 0xFF & & Reserved for future \\
\hline
\end{tabularx}
\end{table}

When VIRTIO_NET_FF_MASK_F_PARTIAL_MASK (bit 0) is set, it indicates that
partial masking is supported for all the fields of the selector identified by \field{type}.

For the selector \field{type} VIRTIO_NET_FF_MASK_TYPE_IPV4, if a partial mask is unsupported,
then matching on an individual bit of \field{Flags} in the
\field{IPv4: Internet Header Format} is unsupported. \field{Flags} has to match as a whole
if it is supported.

For the selector \field{type} VIRTIO_NET_FF_MASK_TYPE_IPV4, \field{mask} includes fields
up to the \field{Destination Address}; that is, \field{Options} and
\field{Padding} are excluded.

For the selector \field{type} VIRTIO_NET_FF_MASK_TYPE_IPV6, the \field{Next Header} field
of the \field{mask} corresponds to the \field{Next Header} in the packet
when \field{IPv6 Extension Headers} are not present. When the packet includes
one or more \field{IPv6 Extension Headers}, the \field{Next Header} field of
the \field{mask} corresponds to the \field{Next Header} of the last
\field{IPv6 Extension Header} in the packet.

For the selector \field{type} VIRTIO_NET_FF_MASK_TYPE_TCP, \field{Control bits}
are treated as individual fields for matching; that is, matching individual
\field{Control bits} does not depend on the partial mask support.

\subparagraph{VIRTIO_NET_FF_ACTION_CAP}
\label{par:Device Types / Network Device / Device Operation / Flow filter / Device and driver capabilities / VIRTIO-NET-FF-ACTION-CAP}

The capability VIRTIO_NET_FF_ACTION_CAP lists the supported actions in a rule.
\field{cap_specific_data} is in the format \field{struct virtio_net_ff_cap_actions}.

\begin{lstlisting}
struct virtio_net_ff_actions {
        u8 count;
        u8 reserved[7];
        u8 actions[];
};
\end{lstlisting}

\field{actions} is an array listing all possible actions.
The entries in the array are ordered from the smallest to the largest,
with each supported value appearing exactly once. Each entry can have the
following values:

\begin{table}[H]
\caption{Flow filter rule actions}
\label{table:Device Types / Network Device / Device Operation / Flow filter / Device and driver capabilities / VIRTIO-NET-FF-ACTION-CAP / flow filter rule actions}
\begin{tabularx}{\textwidth}{ |l|X|X| }
\hline
Action & Name & Description \\
\hline \hline
0x0 & - & reserved \\
\hline
0x1 & VIRTIO_NET_FF_ACTION_DROP & Matching packet will be dropped by the device \\
\hline
0x2 & VIRTIO_NET_FF_ACTION_DIRECT_RX_VQ & Matching packet will be directed to a receive queue \\
\hline
0x3 - 0xFF & & Reserved for future \\
\hline
\end{tabularx}
\end{table}

\paragraph{Resource objects}
\label{par:Device Types / Network Device / Device Operation / Flow filter / Resource objects}

\subparagraph{VIRTIO_NET_RESOURCE_OBJ_FF_GROUP}\label{par:Device Types / Network Device / Device Operation / Flow filter / Resource objects / VIRTIO-NET-RESOURCE-OBJ-FF-GROUP}

A flow filter group contains between 0 and \field{rules_limit} rules, as specified by the
capability VIRTIO_NET_FF_RESOURCE_CAP. For the flow filter group object both
\field{resource_obj_specific_data} and
\field{resource_obj_specific_result} are in the format
\field{struct virtio_net_resource_obj_ff_group}.

\begin{lstlisting}
struct virtio_net_resource_obj_ff_group {
        le16 group_priority;
};
\end{lstlisting}

\field{group_priority} specifies the priority for the group. Each group has a
distinct priority. For each incoming packet, the device tries to apply rules
from groups from higher \field{group_priority} value to lower, until either a
rule matches the packet or all groups have been tried.

\subparagraph{VIRTIO_NET_RESOURCE_OBJ_FF_CLASSIFIER}\label{par:Device Types / Network Device / Device Operation / Flow filter / Resource objects / VIRTIO-NET-RESOURCE-OBJ-FF-CLASSIFIER}

A classifier is used to match a flow filter key against a packet. The
classifier defines the desired packet fields to match, and is represented by
the VIRTIO_NET_RESOURCE_OBJ_FF_CLASSIFIER device resource object.

For the flow filter classifier object both \field{resource_obj_specific_data} and
\field{resource_obj_specific_result} are in the format
\field{struct virtio_net_resource_obj_ff_classifier}.

\begin{lstlisting}
struct virtio_net_resource_obj_ff_classifier {
        u8 count;
        u8 reserved[7];
        struct virtio_net_ff_selector selectors[];
};
\end{lstlisting}

A classifier is an array of \field{selectors}. The number of selectors in the
array is indicated by \field{count}. The selector has a type that specifies
the header fields to be matched against, and a mask.
See \ref{lst:Device Types / Network Device / Device Operation / Flow filter / Device and driver capabilities / VIRTIO-NET-FF-SELECTOR-CAP / virtio-net-ff-selector}
for details about selectors.

The first selector is always VIRTIO_NET_FF_MASK_TYPE_ETH. When there are multiple
selectors, a second selector can be either VIRTIO_NET_FF_MASK_TYPE_IPV4
or VIRTIO_NET_FF_MASK_TYPE_IPV6. If the third selector exists, the third
selector can be either VIRTIO_NET_FF_MASK_TYPE_UDP or VIRTIO_NET_FF_MASK_TYPE_TCP.
For example, to match a Ethernet IPv6 UDP packet,
\field{selectors[0].type} is set to VIRTIO_NET_FF_MASK_TYPE_ETH, \field{selectors[1].type}
is set to VIRTIO_NET_FF_MASK_TYPE_IPV6 and \field{selectors[2].type} is
set to VIRTIO_NET_FF_MASK_TYPE_UDP; accordingly, \field{selectors[0].mask[0-13]} is
for Ethernet header fields, \field{selectors[1].mask[0-39]} is set for IPV6 header
and \field{selectors[2].mask[0-7]} is set for UDP header.

When there are multiple selectors, the type of the (N+1)\textsuperscript{th} selector
affects the mask of the (N)\textsuperscript{th} selector. If
\field{count} is 2 or more, all the mask bits within \field{selectors[0]}
corresponding to \field{EtherType} of an Ethernet header are set.

If \field{count} is more than 2:
\begin{itemize}
\item if \field{selector[1].type} is, VIRTIO_NET_FF_MASK_TYPE_IPV4, then, all the mask bits within
\field{selector[1]} for \field{Protocol} is set.
\item if \field{selector[1].type} is, VIRTIO_NET_FF_MASK_TYPE_IPV6, then, all the mask bits within
\field{selector[1]} for \field{Next Header} is set.
\end{itemize}

If for a given packet header field, a subset of bits of a field is to be matched,
and if the partial mask is supported, the flow filter
mask object can specify a mask which has fewer bits set than the packet header
field size. For example, a partial mask for the Ethernet header source mac
address can be of 1-bit for multicast detection instead of 48-bits.

\subparagraph{VIRTIO_NET_RESOURCE_OBJ_FF_RULE}\label{par:Device Types / Network Device / Device Operation / Flow filter / Resource objects / VIRTIO-NET-RESOURCE-OBJ-FF-RULE}

Each flow filter rule resource object comprises a key, a priority, and an action.
For the flow filter rule object,
\field{resource_obj_specific_data} and
\field{resource_obj_specific_result} are in the format
\field{struct virtio_net_resource_obj_ff_rule}.

\begin{lstlisting}
struct virtio_net_resource_obj_ff_rule {
        le32 group_id;
        le32 classifier_id;
        u8 rule_priority;
        u8 key_length; /* length of key in bytes */
        u8 action;
        u8 reserved;
        le16 vq_index;
        u8 reserved1[2];
        u8 keys[][];
};
\end{lstlisting}

\field{group_id} is the resource object ID of the flow filter group to which
this rule belongs. \field{classifier_id} is the resource object ID of the
classifier used to match a packet against the \field{key}.

\field{rule_priority} denotes the priority of the rule within the group
specified by the \field{group_id}.
Rules within the group are applied from the highest to the lowest priority
until a rule matches the packet and an
action is taken. Rules with the same priority can be applied in any order.

\field{reserved} and \field{reserved1} are reserved and set to 0.

\field{keys[][]} is an array of keys to match against packets, using
the classifier specified by \field{classifier_id}. Each entry (key) comprises
a byte array, and they are located one immediately after another.
The size (number of entries) of the array is exactly the same as that of
\field{selectors} in the classifier, or in other words, \field{count}
in the classifier.

\field{key_length} specifies the total length of \field{keys} in bytes.
In other words, it equals the sum total of \field{length} of all
selectors in \field{selectors} in the classifier specified by
\field{classifier_id}.

For example, if a classifier object's \field{selectors[0].type} is
VIRTIO_NET_FF_MASK_TYPE_ETH and \field{selectors[1].type} is
VIRTIO_NET_FF_MASK_TYPE_IPV6,
then selectors[0].length is 14 and selectors[1].length is 40.
Accordingly, the \field{key_length} is set to 54.
This setting indicates that the \field{key} array's length is 54 bytes
comprising a first byte array of 14 bytes for the
Ethernet MAC header in bytes 0-13, immediately followed by 40 bytes for the
IPv6 header in bytes 14-53.

When there are multiple selectors in the classifier object, the key bytes
for (N)\textsuperscript{th} selector are set so that
(N+1)\textsuperscript{th} selector can be matched.

If \field{count} is 2 or more, key bytes of \field{EtherType}
are set according to \hyperref[intro:IEEE 802 Ethertypes]{IEEE 802 Ethertypes}
for VIRTIO_NET_FF_MASK_TYPE_IPV4 or VIRTIO_NET_FF_MASK_TYPE_IPV6 respectively.

If \field{count} is more than 2, when \field{selector[1].type} is
VIRTIO_NET_FF_MASK_TYPE_IPV4 or VIRTIO_NET_FF_MASK_TYPE_IPV6, key
bytes of \field{Protocol} or \field{Next Header} is set as per
\field{Protocol Numbers} defined \hyperref[intro:IANA Protocol Numbers]{IANA Protocol Numbers}
respectively.

\field{action} is the action to take when a packet matches the
\field{key} using the \field{classifier_id}. Supported actions are described in
\ref{table:Device Types / Network Device / Device Operation / Flow filter / Device and driver capabilities / VIRTIO-NET-FF-ACTION-CAP / flow filter rule actions}.

\field{vq_index} specifies a receive virtqueue. When the \field{action} is set
to VIRTIO_NET_FF_ACTION_DIRECT_RX_VQ, and the packet matches the \field{key},
the matching packet is directed to this virtqueue.

Note that at most one action is ever taken for a given packet. If a rule is
applied and an action is taken, the action of other rules is not taken.

\devicenormative{\paragraph}{Flow filter}{Device Types / Network Device / Device Operation / Flow filter}

When the device supports flow filter operations,
\begin{itemize}
\item the device MUST set VIRTIO_NET_FF_RESOURCE_CAP, VIRTIO_NET_FF_SELECTOR_CAP
and VIRTIO_NET_FF_ACTION_CAP capability in the \field{supported_caps} in the
command VIRTIO_ADMIN_CMD_CAP_SUPPORT_QUERY.
\item the device MUST support the administration commands
VIRTIO_ADMIN_CMD_RESOURCE_OBJ_CREATE,
VIRTIO_ADMIN_CMD_RESOURCE_OBJ_MODIFY, VIRTIO_ADMIN_CMD_RESOURCE_OBJ_QUERY,
VIRTIO_ADMIN_CMD_RESOURCE_OBJ_DESTROY for the resource types
VIRTIO_NET_RESOURCE_OBJ_FF_GROUP, VIRTIO_NET_RESOURCE_OBJ_FF_CLASSIFIER and
VIRTIO_NET_RESOURCE_OBJ_FF_RULE.
\end{itemize}

When any of the VIRTIO_NET_FF_RESOURCE_CAP, VIRTIO_NET_FF_SELECTOR_CAP, or
VIRTIO_NET_FF_ACTION_CAP capability is disabled, the device SHOULD set
\field{status} to VIRTIO_ADMIN_STATUS_Q_INVALID_OPCODE for the commands
VIRTIO_ADMIN_CMD_RESOURCE_OBJ_CREATE,
VIRTIO_ADMIN_CMD_RESOURCE_OBJ_MODIFY, VIRTIO_ADMIN_CMD_RESOURCE_OBJ_QUERY,
and VIRTIO_ADMIN_CMD_RESOURCE_OBJ_DESTROY. These commands apply to the resource
\field{type} of VIRTIO_NET_RESOURCE_OBJ_FF_GROUP, VIRTIO_NET_RESOURCE_OBJ_FF_CLASSIFIER, and
VIRTIO_NET_RESOURCE_OBJ_FF_RULE.

The device SHOULD set \field{status} to VIRTIO_ADMIN_STATUS_EINVAL for the
command VIRTIO_ADMIN_CMD_RESOURCE_OBJ_CREATE when the resource \field{type}
is VIRTIO_NET_RESOURCE_OBJ_FF_GROUP, if a flow filter group already exists
with the supplied \field{group_priority}.

The device SHOULD set \field{status} to VIRTIO_ADMIN_STATUS_ENOSPC for the
command VIRTIO_ADMIN_CMD_RESOURCE_OBJ_CREATE when the resource \field{type}
is VIRTIO_NET_RESOURCE_OBJ_FF_GROUP, if the number of flow filter group
objects in the device exceeds the lower of the configured driver
capabilities \field{groups_limit} and \field{rules_per_group_limit}.

The device SHOULD set \field{status} to VIRTIO_ADMIN_STATUS_ENOSPC for the
command VIRTIO_ADMIN_CMD_RESOURCE_OBJ_CREATE when the resource \field{type} is
VIRTIO_NET_RESOURCE_OBJ_FF_CLASSIFIER, if the number of flow filter selector
objects in the device exceeds the configured driver capability
\field{selectors_limit}.

The device SHOULD set \field{status} to VIRTIO_ADMIN_STATUS_EBUSY for the
command VIRTIO_ADMIN_CMD_RESOURCE_OBJ_DESTROY for a flow filter group when
the flow filter group has one or more flow filter rules depending on it.

The device SHOULD set \field{status} to VIRTIO_ADMIN_STATUS_EBUSY for the
command VIRTIO_ADMIN_CMD_RESOURCE_OBJ_DESTROY for a flow filter classifier when
the flow filter classifier has one or more flow filter rules depending on it.

The device SHOULD fail the command VIRTIO_ADMIN_CMD_RESOURCE_OBJ_CREATE for the
flow filter rule resource object if,
\begin{itemize}
\item \field{vq_index} is not a valid receive virtqueue index for
the VIRTIO_NET_FF_ACTION_DIRECT_RX_VQ action,
\item \field{priority} is greater than or equal to
      \field{last_rule_priority},
\item \field{id} is greater than or equal to \field{rules_limit} or
      greater than or equal to \field{rules_per_group_limit}, whichever is lower,
\item the length of \field{keys} and the length of all the mask bytes of
      \field{selectors[].mask} as referred by \field{classifier_id} differs,
\item the supplied \field{action} is not supported in the capability VIRTIO_NET_FF_ACTION_CAP.
\end{itemize}

When the flow filter directs a packet to the virtqueue identified by
\field{vq_index} and if the receive virtqueue is reset, the device
MUST drop such packets.

Upon applying a flow filter rule to a packet, the device MUST STOP any further
application of rules and cease applying any other steering configurations.

For multiple flow filter groups, the device MUST apply the rules from
the group with the highest priority. If any rule from this group is applied,
the device MUST ignore the remaining groups. If none of the rules from the
highest priority group match, the device MUST apply the rules from
the group with the next highest priority, until either a rule matches or
all groups have been attempted.

The device MUST apply the rules within the group from the highest to the
lowest priority until a rule matches the packet, and the device MUST take
the action. If an action is taken, the device MUST not take any other
action for this packet.

The device MAY apply the rules with the same \field{rule_priority} in any
order within the group.

The device MUST process incoming packets in the following order:
\begin{itemize}
\item apply the steering configuration received using control virtqueue
      commands VIRTIO_NET_CTRL_RX, VIRTIO_NET_CTRL_MAC, and
      VIRTIO_NET_CTRL_VLAN.
\item apply flow filter rules if any.
\item if no filter rule is applied, apply the steering configuration
      received using the command VIRTIO_NET_CTRL_MQ_RSS_CONFIG
      or according to automatic receive steering.
\end{itemize}

When processing an incoming packet, if the packet is dropped at any stage, the device
MUST skip further processing.

When the device drops the packet due to the configuration done using the control
virtqueue commands VIRTIO_NET_CTRL_RX or VIRTIO_NET_CTRL_MAC or VIRTIO_NET_CTRL_VLAN,
the device MUST skip flow filter rules for this packet.

When the device performs flow filter match operations and if the operation
result did not have any match in all the groups, the receive packet processing
continues to next level, i.e. to apply configuration done using
VIRTIO_NET_CTRL_MQ_RSS_CONFIG command.

The device MUST support the creation of flow filter classifier objects
using the command VIRTIO_ADMIN_CMD_RESOURCE_OBJ_CREATE with \field{flags}
set to VIRTIO_NET_FF_MASK_F_PARTIAL_MASK;
this support is required even if all the bits of the masks are set for
a field in \field{selectors}, provided that partial masking is supported
for the selectors.

\drivernormative{\paragraph}{Flow filter}{Device Types / Network Device / Device Operation / Flow filter}

The driver MUST enable VIRTIO_NET_FF_RESOURCE_CAP, VIRTIO_NET_FF_SELECTOR_CAP,
and VIRTIO_NET_FF_ACTION_CAP capabilities to use flow filter.

The driver SHOULD NOT remove a flow filter group using the command
VIRTIO_ADMIN_CMD_RESOURCE_OBJ_DESTROY when one or more flow filter rules
depend on that group. The driver SHOULD only destroy the group after
all the associated rules have been destroyed.

The driver SHOULD NOT remove a flow filter classifier using the command
VIRTIO_ADMIN_CMD_RESOURCE_OBJ_DESTROY when one or more flow filter rules
depend on the classifier. The driver SHOULD only destroy the classifier
after all the associated rules have been destroyed.

The driver SHOULD NOT add multiple flow filter rules with the same
\field{rule_priority} within a flow filter group, as these rules MAY match
the same packet. The driver SHOULD assign different \field{rule_priority}
values to different flow filter rules if multiple rules may match a single
packet.

For the command VIRTIO_ADMIN_CMD_RESOURCE_OBJ_CREATE, when creating a resource
of \field{type} VIRTIO_NET_RESOURCE_OBJ_FF_CLASSIFIER, the driver MUST set:
\begin{itemize}
\item \field{selectors[0].type} to VIRTIO_NET_FF_MASK_TYPE_ETH.
\item \field{selectors[1].type} to VIRTIO_NET_FF_MASK_TYPE_IPV4 or
      VIRTIO_NET_FF_MASK_TYPE_IPV6 when \field{count} is more than 1,
\item \field{selectors[2].type} VIRTIO_NET_FF_MASK_TYPE_UDP or
      VIRTIO_NET_FF_MASK_TYPE_TCP when \field{count} is more than 2.
\end{itemize}

For the command VIRTIO_ADMIN_CMD_RESOURCE_OBJ_CREATE, when creating a resource
of \field{type} VIRTIO_NET_RESOURCE_OBJ_FF_CLASSIFIER, the driver MUST set:
\begin{itemize}
\item \field{selectors[0].mask} bytes to all 1s for the \field{EtherType}
       when \field{count} is 2 or more.
\item \field{selectors[1].mask} bytes to all 1s for \field{Protocol} or \field{Next Header}
       when \field{selector[1].type} is VIRTIO_NET_FF_MASK_TYPE_IPV4 or VIRTIO_NET_FF_MASK_TYPE_IPV6,
       and when \field{count} is more than 2.
\end{itemize}

For the command VIRTIO_ADMIN_CMD_RESOURCE_OBJ_CREATE, the resource \field{type}
VIRTIO_NET_RESOURCE_OBJ_FF_RULE, if the corresponding classifier object's
\field{count} is 2 or more, the driver MUST SET the \field{keys} bytes of
\field{EtherType} in accordance with
\hyperref[intro:IEEE 802 Ethertypes]{IEEE 802 Ethertypes}
for either VIRTIO_NET_FF_MASK_TYPE_IPV4 or VIRTIO_NET_FF_MASK_TYPE_IPV6.

For the command VIRTIO_ADMIN_CMD_RESOURCE_OBJ_CREATE, when creating a resource of
\field{type} VIRTIO_NET_RESOURCE_OBJ_FF_RULE, if the corresponding classifier
object's \field{count} is more than 2, and the \field{selector[1].type} is either
VIRTIO_NET_FF_MASK_TYPE_IPV4 or VIRTIO_NET_FF_MASK_TYPE_IPV6, the driver MUST
set the \field{keys} bytes for the \field{Protocol} or \field{Next Header}
according to \hyperref[intro:IANA Protocol Numbers]{IANA Protocol Numbers} respectively.

The driver SHOULD set all the bits for a field in the mask of a selector in both the
capability and the classifier object, unless the VIRTIO_NET_FF_MASK_F_PARTIAL_MASK
is enabled.

\subsubsection{Legacy Interface: Framing Requirements}\label{sec:Device
Types / Network Device / Legacy Interface: Framing Requirements}

When using legacy interfaces, transitional drivers which have not
negotiated VIRTIO_F_ANY_LAYOUT MUST use a single descriptor for the
\field{struct virtio_net_hdr} on both transmit and receive, with the
network data in the following descriptors.

Additionally, when using the control virtqueue (see \ref{sec:Device
Types / Network Device / Device Operation / Control Virtqueue})
, transitional drivers which have not
negotiated VIRTIO_F_ANY_LAYOUT MUST:
\begin{itemize}
\item for all commands, use a single 2-byte descriptor including the first two
fields: \field{class} and \field{command}
\item for all commands except VIRTIO_NET_CTRL_MAC_TABLE_SET
use a single descriptor including command-specific-data
with no padding.
\item for the VIRTIO_NET_CTRL_MAC_TABLE_SET command use exactly
two descriptors including command-specific-data with no padding:
the first of these descriptors MUST include the
virtio_net_ctrl_mac table structure for the unicast addresses with no padding,
the second of these descriptors MUST include the
virtio_net_ctrl_mac table structure for the multicast addresses
with no padding.
\item for all commands, use a single 1-byte descriptor for the
\field{ack} field
\end{itemize}

See \ref{sec:Basic
Facilities of a Virtio Device / Virtqueues / Message Framing}.

\section{Network Device}\label{sec:Device Types / Network Device}

The virtio network device is a virtual network interface controller.
It consists of a virtual Ethernet link which connects the device
to the Ethernet network. The device has transmit and receive
queues. The driver adds empty buffers to the receive virtqueue.
The device receives incoming packets from the link; the device
places these incoming packets in the receive virtqueue buffers.
The driver adds outgoing packets to the transmit virtqueue. The device
removes these packets from the transmit virtqueue and sends them to
the link. The device may have a control virtqueue. The driver
uses the control virtqueue to dynamically manipulate various
features of the initialized device.

\subsection{Device ID}\label{sec:Device Types / Network Device / Device ID}

 1

\subsection{Virtqueues}\label{sec:Device Types / Network Device / Virtqueues}

\begin{description}
\item[0] receiveq1
\item[1] transmitq1
\item[\ldots]
\item[2(N-1)] receiveqN
\item[2(N-1)+1] transmitqN
\item[2N] controlq
\end{description}

 N=1 if neither VIRTIO_NET_F_MQ nor VIRTIO_NET_F_RSS are negotiated, otherwise N is set by
 \field{max_virtqueue_pairs}.

controlq is optional; it only exists if VIRTIO_NET_F_CTRL_VQ is
negotiated.

\subsection{Feature bits}\label{sec:Device Types / Network Device / Feature bits}

\begin{description}
\item[VIRTIO_NET_F_CSUM (0)] Device handles packets with partial checksum offload.

\item[VIRTIO_NET_F_GUEST_CSUM (1)] Driver handles packets with partial checksum.

\item[VIRTIO_NET_F_CTRL_GUEST_OFFLOADS (2)] Control channel offloads
        reconfiguration support.

\item[VIRTIO_NET_F_MTU(3)] Device maximum MTU reporting is supported. If
    offered by the device, device advises driver about the value of
    its maximum MTU. If negotiated, the driver uses \field{mtu} as
    the maximum MTU value.

\item[VIRTIO_NET_F_MAC (5)] Device has given MAC address.

\item[VIRTIO_NET_F_GUEST_TSO4 (7)] Driver can receive TSOv4.

\item[VIRTIO_NET_F_GUEST_TSO6 (8)] Driver can receive TSOv6.

\item[VIRTIO_NET_F_GUEST_ECN (9)] Driver can receive TSO with ECN.

\item[VIRTIO_NET_F_GUEST_UFO (10)] Driver can receive UFO.

\item[VIRTIO_NET_F_HOST_TSO4 (11)] Device can receive TSOv4.

\item[VIRTIO_NET_F_HOST_TSO6 (12)] Device can receive TSOv6.

\item[VIRTIO_NET_F_HOST_ECN (13)] Device can receive TSO with ECN.

\item[VIRTIO_NET_F_HOST_UFO (14)] Device can receive UFO.

\item[VIRTIO_NET_F_MRG_RXBUF (15)] Driver can merge receive buffers.

\item[VIRTIO_NET_F_STATUS (16)] Configuration status field is
    available.

\item[VIRTIO_NET_F_CTRL_VQ (17)] Control channel is available.

\item[VIRTIO_NET_F_CTRL_RX (18)] Control channel RX mode support.

\item[VIRTIO_NET_F_CTRL_VLAN (19)] Control channel VLAN filtering.

\item[VIRTIO_NET_F_CTRL_RX_EXTRA (20)]	Control channel RX extra mode support.

\item[VIRTIO_NET_F_GUEST_ANNOUNCE(21)] Driver can send gratuitous
    packets.

\item[VIRTIO_NET_F_MQ(22)] Device supports multiqueue with automatic
    receive steering.

\item[VIRTIO_NET_F_CTRL_MAC_ADDR(23)] Set MAC address through control
    channel.

\item[VIRTIO_NET_F_DEVICE_STATS(50)] Device can provide device-level statistics
    to the driver through the control virtqueue.

\item[VIRTIO_NET_F_HASH_TUNNEL(51)] Device supports inner header hash for encapsulated packets.

\item[VIRTIO_NET_F_VQ_NOTF_COAL(52)] Device supports virtqueue notification coalescing.

\item[VIRTIO_NET_F_NOTF_COAL(53)] Device supports notifications coalescing.

\item[VIRTIO_NET_F_GUEST_USO4 (54)] Driver can receive USOv4 packets.

\item[VIRTIO_NET_F_GUEST_USO6 (55)] Driver can receive USOv6 packets.

\item[VIRTIO_NET_F_HOST_USO (56)] Device can receive USO packets. Unlike UFO
 (fragmenting the packet) the USO splits large UDP packet
 to several segments when each of these smaller packets has UDP header.

\item[VIRTIO_NET_F_HASH_REPORT(57)] Device can report per-packet hash
    value and a type of calculated hash.

\item[VIRTIO_NET_F_GUEST_HDRLEN(59)] Driver can provide the exact \field{hdr_len}
    value. Device benefits from knowing the exact header length.

\item[VIRTIO_NET_F_RSS(60)] Device supports RSS (receive-side scaling)
    with Toeplitz hash calculation and configurable hash
    parameters for receive steering.

\item[VIRTIO_NET_F_RSC_EXT(61)] Device can process duplicated ACKs
    and report number of coalesced segments and duplicated ACKs.

\item[VIRTIO_NET_F_STANDBY(62)] Device may act as a standby for a primary
    device with the same MAC address.

\item[VIRTIO_NET_F_SPEED_DUPLEX(63)] Device reports speed and duplex.

\item[VIRTIO_NET_F_RSS_CONTEXT(64)] Device supports multiple RSS contexts.

\item[VIRTIO_NET_F_GUEST_UDP_TUNNEL_GSO (65)] Driver can receive GSO packets
  carried by a UDP tunnel.

\item[VIRTIO_NET_F_GUEST_UDP_TUNNEL_GSO_CSUM (66)] Driver handles packets
  carried by a UDP tunnel with partial csum for the outer header.

\item[VIRTIO_NET_F_HOST_UDP_TUNNEL_GSO (67)] Device can receive GSO packets
  carried by a UDP tunnel.

\item[VIRTIO_NET_F_HOST_UDP_TUNNEL_GSO_CSUM (68)] Device handles packets
  carried by a UDP tunnel with partial csum for the outer header.
\end{description}

\subsubsection{Feature bit requirements}\label{sec:Device Types / Network Device / Feature bits / Feature bit requirements}

Some networking feature bits require other networking feature bits
(see \ref{drivernormative:Basic Facilities of a Virtio Device / Feature Bits}):

\begin{description}
\item[VIRTIO_NET_F_GUEST_TSO4] Requires VIRTIO_NET_F_GUEST_CSUM.
\item[VIRTIO_NET_F_GUEST_TSO6] Requires VIRTIO_NET_F_GUEST_CSUM.
\item[VIRTIO_NET_F_GUEST_ECN] Requires VIRTIO_NET_F_GUEST_TSO4 or VIRTIO_NET_F_GUEST_TSO6.
\item[VIRTIO_NET_F_GUEST_UFO] Requires VIRTIO_NET_F_GUEST_CSUM.
\item[VIRTIO_NET_F_GUEST_USO4] Requires VIRTIO_NET_F_GUEST_CSUM.
\item[VIRTIO_NET_F_GUEST_USO6] Requires VIRTIO_NET_F_GUEST_CSUM.
\item[VIRTIO_NET_F_GUEST_UDP_TUNNEL_GSO] Requires VIRTIO_NET_F_GUEST_TSO4, VIRTIO_NET_F_GUEST_TSO6,
   VIRTIO_NET_F_GUEST_USO4 and VIRTIO_NET_F_GUEST_USO6.
\item[VIRTIO_NET_F_GUEST_UDP_TUNNEL_GSO_CSUM] Requires VIRTIO_NET_F_GUEST_UDP_TUNNEL_GSO

\item[VIRTIO_NET_F_HOST_TSO4] Requires VIRTIO_NET_F_CSUM.
\item[VIRTIO_NET_F_HOST_TSO6] Requires VIRTIO_NET_F_CSUM.
\item[VIRTIO_NET_F_HOST_ECN] Requires VIRTIO_NET_F_HOST_TSO4 or VIRTIO_NET_F_HOST_TSO6.
\item[VIRTIO_NET_F_HOST_UFO] Requires VIRTIO_NET_F_CSUM.
\item[VIRTIO_NET_F_HOST_USO] Requires VIRTIO_NET_F_CSUM.
\item[VIRTIO_NET_F_HOST_UDP_TUNNEL_GSO] Requires VIRTIO_NET_F_HOST_TSO4, VIRTIO_NET_F_HOST_TSO6
   and VIRTIO_NET_F_HOST_USO.
\item[VIRTIO_NET_F_HOST_UDP_TUNNEL_GSO_CSUM] Requires VIRTIO_NET_F_HOST_UDP_TUNNEL_GSO

\item[VIRTIO_NET_F_CTRL_RX] Requires VIRTIO_NET_F_CTRL_VQ.
\item[VIRTIO_NET_F_CTRL_VLAN] Requires VIRTIO_NET_F_CTRL_VQ.
\item[VIRTIO_NET_F_GUEST_ANNOUNCE] Requires VIRTIO_NET_F_CTRL_VQ.
\item[VIRTIO_NET_F_MQ] Requires VIRTIO_NET_F_CTRL_VQ.
\item[VIRTIO_NET_F_CTRL_MAC_ADDR] Requires VIRTIO_NET_F_CTRL_VQ.
\item[VIRTIO_NET_F_NOTF_COAL] Requires VIRTIO_NET_F_CTRL_VQ.
\item[VIRTIO_NET_F_RSC_EXT] Requires VIRTIO_NET_F_HOST_TSO4 or VIRTIO_NET_F_HOST_TSO6.
\item[VIRTIO_NET_F_RSS] Requires VIRTIO_NET_F_CTRL_VQ.
\item[VIRTIO_NET_F_VQ_NOTF_COAL] Requires VIRTIO_NET_F_CTRL_VQ.
\item[VIRTIO_NET_F_HASH_TUNNEL] Requires VIRTIO_NET_F_CTRL_VQ along with VIRTIO_NET_F_RSS or VIRTIO_NET_F_HASH_REPORT.
\item[VIRTIO_NET_F_RSS_CONTEXT] Requires VIRTIO_NET_F_CTRL_VQ and VIRTIO_NET_F_RSS.
\end{description}

\begin{note}
The dependency between UDP_TUNNEL_GSO_CSUM and UDP_TUNNEL_GSO is intentionally
in the opposite direction with respect to the plain GSO features and the plain
checksum offload because UDP tunnel checksum offload gives very little gain
for non GSO packets and is quite complex to implement in H/W.
\end{note}

\subsubsection{Legacy Interface: Feature bits}\label{sec:Device Types / Network Device / Feature bits / Legacy Interface: Feature bits}
\begin{description}
\item[VIRTIO_NET_F_GSO (6)] Device handles packets with any GSO type. This was supposed to indicate segmentation offload support, but
upon further investigation it became clear that multiple bits were needed.
\item[VIRTIO_NET_F_GUEST_RSC4 (41)] Device coalesces TCPIP v4 packets. This was implemented by hypervisor patch for certification
purposes and current Windows driver depends on it. It will not function if virtio-net device reports this feature.
\item[VIRTIO_NET_F_GUEST_RSC6 (42)] Device coalesces TCPIP v6 packets. Similar to VIRTIO_NET_F_GUEST_RSC4.
\end{description}

\subsection{Device configuration layout}\label{sec:Device Types / Network Device / Device configuration layout}
\label{sec:Device Types / Block Device / Feature bits / Device configuration layout}

The network device has the following device configuration layout.
All of the device configuration fields are read-only for the driver.

\begin{lstlisting}
struct virtio_net_config {
        u8 mac[6];
        le16 status;
        le16 max_virtqueue_pairs;
        le16 mtu;
        le32 speed;
        u8 duplex;
        u8 rss_max_key_size;
        le16 rss_max_indirection_table_length;
        le32 supported_hash_types;
        le32 supported_tunnel_types;
};
\end{lstlisting}

The \field{mac} address field always exists (although it is only
valid if VIRTIO_NET_F_MAC is set).

The \field{status} only exists if VIRTIO_NET_F_STATUS is set.
Two bits are currently defined for the status field: VIRTIO_NET_S_LINK_UP
and VIRTIO_NET_S_ANNOUNCE.

\begin{lstlisting}
#define VIRTIO_NET_S_LINK_UP     1
#define VIRTIO_NET_S_ANNOUNCE    2
\end{lstlisting}

The following field, \field{max_virtqueue_pairs} only exists if
VIRTIO_NET_F_MQ or VIRTIO_NET_F_RSS is set. This field specifies the maximum number
of each of transmit and receive virtqueues (receiveq1\ldots receiveqN
and transmitq1\ldots transmitqN respectively) that can be configured once at least one of these features
is negotiated.

The following field, \field{mtu} only exists if VIRTIO_NET_F_MTU
is set. This field specifies the maximum MTU for the driver to
use.

The following two fields, \field{speed} and \field{duplex}, only
exist if VIRTIO_NET_F_SPEED_DUPLEX is set.

\field{speed} contains the device speed, in units of 1 MBit per
second, 0 to 0x7fffffff, or 0xffffffff for unknown speed.

\field{duplex} has the values of 0x01 for full duplex, 0x00 for
half duplex and 0xff for unknown duplex state.

Both \field{speed} and \field{duplex} can change, thus the driver
is expected to re-read these values after receiving a
configuration change notification.

The following field, \field{rss_max_key_size} only exists if VIRTIO_NET_F_RSS or VIRTIO_NET_F_HASH_REPORT is set.
It specifies the maximum supported length of RSS key in bytes.

The following field, \field{rss_max_indirection_table_length} only exists if VIRTIO_NET_F_RSS is set.
It specifies the maximum number of 16-bit entries in RSS indirection table.

The next field, \field{supported_hash_types} only exists if the device supports hash calculation,
i.e. if VIRTIO_NET_F_RSS or VIRTIO_NET_F_HASH_REPORT is set.

Field \field{supported_hash_types} contains the bitmask of supported hash types.
See \ref{sec:Device Types / Network Device / Device Operation / Processing of Incoming Packets / Hash calculation for incoming packets / Supported/enabled hash types} for details of supported hash types.

Field \field{supported_tunnel_types} only exists if the device supports inner header hash, i.e. if VIRTIO_NET_F_HASH_TUNNEL is set.

Field \field{supported_tunnel_types} contains the bitmask of encapsulation types supported by the device for inner header hash.
Encapsulation types are defined in \ref{sec:Device Types / Network Device / Device Operation / Processing of Incoming Packets /
Hash calculation for incoming packets / Encapsulation types supported/enabled for inner header hash}.

\devicenormative{\subsubsection}{Device configuration layout}{Device Types / Network Device / Device configuration layout}

The device MUST set \field{max_virtqueue_pairs} to between 1 and 0x8000 inclusive,
if it offers VIRTIO_NET_F_MQ.

The device MUST set \field{mtu} to between 68 and 65535 inclusive,
if it offers VIRTIO_NET_F_MTU.

The device SHOULD set \field{mtu} to at least 1280, if it offers
VIRTIO_NET_F_MTU.

The device MUST NOT modify \field{mtu} once it has been set.

The device MUST NOT pass received packets that exceed \field{mtu} (plus low
level ethernet header length) size with \field{gso_type} NONE or ECN
after VIRTIO_NET_F_MTU has been successfully negotiated.

The device MUST forward transmitted packets of up to \field{mtu} (plus low
level ethernet header length) size with \field{gso_type} NONE or ECN, and do
so without fragmentation, after VIRTIO_NET_F_MTU has been successfully
negotiated.

The device MUST set \field{rss_max_key_size} to at least 40, if it offers
VIRTIO_NET_F_RSS or VIRTIO_NET_F_HASH_REPORT.

The device MUST set \field{rss_max_indirection_table_length} to at least 128, if it offers
VIRTIO_NET_F_RSS.

If the driver negotiates the VIRTIO_NET_F_STANDBY feature, the device MAY act
as a standby device for a primary device with the same MAC address.

If VIRTIO_NET_F_SPEED_DUPLEX has been negotiated, \field{speed}
MUST contain the device speed, in units of 1 MBit per second, 0 to
0x7ffffffff, or 0xfffffffff for unknown.

If VIRTIO_NET_F_SPEED_DUPLEX has been negotiated, \field{duplex}
MUST have the values of 0x00 for full duplex, 0x01 for half
duplex, or 0xff for unknown.

If VIRTIO_NET_F_SPEED_DUPLEX and VIRTIO_NET_F_STATUS have both
been negotiated, the device SHOULD NOT change the \field{speed} and
\field{duplex} fields as long as VIRTIO_NET_S_LINK_UP is set in
the \field{status}.

The device SHOULD NOT offer VIRTIO_NET_F_HASH_REPORT if it
does not offer VIRTIO_NET_F_CTRL_VQ.

The device SHOULD NOT offer VIRTIO_NET_F_CTRL_RX_EXTRA if it
does not offer VIRTIO_NET_F_CTRL_VQ.

\drivernormative{\subsubsection}{Device configuration layout}{Device Types / Network Device / Device configuration layout}

The driver MUST NOT write to any of the device configuration fields.

A driver SHOULD negotiate VIRTIO_NET_F_MAC if the device offers it.
If the driver negotiates the VIRTIO_NET_F_MAC feature, the driver MUST set
the physical address of the NIC to \field{mac}.  Otherwise, it SHOULD
use a locally-administered MAC address (see \hyperref[intro:IEEE 802]{IEEE 802},
``9.2 48-bit universal LAN MAC addresses'').

If the driver does not negotiate the VIRTIO_NET_F_STATUS feature, it SHOULD
assume the link is active, otherwise it SHOULD read the link status from
the bottom bit of \field{status}.

A driver SHOULD negotiate VIRTIO_NET_F_MTU if the device offers it.

If the driver negotiates VIRTIO_NET_F_MTU, it MUST supply enough receive
buffers to receive at least one receive packet of size \field{mtu} (plus low
level ethernet header length) with \field{gso_type} NONE or ECN.

If the driver negotiates VIRTIO_NET_F_MTU, it MUST NOT transmit packets of
size exceeding the value of \field{mtu} (plus low level ethernet header length)
with \field{gso_type} NONE or ECN.

A driver SHOULD negotiate the VIRTIO_NET_F_STANDBY feature if the device offers it.

If VIRTIO_NET_F_SPEED_DUPLEX has been negotiated,
the driver MUST treat any value of \field{speed} above
0x7fffffff as well as any value of \field{duplex} not
matching 0x00 or 0x01 as an unknown value.

If VIRTIO_NET_F_SPEED_DUPLEX has been negotiated, the driver
SHOULD re-read \field{speed} and \field{duplex} after a
configuration change notification.

A driver SHOULD NOT negotiate VIRTIO_NET_F_HASH_REPORT if it
does not negotiate VIRTIO_NET_F_CTRL_VQ.

A driver SHOULD NOT negotiate VIRTIO_NET_F_CTRL_RX_EXTRA if it
does not negotiate VIRTIO_NET_F_CTRL_VQ.

\subsubsection{Legacy Interface: Device configuration layout}\label{sec:Device Types / Network Device / Device configuration layout / Legacy Interface: Device configuration layout}
\label{sec:Device Types / Block Device / Feature bits / Device configuration layout / Legacy Interface: Device configuration layout}
When using the legacy interface, transitional devices and drivers
MUST format \field{status} and
\field{max_virtqueue_pairs} in struct virtio_net_config
according to the native endian of the guest rather than
(necessarily when not using the legacy interface) little-endian.

When using the legacy interface, \field{mac} is driver-writable
which provided a way for drivers to update the MAC without
negotiating VIRTIO_NET_F_CTRL_MAC_ADDR.

\subsection{Device Initialization}\label{sec:Device Types / Network Device / Device Initialization}

A driver would perform a typical initialization routine like so:

\begin{enumerate}
\item Identify and initialize the receive and
  transmission virtqueues, up to N of each kind. If
  VIRTIO_NET_F_MQ feature bit is negotiated,
  N=\field{max_virtqueue_pairs}, otherwise identify N=1.

\item If the VIRTIO_NET_F_CTRL_VQ feature bit is negotiated,
  identify the control virtqueue.

\item Fill the receive queues with buffers: see \ref{sec:Device Types / Network Device / Device Operation / Setting Up Receive Buffers}.

\item Even with VIRTIO_NET_F_MQ, only receiveq1, transmitq1 and
  controlq are used by default.  The driver would send the
  VIRTIO_NET_CTRL_MQ_VQ_PAIRS_SET command specifying the
  number of the transmit and receive queues to use.

\item If the VIRTIO_NET_F_MAC feature bit is set, the configuration
  space \field{mac} entry indicates the ``physical'' address of the
  device, otherwise the driver would typically generate a random
  local MAC address.

\item If the VIRTIO_NET_F_STATUS feature bit is negotiated, the link
  status comes from the bottom bit of \field{status}.
  Otherwise, the driver assumes it's active.

\item A performant driver would indicate that it will generate checksumless
  packets by negotiating the VIRTIO_NET_F_CSUM feature.

\item If that feature is negotiated, a driver can use TCP segmentation or UDP
  segmentation/fragmentation offload by negotiating the VIRTIO_NET_F_HOST_TSO4 (IPv4
  TCP), VIRTIO_NET_F_HOST_TSO6 (IPv6 TCP), VIRTIO_NET_F_HOST_UFO
  (UDP fragmentation) and VIRTIO_NET_F_HOST_USO (UDP segmentation) features.

\item If the VIRTIO_NET_F_HOST_TSO6, VIRTIO_NET_F_HOST_TSO4 and VIRTIO_NET_F_HOST_USO
  segmentation features are negotiated, a driver can
  use TCP segmentation or UDP segmentation on top of UDP encapsulation
  offload, when the outer header does not require checksumming - e.g.
  the outer UDP checksum is zero - by negotiating the
  VIRTIO_NET_F_HOST_UDP_TUNNEL_GSO feature.
  GSO over UDP tunnels packets carry two sets of headers: the outer ones
  and the inner ones. The outer transport protocol is UDP, the inner
  could be either TCP or UDP. Only a single level of encapsulation
  offload is supported.

\item If VIRTIO_NET_F_HOST_UDP_TUNNEL_GSO is negotiated, a driver can
  additionally use TCP segmentation or UDP segmentation on top of UDP
  encapsulation with the outer header requiring checksum offload,
  negotiating the VIRTIO_NET_F_HOST_UDP_TUNNEL_GSO_CSUM feature.

\item The converse features are also available: a driver can save
  the virtual device some work by negotiating these features.\note{For example, a network packet transported between two guests on
the same system might not need checksumming at all, nor segmentation,
if both guests are amenable.}
   The VIRTIO_NET_F_GUEST_CSUM feature indicates that partially
  checksummed packets can be received, and if it can do that then
  the VIRTIO_NET_F_GUEST_TSO4, VIRTIO_NET_F_GUEST_TSO6,
  VIRTIO_NET_F_GUEST_UFO, VIRTIO_NET_F_GUEST_ECN, VIRTIO_NET_F_GUEST_USO4,
  VIRTIO_NET_F_GUEST_USO6 VIRTIO_NET_F_GUEST_UDP_TUNNEL_GSO and
  VIRTIO_NET_F_GUEST_UDP_TUNNEL_GSO_CSUM are the input equivalents of
  the features described above.
  See \ref{sec:Device Types / Network Device / Device Operation /
Setting Up Receive Buffers}~\nameref{sec:Device Types / Network
Device / Device Operation / Setting Up Receive Buffers} and
\ref{sec:Device Types / Network Device / Device Operation /
Processing of Incoming Packets}~\nameref{sec:Device Types /
Network Device / Device Operation / Processing of Incoming Packets} below.
\end{enumerate}

A truly minimal driver would only accept VIRTIO_NET_F_MAC and ignore
everything else.

\subsection{Device and driver capabilities}\label{sec:Device Types / Network Device / Device and driver capabilities}

The network device has the following capabilities.

\begin{tabularx}{\textwidth}{ |l||l|X| }
\hline
Identifier & Name & Description \\
\hline \hline
0x0800 & \hyperref[par:Device Types / Network Device / Device Operation / Flow filter / Device and driver capabilities / VIRTIO-NET-FF-RESOURCE-CAP]{VIRTIO_NET_FF_RESOURCE_CAP} & Flow filter resource capability \\
\hline
0x0801 & \hyperref[par:Device Types / Network Device / Device Operation / Flow filter / Device and driver capabilities / VIRTIO-NET-FF-SELECTOR-CAP]{VIRTIO_NET_FF_SELECTOR_CAP} & Flow filter classifier capability \\
\hline
0x0802 & \hyperref[par:Device Types / Network Device / Device Operation / Flow filter / Device and driver capabilities / VIRTIO-NET-FF-ACTION-CAP]{VIRTIO_NET_FF_ACTION_CAP} & Flow filter action capability \\
\hline
\end{tabularx}

\subsection{Device resource objects}\label{sec:Device Types / Network Device / Device resource objects}

The network device has the following resource objects.

\begin{tabularx}{\textwidth}{ |l||l|X| }
\hline
type & Name & Description \\
\hline \hline
0x0200 & \hyperref[par:Device Types / Network Device / Device Operation / Flow filter / Resource objects / VIRTIO-NET-RESOURCE-OBJ-FF-GROUP]{VIRTIO_NET_RESOURCE_OBJ_FF_GROUP} & Flow filter group resource object \\
\hline
0x0201 & \hyperref[par:Device Types / Network Device / Device Operation / Flow filter / Resource objects / VIRTIO-NET-RESOURCE-OBJ-FF-CLASSIFIER]{VIRTIO_NET_RESOURCE_OBJ_FF_CLASSIFIER} & Flow filter mask object \\
\hline
0x0202 & \hyperref[par:Device Types / Network Device / Device Operation / Flow filter / Resource objects / VIRTIO-NET-RESOURCE-OBJ-FF-RULE]{VIRTIO_NET_RESOURCE_OBJ_FF_RULE} & Flow filter rule object \\
\hline
\end{tabularx}

\subsection{Device parts}\label{sec:Device Types / Network Device / Device parts}

Network device parts represent the configuration done by the driver using control
virtqueue commands. Network device part is in the format of
\field{struct virtio_dev_part}.

\begin{tabularx}{\textwidth}{ |l||l|X| }
\hline
Type & Name & Description \\
\hline \hline
0x200 & VIRTIO_NET_DEV_PART_CVQ_CFG_PART & Represents device configuration done through a control virtqueue command, see \ref{sec:Device Types / Network Device / Device parts / VIRTIO-NET-DEV-PART-CVQ-CFG-PART} \\
\hline
0x201 - 0x5FF & - & reserved for future \\
\hline
\hline
\end{tabularx}

\subsubsection{VIRTIO_NET_DEV_PART_CVQ_CFG_PART}\label{sec:Device Types / Network Device / Device parts / VIRTIO-NET-DEV-PART-CVQ-CFG-PART}

For VIRTIO_NET_DEV_PART_CVQ_CFG_PART, \field{part_type} is set to 0x200. The
VIRTIO_NET_DEV_PART_CVQ_CFG_PART part indicates configuration performed by the
driver using a control virtqueue command.

\begin{lstlisting}
struct virtio_net_dev_part_cvq_selector {
        u8 class;
        u8 command;
        u8 reserved[6];
};
\end{lstlisting}

There is one device part of type VIRTIO_NET_DEV_PART_CVQ_CFG_PART for each
individual configuration. Each part is identified by a unique selector value.
The selector, \field{device_type_raw}, is in the format
\field{struct virtio_net_dev_part_cvq_selector}.

The selector consists of two fields: \field{class} and \field{command}. These
fields correspond to the \field{class} and \field{command} defined in
\field{struct virtio_net_ctrl}, as described in the relevant sections of
\ref{sec:Device Types / Network Device / Device Operation / Control Virtqueue}.

The value corresponding to each part’s selector follows the same format as the
respective \field{command-specific-data} described in the relevant sections of
\ref{sec:Device Types / Network Device / Device Operation / Control Virtqueue}.

For example, when the \field{class} is VIRTIO_NET_CTRL_MAC, the \field{command}
can be either VIRTIO_NET_CTRL_MAC_TABLE_SET or VIRTIO_NET_CTRL_MAC_ADDR_SET;
when \field{command} is set to VIRTIO_NET_CTRL_MAC_TABLE_SET, \field{value}
is in the format of \field{struct virtio_net_ctrl_mac}.

Supported selectors are listed in the table:

\begin{tabularx}{\textwidth}{ |l|X| }
\hline
Class selector & Command selector \\
\hline \hline
VIRTIO_NET_CTRL_RX & VIRTIO_NET_CTRL_RX_PROMISC \\
\hline
VIRTIO_NET_CTRL_RX & VIRTIO_NET_CTRL_RX_ALLMULTI \\
\hline
VIRTIO_NET_CTRL_RX & VIRTIO_NET_CTRL_RX_ALLUNI \\
\hline
VIRTIO_NET_CTRL_RX & VIRTIO_NET_CTRL_RX_NOMULTI \\
\hline
VIRTIO_NET_CTRL_RX & VIRTIO_NET_CTRL_RX_NOUNI \\
\hline
VIRTIO_NET_CTRL_RX & VIRTIO_NET_CTRL_RX_NOBCAST \\
\hline
VIRTIO_NET_CTRL_MAC & VIRTIO_NET_CTRL_MAC_TABLE_SET \\
\hline
VIRTIO_NET_CTRL_MAC & VIRTIO_NET_CTRL_MAC_ADDR_SET \\
\hline
VIRTIO_NET_CTRL_VLAN & VIRTIO_NET_CTRL_VLAN_ADD \\
\hline
VIRTIO_NET_CTRL_ANNOUNCE & VIRTIO_NET_CTRL_ANNOUNCE_ACK \\
\hline
VIRTIO_NET_CTRL_MQ & VIRTIO_NET_CTRL_MQ_VQ_PAIRS_SET \\
\hline
VIRTIO_NET_CTRL_MQ & VIRTIO_NET_CTRL_MQ_RSS_CONFIG \\
\hline
VIRTIO_NET_CTRL_MQ & VIRTIO_NET_CTRL_MQ_HASH_CONFIG \\
\hline
\hline
\end{tabularx}

For command selector VIRTIO_NET_CTRL_VLAN_ADD, device part consists of a whole
VLAN table.

\field{reserved} is reserved and set to zero.

\subsection{Device Operation}\label{sec:Device Types / Network Device / Device Operation}

Packets are transmitted by placing them in the
transmitq1\ldots transmitqN, and buffers for incoming packets are
placed in the receiveq1\ldots receiveqN. In each case, the packet
itself is preceded by a header:

\begin{lstlisting}
struct virtio_net_hdr {
#define VIRTIO_NET_HDR_F_NEEDS_CSUM    1
#define VIRTIO_NET_HDR_F_DATA_VALID    2
#define VIRTIO_NET_HDR_F_RSC_INFO      4
#define VIRTIO_NET_HDR_F_UDP_TUNNEL_CSUM 8
        u8 flags;
#define VIRTIO_NET_HDR_GSO_NONE        0
#define VIRTIO_NET_HDR_GSO_TCPV4       1
#define VIRTIO_NET_HDR_GSO_UDP         3
#define VIRTIO_NET_HDR_GSO_TCPV6       4
#define VIRTIO_NET_HDR_GSO_UDP_L4      5
#define VIRTIO_NET_HDR_GSO_UDP_TUNNEL_IPV4 0x20
#define VIRTIO_NET_HDR_GSO_UDP_TUNNEL_IPV6 0x40
#define VIRTIO_NET_HDR_GSO_ECN      0x80
        u8 gso_type;
        le16 hdr_len;
        le16 gso_size;
        le16 csum_start;
        le16 csum_offset;
        le16 num_buffers;
        le32 hash_value;        (Only if VIRTIO_NET_F_HASH_REPORT negotiated)
        le16 hash_report;       (Only if VIRTIO_NET_F_HASH_REPORT negotiated)
        le16 padding_reserved;  (Only if VIRTIO_NET_F_HASH_REPORT negotiated)
        le16 outer_th_offset    (Only if VIRTIO_NET_F_HOST_UDP_TUNNEL_GSO or VIRTIO_NET_F_GUEST_UDP_TUNNEL_GSO negotiated)
        le16 inner_nh_offset;   (Only if VIRTIO_NET_F_HOST_UDP_TUNNEL_GSO or VIRTIO_NET_F_GUEST_UDP_TUNNEL_GSO negotiated)
};
\end{lstlisting}

The controlq is used to control various device features described further in
section \ref{sec:Device Types / Network Device / Device Operation / Control Virtqueue}.

\subsubsection{Legacy Interface: Device Operation}\label{sec:Device Types / Network Device / Device Operation / Legacy Interface: Device Operation}
When using the legacy interface, transitional devices and drivers
MUST format the fields in \field{struct virtio_net_hdr}
according to the native endian of the guest rather than
(necessarily when not using the legacy interface) little-endian.

The legacy driver only presented \field{num_buffers} in the \field{struct virtio_net_hdr}
when VIRTIO_NET_F_MRG_RXBUF was negotiated; without that feature the
structure was 2 bytes shorter.

When using the legacy interface, the driver SHOULD ignore the
used length for the transmit queues
and the controlq queue.
\begin{note}
Historically, some devices put
the total descriptor length there, even though no data was
actually written.
\end{note}

\subsubsection{Packet Transmission}\label{sec:Device Types / Network Device / Device Operation / Packet Transmission}

Transmitting a single packet is simple, but varies depending on
the different features the driver negotiated.

\begin{enumerate}
\item The driver can send a completely checksummed packet.  In this case,
  \field{flags} will be zero, and \field{gso_type} will be VIRTIO_NET_HDR_GSO_NONE.

\item If the driver negotiated VIRTIO_NET_F_CSUM, it can skip
  checksumming the packet:
  \begin{itemize}
  \item \field{flags} has the VIRTIO_NET_HDR_F_NEEDS_CSUM set,

  \item \field{csum_start} is set to the offset within the packet to begin checksumming,
    and

  \item \field{csum_offset} indicates how many bytes after the csum_start the
    new (16 bit ones' complement) checksum is placed by the device.

  \item The TCP checksum field in the packet is set to the sum
    of the TCP pseudo header, so that replacing it by the ones'
    complement checksum of the TCP header and body will give the
    correct result.
  \end{itemize}

\begin{note}
For example, consider a partially checksummed TCP (IPv4) packet.
It will have a 14 byte ethernet header and 20 byte IP header
followed by the TCP header (with the TCP checksum field 16 bytes
into that header). \field{csum_start} will be 14+20 = 34 (the TCP
checksum includes the header), and \field{csum_offset} will be 16.
If the given packet has the VIRTIO_NET_HDR_GSO_UDP_TUNNEL_IPV4 bit or the
VIRTIO_NET_HDR_GSO_UDP_TUNNEL_IPV6 bit set,
the above checksum fields refer to the inner header checksum, see
the example below.
\end{note}

\item If the driver negotiated
  VIRTIO_NET_F_HOST_TSO4, TSO6, USO or UFO, and the packet requires
  TCP segmentation, UDP segmentation or fragmentation, then \field{gso_type}
  is set to VIRTIO_NET_HDR_GSO_TCPV4, TCPV6, UDP_L4 or UDP.
  (Otherwise, it is set to VIRTIO_NET_HDR_GSO_NONE). In this
  case, packets larger than 1514 bytes can be transmitted: the
  metadata indicates how to replicate the packet header to cut it
  into smaller packets. The other gso fields are set:

  \begin{itemize}
  \item If the VIRTIO_NET_F_GUEST_HDRLEN feature has been negotiated,
    \field{hdr_len} indicates the header length that needs to be replicated
    for each packet. It's the number of bytes from the beginning of the packet
    to the beginning of the transport payload.
    If the \field{gso_type} has the VIRTIO_NET_HDR_GSO_UDP_TUNNEL_IPV4 bit or
    VIRTIO_NET_HDR_GSO_UDP_TUNNEL_IPV6 bit set, \field{hdr_len} accounts for
    all the headers up to and including the inner transport.
    Otherwise, if the VIRTIO_NET_F_GUEST_HDRLEN feature has not been negotiated,
    \field{hdr_len} is a hint to the device as to how much of the header
    needs to be kept to copy into each packet, usually set to the
    length of the headers, including the transport header\footnote{Due to various bugs in implementations, this field is not useful
as a guarantee of the transport header size.
}.

  \begin{note}
  Some devices benefit from knowledge of the exact header length.
  \end{note}

  \item \field{gso_size} is the maximum size of each packet beyond that
    header (ie. MSS).

  \item If the driver negotiated the VIRTIO_NET_F_HOST_ECN feature,
    the VIRTIO_NET_HDR_GSO_ECN bit in \field{gso_type}
    indicates that the TCP packet has the ECN bit set\footnote{This case is not handled by some older hardware, so is called out
specifically in the protocol.}.
   \end{itemize}

\item If the driver negotiated the VIRTIO_NET_F_HOST_UDP_TUNNEL_GSO feature and the
  \field{gso_type} has the VIRTIO_NET_HDR_GSO_UDP_TUNNEL_IPV4 bit or
  VIRTIO_NET_HDR_GSO_UDP_TUNNEL_IPV6 bit set, the GSO protocol is encapsulated
  in a UDP tunnel.
  If the outer UDP header requires checksumming, the driver must have
  additionally negotiated the VIRTIO_NET_F_HOST_UDP_TUNNEL_GSO_CSUM feature
  and offloaded the outer checksum accordingly, otherwise
  the outer UDP header must not require checksum validation, i.e. the outer
  UDP checksum must be positive zero (0x0) as defined in UDP RFC 768.
  The other tunnel-related fields indicate how to replicate the packet
  headers to cut it into smaller packets:

  \begin{itemize}
  \item \field{outer_th_offset} field indicates the outer transport header within
      the packet. This field differs from \field{csum_start} as the latter
      points to the inner transport header within the packet.

  \item \field{inner_nh_offset} field indicates the inner network header within
      the packet.
  \end{itemize}

\begin{note}
For example, consider a partially checksummed TCP (IPv4) packet carried over a
Geneve UDP tunnel (again IPv4) with no tunnel options. The
only relevant variable related to the tunnel type is the tunnel header length.
The packet will have a 14 byte outer ethernet header, 20 byte outer IP header
followed by the 8 byte UDP header (with a 0 checksum value), 8 byte Geneve header,
14 byte inner ethernet header, 20 byte inner IP header
and the TCP header (with the TCP checksum field 16 bytes
into that header). \field{csum_start} will be 14+20+8+8+14+20 = 84 (the TCP
checksum includes the header), \field{csum_offset} will be 16.
\field{inner_nh_offset} will be 14+20+8+8+14 = 62, \field{outer_th_offset} will be
14+20+8 = 42 and \field{gso_type} will be
VIRTIO_NET_HDR_GSO_TCPV4 | VIRTIO_NET_HDR_GSO_UDP_TUNNEL_IPV4 = 0x21
\end{note}

\item If the driver negotiated the VIRTIO_NET_F_HOST_UDP_TUNNEL_GSO_CSUM feature,
  the transmitted packet is a GSO one encapsulated in a UDP tunnel, and
  the outer UDP header requires checksumming, the driver can skip checksumming
  the outer header:

  \begin{itemize}
  \item \field{flags} has the VIRTIO_NET_HDR_F_UDP_TUNNEL_CSUM set,

  \item The outer UDP checksum field in the packet is set to the sum
    of the UDP pseudo header, so that replacing it by the ones'
    complement checksum of the outer UDP header and payload will give the
    correct result.
  \end{itemize}

\item \field{num_buffers} is set to zero.  This field is unused on transmitted packets.

\item The header and packet are added as one output descriptor to the
  transmitq, and the device is notified of the new entry
  (see \ref{sec:Device Types / Network Device / Device Initialization}~\nameref{sec:Device Types / Network Device / Device Initialization}).
\end{enumerate}

\drivernormative{\paragraph}{Packet Transmission}{Device Types / Network Device / Device Operation / Packet Transmission}

For the transmit packet buffer, the driver MUST use the size of the
structure \field{struct virtio_net_hdr} same as the receive packet buffer.

The driver MUST set \field{num_buffers} to zero.

If VIRTIO_NET_F_CSUM is not negotiated, the driver MUST set
\field{flags} to zero and SHOULD supply a fully checksummed
packet to the device.

If VIRTIO_NET_F_HOST_TSO4 is negotiated, the driver MAY set
\field{gso_type} to VIRTIO_NET_HDR_GSO_TCPV4 to request TCPv4
segmentation, otherwise the driver MUST NOT set
\field{gso_type} to VIRTIO_NET_HDR_GSO_TCPV4.

If VIRTIO_NET_F_HOST_TSO6 is negotiated, the driver MAY set
\field{gso_type} to VIRTIO_NET_HDR_GSO_TCPV6 to request TCPv6
segmentation, otherwise the driver MUST NOT set
\field{gso_type} to VIRTIO_NET_HDR_GSO_TCPV6.

If VIRTIO_NET_F_HOST_UFO is negotiated, the driver MAY set
\field{gso_type} to VIRTIO_NET_HDR_GSO_UDP to request UDP
fragmentation, otherwise the driver MUST NOT set
\field{gso_type} to VIRTIO_NET_HDR_GSO_UDP.

If VIRTIO_NET_F_HOST_USO is negotiated, the driver MAY set
\field{gso_type} to VIRTIO_NET_HDR_GSO_UDP_L4 to request UDP
segmentation, otherwise the driver MUST NOT set
\field{gso_type} to VIRTIO_NET_HDR_GSO_UDP_L4.

The driver SHOULD NOT send to the device TCP packets requiring segmentation offload
which have the Explicit Congestion Notification bit set, unless the
VIRTIO_NET_F_HOST_ECN feature is negotiated, in which case the
driver MUST set the VIRTIO_NET_HDR_GSO_ECN bit in
\field{gso_type}.

If VIRTIO_NET_F_HOST_UDP_TUNNEL_GSO is negotiated, the driver MAY set
VIRTIO_NET_HDR_GSO_UDP_TUNNEL_IPV4 bit or the VIRTIO_NET_HDR_GSO_UDP_TUNNEL_IPV6 bit
in \field{gso_type} according to the inner network header protocol type
to request GSO packets over UDPv4 or UDPv6 tunnel segmentation,
otherwise the driver MUST NOT set either the
VIRTIO_NET_HDR_GSO_UDP_TUNNEL_IPV4 bit or the VIRTIO_NET_HDR_GSO_UDP_TUNNEL_IPV6 bit
in \field{gso_type}.

When requesting GSO segmentation over UDP tunnel, the driver MUST SET the
VIRTIO_NET_HDR_GSO_UDP_TUNNEL_IPV4 bit if the inner network header is IPv4, i.e. the
packet is a TCPv4 GSO one, otherwise, if the inner network header is IPv6, the driver
MUST SET the VIRTIO_NET_HDR_GSO_UDP_TUNNEL_IPV6 bit.

The driver MUST NOT send to the device GSO packets over UDP tunnel
requiring segmentation and outer UDP checksum offload, unless both the
VIRTIO_NET_F_HOST_UDP_TUNNEL_GSO and VIRTIO_NET_F_HOST_UDP_TUNNEL_GSO_CSUM features
are negotiated, in which case the driver MUST set either the
VIRTIO_NET_HDR_GSO_UDP_TUNNEL_IPV4 bit or the VIRTIO_NET_HDR_GSO_UDP_TUNNEL_IPV6
bit in the \field{gso_type} and the VIRTIO_NET_HDR_F_UDP_TUNNEL_CSUM bit in
the \field{flags}.

If VIRTIO_NET_F_HOST_UDP_TUNNEL_GSO_CSUM is not negotiated, the driver MUST not set
the VIRTIO_NET_HDR_F_UDP_TUNNEL_CSUM bit in the \field{flags} and
MUST NOT send to the device GSO packets over UDP tunnel
requiring segmentation and outer UDP checksum offload.

The driver MUST NOT set the VIRTIO_NET_HDR_GSO_UDP_TUNNEL_IPV4 bit or the
VIRTIO_NET_HDR_GSO_UDP_TUNNEL_IPV6 bit together with VIRTIO_NET_HDR_GSO_UDP, as the
latter is deprecated in favor of UDP_L4 and no new feature will support it.

The driver MUST NOT set the VIRTIO_NET_HDR_GSO_UDP_TUNNEL_IPV4 bit and the
VIRTIO_NET_HDR_GSO_UDP_TUNNEL_IPV6 bit together.

The driver MUST NOT set the VIRTIO_NET_HDR_F_UDP_TUNNEL_CSUM bit \field{flags}
without setting either the VIRTIO_NET_HDR_GSO_UDP_TUNNEL_IPV4 bit or
the VIRTIO_NET_HDR_GSO_UDP_TUNNEL_IPV6 bit in \field{gso_type}.

If the VIRTIO_NET_F_CSUM feature has been negotiated, the
driver MAY set the VIRTIO_NET_HDR_F_NEEDS_CSUM bit in
\field{flags}, if so:
\begin{enumerate}
\item the driver MUST validate the packet checksum at
	offset \field{csum_offset} from \field{csum_start} as well as all
	preceding offsets;
\begin{note}
If \field{gso_type} differs from VIRTIO_NET_HDR_GSO_NONE and the
VIRTIO_NET_HDR_GSO_UDP_TUNNEL_IPV4 bit or the VIRTIO_NET_HDR_GSO_UDP_TUNNEL_IPV6
bit are not set in \field{gso_type}, \field{csum_offset}
points to the only transport header present in the packet, and there are no
additional preceding checksums validated by VIRTIO_NET_HDR_F_NEEDS_CSUM.
\end{note}
\item the driver MUST set the packet checksum stored in the
	buffer to the TCP/UDP pseudo header;
\item the driver MUST set \field{csum_start} and
	\field{csum_offset} such that calculating a ones'
	complement checksum from \field{csum_start} up until the end of
	the packet and storing the result at offset \field{csum_offset}
	from  \field{csum_start} will result in a fully checksummed
	packet;
\end{enumerate}

If none of the VIRTIO_NET_F_HOST_TSO4, TSO6, USO or UFO options have
been negotiated, the driver MUST set \field{gso_type} to
VIRTIO_NET_HDR_GSO_NONE.

If \field{gso_type} differs from VIRTIO_NET_HDR_GSO_NONE, then
the driver MUST also set the VIRTIO_NET_HDR_F_NEEDS_CSUM bit in
\field{flags} and MUST set \field{gso_size} to indicate the
desired MSS.

If one of the VIRTIO_NET_F_HOST_TSO4, TSO6, USO or UFO options have
been negotiated:
\begin{itemize}
\item If the VIRTIO_NET_F_GUEST_HDRLEN feature has been negotiated,
	and \field{gso_type} differs from VIRTIO_NET_HDR_GSO_NONE,
	the driver MUST set \field{hdr_len} to a value equal to the length
	of the headers, including the transport header. If \field{gso_type}
	has the VIRTIO_NET_HDR_GSO_UDP_TUNNEL_IPV4 bit or the
	VIRTIO_NET_HDR_GSO_UDP_TUNNEL_IPV6 bit set, \field{hdr_len} includes
	the inner transport header.

\item If the VIRTIO_NET_F_GUEST_HDRLEN feature has not been negotiated,
	or \field{gso_type} is VIRTIO_NET_HDR_GSO_NONE,
	the driver SHOULD set \field{hdr_len} to a value
	not less than the length of the headers, including the transport
	header.
\end{itemize}

If the VIRTIO_NET_F_HOST_UDP_TUNNEL_GSO option has been negotiated, the
driver MAY set the VIRTIO_NET_HDR_GSO_UDP_TUNNEL_IPV4 bit or the
VIRTIO_NET_HDR_GSO_UDP_TUNNEL_IPV6 bit in \field{gso_type}, if so:
\begin{itemize}
\item the driver MUST set \field{outer_th_offset} to the outer UDP header
  offset and \field{inner_nh_offset} to the inner network header offset.
  The \field{csum_start} and \field{csum_offset} fields point respectively
  to the inner transport header and inner transport checksum field.
\end{itemize}

If the VIRTIO_NET_F_HOST_UDP_TUNNEL_GSO_CSUM feature has been negotiated,
and the VIRTIO_NET_HDR_GSO_UDP_TUNNEL_IPV4 bit or
VIRTIO_NET_HDR_GSO_UDP_TUNNEL_IPV6 bit in \field{gso_type} are set,
the driver MAY set the VIRTIO_NET_HDR_F_UDP_TUNNEL_CSUM bit in
\field{flags}, if so the driver MUST set the packet outer UDP header checksum
to the outer UDP pseudo header checksum.

\begin{note}
calculating a ones' complement checksum from \field{outer_th_offset}
up until the end of the packet and storing the result at offset 6
from \field{outer_th_offset} will result in a fully checksummed outer UDP packet;
\end{note}

If the VIRTIO_NET_HDR_GSO_UDP_TUNNEL_IPV4 bit or the
VIRTIO_NET_HDR_GSO_UDP_TUNNEL_IPV6 bit in \field{gso_type} are set
and the VIRTIO_NET_F_HOST_UDP_TUNNEL_GSO_CSUM feature has not
been negotiated, the
outer UDP header MUST NOT require checksum validation. That is, the
outer UDP checksum value MUST be 0 or the validated complete checksum
for such header.

\begin{note}
The valid complete checksum of the outer UDP header of individual segments
can be computed by the driver prior to segmentation only if the GSO packet
size is a multiple of \field{gso_size}, because then all segments
have the same size and thus all data included in the outer UDP
checksum is the same for every segment. These pre-computed segment
length and checksum fields are different from those of the GSO
packet.
In this scenario the outer UDP header of the GSO packet must carry the
segmented UDP packet length.
\end{note}

If the VIRTIO_NET_F_HOST_UDP_TUNNEL_GSO option has not
been negotiated, the driver MUST NOT set either the VIRTIO_NET_HDR_F_GSO_UDP_TUNNEL_IPV4
bit or the VIRTIO_NET_HDR_F_GSO_UDP_TUNNEL_IPV6 in \field{gso_type}.

If the VIRTIO_NET_F_HOST_UDP_TUNNEL_GSO_CSUM option has not been negotiated,
the driver MUST NOT set the VIRTIO_NET_HDR_F_UDP_TUNNEL_CSUM bit
in \field{flags}.

The driver SHOULD accept the VIRTIO_NET_F_GUEST_HDRLEN feature if it has
been offered, and if it's able to provide the exact header length.

The driver MUST NOT set the VIRTIO_NET_HDR_F_DATA_VALID and
VIRTIO_NET_HDR_F_RSC_INFO bits in \field{flags}.

The driver MUST NOT set the VIRTIO_NET_HDR_F_DATA_VALID bit in \field{flags}
together with the VIRTIO_NET_HDR_F_GSO_UDP_TUNNEL_IPV4 bit or the
VIRTIO_NET_HDR_F_GSO_UDP_TUNNEL_IPV6 bit in \field{gso_type}.

\devicenormative{\paragraph}{Packet Transmission}{Device Types / Network Device / Device Operation / Packet Transmission}
The device MUST ignore \field{flag} bits that it does not recognize.

If VIRTIO_NET_HDR_F_NEEDS_CSUM bit in \field{flags} is not set, the
device MUST NOT use the \field{csum_start} and \field{csum_offset}.

If one of the VIRTIO_NET_F_HOST_TSO4, TSO6, USO or UFO options have
been negotiated:
\begin{itemize}
\item If the VIRTIO_NET_F_GUEST_HDRLEN feature has been negotiated,
	and \field{gso_type} differs from VIRTIO_NET_HDR_GSO_NONE,
	the device MAY use \field{hdr_len} as the transport header size.

	\begin{note}
	Caution should be taken by the implementation so as to prevent
	a malicious driver from attacking the device by setting an incorrect hdr_len.
	\end{note}

\item If the VIRTIO_NET_F_GUEST_HDRLEN feature has not been negotiated,
	or \field{gso_type} is VIRTIO_NET_HDR_GSO_NONE,
	the device MAY use \field{hdr_len} only as a hint about the
	transport header size.
	The device MUST NOT rely on \field{hdr_len} to be correct.

	\begin{note}
	This is due to various bugs in implementations.
	\end{note}
\end{itemize}

If both the VIRTIO_NET_HDR_GSO_UDP_TUNNEL_IPV4 bit and
the VIRTIO_NET_HDR_GSO_UDP_TUNNEL_IPV6 bit in in \field{gso_type} are set,
the device MUST NOT accept the packet.

If the VIRTIO_NET_HDR_GSO_UDP_TUNNEL_IPV4 bit and the VIRTIO_NET_HDR_GSO_UDP_TUNNEL_IPV6
bit in \field{gso_type} are not set, the device MUST NOT use the
\field{outer_th_offset} and \field{inner_nh_offset}.

If either the VIRTIO_NET_HDR_GSO_UDP_TUNNEL_IPV4 bit or
the VIRTIO_NET_HDR_GSO_UDP_TUNNEL_IPV6 bit in \field{gso_type} are set, and any of
the following is true:
\begin{itemize}
\item the VIRTIO_NET_HDR_F_NEEDS_CSUM is not set in \field{flags}
\item the VIRTIO_NET_HDR_F_DATA_VALID is set in \field{flags}
\item the \field{gso_type} excluding the VIRTIO_NET_HDR_GSO_UDP_TUNNEL_IPV4
bit and the VIRTIO_NET_HDR_GSO_UDP_TUNNEL_IPV6 bit is VIRTIO_NET_HDR_GSO_NONE
\end{itemize}
the device MUST NOT accept the packet.

If the VIRTIO_NET_HDR_F_UDP_TUNNEL_CSUM bit in \field{flags} is set,
and both the bits VIRTIO_NET_HDR_GSO_UDP_TUNNEL_IPV4 and
VIRTIO_NET_HDR_GSO_UDP_TUNNEL_IPV6 in \field{gso_type} are not set,
the device MOST NOT accept the packet.

If VIRTIO_NET_HDR_F_NEEDS_CSUM is not set, the device MUST NOT
rely on the packet checksum being correct.
\paragraph{Packet Transmission Interrupt}\label{sec:Device Types / Network Device / Device Operation / Packet Transmission / Packet Transmission Interrupt}

Often a driver will suppress transmission virtqueue interrupts
and check for used packets in the transmit path of following
packets.

The normal behavior in this interrupt handler is to retrieve
used buffers from the virtqueue and free the corresponding
headers and packets.

\subsubsection{Setting Up Receive Buffers}\label{sec:Device Types / Network Device / Device Operation / Setting Up Receive Buffers}

It is generally a good idea to keep the receive virtqueue as
fully populated as possible: if it runs out, network performance
will suffer.

If the VIRTIO_NET_F_GUEST_TSO4, VIRTIO_NET_F_GUEST_TSO6,
VIRTIO_NET_F_GUEST_UFO, VIRTIO_NET_F_GUEST_USO4 or VIRTIO_NET_F_GUEST_USO6
features are used, the maximum incoming packet
will be 65589 bytes long (14 bytes of Ethernet header, plus 40 bytes of
the IPv6 header, plus 65535 bytes of maximum IPv6 payload including any
extension header), otherwise 1514 bytes.
When VIRTIO_NET_F_HASH_REPORT is not negotiated, the required receive buffer
size is either 65601 or 1526 bytes accounting for 20 bytes of
\field{struct virtio_net_hdr} followed by receive packet.
When VIRTIO_NET_F_HASH_REPORT is negotiated, the required receive buffer
size is either 65609 or 1534 bytes accounting for 12 bytes of
\field{struct virtio_net_hdr} followed by receive packet.

\drivernormative{\paragraph}{Setting Up Receive Buffers}{Device Types / Network Device / Device Operation / Setting Up Receive Buffers}

\begin{itemize}
\item If VIRTIO_NET_F_MRG_RXBUF is not negotiated:
  \begin{itemize}
    \item If VIRTIO_NET_F_GUEST_TSO4, VIRTIO_NET_F_GUEST_TSO6, VIRTIO_NET_F_GUEST_UFO,
	VIRTIO_NET_F_GUEST_USO4 or VIRTIO_NET_F_GUEST_USO6 are negotiated, the driver SHOULD populate
      the receive queue(s) with buffers of at least 65609 bytes if
      VIRTIO_NET_F_HASH_REPORT is negotiated, and of at least 65601 bytes if not.
    \item Otherwise, the driver SHOULD populate the receive queue(s)
      with buffers of at least 1534 bytes if VIRTIO_NET_F_HASH_REPORT
      is negotiated, and of at least 1526 bytes if not.
  \end{itemize}
\item If VIRTIO_NET_F_MRG_RXBUF is negotiated, each buffer MUST be at
least size of \field{struct virtio_net_hdr},
i.e. 20 bytes if VIRTIO_NET_F_HASH_REPORT is negotiated, and 12 bytes if not.
\end{itemize}

\begin{note}
Obviously each buffer can be split across multiple descriptor elements.
\end{note}

When calculating the size of \field{struct virtio_net_hdr}, the driver
MUST consider all the fields inclusive up to \field{padding_reserved},
i.e. 20 bytes if VIRTIO_NET_F_HASH_REPORT is negotiated, and 12 bytes if not.

If VIRTIO_NET_F_MQ is negotiated, each of receiveq1\ldots receiveqN
that will be used SHOULD be populated with receive buffers.

\devicenormative{\paragraph}{Setting Up Receive Buffers}{Device Types / Network Device / Device Operation / Setting Up Receive Buffers}

The device MUST set \field{num_buffers} to the number of descriptors used to
hold the incoming packet.

The device MUST use only a single descriptor if VIRTIO_NET_F_MRG_RXBUF
was not negotiated.
\begin{note}
{This means that \field{num_buffers} will always be 1
if VIRTIO_NET_F_MRG_RXBUF is not negotiated.}
\end{note}

\subsubsection{Processing of Incoming Packets}\label{sec:Device Types / Network Device / Device Operation / Processing of Incoming Packets}
\label{sec:Device Types / Network Device / Device Operation / Processing of Packets}%old label for latexdiff

When a packet is copied into a buffer in the receiveq, the
optimal path is to disable further used buffer notifications for the
receiveq and process packets until no more are found, then re-enable
them.

Processing incoming packets involves:

\begin{enumerate}
\item \field{num_buffers} indicates how many descriptors
  this packet is spread over (including this one): this will
  always be 1 if VIRTIO_NET_F_MRG_RXBUF was not negotiated.
  This allows receipt of large packets without having to allocate large
  buffers: a packet that does not fit in a single buffer can flow
  over to the next buffer, and so on. In this case, there will be
  at least \field{num_buffers} used buffers in the virtqueue, and the device
  chains them together to form a single packet in a way similar to
  how it would store it in a single buffer spread over multiple
  descriptors.
  The other buffers will not begin with a \field{struct virtio_net_hdr}.

\item If
  \field{num_buffers} is one, then the entire packet will be
  contained within this buffer, immediately following the struct
  virtio_net_hdr.
\item If the VIRTIO_NET_F_GUEST_CSUM feature was negotiated, the
  VIRTIO_NET_HDR_F_DATA_VALID bit in \field{flags} can be
  set: if so, device has validated the packet checksum.
  If the VIRTIO_NET_F_GUEST_UDP_TUNNEL_GSO_CSUM feature has been negotiated,
  and the VIRTIO_NET_HDR_F_UDP_TUNNEL_CSUM bit is set in \field{flags},
  both the outer UDP checksum and the inner transport checksum
  have been validated, otherwise only one level of checksums (the outer one
  in case of tunnels) has been validated.
\end{enumerate}

Additionally, VIRTIO_NET_F_GUEST_CSUM, TSO4, TSO6, UDP, UDP_TUNNEL
and ECN features enable receive checksum, large receive offload and ECN
support which are the input equivalents of the transmit checksum,
transmit segmentation offloading and ECN features, as described
in \ref{sec:Device Types / Network Device / Device Operation /
Packet Transmission}:
\begin{enumerate}
\item If the VIRTIO_NET_F_GUEST_TSO4, TSO6, UFO, USO4 or USO6 options were
  negotiated, then \field{gso_type} MAY be something other than
  VIRTIO_NET_HDR_GSO_NONE, and \field{gso_size} field indicates the
  desired MSS (see Packet Transmission point 2).
\item If the VIRTIO_NET_F_RSC_EXT option was negotiated (this
  implies one of VIRTIO_NET_F_GUEST_TSO4, TSO6), the
  device processes also duplicated ACK segments, reports
  number of coalesced TCP segments in \field{csum_start} field and
  number of duplicated ACK segments in \field{csum_offset} field
  and sets bit VIRTIO_NET_HDR_F_RSC_INFO in \field{flags}.
\item If the VIRTIO_NET_F_GUEST_CSUM feature was negotiated, the
  VIRTIO_NET_HDR_F_NEEDS_CSUM bit in \field{flags} can be
  set: if so, the packet checksum at offset \field{csum_offset}
  from \field{csum_start} and any preceding checksums
  have been validated.  The checksum on the packet is incomplete and
  if bit VIRTIO_NET_HDR_F_RSC_INFO is not set in \field{flags},
  then \field{csum_start} and \field{csum_offset} indicate how to calculate it
  (see Packet Transmission point 1).
\begin{note}
If \field{gso_type} differs from VIRTIO_NET_HDR_GSO_NONE and the
VIRTIO_NET_HDR_GSO_UDP_TUNNEL_IPV4 bit or the VIRTIO_NET_HDR_GSO_UDP_TUNNEL_IPV6
bit are not set, \field{csum_offset}
points to the only transport header present in the packet, and there are no
additional preceding checksums validated by VIRTIO_NET_HDR_F_NEEDS_CSUM.
\end{note}
\item If the VIRTIO_NET_F_GUEST_UDP_TUNNEL_GSO option was negotiated and
  \field{gso_type} is not VIRTIO_NET_HDR_GSO_NONE, the
  VIRTIO_NET_HDR_GSO_UDP_TUNNEL_IPV4 bit or the VIRTIO_NET_HDR_GSO_UDP_TUNNEL_IPV6
  bit MAY be set. In such case the \field{outer_th_offset} and
  \field{inner_nh_offset} fields indicate the corresponding
  headers information.
\item If the VIRTIO_NET_F_GUEST_UDP_TUNNEL_GSO_CSUM feature was
negotiated, and
  the VIRTIO_NET_HDR_GSO_UDP_TUNNEL_IPV4 bit or the VIRTIO_NET_HDR_GSO_UDP_TUNNEL_IPV6
  are set in \field{gso_type}, the VIRTIO_NET_HDR_F_UDP_TUNNEL_CSUM bit in the
  \field{flags} can be set: if so, the outer UDP checksum has been validated
  and the UDP header checksum at offset 6 from from \field{outer_th_offset}
  is set to the outer UDP pseudo header checksum.

\begin{note}
If the VIRTIO_NET_HDR_GSO_UDP_TUNNEL_IPV4 bit or VIRTIO_NET_HDR_GSO_UDP_TUNNEL_IPV6
bit are set in \field{gso_type}, the \field{csum_start} field refers to
the inner transport header offset (see Packet Transmission point 1).
If the VIRTIO_NET_HDR_F_UDP_TUNNEL_CSUM bit in \field{flags} is set both
the inner and the outer header checksums have been validated by
VIRTIO_NET_HDR_F_NEEDS_CSUM, otherwise only the inner transport header
checksum has been validated.
\end{note}
\end{enumerate}

If applicable, the device calculates per-packet hash for incoming packets as
defined in \ref{sec:Device Types / Network Device / Device Operation / Processing of Incoming Packets / Hash calculation for incoming packets}.

If applicable, the device reports hash information for incoming packets as
defined in \ref{sec:Device Types / Network Device / Device Operation / Processing of Incoming Packets / Hash reporting for incoming packets}.

\devicenormative{\paragraph}{Processing of Incoming Packets}{Device Types / Network Device / Device Operation / Processing of Incoming Packets}
\label{devicenormative:Device Types / Network Device / Device Operation / Processing of Packets}%old label for latexdiff

If VIRTIO_NET_F_MRG_RXBUF has not been negotiated, the device MUST set
\field{num_buffers} to 1.

If VIRTIO_NET_F_MRG_RXBUF has been negotiated, the device MUST set
\field{num_buffers} to indicate the number of buffers
the packet (including the header) is spread over.

If a receive packet is spread over multiple buffers, the device
MUST use all buffers but the last (i.e. the first \field{num_buffers} -
1 buffers) completely up to the full length of each buffer
supplied by the driver.

The device MUST use all buffers used by a single receive
packet together, such that at least \field{num_buffers} are
observed by driver as used.

If VIRTIO_NET_F_GUEST_CSUM is not negotiated, the device MUST set
\field{flags} to zero and SHOULD supply a fully checksummed
packet to the driver.

If VIRTIO_NET_F_GUEST_TSO4 is not negotiated, the device MUST NOT set
\field{gso_type} to VIRTIO_NET_HDR_GSO_TCPV4.

If VIRTIO_NET_F_GUEST_UDP is not negotiated, the device MUST NOT set
\field{gso_type} to VIRTIO_NET_HDR_GSO_UDP.

If VIRTIO_NET_F_GUEST_TSO6 is not negotiated, the device MUST NOT set
\field{gso_type} to VIRTIO_NET_HDR_GSO_TCPV6.

If none of VIRTIO_NET_F_GUEST_USO4 or VIRTIO_NET_F_GUEST_USO6 have been negotiated,
the device MUST NOT set \field{gso_type} to VIRTIO_NET_HDR_GSO_UDP_L4.

If VIRTIO_NET_F_GUEST_UDP_TUNNEL_GSO is not negotiated, the device MUST NOT set
either the VIRTIO_NET_HDR_GSO_UDP_TUNNEL_IPV4 bit or the
VIRTIO_NET_HDR_GSO_UDP_TUNNEL_IPV6 bit in \field{gso_type}.

If VIRTIO_NET_F_GUEST_UDP_TUNNEL_GSO_CSUM is not negotiated the device MUST NOT set
the VIRTIO_NET_HDR_F_UDP_TUNNEL_CSUM bit in \field{flags}.

The device SHOULD NOT send to the driver TCP packets requiring segmentation offload
which have the Explicit Congestion Notification bit set, unless the
VIRTIO_NET_F_GUEST_ECN feature is negotiated, in which case the
device MUST set the VIRTIO_NET_HDR_GSO_ECN bit in
\field{gso_type}.

If the VIRTIO_NET_F_GUEST_CSUM feature has been negotiated, the
device MAY set the VIRTIO_NET_HDR_F_NEEDS_CSUM bit in
\field{flags}, if so:
\begin{enumerate}
\item the device MUST validate the packet checksum at
	offset \field{csum_offset} from \field{csum_start} as well as all
	preceding offsets;
\item the device MUST set the packet checksum stored in the
	receive buffer to the TCP/UDP pseudo header;
\item the device MUST set \field{csum_start} and
	\field{csum_offset} such that calculating a ones'
	complement checksum from \field{csum_start} up until the
	end of the packet and storing the result at offset
	\field{csum_offset} from  \field{csum_start} will result in a
	fully checksummed packet;
\end{enumerate}

The device MUST NOT send to the driver GSO packets encapsulated in UDP
tunnel and requiring segmentation offload, unless the
VIRTIO_NET_F_GUEST_UDP_TUNNEL_GSO is negotiated, in which case the device MUST set
the VIRTIO_NET_HDR_GSO_UDP_TUNNEL_IPV4 bit or the VIRTIO_NET_HDR_GSO_UDP_TUNNEL_IPV6
bit in \field{gso_type} according to the inner network header protocol type,
MUST set the \field{outer_th_offset} and \field{inner_nh_offset} fields
to the corresponding header information, and the outer UDP header MUST NOT
require checksum offload.

If the VIRTIO_NET_F_GUEST_UDP_TUNNEL_GSO_CSUM feature has not been negotiated,
the device MUST NOT send the driver GSO packets encapsulated in UDP
tunnel and requiring segmentation and outer checksum offload.

If none of the VIRTIO_NET_F_GUEST_TSO4, TSO6, UFO, USO4 or USO6 options have
been negotiated, the device MUST set \field{gso_type} to
VIRTIO_NET_HDR_GSO_NONE.

If \field{gso_type} differs from VIRTIO_NET_HDR_GSO_NONE, then
the device MUST also set the VIRTIO_NET_HDR_F_NEEDS_CSUM bit in
\field{flags} MUST set \field{gso_size} to indicate the desired MSS.
If VIRTIO_NET_F_RSC_EXT was negotiated, the device MUST also
set VIRTIO_NET_HDR_F_RSC_INFO bit in \field{flags},
set \field{csum_start} to number of coalesced TCP segments and
set \field{csum_offset} to number of received duplicated ACK segments.

If VIRTIO_NET_F_RSC_EXT was not negotiated, the device MUST
not set VIRTIO_NET_HDR_F_RSC_INFO bit in \field{flags}.

If one of the VIRTIO_NET_F_GUEST_TSO4, TSO6, UFO, USO4 or USO6 options have
been negotiated, the device SHOULD set \field{hdr_len} to a value
not less than the length of the headers, including the transport
header. If \field{gso_type} has the VIRTIO_NET_HDR_GSO_UDP_TUNNEL_IPV4 bit
or the VIRTIO_NET_HDR_GSO_UDP_TUNNEL_IPV6 bit set, the referenced transport
header is the inner one.

If the VIRTIO_NET_F_GUEST_CSUM feature has been negotiated, the
device MAY set the VIRTIO_NET_HDR_F_DATA_VALID bit in
\field{flags}, if so, the device MUST validate the packet
checksum. If the VIRTIO_NET_F_GUEST_UDP_TUNNEL_GSO_CSUM feature has
been negotiated, and the VIRTIO_NET_HDR_F_UDP_TUNNEL_CSUM bit set in
\field{flags}, both the outer UDP checksum and the inner transport
checksum have been validated.
Otherwise level of checksum is validated: in case of multiple
encapsulated protocols the outermost one.

If either the VIRTIO_NET_HDR_GSO_UDP_TUNNEL_IPV4 bit or the
VIRTIO_NET_HDR_GSO_UDP_TUNNEL_IPV6 bit in \field{gso_type} are set,
the device MUST NOT set the VIRTIO_NET_HDR_F_DATA_VALID bit in
\field{flags}.

If the VIRTIO_NET_F_GUEST_UDP_TUNNEL_GSO_CSUM feature has been negotiated
and either the VIRTIO_NET_HDR_GSO_UDP_TUNNEL_IPV4 bit is set or the
VIRTIO_NET_HDR_GSO_UDP_TUNNEL_IPV6 bit is set in \field{gso_type}, the
device MAY set the VIRTIO_NET_HDR_F_UDP_TUNNEL_CSUM bit in
\field{flags}, if so the device MUST set the packet outer UDP checksum
stored in the receive buffer to the outer UDP pseudo header.

Otherwise, the VIRTIO_NET_F_GUEST_UDP_TUNNEL_GSO_CSUM feature has been
negotiated, either the VIRTIO_NET_HDR_GSO_UDP_TUNNEL_IPV4 bit is set or the
VIRTIO_NET_HDR_GSO_UDP_TUNNEL_IPV6 bit is set in \field{gso_type},
and the bit VIRTIO_NET_HDR_F_UDP_TUNNEL_CSUM is not set in
\field{flags}, the device MUST either provide a zero outer UDP header
checksum or a fully checksummed outer UDP header.

\drivernormative{\paragraph}{Processing of Incoming
Packets}{Device Types / Network Device / Device Operation /
Processing of Incoming Packets}

The driver MUST ignore \field{flag} bits that it does not recognize.

If VIRTIO_NET_HDR_F_NEEDS_CSUM bit in \field{flags} is not set or
if VIRTIO_NET_HDR_F_RSC_INFO bit \field{flags} is set, the
driver MUST NOT use the \field{csum_start} and \field{csum_offset}.

If one of the VIRTIO_NET_F_GUEST_TSO4, TSO6, UFO, USO4 or USO6 options have
been negotiated, the driver MAY use \field{hdr_len} only as a hint about the
transport header size.
The driver MUST NOT rely on \field{hdr_len} to be correct.
\begin{note}
This is due to various bugs in implementations.
\end{note}

If neither VIRTIO_NET_HDR_F_NEEDS_CSUM nor
VIRTIO_NET_HDR_F_DATA_VALID is set, the driver MUST NOT
rely on the packet checksum being correct.

If both the VIRTIO_NET_HDR_GSO_UDP_TUNNEL_IPV4 bit and
the VIRTIO_NET_HDR_GSO_UDP_TUNNEL_IPV6 bit in in \field{gso_type} are set,
the driver MUST NOT accept the packet.

If the VIRTIO_NET_HDR_GSO_UDP_TUNNEL_IPV4 bit or the VIRTIO_NET_HDR_GSO_UDP_TUNNEL_IPV6
bit in \field{gso_type} are not set, the driver MUST NOT use the
\field{outer_th_offset} and \field{inner_nh_offset}.

If either the VIRTIO_NET_HDR_GSO_UDP_TUNNEL_IPV4 bit or
the VIRTIO_NET_HDR_GSO_UDP_TUNNEL_IPV6 bit in \field{gso_type} are set, and any of
the following is true:
\begin{itemize}
\item the VIRTIO_NET_HDR_F_NEEDS_CSUM bit is not set in \field{flags}
\item the VIRTIO_NET_HDR_F_DATA_VALID bit is set in \field{flags}
\item the \field{gso_type} excluding the VIRTIO_NET_HDR_GSO_UDP_TUNNEL_IPV4
bit and the VIRTIO_NET_HDR_GSO_UDP_TUNNEL_IPV6 bit is VIRTIO_NET_HDR_GSO_NONE
\end{itemize}
the driver MUST NOT accept the packet.

If the VIRTIO_NET_HDR_F_UDP_TUNNEL_CSUM bit and the VIRTIO_NET_HDR_F_NEEDS_CSUM
bit in \field{flags} are set,
and both the bits VIRTIO_NET_HDR_GSO_UDP_TUNNEL_IPV4 and
VIRTIO_NET_HDR_GSO_UDP_TUNNEL_IPV6 in \field{gso_type} are not set,
the driver MOST NOT accept the packet.

\paragraph{Hash calculation for incoming packets}
\label{sec:Device Types / Network Device / Device Operation / Processing of Incoming Packets / Hash calculation for incoming packets}

A device attempts to calculate a per-packet hash in the following cases:
\begin{itemize}
\item The feature VIRTIO_NET_F_RSS was negotiated. The device uses the hash to determine the receive virtqueue to place incoming packets.
\item The feature VIRTIO_NET_F_HASH_REPORT was negotiated. The device reports the hash value and the hash type with the packet.
\end{itemize}

If the feature VIRTIO_NET_F_RSS was negotiated:
\begin{itemize}
\item The device uses \field{hash_types} of the virtio_net_rss_config structure as 'Enabled hash types' bitmask.
\item If additionally the feature VIRTIO_NET_F_HASH_TUNNEL was negotiated, the device uses \field{enabled_tunnel_types} of the
      virtnet_hash_tunnel structure as 'Encapsulation types enabled for inner header hash' bitmask.
\item The device uses a key as defined in \field{hash_key_data} and \field{hash_key_length} of the virtio_net_rss_config structure (see
\ref{sec:Device Types / Network Device / Device Operation / Control Virtqueue / Receive-side scaling (RSS) / Setting RSS parameters}).
\end{itemize}

If the feature VIRTIO_NET_F_RSS was not negotiated:
\begin{itemize}
\item The device uses \field{hash_types} of the virtio_net_hash_config structure as 'Enabled hash types' bitmask.
\item If additionally the feature VIRTIO_NET_F_HASH_TUNNEL was negotiated, the device uses \field{enabled_tunnel_types} of the
      virtnet_hash_tunnel structure as 'Encapsulation types enabled for inner header hash' bitmask.
\item The device uses a key as defined in \field{hash_key_data} and \field{hash_key_length} of the virtio_net_hash_config structure (see
\ref{sec:Device Types / Network Device / Device Operation / Control Virtqueue / Automatic receive steering in multiqueue mode / Hash calculation}).
\end{itemize}

Note that if the device offers VIRTIO_NET_F_HASH_REPORT, even if it supports only one pair of virtqueues, it MUST support
at least one of commands of VIRTIO_NET_CTRL_MQ class to configure reported hash parameters:
\begin{itemize}
\item If the device offers VIRTIO_NET_F_RSS, it MUST support VIRTIO_NET_CTRL_MQ_RSS_CONFIG command per
 \ref{sec:Device Types / Network Device / Device Operation / Control Virtqueue / Receive-side scaling (RSS) / Setting RSS parameters}.
\item Otherwise the device MUST support VIRTIO_NET_CTRL_MQ_HASH_CONFIG command per
 \ref{sec:Device Types / Network Device / Device Operation / Control Virtqueue / Automatic receive steering in multiqueue mode / Hash calculation}.
\end{itemize}

The per-packet hash calculation can depend on the IP packet type. See
\hyperref[intro:IP]{[IP]}, \hyperref[intro:UDP]{[UDP]} and \hyperref[intro:TCP]{[TCP]}.

\subparagraph{Supported/enabled hash types}
\label{sec:Device Types / Network Device / Device Operation / Processing of Incoming Packets / Hash calculation for incoming packets / Supported/enabled hash types}
Hash types applicable for IPv4 packets:
\begin{lstlisting}
#define VIRTIO_NET_HASH_TYPE_IPv4              (1 << 0)
#define VIRTIO_NET_HASH_TYPE_TCPv4             (1 << 1)
#define VIRTIO_NET_HASH_TYPE_UDPv4             (1 << 2)
\end{lstlisting}
Hash types applicable for IPv6 packets without extension headers
\begin{lstlisting}
#define VIRTIO_NET_HASH_TYPE_IPv6              (1 << 3)
#define VIRTIO_NET_HASH_TYPE_TCPv6             (1 << 4)
#define VIRTIO_NET_HASH_TYPE_UDPv6             (1 << 5)
\end{lstlisting}
Hash types applicable for IPv6 packets with extension headers
\begin{lstlisting}
#define VIRTIO_NET_HASH_TYPE_IP_EX             (1 << 6)
#define VIRTIO_NET_HASH_TYPE_TCP_EX            (1 << 7)
#define VIRTIO_NET_HASH_TYPE_UDP_EX            (1 << 8)
\end{lstlisting}

\subparagraph{IPv4 packets}
\label{sec:Device Types / Network Device / Device Operation / Processing of Incoming Packets / Hash calculation for incoming packets / IPv4 packets}
The device calculates the hash on IPv4 packets according to 'Enabled hash types' bitmask as follows:
\begin{itemize}
\item If VIRTIO_NET_HASH_TYPE_TCPv4 is set and the packet has
a TCP header, the hash is calculated over the following fields:
\begin{itemize}
\item Source IP address
\item Destination IP address
\item Source TCP port
\item Destination TCP port
\end{itemize}
\item Else if VIRTIO_NET_HASH_TYPE_UDPv4 is set and the
packet has a UDP header, the hash is calculated over the following fields:
\begin{itemize}
\item Source IP address
\item Destination IP address
\item Source UDP port
\item Destination UDP port
\end{itemize}
\item Else if VIRTIO_NET_HASH_TYPE_IPv4 is set, the hash is
calculated over the following fields:
\begin{itemize}
\item Source IP address
\item Destination IP address
\end{itemize}
\item Else the device does not calculate the hash
\end{itemize}

\subparagraph{IPv6 packets without extension header}
\label{sec:Device Types / Network Device / Device Operation / Processing of Incoming Packets / Hash calculation for incoming packets / IPv6 packets without extension header}
The device calculates the hash on IPv6 packets without extension
headers according to 'Enabled hash types' bitmask as follows:
\begin{itemize}
\item If VIRTIO_NET_HASH_TYPE_TCPv6 is set and the packet has
a TCPv6 header, the hash is calculated over the following fields:
\begin{itemize}
\item Source IPv6 address
\item Destination IPv6 address
\item Source TCP port
\item Destination TCP port
\end{itemize}
\item Else if VIRTIO_NET_HASH_TYPE_UDPv6 is set and the
packet has a UDPv6 header, the hash is calculated over the following fields:
\begin{itemize}
\item Source IPv6 address
\item Destination IPv6 address
\item Source UDP port
\item Destination UDP port
\end{itemize}
\item Else if VIRTIO_NET_HASH_TYPE_IPv6 is set, the hash is
calculated over the following fields:
\begin{itemize}
\item Source IPv6 address
\item Destination IPv6 address
\end{itemize}
\item Else the device does not calculate the hash
\end{itemize}

\subparagraph{IPv6 packets with extension header}
\label{sec:Device Types / Network Device / Device Operation / Processing of Incoming Packets / Hash calculation for incoming packets / IPv6 packets with extension header}
The device calculates the hash on IPv6 packets with extension
headers according to 'Enabled hash types' bitmask as follows:
\begin{itemize}
\item If VIRTIO_NET_HASH_TYPE_TCP_EX is set and the packet
has a TCPv6 header, the hash is calculated over the following fields:
\begin{itemize}
\item Home address from the home address option in the IPv6 destination options header. If the extension header is not present, use the Source IPv6 address.
\item IPv6 address that is contained in the Routing-Header-Type-2 from the associated extension header. If the extension header is not present, use the Destination IPv6 address.
\item Source TCP port
\item Destination TCP port
\end{itemize}
\item Else if VIRTIO_NET_HASH_TYPE_UDP_EX is set and the
packet has a UDPv6 header, the hash is calculated over the following fields:
\begin{itemize}
\item Home address from the home address option in the IPv6 destination options header. If the extension header is not present, use the Source IPv6 address.
\item IPv6 address that is contained in the Routing-Header-Type-2 from the associated extension header. If the extension header is not present, use the Destination IPv6 address.
\item Source UDP port
\item Destination UDP port
\end{itemize}
\item Else if VIRTIO_NET_HASH_TYPE_IP_EX is set, the hash is
calculated over the following fields:
\begin{itemize}
\item Home address from the home address option in the IPv6 destination options header. If the extension header is not present, use the Source IPv6 address.
\item IPv6 address that is contained in the Routing-Header-Type-2 from the associated extension header. If the extension header is not present, use the Destination IPv6 address.
\end{itemize}
\item Else skip IPv6 extension headers and calculate the hash as
defined for an IPv6 packet without extension headers
(see \ref{sec:Device Types / Network Device / Device Operation / Processing of Incoming Packets / Hash calculation for incoming packets / IPv6 packets without extension header}).
\end{itemize}

\paragraph{Inner Header Hash}
\label{sec:Device Types / Network Device / Device Operation / Processing of Incoming Packets / Inner Header Hash}

If VIRTIO_NET_F_HASH_TUNNEL has been negotiated, the driver can send the command
VIRTIO_NET_CTRL_HASH_TUNNEL_SET to configure the calculation of the inner header hash.

\begin{lstlisting}
struct virtnet_hash_tunnel {
    le32 enabled_tunnel_types;
};

#define VIRTIO_NET_CTRL_HASH_TUNNEL 7
 #define VIRTIO_NET_CTRL_HASH_TUNNEL_SET 0
\end{lstlisting}

Field \field{enabled_tunnel_types} contains the bitmask of encapsulation types enabled for inner header hash.
See \ref{sec:Device Types / Network Device / Device Operation / Processing of Incoming Packets /
Hash calculation for incoming packets / Encapsulation types supported/enabled for inner header hash}.

The class VIRTIO_NET_CTRL_HASH_TUNNEL has one command:
VIRTIO_NET_CTRL_HASH_TUNNEL_SET sets \field{enabled_tunnel_types} for the device using the
virtnet_hash_tunnel structure, which is read-only for the device.

Inner header hash is disabled by VIRTIO_NET_CTRL_HASH_TUNNEL_SET with \field{enabled_tunnel_types} set to 0.

Initially (before the driver sends any VIRTIO_NET_CTRL_HASH_TUNNEL_SET command) all
encapsulation types are disabled for inner header hash.

\subparagraph{Encapsulated packet}
\label{sec:Device Types / Network Device / Device Operation / Processing of Incoming Packets / Hash calculation for incoming packets / Encapsulated packet}

Multiple tunneling protocols allow encapsulating an inner, payload packet in an outer, encapsulated packet.
The encapsulated packet thus contains an outer header and an inner header, and the device calculates the
hash over either the inner header or the outer header.

If VIRTIO_NET_F_HASH_TUNNEL is negotiated and a received encapsulated packet's outer header matches one of the
encapsulation types enabled in \field{enabled_tunnel_types}, then the device uses the inner header for hash
calculations (only a single level of encapsulation is currently supported).

If VIRTIO_NET_F_HASH_TUNNEL is negotiated and a received packet's (outer) header does not match any encapsulation
types enabled in \field{enabled_tunnel_types}, then the device uses the outer header for hash calculations.

\subparagraph{Encapsulation types supported/enabled for inner header hash}
\label{sec:Device Types / Network Device / Device Operation / Processing of Incoming Packets /
Hash calculation for incoming packets / Encapsulation types supported/enabled for inner header hash}

Encapsulation types applicable for inner header hash:
\begin{lstlisting}[escapechar=|]
#define VIRTIO_NET_HASH_TUNNEL_TYPE_GRE_2784    (1 << 0) /* |\hyperref[intro:rfc2784]{[RFC2784]}| */
#define VIRTIO_NET_HASH_TUNNEL_TYPE_GRE_2890    (1 << 1) /* |\hyperref[intro:rfc2890]{[RFC2890]}| */
#define VIRTIO_NET_HASH_TUNNEL_TYPE_GRE_7676    (1 << 2) /* |\hyperref[intro:rfc7676]{[RFC7676]}| */
#define VIRTIO_NET_HASH_TUNNEL_TYPE_GRE_UDP     (1 << 3) /* |\hyperref[intro:rfc8086]{[GRE-in-UDP]}| */
#define VIRTIO_NET_HASH_TUNNEL_TYPE_VXLAN       (1 << 4) /* |\hyperref[intro:vxlan]{[VXLAN]}| */
#define VIRTIO_NET_HASH_TUNNEL_TYPE_VXLAN_GPE   (1 << 5) /* |\hyperref[intro:vxlan-gpe]{[VXLAN-GPE]}| */
#define VIRTIO_NET_HASH_TUNNEL_TYPE_GENEVE      (1 << 6) /* |\hyperref[intro:geneve]{[GENEVE]}| */
#define VIRTIO_NET_HASH_TUNNEL_TYPE_IPIP        (1 << 7) /* |\hyperref[intro:ipip]{[IPIP]}| */
#define VIRTIO_NET_HASH_TUNNEL_TYPE_NVGRE       (1 << 8) /* |\hyperref[intro:nvgre]{[NVGRE]}| */
\end{lstlisting}

\subparagraph{Advice}
Example uses of the inner header hash:
\begin{itemize}
\item Legacy tunneling protocols, lacking the outer header entropy, can use RSS with the inner header hash to
      distribute flows with identical outer but different inner headers across various queues, improving performance.
\item Identify an inner flow distributed across multiple outer tunnels.
\end{itemize}

As using the inner header hash completely discards the outer header entropy, care must be taken
if the inner header is controlled by an adversary, as the adversary can then intentionally create
configurations with insufficient entropy.

Besides disabling the inner header hash, mitigations would depend on how the hash is used. When the hash
use is limited to the RSS queue selection, the inner header hash may have quality of service (QoS) limitations.

\devicenormative{\subparagraph}{Inner Header Hash}{Device Types / Network Device / Device Operation / Control Virtqueue / Inner Header Hash}

If the (outer) header of the received packet does not match any encapsulation types enabled
in \field{enabled_tunnel_types}, the device MUST calculate the hash on the outer header.

If the device receives any bits in \field{enabled_tunnel_types} which are not set in \field{supported_tunnel_types},
it SHOULD respond to the VIRTIO_NET_CTRL_HASH_TUNNEL_SET command with VIRTIO_NET_ERR.

If the driver sets \field{enabled_tunnel_types} to 0 through VIRTIO_NET_CTRL_HASH_TUNNEL_SET or upon the device reset,
the device MUST disable the inner header hash for all encapsulation types.

\drivernormative{\subparagraph}{Inner Header Hash}{Device Types / Network Device / Device Operation / Control Virtqueue / Inner Header Hash}

The driver MUST have negotiated the VIRTIO_NET_F_HASH_TUNNEL feature when issuing the VIRTIO_NET_CTRL_HASH_TUNNEL_SET command.

The driver MUST NOT set any bits in \field{enabled_tunnel_types} which are not set in \field{supported_tunnel_types}.

The driver MUST ignore bits in \field{supported_tunnel_types} which are not documented in this specification.

\paragraph{Hash reporting for incoming packets}
\label{sec:Device Types / Network Device / Device Operation / Processing of Incoming Packets / Hash reporting for incoming packets}

If VIRTIO_NET_F_HASH_REPORT was negotiated and
 the device has calculated the hash for the packet, the device fills \field{hash_report} with the report type of calculated hash
and \field{hash_value} with the value of calculated hash.

If VIRTIO_NET_F_HASH_REPORT was negotiated but due to any reason the
hash was not calculated, the device sets \field{hash_report} to VIRTIO_NET_HASH_REPORT_NONE.

Possible values that the device can report in \field{hash_report} are defined below.
They correspond to supported hash types defined in
\ref{sec:Device Types / Network Device / Device Operation / Processing of Incoming Packets / Hash calculation for incoming packets / Supported/enabled hash types}
as follows:

VIRTIO_NET_HASH_TYPE_XXX = 1 << (VIRTIO_NET_HASH_REPORT_XXX - 1)

\begin{lstlisting}
#define VIRTIO_NET_HASH_REPORT_NONE            0
#define VIRTIO_NET_HASH_REPORT_IPv4            1
#define VIRTIO_NET_HASH_REPORT_TCPv4           2
#define VIRTIO_NET_HASH_REPORT_UDPv4           3
#define VIRTIO_NET_HASH_REPORT_IPv6            4
#define VIRTIO_NET_HASH_REPORT_TCPv6           5
#define VIRTIO_NET_HASH_REPORT_UDPv6           6
#define VIRTIO_NET_HASH_REPORT_IPv6_EX         7
#define VIRTIO_NET_HASH_REPORT_TCPv6_EX        8
#define VIRTIO_NET_HASH_REPORT_UDPv6_EX        9
\end{lstlisting}

\subsubsection{Control Virtqueue}\label{sec:Device Types / Network Device / Device Operation / Control Virtqueue}

The driver uses the control virtqueue (if VIRTIO_NET_F_CTRL_VQ is
negotiated) to send commands to manipulate various features of
the device which would not easily map into the configuration
space.

All commands are of the following form:

\begin{lstlisting}
struct virtio_net_ctrl {
        u8 class;
        u8 command;
        u8 command-specific-data[];
        u8 ack;
        u8 command-specific-result[];
};

/* ack values */
#define VIRTIO_NET_OK     0
#define VIRTIO_NET_ERR    1
\end{lstlisting}

The \field{class}, \field{command} and command-specific-data are set by the
driver, and the device sets the \field{ack} byte and optionally
\field{command-specific-result}. There is little the driver can
do except issue a diagnostic if \field{ack} is not VIRTIO_NET_OK.

The command VIRTIO_NET_CTRL_STATS_QUERY and VIRTIO_NET_CTRL_STATS_GET contain
\field{command-specific-result}.

\paragraph{Packet Receive Filtering}\label{sec:Device Types / Network Device / Device Operation / Control Virtqueue / Packet Receive Filtering}
\label{sec:Device Types / Network Device / Device Operation / Control Virtqueue / Setting Promiscuous Mode}%old label for latexdiff

If the VIRTIO_NET_F_CTRL_RX and VIRTIO_NET_F_CTRL_RX_EXTRA
features are negotiated, the driver can send control commands for
promiscuous mode, multicast, unicast and broadcast receiving.

\begin{note}
In general, these commands are best-effort: unwanted
packets could still arrive.
\end{note}

\begin{lstlisting}
#define VIRTIO_NET_CTRL_RX    0
 #define VIRTIO_NET_CTRL_RX_PROMISC      0
 #define VIRTIO_NET_CTRL_RX_ALLMULTI     1
 #define VIRTIO_NET_CTRL_RX_ALLUNI       2
 #define VIRTIO_NET_CTRL_RX_NOMULTI      3
 #define VIRTIO_NET_CTRL_RX_NOUNI        4
 #define VIRTIO_NET_CTRL_RX_NOBCAST      5
\end{lstlisting}


\devicenormative{\subparagraph}{Packet Receive Filtering}{Device Types / Network Device / Device Operation / Control Virtqueue / Packet Receive Filtering}

If the VIRTIO_NET_F_CTRL_RX feature has been negotiated,
the device MUST support the following VIRTIO_NET_CTRL_RX class
commands:
\begin{itemize}
\item VIRTIO_NET_CTRL_RX_PROMISC turns promiscuous mode on and
off. The command-specific-data is one byte containing 0 (off) or
1 (on). If promiscuous mode is on, the device SHOULD receive all
incoming packets.
This SHOULD take effect even if one of the other modes set by
a VIRTIO_NET_CTRL_RX class command is on.
\item VIRTIO_NET_CTRL_RX_ALLMULTI turns all-multicast receive on and
off. The command-specific-data is one byte containing 0 (off) or
1 (on). When all-multicast receive is on the device SHOULD allow
all incoming multicast packets.
\end{itemize}

If the VIRTIO_NET_F_CTRL_RX_EXTRA feature has been negotiated,
the device MUST support the following VIRTIO_NET_CTRL_RX class
commands:
\begin{itemize}
\item VIRTIO_NET_CTRL_RX_ALLUNI turns all-unicast receive on and
off. The command-specific-data is one byte containing 0 (off) or
1 (on). When all-unicast receive is on the device SHOULD allow
all incoming unicast packets.
\item VIRTIO_NET_CTRL_RX_NOMULTI suppresses multicast receive.
The command-specific-data is one byte containing 0 (multicast
receive allowed) or 1 (multicast receive suppressed).
When multicast receive is suppressed, the device SHOULD NOT
send multicast packets to the driver.
This SHOULD take effect even if VIRTIO_NET_CTRL_RX_ALLMULTI is on.
This filter SHOULD NOT apply to broadcast packets.
\item VIRTIO_NET_CTRL_RX_NOUNI suppresses unicast receive.
The command-specific-data is one byte containing 0 (unicast
receive allowed) or 1 (unicast receive suppressed).
When unicast receive is suppressed, the device SHOULD NOT
send unicast packets to the driver.
This SHOULD take effect even if VIRTIO_NET_CTRL_RX_ALLUNI is on.
\item VIRTIO_NET_CTRL_RX_NOBCAST suppresses broadcast receive.
The command-specific-data is one byte containing 0 (broadcast
receive allowed) or 1 (broadcast receive suppressed).
When broadcast receive is suppressed, the device SHOULD NOT
send broadcast packets to the driver.
This SHOULD take effect even if VIRTIO_NET_CTRL_RX_ALLMULTI is on.
\end{itemize}

\drivernormative{\subparagraph}{Packet Receive Filtering}{Device Types / Network Device / Device Operation / Control Virtqueue / Packet Receive Filtering}

If the VIRTIO_NET_F_CTRL_RX feature has not been negotiated,
the driver MUST NOT issue commands VIRTIO_NET_CTRL_RX_PROMISC or
VIRTIO_NET_CTRL_RX_ALLMULTI.

If the VIRTIO_NET_F_CTRL_RX_EXTRA feature has not been negotiated,
the driver MUST NOT issue commands
 VIRTIO_NET_CTRL_RX_ALLUNI,
 VIRTIO_NET_CTRL_RX_NOMULTI,
 VIRTIO_NET_CTRL_RX_NOUNI or
 VIRTIO_NET_CTRL_RX_NOBCAST.

\paragraph{Setting MAC Address Filtering}\label{sec:Device Types / Network Device / Device Operation / Control Virtqueue / Setting MAC Address Filtering}

If the VIRTIO_NET_F_CTRL_RX feature is negotiated, the driver can
send control commands for MAC address filtering.

\begin{lstlisting}
struct virtio_net_ctrl_mac {
        le32 entries;
        u8 macs[entries][6];
};

#define VIRTIO_NET_CTRL_MAC    1
 #define VIRTIO_NET_CTRL_MAC_TABLE_SET        0
 #define VIRTIO_NET_CTRL_MAC_ADDR_SET         1
\end{lstlisting}

The device can filter incoming packets by any number of destination
MAC addresses\footnote{Since there are no guarantees, it can use a hash filter or
silently switch to allmulti or promiscuous mode if it is given too
many addresses.
}. This table is set using the class
VIRTIO_NET_CTRL_MAC and the command VIRTIO_NET_CTRL_MAC_TABLE_SET. The
command-specific-data is two variable length tables of 6-byte MAC
addresses (as described in struct virtio_net_ctrl_mac). The first table contains unicast addresses, and the second
contains multicast addresses.

The VIRTIO_NET_CTRL_MAC_ADDR_SET command is used to set the
default MAC address which rx filtering
accepts (and if VIRTIO_NET_F_MAC has been negotiated,
this will be reflected in \field{mac} in config space).

The command-specific-data for VIRTIO_NET_CTRL_MAC_ADDR_SET is
the 6-byte MAC address.

\devicenormative{\subparagraph}{Setting MAC Address Filtering}{Device Types / Network Device / Device Operation / Control Virtqueue / Setting MAC Address Filtering}

The device MUST have an empty MAC filtering table on reset.

The device MUST update the MAC filtering table before it consumes
the VIRTIO_NET_CTRL_MAC_TABLE_SET command.

The device MUST update \field{mac} in config space before it consumes
the VIRTIO_NET_CTRL_MAC_ADDR_SET command, if VIRTIO_NET_F_MAC has
been negotiated.

The device SHOULD drop incoming packets which have a destination MAC which
matches neither the \field{mac} (or that set with VIRTIO_NET_CTRL_MAC_ADDR_SET)
nor the MAC filtering table.

\drivernormative{\subparagraph}{Setting MAC Address Filtering}{Device Types / Network Device / Device Operation / Control Virtqueue / Setting MAC Address Filtering}

If VIRTIO_NET_F_CTRL_RX has not been negotiated,
the driver MUST NOT issue VIRTIO_NET_CTRL_MAC class commands.

If VIRTIO_NET_F_CTRL_RX has been negotiated,
the driver SHOULD issue VIRTIO_NET_CTRL_MAC_ADDR_SET
to set the default mac if it is different from \field{mac}.

The driver MUST follow the VIRTIO_NET_CTRL_MAC_TABLE_SET command
by a le32 number, followed by that number of non-multicast
MAC addresses, followed by another le32 number, followed by
that number of multicast addresses.  Either number MAY be 0.

\subparagraph{Legacy Interface: Setting MAC Address Filtering}\label{sec:Device Types / Network Device / Device Operation / Control Virtqueue / Setting MAC Address Filtering / Legacy Interface: Setting MAC Address Filtering}
When using the legacy interface, transitional devices and drivers
MUST format \field{entries} in struct virtio_net_ctrl_mac
according to the native endian of the guest rather than
(necessarily when not using the legacy interface) little-endian.

Legacy drivers that didn't negotiate VIRTIO_NET_F_CTRL_MAC_ADDR
changed \field{mac} in config space when NIC is accepting
incoming packets. These drivers always wrote the mac value from
first to last byte, therefore after detecting such drivers,
a transitional device MAY defer MAC update, or MAY defer
processing incoming packets until driver writes the last byte
of \field{mac} in the config space.

\paragraph{VLAN Filtering}\label{sec:Device Types / Network Device / Device Operation / Control Virtqueue / VLAN Filtering}

If the driver negotiates the VIRTIO_NET_F_CTRL_VLAN feature, it
can control a VLAN filter table in the device. The VLAN filter
table applies only to VLAN tagged packets.

When VIRTIO_NET_F_CTRL_VLAN is negotiated, the device starts with
an empty VLAN filter table.

\begin{note}
Similar to the MAC address based filtering, the VLAN filtering
is also best-effort: unwanted packets could still arrive.
\end{note}

\begin{lstlisting}
#define VIRTIO_NET_CTRL_VLAN       2
 #define VIRTIO_NET_CTRL_VLAN_ADD             0
 #define VIRTIO_NET_CTRL_VLAN_DEL             1
\end{lstlisting}

Both the VIRTIO_NET_CTRL_VLAN_ADD and VIRTIO_NET_CTRL_VLAN_DEL
command take a little-endian 16-bit VLAN id as the command-specific-data.

VIRTIO_NET_CTRL_VLAN_ADD command adds the specified VLAN to the
VLAN filter table.

VIRTIO_NET_CTRL_VLAN_DEL command removes the specified VLAN from
the VLAN filter table.

\devicenormative{\subparagraph}{VLAN Filtering}{Device Types / Network Device / Device Operation / Control Virtqueue / VLAN Filtering}

When VIRTIO_NET_F_CTRL_VLAN is not negotiated, the device MUST
accept all VLAN tagged packets.

When VIRTIO_NET_F_CTRL_VLAN is negotiated, the device MUST
accept all VLAN tagged packets whose VLAN tag is present in
the VLAN filter table and SHOULD drop all VLAN tagged packets
whose VLAN tag is absent in the VLAN filter table.

\subparagraph{Legacy Interface: VLAN Filtering}\label{sec:Device Types / Network Device / Device Operation / Control Virtqueue / VLAN Filtering / Legacy Interface: VLAN Filtering}
When using the legacy interface, transitional devices and drivers
MUST format the VLAN id
according to the native endian of the guest rather than
(necessarily when not using the legacy interface) little-endian.

\paragraph{Gratuitous Packet Sending}\label{sec:Device Types / Network Device / Device Operation / Control Virtqueue / Gratuitous Packet Sending}

If the driver negotiates the VIRTIO_NET_F_GUEST_ANNOUNCE (depends
on VIRTIO_NET_F_CTRL_VQ), the device can ask the driver to send gratuitous
packets; this is usually done after the guest has been physically
migrated, and needs to announce its presence on the new network
links. (As hypervisor does not have the knowledge of guest
network configuration (eg. tagged vlan) it is simplest to prod
the guest in this way).

\begin{lstlisting}
#define VIRTIO_NET_CTRL_ANNOUNCE       3
 #define VIRTIO_NET_CTRL_ANNOUNCE_ACK             0
\end{lstlisting}

The driver checks VIRTIO_NET_S_ANNOUNCE bit in the device configuration \field{status} field
when it notices the changes of device configuration. The
command VIRTIO_NET_CTRL_ANNOUNCE_ACK is used to indicate that
driver has received the notification and device clears the
VIRTIO_NET_S_ANNOUNCE bit in \field{status}.

Processing this notification involves:

\begin{enumerate}
\item Sending the gratuitous packets (eg. ARP) or marking there are pending
  gratuitous packets to be sent and letting deferred routine to
  send them.

\item Sending VIRTIO_NET_CTRL_ANNOUNCE_ACK command through control
  vq.
\end{enumerate}

\drivernormative{\subparagraph}{Gratuitous Packet Sending}{Device Types / Network Device / Device Operation / Control Virtqueue / Gratuitous Packet Sending}

If the driver negotiates VIRTIO_NET_F_GUEST_ANNOUNCE, it SHOULD notify
network peers of its new location after it sees the VIRTIO_NET_S_ANNOUNCE bit
in \field{status}.  The driver MUST send a command on the command queue
with class VIRTIO_NET_CTRL_ANNOUNCE and command VIRTIO_NET_CTRL_ANNOUNCE_ACK.

\devicenormative{\subparagraph}{Gratuitous Packet Sending}{Device Types / Network Device / Device Operation / Control Virtqueue / Gratuitous Packet Sending}

If VIRTIO_NET_F_GUEST_ANNOUNCE is negotiated, the device MUST clear the
VIRTIO_NET_S_ANNOUNCE bit in \field{status} upon receipt of a command buffer
with class VIRTIO_NET_CTRL_ANNOUNCE and command VIRTIO_NET_CTRL_ANNOUNCE_ACK
before marking the buffer as used.

\paragraph{Device operation in multiqueue mode}\label{sec:Device Types / Network Device / Device Operation / Control Virtqueue / Device operation in multiqueue mode}

This specification defines the following modes that a device MAY implement for operation with multiple transmit/receive virtqueues:
\begin{itemize}
\item Automatic receive steering as defined in \ref{sec:Device Types / Network Device / Device Operation / Control Virtqueue / Automatic receive steering in multiqueue mode}.
 If a device supports this mode, it offers the VIRTIO_NET_F_MQ feature bit.
\item Receive-side scaling as defined in \ref{devicenormative:Device Types / Network Device / Device Operation / Control Virtqueue / Receive-side scaling (RSS) / RSS processing}.
 If a device supports this mode, it offers the VIRTIO_NET_F_RSS feature bit.
\end{itemize}

A device MAY support one of these features or both. The driver MAY negotiate any set of these features that the device supports.

Multiqueue is disabled by default.

The driver enables multiqueue by sending a command using \field{class} VIRTIO_NET_CTRL_MQ. The \field{command} selects the mode of multiqueue operation, as follows:
\begin{lstlisting}
#define VIRTIO_NET_CTRL_MQ    4
 #define VIRTIO_NET_CTRL_MQ_VQ_PAIRS_SET        0 (for automatic receive steering)
 #define VIRTIO_NET_CTRL_MQ_RSS_CONFIG          1 (for configurable receive steering)
 #define VIRTIO_NET_CTRL_MQ_HASH_CONFIG         2 (for configurable hash calculation)
\end{lstlisting}

If more than one multiqueue mode is negotiated, the resulting device configuration is defined by the last command sent by the driver.

\paragraph{Automatic receive steering in multiqueue mode}\label{sec:Device Types / Network Device / Device Operation / Control Virtqueue / Automatic receive steering in multiqueue mode}

If the driver negotiates the VIRTIO_NET_F_MQ feature bit (depends on VIRTIO_NET_F_CTRL_VQ), it MAY transmit outgoing packets on one
of the multiple transmitq1\ldots transmitqN and ask the device to
queue incoming packets into one of the multiple receiveq1\ldots receiveqN
depending on the packet flow.

The driver enables multiqueue by
sending the VIRTIO_NET_CTRL_MQ_VQ_PAIRS_SET command, specifying
the number of the transmit and receive queues to be used up to
\field{max_virtqueue_pairs}; subsequently,
transmitq1\ldots transmitqn and receiveq1\ldots receiveqn where
n=\field{virtqueue_pairs} MAY be used.
\begin{lstlisting}
struct virtio_net_ctrl_mq_pairs_set {
       le16 virtqueue_pairs;
};
#define VIRTIO_NET_CTRL_MQ_VQ_PAIRS_MIN        1
#define VIRTIO_NET_CTRL_MQ_VQ_PAIRS_MAX        0x8000

\end{lstlisting}

When multiqueue is enabled by VIRTIO_NET_CTRL_MQ_VQ_PAIRS_SET command, the device MUST use automatic receive steering
based on packet flow. Programming of the receive steering
classificator is implicit. After the driver transmitted a packet of a
flow on transmitqX, the device SHOULD cause incoming packets for that flow to
be steered to receiveqX. For uni-directional protocols, or where
no packets have been transmitted yet, the device MAY steer a packet
to a random queue out of the specified receiveq1\ldots receiveqn.

Multiqueue is disabled by VIRTIO_NET_CTRL_MQ_VQ_PAIRS_SET with \field{virtqueue_pairs} to 1 (this is
the default) and waiting for the device to use the command buffer.

\drivernormative{\subparagraph}{Automatic receive steering in multiqueue mode}{Device Types / Network Device / Device Operation / Control Virtqueue / Automatic receive steering in multiqueue mode}

The driver MUST configure the virtqueues before enabling them with the
VIRTIO_NET_CTRL_MQ_VQ_PAIRS_SET command.

The driver MUST NOT request a \field{virtqueue_pairs} of 0 or
greater than \field{max_virtqueue_pairs} in the device configuration space.

The driver MUST queue packets only on any transmitq1 before the
VIRTIO_NET_CTRL_MQ_VQ_PAIRS_SET command.

The driver MUST NOT queue packets on transmit queues greater than
\field{virtqueue_pairs} once it has placed the VIRTIO_NET_CTRL_MQ_VQ_PAIRS_SET command in the available ring.

\devicenormative{\subparagraph}{Automatic receive steering in multiqueue mode}{Device Types / Network Device / Device Operation / Control Virtqueue / Automatic receive steering in multiqueue mode}

After initialization of reset, the device MUST queue packets only on receiveq1.

The device MUST NOT queue packets on receive queues greater than
\field{virtqueue_pairs} once it has placed the
VIRTIO_NET_CTRL_MQ_VQ_PAIRS_SET command in a used buffer.

If the destination receive queue is being reset (See \ref{sec:Basic Facilities of a Virtio Device / Virtqueues / Virtqueue Reset}),
the device SHOULD re-select another random queue. If all receive queues are
being reset, the device MUST drop the packet.

\subparagraph{Legacy Interface: Automatic receive steering in multiqueue mode}\label{sec:Device Types / Network Device / Device Operation / Control Virtqueue / Automatic receive steering in multiqueue mode / Legacy Interface: Automatic receive steering in multiqueue mode}
When using the legacy interface, transitional devices and drivers
MUST format \field{virtqueue_pairs}
according to the native endian of the guest rather than
(necessarily when not using the legacy interface) little-endian.

\subparagraph{Hash calculation}\label{sec:Device Types / Network Device / Device Operation / Control Virtqueue / Automatic receive steering in multiqueue mode / Hash calculation}
If VIRTIO_NET_F_HASH_REPORT was negotiated and the device uses automatic receive steering,
the device MUST support a command to configure hash calculation parameters.

The driver provides parameters for hash calculation as follows:

\field{class} VIRTIO_NET_CTRL_MQ, \field{command} VIRTIO_NET_CTRL_MQ_HASH_CONFIG.

The \field{command-specific-data} has following format:
\begin{lstlisting}
struct virtio_net_hash_config {
    le32 hash_types;
    le16 reserved[4];
    u8 hash_key_length;
    u8 hash_key_data[hash_key_length];
};
\end{lstlisting}
Field \field{hash_types} contains a bitmask of allowed hash types as
defined in
\ref{sec:Device Types / Network Device / Device Operation / Processing of Incoming Packets / Hash calculation for incoming packets / Supported/enabled hash types}.
Initially the device has all hash types disabled and reports only VIRTIO_NET_HASH_REPORT_NONE.

Field \field{reserved} MUST contain zeroes. It is defined to make the structure to match the layout of virtio_net_rss_config structure,
defined in \ref{sec:Device Types / Network Device / Device Operation / Control Virtqueue / Receive-side scaling (RSS)}.

Fields \field{hash_key_length} and \field{hash_key_data} define the key to be used in hash calculation.

\paragraph{Receive-side scaling (RSS)}\label{sec:Device Types / Network Device / Device Operation / Control Virtqueue / Receive-side scaling (RSS)}
A device offers the feature VIRTIO_NET_F_RSS if it supports RSS receive steering with Toeplitz hash calculation and configurable parameters.

A driver queries RSS capabilities of the device by reading device configuration as defined in \ref{sec:Device Types / Network Device / Device configuration layout}

\subparagraph{Setting RSS parameters}\label{sec:Device Types / Network Device / Device Operation / Control Virtqueue / Receive-side scaling (RSS) / Setting RSS parameters}

Driver sends a VIRTIO_NET_CTRL_MQ_RSS_CONFIG command using the following format for \field{command-specific-data}:
\begin{lstlisting}
struct rss_rq_id {
   le16 vq_index_1_16: 15; /* Bits 1 to 16 of the virtqueue index */
   le16 reserved: 1; /* Set to zero */
};

struct virtio_net_rss_config {
    le32 hash_types;
    le16 indirection_table_mask;
    struct rss_rq_id unclassified_queue;
    struct rss_rq_id indirection_table[indirection_table_length];
    le16 max_tx_vq;
    u8 hash_key_length;
    u8 hash_key_data[hash_key_length];
};
\end{lstlisting}
Field \field{hash_types} contains a bitmask of allowed hash types as
defined in
\ref{sec:Device Types / Network Device / Device Operation / Processing of Incoming Packets / Hash calculation for incoming packets / Supported/enabled hash types}.

Field \field{indirection_table_mask} is a mask to be applied to
the calculated hash to produce an index in the
\field{indirection_table} array.
Number of entries in \field{indirection_table} is (\field{indirection_table_mask} + 1).

\field{rss_rq_id} is a receive virtqueue id. \field{vq_index_1_16}
consists of bits 1 to 16 of a virtqueue index. For example, a
\field{vq_index_1_16} value of 3 corresponds to virtqueue index 6,
which maps to receiveq4.

Field \field{unclassified_queue} specifies the receive virtqueue id in which to
place unclassified packets.

Field \field{indirection_table} is an array of receive virtqueues ids.

A driver sets \field{max_tx_vq} to inform a device how many transmit virtqueues it may use (transmitq1\ldots transmitq \field{max_tx_vq}).

Fields \field{hash_key_length} and \field{hash_key_data} define the key to be used in hash calculation.

\drivernormative{\subparagraph}{Setting RSS parameters}{Device Types / Network Device / Device Operation / Control Virtqueue / Receive-side scaling (RSS) }

A driver MUST NOT send the VIRTIO_NET_CTRL_MQ_RSS_CONFIG command if the feature VIRTIO_NET_F_RSS has not been negotiated.

A driver MUST fill the \field{indirection_table} array only with
enabled receive virtqueues ids.

The number of entries in \field{indirection_table} (\field{indirection_table_mask} + 1) MUST be a power of two.

A driver MUST use \field{indirection_table_mask} values that are less than \field{rss_max_indirection_table_length} reported by a device.

A driver MUST NOT set any VIRTIO_NET_HASH_TYPE_ flags that are not supported by a device.

\devicenormative{\subparagraph}{RSS processing}{Device Types / Network Device / Device Operation / Control Virtqueue / Receive-side scaling (RSS) / RSS processing}
The device MUST determine the destination queue for a network packet as follows:
\begin{itemize}
\item Calculate the hash of the packet as defined in \ref{sec:Device Types / Network Device / Device Operation / Processing of Incoming Packets / Hash calculation for incoming packets}.
\item If the device did not calculate the hash for the specific packet, the device directs the packet to the receiveq specified by \field{unclassified_queue} of virtio_net_rss_config structure.
\item Apply \field{indirection_table_mask} to the calculated hash
and use the result as the index in the indirection table to get
the destination receive virtqueue id.
\item If the destination receive queue is being reset (See \ref{sec:Basic Facilities of a Virtio Device / Virtqueues / Virtqueue Reset}), the device MUST drop the packet.
\end{itemize}

\paragraph{RSS Context}\label{sec:Device Types / Network Device / Device Operation / Control Virtqueue / RSS Context}

An RSS context consists of configurable parameters specified by \ref{sec:Device Types / Network Device
/ Device Operation / Control Virtqueue / Receive-side scaling (RSS)}.

The RSS configuration supported by VIRTIO_NET_F_RSS is considered the default RSS configuration.

The device offers the feature VIRTIO_NET_F_RSS_CONTEXT if it supports one or multiple RSS contexts
(excluding the default RSS configuration) and configurable parameters.

\subparagraph{Querying RSS Context Capability}\label{sec:Device Types / Network Device / Device Operation / Control Virtqueue / RSS Context / Querying RSS Context Capability}

\begin{lstlisting}
#define VIRTNET_RSS_CTX_CTRL 9
 #define VIRTNET_RSS_CTX_CTRL_CAP_GET  0
 #define VIRTNET_RSS_CTX_CTRL_ADD      1
 #define VIRTNET_RSS_CTX_CTRL_MOD      2
 #define VIRTNET_RSS_CTX_CTRL_DEL      3

struct virtnet_rss_ctx_cap {
    le16 max_rss_contexts;
}
\end{lstlisting}

Field \field{max_rss_contexts} specifies the maximum number of RSS contexts \ref{sec:Device Types / Network Device /
Device Operation / Control Virtqueue / RSS Context} supported by the device.

The driver queries the RSS context capability of the device by sending the command VIRTNET_RSS_CTX_CTRL_CAP_GET
with the structure virtnet_rss_ctx_cap.

For the command VIRTNET_RSS_CTX_CTRL_CAP_GET, the structure virtnet_rss_ctx_cap is write-only for the device.

\subparagraph{Setting RSS Context Parameters}\label{sec:Device Types / Network Device / Device Operation / Control Virtqueue / RSS Context / Setting RSS Context Parameters}

\begin{lstlisting}
struct virtnet_rss_ctx_add_modify {
    le16 rss_ctx_id;
    u8 reserved[6];
    struct virtio_net_rss_config rss;
};

struct virtnet_rss_ctx_del {
    le16 rss_ctx_id;
};
\end{lstlisting}

RSS context parameters:
\begin{itemize}
\item  \field{rss_ctx_id}: ID of the specific RSS context.
\item  \field{rss}: RSS context parameters of the specific RSS context whose id is \field{rss_ctx_id}.
\end{itemize}

\field{reserved} is reserved and it is ignored by the device.

If the feature VIRTIO_NET_F_RSS_CONTEXT has been negotiated, the driver can send the following
VIRTNET_RSS_CTX_CTRL class commands:
\begin{enumerate}
\item VIRTNET_RSS_CTX_CTRL_ADD: use the structure virtnet_rss_ctx_add_modify to
       add an RSS context configured as \field{rss} and id as \field{rss_ctx_id} for the device.
\item VIRTNET_RSS_CTX_CTRL_MOD: use the structure virtnet_rss_ctx_add_modify to
       configure parameters of the RSS context whose id is \field{rss_ctx_id} as \field{rss} for the device.
\item VIRTNET_RSS_CTX_CTRL_DEL: use the structure virtnet_rss_ctx_del to delete
       the RSS context whose id is \field{rss_ctx_id} for the device.
\end{enumerate}

For commands VIRTNET_RSS_CTX_CTRL_ADD and VIRTNET_RSS_CTX_CTRL_MOD, the structure virtnet_rss_ctx_add_modify is read-only for the device.
For the command VIRTNET_RSS_CTX_CTRL_DEL, the structure virtnet_rss_ctx_del is read-only for the device.

\devicenormative{\subparagraph}{RSS Context}{Device Types / Network Device / Device Operation / Control Virtqueue / RSS Context}

The device MUST set \field{max_rss_contexts} to at least 1 if it offers VIRTIO_NET_F_RSS_CONTEXT.

Upon reset, the device MUST clear all previously configured RSS contexts.

\drivernormative{\subparagraph}{RSS Context}{Device Types / Network Device / Device Operation / Control Virtqueue / RSS Context}

The driver MUST have negotiated the VIRTIO_NET_F_RSS_CONTEXT feature when issuing the VIRTNET_RSS_CTX_CTRL class commands.

The driver MUST set \field{rss_ctx_id} to between 1 and \field{max_rss_contexts} inclusive.

The driver MUST NOT send the command VIRTIO_NET_CTRL_MQ_VQ_PAIRS_SET when the device has successfully configured at least one RSS context.

\paragraph{Offloads State Configuration}\label{sec:Device Types / Network Device / Device Operation / Control Virtqueue / Offloads State Configuration}

If the VIRTIO_NET_F_CTRL_GUEST_OFFLOADS feature is negotiated, the driver can
send control commands for dynamic offloads state configuration.

\subparagraph{Setting Offloads State}\label{sec:Device Types / Network Device / Device Operation / Control Virtqueue / Offloads State Configuration / Setting Offloads State}

To configure the offloads, the following layout structure and
definitions are used:

\begin{lstlisting}
le64 offloads;

#define VIRTIO_NET_F_GUEST_CSUM       1
#define VIRTIO_NET_F_GUEST_TSO4       7
#define VIRTIO_NET_F_GUEST_TSO6       8
#define VIRTIO_NET_F_GUEST_ECN        9
#define VIRTIO_NET_F_GUEST_UFO        10
#define VIRTIO_NET_F_GUEST_UDP_TUNNEL_GSO  46
#define VIRTIO_NET_F_GUEST_UDP_TUNNEL_GSO_CSUM 47
#define VIRTIO_NET_F_GUEST_USO4       54
#define VIRTIO_NET_F_GUEST_USO6       55

#define VIRTIO_NET_CTRL_GUEST_OFFLOADS       5
 #define VIRTIO_NET_CTRL_GUEST_OFFLOADS_SET   0
\end{lstlisting}

The class VIRTIO_NET_CTRL_GUEST_OFFLOADS has one command:
VIRTIO_NET_CTRL_GUEST_OFFLOADS_SET applies the new offloads configuration.

le64 value passed as command data is a bitmask, bits set define
offloads to be enabled, bits cleared - offloads to be disabled.

There is a corresponding device feature for each offload. Upon feature
negotiation corresponding offload gets enabled to preserve backward
compatibility.

\drivernormative{\subparagraph}{Setting Offloads State}{Device Types / Network Device / Device Operation / Control Virtqueue / Offloads State Configuration / Setting Offloads State}

A driver MUST NOT enable an offload for which the appropriate feature
has not been negotiated.

\subparagraph{Legacy Interface: Setting Offloads State}\label{sec:Device Types / Network Device / Device Operation / Control Virtqueue / Offloads State Configuration / Setting Offloads State / Legacy Interface: Setting Offloads State}
When using the legacy interface, transitional devices and drivers
MUST format \field{offloads}
according to the native endian of the guest rather than
(necessarily when not using the legacy interface) little-endian.


\paragraph{Notifications Coalescing}\label{sec:Device Types / Network Device / Device Operation / Control Virtqueue / Notifications Coalescing}

If the VIRTIO_NET_F_NOTF_COAL feature is negotiated, the driver can
send commands VIRTIO_NET_CTRL_NOTF_COAL_TX_SET and VIRTIO_NET_CTRL_NOTF_COAL_RX_SET
for notification coalescing.

If the VIRTIO_NET_F_VQ_NOTF_COAL feature is negotiated, the driver can
send commands VIRTIO_NET_CTRL_NOTF_COAL_VQ_SET and VIRTIO_NET_CTRL_NOTF_COAL_VQ_GET
for virtqueue notification coalescing.

\begin{lstlisting}
struct virtio_net_ctrl_coal {
    le32 max_packets;
    le32 max_usecs;
};

struct virtio_net_ctrl_coal_vq {
    le16 vq_index;
    le16 reserved;
    struct virtio_net_ctrl_coal coal;
};

#define VIRTIO_NET_CTRL_NOTF_COAL 6
 #define VIRTIO_NET_CTRL_NOTF_COAL_TX_SET  0
 #define VIRTIO_NET_CTRL_NOTF_COAL_RX_SET 1
 #define VIRTIO_NET_CTRL_NOTF_COAL_VQ_SET 2
 #define VIRTIO_NET_CTRL_NOTF_COAL_VQ_GET 3
\end{lstlisting}

Coalescing parameters:
\begin{itemize}
\item \field{vq_index}: The virtqueue index of an enabled transmit or receive virtqueue.
\item \field{max_usecs} for RX: Maximum number of microseconds to delay a RX notification.
\item \field{max_usecs} for TX: Maximum number of microseconds to delay a TX notification.
\item \field{max_packets} for RX: Maximum number of packets to receive before a RX notification.
\item \field{max_packets} for TX: Maximum number of packets to send before a TX notification.
\end{itemize}

\field{reserved} is reserved and it is ignored by the device.

Read/Write attributes for coalescing parameters:
\begin{itemize}
\item For commands VIRTIO_NET_CTRL_NOTF_COAL_TX_SET and VIRTIO_NET_CTRL_NOTF_COAL_RX_SET, the structure virtio_net_ctrl_coal is write-only for the driver.
\item For the command VIRTIO_NET_CTRL_NOTF_COAL_VQ_SET, the structure virtio_net_ctrl_coal_vq is write-only for the driver.
\item For the command VIRTIO_NET_CTRL_NOTF_COAL_VQ_GET, \field{vq_index} and \field{reserved} are write-only
      for the driver, and the structure virtio_net_ctrl_coal is read-only for the driver.
\end{itemize}

The class VIRTIO_NET_CTRL_NOTF_COAL has the following commands:
\begin{enumerate}
\item VIRTIO_NET_CTRL_NOTF_COAL_TX_SET: use the structure virtio_net_ctrl_coal to set the \field{max_usecs} and \field{max_packets} parameters for all transmit virtqueues.
\item VIRTIO_NET_CTRL_NOTF_COAL_RX_SET: use the structure virtio_net_ctrl_coal to set the \field{max_usecs} and \field{max_packets} parameters for all receive virtqueues.
\item VIRTIO_NET_CTRL_NOTF_COAL_VQ_SET: use the structure virtio_net_ctrl_coal_vq to set the \field{max_usecs} and \field{max_packets} parameters
                                        for an enabled transmit/receive virtqueue whose index is \field{vq_index}.
\item VIRTIO_NET_CTRL_NOTF_COAL_VQ_GET: use the structure virtio_net_ctrl_coal_vq to get the \field{max_usecs} and \field{max_packets} parameters
                                        for an enabled transmit/receive virtqueue whose index is \field{vq_index}.
\end{enumerate}

The device may generate notifications more or less frequently than specified by set commands of the VIRTIO_NET_CTRL_NOTF_COAL class.

If coalescing parameters are being set, the device applies the last coalescing parameters set for a
virtqueue, regardless of the command used to set the parameters. Use the following command sequence
with two pairs of virtqueues as an example:
Each of the following commands sets \field{max_usecs} and \field{max_packets} parameters for virtqueues.
\begin{itemize}
\item Command1: VIRTIO_NET_CTRL_NOTF_COAL_RX_SET sets coalescing parameters for virtqueues having index 0 and index 2. Virtqueues having index 1 and index 3 retain their previous parameters.
\item Command2: VIRTIO_NET_CTRL_NOTF_COAL_VQ_SET with \field{vq_index} = 0 sets coalescing parameters for virtqueue having index 0. Virtqueue having index 2 retains the parameters from command1.
\item Command3: VIRTIO_NET_CTRL_NOTF_COAL_VQ_GET with \field{vq_index} = 0, the device responds with coalescing parameters of vq_index 0 set by command2.
\item Command4: VIRTIO_NET_CTRL_NOTF_COAL_VQ_SET with \field{vq_index} = 1 sets coalescing parameters for virtqueue having index 1. Virtqueue having index 3 retains its previous parameters.
\item Command5: VIRTIO_NET_CTRL_NOTF_COAL_TX_SET sets coalescing parameters for virtqueues having index 1 and index 3, and overrides the parameters set by command4.
\item Command6: VIRTIO_NET_CTRL_NOTF_COAL_VQ_GET with \field{vq_index} = 1, the device responds with coalescing parameters of index 1 set by command5.
\end{itemize}

\subparagraph{Operation}\label{sec:Device Types / Network Device / Device Operation / Control Virtqueue / Notifications Coalescing / Operation}

The device sends a used buffer notification once the notification conditions are met and if the notifications are not suppressed as explained in \ref{sec:Basic Facilities of a Virtio Device / Virtqueues / Used Buffer Notification Suppression}.

When the device has non-zero \field{max_usecs} and non-zero \field{max_packets}, it starts counting microseconds and packets upon receiving/sending a packet.
The device counts packets and microseconds for each receive virtqueue and transmit virtqueue separately.
In this case, the notification conditions are met when \field{max_usecs} microseconds elapse, or upon sending/receiving \field{max_packets} packets, whichever happens first.
Afterwards, the device waits for the next packet and starts counting packets and microseconds again.

When the device has \field{max_usecs} = 0 or \field{max_packets} = 0, the notification conditions are met after every packet received/sent.

\subparagraph{RX Example}\label{sec:Device Types / Network Device / Device Operation / Control Virtqueue / Notifications Coalescing / RX Example}

If, for example:
\begin{itemize}
\item \field{max_usecs} = 10.
\item \field{max_packets} = 15.
\end{itemize}
then each receive virtqueue of a device will operate as follows:
\begin{itemize}
\item The device will count packets received on each virtqueue until it accumulates 15, or until 10 microseconds elapsed since the first one was received.
\item If the notifications are not suppressed by the driver, the device will send an used buffer notification, otherwise, the device will not send an used buffer notification as long as the notifications are suppressed.
\end{itemize}

\subparagraph{TX Example}\label{sec:Device Types / Network Device / Device Operation / Control Virtqueue / Notifications Coalescing / TX Example}

If, for example:
\begin{itemize}
\item \field{max_usecs} = 10.
\item \field{max_packets} = 15.
\end{itemize}
then each transmit virtqueue of a device will operate as follows:
\begin{itemize}
\item The device will count packets sent on each virtqueue until it accumulates 15, or until 10 microseconds elapsed since the first one was sent.
\item If the notifications are not suppressed by the driver, the device will send an used buffer notification, otherwise, the device will not send an used buffer notification as long as the notifications are suppressed.
\end{itemize}

\subparagraph{Notifications When Coalescing Parameters Change}\label{sec:Device Types / Network Device / Device Operation / Control Virtqueue / Notifications Coalescing / Notifications When Coalescing Parameters Change}

When the coalescing parameters of a device change, the device needs to check if the new notification conditions are met and send a used buffer notification if so.

For example, \field{max_packets} = 15 for a device with a single transmit virtqueue: if the device sends 10 packets and afterwards receives a
VIRTIO_NET_CTRL_NOTF_COAL_TX_SET command with \field{max_packets} = 8, then the notification condition is immediately considered to be met;
the device needs to immediately send a used buffer notification, if the notifications are not suppressed by the driver.

\drivernormative{\subparagraph}{Notifications Coalescing}{Device Types / Network Device / Device Operation / Control Virtqueue / Notifications Coalescing}

The driver MUST set \field{vq_index} to the virtqueue index of an enabled transmit or receive virtqueue.

The driver MUST have negotiated the VIRTIO_NET_F_NOTF_COAL feature when issuing commands VIRTIO_NET_CTRL_NOTF_COAL_TX_SET and VIRTIO_NET_CTRL_NOTF_COAL_RX_SET.

The driver MUST have negotiated the VIRTIO_NET_F_VQ_NOTF_COAL feature when issuing commands VIRTIO_NET_CTRL_NOTF_COAL_VQ_SET and VIRTIO_NET_CTRL_NOTF_COAL_VQ_GET.

The driver MUST ignore the values of coalescing parameters received from the VIRTIO_NET_CTRL_NOTF_COAL_VQ_GET command if the device responds with VIRTIO_NET_ERR.

\devicenormative{\subparagraph}{Notifications Coalescing}{Device Types / Network Device / Device Operation / Control Virtqueue / Notifications Coalescing}

The device MUST ignore \field{reserved}.

The device SHOULD respond to VIRTIO_NET_CTRL_NOTF_COAL_TX_SET and VIRTIO_NET_CTRL_NOTF_COAL_RX_SET commands with VIRTIO_NET_ERR if it was not able to change the parameters.

The device MUST respond to the VIRTIO_NET_CTRL_NOTF_COAL_VQ_SET command with VIRTIO_NET_ERR if it was not able to change the parameters.

The device MUST respond to VIRTIO_NET_CTRL_NOTF_COAL_VQ_SET and VIRTIO_NET_CTRL_NOTF_COAL_VQ_GET commands with
VIRTIO_NET_ERR if the designated virtqueue is not an enabled transmit or receive virtqueue.

Upon disabling and re-enabling a transmit virtqueue, the device MUST set the coalescing parameters of the virtqueue
to those configured through the VIRTIO_NET_CTRL_NOTF_COAL_TX_SET command, or, if the driver did not set any TX coalescing parameters, to 0.

Upon disabling and re-enabling a receive virtqueue, the device MUST set the coalescing parameters of the virtqueue
to those configured through the VIRTIO_NET_CTRL_NOTF_COAL_RX_SET command, or, if the driver did not set any RX coalescing parameters, to 0.

The behavior of the device in response to set commands of the VIRTIO_NET_CTRL_NOTF_COAL class is best-effort:
the device MAY generate notifications more or less frequently than specified.

A device SHOULD NOT send used buffer notifications to the driver if the notifications are suppressed, even if the notification conditions are met.

Upon reset, a device MUST initialize all coalescing parameters to 0.

\paragraph{Device Statistics}\label{sec:Device Types / Network Device / Device Operation / Control Virtqueue / Device Statistics}

If the VIRTIO_NET_F_DEVICE_STATS feature is negotiated, the driver can obtain
device statistics from the device by using the following command.

Different types of virtqueues have different statistics. The statistics of the
receiveq are different from those of the transmitq.

The statistics of a certain type of virtqueue are also divided into multiple types
because different types require different features. This enables the expansion
of new statistics.

In one command, the driver can obtain the statistics of one or multiple virtqueues.
Additionally, the driver can obtain multiple type statistics of each virtqueue.

\subparagraph{Query Statistic Capabilities}\label{sec:Device Types / Network Device / Device Operation / Control Virtqueue / Device Statistics / Query Statistic Capabilities}

\begin{lstlisting}
#define VIRTIO_NET_CTRL_STATS         8
#define VIRTIO_NET_CTRL_STATS_QUERY   0
#define VIRTIO_NET_CTRL_STATS_GET     1

struct virtio_net_stats_capabilities {

#define VIRTIO_NET_STATS_TYPE_CVQ       (1 << 32)

#define VIRTIO_NET_STATS_TYPE_RX_BASIC  (1 << 0)
#define VIRTIO_NET_STATS_TYPE_RX_CSUM   (1 << 1)
#define VIRTIO_NET_STATS_TYPE_RX_GSO    (1 << 2)
#define VIRTIO_NET_STATS_TYPE_RX_SPEED  (1 << 3)

#define VIRTIO_NET_STATS_TYPE_TX_BASIC  (1 << 16)
#define VIRTIO_NET_STATS_TYPE_TX_CSUM   (1 << 17)
#define VIRTIO_NET_STATS_TYPE_TX_GSO    (1 << 18)
#define VIRTIO_NET_STATS_TYPE_TX_SPEED  (1 << 19)

    le64 supported_stats_types[1];
}
\end{lstlisting}

To obtain device statistic capability, use the VIRTIO_NET_CTRL_STATS_QUERY
command. When the command completes successfully, \field{command-specific-result}
is in the format of \field{struct virtio_net_stats_capabilities}.

\subparagraph{Get Statistics}\label{sec:Device Types / Network Device / Device Operation / Control Virtqueue / Device Statistics / Get Statistics}

\begin{lstlisting}
struct virtio_net_ctrl_queue_stats {
       struct {
           le16 vq_index;
           le16 reserved[3];
           le64 types_bitmap[1];
       } stats[];
};

struct virtio_net_stats_reply_hdr {
#define VIRTIO_NET_STATS_TYPE_REPLY_CVQ       32

#define VIRTIO_NET_STATS_TYPE_REPLY_RX_BASIC  0
#define VIRTIO_NET_STATS_TYPE_REPLY_RX_CSUM   1
#define VIRTIO_NET_STATS_TYPE_REPLY_RX_GSO    2
#define VIRTIO_NET_STATS_TYPE_REPLY_RX_SPEED  3

#define VIRTIO_NET_STATS_TYPE_REPLY_TX_BASIC  16
#define VIRTIO_NET_STATS_TYPE_REPLY_TX_CSUM   17
#define VIRTIO_NET_STATS_TYPE_REPLY_TX_GSO    18
#define VIRTIO_NET_STATS_TYPE_REPLY_TX_SPEED  19
    u8 type;
    u8 reserved;
    le16 vq_index;
    le16 reserved1;
    le16 size;
}
\end{lstlisting}

To obtain device statistics, use the VIRTIO_NET_CTRL_STATS_GET command with the
\field{command-specific-data} which is in the format of
\field{struct virtio_net_ctrl_queue_stats}. When the command completes
successfully, \field{command-specific-result} contains multiple statistic
results, each statistic result has the \field{struct virtio_net_stats_reply_hdr}
as the header.

The fields of the \field{struct virtio_net_ctrl_queue_stats}:
\begin{description}
    \item [vq_index]
        The index of the virtqueue to obtain the statistics.

    \item [types_bitmap]
        This is a bitmask of the types of statistics to be obtained. Therefore, a
        \field{stats} inside \field{struct virtio_net_ctrl_queue_stats} may
        indicate multiple statistic replies for the virtqueue.
\end{description}

The fields of the \field{struct virtio_net_stats_reply_hdr}:
\begin{description}
    \item [type]
        The type of the reply statistic.

    \item [vq_index]
        The virtqueue index of the reply statistic.

    \item [size]
        The number of bytes for the statistics entry including size of \field{struct virtio_net_stats_reply_hdr}.

\end{description}

\subparagraph{Controlq Statistics}\label{sec:Device Types / Network Device / Device Operation / Control Virtqueue / Device Statistics / Controlq Statistics}

The structure corresponding to the controlq statistics is
\field{struct virtio_net_stats_cvq}. The corresponding type is
VIRTIO_NET_STATS_TYPE_CVQ. This is for the controlq.

\begin{lstlisting}
struct virtio_net_stats_cvq {
    struct virtio_net_stats_reply_hdr hdr;

    le64 command_num;
    le64 ok_num;
};
\end{lstlisting}

\begin{description}
    \item [command_num]
        The number of commands received by the device including the current command.

    \item [ok_num]
        The number of commands completed successfully by the device including the current command.
\end{description}


\subparagraph{Receiveq Basic Statistics}\label{sec:Device Types / Network Device / Device Operation / Control Virtqueue / Device Statistics / Receiveq Basic Statistics}

The structure corresponding to the receiveq basic statistics is
\field{struct virtio_net_stats_rx_basic}. The corresponding type is
VIRTIO_NET_STATS_TYPE_RX_BASIC. This is for the receiveq.

Receiveq basic statistics do not require any feature. As long as the device supports
VIRTIO_NET_F_DEVICE_STATS, the following are the receiveq basic statistics.

\begin{lstlisting}
struct virtio_net_stats_rx_basic {
    struct virtio_net_stats_reply_hdr hdr;

    le64 rx_notifications;

    le64 rx_packets;
    le64 rx_bytes;

    le64 rx_interrupts;

    le64 rx_drops;
    le64 rx_drop_overruns;
};
\end{lstlisting}

The packets described below were all presented on the specified virtqueue.
\begin{description}
    \item [rx_notifications]
        The number of driver notifications received by the device for this
        receiveq.

    \item [rx_packets]
        This is the number of packets passed to the driver by the device.

    \item [rx_bytes]
        This is the bytes of packets passed to the driver by the device.

    \item [rx_interrupts]
        The number of interrupts generated by the device for this receiveq.

    \item [rx_drops]
        This is the number of packets dropped by the device. The count includes
        all types of packets dropped by the device.

    \item [rx_drop_overruns]
        This is the number of packets dropped by the device when no more
        descriptors were available.

\end{description}

\subparagraph{Transmitq Basic Statistics}\label{sec:Device Types / Network Device / Device Operation / Control Virtqueue / Device Statistics / Transmitq Basic Statistics}

The structure corresponding to the transmitq basic statistics is
\field{struct virtio_net_stats_tx_basic}. The corresponding type is
VIRTIO_NET_STATS_TYPE_TX_BASIC. This is for the transmitq.

Transmitq basic statistics do not require any feature. As long as the device supports
VIRTIO_NET_F_DEVICE_STATS, the following are the transmitq basic statistics.

\begin{lstlisting}
struct virtio_net_stats_tx_basic {
    struct virtio_net_stats_reply_hdr hdr;

    le64 tx_notifications;

    le64 tx_packets;
    le64 tx_bytes;

    le64 tx_interrupts;

    le64 tx_drops;
    le64 tx_drop_malformed;
};
\end{lstlisting}

The packets described below are all for a specific virtqueue.
\begin{description}
    \item [tx_notifications]
        The number of driver notifications received by the device for this
        transmitq.

    \item [tx_packets]
        This is the number of packets sent by the device (not the packets
        got from the driver).

    \item [tx_bytes]
        This is the number of bytes sent by the device for all the sent packets
        (not the bytes sent got from the driver).

    \item [tx_interrupts]
        The number of interrupts generated by the device for this transmitq.

    \item [tx_drops]
        The number of packets dropped by the device. The count includes all
        types of packets dropped by the device.

    \item [tx_drop_malformed]
        The number of packets dropped by the device, when the descriptors are
        malformed. For example, the buffer is too short.
\end{description}

\subparagraph{Receiveq CSUM Statistics}\label{sec:Device Types / Network Device / Device Operation / Control Virtqueue / Device Statistics / Receiveq CSUM Statistics}

The structure corresponding to the receiveq checksum statistics is
\field{struct virtio_net_stats_rx_csum}. The corresponding type is
VIRTIO_NET_STATS_TYPE_RX_CSUM. This is for the receiveq.

Only after the VIRTIO_NET_F_GUEST_CSUM is negotiated, the receiveq checksum
statistics can be obtained.

\begin{lstlisting}
struct virtio_net_stats_rx_csum {
    struct virtio_net_stats_reply_hdr hdr;

    le64 rx_csum_valid;
    le64 rx_needs_csum;
    le64 rx_csum_none;
    le64 rx_csum_bad;
};
\end{lstlisting}

The packets described below were all presented on the specified virtqueue.
\begin{description}
    \item [rx_csum_valid]
        The number of packets with VIRTIO_NET_HDR_F_DATA_VALID.

    \item [rx_needs_csum]
        The number of packets with VIRTIO_NET_HDR_F_NEEDS_CSUM.

    \item [rx_csum_none]
        The number of packets without hardware checksum. The packet here refers
        to the non-TCP/UDP packet that the device cannot recognize.

    \item [rx_csum_bad]
        The number of packets with checksum mismatch.

\end{description}

\subparagraph{Transmitq CSUM Statistics}\label{sec:Device Types / Network Device / Device Operation / Control Virtqueue / Device Statistics / Transmitq CSUM Statistics}

The structure corresponding to the transmitq checksum statistics is
\field{struct virtio_net_stats_tx_csum}. The corresponding type is
VIRTIO_NET_STATS_TYPE_TX_CSUM. This is for the transmitq.

Only after the VIRTIO_NET_F_CSUM is negotiated, the transmitq checksum
statistics can be obtained.

The following are the transmitq checksum statistics:

\begin{lstlisting}
struct virtio_net_stats_tx_csum {
    struct virtio_net_stats_reply_hdr hdr;

    le64 tx_csum_none;
    le64 tx_needs_csum;
};
\end{lstlisting}

The packets described below are all for a specific virtqueue.
\begin{description}
    \item [tx_csum_none]
        The number of packets which do not require hardware checksum.

    \item [tx_needs_csum]
        The number of packets which require checksum calculation by the device.

\end{description}

\subparagraph{Receiveq GSO Statistics}\label{sec:Device Types / Network Device / Device Operation / Control Virtqueue / Device Statistics / Receiveq GSO Statistics}

The structure corresponding to the receivq GSO statistics is
\field{struct virtio_net_stats_rx_gso}. The corresponding type is
VIRTIO_NET_STATS_TYPE_RX_GSO. This is for the receiveq.

If one or more of the VIRTIO_NET_F_GUEST_TSO4, VIRTIO_NET_F_GUEST_TSO6
have been negotiated, the receiveq GSO statistics can be obtained.

GSO packets refer to packets passed by the device to the driver where
\field{gso_type} is not VIRTIO_NET_HDR_GSO_NONE.

\begin{lstlisting}
struct virtio_net_stats_rx_gso {
    struct virtio_net_stats_reply_hdr hdr;

    le64 rx_gso_packets;
    le64 rx_gso_bytes;
    le64 rx_gso_packets_coalesced;
    le64 rx_gso_bytes_coalesced;
};
\end{lstlisting}

The packets described below were all presented on the specified virtqueue.
\begin{description}
    \item [rx_gso_packets]
        The number of the GSO packets received by the device.

    \item [rx_gso_bytes]
        The bytes of the GSO packets received by the device.
        This includes the header size of the GSO packet.

    \item [rx_gso_packets_coalesced]
        The number of the GSO packets coalesced by the device.

    \item [rx_gso_bytes_coalesced]
        The bytes of the GSO packets coalesced by the device.
        This includes the header size of the GSO packet.
\end{description}

\subparagraph{Transmitq GSO Statistics}\label{sec:Device Types / Network Device / Device Operation / Control Virtqueue / Device Statistics / Transmitq GSO Statistics}

The structure corresponding to the transmitq GSO statistics is
\field{struct virtio_net_stats_tx_gso}. The corresponding type is
VIRTIO_NET_STATS_TYPE_TX_GSO. This is for the transmitq.

If one or more of the VIRTIO_NET_F_HOST_TSO4, VIRTIO_NET_F_HOST_TSO6,
VIRTIO_NET_F_HOST_USO options have been negotiated, the transmitq GSO statistics
can be obtained.

GSO packets refer to packets passed by the driver to the device where
\field{gso_type} is not VIRTIO_NET_HDR_GSO_NONE.
See more \ref{sec:Device Types / Network Device / Device Operation / Packet
Transmission}.

\begin{lstlisting}
struct virtio_net_stats_tx_gso {
    struct virtio_net_stats_reply_hdr hdr;

    le64 tx_gso_packets;
    le64 tx_gso_bytes;
    le64 tx_gso_segments;
    le64 tx_gso_segments_bytes;
    le64 tx_gso_packets_noseg;
    le64 tx_gso_bytes_noseg;
};
\end{lstlisting}

The packets described below are all for a specific virtqueue.
\begin{description}
    \item [tx_gso_packets]
        The number of the GSO packets sent by the device.

    \item [tx_gso_bytes]
        The bytes of the GSO packets sent by the device.

    \item [tx_gso_segments]
        The number of segments prepared from GSO packets.

    \item [tx_gso_segments_bytes]
        The bytes of segments prepared from GSO packets.

    \item [tx_gso_packets_noseg]
        The number of the GSO packets without segmentation.

    \item [tx_gso_bytes_noseg]
        The bytes of the GSO packets without segmentation.

\end{description}

\subparagraph{Receiveq Speed Statistics}\label{sec:Device Types / Network Device / Device Operation / Control Virtqueue / Device Statistics / Receiveq Speed Statistics}

The structure corresponding to the receiveq speed statistics is
\field{struct virtio_net_stats_rx_speed}. The corresponding type is
VIRTIO_NET_STATS_TYPE_RX_SPEED. This is for the receiveq.

The device has the allowance for the speed. If VIRTIO_NET_F_SPEED_DUPLEX has
been negotiated, the driver can get this by \field{speed}. When the received
packets bitrate exceeds the \field{speed}, some packets may be dropped by the
device.

\begin{lstlisting}
struct virtio_net_stats_rx_speed {
    struct virtio_net_stats_reply_hdr hdr;

    le64 rx_packets_allowance_exceeded;
    le64 rx_bytes_allowance_exceeded;
};
\end{lstlisting}

The packets described below were all presented on the specified virtqueue.
\begin{description}
    \item [rx_packets_allowance_exceeded]
        The number of the packets dropped by the device due to the received
        packets bitrate exceeding the \field{speed}.

    \item [rx_bytes_allowance_exceeded]
        The bytes of the packets dropped by the device due to the received
        packets bitrate exceeding the \field{speed}.

\end{description}

\subparagraph{Transmitq Speed Statistics}\label{sec:Device Types / Network Device / Device Operation / Control Virtqueue / Device Statistics / Transmitq Speed Statistics}

The structure corresponding to the transmitq speed statistics is
\field{struct virtio_net_stats_tx_speed}. The corresponding type is
VIRTIO_NET_STATS_TYPE_TX_SPEED. This is for the transmitq.

The device has the allowance for the speed. If VIRTIO_NET_F_SPEED_DUPLEX has
been negotiated, the driver can get this by \field{speed}. When the transmit
packets bitrate exceeds the \field{speed}, some packets may be dropped by the
device.

\begin{lstlisting}
struct virtio_net_stats_tx_speed {
    struct virtio_net_stats_reply_hdr hdr;

    le64 tx_packets_allowance_exceeded;
    le64 tx_bytes_allowance_exceeded;
};
\end{lstlisting}

The packets described below were all presented on the specified virtqueue.
\begin{description}
    \item [tx_packets_allowance_exceeded]
        The number of the packets dropped by the device due to the transmit packets
        bitrate exceeding the \field{speed}.

    \item [tx_bytes_allowance_exceeded]
        The bytes of the packets dropped by the device due to the transmit packets
        bitrate exceeding the \field{speed}.

\end{description}

\devicenormative{\subparagraph}{Device Statistics}{Device Types / Network Device / Device Operation / Control Virtqueue / Device Statistics}

When the VIRTIO_NET_F_DEVICE_STATS feature is negotiated, the device MUST reply
to the command VIRTIO_NET_CTRL_STATS_QUERY with the
\field{struct virtio_net_stats_capabilities}. \field{supported_stats_types}
includes all the statistic types supported by the device.

If \field{struct virtio_net_ctrl_queue_stats} is incorrect (such as the
following), the device MUST set \field{ack} to VIRTIO_NET_ERR. Even if there is
only one error, the device MUST fail the entire command.
\begin{itemize}
    \item \field{vq_index} exceeds the queue range.
    \item \field{types_bitmap} contains unknown types.
    \item One or more of the bits present in \field{types_bitmap} is not valid
        for the specified virtqueue.
    \item The feature corresponding to the specified \field{types_bitmap} was
        not negotiated.
\end{itemize}

The device MUST set the actual size of the bytes occupied by the reply to the
\field{size} of the \field{hdr}. And the device MUST set the \field{type} and
the \field{vq_index} of the statistic header.

The \field{command-specific-result} buffer allocated by the driver may be
smaller or bigger than all the statistics specified by
\field{struct virtio_net_ctrl_queue_stats}. The device MUST fill up only upto
the valid bytes.

The statistics counter replied by the device MUST wrap around to zero by the
device on the overflow.

\drivernormative{\subparagraph}{Device Statistics}{Device Types / Network Device / Device Operation / Control Virtqueue / Device Statistics}

The types contained in the \field{types_bitmap} MUST be queried from the device
via command VIRTIO_NET_CTRL_STATS_QUERY.

\field{types_bitmap} in \field{struct virtio_net_ctrl_queue_stats} MUST be valid to the
vq specified by \field{vq_index}.

The \field{command-specific-result} buffer allocated by the driver MUST have
enough capacity to store all the statistics reply headers defined in
\field{struct virtio_net_ctrl_queue_stats}. If the
\field{command-specific-result} buffer is fully utilized by the device but some
replies are missed, it is possible that some statistics may exceed the capacity
of the driver's records. In such cases, the driver should allocate additional
space for the \field{command-specific-result} buffer.

\subsubsection{Flow filter}\label{sec:Device Types / Network Device / Device Operation / Flow filter}

A network device can support one or more flow filter rules. Each flow filter rule
is applied by matching a packet and then taking an action, such as directing the packet
to a specific receiveq or dropping the packet. An example of a match is
matching on specific source and destination IP addresses.

A flow filter rule is a device resource object that consists of a key,
a processing priority, and an action to either direct a packet to a
receive queue or drop the packet.

Each rule uses a classifier. The key is matched against the packet using
a classifier, defining which fields in the packet are matched.
A classifier resource object consists of one or more field selectors, each with
a type that specifies the header fields to be matched against, and a mask.
The mask can match whole fields or parts of a field in a header. Each
rule resource object depends on the classifier resource object.

When a packet is received, relevant fields are extracted
(in the same way) from both the packet and the key according to the
classifier. The resulting field contents are then compared -
if they are identical the rule action is taken, if they are not, the rule is ignored.

Multiple flow filter rules are part of a group. The rule resource object
depends on the group. Each rule within a
group has a rule priority, and each group also has a group priority. For a
packet, a group with the highest priority is selected first. Within a group,
rules are applied from highest to lowest priority, until one of the rules
matches the packet and an action is taken. If all the rules within a group
are ignored, the group with the next highest priority is selected, and so on.

The device and the driver indicates flow filter resource limits using the capability
\ref{par:Device Types / Network Device / Device Operation / Flow filter / Device and driver capabilities / VIRTIO-NET-FF-RESOURCE-CAP} specifying the limits on the number of flow filter rule,
group and classifier resource objects. The capability \ref{par:Device Types / Network Device / Device Operation / Flow filter / Device and driver capabilities / VIRTIO-NET-FF-SELECTOR-CAP} specifies which selectors the device supports.
The driver indicates the selectors it is using by setting the flow
filter selector capability, prior to adding any resource objects.

The capability \ref{par:Device Types / Network Device / Device Operation / Flow filter / Device and driver capabilities / VIRTIO-NET-FF-ACTION-CAP} specifies which actions the device supports.

The driver controls the flow filter rule, classifier and group resource objects using
administration commands described in
\ref{sec:Basic Facilities of a Virtio Device / Device groups / Group administration commands / Device resource objects}.

\paragraph{Packet processing order}\label{sec:sec:Device Types / Network Device / Device Operation / Flow filter / Packet processing order}

Note that flow filter rules are applied after MAC/VLAN filtering. Flow filter
rules take precedence over steering: if a flow filter rule results in an action,
the steering configuration does not apply. The steering configuration only applies
to packets for which no flow filter rule action was performed. For example,
incoming packets can be processed in the following order:

\begin{itemize}
\item apply steering configuration received using control virtqueue commands
      VIRTIO_NET_CTRL_RX, VIRTIO_NET_CTRL_MAC and VIRTIO_NET_CTRL_VLAN.
\item apply flow filter rules if any.
\item if no filter rule applied, apply steering configuration received using command
      VIRTIO_NET_CTRL_MQ_RSS_CONFIG or as per automatic receive steering.
\end{itemize}

Some incoming packet processing examples:
\begin{itemize}
\item If the packet is dropped by the flow filter rule, RSS
      steering is ignored for the packet.
\item If the packet is directed to a specific receiveq using flow filter rule,
      the RSS steering is ignored for the packet.
\item If a packet is dropped due to the VIRTIO_NET_CTRL_MAC configuration,
      both flow filter rules and the RSS steering are ignored for the packet.
\item If a packet does not match any flow filter rules,
      the RSS steering is used to select the receiveq for the packet (if enabled).
\item If there are two flow filter groups configured as group_A and group_B
      with respective group priorities as 4, and 5; flow filter rules of
      group_B are applied first having highest group priority, if there is a match,
      the flow filter rules of group_A are ignored; if there is no match for
      the flow filter rules in group_B, the flow filter rules of next level group_A are applied.
\end{itemize}

\paragraph{Device and driver capabilities}
\label{par:Device Types / Network Device / Device Operation / Flow filter / Device and driver capabilities}

\subparagraph{VIRTIO_NET_FF_RESOURCE_CAP}
\label{par:Device Types / Network Device / Device Operation / Flow filter / Device and driver capabilities / VIRTIO-NET-FF-RESOURCE-CAP}

The capability VIRTIO_NET_FF_RESOURCE_CAP indicates the flow filter resource limits.
\field{cap_specific_data} is in the format
\field{struct virtio_net_ff_cap_data}.

\begin{lstlisting}
struct virtio_net_ff_cap_data {
        le32 groups_limit;
        le32 selectors_limit;
        le32 rules_limit;
        le32 rules_per_group_limit;
        u8 last_rule_priority;
        u8 selectors_per_classifier_limit;
};
\end{lstlisting}

\field{groups_limit}, and \field{selectors_limit} represent the maximum
number of flow filter groups and selectors, respectively, that the driver can create.
 \field{rules_limit} is the maximum number of
flow fiilter rules that the driver can create across all the groups.
\field{rules_per_group_limit} is the maximum number of flow filter rules that the driver
can create for each flow filter group.

\field{last_rule_priority} is the highest priority that can be assigned to a
flow filter rule.

\field{selectors_per_classifier_limit} is the maximum number of selectors
that a classifier can have.

\subparagraph{VIRTIO_NET_FF_SELECTOR_CAP}
\label{par:Device Types / Network Device / Device Operation / Flow filter / Device and driver capabilities / VIRTIO-NET-FF-SELECTOR-CAP}

The capability VIRTIO_NET_FF_SELECTOR_CAP lists the supported selectors and the
supported packet header fields for each selector.
\field{cap_specific_data} is in the format \field{struct virtio_net_ff_cap_mask_data}.

\begin{lstlisting}[label={lst:Device Types / Network Device / Device Operation / Flow filter / Device and driver capabilities / VIRTIO-NET-FF-SELECTOR-CAP / virtio-net-ff-selector}]
struct virtio_net_ff_selector {
        u8 type;
        u8 flags;
        u8 reserved[2];
        u8 length;
        u8 reserved1[3];
        u8 mask[];
};

struct virtio_net_ff_cap_mask_data {
        u8 count;
        u8 reserved[7];
        struct virtio_net_ff_selector selectors[];
};

#define VIRTIO_NET_FF_MASK_F_PARTIAL_MASK (1 << 0)
\end{lstlisting}

\field{count} indicates number of valid entries in the \field{selectors} array.
\field{selectors[]} is an array of supported selectors. Within each array entry:
\field{type} specifies the type of the packet header, as defined in table
\ref{table:Device Types / Network Device / Device Operation / Flow filter / Device and driver capabilities / VIRTIO-NET-FF-SELECTOR-CAP / flow filter selector types}. \field{mask} specifies which fields of the
packet header can be matched in a flow filter rule.

Each \field{type} is also listed in table
\ref{table:Device Types / Network Device / Device Operation / Flow filter / Device and driver capabilities / VIRTIO-NET-FF-SELECTOR-CAP / flow filter selector types}. \field{mask} is a byte array
in network byte order. For example, when \field{type} is VIRTIO_NET_FF_MASK_TYPE_IPV6,
the \field{mask} is in the format \hyperref[intro:IPv6-Header-Format]{IPv6 Header Format}.

If partial masking is not set, then all bits in each field have to be either all 0s
to ignore this field or all 1s to match on this field. If partial masking is set,
then any combination of bits can bit set to match on these bits.
For example, when a selector \field{type} is VIRTIO_NET_FF_MASK_TYPE_ETH, if
\field{mask[0-12]} are zero and \field{mask[13-14]} are 0xff (all 1s), it
indicates that matching is only supported for \field{EtherType} of
\field{Ethernet MAC frame}, matching is not supported for
\field{Destination Address} and \field{Source Address}.

The entries in the array \field{selectors} are ordered by
\field{type}, with each \field{type} value only appearing once.

\field{length} is the length of a dynamic array \field{mask} in bytes.
\field{reserved} and \field{reserved1} are reserved and set to zero.

\begin{table}[H]
\caption{Flow filter selector types}
\label{table:Device Types / Network Device / Device Operation / Flow filter / Device and driver capabilities / VIRTIO-NET-FF-SELECTOR-CAP / flow filter selector types}
\begin{tabularx}{\textwidth}{ |l|X|X| }
\hline
Type & Name & Description \\
\hline \hline
0x0 & - & Reserved \\
\hline
0x1 & VIRTIO_NET_FF_MASK_TYPE_ETH & 14 bytes of frame header starting from destination address described in \hyperref[intro:IEEE 802.3-2022]{IEEE 802.3-2022} \\
\hline
0x2 & VIRTIO_NET_FF_MASK_TYPE_IPV4 & 20 bytes of \hyperref[intro:Internet-Header-Format]{IPv4: Internet Header Format} \\
\hline
0x3 & VIRTIO_NET_FF_MASK_TYPE_IPV6 & 40 bytes of \hyperref[intro:IPv6-Header-Format]{IPv6 Header Format} \\
\hline
0x4 & VIRTIO_NET_FF_MASK_TYPE_TCP & 20 bytes of \hyperref[intro:TCP-Header-Format]{TCP Header Format} \\
\hline
0x5 & VIRTIO_NET_FF_MASK_TYPE_UDP & 8 bytes of UDP header described in \hyperref[intro:UDP]{UDP} \\
\hline
0x6 - 0xFF & & Reserved for future \\
\hline
\end{tabularx}
\end{table}

When VIRTIO_NET_FF_MASK_F_PARTIAL_MASK (bit 0) is set, it indicates that
partial masking is supported for all the fields of the selector identified by \field{type}.

For the selector \field{type} VIRTIO_NET_FF_MASK_TYPE_IPV4, if a partial mask is unsupported,
then matching on an individual bit of \field{Flags} in the
\field{IPv4: Internet Header Format} is unsupported. \field{Flags} has to match as a whole
if it is supported.

For the selector \field{type} VIRTIO_NET_FF_MASK_TYPE_IPV4, \field{mask} includes fields
up to the \field{Destination Address}; that is, \field{Options} and
\field{Padding} are excluded.

For the selector \field{type} VIRTIO_NET_FF_MASK_TYPE_IPV6, the \field{Next Header} field
of the \field{mask} corresponds to the \field{Next Header} in the packet
when \field{IPv6 Extension Headers} are not present. When the packet includes
one or more \field{IPv6 Extension Headers}, the \field{Next Header} field of
the \field{mask} corresponds to the \field{Next Header} of the last
\field{IPv6 Extension Header} in the packet.

For the selector \field{type} VIRTIO_NET_FF_MASK_TYPE_TCP, \field{Control bits}
are treated as individual fields for matching; that is, matching individual
\field{Control bits} does not depend on the partial mask support.

\subparagraph{VIRTIO_NET_FF_ACTION_CAP}
\label{par:Device Types / Network Device / Device Operation / Flow filter / Device and driver capabilities / VIRTIO-NET-FF-ACTION-CAP}

The capability VIRTIO_NET_FF_ACTION_CAP lists the supported actions in a rule.
\field{cap_specific_data} is in the format \field{struct virtio_net_ff_cap_actions}.

\begin{lstlisting}
struct virtio_net_ff_actions {
        u8 count;
        u8 reserved[7];
        u8 actions[];
};
\end{lstlisting}

\field{actions} is an array listing all possible actions.
The entries in the array are ordered from the smallest to the largest,
with each supported value appearing exactly once. Each entry can have the
following values:

\begin{table}[H]
\caption{Flow filter rule actions}
\label{table:Device Types / Network Device / Device Operation / Flow filter / Device and driver capabilities / VIRTIO-NET-FF-ACTION-CAP / flow filter rule actions}
\begin{tabularx}{\textwidth}{ |l|X|X| }
\hline
Action & Name & Description \\
\hline \hline
0x0 & - & reserved \\
\hline
0x1 & VIRTIO_NET_FF_ACTION_DROP & Matching packet will be dropped by the device \\
\hline
0x2 & VIRTIO_NET_FF_ACTION_DIRECT_RX_VQ & Matching packet will be directed to a receive queue \\
\hline
0x3 - 0xFF & & Reserved for future \\
\hline
\end{tabularx}
\end{table}

\paragraph{Resource objects}
\label{par:Device Types / Network Device / Device Operation / Flow filter / Resource objects}

\subparagraph{VIRTIO_NET_RESOURCE_OBJ_FF_GROUP}\label{par:Device Types / Network Device / Device Operation / Flow filter / Resource objects / VIRTIO-NET-RESOURCE-OBJ-FF-GROUP}

A flow filter group contains between 0 and \field{rules_limit} rules, as specified by the
capability VIRTIO_NET_FF_RESOURCE_CAP. For the flow filter group object both
\field{resource_obj_specific_data} and
\field{resource_obj_specific_result} are in the format
\field{struct virtio_net_resource_obj_ff_group}.

\begin{lstlisting}
struct virtio_net_resource_obj_ff_group {
        le16 group_priority;
};
\end{lstlisting}

\field{group_priority} specifies the priority for the group. Each group has a
distinct priority. For each incoming packet, the device tries to apply rules
from groups from higher \field{group_priority} value to lower, until either a
rule matches the packet or all groups have been tried.

\subparagraph{VIRTIO_NET_RESOURCE_OBJ_FF_CLASSIFIER}\label{par:Device Types / Network Device / Device Operation / Flow filter / Resource objects / VIRTIO-NET-RESOURCE-OBJ-FF-CLASSIFIER}

A classifier is used to match a flow filter key against a packet. The
classifier defines the desired packet fields to match, and is represented by
the VIRTIO_NET_RESOURCE_OBJ_FF_CLASSIFIER device resource object.

For the flow filter classifier object both \field{resource_obj_specific_data} and
\field{resource_obj_specific_result} are in the format
\field{struct virtio_net_resource_obj_ff_classifier}.

\begin{lstlisting}
struct virtio_net_resource_obj_ff_classifier {
        u8 count;
        u8 reserved[7];
        struct virtio_net_ff_selector selectors[];
};
\end{lstlisting}

A classifier is an array of \field{selectors}. The number of selectors in the
array is indicated by \field{count}. The selector has a type that specifies
the header fields to be matched against, and a mask.
See \ref{lst:Device Types / Network Device / Device Operation / Flow filter / Device and driver capabilities / VIRTIO-NET-FF-SELECTOR-CAP / virtio-net-ff-selector}
for details about selectors.

The first selector is always VIRTIO_NET_FF_MASK_TYPE_ETH. When there are multiple
selectors, a second selector can be either VIRTIO_NET_FF_MASK_TYPE_IPV4
or VIRTIO_NET_FF_MASK_TYPE_IPV6. If the third selector exists, the third
selector can be either VIRTIO_NET_FF_MASK_TYPE_UDP or VIRTIO_NET_FF_MASK_TYPE_TCP.
For example, to match a Ethernet IPv6 UDP packet,
\field{selectors[0].type} is set to VIRTIO_NET_FF_MASK_TYPE_ETH, \field{selectors[1].type}
is set to VIRTIO_NET_FF_MASK_TYPE_IPV6 and \field{selectors[2].type} is
set to VIRTIO_NET_FF_MASK_TYPE_UDP; accordingly, \field{selectors[0].mask[0-13]} is
for Ethernet header fields, \field{selectors[1].mask[0-39]} is set for IPV6 header
and \field{selectors[2].mask[0-7]} is set for UDP header.

When there are multiple selectors, the type of the (N+1)\textsuperscript{th} selector
affects the mask of the (N)\textsuperscript{th} selector. If
\field{count} is 2 or more, all the mask bits within \field{selectors[0]}
corresponding to \field{EtherType} of an Ethernet header are set.

If \field{count} is more than 2:
\begin{itemize}
\item if \field{selector[1].type} is, VIRTIO_NET_FF_MASK_TYPE_IPV4, then, all the mask bits within
\field{selector[1]} for \field{Protocol} is set.
\item if \field{selector[1].type} is, VIRTIO_NET_FF_MASK_TYPE_IPV6, then, all the mask bits within
\field{selector[1]} for \field{Next Header} is set.
\end{itemize}

If for a given packet header field, a subset of bits of a field is to be matched,
and if the partial mask is supported, the flow filter
mask object can specify a mask which has fewer bits set than the packet header
field size. For example, a partial mask for the Ethernet header source mac
address can be of 1-bit for multicast detection instead of 48-bits.

\subparagraph{VIRTIO_NET_RESOURCE_OBJ_FF_RULE}\label{par:Device Types / Network Device / Device Operation / Flow filter / Resource objects / VIRTIO-NET-RESOURCE-OBJ-FF-RULE}

Each flow filter rule resource object comprises a key, a priority, and an action.
For the flow filter rule object,
\field{resource_obj_specific_data} and
\field{resource_obj_specific_result} are in the format
\field{struct virtio_net_resource_obj_ff_rule}.

\begin{lstlisting}
struct virtio_net_resource_obj_ff_rule {
        le32 group_id;
        le32 classifier_id;
        u8 rule_priority;
        u8 key_length; /* length of key in bytes */
        u8 action;
        u8 reserved;
        le16 vq_index;
        u8 reserved1[2];
        u8 keys[][];
};
\end{lstlisting}

\field{group_id} is the resource object ID of the flow filter group to which
this rule belongs. \field{classifier_id} is the resource object ID of the
classifier used to match a packet against the \field{key}.

\field{rule_priority} denotes the priority of the rule within the group
specified by the \field{group_id}.
Rules within the group are applied from the highest to the lowest priority
until a rule matches the packet and an
action is taken. Rules with the same priority can be applied in any order.

\field{reserved} and \field{reserved1} are reserved and set to 0.

\field{keys[][]} is an array of keys to match against packets, using
the classifier specified by \field{classifier_id}. Each entry (key) comprises
a byte array, and they are located one immediately after another.
The size (number of entries) of the array is exactly the same as that of
\field{selectors} in the classifier, or in other words, \field{count}
in the classifier.

\field{key_length} specifies the total length of \field{keys} in bytes.
In other words, it equals the sum total of \field{length} of all
selectors in \field{selectors} in the classifier specified by
\field{classifier_id}.

For example, if a classifier object's \field{selectors[0].type} is
VIRTIO_NET_FF_MASK_TYPE_ETH and \field{selectors[1].type} is
VIRTIO_NET_FF_MASK_TYPE_IPV6,
then selectors[0].length is 14 and selectors[1].length is 40.
Accordingly, the \field{key_length} is set to 54.
This setting indicates that the \field{key} array's length is 54 bytes
comprising a first byte array of 14 bytes for the
Ethernet MAC header in bytes 0-13, immediately followed by 40 bytes for the
IPv6 header in bytes 14-53.

When there are multiple selectors in the classifier object, the key bytes
for (N)\textsuperscript{th} selector are set so that
(N+1)\textsuperscript{th} selector can be matched.

If \field{count} is 2 or more, key bytes of \field{EtherType}
are set according to \hyperref[intro:IEEE 802 Ethertypes]{IEEE 802 Ethertypes}
for VIRTIO_NET_FF_MASK_TYPE_IPV4 or VIRTIO_NET_FF_MASK_TYPE_IPV6 respectively.

If \field{count} is more than 2, when \field{selector[1].type} is
VIRTIO_NET_FF_MASK_TYPE_IPV4 or VIRTIO_NET_FF_MASK_TYPE_IPV6, key
bytes of \field{Protocol} or \field{Next Header} is set as per
\field{Protocol Numbers} defined \hyperref[intro:IANA Protocol Numbers]{IANA Protocol Numbers}
respectively.

\field{action} is the action to take when a packet matches the
\field{key} using the \field{classifier_id}. Supported actions are described in
\ref{table:Device Types / Network Device / Device Operation / Flow filter / Device and driver capabilities / VIRTIO-NET-FF-ACTION-CAP / flow filter rule actions}.

\field{vq_index} specifies a receive virtqueue. When the \field{action} is set
to VIRTIO_NET_FF_ACTION_DIRECT_RX_VQ, and the packet matches the \field{key},
the matching packet is directed to this virtqueue.

Note that at most one action is ever taken for a given packet. If a rule is
applied and an action is taken, the action of other rules is not taken.

\devicenormative{\paragraph}{Flow filter}{Device Types / Network Device / Device Operation / Flow filter}

When the device supports flow filter operations,
\begin{itemize}
\item the device MUST set VIRTIO_NET_FF_RESOURCE_CAP, VIRTIO_NET_FF_SELECTOR_CAP
and VIRTIO_NET_FF_ACTION_CAP capability in the \field{supported_caps} in the
command VIRTIO_ADMIN_CMD_CAP_SUPPORT_QUERY.
\item the device MUST support the administration commands
VIRTIO_ADMIN_CMD_RESOURCE_OBJ_CREATE,
VIRTIO_ADMIN_CMD_RESOURCE_OBJ_MODIFY, VIRTIO_ADMIN_CMD_RESOURCE_OBJ_QUERY,
VIRTIO_ADMIN_CMD_RESOURCE_OBJ_DESTROY for the resource types
VIRTIO_NET_RESOURCE_OBJ_FF_GROUP, VIRTIO_NET_RESOURCE_OBJ_FF_CLASSIFIER and
VIRTIO_NET_RESOURCE_OBJ_FF_RULE.
\end{itemize}

When any of the VIRTIO_NET_FF_RESOURCE_CAP, VIRTIO_NET_FF_SELECTOR_CAP, or
VIRTIO_NET_FF_ACTION_CAP capability is disabled, the device SHOULD set
\field{status} to VIRTIO_ADMIN_STATUS_Q_INVALID_OPCODE for the commands
VIRTIO_ADMIN_CMD_RESOURCE_OBJ_CREATE,
VIRTIO_ADMIN_CMD_RESOURCE_OBJ_MODIFY, VIRTIO_ADMIN_CMD_RESOURCE_OBJ_QUERY,
and VIRTIO_ADMIN_CMD_RESOURCE_OBJ_DESTROY. These commands apply to the resource
\field{type} of VIRTIO_NET_RESOURCE_OBJ_FF_GROUP, VIRTIO_NET_RESOURCE_OBJ_FF_CLASSIFIER, and
VIRTIO_NET_RESOURCE_OBJ_FF_RULE.

The device SHOULD set \field{status} to VIRTIO_ADMIN_STATUS_EINVAL for the
command VIRTIO_ADMIN_CMD_RESOURCE_OBJ_CREATE when the resource \field{type}
is VIRTIO_NET_RESOURCE_OBJ_FF_GROUP, if a flow filter group already exists
with the supplied \field{group_priority}.

The device SHOULD set \field{status} to VIRTIO_ADMIN_STATUS_ENOSPC for the
command VIRTIO_ADMIN_CMD_RESOURCE_OBJ_CREATE when the resource \field{type}
is VIRTIO_NET_RESOURCE_OBJ_FF_GROUP, if the number of flow filter group
objects in the device exceeds the lower of the configured driver
capabilities \field{groups_limit} and \field{rules_per_group_limit}.

The device SHOULD set \field{status} to VIRTIO_ADMIN_STATUS_ENOSPC for the
command VIRTIO_ADMIN_CMD_RESOURCE_OBJ_CREATE when the resource \field{type} is
VIRTIO_NET_RESOURCE_OBJ_FF_CLASSIFIER, if the number of flow filter selector
objects in the device exceeds the configured driver capability
\field{selectors_limit}.

The device SHOULD set \field{status} to VIRTIO_ADMIN_STATUS_EBUSY for the
command VIRTIO_ADMIN_CMD_RESOURCE_OBJ_DESTROY for a flow filter group when
the flow filter group has one or more flow filter rules depending on it.

The device SHOULD set \field{status} to VIRTIO_ADMIN_STATUS_EBUSY for the
command VIRTIO_ADMIN_CMD_RESOURCE_OBJ_DESTROY for a flow filter classifier when
the flow filter classifier has one or more flow filter rules depending on it.

The device SHOULD fail the command VIRTIO_ADMIN_CMD_RESOURCE_OBJ_CREATE for the
flow filter rule resource object if,
\begin{itemize}
\item \field{vq_index} is not a valid receive virtqueue index for
the VIRTIO_NET_FF_ACTION_DIRECT_RX_VQ action,
\item \field{priority} is greater than or equal to
      \field{last_rule_priority},
\item \field{id} is greater than or equal to \field{rules_limit} or
      greater than or equal to \field{rules_per_group_limit}, whichever is lower,
\item the length of \field{keys} and the length of all the mask bytes of
      \field{selectors[].mask} as referred by \field{classifier_id} differs,
\item the supplied \field{action} is not supported in the capability VIRTIO_NET_FF_ACTION_CAP.
\end{itemize}

When the flow filter directs a packet to the virtqueue identified by
\field{vq_index} and if the receive virtqueue is reset, the device
MUST drop such packets.

Upon applying a flow filter rule to a packet, the device MUST STOP any further
application of rules and cease applying any other steering configurations.

For multiple flow filter groups, the device MUST apply the rules from
the group with the highest priority. If any rule from this group is applied,
the device MUST ignore the remaining groups. If none of the rules from the
highest priority group match, the device MUST apply the rules from
the group with the next highest priority, until either a rule matches or
all groups have been attempted.

The device MUST apply the rules within the group from the highest to the
lowest priority until a rule matches the packet, and the device MUST take
the action. If an action is taken, the device MUST not take any other
action for this packet.

The device MAY apply the rules with the same \field{rule_priority} in any
order within the group.

The device MUST process incoming packets in the following order:
\begin{itemize}
\item apply the steering configuration received using control virtqueue
      commands VIRTIO_NET_CTRL_RX, VIRTIO_NET_CTRL_MAC, and
      VIRTIO_NET_CTRL_VLAN.
\item apply flow filter rules if any.
\item if no filter rule is applied, apply the steering configuration
      received using the command VIRTIO_NET_CTRL_MQ_RSS_CONFIG
      or according to automatic receive steering.
\end{itemize}

When processing an incoming packet, if the packet is dropped at any stage, the device
MUST skip further processing.

When the device drops the packet due to the configuration done using the control
virtqueue commands VIRTIO_NET_CTRL_RX or VIRTIO_NET_CTRL_MAC or VIRTIO_NET_CTRL_VLAN,
the device MUST skip flow filter rules for this packet.

When the device performs flow filter match operations and if the operation
result did not have any match in all the groups, the receive packet processing
continues to next level, i.e. to apply configuration done using
VIRTIO_NET_CTRL_MQ_RSS_CONFIG command.

The device MUST support the creation of flow filter classifier objects
using the command VIRTIO_ADMIN_CMD_RESOURCE_OBJ_CREATE with \field{flags}
set to VIRTIO_NET_FF_MASK_F_PARTIAL_MASK;
this support is required even if all the bits of the masks are set for
a field in \field{selectors}, provided that partial masking is supported
for the selectors.

\drivernormative{\paragraph}{Flow filter}{Device Types / Network Device / Device Operation / Flow filter}

The driver MUST enable VIRTIO_NET_FF_RESOURCE_CAP, VIRTIO_NET_FF_SELECTOR_CAP,
and VIRTIO_NET_FF_ACTION_CAP capabilities to use flow filter.

The driver SHOULD NOT remove a flow filter group using the command
VIRTIO_ADMIN_CMD_RESOURCE_OBJ_DESTROY when one or more flow filter rules
depend on that group. The driver SHOULD only destroy the group after
all the associated rules have been destroyed.

The driver SHOULD NOT remove a flow filter classifier using the command
VIRTIO_ADMIN_CMD_RESOURCE_OBJ_DESTROY when one or more flow filter rules
depend on the classifier. The driver SHOULD only destroy the classifier
after all the associated rules have been destroyed.

The driver SHOULD NOT add multiple flow filter rules with the same
\field{rule_priority} within a flow filter group, as these rules MAY match
the same packet. The driver SHOULD assign different \field{rule_priority}
values to different flow filter rules if multiple rules may match a single
packet.

For the command VIRTIO_ADMIN_CMD_RESOURCE_OBJ_CREATE, when creating a resource
of \field{type} VIRTIO_NET_RESOURCE_OBJ_FF_CLASSIFIER, the driver MUST set:
\begin{itemize}
\item \field{selectors[0].type} to VIRTIO_NET_FF_MASK_TYPE_ETH.
\item \field{selectors[1].type} to VIRTIO_NET_FF_MASK_TYPE_IPV4 or
      VIRTIO_NET_FF_MASK_TYPE_IPV6 when \field{count} is more than 1,
\item \field{selectors[2].type} VIRTIO_NET_FF_MASK_TYPE_UDP or
      VIRTIO_NET_FF_MASK_TYPE_TCP when \field{count} is more than 2.
\end{itemize}

For the command VIRTIO_ADMIN_CMD_RESOURCE_OBJ_CREATE, when creating a resource
of \field{type} VIRTIO_NET_RESOURCE_OBJ_FF_CLASSIFIER, the driver MUST set:
\begin{itemize}
\item \field{selectors[0].mask} bytes to all 1s for the \field{EtherType}
       when \field{count} is 2 or more.
\item \field{selectors[1].mask} bytes to all 1s for \field{Protocol} or \field{Next Header}
       when \field{selector[1].type} is VIRTIO_NET_FF_MASK_TYPE_IPV4 or VIRTIO_NET_FF_MASK_TYPE_IPV6,
       and when \field{count} is more than 2.
\end{itemize}

For the command VIRTIO_ADMIN_CMD_RESOURCE_OBJ_CREATE, the resource \field{type}
VIRTIO_NET_RESOURCE_OBJ_FF_RULE, if the corresponding classifier object's
\field{count} is 2 or more, the driver MUST SET the \field{keys} bytes of
\field{EtherType} in accordance with
\hyperref[intro:IEEE 802 Ethertypes]{IEEE 802 Ethertypes}
for either VIRTIO_NET_FF_MASK_TYPE_IPV4 or VIRTIO_NET_FF_MASK_TYPE_IPV6.

For the command VIRTIO_ADMIN_CMD_RESOURCE_OBJ_CREATE, when creating a resource of
\field{type} VIRTIO_NET_RESOURCE_OBJ_FF_RULE, if the corresponding classifier
object's \field{count} is more than 2, and the \field{selector[1].type} is either
VIRTIO_NET_FF_MASK_TYPE_IPV4 or VIRTIO_NET_FF_MASK_TYPE_IPV6, the driver MUST
set the \field{keys} bytes for the \field{Protocol} or \field{Next Header}
according to \hyperref[intro:IANA Protocol Numbers]{IANA Protocol Numbers} respectively.

The driver SHOULD set all the bits for a field in the mask of a selector in both the
capability and the classifier object, unless the VIRTIO_NET_FF_MASK_F_PARTIAL_MASK
is enabled.

\subsubsection{Legacy Interface: Framing Requirements}\label{sec:Device
Types / Network Device / Legacy Interface: Framing Requirements}

When using legacy interfaces, transitional drivers which have not
negotiated VIRTIO_F_ANY_LAYOUT MUST use a single descriptor for the
\field{struct virtio_net_hdr} on both transmit and receive, with the
network data in the following descriptors.

Additionally, when using the control virtqueue (see \ref{sec:Device
Types / Network Device / Device Operation / Control Virtqueue})
, transitional drivers which have not
negotiated VIRTIO_F_ANY_LAYOUT MUST:
\begin{itemize}
\item for all commands, use a single 2-byte descriptor including the first two
fields: \field{class} and \field{command}
\item for all commands except VIRTIO_NET_CTRL_MAC_TABLE_SET
use a single descriptor including command-specific-data
with no padding.
\item for the VIRTIO_NET_CTRL_MAC_TABLE_SET command use exactly
two descriptors including command-specific-data with no padding:
the first of these descriptors MUST include the
virtio_net_ctrl_mac table structure for the unicast addresses with no padding,
the second of these descriptors MUST include the
virtio_net_ctrl_mac table structure for the multicast addresses
with no padding.
\item for all commands, use a single 1-byte descriptor for the
\field{ack} field
\end{itemize}

See \ref{sec:Basic
Facilities of a Virtio Device / Virtqueues / Message Framing}.

\section{Memory Device}\label{sec:Device Types / Memory Device}

The virtio memory device provides and manages a memory region in guest
physical address space.  This memory region is partitioned into memory
blocks of fixed size that can either be in the state plugged or unplugged.
Once plugged, a memory block can be used like ordinary RAM.  The driver
selects memory blocks to (un)plug and requests the device to perform the
(un)plug.

The device requests the driver to plug a certain amount of memory, by
setting the \field{requested_size} in the device configuration, which can
change at runtime.  It is up to the device driver to fulfill this request
by (un)plugging memory blocks.  Once the \field{plugged_size} is greater or
equal to the \field{requested_size}, requests to plug memory blocks will be
rejected by the device.

The device-managed memory region is split into two parts, the usable region
and the unusable region.  All memory blocks in the unusable region are
unplugged and requests to plug them will be rejected.  The device will grow
the usable region to fit the \field{requested_size}.  Usually, the usable
region is bigger than the \field{requested_size} of the device, to give the
driver some flexibility when selecting memory blocks to plug.

On initial start, and after a system reset, all memory blocks are
unplugged.  In corner cases, memory blocks might still be plugged after a
system reset, and the driver usually requests to unplug all memory while
initializing, before starting to select memory blocks to plug.

The device-managed memory region is not exposed as RAM via other firmware
/ hw interfaces (e.g., e820 on x86).  The driver is responsible for
deciding how plugged memory blocks will be used.  A common use case is to
expose plugged memory blocks to the operating system as system RAM,
available for the page allocator.

Some platforms provide memory properties for system RAM that are usually
queried and modified using special CPU instructions. Memory properties might
be implicitly queried or modified on memory access. Memory properties can
include advanced memory protection, access and change indication, or memory
usage indication relevant in virtualized environments. \footnote{For example,
s390x provides storage keys for each 4 KiB page and may, depending on the
configuration, provide storage attributes for each 4 KiB page.} The device
provides the exact same properties with the exact same semantics for
plugged device memory as available for comparable RAM in the same configuration.

\subsection{Device ID}\label{sec:Device Types / Memory Device / Device ID}
24

\subsection{Virtqueues}\label{sec:Device Types / Memory Device / Virtqueues}

\begin{description}
\item[0] guest-request
\end{description}

\subsection{Feature bits}\label{sec:Device Types / Memory Device / Feature bits}

\begin{description}
\item[VIRTIO_MEM_F_ACPI_PXM (0)] The field \field{node_id} in the device
configuration is valid and corresponds to an ACPI PXM.
\item[VIRTIO_MEM_F_UNPLUGGED_INACCESSIBLE (1)] The driver is not allowed to
access unplugged memory. \footnote{On platforms with memory properties that
might get modified implicitly on memory access, this feature is expected to
be offered by the device.}
\item[VIRTIO_MEM_F_PERSISTENT_SUSPEND (2)] The driver can allow the guest
to enter suspended state (deep sleep, suspend-to-RAM).
\end{description}

\subsection{Device configuration layout}\label{sec:Device Types / Memory Device / Device configuration layout}

All fields of this configuration are always available and read-only for the
driver.

\begin{lstlisting}
struct virtio_mem_config {
  le64 block_size;
  le16 node_id;
  le8 padding[6];
  le64 addr;
  le64 region_size;
  le64 usable_region_size;
  le64 plugged_size;
  le64 requested_size;
};
\end{lstlisting}

\begin{description}
\item[\field{block_size}] is the size and the alignment in bytes of a
memory block.  Cannot change.
\item[\field{node_id}] has no meaning without VIRTIO_MEM_F_ACPI_PXM.  With
VIRTIO_MEM_F_ACPI_PXM, this field is valid and corresponds to an ACPI PXM.
Cannot change.
\item[\field{padding}] has no meaning and is reserved for future use.
\item[\field{addr}] is the guest physical address of the start of the
device-managed memory region in bytes.  Cannot change.
\item[\field{region_size}] is the size of device-managed memory region in
bytes.  Cannot change.
\item[\field{usable_region_size}] is the size of the usable device-managed
memory region.  Can grow up to \field{region_size}.  Can only shrink due to
VIRTIO_MEM_REQ_UNPLUG_ALL requests.
\item[\field{plugged_size}] is the amount of plugged memory in bytes within
the usable device-managed memory region.
\item[\field{requested_size}] is the requested amount of plugged memory
within the usable device-managed memory region.
\end{description}

\drivernormative{\subsubsection}{Device configuration layout}{Device Types / Memory Device / Device configuration layout}

The driver MUST NOT write to device configuration fields.

The driver MUST ignore the value of \field{padding}.

The driver MUST ignore the value of \field{node_id} without
VIRTIO_MEM_F_ACPI_PXM.

\devicenormative{\subsubsection}{Device configuration layout}{Device Types / Memory Device / Device configuration layout}

The device MAY change \field{usable_region_size} and
\field{requested_size}.

The device MUST NOT change \field{block_size}, \field{node_id},
\field{addr}, and \field{region_size}, except during a system reset.

The device MUST change \field{plugged_size} to reflect the size of plugged
memory blocks.

The device MUST set \field{usable_region_size} to \field{requested_size} or
greater.

The device MUST set \field{block_size} to a power of two.

The device MUST set \field{addr}, \field{region_size},
\field{usable_region_size}, \field{plugged_size}, \field{requested_size} to
multiples of \field{block_size}.

The device MUST set \field{region_size} to 0 or greater.

The device MUST NOT shrink \field{usable_region_size}, except when
processing an UNPLUG ALL request, or during a system reset.

The device MUST send a configuration update notification when changing
\field{usable_region_size} or \field{requested_size}, except when
processing an UNPLUG ALL request.

The device SHOULD NOT send a configuration update notification when
changing \field{plugged_size}.

The device MAY send a configuration update notification even if nothing
changed.

\subsection{Device Initialization}\label{Device Types / Memory Device / Device Initialization}

On initialization, the driver first discovers the device's virtqueues.  It
then reads the device configuration.

In case the driver detects that the device still has memory plugged
(\field{plugged_size} in the device configuration is greater than 0), the
driver will either try to re-initialize by issuing STATE requests, or try
to unplug all memory before continuing.  Special handling might be
necessary in case some plugged memory might still be relevant (e.g., system
dump, memory still in use after unloading the driver).

\drivernormative{\subsubsection}{Device Initialization}{Device Types / Memory Device / Device Initialization}

The driver SHOULD accept VIRTIO_MEM_F_UNPLUGGED_INACCESSIBLE if it is
offered and the driver supports it.

The driver SHOULD issue UNPLUG ALL requests until successful if the device
still has memory plugged and the plugged memory is not in use.

\devicenormative{\subsubsection}{Device Initialization}{Device Types / Memory Device / Device Initialization}

A device MAY fail to operate further if VIRTIO_MEM_F_UNPLUGGED_INACCESSIBLE
is not accepted.

The device MUST NOT change the state of memory blocks during device reset.

The device MUST NOT modify memory or memory properties of plugged memory
blocks during device reset.

The device SHOULD offer VIRTIO_MEM_F_PERSISTENT_SUSPEND if the platform
supports suspending (deep sleep, suspend-to-RAM) with plugged memory blocks.

\subsection{Device Operation}\label{sec:Device Types / Memory Device / Device Operation}

The device notifies the driver about the amount of memory the device wants
the driver to consume via the device.  These resize requests from the
device are communciated via the \field{requested_size} in the device
configuration.  The driver will react by requesting to (un)plug specific
memory blocks, to make the \field{plugged_size} match the
\field{requested_size} as close as possible.

The driver sends requests to the device on the guest-request virtqueue,
notifies the device, and waits for the device to respond.  Requests have a
common header, defining the request type, followed by request-specific
data.  All requests are 24 bytes long and have the layout:

\begin{lstlisting}
struct virtio_mem_req {
  le16 type;
  le16 padding[3];

  union {
    struct virtio_mem_req_plug plug;
    struct virtio_mem_req_unplug unplug;
    struct virtio_mem_req_state state;
  } u;
}
\end{lstlisting}

Possible request types are:

\begin{lstlisting}
#define VIRTIO_MEM_REQ_PLUG            0
#define VIRTIO_MEM_REQ_UNPLUG          1
#define VIRTIO_MEM_REQ_UNPLUG_ALL      2
#define VIRTIO_MEM_REQ_STATE           3
\end{lstlisting}

Responses have a common header, defining the response type, followed by
request-specific data.  All responses are 10 bytes long and have the layout:

\begin{lstlisting}
struct virtio_mem_resp {
  le16 type;
  le16 padding[3];

  union {
    struct virtio_mem_resp_state state;
  } u;
}
\end{lstlisting}

Possible response types, in general, are:

\begin{lstlisting}
#define VIRTIO_MEM_RESP_ACK            0
#define VIRTIO_MEM_RESP_NACK           1
#define VIRTIO_MEM_RESP_BUSY           2
#define VIRTIO_MEM_RESP_ERROR          3
\end{lstlisting}

\drivernormative{\subsubsection}{Device Operation}{Device Types / Memory Device / Device Operation}

The driver MUST NOT write memory or modify memory properties of
unplugged memory blocks.

The driver MUST NOT read memory or query memory properties of unplugged memory
blocks outside \field{usable_region_size}.

The driver MUST NOT read memory or query memory properties of unplugged memory
blocks inside \field{usable_region_size} via DMA.

If VIRTIO_MEM_F_UNPLUGGED_INACCESSIBLE has not been negotiated, the driver
SHOULD NOT read memory or query memory properties of unplugged memory blocks
inside \field{usable_region_size} via the CPU.

If VIRTIO_MEM_F_UNPLUGGED_INACCESSIBLE has been negotiated, the driver
MUST NOT read memory or query memory properties of unplugged memory blocks.

The driver MUST NOT request unplug of memory blocks while corresponding memory
or memory properties are still in use.

The driver SHOULD initialize memory blocks after plugging them, the content
is undefined.

The driver SHOULD react to resize requests from the device
(\field{requested_size} in the device configuration changed) by
(un)plugging memory blocks.

The driver SHOULD only plug memory blocks it can actually use.

The driver MAY not reach the requested size (\field{requested_size} in the
device configuration), for example, because it cannot free up any plugged
memory blocks to unplug them, or it would not be able to make use of
unplugged memory blocks after plugging them (e.g., alignment).

If VIRTIO_MEM_F_PERSISTENT_SUSPEND has not been negotiated, the driver MUST NOT
allow the guest to enter a suspended state (deep sleep, suspend-to-RAM).

\devicenormative{\subsubsection}{Device Operation}{Device Types / Memory Device / Device Operation}

The device MUST provide the exact same memory properties with the exact same
semantics for device memory the platform provides in the same configuration for
comparable RAM.

The device MAY modify memory of unplugged memory blocks or reset memory
properties of such memory blocks to platform defaults at any time.

The device MUST NOT modify memory or memory properties of plugged memory
blocks.

The device MUST allow the driver to read and write memory and to query
and modify memory attributes of plugged memory blocks.

If VIRTIO_MEM_F_UNPLUGGED_INACCESSIBLE has not been negotiated, the device
MUST allow the driver to read memory and to query memory properties of
unplugged memory blocks inside \field{usable_region_size} via the CPU.
\footnote{To allow for simplified dumping of memory. The CPU is expected to
copy such memory to another location before starting DMA.}

The device MAY change the state of memory blocks during system resets.

The device SHOULD unplug all memory blocks during system resets.

If VIRTIO_MEM_F_PERSISTENT_SUSPEND has been negotiated, the device MUST NOT not
change the state of memory blocks when suspending or when waking up from
suspended state (deep sleep, suspend-to-RAM).

\subsubsection{PLUG request}\label{sec:Device Types / Memory Device / Device Operation / PLUG request}

Request to plug consecutive memory blocks that are currently unplugged.

The request-specific data in a PLUG request has the format:

\begin{lstlisting}
struct virtio_mem_req_plug {
  le64 addr;
  le16 nb_blocks;
  le16 padding[3];
}
\end{lstlisting}

\field{addr} is the guest physical address of the first memory block.
\field{nb_blocks} is the number of consecutive memory blocks

Responses don't have request-specific data defined.

\drivernormative{\paragraph}{PLUG request}{Device Types / Memory Device / Device Operation / PLUG request}

The driver MUST ignore anything except the response type in a response.

\devicenormative{\paragraph}{PLUG request}{Device Types / Memory Device / Device Operation / PLUG request}

The device MUST ignore anything except the request type and the
request-specific data in a request.

The device MUST ignore the \field{padding} in the request-specific data in
a request.

The device MUST reject requests with VIRTIO_MEM_RESP_ERROR if \field{addr}
is not aligned to the \field{block_size} in the device configuration, if
\field{nb_blocks} is not greater than 0, or if any memory block outside of
the usable device-managed memory region is covered by the request.

The device MUST reject requests with VIRTIO_MEM_RESP_ERROR if any memory
block covered by the request is already plugged.

The device MAY reject requests with VIRTIO_MEM_RESP_BUSY if the request can
currently not be processed.

The device MUST acknowledge requests with VIRTIO_MEM_RESP_ACK in case all
memory blocks were successfully plugged.  The device MUST reflect the
change in the device configuration \field{plugged_size}.

\subsubsection{UNPLUG request}\label{sec:Device Types / Memory Device / Device Operation / UNPLUG request}

Request to unplug consecutive memory blocks that are currently plugged.

The request-specific data in an UNPLUG request has the format:

\begin{lstlisting}
struct virtio_mem_req_unplug {
  le64 addr;
  le16 nb_blocks;
  le16 padding[3];
}
\end{lstlisting}

\field{addr} is the guest physical address of the first memory block.
\field{nb_blocks} is the number of consecutive memory blocks

Responses don't have request-specific data defined.

\drivernormative{\paragraph}{UNPLUG request}{Device Types / Memory Device / Device Operation / UNPLUG request}

The driver MUST ignore anything except the response type in a response.

\devicenormative{\paragraph}{UNPLUG request}{Device Types / Memory Device / Device Operation / UNPLUG request}

The device MUST ignore anything except the request type and the
request-specific data in a request.

The device MUST ignore the \field{padding} in the request-specific data in
a request.

The device MUST reject requests with VIRTIO_MEM_RESP_ERROR if \field{addr}
is not aligned to the \field{block_size} in the device configuration, if
\field{nb_blocks} is not greater than 0, or if any memory block outside of
the usable device-managed memory region is covered by the request.

The device MUST reject requests with VIRTIO_MEM_RESP_ERROR if any memory
block covered by the request is already unplugged.

The device MAY reject requests with VIRTIO_MEM_RESP_BUSY if the request can
currently not be processed.

The device MUST acknowledge requests with VIRTIO_MEM_RESP_ACK in case all
memory blocks were successfully unplugged.  The device MUST reflect the
change in the device configuration \field{plugged_size}.

\subsubsection{UNPLUG ALL request}\label{sec:Device Types / Memory Device / Device Operation / UNPLUG ALL request}

Request to unplug all memory blocks the device has currently plugged.  If
successful, the \field{plugged_size} in the device configuration will be 0
and \field{usable_region_size} might have changed.

Requests don't have request-specific data defined, only the request type is
relevant.  Responses don't have request-specific data defined, only the
response type is relevant.

\drivernormative{\paragraph}{UNPLUG request}{Device Types / Memory Device / Device Operation / UNPLUG ALL request}

The driver MUST ignore any data in a response except the response type.

\devicenormative{\paragraph}{UNPLUG request}{Device Types / Memory Device / Device Operation / UNPLUG ALL request}

The device MUST ignore any data in a request except the request type.

The device MUST ignore the \field{padding} in the request-specific data in
a request.

The device MAY reject requests with VIRTIO_MEM_RESP_BUSY if the request can
currently not be processed.

The device MUST acknowledge requests with VIRTIO_MEM_RESP_ACK in case all
memory blocks were successfully unplugged.

The device MUST set \field{plugged_size} to 0 in case the request is
acknowledged with VIRTIO_MEM_RESP_ACK.

The device MAY modify \field{usable_region_size} before responding with
VIRTIO_MEM_RESP_ACK.

\subsubsection{STATE request}\label{sec:Device Types / Memory Device / Device Operation / STATE request}

Request the state (plugged, unplugged, mixed) of consecutive memory blocks.

The request-specific data in a STATE request has the format:

\begin{lstlisting}
struct virtio_mem_req_state {
  le64 addr;
  le16 nb_blocks;
  le16 padding[3];
};
\end{lstlisting}

\field{addr} is the guest physical address of the first memory block.
\field{nb_blocks} is the number of consecutive memory blocks.

The request-specific data in a STATE response has the format:

\begin{lstlisting}
struct virtio_mem_resp_state {
  le16 type;
};
\end{lstlisting}

Whereby \field{type} defines one of three different state types:

\begin{lstlisting}
#define VIRTIO_MEM_STATE_PLUGGED        0
#define VIRTIO_MEM_STATE_UNPLUGGED      1
#define VIRTIO_MEM_STATE_MIXED          2
\end{lstlisting}

\drivernormative{\paragraph}{STATE request}{Device Types / Memory Device / Device Operation / STATE request}

The driver MUST ignore anything except the response type and the
request-specific data in a response.

The driver MUST ignore the request-specific data in a response in case the
response type is not VIRTIO_MEM_RESP_ACK.

\devicenormative{\paragraph}{STATE request}{Device Types / Memory Device / Device Operation / STATE request}

The device MUST ignore anything except the request type and the
request-specific data in a request.

The device MUST ignore the \field{padding} in the request-specific data in
a request.

The device MUST reject requests with VIRTIO_MEM_RESP_ERROR if \field{addr}
is not aligned to the \field{block_size} in the device configuration, if
\field{nb_blocks} is not greater than 0, or if any memory block outside of
the usable device-managed memory region is covered by the request.

The device MUST acknowledge requests with VIRTIO_MEM_RESP_ACK, supplying
the state of the memory blocks.

The device MUST set the state type in the response to
VIRTIO_MEM_STATE_PLUGGED if all requested memory blocks are plugged.  The
device MUST set the state type in the response to
VIRTIO_MEM_STATE_UNPLUGGED if all requested memory blocks are unplugged.
Otherwise, the device MUST set state type in the response to
VIRTIO_MEM_STATE_MIXED.

\section{Network Device}\label{sec:Device Types / Network Device}

The virtio network device is a virtual network interface controller.
It consists of a virtual Ethernet link which connects the device
to the Ethernet network. The device has transmit and receive
queues. The driver adds empty buffers to the receive virtqueue.
The device receives incoming packets from the link; the device
places these incoming packets in the receive virtqueue buffers.
The driver adds outgoing packets to the transmit virtqueue. The device
removes these packets from the transmit virtqueue and sends them to
the link. The device may have a control virtqueue. The driver
uses the control virtqueue to dynamically manipulate various
features of the initialized device.

\subsection{Device ID}\label{sec:Device Types / Network Device / Device ID}

 1

\subsection{Virtqueues}\label{sec:Device Types / Network Device / Virtqueues}

\begin{description}
\item[0] receiveq1
\item[1] transmitq1
\item[\ldots]
\item[2(N-1)] receiveqN
\item[2(N-1)+1] transmitqN
\item[2N] controlq
\end{description}

 N=1 if neither VIRTIO_NET_F_MQ nor VIRTIO_NET_F_RSS are negotiated, otherwise N is set by
 \field{max_virtqueue_pairs}.

controlq is optional; it only exists if VIRTIO_NET_F_CTRL_VQ is
negotiated.

\subsection{Feature bits}\label{sec:Device Types / Network Device / Feature bits}

\begin{description}
\item[VIRTIO_NET_F_CSUM (0)] Device handles packets with partial checksum offload.

\item[VIRTIO_NET_F_GUEST_CSUM (1)] Driver handles packets with partial checksum.

\item[VIRTIO_NET_F_CTRL_GUEST_OFFLOADS (2)] Control channel offloads
        reconfiguration support.

\item[VIRTIO_NET_F_MTU(3)] Device maximum MTU reporting is supported. If
    offered by the device, device advises driver about the value of
    its maximum MTU. If negotiated, the driver uses \field{mtu} as
    the maximum MTU value.

\item[VIRTIO_NET_F_MAC (5)] Device has given MAC address.

\item[VIRTIO_NET_F_GUEST_TSO4 (7)] Driver can receive TSOv4.

\item[VIRTIO_NET_F_GUEST_TSO6 (8)] Driver can receive TSOv6.

\item[VIRTIO_NET_F_GUEST_ECN (9)] Driver can receive TSO with ECN.

\item[VIRTIO_NET_F_GUEST_UFO (10)] Driver can receive UFO.

\item[VIRTIO_NET_F_HOST_TSO4 (11)] Device can receive TSOv4.

\item[VIRTIO_NET_F_HOST_TSO6 (12)] Device can receive TSOv6.

\item[VIRTIO_NET_F_HOST_ECN (13)] Device can receive TSO with ECN.

\item[VIRTIO_NET_F_HOST_UFO (14)] Device can receive UFO.

\item[VIRTIO_NET_F_MRG_RXBUF (15)] Driver can merge receive buffers.

\item[VIRTIO_NET_F_STATUS (16)] Configuration status field is
    available.

\item[VIRTIO_NET_F_CTRL_VQ (17)] Control channel is available.

\item[VIRTIO_NET_F_CTRL_RX (18)] Control channel RX mode support.

\item[VIRTIO_NET_F_CTRL_VLAN (19)] Control channel VLAN filtering.

\item[VIRTIO_NET_F_CTRL_RX_EXTRA (20)]	Control channel RX extra mode support.

\item[VIRTIO_NET_F_GUEST_ANNOUNCE(21)] Driver can send gratuitous
    packets.

\item[VIRTIO_NET_F_MQ(22)] Device supports multiqueue with automatic
    receive steering.

\item[VIRTIO_NET_F_CTRL_MAC_ADDR(23)] Set MAC address through control
    channel.

\item[VIRTIO_NET_F_DEVICE_STATS(50)] Device can provide device-level statistics
    to the driver through the control virtqueue.

\item[VIRTIO_NET_F_HASH_TUNNEL(51)] Device supports inner header hash for encapsulated packets.

\item[VIRTIO_NET_F_VQ_NOTF_COAL(52)] Device supports virtqueue notification coalescing.

\item[VIRTIO_NET_F_NOTF_COAL(53)] Device supports notifications coalescing.

\item[VIRTIO_NET_F_GUEST_USO4 (54)] Driver can receive USOv4 packets.

\item[VIRTIO_NET_F_GUEST_USO6 (55)] Driver can receive USOv6 packets.

\item[VIRTIO_NET_F_HOST_USO (56)] Device can receive USO packets. Unlike UFO
 (fragmenting the packet) the USO splits large UDP packet
 to several segments when each of these smaller packets has UDP header.

\item[VIRTIO_NET_F_HASH_REPORT(57)] Device can report per-packet hash
    value and a type of calculated hash.

\item[VIRTIO_NET_F_GUEST_HDRLEN(59)] Driver can provide the exact \field{hdr_len}
    value. Device benefits from knowing the exact header length.

\item[VIRTIO_NET_F_RSS(60)] Device supports RSS (receive-side scaling)
    with Toeplitz hash calculation and configurable hash
    parameters for receive steering.

\item[VIRTIO_NET_F_RSC_EXT(61)] Device can process duplicated ACKs
    and report number of coalesced segments and duplicated ACKs.

\item[VIRTIO_NET_F_STANDBY(62)] Device may act as a standby for a primary
    device with the same MAC address.

\item[VIRTIO_NET_F_SPEED_DUPLEX(63)] Device reports speed and duplex.

\item[VIRTIO_NET_F_RSS_CONTEXT(64)] Device supports multiple RSS contexts.

\item[VIRTIO_NET_F_GUEST_UDP_TUNNEL_GSO (65)] Driver can receive GSO packets
  carried by a UDP tunnel.

\item[VIRTIO_NET_F_GUEST_UDP_TUNNEL_GSO_CSUM (66)] Driver handles packets
  carried by a UDP tunnel with partial csum for the outer header.

\item[VIRTIO_NET_F_HOST_UDP_TUNNEL_GSO (67)] Device can receive GSO packets
  carried by a UDP tunnel.

\item[VIRTIO_NET_F_HOST_UDP_TUNNEL_GSO_CSUM (68)] Device handles packets
  carried by a UDP tunnel with partial csum for the outer header.
\end{description}

\subsubsection{Feature bit requirements}\label{sec:Device Types / Network Device / Feature bits / Feature bit requirements}

Some networking feature bits require other networking feature bits
(see \ref{drivernormative:Basic Facilities of a Virtio Device / Feature Bits}):

\begin{description}
\item[VIRTIO_NET_F_GUEST_TSO4] Requires VIRTIO_NET_F_GUEST_CSUM.
\item[VIRTIO_NET_F_GUEST_TSO6] Requires VIRTIO_NET_F_GUEST_CSUM.
\item[VIRTIO_NET_F_GUEST_ECN] Requires VIRTIO_NET_F_GUEST_TSO4 or VIRTIO_NET_F_GUEST_TSO6.
\item[VIRTIO_NET_F_GUEST_UFO] Requires VIRTIO_NET_F_GUEST_CSUM.
\item[VIRTIO_NET_F_GUEST_USO4] Requires VIRTIO_NET_F_GUEST_CSUM.
\item[VIRTIO_NET_F_GUEST_USO6] Requires VIRTIO_NET_F_GUEST_CSUM.
\item[VIRTIO_NET_F_GUEST_UDP_TUNNEL_GSO] Requires VIRTIO_NET_F_GUEST_TSO4, VIRTIO_NET_F_GUEST_TSO6,
   VIRTIO_NET_F_GUEST_USO4 and VIRTIO_NET_F_GUEST_USO6.
\item[VIRTIO_NET_F_GUEST_UDP_TUNNEL_GSO_CSUM] Requires VIRTIO_NET_F_GUEST_UDP_TUNNEL_GSO

\item[VIRTIO_NET_F_HOST_TSO4] Requires VIRTIO_NET_F_CSUM.
\item[VIRTIO_NET_F_HOST_TSO6] Requires VIRTIO_NET_F_CSUM.
\item[VIRTIO_NET_F_HOST_ECN] Requires VIRTIO_NET_F_HOST_TSO4 or VIRTIO_NET_F_HOST_TSO6.
\item[VIRTIO_NET_F_HOST_UFO] Requires VIRTIO_NET_F_CSUM.
\item[VIRTIO_NET_F_HOST_USO] Requires VIRTIO_NET_F_CSUM.
\item[VIRTIO_NET_F_HOST_UDP_TUNNEL_GSO] Requires VIRTIO_NET_F_HOST_TSO4, VIRTIO_NET_F_HOST_TSO6
   and VIRTIO_NET_F_HOST_USO.
\item[VIRTIO_NET_F_HOST_UDP_TUNNEL_GSO_CSUM] Requires VIRTIO_NET_F_HOST_UDP_TUNNEL_GSO

\item[VIRTIO_NET_F_CTRL_RX] Requires VIRTIO_NET_F_CTRL_VQ.
\item[VIRTIO_NET_F_CTRL_VLAN] Requires VIRTIO_NET_F_CTRL_VQ.
\item[VIRTIO_NET_F_GUEST_ANNOUNCE] Requires VIRTIO_NET_F_CTRL_VQ.
\item[VIRTIO_NET_F_MQ] Requires VIRTIO_NET_F_CTRL_VQ.
\item[VIRTIO_NET_F_CTRL_MAC_ADDR] Requires VIRTIO_NET_F_CTRL_VQ.
\item[VIRTIO_NET_F_NOTF_COAL] Requires VIRTIO_NET_F_CTRL_VQ.
\item[VIRTIO_NET_F_RSC_EXT] Requires VIRTIO_NET_F_HOST_TSO4 or VIRTIO_NET_F_HOST_TSO6.
\item[VIRTIO_NET_F_RSS] Requires VIRTIO_NET_F_CTRL_VQ.
\item[VIRTIO_NET_F_VQ_NOTF_COAL] Requires VIRTIO_NET_F_CTRL_VQ.
\item[VIRTIO_NET_F_HASH_TUNNEL] Requires VIRTIO_NET_F_CTRL_VQ along with VIRTIO_NET_F_RSS or VIRTIO_NET_F_HASH_REPORT.
\item[VIRTIO_NET_F_RSS_CONTEXT] Requires VIRTIO_NET_F_CTRL_VQ and VIRTIO_NET_F_RSS.
\end{description}

\begin{note}
The dependency between UDP_TUNNEL_GSO_CSUM and UDP_TUNNEL_GSO is intentionally
in the opposite direction with respect to the plain GSO features and the plain
checksum offload because UDP tunnel checksum offload gives very little gain
for non GSO packets and is quite complex to implement in H/W.
\end{note}

\subsubsection{Legacy Interface: Feature bits}\label{sec:Device Types / Network Device / Feature bits / Legacy Interface: Feature bits}
\begin{description}
\item[VIRTIO_NET_F_GSO (6)] Device handles packets with any GSO type. This was supposed to indicate segmentation offload support, but
upon further investigation it became clear that multiple bits were needed.
\item[VIRTIO_NET_F_GUEST_RSC4 (41)] Device coalesces TCPIP v4 packets. This was implemented by hypervisor patch for certification
purposes and current Windows driver depends on it. It will not function if virtio-net device reports this feature.
\item[VIRTIO_NET_F_GUEST_RSC6 (42)] Device coalesces TCPIP v6 packets. Similar to VIRTIO_NET_F_GUEST_RSC4.
\end{description}

\subsection{Device configuration layout}\label{sec:Device Types / Network Device / Device configuration layout}
\label{sec:Device Types / Block Device / Feature bits / Device configuration layout}

The network device has the following device configuration layout.
All of the device configuration fields are read-only for the driver.

\begin{lstlisting}
struct virtio_net_config {
        u8 mac[6];
        le16 status;
        le16 max_virtqueue_pairs;
        le16 mtu;
        le32 speed;
        u8 duplex;
        u8 rss_max_key_size;
        le16 rss_max_indirection_table_length;
        le32 supported_hash_types;
        le32 supported_tunnel_types;
};
\end{lstlisting}

The \field{mac} address field always exists (although it is only
valid if VIRTIO_NET_F_MAC is set).

The \field{status} only exists if VIRTIO_NET_F_STATUS is set.
Two bits are currently defined for the status field: VIRTIO_NET_S_LINK_UP
and VIRTIO_NET_S_ANNOUNCE.

\begin{lstlisting}
#define VIRTIO_NET_S_LINK_UP     1
#define VIRTIO_NET_S_ANNOUNCE    2
\end{lstlisting}

The following field, \field{max_virtqueue_pairs} only exists if
VIRTIO_NET_F_MQ or VIRTIO_NET_F_RSS is set. This field specifies the maximum number
of each of transmit and receive virtqueues (receiveq1\ldots receiveqN
and transmitq1\ldots transmitqN respectively) that can be configured once at least one of these features
is negotiated.

The following field, \field{mtu} only exists if VIRTIO_NET_F_MTU
is set. This field specifies the maximum MTU for the driver to
use.

The following two fields, \field{speed} and \field{duplex}, only
exist if VIRTIO_NET_F_SPEED_DUPLEX is set.

\field{speed} contains the device speed, in units of 1 MBit per
second, 0 to 0x7fffffff, or 0xffffffff for unknown speed.

\field{duplex} has the values of 0x01 for full duplex, 0x00 for
half duplex and 0xff for unknown duplex state.

Both \field{speed} and \field{duplex} can change, thus the driver
is expected to re-read these values after receiving a
configuration change notification.

The following field, \field{rss_max_key_size} only exists if VIRTIO_NET_F_RSS or VIRTIO_NET_F_HASH_REPORT is set.
It specifies the maximum supported length of RSS key in bytes.

The following field, \field{rss_max_indirection_table_length} only exists if VIRTIO_NET_F_RSS is set.
It specifies the maximum number of 16-bit entries in RSS indirection table.

The next field, \field{supported_hash_types} only exists if the device supports hash calculation,
i.e. if VIRTIO_NET_F_RSS or VIRTIO_NET_F_HASH_REPORT is set.

Field \field{supported_hash_types} contains the bitmask of supported hash types.
See \ref{sec:Device Types / Network Device / Device Operation / Processing of Incoming Packets / Hash calculation for incoming packets / Supported/enabled hash types} for details of supported hash types.

Field \field{supported_tunnel_types} only exists if the device supports inner header hash, i.e. if VIRTIO_NET_F_HASH_TUNNEL is set.

Field \field{supported_tunnel_types} contains the bitmask of encapsulation types supported by the device for inner header hash.
Encapsulation types are defined in \ref{sec:Device Types / Network Device / Device Operation / Processing of Incoming Packets /
Hash calculation for incoming packets / Encapsulation types supported/enabled for inner header hash}.

\devicenormative{\subsubsection}{Device configuration layout}{Device Types / Network Device / Device configuration layout}

The device MUST set \field{max_virtqueue_pairs} to between 1 and 0x8000 inclusive,
if it offers VIRTIO_NET_F_MQ.

The device MUST set \field{mtu} to between 68 and 65535 inclusive,
if it offers VIRTIO_NET_F_MTU.

The device SHOULD set \field{mtu} to at least 1280, if it offers
VIRTIO_NET_F_MTU.

The device MUST NOT modify \field{mtu} once it has been set.

The device MUST NOT pass received packets that exceed \field{mtu} (plus low
level ethernet header length) size with \field{gso_type} NONE or ECN
after VIRTIO_NET_F_MTU has been successfully negotiated.

The device MUST forward transmitted packets of up to \field{mtu} (plus low
level ethernet header length) size with \field{gso_type} NONE or ECN, and do
so without fragmentation, after VIRTIO_NET_F_MTU has been successfully
negotiated.

The device MUST set \field{rss_max_key_size} to at least 40, if it offers
VIRTIO_NET_F_RSS or VIRTIO_NET_F_HASH_REPORT.

The device MUST set \field{rss_max_indirection_table_length} to at least 128, if it offers
VIRTIO_NET_F_RSS.

If the driver negotiates the VIRTIO_NET_F_STANDBY feature, the device MAY act
as a standby device for a primary device with the same MAC address.

If VIRTIO_NET_F_SPEED_DUPLEX has been negotiated, \field{speed}
MUST contain the device speed, in units of 1 MBit per second, 0 to
0x7ffffffff, or 0xfffffffff for unknown.

If VIRTIO_NET_F_SPEED_DUPLEX has been negotiated, \field{duplex}
MUST have the values of 0x00 for full duplex, 0x01 for half
duplex, or 0xff for unknown.

If VIRTIO_NET_F_SPEED_DUPLEX and VIRTIO_NET_F_STATUS have both
been negotiated, the device SHOULD NOT change the \field{speed} and
\field{duplex} fields as long as VIRTIO_NET_S_LINK_UP is set in
the \field{status}.

The device SHOULD NOT offer VIRTIO_NET_F_HASH_REPORT if it
does not offer VIRTIO_NET_F_CTRL_VQ.

The device SHOULD NOT offer VIRTIO_NET_F_CTRL_RX_EXTRA if it
does not offer VIRTIO_NET_F_CTRL_VQ.

\drivernormative{\subsubsection}{Device configuration layout}{Device Types / Network Device / Device configuration layout}

The driver MUST NOT write to any of the device configuration fields.

A driver SHOULD negotiate VIRTIO_NET_F_MAC if the device offers it.
If the driver negotiates the VIRTIO_NET_F_MAC feature, the driver MUST set
the physical address of the NIC to \field{mac}.  Otherwise, it SHOULD
use a locally-administered MAC address (see \hyperref[intro:IEEE 802]{IEEE 802},
``9.2 48-bit universal LAN MAC addresses'').

If the driver does not negotiate the VIRTIO_NET_F_STATUS feature, it SHOULD
assume the link is active, otherwise it SHOULD read the link status from
the bottom bit of \field{status}.

A driver SHOULD negotiate VIRTIO_NET_F_MTU if the device offers it.

If the driver negotiates VIRTIO_NET_F_MTU, it MUST supply enough receive
buffers to receive at least one receive packet of size \field{mtu} (plus low
level ethernet header length) with \field{gso_type} NONE or ECN.

If the driver negotiates VIRTIO_NET_F_MTU, it MUST NOT transmit packets of
size exceeding the value of \field{mtu} (plus low level ethernet header length)
with \field{gso_type} NONE or ECN.

A driver SHOULD negotiate the VIRTIO_NET_F_STANDBY feature if the device offers it.

If VIRTIO_NET_F_SPEED_DUPLEX has been negotiated,
the driver MUST treat any value of \field{speed} above
0x7fffffff as well as any value of \field{duplex} not
matching 0x00 or 0x01 as an unknown value.

If VIRTIO_NET_F_SPEED_DUPLEX has been negotiated, the driver
SHOULD re-read \field{speed} and \field{duplex} after a
configuration change notification.

A driver SHOULD NOT negotiate VIRTIO_NET_F_HASH_REPORT if it
does not negotiate VIRTIO_NET_F_CTRL_VQ.

A driver SHOULD NOT negotiate VIRTIO_NET_F_CTRL_RX_EXTRA if it
does not negotiate VIRTIO_NET_F_CTRL_VQ.

\subsubsection{Legacy Interface: Device configuration layout}\label{sec:Device Types / Network Device / Device configuration layout / Legacy Interface: Device configuration layout}
\label{sec:Device Types / Block Device / Feature bits / Device configuration layout / Legacy Interface: Device configuration layout}
When using the legacy interface, transitional devices and drivers
MUST format \field{status} and
\field{max_virtqueue_pairs} in struct virtio_net_config
according to the native endian of the guest rather than
(necessarily when not using the legacy interface) little-endian.

When using the legacy interface, \field{mac} is driver-writable
which provided a way for drivers to update the MAC without
negotiating VIRTIO_NET_F_CTRL_MAC_ADDR.

\subsection{Device Initialization}\label{sec:Device Types / Network Device / Device Initialization}

A driver would perform a typical initialization routine like so:

\begin{enumerate}
\item Identify and initialize the receive and
  transmission virtqueues, up to N of each kind. If
  VIRTIO_NET_F_MQ feature bit is negotiated,
  N=\field{max_virtqueue_pairs}, otherwise identify N=1.

\item If the VIRTIO_NET_F_CTRL_VQ feature bit is negotiated,
  identify the control virtqueue.

\item Fill the receive queues with buffers: see \ref{sec:Device Types / Network Device / Device Operation / Setting Up Receive Buffers}.

\item Even with VIRTIO_NET_F_MQ, only receiveq1, transmitq1 and
  controlq are used by default.  The driver would send the
  VIRTIO_NET_CTRL_MQ_VQ_PAIRS_SET command specifying the
  number of the transmit and receive queues to use.

\item If the VIRTIO_NET_F_MAC feature bit is set, the configuration
  space \field{mac} entry indicates the ``physical'' address of the
  device, otherwise the driver would typically generate a random
  local MAC address.

\item If the VIRTIO_NET_F_STATUS feature bit is negotiated, the link
  status comes from the bottom bit of \field{status}.
  Otherwise, the driver assumes it's active.

\item A performant driver would indicate that it will generate checksumless
  packets by negotiating the VIRTIO_NET_F_CSUM feature.

\item If that feature is negotiated, a driver can use TCP segmentation or UDP
  segmentation/fragmentation offload by negotiating the VIRTIO_NET_F_HOST_TSO4 (IPv4
  TCP), VIRTIO_NET_F_HOST_TSO6 (IPv6 TCP), VIRTIO_NET_F_HOST_UFO
  (UDP fragmentation) and VIRTIO_NET_F_HOST_USO (UDP segmentation) features.

\item If the VIRTIO_NET_F_HOST_TSO6, VIRTIO_NET_F_HOST_TSO4 and VIRTIO_NET_F_HOST_USO
  segmentation features are negotiated, a driver can
  use TCP segmentation or UDP segmentation on top of UDP encapsulation
  offload, when the outer header does not require checksumming - e.g.
  the outer UDP checksum is zero - by negotiating the
  VIRTIO_NET_F_HOST_UDP_TUNNEL_GSO feature.
  GSO over UDP tunnels packets carry two sets of headers: the outer ones
  and the inner ones. The outer transport protocol is UDP, the inner
  could be either TCP or UDP. Only a single level of encapsulation
  offload is supported.

\item If VIRTIO_NET_F_HOST_UDP_TUNNEL_GSO is negotiated, a driver can
  additionally use TCP segmentation or UDP segmentation on top of UDP
  encapsulation with the outer header requiring checksum offload,
  negotiating the VIRTIO_NET_F_HOST_UDP_TUNNEL_GSO_CSUM feature.

\item The converse features are also available: a driver can save
  the virtual device some work by negotiating these features.\note{For example, a network packet transported between two guests on
the same system might not need checksumming at all, nor segmentation,
if both guests are amenable.}
   The VIRTIO_NET_F_GUEST_CSUM feature indicates that partially
  checksummed packets can be received, and if it can do that then
  the VIRTIO_NET_F_GUEST_TSO4, VIRTIO_NET_F_GUEST_TSO6,
  VIRTIO_NET_F_GUEST_UFO, VIRTIO_NET_F_GUEST_ECN, VIRTIO_NET_F_GUEST_USO4,
  VIRTIO_NET_F_GUEST_USO6 VIRTIO_NET_F_GUEST_UDP_TUNNEL_GSO and
  VIRTIO_NET_F_GUEST_UDP_TUNNEL_GSO_CSUM are the input equivalents of
  the features described above.
  See \ref{sec:Device Types / Network Device / Device Operation /
Setting Up Receive Buffers}~\nameref{sec:Device Types / Network
Device / Device Operation / Setting Up Receive Buffers} and
\ref{sec:Device Types / Network Device / Device Operation /
Processing of Incoming Packets}~\nameref{sec:Device Types /
Network Device / Device Operation / Processing of Incoming Packets} below.
\end{enumerate}

A truly minimal driver would only accept VIRTIO_NET_F_MAC and ignore
everything else.

\subsection{Device and driver capabilities}\label{sec:Device Types / Network Device / Device and driver capabilities}

The network device has the following capabilities.

\begin{tabularx}{\textwidth}{ |l||l|X| }
\hline
Identifier & Name & Description \\
\hline \hline
0x0800 & \hyperref[par:Device Types / Network Device / Device Operation / Flow filter / Device and driver capabilities / VIRTIO-NET-FF-RESOURCE-CAP]{VIRTIO_NET_FF_RESOURCE_CAP} & Flow filter resource capability \\
\hline
0x0801 & \hyperref[par:Device Types / Network Device / Device Operation / Flow filter / Device and driver capabilities / VIRTIO-NET-FF-SELECTOR-CAP]{VIRTIO_NET_FF_SELECTOR_CAP} & Flow filter classifier capability \\
\hline
0x0802 & \hyperref[par:Device Types / Network Device / Device Operation / Flow filter / Device and driver capabilities / VIRTIO-NET-FF-ACTION-CAP]{VIRTIO_NET_FF_ACTION_CAP} & Flow filter action capability \\
\hline
\end{tabularx}

\subsection{Device resource objects}\label{sec:Device Types / Network Device / Device resource objects}

The network device has the following resource objects.

\begin{tabularx}{\textwidth}{ |l||l|X| }
\hline
type & Name & Description \\
\hline \hline
0x0200 & \hyperref[par:Device Types / Network Device / Device Operation / Flow filter / Resource objects / VIRTIO-NET-RESOURCE-OBJ-FF-GROUP]{VIRTIO_NET_RESOURCE_OBJ_FF_GROUP} & Flow filter group resource object \\
\hline
0x0201 & \hyperref[par:Device Types / Network Device / Device Operation / Flow filter / Resource objects / VIRTIO-NET-RESOURCE-OBJ-FF-CLASSIFIER]{VIRTIO_NET_RESOURCE_OBJ_FF_CLASSIFIER} & Flow filter mask object \\
\hline
0x0202 & \hyperref[par:Device Types / Network Device / Device Operation / Flow filter / Resource objects / VIRTIO-NET-RESOURCE-OBJ-FF-RULE]{VIRTIO_NET_RESOURCE_OBJ_FF_RULE} & Flow filter rule object \\
\hline
\end{tabularx}

\subsection{Device parts}\label{sec:Device Types / Network Device / Device parts}

Network device parts represent the configuration done by the driver using control
virtqueue commands. Network device part is in the format of
\field{struct virtio_dev_part}.

\begin{tabularx}{\textwidth}{ |l||l|X| }
\hline
Type & Name & Description \\
\hline \hline
0x200 & VIRTIO_NET_DEV_PART_CVQ_CFG_PART & Represents device configuration done through a control virtqueue command, see \ref{sec:Device Types / Network Device / Device parts / VIRTIO-NET-DEV-PART-CVQ-CFG-PART} \\
\hline
0x201 - 0x5FF & - & reserved for future \\
\hline
\hline
\end{tabularx}

\subsubsection{VIRTIO_NET_DEV_PART_CVQ_CFG_PART}\label{sec:Device Types / Network Device / Device parts / VIRTIO-NET-DEV-PART-CVQ-CFG-PART}

For VIRTIO_NET_DEV_PART_CVQ_CFG_PART, \field{part_type} is set to 0x200. The
VIRTIO_NET_DEV_PART_CVQ_CFG_PART part indicates configuration performed by the
driver using a control virtqueue command.

\begin{lstlisting}
struct virtio_net_dev_part_cvq_selector {
        u8 class;
        u8 command;
        u8 reserved[6];
};
\end{lstlisting}

There is one device part of type VIRTIO_NET_DEV_PART_CVQ_CFG_PART for each
individual configuration. Each part is identified by a unique selector value.
The selector, \field{device_type_raw}, is in the format
\field{struct virtio_net_dev_part_cvq_selector}.

The selector consists of two fields: \field{class} and \field{command}. These
fields correspond to the \field{class} and \field{command} defined in
\field{struct virtio_net_ctrl}, as described in the relevant sections of
\ref{sec:Device Types / Network Device / Device Operation / Control Virtqueue}.

The value corresponding to each part’s selector follows the same format as the
respective \field{command-specific-data} described in the relevant sections of
\ref{sec:Device Types / Network Device / Device Operation / Control Virtqueue}.

For example, when the \field{class} is VIRTIO_NET_CTRL_MAC, the \field{command}
can be either VIRTIO_NET_CTRL_MAC_TABLE_SET or VIRTIO_NET_CTRL_MAC_ADDR_SET;
when \field{command} is set to VIRTIO_NET_CTRL_MAC_TABLE_SET, \field{value}
is in the format of \field{struct virtio_net_ctrl_mac}.

Supported selectors are listed in the table:

\begin{tabularx}{\textwidth}{ |l|X| }
\hline
Class selector & Command selector \\
\hline \hline
VIRTIO_NET_CTRL_RX & VIRTIO_NET_CTRL_RX_PROMISC \\
\hline
VIRTIO_NET_CTRL_RX & VIRTIO_NET_CTRL_RX_ALLMULTI \\
\hline
VIRTIO_NET_CTRL_RX & VIRTIO_NET_CTRL_RX_ALLUNI \\
\hline
VIRTIO_NET_CTRL_RX & VIRTIO_NET_CTRL_RX_NOMULTI \\
\hline
VIRTIO_NET_CTRL_RX & VIRTIO_NET_CTRL_RX_NOUNI \\
\hline
VIRTIO_NET_CTRL_RX & VIRTIO_NET_CTRL_RX_NOBCAST \\
\hline
VIRTIO_NET_CTRL_MAC & VIRTIO_NET_CTRL_MAC_TABLE_SET \\
\hline
VIRTIO_NET_CTRL_MAC & VIRTIO_NET_CTRL_MAC_ADDR_SET \\
\hline
VIRTIO_NET_CTRL_VLAN & VIRTIO_NET_CTRL_VLAN_ADD \\
\hline
VIRTIO_NET_CTRL_ANNOUNCE & VIRTIO_NET_CTRL_ANNOUNCE_ACK \\
\hline
VIRTIO_NET_CTRL_MQ & VIRTIO_NET_CTRL_MQ_VQ_PAIRS_SET \\
\hline
VIRTIO_NET_CTRL_MQ & VIRTIO_NET_CTRL_MQ_RSS_CONFIG \\
\hline
VIRTIO_NET_CTRL_MQ & VIRTIO_NET_CTRL_MQ_HASH_CONFIG \\
\hline
\hline
\end{tabularx}

For command selector VIRTIO_NET_CTRL_VLAN_ADD, device part consists of a whole
VLAN table.

\field{reserved} is reserved and set to zero.

\subsection{Device Operation}\label{sec:Device Types / Network Device / Device Operation}

Packets are transmitted by placing them in the
transmitq1\ldots transmitqN, and buffers for incoming packets are
placed in the receiveq1\ldots receiveqN. In each case, the packet
itself is preceded by a header:

\begin{lstlisting}
struct virtio_net_hdr {
#define VIRTIO_NET_HDR_F_NEEDS_CSUM    1
#define VIRTIO_NET_HDR_F_DATA_VALID    2
#define VIRTIO_NET_HDR_F_RSC_INFO      4
#define VIRTIO_NET_HDR_F_UDP_TUNNEL_CSUM 8
        u8 flags;
#define VIRTIO_NET_HDR_GSO_NONE        0
#define VIRTIO_NET_HDR_GSO_TCPV4       1
#define VIRTIO_NET_HDR_GSO_UDP         3
#define VIRTIO_NET_HDR_GSO_TCPV6       4
#define VIRTIO_NET_HDR_GSO_UDP_L4      5
#define VIRTIO_NET_HDR_GSO_UDP_TUNNEL_IPV4 0x20
#define VIRTIO_NET_HDR_GSO_UDP_TUNNEL_IPV6 0x40
#define VIRTIO_NET_HDR_GSO_ECN      0x80
        u8 gso_type;
        le16 hdr_len;
        le16 gso_size;
        le16 csum_start;
        le16 csum_offset;
        le16 num_buffers;
        le32 hash_value;        (Only if VIRTIO_NET_F_HASH_REPORT negotiated)
        le16 hash_report;       (Only if VIRTIO_NET_F_HASH_REPORT negotiated)
        le16 padding_reserved;  (Only if VIRTIO_NET_F_HASH_REPORT negotiated)
        le16 outer_th_offset    (Only if VIRTIO_NET_F_HOST_UDP_TUNNEL_GSO or VIRTIO_NET_F_GUEST_UDP_TUNNEL_GSO negotiated)
        le16 inner_nh_offset;   (Only if VIRTIO_NET_F_HOST_UDP_TUNNEL_GSO or VIRTIO_NET_F_GUEST_UDP_TUNNEL_GSO negotiated)
};
\end{lstlisting}

The controlq is used to control various device features described further in
section \ref{sec:Device Types / Network Device / Device Operation / Control Virtqueue}.

\subsubsection{Legacy Interface: Device Operation}\label{sec:Device Types / Network Device / Device Operation / Legacy Interface: Device Operation}
When using the legacy interface, transitional devices and drivers
MUST format the fields in \field{struct virtio_net_hdr}
according to the native endian of the guest rather than
(necessarily when not using the legacy interface) little-endian.

The legacy driver only presented \field{num_buffers} in the \field{struct virtio_net_hdr}
when VIRTIO_NET_F_MRG_RXBUF was negotiated; without that feature the
structure was 2 bytes shorter.

When using the legacy interface, the driver SHOULD ignore the
used length for the transmit queues
and the controlq queue.
\begin{note}
Historically, some devices put
the total descriptor length there, even though no data was
actually written.
\end{note}

\subsubsection{Packet Transmission}\label{sec:Device Types / Network Device / Device Operation / Packet Transmission}

Transmitting a single packet is simple, but varies depending on
the different features the driver negotiated.

\begin{enumerate}
\item The driver can send a completely checksummed packet.  In this case,
  \field{flags} will be zero, and \field{gso_type} will be VIRTIO_NET_HDR_GSO_NONE.

\item If the driver negotiated VIRTIO_NET_F_CSUM, it can skip
  checksumming the packet:
  \begin{itemize}
  \item \field{flags} has the VIRTIO_NET_HDR_F_NEEDS_CSUM set,

  \item \field{csum_start} is set to the offset within the packet to begin checksumming,
    and

  \item \field{csum_offset} indicates how many bytes after the csum_start the
    new (16 bit ones' complement) checksum is placed by the device.

  \item The TCP checksum field in the packet is set to the sum
    of the TCP pseudo header, so that replacing it by the ones'
    complement checksum of the TCP header and body will give the
    correct result.
  \end{itemize}

\begin{note}
For example, consider a partially checksummed TCP (IPv4) packet.
It will have a 14 byte ethernet header and 20 byte IP header
followed by the TCP header (with the TCP checksum field 16 bytes
into that header). \field{csum_start} will be 14+20 = 34 (the TCP
checksum includes the header), and \field{csum_offset} will be 16.
If the given packet has the VIRTIO_NET_HDR_GSO_UDP_TUNNEL_IPV4 bit or the
VIRTIO_NET_HDR_GSO_UDP_TUNNEL_IPV6 bit set,
the above checksum fields refer to the inner header checksum, see
the example below.
\end{note}

\item If the driver negotiated
  VIRTIO_NET_F_HOST_TSO4, TSO6, USO or UFO, and the packet requires
  TCP segmentation, UDP segmentation or fragmentation, then \field{gso_type}
  is set to VIRTIO_NET_HDR_GSO_TCPV4, TCPV6, UDP_L4 or UDP.
  (Otherwise, it is set to VIRTIO_NET_HDR_GSO_NONE). In this
  case, packets larger than 1514 bytes can be transmitted: the
  metadata indicates how to replicate the packet header to cut it
  into smaller packets. The other gso fields are set:

  \begin{itemize}
  \item If the VIRTIO_NET_F_GUEST_HDRLEN feature has been negotiated,
    \field{hdr_len} indicates the header length that needs to be replicated
    for each packet. It's the number of bytes from the beginning of the packet
    to the beginning of the transport payload.
    If the \field{gso_type} has the VIRTIO_NET_HDR_GSO_UDP_TUNNEL_IPV4 bit or
    VIRTIO_NET_HDR_GSO_UDP_TUNNEL_IPV6 bit set, \field{hdr_len} accounts for
    all the headers up to and including the inner transport.
    Otherwise, if the VIRTIO_NET_F_GUEST_HDRLEN feature has not been negotiated,
    \field{hdr_len} is a hint to the device as to how much of the header
    needs to be kept to copy into each packet, usually set to the
    length of the headers, including the transport header\footnote{Due to various bugs in implementations, this field is not useful
as a guarantee of the transport header size.
}.

  \begin{note}
  Some devices benefit from knowledge of the exact header length.
  \end{note}

  \item \field{gso_size} is the maximum size of each packet beyond that
    header (ie. MSS).

  \item If the driver negotiated the VIRTIO_NET_F_HOST_ECN feature,
    the VIRTIO_NET_HDR_GSO_ECN bit in \field{gso_type}
    indicates that the TCP packet has the ECN bit set\footnote{This case is not handled by some older hardware, so is called out
specifically in the protocol.}.
   \end{itemize}

\item If the driver negotiated the VIRTIO_NET_F_HOST_UDP_TUNNEL_GSO feature and the
  \field{gso_type} has the VIRTIO_NET_HDR_GSO_UDP_TUNNEL_IPV4 bit or
  VIRTIO_NET_HDR_GSO_UDP_TUNNEL_IPV6 bit set, the GSO protocol is encapsulated
  in a UDP tunnel.
  If the outer UDP header requires checksumming, the driver must have
  additionally negotiated the VIRTIO_NET_F_HOST_UDP_TUNNEL_GSO_CSUM feature
  and offloaded the outer checksum accordingly, otherwise
  the outer UDP header must not require checksum validation, i.e. the outer
  UDP checksum must be positive zero (0x0) as defined in UDP RFC 768.
  The other tunnel-related fields indicate how to replicate the packet
  headers to cut it into smaller packets:

  \begin{itemize}
  \item \field{outer_th_offset} field indicates the outer transport header within
      the packet. This field differs from \field{csum_start} as the latter
      points to the inner transport header within the packet.

  \item \field{inner_nh_offset} field indicates the inner network header within
      the packet.
  \end{itemize}

\begin{note}
For example, consider a partially checksummed TCP (IPv4) packet carried over a
Geneve UDP tunnel (again IPv4) with no tunnel options. The
only relevant variable related to the tunnel type is the tunnel header length.
The packet will have a 14 byte outer ethernet header, 20 byte outer IP header
followed by the 8 byte UDP header (with a 0 checksum value), 8 byte Geneve header,
14 byte inner ethernet header, 20 byte inner IP header
and the TCP header (with the TCP checksum field 16 bytes
into that header). \field{csum_start} will be 14+20+8+8+14+20 = 84 (the TCP
checksum includes the header), \field{csum_offset} will be 16.
\field{inner_nh_offset} will be 14+20+8+8+14 = 62, \field{outer_th_offset} will be
14+20+8 = 42 and \field{gso_type} will be
VIRTIO_NET_HDR_GSO_TCPV4 | VIRTIO_NET_HDR_GSO_UDP_TUNNEL_IPV4 = 0x21
\end{note}

\item If the driver negotiated the VIRTIO_NET_F_HOST_UDP_TUNNEL_GSO_CSUM feature,
  the transmitted packet is a GSO one encapsulated in a UDP tunnel, and
  the outer UDP header requires checksumming, the driver can skip checksumming
  the outer header:

  \begin{itemize}
  \item \field{flags} has the VIRTIO_NET_HDR_F_UDP_TUNNEL_CSUM set,

  \item The outer UDP checksum field in the packet is set to the sum
    of the UDP pseudo header, so that replacing it by the ones'
    complement checksum of the outer UDP header and payload will give the
    correct result.
  \end{itemize}

\item \field{num_buffers} is set to zero.  This field is unused on transmitted packets.

\item The header and packet are added as one output descriptor to the
  transmitq, and the device is notified of the new entry
  (see \ref{sec:Device Types / Network Device / Device Initialization}~\nameref{sec:Device Types / Network Device / Device Initialization}).
\end{enumerate}

\drivernormative{\paragraph}{Packet Transmission}{Device Types / Network Device / Device Operation / Packet Transmission}

For the transmit packet buffer, the driver MUST use the size of the
structure \field{struct virtio_net_hdr} same as the receive packet buffer.

The driver MUST set \field{num_buffers} to zero.

If VIRTIO_NET_F_CSUM is not negotiated, the driver MUST set
\field{flags} to zero and SHOULD supply a fully checksummed
packet to the device.

If VIRTIO_NET_F_HOST_TSO4 is negotiated, the driver MAY set
\field{gso_type} to VIRTIO_NET_HDR_GSO_TCPV4 to request TCPv4
segmentation, otherwise the driver MUST NOT set
\field{gso_type} to VIRTIO_NET_HDR_GSO_TCPV4.

If VIRTIO_NET_F_HOST_TSO6 is negotiated, the driver MAY set
\field{gso_type} to VIRTIO_NET_HDR_GSO_TCPV6 to request TCPv6
segmentation, otherwise the driver MUST NOT set
\field{gso_type} to VIRTIO_NET_HDR_GSO_TCPV6.

If VIRTIO_NET_F_HOST_UFO is negotiated, the driver MAY set
\field{gso_type} to VIRTIO_NET_HDR_GSO_UDP to request UDP
fragmentation, otherwise the driver MUST NOT set
\field{gso_type} to VIRTIO_NET_HDR_GSO_UDP.

If VIRTIO_NET_F_HOST_USO is negotiated, the driver MAY set
\field{gso_type} to VIRTIO_NET_HDR_GSO_UDP_L4 to request UDP
segmentation, otherwise the driver MUST NOT set
\field{gso_type} to VIRTIO_NET_HDR_GSO_UDP_L4.

The driver SHOULD NOT send to the device TCP packets requiring segmentation offload
which have the Explicit Congestion Notification bit set, unless the
VIRTIO_NET_F_HOST_ECN feature is negotiated, in which case the
driver MUST set the VIRTIO_NET_HDR_GSO_ECN bit in
\field{gso_type}.

If VIRTIO_NET_F_HOST_UDP_TUNNEL_GSO is negotiated, the driver MAY set
VIRTIO_NET_HDR_GSO_UDP_TUNNEL_IPV4 bit or the VIRTIO_NET_HDR_GSO_UDP_TUNNEL_IPV6 bit
in \field{gso_type} according to the inner network header protocol type
to request GSO packets over UDPv4 or UDPv6 tunnel segmentation,
otherwise the driver MUST NOT set either the
VIRTIO_NET_HDR_GSO_UDP_TUNNEL_IPV4 bit or the VIRTIO_NET_HDR_GSO_UDP_TUNNEL_IPV6 bit
in \field{gso_type}.

When requesting GSO segmentation over UDP tunnel, the driver MUST SET the
VIRTIO_NET_HDR_GSO_UDP_TUNNEL_IPV4 bit if the inner network header is IPv4, i.e. the
packet is a TCPv4 GSO one, otherwise, if the inner network header is IPv6, the driver
MUST SET the VIRTIO_NET_HDR_GSO_UDP_TUNNEL_IPV6 bit.

The driver MUST NOT send to the device GSO packets over UDP tunnel
requiring segmentation and outer UDP checksum offload, unless both the
VIRTIO_NET_F_HOST_UDP_TUNNEL_GSO and VIRTIO_NET_F_HOST_UDP_TUNNEL_GSO_CSUM features
are negotiated, in which case the driver MUST set either the
VIRTIO_NET_HDR_GSO_UDP_TUNNEL_IPV4 bit or the VIRTIO_NET_HDR_GSO_UDP_TUNNEL_IPV6
bit in the \field{gso_type} and the VIRTIO_NET_HDR_F_UDP_TUNNEL_CSUM bit in
the \field{flags}.

If VIRTIO_NET_F_HOST_UDP_TUNNEL_GSO_CSUM is not negotiated, the driver MUST not set
the VIRTIO_NET_HDR_F_UDP_TUNNEL_CSUM bit in the \field{flags} and
MUST NOT send to the device GSO packets over UDP tunnel
requiring segmentation and outer UDP checksum offload.

The driver MUST NOT set the VIRTIO_NET_HDR_GSO_UDP_TUNNEL_IPV4 bit or the
VIRTIO_NET_HDR_GSO_UDP_TUNNEL_IPV6 bit together with VIRTIO_NET_HDR_GSO_UDP, as the
latter is deprecated in favor of UDP_L4 and no new feature will support it.

The driver MUST NOT set the VIRTIO_NET_HDR_GSO_UDP_TUNNEL_IPV4 bit and the
VIRTIO_NET_HDR_GSO_UDP_TUNNEL_IPV6 bit together.

The driver MUST NOT set the VIRTIO_NET_HDR_F_UDP_TUNNEL_CSUM bit \field{flags}
without setting either the VIRTIO_NET_HDR_GSO_UDP_TUNNEL_IPV4 bit or
the VIRTIO_NET_HDR_GSO_UDP_TUNNEL_IPV6 bit in \field{gso_type}.

If the VIRTIO_NET_F_CSUM feature has been negotiated, the
driver MAY set the VIRTIO_NET_HDR_F_NEEDS_CSUM bit in
\field{flags}, if so:
\begin{enumerate}
\item the driver MUST validate the packet checksum at
	offset \field{csum_offset} from \field{csum_start} as well as all
	preceding offsets;
\begin{note}
If \field{gso_type} differs from VIRTIO_NET_HDR_GSO_NONE and the
VIRTIO_NET_HDR_GSO_UDP_TUNNEL_IPV4 bit or the VIRTIO_NET_HDR_GSO_UDP_TUNNEL_IPV6
bit are not set in \field{gso_type}, \field{csum_offset}
points to the only transport header present in the packet, and there are no
additional preceding checksums validated by VIRTIO_NET_HDR_F_NEEDS_CSUM.
\end{note}
\item the driver MUST set the packet checksum stored in the
	buffer to the TCP/UDP pseudo header;
\item the driver MUST set \field{csum_start} and
	\field{csum_offset} such that calculating a ones'
	complement checksum from \field{csum_start} up until the end of
	the packet and storing the result at offset \field{csum_offset}
	from  \field{csum_start} will result in a fully checksummed
	packet;
\end{enumerate}

If none of the VIRTIO_NET_F_HOST_TSO4, TSO6, USO or UFO options have
been negotiated, the driver MUST set \field{gso_type} to
VIRTIO_NET_HDR_GSO_NONE.

If \field{gso_type} differs from VIRTIO_NET_HDR_GSO_NONE, then
the driver MUST also set the VIRTIO_NET_HDR_F_NEEDS_CSUM bit in
\field{flags} and MUST set \field{gso_size} to indicate the
desired MSS.

If one of the VIRTIO_NET_F_HOST_TSO4, TSO6, USO or UFO options have
been negotiated:
\begin{itemize}
\item If the VIRTIO_NET_F_GUEST_HDRLEN feature has been negotiated,
	and \field{gso_type} differs from VIRTIO_NET_HDR_GSO_NONE,
	the driver MUST set \field{hdr_len} to a value equal to the length
	of the headers, including the transport header. If \field{gso_type}
	has the VIRTIO_NET_HDR_GSO_UDP_TUNNEL_IPV4 bit or the
	VIRTIO_NET_HDR_GSO_UDP_TUNNEL_IPV6 bit set, \field{hdr_len} includes
	the inner transport header.

\item If the VIRTIO_NET_F_GUEST_HDRLEN feature has not been negotiated,
	or \field{gso_type} is VIRTIO_NET_HDR_GSO_NONE,
	the driver SHOULD set \field{hdr_len} to a value
	not less than the length of the headers, including the transport
	header.
\end{itemize}

If the VIRTIO_NET_F_HOST_UDP_TUNNEL_GSO option has been negotiated, the
driver MAY set the VIRTIO_NET_HDR_GSO_UDP_TUNNEL_IPV4 bit or the
VIRTIO_NET_HDR_GSO_UDP_TUNNEL_IPV6 bit in \field{gso_type}, if so:
\begin{itemize}
\item the driver MUST set \field{outer_th_offset} to the outer UDP header
  offset and \field{inner_nh_offset} to the inner network header offset.
  The \field{csum_start} and \field{csum_offset} fields point respectively
  to the inner transport header and inner transport checksum field.
\end{itemize}

If the VIRTIO_NET_F_HOST_UDP_TUNNEL_GSO_CSUM feature has been negotiated,
and the VIRTIO_NET_HDR_GSO_UDP_TUNNEL_IPV4 bit or
VIRTIO_NET_HDR_GSO_UDP_TUNNEL_IPV6 bit in \field{gso_type} are set,
the driver MAY set the VIRTIO_NET_HDR_F_UDP_TUNNEL_CSUM bit in
\field{flags}, if so the driver MUST set the packet outer UDP header checksum
to the outer UDP pseudo header checksum.

\begin{note}
calculating a ones' complement checksum from \field{outer_th_offset}
up until the end of the packet and storing the result at offset 6
from \field{outer_th_offset} will result in a fully checksummed outer UDP packet;
\end{note}

If the VIRTIO_NET_HDR_GSO_UDP_TUNNEL_IPV4 bit or the
VIRTIO_NET_HDR_GSO_UDP_TUNNEL_IPV6 bit in \field{gso_type} are set
and the VIRTIO_NET_F_HOST_UDP_TUNNEL_GSO_CSUM feature has not
been negotiated, the
outer UDP header MUST NOT require checksum validation. That is, the
outer UDP checksum value MUST be 0 or the validated complete checksum
for such header.

\begin{note}
The valid complete checksum of the outer UDP header of individual segments
can be computed by the driver prior to segmentation only if the GSO packet
size is a multiple of \field{gso_size}, because then all segments
have the same size and thus all data included in the outer UDP
checksum is the same for every segment. These pre-computed segment
length and checksum fields are different from those of the GSO
packet.
In this scenario the outer UDP header of the GSO packet must carry the
segmented UDP packet length.
\end{note}

If the VIRTIO_NET_F_HOST_UDP_TUNNEL_GSO option has not
been negotiated, the driver MUST NOT set either the VIRTIO_NET_HDR_F_GSO_UDP_TUNNEL_IPV4
bit or the VIRTIO_NET_HDR_F_GSO_UDP_TUNNEL_IPV6 in \field{gso_type}.

If the VIRTIO_NET_F_HOST_UDP_TUNNEL_GSO_CSUM option has not been negotiated,
the driver MUST NOT set the VIRTIO_NET_HDR_F_UDP_TUNNEL_CSUM bit
in \field{flags}.

The driver SHOULD accept the VIRTIO_NET_F_GUEST_HDRLEN feature if it has
been offered, and if it's able to provide the exact header length.

The driver MUST NOT set the VIRTIO_NET_HDR_F_DATA_VALID and
VIRTIO_NET_HDR_F_RSC_INFO bits in \field{flags}.

The driver MUST NOT set the VIRTIO_NET_HDR_F_DATA_VALID bit in \field{flags}
together with the VIRTIO_NET_HDR_F_GSO_UDP_TUNNEL_IPV4 bit or the
VIRTIO_NET_HDR_F_GSO_UDP_TUNNEL_IPV6 bit in \field{gso_type}.

\devicenormative{\paragraph}{Packet Transmission}{Device Types / Network Device / Device Operation / Packet Transmission}
The device MUST ignore \field{flag} bits that it does not recognize.

If VIRTIO_NET_HDR_F_NEEDS_CSUM bit in \field{flags} is not set, the
device MUST NOT use the \field{csum_start} and \field{csum_offset}.

If one of the VIRTIO_NET_F_HOST_TSO4, TSO6, USO or UFO options have
been negotiated:
\begin{itemize}
\item If the VIRTIO_NET_F_GUEST_HDRLEN feature has been negotiated,
	and \field{gso_type} differs from VIRTIO_NET_HDR_GSO_NONE,
	the device MAY use \field{hdr_len} as the transport header size.

	\begin{note}
	Caution should be taken by the implementation so as to prevent
	a malicious driver from attacking the device by setting an incorrect hdr_len.
	\end{note}

\item If the VIRTIO_NET_F_GUEST_HDRLEN feature has not been negotiated,
	or \field{gso_type} is VIRTIO_NET_HDR_GSO_NONE,
	the device MAY use \field{hdr_len} only as a hint about the
	transport header size.
	The device MUST NOT rely on \field{hdr_len} to be correct.

	\begin{note}
	This is due to various bugs in implementations.
	\end{note}
\end{itemize}

If both the VIRTIO_NET_HDR_GSO_UDP_TUNNEL_IPV4 bit and
the VIRTIO_NET_HDR_GSO_UDP_TUNNEL_IPV6 bit in in \field{gso_type} are set,
the device MUST NOT accept the packet.

If the VIRTIO_NET_HDR_GSO_UDP_TUNNEL_IPV4 bit and the VIRTIO_NET_HDR_GSO_UDP_TUNNEL_IPV6
bit in \field{gso_type} are not set, the device MUST NOT use the
\field{outer_th_offset} and \field{inner_nh_offset}.

If either the VIRTIO_NET_HDR_GSO_UDP_TUNNEL_IPV4 bit or
the VIRTIO_NET_HDR_GSO_UDP_TUNNEL_IPV6 bit in \field{gso_type} are set, and any of
the following is true:
\begin{itemize}
\item the VIRTIO_NET_HDR_F_NEEDS_CSUM is not set in \field{flags}
\item the VIRTIO_NET_HDR_F_DATA_VALID is set in \field{flags}
\item the \field{gso_type} excluding the VIRTIO_NET_HDR_GSO_UDP_TUNNEL_IPV4
bit and the VIRTIO_NET_HDR_GSO_UDP_TUNNEL_IPV6 bit is VIRTIO_NET_HDR_GSO_NONE
\end{itemize}
the device MUST NOT accept the packet.

If the VIRTIO_NET_HDR_F_UDP_TUNNEL_CSUM bit in \field{flags} is set,
and both the bits VIRTIO_NET_HDR_GSO_UDP_TUNNEL_IPV4 and
VIRTIO_NET_HDR_GSO_UDP_TUNNEL_IPV6 in \field{gso_type} are not set,
the device MOST NOT accept the packet.

If VIRTIO_NET_HDR_F_NEEDS_CSUM is not set, the device MUST NOT
rely on the packet checksum being correct.
\paragraph{Packet Transmission Interrupt}\label{sec:Device Types / Network Device / Device Operation / Packet Transmission / Packet Transmission Interrupt}

Often a driver will suppress transmission virtqueue interrupts
and check for used packets in the transmit path of following
packets.

The normal behavior in this interrupt handler is to retrieve
used buffers from the virtqueue and free the corresponding
headers and packets.

\subsubsection{Setting Up Receive Buffers}\label{sec:Device Types / Network Device / Device Operation / Setting Up Receive Buffers}

It is generally a good idea to keep the receive virtqueue as
fully populated as possible: if it runs out, network performance
will suffer.

If the VIRTIO_NET_F_GUEST_TSO4, VIRTIO_NET_F_GUEST_TSO6,
VIRTIO_NET_F_GUEST_UFO, VIRTIO_NET_F_GUEST_USO4 or VIRTIO_NET_F_GUEST_USO6
features are used, the maximum incoming packet
will be 65589 bytes long (14 bytes of Ethernet header, plus 40 bytes of
the IPv6 header, plus 65535 bytes of maximum IPv6 payload including any
extension header), otherwise 1514 bytes.
When VIRTIO_NET_F_HASH_REPORT is not negotiated, the required receive buffer
size is either 65601 or 1526 bytes accounting for 20 bytes of
\field{struct virtio_net_hdr} followed by receive packet.
When VIRTIO_NET_F_HASH_REPORT is negotiated, the required receive buffer
size is either 65609 or 1534 bytes accounting for 12 bytes of
\field{struct virtio_net_hdr} followed by receive packet.

\drivernormative{\paragraph}{Setting Up Receive Buffers}{Device Types / Network Device / Device Operation / Setting Up Receive Buffers}

\begin{itemize}
\item If VIRTIO_NET_F_MRG_RXBUF is not negotiated:
  \begin{itemize}
    \item If VIRTIO_NET_F_GUEST_TSO4, VIRTIO_NET_F_GUEST_TSO6, VIRTIO_NET_F_GUEST_UFO,
	VIRTIO_NET_F_GUEST_USO4 or VIRTIO_NET_F_GUEST_USO6 are negotiated, the driver SHOULD populate
      the receive queue(s) with buffers of at least 65609 bytes if
      VIRTIO_NET_F_HASH_REPORT is negotiated, and of at least 65601 bytes if not.
    \item Otherwise, the driver SHOULD populate the receive queue(s)
      with buffers of at least 1534 bytes if VIRTIO_NET_F_HASH_REPORT
      is negotiated, and of at least 1526 bytes if not.
  \end{itemize}
\item If VIRTIO_NET_F_MRG_RXBUF is negotiated, each buffer MUST be at
least size of \field{struct virtio_net_hdr},
i.e. 20 bytes if VIRTIO_NET_F_HASH_REPORT is negotiated, and 12 bytes if not.
\end{itemize}

\begin{note}
Obviously each buffer can be split across multiple descriptor elements.
\end{note}

When calculating the size of \field{struct virtio_net_hdr}, the driver
MUST consider all the fields inclusive up to \field{padding_reserved},
i.e. 20 bytes if VIRTIO_NET_F_HASH_REPORT is negotiated, and 12 bytes if not.

If VIRTIO_NET_F_MQ is negotiated, each of receiveq1\ldots receiveqN
that will be used SHOULD be populated with receive buffers.

\devicenormative{\paragraph}{Setting Up Receive Buffers}{Device Types / Network Device / Device Operation / Setting Up Receive Buffers}

The device MUST set \field{num_buffers} to the number of descriptors used to
hold the incoming packet.

The device MUST use only a single descriptor if VIRTIO_NET_F_MRG_RXBUF
was not negotiated.
\begin{note}
{This means that \field{num_buffers} will always be 1
if VIRTIO_NET_F_MRG_RXBUF is not negotiated.}
\end{note}

\subsubsection{Processing of Incoming Packets}\label{sec:Device Types / Network Device / Device Operation / Processing of Incoming Packets}
\label{sec:Device Types / Network Device / Device Operation / Processing of Packets}%old label for latexdiff

When a packet is copied into a buffer in the receiveq, the
optimal path is to disable further used buffer notifications for the
receiveq and process packets until no more are found, then re-enable
them.

Processing incoming packets involves:

\begin{enumerate}
\item \field{num_buffers} indicates how many descriptors
  this packet is spread over (including this one): this will
  always be 1 if VIRTIO_NET_F_MRG_RXBUF was not negotiated.
  This allows receipt of large packets without having to allocate large
  buffers: a packet that does not fit in a single buffer can flow
  over to the next buffer, and so on. In this case, there will be
  at least \field{num_buffers} used buffers in the virtqueue, and the device
  chains them together to form a single packet in a way similar to
  how it would store it in a single buffer spread over multiple
  descriptors.
  The other buffers will not begin with a \field{struct virtio_net_hdr}.

\item If
  \field{num_buffers} is one, then the entire packet will be
  contained within this buffer, immediately following the struct
  virtio_net_hdr.
\item If the VIRTIO_NET_F_GUEST_CSUM feature was negotiated, the
  VIRTIO_NET_HDR_F_DATA_VALID bit in \field{flags} can be
  set: if so, device has validated the packet checksum.
  If the VIRTIO_NET_F_GUEST_UDP_TUNNEL_GSO_CSUM feature has been negotiated,
  and the VIRTIO_NET_HDR_F_UDP_TUNNEL_CSUM bit is set in \field{flags},
  both the outer UDP checksum and the inner transport checksum
  have been validated, otherwise only one level of checksums (the outer one
  in case of tunnels) has been validated.
\end{enumerate}

Additionally, VIRTIO_NET_F_GUEST_CSUM, TSO4, TSO6, UDP, UDP_TUNNEL
and ECN features enable receive checksum, large receive offload and ECN
support which are the input equivalents of the transmit checksum,
transmit segmentation offloading and ECN features, as described
in \ref{sec:Device Types / Network Device / Device Operation /
Packet Transmission}:
\begin{enumerate}
\item If the VIRTIO_NET_F_GUEST_TSO4, TSO6, UFO, USO4 or USO6 options were
  negotiated, then \field{gso_type} MAY be something other than
  VIRTIO_NET_HDR_GSO_NONE, and \field{gso_size} field indicates the
  desired MSS (see Packet Transmission point 2).
\item If the VIRTIO_NET_F_RSC_EXT option was negotiated (this
  implies one of VIRTIO_NET_F_GUEST_TSO4, TSO6), the
  device processes also duplicated ACK segments, reports
  number of coalesced TCP segments in \field{csum_start} field and
  number of duplicated ACK segments in \field{csum_offset} field
  and sets bit VIRTIO_NET_HDR_F_RSC_INFO in \field{flags}.
\item If the VIRTIO_NET_F_GUEST_CSUM feature was negotiated, the
  VIRTIO_NET_HDR_F_NEEDS_CSUM bit in \field{flags} can be
  set: if so, the packet checksum at offset \field{csum_offset}
  from \field{csum_start} and any preceding checksums
  have been validated.  The checksum on the packet is incomplete and
  if bit VIRTIO_NET_HDR_F_RSC_INFO is not set in \field{flags},
  then \field{csum_start} and \field{csum_offset} indicate how to calculate it
  (see Packet Transmission point 1).
\begin{note}
If \field{gso_type} differs from VIRTIO_NET_HDR_GSO_NONE and the
VIRTIO_NET_HDR_GSO_UDP_TUNNEL_IPV4 bit or the VIRTIO_NET_HDR_GSO_UDP_TUNNEL_IPV6
bit are not set, \field{csum_offset}
points to the only transport header present in the packet, and there are no
additional preceding checksums validated by VIRTIO_NET_HDR_F_NEEDS_CSUM.
\end{note}
\item If the VIRTIO_NET_F_GUEST_UDP_TUNNEL_GSO option was negotiated and
  \field{gso_type} is not VIRTIO_NET_HDR_GSO_NONE, the
  VIRTIO_NET_HDR_GSO_UDP_TUNNEL_IPV4 bit or the VIRTIO_NET_HDR_GSO_UDP_TUNNEL_IPV6
  bit MAY be set. In such case the \field{outer_th_offset} and
  \field{inner_nh_offset} fields indicate the corresponding
  headers information.
\item If the VIRTIO_NET_F_GUEST_UDP_TUNNEL_GSO_CSUM feature was
negotiated, and
  the VIRTIO_NET_HDR_GSO_UDP_TUNNEL_IPV4 bit or the VIRTIO_NET_HDR_GSO_UDP_TUNNEL_IPV6
  are set in \field{gso_type}, the VIRTIO_NET_HDR_F_UDP_TUNNEL_CSUM bit in the
  \field{flags} can be set: if so, the outer UDP checksum has been validated
  and the UDP header checksum at offset 6 from from \field{outer_th_offset}
  is set to the outer UDP pseudo header checksum.

\begin{note}
If the VIRTIO_NET_HDR_GSO_UDP_TUNNEL_IPV4 bit or VIRTIO_NET_HDR_GSO_UDP_TUNNEL_IPV6
bit are set in \field{gso_type}, the \field{csum_start} field refers to
the inner transport header offset (see Packet Transmission point 1).
If the VIRTIO_NET_HDR_F_UDP_TUNNEL_CSUM bit in \field{flags} is set both
the inner and the outer header checksums have been validated by
VIRTIO_NET_HDR_F_NEEDS_CSUM, otherwise only the inner transport header
checksum has been validated.
\end{note}
\end{enumerate}

If applicable, the device calculates per-packet hash for incoming packets as
defined in \ref{sec:Device Types / Network Device / Device Operation / Processing of Incoming Packets / Hash calculation for incoming packets}.

If applicable, the device reports hash information for incoming packets as
defined in \ref{sec:Device Types / Network Device / Device Operation / Processing of Incoming Packets / Hash reporting for incoming packets}.

\devicenormative{\paragraph}{Processing of Incoming Packets}{Device Types / Network Device / Device Operation / Processing of Incoming Packets}
\label{devicenormative:Device Types / Network Device / Device Operation / Processing of Packets}%old label for latexdiff

If VIRTIO_NET_F_MRG_RXBUF has not been negotiated, the device MUST set
\field{num_buffers} to 1.

If VIRTIO_NET_F_MRG_RXBUF has been negotiated, the device MUST set
\field{num_buffers} to indicate the number of buffers
the packet (including the header) is spread over.

If a receive packet is spread over multiple buffers, the device
MUST use all buffers but the last (i.e. the first \field{num_buffers} -
1 buffers) completely up to the full length of each buffer
supplied by the driver.

The device MUST use all buffers used by a single receive
packet together, such that at least \field{num_buffers} are
observed by driver as used.

If VIRTIO_NET_F_GUEST_CSUM is not negotiated, the device MUST set
\field{flags} to zero and SHOULD supply a fully checksummed
packet to the driver.

If VIRTIO_NET_F_GUEST_TSO4 is not negotiated, the device MUST NOT set
\field{gso_type} to VIRTIO_NET_HDR_GSO_TCPV4.

If VIRTIO_NET_F_GUEST_UDP is not negotiated, the device MUST NOT set
\field{gso_type} to VIRTIO_NET_HDR_GSO_UDP.

If VIRTIO_NET_F_GUEST_TSO6 is not negotiated, the device MUST NOT set
\field{gso_type} to VIRTIO_NET_HDR_GSO_TCPV6.

If none of VIRTIO_NET_F_GUEST_USO4 or VIRTIO_NET_F_GUEST_USO6 have been negotiated,
the device MUST NOT set \field{gso_type} to VIRTIO_NET_HDR_GSO_UDP_L4.

If VIRTIO_NET_F_GUEST_UDP_TUNNEL_GSO is not negotiated, the device MUST NOT set
either the VIRTIO_NET_HDR_GSO_UDP_TUNNEL_IPV4 bit or the
VIRTIO_NET_HDR_GSO_UDP_TUNNEL_IPV6 bit in \field{gso_type}.

If VIRTIO_NET_F_GUEST_UDP_TUNNEL_GSO_CSUM is not negotiated the device MUST NOT set
the VIRTIO_NET_HDR_F_UDP_TUNNEL_CSUM bit in \field{flags}.

The device SHOULD NOT send to the driver TCP packets requiring segmentation offload
which have the Explicit Congestion Notification bit set, unless the
VIRTIO_NET_F_GUEST_ECN feature is negotiated, in which case the
device MUST set the VIRTIO_NET_HDR_GSO_ECN bit in
\field{gso_type}.

If the VIRTIO_NET_F_GUEST_CSUM feature has been negotiated, the
device MAY set the VIRTIO_NET_HDR_F_NEEDS_CSUM bit in
\field{flags}, if so:
\begin{enumerate}
\item the device MUST validate the packet checksum at
	offset \field{csum_offset} from \field{csum_start} as well as all
	preceding offsets;
\item the device MUST set the packet checksum stored in the
	receive buffer to the TCP/UDP pseudo header;
\item the device MUST set \field{csum_start} and
	\field{csum_offset} such that calculating a ones'
	complement checksum from \field{csum_start} up until the
	end of the packet and storing the result at offset
	\field{csum_offset} from  \field{csum_start} will result in a
	fully checksummed packet;
\end{enumerate}

The device MUST NOT send to the driver GSO packets encapsulated in UDP
tunnel and requiring segmentation offload, unless the
VIRTIO_NET_F_GUEST_UDP_TUNNEL_GSO is negotiated, in which case the device MUST set
the VIRTIO_NET_HDR_GSO_UDP_TUNNEL_IPV4 bit or the VIRTIO_NET_HDR_GSO_UDP_TUNNEL_IPV6
bit in \field{gso_type} according to the inner network header protocol type,
MUST set the \field{outer_th_offset} and \field{inner_nh_offset} fields
to the corresponding header information, and the outer UDP header MUST NOT
require checksum offload.

If the VIRTIO_NET_F_GUEST_UDP_TUNNEL_GSO_CSUM feature has not been negotiated,
the device MUST NOT send the driver GSO packets encapsulated in UDP
tunnel and requiring segmentation and outer checksum offload.

If none of the VIRTIO_NET_F_GUEST_TSO4, TSO6, UFO, USO4 or USO6 options have
been negotiated, the device MUST set \field{gso_type} to
VIRTIO_NET_HDR_GSO_NONE.

If \field{gso_type} differs from VIRTIO_NET_HDR_GSO_NONE, then
the device MUST also set the VIRTIO_NET_HDR_F_NEEDS_CSUM bit in
\field{flags} MUST set \field{gso_size} to indicate the desired MSS.
If VIRTIO_NET_F_RSC_EXT was negotiated, the device MUST also
set VIRTIO_NET_HDR_F_RSC_INFO bit in \field{flags},
set \field{csum_start} to number of coalesced TCP segments and
set \field{csum_offset} to number of received duplicated ACK segments.

If VIRTIO_NET_F_RSC_EXT was not negotiated, the device MUST
not set VIRTIO_NET_HDR_F_RSC_INFO bit in \field{flags}.

If one of the VIRTIO_NET_F_GUEST_TSO4, TSO6, UFO, USO4 or USO6 options have
been negotiated, the device SHOULD set \field{hdr_len} to a value
not less than the length of the headers, including the transport
header. If \field{gso_type} has the VIRTIO_NET_HDR_GSO_UDP_TUNNEL_IPV4 bit
or the VIRTIO_NET_HDR_GSO_UDP_TUNNEL_IPV6 bit set, the referenced transport
header is the inner one.

If the VIRTIO_NET_F_GUEST_CSUM feature has been negotiated, the
device MAY set the VIRTIO_NET_HDR_F_DATA_VALID bit in
\field{flags}, if so, the device MUST validate the packet
checksum. If the VIRTIO_NET_F_GUEST_UDP_TUNNEL_GSO_CSUM feature has
been negotiated, and the VIRTIO_NET_HDR_F_UDP_TUNNEL_CSUM bit set in
\field{flags}, both the outer UDP checksum and the inner transport
checksum have been validated.
Otherwise level of checksum is validated: in case of multiple
encapsulated protocols the outermost one.

If either the VIRTIO_NET_HDR_GSO_UDP_TUNNEL_IPV4 bit or the
VIRTIO_NET_HDR_GSO_UDP_TUNNEL_IPV6 bit in \field{gso_type} are set,
the device MUST NOT set the VIRTIO_NET_HDR_F_DATA_VALID bit in
\field{flags}.

If the VIRTIO_NET_F_GUEST_UDP_TUNNEL_GSO_CSUM feature has been negotiated
and either the VIRTIO_NET_HDR_GSO_UDP_TUNNEL_IPV4 bit is set or the
VIRTIO_NET_HDR_GSO_UDP_TUNNEL_IPV6 bit is set in \field{gso_type}, the
device MAY set the VIRTIO_NET_HDR_F_UDP_TUNNEL_CSUM bit in
\field{flags}, if so the device MUST set the packet outer UDP checksum
stored in the receive buffer to the outer UDP pseudo header.

Otherwise, the VIRTIO_NET_F_GUEST_UDP_TUNNEL_GSO_CSUM feature has been
negotiated, either the VIRTIO_NET_HDR_GSO_UDP_TUNNEL_IPV4 bit is set or the
VIRTIO_NET_HDR_GSO_UDP_TUNNEL_IPV6 bit is set in \field{gso_type},
and the bit VIRTIO_NET_HDR_F_UDP_TUNNEL_CSUM is not set in
\field{flags}, the device MUST either provide a zero outer UDP header
checksum or a fully checksummed outer UDP header.

\drivernormative{\paragraph}{Processing of Incoming
Packets}{Device Types / Network Device / Device Operation /
Processing of Incoming Packets}

The driver MUST ignore \field{flag} bits that it does not recognize.

If VIRTIO_NET_HDR_F_NEEDS_CSUM bit in \field{flags} is not set or
if VIRTIO_NET_HDR_F_RSC_INFO bit \field{flags} is set, the
driver MUST NOT use the \field{csum_start} and \field{csum_offset}.

If one of the VIRTIO_NET_F_GUEST_TSO4, TSO6, UFO, USO4 or USO6 options have
been negotiated, the driver MAY use \field{hdr_len} only as a hint about the
transport header size.
The driver MUST NOT rely on \field{hdr_len} to be correct.
\begin{note}
This is due to various bugs in implementations.
\end{note}

If neither VIRTIO_NET_HDR_F_NEEDS_CSUM nor
VIRTIO_NET_HDR_F_DATA_VALID is set, the driver MUST NOT
rely on the packet checksum being correct.

If both the VIRTIO_NET_HDR_GSO_UDP_TUNNEL_IPV4 bit and
the VIRTIO_NET_HDR_GSO_UDP_TUNNEL_IPV6 bit in in \field{gso_type} are set,
the driver MUST NOT accept the packet.

If the VIRTIO_NET_HDR_GSO_UDP_TUNNEL_IPV4 bit or the VIRTIO_NET_HDR_GSO_UDP_TUNNEL_IPV6
bit in \field{gso_type} are not set, the driver MUST NOT use the
\field{outer_th_offset} and \field{inner_nh_offset}.

If either the VIRTIO_NET_HDR_GSO_UDP_TUNNEL_IPV4 bit or
the VIRTIO_NET_HDR_GSO_UDP_TUNNEL_IPV6 bit in \field{gso_type} are set, and any of
the following is true:
\begin{itemize}
\item the VIRTIO_NET_HDR_F_NEEDS_CSUM bit is not set in \field{flags}
\item the VIRTIO_NET_HDR_F_DATA_VALID bit is set in \field{flags}
\item the \field{gso_type} excluding the VIRTIO_NET_HDR_GSO_UDP_TUNNEL_IPV4
bit and the VIRTIO_NET_HDR_GSO_UDP_TUNNEL_IPV6 bit is VIRTIO_NET_HDR_GSO_NONE
\end{itemize}
the driver MUST NOT accept the packet.

If the VIRTIO_NET_HDR_F_UDP_TUNNEL_CSUM bit and the VIRTIO_NET_HDR_F_NEEDS_CSUM
bit in \field{flags} are set,
and both the bits VIRTIO_NET_HDR_GSO_UDP_TUNNEL_IPV4 and
VIRTIO_NET_HDR_GSO_UDP_TUNNEL_IPV6 in \field{gso_type} are not set,
the driver MOST NOT accept the packet.

\paragraph{Hash calculation for incoming packets}
\label{sec:Device Types / Network Device / Device Operation / Processing of Incoming Packets / Hash calculation for incoming packets}

A device attempts to calculate a per-packet hash in the following cases:
\begin{itemize}
\item The feature VIRTIO_NET_F_RSS was negotiated. The device uses the hash to determine the receive virtqueue to place incoming packets.
\item The feature VIRTIO_NET_F_HASH_REPORT was negotiated. The device reports the hash value and the hash type with the packet.
\end{itemize}

If the feature VIRTIO_NET_F_RSS was negotiated:
\begin{itemize}
\item The device uses \field{hash_types} of the virtio_net_rss_config structure as 'Enabled hash types' bitmask.
\item If additionally the feature VIRTIO_NET_F_HASH_TUNNEL was negotiated, the device uses \field{enabled_tunnel_types} of the
      virtnet_hash_tunnel structure as 'Encapsulation types enabled for inner header hash' bitmask.
\item The device uses a key as defined in \field{hash_key_data} and \field{hash_key_length} of the virtio_net_rss_config structure (see
\ref{sec:Device Types / Network Device / Device Operation / Control Virtqueue / Receive-side scaling (RSS) / Setting RSS parameters}).
\end{itemize}

If the feature VIRTIO_NET_F_RSS was not negotiated:
\begin{itemize}
\item The device uses \field{hash_types} of the virtio_net_hash_config structure as 'Enabled hash types' bitmask.
\item If additionally the feature VIRTIO_NET_F_HASH_TUNNEL was negotiated, the device uses \field{enabled_tunnel_types} of the
      virtnet_hash_tunnel structure as 'Encapsulation types enabled for inner header hash' bitmask.
\item The device uses a key as defined in \field{hash_key_data} and \field{hash_key_length} of the virtio_net_hash_config structure (see
\ref{sec:Device Types / Network Device / Device Operation / Control Virtqueue / Automatic receive steering in multiqueue mode / Hash calculation}).
\end{itemize}

Note that if the device offers VIRTIO_NET_F_HASH_REPORT, even if it supports only one pair of virtqueues, it MUST support
at least one of commands of VIRTIO_NET_CTRL_MQ class to configure reported hash parameters:
\begin{itemize}
\item If the device offers VIRTIO_NET_F_RSS, it MUST support VIRTIO_NET_CTRL_MQ_RSS_CONFIG command per
 \ref{sec:Device Types / Network Device / Device Operation / Control Virtqueue / Receive-side scaling (RSS) / Setting RSS parameters}.
\item Otherwise the device MUST support VIRTIO_NET_CTRL_MQ_HASH_CONFIG command per
 \ref{sec:Device Types / Network Device / Device Operation / Control Virtqueue / Automatic receive steering in multiqueue mode / Hash calculation}.
\end{itemize}

The per-packet hash calculation can depend on the IP packet type. See
\hyperref[intro:IP]{[IP]}, \hyperref[intro:UDP]{[UDP]} and \hyperref[intro:TCP]{[TCP]}.

\subparagraph{Supported/enabled hash types}
\label{sec:Device Types / Network Device / Device Operation / Processing of Incoming Packets / Hash calculation for incoming packets / Supported/enabled hash types}
Hash types applicable for IPv4 packets:
\begin{lstlisting}
#define VIRTIO_NET_HASH_TYPE_IPv4              (1 << 0)
#define VIRTIO_NET_HASH_TYPE_TCPv4             (1 << 1)
#define VIRTIO_NET_HASH_TYPE_UDPv4             (1 << 2)
\end{lstlisting}
Hash types applicable for IPv6 packets without extension headers
\begin{lstlisting}
#define VIRTIO_NET_HASH_TYPE_IPv6              (1 << 3)
#define VIRTIO_NET_HASH_TYPE_TCPv6             (1 << 4)
#define VIRTIO_NET_HASH_TYPE_UDPv6             (1 << 5)
\end{lstlisting}
Hash types applicable for IPv6 packets with extension headers
\begin{lstlisting}
#define VIRTIO_NET_HASH_TYPE_IP_EX             (1 << 6)
#define VIRTIO_NET_HASH_TYPE_TCP_EX            (1 << 7)
#define VIRTIO_NET_HASH_TYPE_UDP_EX            (1 << 8)
\end{lstlisting}

\subparagraph{IPv4 packets}
\label{sec:Device Types / Network Device / Device Operation / Processing of Incoming Packets / Hash calculation for incoming packets / IPv4 packets}
The device calculates the hash on IPv4 packets according to 'Enabled hash types' bitmask as follows:
\begin{itemize}
\item If VIRTIO_NET_HASH_TYPE_TCPv4 is set and the packet has
a TCP header, the hash is calculated over the following fields:
\begin{itemize}
\item Source IP address
\item Destination IP address
\item Source TCP port
\item Destination TCP port
\end{itemize}
\item Else if VIRTIO_NET_HASH_TYPE_UDPv4 is set and the
packet has a UDP header, the hash is calculated over the following fields:
\begin{itemize}
\item Source IP address
\item Destination IP address
\item Source UDP port
\item Destination UDP port
\end{itemize}
\item Else if VIRTIO_NET_HASH_TYPE_IPv4 is set, the hash is
calculated over the following fields:
\begin{itemize}
\item Source IP address
\item Destination IP address
\end{itemize}
\item Else the device does not calculate the hash
\end{itemize}

\subparagraph{IPv6 packets without extension header}
\label{sec:Device Types / Network Device / Device Operation / Processing of Incoming Packets / Hash calculation for incoming packets / IPv6 packets without extension header}
The device calculates the hash on IPv6 packets without extension
headers according to 'Enabled hash types' bitmask as follows:
\begin{itemize}
\item If VIRTIO_NET_HASH_TYPE_TCPv6 is set and the packet has
a TCPv6 header, the hash is calculated over the following fields:
\begin{itemize}
\item Source IPv6 address
\item Destination IPv6 address
\item Source TCP port
\item Destination TCP port
\end{itemize}
\item Else if VIRTIO_NET_HASH_TYPE_UDPv6 is set and the
packet has a UDPv6 header, the hash is calculated over the following fields:
\begin{itemize}
\item Source IPv6 address
\item Destination IPv6 address
\item Source UDP port
\item Destination UDP port
\end{itemize}
\item Else if VIRTIO_NET_HASH_TYPE_IPv6 is set, the hash is
calculated over the following fields:
\begin{itemize}
\item Source IPv6 address
\item Destination IPv6 address
\end{itemize}
\item Else the device does not calculate the hash
\end{itemize}

\subparagraph{IPv6 packets with extension header}
\label{sec:Device Types / Network Device / Device Operation / Processing of Incoming Packets / Hash calculation for incoming packets / IPv6 packets with extension header}
The device calculates the hash on IPv6 packets with extension
headers according to 'Enabled hash types' bitmask as follows:
\begin{itemize}
\item If VIRTIO_NET_HASH_TYPE_TCP_EX is set and the packet
has a TCPv6 header, the hash is calculated over the following fields:
\begin{itemize}
\item Home address from the home address option in the IPv6 destination options header. If the extension header is not present, use the Source IPv6 address.
\item IPv6 address that is contained in the Routing-Header-Type-2 from the associated extension header. If the extension header is not present, use the Destination IPv6 address.
\item Source TCP port
\item Destination TCP port
\end{itemize}
\item Else if VIRTIO_NET_HASH_TYPE_UDP_EX is set and the
packet has a UDPv6 header, the hash is calculated over the following fields:
\begin{itemize}
\item Home address from the home address option in the IPv6 destination options header. If the extension header is not present, use the Source IPv6 address.
\item IPv6 address that is contained in the Routing-Header-Type-2 from the associated extension header. If the extension header is not present, use the Destination IPv6 address.
\item Source UDP port
\item Destination UDP port
\end{itemize}
\item Else if VIRTIO_NET_HASH_TYPE_IP_EX is set, the hash is
calculated over the following fields:
\begin{itemize}
\item Home address from the home address option in the IPv6 destination options header. If the extension header is not present, use the Source IPv6 address.
\item IPv6 address that is contained in the Routing-Header-Type-2 from the associated extension header. If the extension header is not present, use the Destination IPv6 address.
\end{itemize}
\item Else skip IPv6 extension headers and calculate the hash as
defined for an IPv6 packet without extension headers
(see \ref{sec:Device Types / Network Device / Device Operation / Processing of Incoming Packets / Hash calculation for incoming packets / IPv6 packets without extension header}).
\end{itemize}

\paragraph{Inner Header Hash}
\label{sec:Device Types / Network Device / Device Operation / Processing of Incoming Packets / Inner Header Hash}

If VIRTIO_NET_F_HASH_TUNNEL has been negotiated, the driver can send the command
VIRTIO_NET_CTRL_HASH_TUNNEL_SET to configure the calculation of the inner header hash.

\begin{lstlisting}
struct virtnet_hash_tunnel {
    le32 enabled_tunnel_types;
};

#define VIRTIO_NET_CTRL_HASH_TUNNEL 7
 #define VIRTIO_NET_CTRL_HASH_TUNNEL_SET 0
\end{lstlisting}

Field \field{enabled_tunnel_types} contains the bitmask of encapsulation types enabled for inner header hash.
See \ref{sec:Device Types / Network Device / Device Operation / Processing of Incoming Packets /
Hash calculation for incoming packets / Encapsulation types supported/enabled for inner header hash}.

The class VIRTIO_NET_CTRL_HASH_TUNNEL has one command:
VIRTIO_NET_CTRL_HASH_TUNNEL_SET sets \field{enabled_tunnel_types} for the device using the
virtnet_hash_tunnel structure, which is read-only for the device.

Inner header hash is disabled by VIRTIO_NET_CTRL_HASH_TUNNEL_SET with \field{enabled_tunnel_types} set to 0.

Initially (before the driver sends any VIRTIO_NET_CTRL_HASH_TUNNEL_SET command) all
encapsulation types are disabled for inner header hash.

\subparagraph{Encapsulated packet}
\label{sec:Device Types / Network Device / Device Operation / Processing of Incoming Packets / Hash calculation for incoming packets / Encapsulated packet}

Multiple tunneling protocols allow encapsulating an inner, payload packet in an outer, encapsulated packet.
The encapsulated packet thus contains an outer header and an inner header, and the device calculates the
hash over either the inner header or the outer header.

If VIRTIO_NET_F_HASH_TUNNEL is negotiated and a received encapsulated packet's outer header matches one of the
encapsulation types enabled in \field{enabled_tunnel_types}, then the device uses the inner header for hash
calculations (only a single level of encapsulation is currently supported).

If VIRTIO_NET_F_HASH_TUNNEL is negotiated and a received packet's (outer) header does not match any encapsulation
types enabled in \field{enabled_tunnel_types}, then the device uses the outer header for hash calculations.

\subparagraph{Encapsulation types supported/enabled for inner header hash}
\label{sec:Device Types / Network Device / Device Operation / Processing of Incoming Packets /
Hash calculation for incoming packets / Encapsulation types supported/enabled for inner header hash}

Encapsulation types applicable for inner header hash:
\begin{lstlisting}[escapechar=|]
#define VIRTIO_NET_HASH_TUNNEL_TYPE_GRE_2784    (1 << 0) /* |\hyperref[intro:rfc2784]{[RFC2784]}| */
#define VIRTIO_NET_HASH_TUNNEL_TYPE_GRE_2890    (1 << 1) /* |\hyperref[intro:rfc2890]{[RFC2890]}| */
#define VIRTIO_NET_HASH_TUNNEL_TYPE_GRE_7676    (1 << 2) /* |\hyperref[intro:rfc7676]{[RFC7676]}| */
#define VIRTIO_NET_HASH_TUNNEL_TYPE_GRE_UDP     (1 << 3) /* |\hyperref[intro:rfc8086]{[GRE-in-UDP]}| */
#define VIRTIO_NET_HASH_TUNNEL_TYPE_VXLAN       (1 << 4) /* |\hyperref[intro:vxlan]{[VXLAN]}| */
#define VIRTIO_NET_HASH_TUNNEL_TYPE_VXLAN_GPE   (1 << 5) /* |\hyperref[intro:vxlan-gpe]{[VXLAN-GPE]}| */
#define VIRTIO_NET_HASH_TUNNEL_TYPE_GENEVE      (1 << 6) /* |\hyperref[intro:geneve]{[GENEVE]}| */
#define VIRTIO_NET_HASH_TUNNEL_TYPE_IPIP        (1 << 7) /* |\hyperref[intro:ipip]{[IPIP]}| */
#define VIRTIO_NET_HASH_TUNNEL_TYPE_NVGRE       (1 << 8) /* |\hyperref[intro:nvgre]{[NVGRE]}| */
\end{lstlisting}

\subparagraph{Advice}
Example uses of the inner header hash:
\begin{itemize}
\item Legacy tunneling protocols, lacking the outer header entropy, can use RSS with the inner header hash to
      distribute flows with identical outer but different inner headers across various queues, improving performance.
\item Identify an inner flow distributed across multiple outer tunnels.
\end{itemize}

As using the inner header hash completely discards the outer header entropy, care must be taken
if the inner header is controlled by an adversary, as the adversary can then intentionally create
configurations with insufficient entropy.

Besides disabling the inner header hash, mitigations would depend on how the hash is used. When the hash
use is limited to the RSS queue selection, the inner header hash may have quality of service (QoS) limitations.

\devicenormative{\subparagraph}{Inner Header Hash}{Device Types / Network Device / Device Operation / Control Virtqueue / Inner Header Hash}

If the (outer) header of the received packet does not match any encapsulation types enabled
in \field{enabled_tunnel_types}, the device MUST calculate the hash on the outer header.

If the device receives any bits in \field{enabled_tunnel_types} which are not set in \field{supported_tunnel_types},
it SHOULD respond to the VIRTIO_NET_CTRL_HASH_TUNNEL_SET command with VIRTIO_NET_ERR.

If the driver sets \field{enabled_tunnel_types} to 0 through VIRTIO_NET_CTRL_HASH_TUNNEL_SET or upon the device reset,
the device MUST disable the inner header hash for all encapsulation types.

\drivernormative{\subparagraph}{Inner Header Hash}{Device Types / Network Device / Device Operation / Control Virtqueue / Inner Header Hash}

The driver MUST have negotiated the VIRTIO_NET_F_HASH_TUNNEL feature when issuing the VIRTIO_NET_CTRL_HASH_TUNNEL_SET command.

The driver MUST NOT set any bits in \field{enabled_tunnel_types} which are not set in \field{supported_tunnel_types}.

The driver MUST ignore bits in \field{supported_tunnel_types} which are not documented in this specification.

\paragraph{Hash reporting for incoming packets}
\label{sec:Device Types / Network Device / Device Operation / Processing of Incoming Packets / Hash reporting for incoming packets}

If VIRTIO_NET_F_HASH_REPORT was negotiated and
 the device has calculated the hash for the packet, the device fills \field{hash_report} with the report type of calculated hash
and \field{hash_value} with the value of calculated hash.

If VIRTIO_NET_F_HASH_REPORT was negotiated but due to any reason the
hash was not calculated, the device sets \field{hash_report} to VIRTIO_NET_HASH_REPORT_NONE.

Possible values that the device can report in \field{hash_report} are defined below.
They correspond to supported hash types defined in
\ref{sec:Device Types / Network Device / Device Operation / Processing of Incoming Packets / Hash calculation for incoming packets / Supported/enabled hash types}
as follows:

VIRTIO_NET_HASH_TYPE_XXX = 1 << (VIRTIO_NET_HASH_REPORT_XXX - 1)

\begin{lstlisting}
#define VIRTIO_NET_HASH_REPORT_NONE            0
#define VIRTIO_NET_HASH_REPORT_IPv4            1
#define VIRTIO_NET_HASH_REPORT_TCPv4           2
#define VIRTIO_NET_HASH_REPORT_UDPv4           3
#define VIRTIO_NET_HASH_REPORT_IPv6            4
#define VIRTIO_NET_HASH_REPORT_TCPv6           5
#define VIRTIO_NET_HASH_REPORT_UDPv6           6
#define VIRTIO_NET_HASH_REPORT_IPv6_EX         7
#define VIRTIO_NET_HASH_REPORT_TCPv6_EX        8
#define VIRTIO_NET_HASH_REPORT_UDPv6_EX        9
\end{lstlisting}

\subsubsection{Control Virtqueue}\label{sec:Device Types / Network Device / Device Operation / Control Virtqueue}

The driver uses the control virtqueue (if VIRTIO_NET_F_CTRL_VQ is
negotiated) to send commands to manipulate various features of
the device which would not easily map into the configuration
space.

All commands are of the following form:

\begin{lstlisting}
struct virtio_net_ctrl {
        u8 class;
        u8 command;
        u8 command-specific-data[];
        u8 ack;
        u8 command-specific-result[];
};

/* ack values */
#define VIRTIO_NET_OK     0
#define VIRTIO_NET_ERR    1
\end{lstlisting}

The \field{class}, \field{command} and command-specific-data are set by the
driver, and the device sets the \field{ack} byte and optionally
\field{command-specific-result}. There is little the driver can
do except issue a diagnostic if \field{ack} is not VIRTIO_NET_OK.

The command VIRTIO_NET_CTRL_STATS_QUERY and VIRTIO_NET_CTRL_STATS_GET contain
\field{command-specific-result}.

\paragraph{Packet Receive Filtering}\label{sec:Device Types / Network Device / Device Operation / Control Virtqueue / Packet Receive Filtering}
\label{sec:Device Types / Network Device / Device Operation / Control Virtqueue / Setting Promiscuous Mode}%old label for latexdiff

If the VIRTIO_NET_F_CTRL_RX and VIRTIO_NET_F_CTRL_RX_EXTRA
features are negotiated, the driver can send control commands for
promiscuous mode, multicast, unicast and broadcast receiving.

\begin{note}
In general, these commands are best-effort: unwanted
packets could still arrive.
\end{note}

\begin{lstlisting}
#define VIRTIO_NET_CTRL_RX    0
 #define VIRTIO_NET_CTRL_RX_PROMISC      0
 #define VIRTIO_NET_CTRL_RX_ALLMULTI     1
 #define VIRTIO_NET_CTRL_RX_ALLUNI       2
 #define VIRTIO_NET_CTRL_RX_NOMULTI      3
 #define VIRTIO_NET_CTRL_RX_NOUNI        4
 #define VIRTIO_NET_CTRL_RX_NOBCAST      5
\end{lstlisting}


\devicenormative{\subparagraph}{Packet Receive Filtering}{Device Types / Network Device / Device Operation / Control Virtqueue / Packet Receive Filtering}

If the VIRTIO_NET_F_CTRL_RX feature has been negotiated,
the device MUST support the following VIRTIO_NET_CTRL_RX class
commands:
\begin{itemize}
\item VIRTIO_NET_CTRL_RX_PROMISC turns promiscuous mode on and
off. The command-specific-data is one byte containing 0 (off) or
1 (on). If promiscuous mode is on, the device SHOULD receive all
incoming packets.
This SHOULD take effect even if one of the other modes set by
a VIRTIO_NET_CTRL_RX class command is on.
\item VIRTIO_NET_CTRL_RX_ALLMULTI turns all-multicast receive on and
off. The command-specific-data is one byte containing 0 (off) or
1 (on). When all-multicast receive is on the device SHOULD allow
all incoming multicast packets.
\end{itemize}

If the VIRTIO_NET_F_CTRL_RX_EXTRA feature has been negotiated,
the device MUST support the following VIRTIO_NET_CTRL_RX class
commands:
\begin{itemize}
\item VIRTIO_NET_CTRL_RX_ALLUNI turns all-unicast receive on and
off. The command-specific-data is one byte containing 0 (off) or
1 (on). When all-unicast receive is on the device SHOULD allow
all incoming unicast packets.
\item VIRTIO_NET_CTRL_RX_NOMULTI suppresses multicast receive.
The command-specific-data is one byte containing 0 (multicast
receive allowed) or 1 (multicast receive suppressed).
When multicast receive is suppressed, the device SHOULD NOT
send multicast packets to the driver.
This SHOULD take effect even if VIRTIO_NET_CTRL_RX_ALLMULTI is on.
This filter SHOULD NOT apply to broadcast packets.
\item VIRTIO_NET_CTRL_RX_NOUNI suppresses unicast receive.
The command-specific-data is one byte containing 0 (unicast
receive allowed) or 1 (unicast receive suppressed).
When unicast receive is suppressed, the device SHOULD NOT
send unicast packets to the driver.
This SHOULD take effect even if VIRTIO_NET_CTRL_RX_ALLUNI is on.
\item VIRTIO_NET_CTRL_RX_NOBCAST suppresses broadcast receive.
The command-specific-data is one byte containing 0 (broadcast
receive allowed) or 1 (broadcast receive suppressed).
When broadcast receive is suppressed, the device SHOULD NOT
send broadcast packets to the driver.
This SHOULD take effect even if VIRTIO_NET_CTRL_RX_ALLMULTI is on.
\end{itemize}

\drivernormative{\subparagraph}{Packet Receive Filtering}{Device Types / Network Device / Device Operation / Control Virtqueue / Packet Receive Filtering}

If the VIRTIO_NET_F_CTRL_RX feature has not been negotiated,
the driver MUST NOT issue commands VIRTIO_NET_CTRL_RX_PROMISC or
VIRTIO_NET_CTRL_RX_ALLMULTI.

If the VIRTIO_NET_F_CTRL_RX_EXTRA feature has not been negotiated,
the driver MUST NOT issue commands
 VIRTIO_NET_CTRL_RX_ALLUNI,
 VIRTIO_NET_CTRL_RX_NOMULTI,
 VIRTIO_NET_CTRL_RX_NOUNI or
 VIRTIO_NET_CTRL_RX_NOBCAST.

\paragraph{Setting MAC Address Filtering}\label{sec:Device Types / Network Device / Device Operation / Control Virtqueue / Setting MAC Address Filtering}

If the VIRTIO_NET_F_CTRL_RX feature is negotiated, the driver can
send control commands for MAC address filtering.

\begin{lstlisting}
struct virtio_net_ctrl_mac {
        le32 entries;
        u8 macs[entries][6];
};

#define VIRTIO_NET_CTRL_MAC    1
 #define VIRTIO_NET_CTRL_MAC_TABLE_SET        0
 #define VIRTIO_NET_CTRL_MAC_ADDR_SET         1
\end{lstlisting}

The device can filter incoming packets by any number of destination
MAC addresses\footnote{Since there are no guarantees, it can use a hash filter or
silently switch to allmulti or promiscuous mode if it is given too
many addresses.
}. This table is set using the class
VIRTIO_NET_CTRL_MAC and the command VIRTIO_NET_CTRL_MAC_TABLE_SET. The
command-specific-data is two variable length tables of 6-byte MAC
addresses (as described in struct virtio_net_ctrl_mac). The first table contains unicast addresses, and the second
contains multicast addresses.

The VIRTIO_NET_CTRL_MAC_ADDR_SET command is used to set the
default MAC address which rx filtering
accepts (and if VIRTIO_NET_F_MAC has been negotiated,
this will be reflected in \field{mac} in config space).

The command-specific-data for VIRTIO_NET_CTRL_MAC_ADDR_SET is
the 6-byte MAC address.

\devicenormative{\subparagraph}{Setting MAC Address Filtering}{Device Types / Network Device / Device Operation / Control Virtqueue / Setting MAC Address Filtering}

The device MUST have an empty MAC filtering table on reset.

The device MUST update the MAC filtering table before it consumes
the VIRTIO_NET_CTRL_MAC_TABLE_SET command.

The device MUST update \field{mac} in config space before it consumes
the VIRTIO_NET_CTRL_MAC_ADDR_SET command, if VIRTIO_NET_F_MAC has
been negotiated.

The device SHOULD drop incoming packets which have a destination MAC which
matches neither the \field{mac} (or that set with VIRTIO_NET_CTRL_MAC_ADDR_SET)
nor the MAC filtering table.

\drivernormative{\subparagraph}{Setting MAC Address Filtering}{Device Types / Network Device / Device Operation / Control Virtqueue / Setting MAC Address Filtering}

If VIRTIO_NET_F_CTRL_RX has not been negotiated,
the driver MUST NOT issue VIRTIO_NET_CTRL_MAC class commands.

If VIRTIO_NET_F_CTRL_RX has been negotiated,
the driver SHOULD issue VIRTIO_NET_CTRL_MAC_ADDR_SET
to set the default mac if it is different from \field{mac}.

The driver MUST follow the VIRTIO_NET_CTRL_MAC_TABLE_SET command
by a le32 number, followed by that number of non-multicast
MAC addresses, followed by another le32 number, followed by
that number of multicast addresses.  Either number MAY be 0.

\subparagraph{Legacy Interface: Setting MAC Address Filtering}\label{sec:Device Types / Network Device / Device Operation / Control Virtqueue / Setting MAC Address Filtering / Legacy Interface: Setting MAC Address Filtering}
When using the legacy interface, transitional devices and drivers
MUST format \field{entries} in struct virtio_net_ctrl_mac
according to the native endian of the guest rather than
(necessarily when not using the legacy interface) little-endian.

Legacy drivers that didn't negotiate VIRTIO_NET_F_CTRL_MAC_ADDR
changed \field{mac} in config space when NIC is accepting
incoming packets. These drivers always wrote the mac value from
first to last byte, therefore after detecting such drivers,
a transitional device MAY defer MAC update, or MAY defer
processing incoming packets until driver writes the last byte
of \field{mac} in the config space.

\paragraph{VLAN Filtering}\label{sec:Device Types / Network Device / Device Operation / Control Virtqueue / VLAN Filtering}

If the driver negotiates the VIRTIO_NET_F_CTRL_VLAN feature, it
can control a VLAN filter table in the device. The VLAN filter
table applies only to VLAN tagged packets.

When VIRTIO_NET_F_CTRL_VLAN is negotiated, the device starts with
an empty VLAN filter table.

\begin{note}
Similar to the MAC address based filtering, the VLAN filtering
is also best-effort: unwanted packets could still arrive.
\end{note}

\begin{lstlisting}
#define VIRTIO_NET_CTRL_VLAN       2
 #define VIRTIO_NET_CTRL_VLAN_ADD             0
 #define VIRTIO_NET_CTRL_VLAN_DEL             1
\end{lstlisting}

Both the VIRTIO_NET_CTRL_VLAN_ADD and VIRTIO_NET_CTRL_VLAN_DEL
command take a little-endian 16-bit VLAN id as the command-specific-data.

VIRTIO_NET_CTRL_VLAN_ADD command adds the specified VLAN to the
VLAN filter table.

VIRTIO_NET_CTRL_VLAN_DEL command removes the specified VLAN from
the VLAN filter table.

\devicenormative{\subparagraph}{VLAN Filtering}{Device Types / Network Device / Device Operation / Control Virtqueue / VLAN Filtering}

When VIRTIO_NET_F_CTRL_VLAN is not negotiated, the device MUST
accept all VLAN tagged packets.

When VIRTIO_NET_F_CTRL_VLAN is negotiated, the device MUST
accept all VLAN tagged packets whose VLAN tag is present in
the VLAN filter table and SHOULD drop all VLAN tagged packets
whose VLAN tag is absent in the VLAN filter table.

\subparagraph{Legacy Interface: VLAN Filtering}\label{sec:Device Types / Network Device / Device Operation / Control Virtqueue / VLAN Filtering / Legacy Interface: VLAN Filtering}
When using the legacy interface, transitional devices and drivers
MUST format the VLAN id
according to the native endian of the guest rather than
(necessarily when not using the legacy interface) little-endian.

\paragraph{Gratuitous Packet Sending}\label{sec:Device Types / Network Device / Device Operation / Control Virtqueue / Gratuitous Packet Sending}

If the driver negotiates the VIRTIO_NET_F_GUEST_ANNOUNCE (depends
on VIRTIO_NET_F_CTRL_VQ), the device can ask the driver to send gratuitous
packets; this is usually done after the guest has been physically
migrated, and needs to announce its presence on the new network
links. (As hypervisor does not have the knowledge of guest
network configuration (eg. tagged vlan) it is simplest to prod
the guest in this way).

\begin{lstlisting}
#define VIRTIO_NET_CTRL_ANNOUNCE       3
 #define VIRTIO_NET_CTRL_ANNOUNCE_ACK             0
\end{lstlisting}

The driver checks VIRTIO_NET_S_ANNOUNCE bit in the device configuration \field{status} field
when it notices the changes of device configuration. The
command VIRTIO_NET_CTRL_ANNOUNCE_ACK is used to indicate that
driver has received the notification and device clears the
VIRTIO_NET_S_ANNOUNCE bit in \field{status}.

Processing this notification involves:

\begin{enumerate}
\item Sending the gratuitous packets (eg. ARP) or marking there are pending
  gratuitous packets to be sent and letting deferred routine to
  send them.

\item Sending VIRTIO_NET_CTRL_ANNOUNCE_ACK command through control
  vq.
\end{enumerate}

\drivernormative{\subparagraph}{Gratuitous Packet Sending}{Device Types / Network Device / Device Operation / Control Virtqueue / Gratuitous Packet Sending}

If the driver negotiates VIRTIO_NET_F_GUEST_ANNOUNCE, it SHOULD notify
network peers of its new location after it sees the VIRTIO_NET_S_ANNOUNCE bit
in \field{status}.  The driver MUST send a command on the command queue
with class VIRTIO_NET_CTRL_ANNOUNCE and command VIRTIO_NET_CTRL_ANNOUNCE_ACK.

\devicenormative{\subparagraph}{Gratuitous Packet Sending}{Device Types / Network Device / Device Operation / Control Virtqueue / Gratuitous Packet Sending}

If VIRTIO_NET_F_GUEST_ANNOUNCE is negotiated, the device MUST clear the
VIRTIO_NET_S_ANNOUNCE bit in \field{status} upon receipt of a command buffer
with class VIRTIO_NET_CTRL_ANNOUNCE and command VIRTIO_NET_CTRL_ANNOUNCE_ACK
before marking the buffer as used.

\paragraph{Device operation in multiqueue mode}\label{sec:Device Types / Network Device / Device Operation / Control Virtqueue / Device operation in multiqueue mode}

This specification defines the following modes that a device MAY implement for operation with multiple transmit/receive virtqueues:
\begin{itemize}
\item Automatic receive steering as defined in \ref{sec:Device Types / Network Device / Device Operation / Control Virtqueue / Automatic receive steering in multiqueue mode}.
 If a device supports this mode, it offers the VIRTIO_NET_F_MQ feature bit.
\item Receive-side scaling as defined in \ref{devicenormative:Device Types / Network Device / Device Operation / Control Virtqueue / Receive-side scaling (RSS) / RSS processing}.
 If a device supports this mode, it offers the VIRTIO_NET_F_RSS feature bit.
\end{itemize}

A device MAY support one of these features or both. The driver MAY negotiate any set of these features that the device supports.

Multiqueue is disabled by default.

The driver enables multiqueue by sending a command using \field{class} VIRTIO_NET_CTRL_MQ. The \field{command} selects the mode of multiqueue operation, as follows:
\begin{lstlisting}
#define VIRTIO_NET_CTRL_MQ    4
 #define VIRTIO_NET_CTRL_MQ_VQ_PAIRS_SET        0 (for automatic receive steering)
 #define VIRTIO_NET_CTRL_MQ_RSS_CONFIG          1 (for configurable receive steering)
 #define VIRTIO_NET_CTRL_MQ_HASH_CONFIG         2 (for configurable hash calculation)
\end{lstlisting}

If more than one multiqueue mode is negotiated, the resulting device configuration is defined by the last command sent by the driver.

\paragraph{Automatic receive steering in multiqueue mode}\label{sec:Device Types / Network Device / Device Operation / Control Virtqueue / Automatic receive steering in multiqueue mode}

If the driver negotiates the VIRTIO_NET_F_MQ feature bit (depends on VIRTIO_NET_F_CTRL_VQ), it MAY transmit outgoing packets on one
of the multiple transmitq1\ldots transmitqN and ask the device to
queue incoming packets into one of the multiple receiveq1\ldots receiveqN
depending on the packet flow.

The driver enables multiqueue by
sending the VIRTIO_NET_CTRL_MQ_VQ_PAIRS_SET command, specifying
the number of the transmit and receive queues to be used up to
\field{max_virtqueue_pairs}; subsequently,
transmitq1\ldots transmitqn and receiveq1\ldots receiveqn where
n=\field{virtqueue_pairs} MAY be used.
\begin{lstlisting}
struct virtio_net_ctrl_mq_pairs_set {
       le16 virtqueue_pairs;
};
#define VIRTIO_NET_CTRL_MQ_VQ_PAIRS_MIN        1
#define VIRTIO_NET_CTRL_MQ_VQ_PAIRS_MAX        0x8000

\end{lstlisting}

When multiqueue is enabled by VIRTIO_NET_CTRL_MQ_VQ_PAIRS_SET command, the device MUST use automatic receive steering
based on packet flow. Programming of the receive steering
classificator is implicit. After the driver transmitted a packet of a
flow on transmitqX, the device SHOULD cause incoming packets for that flow to
be steered to receiveqX. For uni-directional protocols, or where
no packets have been transmitted yet, the device MAY steer a packet
to a random queue out of the specified receiveq1\ldots receiveqn.

Multiqueue is disabled by VIRTIO_NET_CTRL_MQ_VQ_PAIRS_SET with \field{virtqueue_pairs} to 1 (this is
the default) and waiting for the device to use the command buffer.

\drivernormative{\subparagraph}{Automatic receive steering in multiqueue mode}{Device Types / Network Device / Device Operation / Control Virtqueue / Automatic receive steering in multiqueue mode}

The driver MUST configure the virtqueues before enabling them with the
VIRTIO_NET_CTRL_MQ_VQ_PAIRS_SET command.

The driver MUST NOT request a \field{virtqueue_pairs} of 0 or
greater than \field{max_virtqueue_pairs} in the device configuration space.

The driver MUST queue packets only on any transmitq1 before the
VIRTIO_NET_CTRL_MQ_VQ_PAIRS_SET command.

The driver MUST NOT queue packets on transmit queues greater than
\field{virtqueue_pairs} once it has placed the VIRTIO_NET_CTRL_MQ_VQ_PAIRS_SET command in the available ring.

\devicenormative{\subparagraph}{Automatic receive steering in multiqueue mode}{Device Types / Network Device / Device Operation / Control Virtqueue / Automatic receive steering in multiqueue mode}

After initialization of reset, the device MUST queue packets only on receiveq1.

The device MUST NOT queue packets on receive queues greater than
\field{virtqueue_pairs} once it has placed the
VIRTIO_NET_CTRL_MQ_VQ_PAIRS_SET command in a used buffer.

If the destination receive queue is being reset (See \ref{sec:Basic Facilities of a Virtio Device / Virtqueues / Virtqueue Reset}),
the device SHOULD re-select another random queue. If all receive queues are
being reset, the device MUST drop the packet.

\subparagraph{Legacy Interface: Automatic receive steering in multiqueue mode}\label{sec:Device Types / Network Device / Device Operation / Control Virtqueue / Automatic receive steering in multiqueue mode / Legacy Interface: Automatic receive steering in multiqueue mode}
When using the legacy interface, transitional devices and drivers
MUST format \field{virtqueue_pairs}
according to the native endian of the guest rather than
(necessarily when not using the legacy interface) little-endian.

\subparagraph{Hash calculation}\label{sec:Device Types / Network Device / Device Operation / Control Virtqueue / Automatic receive steering in multiqueue mode / Hash calculation}
If VIRTIO_NET_F_HASH_REPORT was negotiated and the device uses automatic receive steering,
the device MUST support a command to configure hash calculation parameters.

The driver provides parameters for hash calculation as follows:

\field{class} VIRTIO_NET_CTRL_MQ, \field{command} VIRTIO_NET_CTRL_MQ_HASH_CONFIG.

The \field{command-specific-data} has following format:
\begin{lstlisting}
struct virtio_net_hash_config {
    le32 hash_types;
    le16 reserved[4];
    u8 hash_key_length;
    u8 hash_key_data[hash_key_length];
};
\end{lstlisting}
Field \field{hash_types} contains a bitmask of allowed hash types as
defined in
\ref{sec:Device Types / Network Device / Device Operation / Processing of Incoming Packets / Hash calculation for incoming packets / Supported/enabled hash types}.
Initially the device has all hash types disabled and reports only VIRTIO_NET_HASH_REPORT_NONE.

Field \field{reserved} MUST contain zeroes. It is defined to make the structure to match the layout of virtio_net_rss_config structure,
defined in \ref{sec:Device Types / Network Device / Device Operation / Control Virtqueue / Receive-side scaling (RSS)}.

Fields \field{hash_key_length} and \field{hash_key_data} define the key to be used in hash calculation.

\paragraph{Receive-side scaling (RSS)}\label{sec:Device Types / Network Device / Device Operation / Control Virtqueue / Receive-side scaling (RSS)}
A device offers the feature VIRTIO_NET_F_RSS if it supports RSS receive steering with Toeplitz hash calculation and configurable parameters.

A driver queries RSS capabilities of the device by reading device configuration as defined in \ref{sec:Device Types / Network Device / Device configuration layout}

\subparagraph{Setting RSS parameters}\label{sec:Device Types / Network Device / Device Operation / Control Virtqueue / Receive-side scaling (RSS) / Setting RSS parameters}

Driver sends a VIRTIO_NET_CTRL_MQ_RSS_CONFIG command using the following format for \field{command-specific-data}:
\begin{lstlisting}
struct rss_rq_id {
   le16 vq_index_1_16: 15; /* Bits 1 to 16 of the virtqueue index */
   le16 reserved: 1; /* Set to zero */
};

struct virtio_net_rss_config {
    le32 hash_types;
    le16 indirection_table_mask;
    struct rss_rq_id unclassified_queue;
    struct rss_rq_id indirection_table[indirection_table_length];
    le16 max_tx_vq;
    u8 hash_key_length;
    u8 hash_key_data[hash_key_length];
};
\end{lstlisting}
Field \field{hash_types} contains a bitmask of allowed hash types as
defined in
\ref{sec:Device Types / Network Device / Device Operation / Processing of Incoming Packets / Hash calculation for incoming packets / Supported/enabled hash types}.

Field \field{indirection_table_mask} is a mask to be applied to
the calculated hash to produce an index in the
\field{indirection_table} array.
Number of entries in \field{indirection_table} is (\field{indirection_table_mask} + 1).

\field{rss_rq_id} is a receive virtqueue id. \field{vq_index_1_16}
consists of bits 1 to 16 of a virtqueue index. For example, a
\field{vq_index_1_16} value of 3 corresponds to virtqueue index 6,
which maps to receiveq4.

Field \field{unclassified_queue} specifies the receive virtqueue id in which to
place unclassified packets.

Field \field{indirection_table} is an array of receive virtqueues ids.

A driver sets \field{max_tx_vq} to inform a device how many transmit virtqueues it may use (transmitq1\ldots transmitq \field{max_tx_vq}).

Fields \field{hash_key_length} and \field{hash_key_data} define the key to be used in hash calculation.

\drivernormative{\subparagraph}{Setting RSS parameters}{Device Types / Network Device / Device Operation / Control Virtqueue / Receive-side scaling (RSS) }

A driver MUST NOT send the VIRTIO_NET_CTRL_MQ_RSS_CONFIG command if the feature VIRTIO_NET_F_RSS has not been negotiated.

A driver MUST fill the \field{indirection_table} array only with
enabled receive virtqueues ids.

The number of entries in \field{indirection_table} (\field{indirection_table_mask} + 1) MUST be a power of two.

A driver MUST use \field{indirection_table_mask} values that are less than \field{rss_max_indirection_table_length} reported by a device.

A driver MUST NOT set any VIRTIO_NET_HASH_TYPE_ flags that are not supported by a device.

\devicenormative{\subparagraph}{RSS processing}{Device Types / Network Device / Device Operation / Control Virtqueue / Receive-side scaling (RSS) / RSS processing}
The device MUST determine the destination queue for a network packet as follows:
\begin{itemize}
\item Calculate the hash of the packet as defined in \ref{sec:Device Types / Network Device / Device Operation / Processing of Incoming Packets / Hash calculation for incoming packets}.
\item If the device did not calculate the hash for the specific packet, the device directs the packet to the receiveq specified by \field{unclassified_queue} of virtio_net_rss_config structure.
\item Apply \field{indirection_table_mask} to the calculated hash
and use the result as the index in the indirection table to get
the destination receive virtqueue id.
\item If the destination receive queue is being reset (See \ref{sec:Basic Facilities of a Virtio Device / Virtqueues / Virtqueue Reset}), the device MUST drop the packet.
\end{itemize}

\paragraph{RSS Context}\label{sec:Device Types / Network Device / Device Operation / Control Virtqueue / RSS Context}

An RSS context consists of configurable parameters specified by \ref{sec:Device Types / Network Device
/ Device Operation / Control Virtqueue / Receive-side scaling (RSS)}.

The RSS configuration supported by VIRTIO_NET_F_RSS is considered the default RSS configuration.

The device offers the feature VIRTIO_NET_F_RSS_CONTEXT if it supports one or multiple RSS contexts
(excluding the default RSS configuration) and configurable parameters.

\subparagraph{Querying RSS Context Capability}\label{sec:Device Types / Network Device / Device Operation / Control Virtqueue / RSS Context / Querying RSS Context Capability}

\begin{lstlisting}
#define VIRTNET_RSS_CTX_CTRL 9
 #define VIRTNET_RSS_CTX_CTRL_CAP_GET  0
 #define VIRTNET_RSS_CTX_CTRL_ADD      1
 #define VIRTNET_RSS_CTX_CTRL_MOD      2
 #define VIRTNET_RSS_CTX_CTRL_DEL      3

struct virtnet_rss_ctx_cap {
    le16 max_rss_contexts;
}
\end{lstlisting}

Field \field{max_rss_contexts} specifies the maximum number of RSS contexts \ref{sec:Device Types / Network Device /
Device Operation / Control Virtqueue / RSS Context} supported by the device.

The driver queries the RSS context capability of the device by sending the command VIRTNET_RSS_CTX_CTRL_CAP_GET
with the structure virtnet_rss_ctx_cap.

For the command VIRTNET_RSS_CTX_CTRL_CAP_GET, the structure virtnet_rss_ctx_cap is write-only for the device.

\subparagraph{Setting RSS Context Parameters}\label{sec:Device Types / Network Device / Device Operation / Control Virtqueue / RSS Context / Setting RSS Context Parameters}

\begin{lstlisting}
struct virtnet_rss_ctx_add_modify {
    le16 rss_ctx_id;
    u8 reserved[6];
    struct virtio_net_rss_config rss;
};

struct virtnet_rss_ctx_del {
    le16 rss_ctx_id;
};
\end{lstlisting}

RSS context parameters:
\begin{itemize}
\item  \field{rss_ctx_id}: ID of the specific RSS context.
\item  \field{rss}: RSS context parameters of the specific RSS context whose id is \field{rss_ctx_id}.
\end{itemize}

\field{reserved} is reserved and it is ignored by the device.

If the feature VIRTIO_NET_F_RSS_CONTEXT has been negotiated, the driver can send the following
VIRTNET_RSS_CTX_CTRL class commands:
\begin{enumerate}
\item VIRTNET_RSS_CTX_CTRL_ADD: use the structure virtnet_rss_ctx_add_modify to
       add an RSS context configured as \field{rss} and id as \field{rss_ctx_id} for the device.
\item VIRTNET_RSS_CTX_CTRL_MOD: use the structure virtnet_rss_ctx_add_modify to
       configure parameters of the RSS context whose id is \field{rss_ctx_id} as \field{rss} for the device.
\item VIRTNET_RSS_CTX_CTRL_DEL: use the structure virtnet_rss_ctx_del to delete
       the RSS context whose id is \field{rss_ctx_id} for the device.
\end{enumerate}

For commands VIRTNET_RSS_CTX_CTRL_ADD and VIRTNET_RSS_CTX_CTRL_MOD, the structure virtnet_rss_ctx_add_modify is read-only for the device.
For the command VIRTNET_RSS_CTX_CTRL_DEL, the structure virtnet_rss_ctx_del is read-only for the device.

\devicenormative{\subparagraph}{RSS Context}{Device Types / Network Device / Device Operation / Control Virtqueue / RSS Context}

The device MUST set \field{max_rss_contexts} to at least 1 if it offers VIRTIO_NET_F_RSS_CONTEXT.

Upon reset, the device MUST clear all previously configured RSS contexts.

\drivernormative{\subparagraph}{RSS Context}{Device Types / Network Device / Device Operation / Control Virtqueue / RSS Context}

The driver MUST have negotiated the VIRTIO_NET_F_RSS_CONTEXT feature when issuing the VIRTNET_RSS_CTX_CTRL class commands.

The driver MUST set \field{rss_ctx_id} to between 1 and \field{max_rss_contexts} inclusive.

The driver MUST NOT send the command VIRTIO_NET_CTRL_MQ_VQ_PAIRS_SET when the device has successfully configured at least one RSS context.

\paragraph{Offloads State Configuration}\label{sec:Device Types / Network Device / Device Operation / Control Virtqueue / Offloads State Configuration}

If the VIRTIO_NET_F_CTRL_GUEST_OFFLOADS feature is negotiated, the driver can
send control commands for dynamic offloads state configuration.

\subparagraph{Setting Offloads State}\label{sec:Device Types / Network Device / Device Operation / Control Virtqueue / Offloads State Configuration / Setting Offloads State}

To configure the offloads, the following layout structure and
definitions are used:

\begin{lstlisting}
le64 offloads;

#define VIRTIO_NET_F_GUEST_CSUM       1
#define VIRTIO_NET_F_GUEST_TSO4       7
#define VIRTIO_NET_F_GUEST_TSO6       8
#define VIRTIO_NET_F_GUEST_ECN        9
#define VIRTIO_NET_F_GUEST_UFO        10
#define VIRTIO_NET_F_GUEST_UDP_TUNNEL_GSO  46
#define VIRTIO_NET_F_GUEST_UDP_TUNNEL_GSO_CSUM 47
#define VIRTIO_NET_F_GUEST_USO4       54
#define VIRTIO_NET_F_GUEST_USO6       55

#define VIRTIO_NET_CTRL_GUEST_OFFLOADS       5
 #define VIRTIO_NET_CTRL_GUEST_OFFLOADS_SET   0
\end{lstlisting}

The class VIRTIO_NET_CTRL_GUEST_OFFLOADS has one command:
VIRTIO_NET_CTRL_GUEST_OFFLOADS_SET applies the new offloads configuration.

le64 value passed as command data is a bitmask, bits set define
offloads to be enabled, bits cleared - offloads to be disabled.

There is a corresponding device feature for each offload. Upon feature
negotiation corresponding offload gets enabled to preserve backward
compatibility.

\drivernormative{\subparagraph}{Setting Offloads State}{Device Types / Network Device / Device Operation / Control Virtqueue / Offloads State Configuration / Setting Offloads State}

A driver MUST NOT enable an offload for which the appropriate feature
has not been negotiated.

\subparagraph{Legacy Interface: Setting Offloads State}\label{sec:Device Types / Network Device / Device Operation / Control Virtqueue / Offloads State Configuration / Setting Offloads State / Legacy Interface: Setting Offloads State}
When using the legacy interface, transitional devices and drivers
MUST format \field{offloads}
according to the native endian of the guest rather than
(necessarily when not using the legacy interface) little-endian.


\paragraph{Notifications Coalescing}\label{sec:Device Types / Network Device / Device Operation / Control Virtqueue / Notifications Coalescing}

If the VIRTIO_NET_F_NOTF_COAL feature is negotiated, the driver can
send commands VIRTIO_NET_CTRL_NOTF_COAL_TX_SET and VIRTIO_NET_CTRL_NOTF_COAL_RX_SET
for notification coalescing.

If the VIRTIO_NET_F_VQ_NOTF_COAL feature is negotiated, the driver can
send commands VIRTIO_NET_CTRL_NOTF_COAL_VQ_SET and VIRTIO_NET_CTRL_NOTF_COAL_VQ_GET
for virtqueue notification coalescing.

\begin{lstlisting}
struct virtio_net_ctrl_coal {
    le32 max_packets;
    le32 max_usecs;
};

struct virtio_net_ctrl_coal_vq {
    le16 vq_index;
    le16 reserved;
    struct virtio_net_ctrl_coal coal;
};

#define VIRTIO_NET_CTRL_NOTF_COAL 6
 #define VIRTIO_NET_CTRL_NOTF_COAL_TX_SET  0
 #define VIRTIO_NET_CTRL_NOTF_COAL_RX_SET 1
 #define VIRTIO_NET_CTRL_NOTF_COAL_VQ_SET 2
 #define VIRTIO_NET_CTRL_NOTF_COAL_VQ_GET 3
\end{lstlisting}

Coalescing parameters:
\begin{itemize}
\item \field{vq_index}: The virtqueue index of an enabled transmit or receive virtqueue.
\item \field{max_usecs} for RX: Maximum number of microseconds to delay a RX notification.
\item \field{max_usecs} for TX: Maximum number of microseconds to delay a TX notification.
\item \field{max_packets} for RX: Maximum number of packets to receive before a RX notification.
\item \field{max_packets} for TX: Maximum number of packets to send before a TX notification.
\end{itemize}

\field{reserved} is reserved and it is ignored by the device.

Read/Write attributes for coalescing parameters:
\begin{itemize}
\item For commands VIRTIO_NET_CTRL_NOTF_COAL_TX_SET and VIRTIO_NET_CTRL_NOTF_COAL_RX_SET, the structure virtio_net_ctrl_coal is write-only for the driver.
\item For the command VIRTIO_NET_CTRL_NOTF_COAL_VQ_SET, the structure virtio_net_ctrl_coal_vq is write-only for the driver.
\item For the command VIRTIO_NET_CTRL_NOTF_COAL_VQ_GET, \field{vq_index} and \field{reserved} are write-only
      for the driver, and the structure virtio_net_ctrl_coal is read-only for the driver.
\end{itemize}

The class VIRTIO_NET_CTRL_NOTF_COAL has the following commands:
\begin{enumerate}
\item VIRTIO_NET_CTRL_NOTF_COAL_TX_SET: use the structure virtio_net_ctrl_coal to set the \field{max_usecs} and \field{max_packets} parameters for all transmit virtqueues.
\item VIRTIO_NET_CTRL_NOTF_COAL_RX_SET: use the structure virtio_net_ctrl_coal to set the \field{max_usecs} and \field{max_packets} parameters for all receive virtqueues.
\item VIRTIO_NET_CTRL_NOTF_COAL_VQ_SET: use the structure virtio_net_ctrl_coal_vq to set the \field{max_usecs} and \field{max_packets} parameters
                                        for an enabled transmit/receive virtqueue whose index is \field{vq_index}.
\item VIRTIO_NET_CTRL_NOTF_COAL_VQ_GET: use the structure virtio_net_ctrl_coal_vq to get the \field{max_usecs} and \field{max_packets} parameters
                                        for an enabled transmit/receive virtqueue whose index is \field{vq_index}.
\end{enumerate}

The device may generate notifications more or less frequently than specified by set commands of the VIRTIO_NET_CTRL_NOTF_COAL class.

If coalescing parameters are being set, the device applies the last coalescing parameters set for a
virtqueue, regardless of the command used to set the parameters. Use the following command sequence
with two pairs of virtqueues as an example:
Each of the following commands sets \field{max_usecs} and \field{max_packets} parameters for virtqueues.
\begin{itemize}
\item Command1: VIRTIO_NET_CTRL_NOTF_COAL_RX_SET sets coalescing parameters for virtqueues having index 0 and index 2. Virtqueues having index 1 and index 3 retain their previous parameters.
\item Command2: VIRTIO_NET_CTRL_NOTF_COAL_VQ_SET with \field{vq_index} = 0 sets coalescing parameters for virtqueue having index 0. Virtqueue having index 2 retains the parameters from command1.
\item Command3: VIRTIO_NET_CTRL_NOTF_COAL_VQ_GET with \field{vq_index} = 0, the device responds with coalescing parameters of vq_index 0 set by command2.
\item Command4: VIRTIO_NET_CTRL_NOTF_COAL_VQ_SET with \field{vq_index} = 1 sets coalescing parameters for virtqueue having index 1. Virtqueue having index 3 retains its previous parameters.
\item Command5: VIRTIO_NET_CTRL_NOTF_COAL_TX_SET sets coalescing parameters for virtqueues having index 1 and index 3, and overrides the parameters set by command4.
\item Command6: VIRTIO_NET_CTRL_NOTF_COAL_VQ_GET with \field{vq_index} = 1, the device responds with coalescing parameters of index 1 set by command5.
\end{itemize}

\subparagraph{Operation}\label{sec:Device Types / Network Device / Device Operation / Control Virtqueue / Notifications Coalescing / Operation}

The device sends a used buffer notification once the notification conditions are met and if the notifications are not suppressed as explained in \ref{sec:Basic Facilities of a Virtio Device / Virtqueues / Used Buffer Notification Suppression}.

When the device has non-zero \field{max_usecs} and non-zero \field{max_packets}, it starts counting microseconds and packets upon receiving/sending a packet.
The device counts packets and microseconds for each receive virtqueue and transmit virtqueue separately.
In this case, the notification conditions are met when \field{max_usecs} microseconds elapse, or upon sending/receiving \field{max_packets} packets, whichever happens first.
Afterwards, the device waits for the next packet and starts counting packets and microseconds again.

When the device has \field{max_usecs} = 0 or \field{max_packets} = 0, the notification conditions are met after every packet received/sent.

\subparagraph{RX Example}\label{sec:Device Types / Network Device / Device Operation / Control Virtqueue / Notifications Coalescing / RX Example}

If, for example:
\begin{itemize}
\item \field{max_usecs} = 10.
\item \field{max_packets} = 15.
\end{itemize}
then each receive virtqueue of a device will operate as follows:
\begin{itemize}
\item The device will count packets received on each virtqueue until it accumulates 15, or until 10 microseconds elapsed since the first one was received.
\item If the notifications are not suppressed by the driver, the device will send an used buffer notification, otherwise, the device will not send an used buffer notification as long as the notifications are suppressed.
\end{itemize}

\subparagraph{TX Example}\label{sec:Device Types / Network Device / Device Operation / Control Virtqueue / Notifications Coalescing / TX Example}

If, for example:
\begin{itemize}
\item \field{max_usecs} = 10.
\item \field{max_packets} = 15.
\end{itemize}
then each transmit virtqueue of a device will operate as follows:
\begin{itemize}
\item The device will count packets sent on each virtqueue until it accumulates 15, or until 10 microseconds elapsed since the first one was sent.
\item If the notifications are not suppressed by the driver, the device will send an used buffer notification, otherwise, the device will not send an used buffer notification as long as the notifications are suppressed.
\end{itemize}

\subparagraph{Notifications When Coalescing Parameters Change}\label{sec:Device Types / Network Device / Device Operation / Control Virtqueue / Notifications Coalescing / Notifications When Coalescing Parameters Change}

When the coalescing parameters of a device change, the device needs to check if the new notification conditions are met and send a used buffer notification if so.

For example, \field{max_packets} = 15 for a device with a single transmit virtqueue: if the device sends 10 packets and afterwards receives a
VIRTIO_NET_CTRL_NOTF_COAL_TX_SET command with \field{max_packets} = 8, then the notification condition is immediately considered to be met;
the device needs to immediately send a used buffer notification, if the notifications are not suppressed by the driver.

\drivernormative{\subparagraph}{Notifications Coalescing}{Device Types / Network Device / Device Operation / Control Virtqueue / Notifications Coalescing}

The driver MUST set \field{vq_index} to the virtqueue index of an enabled transmit or receive virtqueue.

The driver MUST have negotiated the VIRTIO_NET_F_NOTF_COAL feature when issuing commands VIRTIO_NET_CTRL_NOTF_COAL_TX_SET and VIRTIO_NET_CTRL_NOTF_COAL_RX_SET.

The driver MUST have negotiated the VIRTIO_NET_F_VQ_NOTF_COAL feature when issuing commands VIRTIO_NET_CTRL_NOTF_COAL_VQ_SET and VIRTIO_NET_CTRL_NOTF_COAL_VQ_GET.

The driver MUST ignore the values of coalescing parameters received from the VIRTIO_NET_CTRL_NOTF_COAL_VQ_GET command if the device responds with VIRTIO_NET_ERR.

\devicenormative{\subparagraph}{Notifications Coalescing}{Device Types / Network Device / Device Operation / Control Virtqueue / Notifications Coalescing}

The device MUST ignore \field{reserved}.

The device SHOULD respond to VIRTIO_NET_CTRL_NOTF_COAL_TX_SET and VIRTIO_NET_CTRL_NOTF_COAL_RX_SET commands with VIRTIO_NET_ERR if it was not able to change the parameters.

The device MUST respond to the VIRTIO_NET_CTRL_NOTF_COAL_VQ_SET command with VIRTIO_NET_ERR if it was not able to change the parameters.

The device MUST respond to VIRTIO_NET_CTRL_NOTF_COAL_VQ_SET and VIRTIO_NET_CTRL_NOTF_COAL_VQ_GET commands with
VIRTIO_NET_ERR if the designated virtqueue is not an enabled transmit or receive virtqueue.

Upon disabling and re-enabling a transmit virtqueue, the device MUST set the coalescing parameters of the virtqueue
to those configured through the VIRTIO_NET_CTRL_NOTF_COAL_TX_SET command, or, if the driver did not set any TX coalescing parameters, to 0.

Upon disabling and re-enabling a receive virtqueue, the device MUST set the coalescing parameters of the virtqueue
to those configured through the VIRTIO_NET_CTRL_NOTF_COAL_RX_SET command, or, if the driver did not set any RX coalescing parameters, to 0.

The behavior of the device in response to set commands of the VIRTIO_NET_CTRL_NOTF_COAL class is best-effort:
the device MAY generate notifications more or less frequently than specified.

A device SHOULD NOT send used buffer notifications to the driver if the notifications are suppressed, even if the notification conditions are met.

Upon reset, a device MUST initialize all coalescing parameters to 0.

\paragraph{Device Statistics}\label{sec:Device Types / Network Device / Device Operation / Control Virtqueue / Device Statistics}

If the VIRTIO_NET_F_DEVICE_STATS feature is negotiated, the driver can obtain
device statistics from the device by using the following command.

Different types of virtqueues have different statistics. The statistics of the
receiveq are different from those of the transmitq.

The statistics of a certain type of virtqueue are also divided into multiple types
because different types require different features. This enables the expansion
of new statistics.

In one command, the driver can obtain the statistics of one or multiple virtqueues.
Additionally, the driver can obtain multiple type statistics of each virtqueue.

\subparagraph{Query Statistic Capabilities}\label{sec:Device Types / Network Device / Device Operation / Control Virtqueue / Device Statistics / Query Statistic Capabilities}

\begin{lstlisting}
#define VIRTIO_NET_CTRL_STATS         8
#define VIRTIO_NET_CTRL_STATS_QUERY   0
#define VIRTIO_NET_CTRL_STATS_GET     1

struct virtio_net_stats_capabilities {

#define VIRTIO_NET_STATS_TYPE_CVQ       (1 << 32)

#define VIRTIO_NET_STATS_TYPE_RX_BASIC  (1 << 0)
#define VIRTIO_NET_STATS_TYPE_RX_CSUM   (1 << 1)
#define VIRTIO_NET_STATS_TYPE_RX_GSO    (1 << 2)
#define VIRTIO_NET_STATS_TYPE_RX_SPEED  (1 << 3)

#define VIRTIO_NET_STATS_TYPE_TX_BASIC  (1 << 16)
#define VIRTIO_NET_STATS_TYPE_TX_CSUM   (1 << 17)
#define VIRTIO_NET_STATS_TYPE_TX_GSO    (1 << 18)
#define VIRTIO_NET_STATS_TYPE_TX_SPEED  (1 << 19)

    le64 supported_stats_types[1];
}
\end{lstlisting}

To obtain device statistic capability, use the VIRTIO_NET_CTRL_STATS_QUERY
command. When the command completes successfully, \field{command-specific-result}
is in the format of \field{struct virtio_net_stats_capabilities}.

\subparagraph{Get Statistics}\label{sec:Device Types / Network Device / Device Operation / Control Virtqueue / Device Statistics / Get Statistics}

\begin{lstlisting}
struct virtio_net_ctrl_queue_stats {
       struct {
           le16 vq_index;
           le16 reserved[3];
           le64 types_bitmap[1];
       } stats[];
};

struct virtio_net_stats_reply_hdr {
#define VIRTIO_NET_STATS_TYPE_REPLY_CVQ       32

#define VIRTIO_NET_STATS_TYPE_REPLY_RX_BASIC  0
#define VIRTIO_NET_STATS_TYPE_REPLY_RX_CSUM   1
#define VIRTIO_NET_STATS_TYPE_REPLY_RX_GSO    2
#define VIRTIO_NET_STATS_TYPE_REPLY_RX_SPEED  3

#define VIRTIO_NET_STATS_TYPE_REPLY_TX_BASIC  16
#define VIRTIO_NET_STATS_TYPE_REPLY_TX_CSUM   17
#define VIRTIO_NET_STATS_TYPE_REPLY_TX_GSO    18
#define VIRTIO_NET_STATS_TYPE_REPLY_TX_SPEED  19
    u8 type;
    u8 reserved;
    le16 vq_index;
    le16 reserved1;
    le16 size;
}
\end{lstlisting}

To obtain device statistics, use the VIRTIO_NET_CTRL_STATS_GET command with the
\field{command-specific-data} which is in the format of
\field{struct virtio_net_ctrl_queue_stats}. When the command completes
successfully, \field{command-specific-result} contains multiple statistic
results, each statistic result has the \field{struct virtio_net_stats_reply_hdr}
as the header.

The fields of the \field{struct virtio_net_ctrl_queue_stats}:
\begin{description}
    \item [vq_index]
        The index of the virtqueue to obtain the statistics.

    \item [types_bitmap]
        This is a bitmask of the types of statistics to be obtained. Therefore, a
        \field{stats} inside \field{struct virtio_net_ctrl_queue_stats} may
        indicate multiple statistic replies for the virtqueue.
\end{description}

The fields of the \field{struct virtio_net_stats_reply_hdr}:
\begin{description}
    \item [type]
        The type of the reply statistic.

    \item [vq_index]
        The virtqueue index of the reply statistic.

    \item [size]
        The number of bytes for the statistics entry including size of \field{struct virtio_net_stats_reply_hdr}.

\end{description}

\subparagraph{Controlq Statistics}\label{sec:Device Types / Network Device / Device Operation / Control Virtqueue / Device Statistics / Controlq Statistics}

The structure corresponding to the controlq statistics is
\field{struct virtio_net_stats_cvq}. The corresponding type is
VIRTIO_NET_STATS_TYPE_CVQ. This is for the controlq.

\begin{lstlisting}
struct virtio_net_stats_cvq {
    struct virtio_net_stats_reply_hdr hdr;

    le64 command_num;
    le64 ok_num;
};
\end{lstlisting}

\begin{description}
    \item [command_num]
        The number of commands received by the device including the current command.

    \item [ok_num]
        The number of commands completed successfully by the device including the current command.
\end{description}


\subparagraph{Receiveq Basic Statistics}\label{sec:Device Types / Network Device / Device Operation / Control Virtqueue / Device Statistics / Receiveq Basic Statistics}

The structure corresponding to the receiveq basic statistics is
\field{struct virtio_net_stats_rx_basic}. The corresponding type is
VIRTIO_NET_STATS_TYPE_RX_BASIC. This is for the receiveq.

Receiveq basic statistics do not require any feature. As long as the device supports
VIRTIO_NET_F_DEVICE_STATS, the following are the receiveq basic statistics.

\begin{lstlisting}
struct virtio_net_stats_rx_basic {
    struct virtio_net_stats_reply_hdr hdr;

    le64 rx_notifications;

    le64 rx_packets;
    le64 rx_bytes;

    le64 rx_interrupts;

    le64 rx_drops;
    le64 rx_drop_overruns;
};
\end{lstlisting}

The packets described below were all presented on the specified virtqueue.
\begin{description}
    \item [rx_notifications]
        The number of driver notifications received by the device for this
        receiveq.

    \item [rx_packets]
        This is the number of packets passed to the driver by the device.

    \item [rx_bytes]
        This is the bytes of packets passed to the driver by the device.

    \item [rx_interrupts]
        The number of interrupts generated by the device for this receiveq.

    \item [rx_drops]
        This is the number of packets dropped by the device. The count includes
        all types of packets dropped by the device.

    \item [rx_drop_overruns]
        This is the number of packets dropped by the device when no more
        descriptors were available.

\end{description}

\subparagraph{Transmitq Basic Statistics}\label{sec:Device Types / Network Device / Device Operation / Control Virtqueue / Device Statistics / Transmitq Basic Statistics}

The structure corresponding to the transmitq basic statistics is
\field{struct virtio_net_stats_tx_basic}. The corresponding type is
VIRTIO_NET_STATS_TYPE_TX_BASIC. This is for the transmitq.

Transmitq basic statistics do not require any feature. As long as the device supports
VIRTIO_NET_F_DEVICE_STATS, the following are the transmitq basic statistics.

\begin{lstlisting}
struct virtio_net_stats_tx_basic {
    struct virtio_net_stats_reply_hdr hdr;

    le64 tx_notifications;

    le64 tx_packets;
    le64 tx_bytes;

    le64 tx_interrupts;

    le64 tx_drops;
    le64 tx_drop_malformed;
};
\end{lstlisting}

The packets described below are all for a specific virtqueue.
\begin{description}
    \item [tx_notifications]
        The number of driver notifications received by the device for this
        transmitq.

    \item [tx_packets]
        This is the number of packets sent by the device (not the packets
        got from the driver).

    \item [tx_bytes]
        This is the number of bytes sent by the device for all the sent packets
        (not the bytes sent got from the driver).

    \item [tx_interrupts]
        The number of interrupts generated by the device for this transmitq.

    \item [tx_drops]
        The number of packets dropped by the device. The count includes all
        types of packets dropped by the device.

    \item [tx_drop_malformed]
        The number of packets dropped by the device, when the descriptors are
        malformed. For example, the buffer is too short.
\end{description}

\subparagraph{Receiveq CSUM Statistics}\label{sec:Device Types / Network Device / Device Operation / Control Virtqueue / Device Statistics / Receiveq CSUM Statistics}

The structure corresponding to the receiveq checksum statistics is
\field{struct virtio_net_stats_rx_csum}. The corresponding type is
VIRTIO_NET_STATS_TYPE_RX_CSUM. This is for the receiveq.

Only after the VIRTIO_NET_F_GUEST_CSUM is negotiated, the receiveq checksum
statistics can be obtained.

\begin{lstlisting}
struct virtio_net_stats_rx_csum {
    struct virtio_net_stats_reply_hdr hdr;

    le64 rx_csum_valid;
    le64 rx_needs_csum;
    le64 rx_csum_none;
    le64 rx_csum_bad;
};
\end{lstlisting}

The packets described below were all presented on the specified virtqueue.
\begin{description}
    \item [rx_csum_valid]
        The number of packets with VIRTIO_NET_HDR_F_DATA_VALID.

    \item [rx_needs_csum]
        The number of packets with VIRTIO_NET_HDR_F_NEEDS_CSUM.

    \item [rx_csum_none]
        The number of packets without hardware checksum. The packet here refers
        to the non-TCP/UDP packet that the device cannot recognize.

    \item [rx_csum_bad]
        The number of packets with checksum mismatch.

\end{description}

\subparagraph{Transmitq CSUM Statistics}\label{sec:Device Types / Network Device / Device Operation / Control Virtqueue / Device Statistics / Transmitq CSUM Statistics}

The structure corresponding to the transmitq checksum statistics is
\field{struct virtio_net_stats_tx_csum}. The corresponding type is
VIRTIO_NET_STATS_TYPE_TX_CSUM. This is for the transmitq.

Only after the VIRTIO_NET_F_CSUM is negotiated, the transmitq checksum
statistics can be obtained.

The following are the transmitq checksum statistics:

\begin{lstlisting}
struct virtio_net_stats_tx_csum {
    struct virtio_net_stats_reply_hdr hdr;

    le64 tx_csum_none;
    le64 tx_needs_csum;
};
\end{lstlisting}

The packets described below are all for a specific virtqueue.
\begin{description}
    \item [tx_csum_none]
        The number of packets which do not require hardware checksum.

    \item [tx_needs_csum]
        The number of packets which require checksum calculation by the device.

\end{description}

\subparagraph{Receiveq GSO Statistics}\label{sec:Device Types / Network Device / Device Operation / Control Virtqueue / Device Statistics / Receiveq GSO Statistics}

The structure corresponding to the receivq GSO statistics is
\field{struct virtio_net_stats_rx_gso}. The corresponding type is
VIRTIO_NET_STATS_TYPE_RX_GSO. This is for the receiveq.

If one or more of the VIRTIO_NET_F_GUEST_TSO4, VIRTIO_NET_F_GUEST_TSO6
have been negotiated, the receiveq GSO statistics can be obtained.

GSO packets refer to packets passed by the device to the driver where
\field{gso_type} is not VIRTIO_NET_HDR_GSO_NONE.

\begin{lstlisting}
struct virtio_net_stats_rx_gso {
    struct virtio_net_stats_reply_hdr hdr;

    le64 rx_gso_packets;
    le64 rx_gso_bytes;
    le64 rx_gso_packets_coalesced;
    le64 rx_gso_bytes_coalesced;
};
\end{lstlisting}

The packets described below were all presented on the specified virtqueue.
\begin{description}
    \item [rx_gso_packets]
        The number of the GSO packets received by the device.

    \item [rx_gso_bytes]
        The bytes of the GSO packets received by the device.
        This includes the header size of the GSO packet.

    \item [rx_gso_packets_coalesced]
        The number of the GSO packets coalesced by the device.

    \item [rx_gso_bytes_coalesced]
        The bytes of the GSO packets coalesced by the device.
        This includes the header size of the GSO packet.
\end{description}

\subparagraph{Transmitq GSO Statistics}\label{sec:Device Types / Network Device / Device Operation / Control Virtqueue / Device Statistics / Transmitq GSO Statistics}

The structure corresponding to the transmitq GSO statistics is
\field{struct virtio_net_stats_tx_gso}. The corresponding type is
VIRTIO_NET_STATS_TYPE_TX_GSO. This is for the transmitq.

If one or more of the VIRTIO_NET_F_HOST_TSO4, VIRTIO_NET_F_HOST_TSO6,
VIRTIO_NET_F_HOST_USO options have been negotiated, the transmitq GSO statistics
can be obtained.

GSO packets refer to packets passed by the driver to the device where
\field{gso_type} is not VIRTIO_NET_HDR_GSO_NONE.
See more \ref{sec:Device Types / Network Device / Device Operation / Packet
Transmission}.

\begin{lstlisting}
struct virtio_net_stats_tx_gso {
    struct virtio_net_stats_reply_hdr hdr;

    le64 tx_gso_packets;
    le64 tx_gso_bytes;
    le64 tx_gso_segments;
    le64 tx_gso_segments_bytes;
    le64 tx_gso_packets_noseg;
    le64 tx_gso_bytes_noseg;
};
\end{lstlisting}

The packets described below are all for a specific virtqueue.
\begin{description}
    \item [tx_gso_packets]
        The number of the GSO packets sent by the device.

    \item [tx_gso_bytes]
        The bytes of the GSO packets sent by the device.

    \item [tx_gso_segments]
        The number of segments prepared from GSO packets.

    \item [tx_gso_segments_bytes]
        The bytes of segments prepared from GSO packets.

    \item [tx_gso_packets_noseg]
        The number of the GSO packets without segmentation.

    \item [tx_gso_bytes_noseg]
        The bytes of the GSO packets without segmentation.

\end{description}

\subparagraph{Receiveq Speed Statistics}\label{sec:Device Types / Network Device / Device Operation / Control Virtqueue / Device Statistics / Receiveq Speed Statistics}

The structure corresponding to the receiveq speed statistics is
\field{struct virtio_net_stats_rx_speed}. The corresponding type is
VIRTIO_NET_STATS_TYPE_RX_SPEED. This is for the receiveq.

The device has the allowance for the speed. If VIRTIO_NET_F_SPEED_DUPLEX has
been negotiated, the driver can get this by \field{speed}. When the received
packets bitrate exceeds the \field{speed}, some packets may be dropped by the
device.

\begin{lstlisting}
struct virtio_net_stats_rx_speed {
    struct virtio_net_stats_reply_hdr hdr;

    le64 rx_packets_allowance_exceeded;
    le64 rx_bytes_allowance_exceeded;
};
\end{lstlisting}

The packets described below were all presented on the specified virtqueue.
\begin{description}
    \item [rx_packets_allowance_exceeded]
        The number of the packets dropped by the device due to the received
        packets bitrate exceeding the \field{speed}.

    \item [rx_bytes_allowance_exceeded]
        The bytes of the packets dropped by the device due to the received
        packets bitrate exceeding the \field{speed}.

\end{description}

\subparagraph{Transmitq Speed Statistics}\label{sec:Device Types / Network Device / Device Operation / Control Virtqueue / Device Statistics / Transmitq Speed Statistics}

The structure corresponding to the transmitq speed statistics is
\field{struct virtio_net_stats_tx_speed}. The corresponding type is
VIRTIO_NET_STATS_TYPE_TX_SPEED. This is for the transmitq.

The device has the allowance for the speed. If VIRTIO_NET_F_SPEED_DUPLEX has
been negotiated, the driver can get this by \field{speed}. When the transmit
packets bitrate exceeds the \field{speed}, some packets may be dropped by the
device.

\begin{lstlisting}
struct virtio_net_stats_tx_speed {
    struct virtio_net_stats_reply_hdr hdr;

    le64 tx_packets_allowance_exceeded;
    le64 tx_bytes_allowance_exceeded;
};
\end{lstlisting}

The packets described below were all presented on the specified virtqueue.
\begin{description}
    \item [tx_packets_allowance_exceeded]
        The number of the packets dropped by the device due to the transmit packets
        bitrate exceeding the \field{speed}.

    \item [tx_bytes_allowance_exceeded]
        The bytes of the packets dropped by the device due to the transmit packets
        bitrate exceeding the \field{speed}.

\end{description}

\devicenormative{\subparagraph}{Device Statistics}{Device Types / Network Device / Device Operation / Control Virtqueue / Device Statistics}

When the VIRTIO_NET_F_DEVICE_STATS feature is negotiated, the device MUST reply
to the command VIRTIO_NET_CTRL_STATS_QUERY with the
\field{struct virtio_net_stats_capabilities}. \field{supported_stats_types}
includes all the statistic types supported by the device.

If \field{struct virtio_net_ctrl_queue_stats} is incorrect (such as the
following), the device MUST set \field{ack} to VIRTIO_NET_ERR. Even if there is
only one error, the device MUST fail the entire command.
\begin{itemize}
    \item \field{vq_index} exceeds the queue range.
    \item \field{types_bitmap} contains unknown types.
    \item One or more of the bits present in \field{types_bitmap} is not valid
        for the specified virtqueue.
    \item The feature corresponding to the specified \field{types_bitmap} was
        not negotiated.
\end{itemize}

The device MUST set the actual size of the bytes occupied by the reply to the
\field{size} of the \field{hdr}. And the device MUST set the \field{type} and
the \field{vq_index} of the statistic header.

The \field{command-specific-result} buffer allocated by the driver may be
smaller or bigger than all the statistics specified by
\field{struct virtio_net_ctrl_queue_stats}. The device MUST fill up only upto
the valid bytes.

The statistics counter replied by the device MUST wrap around to zero by the
device on the overflow.

\drivernormative{\subparagraph}{Device Statistics}{Device Types / Network Device / Device Operation / Control Virtqueue / Device Statistics}

The types contained in the \field{types_bitmap} MUST be queried from the device
via command VIRTIO_NET_CTRL_STATS_QUERY.

\field{types_bitmap} in \field{struct virtio_net_ctrl_queue_stats} MUST be valid to the
vq specified by \field{vq_index}.

The \field{command-specific-result} buffer allocated by the driver MUST have
enough capacity to store all the statistics reply headers defined in
\field{struct virtio_net_ctrl_queue_stats}. If the
\field{command-specific-result} buffer is fully utilized by the device but some
replies are missed, it is possible that some statistics may exceed the capacity
of the driver's records. In such cases, the driver should allocate additional
space for the \field{command-specific-result} buffer.

\subsubsection{Flow filter}\label{sec:Device Types / Network Device / Device Operation / Flow filter}

A network device can support one or more flow filter rules. Each flow filter rule
is applied by matching a packet and then taking an action, such as directing the packet
to a specific receiveq or dropping the packet. An example of a match is
matching on specific source and destination IP addresses.

A flow filter rule is a device resource object that consists of a key,
a processing priority, and an action to either direct a packet to a
receive queue or drop the packet.

Each rule uses a classifier. The key is matched against the packet using
a classifier, defining which fields in the packet are matched.
A classifier resource object consists of one or more field selectors, each with
a type that specifies the header fields to be matched against, and a mask.
The mask can match whole fields or parts of a field in a header. Each
rule resource object depends on the classifier resource object.

When a packet is received, relevant fields are extracted
(in the same way) from both the packet and the key according to the
classifier. The resulting field contents are then compared -
if they are identical the rule action is taken, if they are not, the rule is ignored.

Multiple flow filter rules are part of a group. The rule resource object
depends on the group. Each rule within a
group has a rule priority, and each group also has a group priority. For a
packet, a group with the highest priority is selected first. Within a group,
rules are applied from highest to lowest priority, until one of the rules
matches the packet and an action is taken. If all the rules within a group
are ignored, the group with the next highest priority is selected, and so on.

The device and the driver indicates flow filter resource limits using the capability
\ref{par:Device Types / Network Device / Device Operation / Flow filter / Device and driver capabilities / VIRTIO-NET-FF-RESOURCE-CAP} specifying the limits on the number of flow filter rule,
group and classifier resource objects. The capability \ref{par:Device Types / Network Device / Device Operation / Flow filter / Device and driver capabilities / VIRTIO-NET-FF-SELECTOR-CAP} specifies which selectors the device supports.
The driver indicates the selectors it is using by setting the flow
filter selector capability, prior to adding any resource objects.

The capability \ref{par:Device Types / Network Device / Device Operation / Flow filter / Device and driver capabilities / VIRTIO-NET-FF-ACTION-CAP} specifies which actions the device supports.

The driver controls the flow filter rule, classifier and group resource objects using
administration commands described in
\ref{sec:Basic Facilities of a Virtio Device / Device groups / Group administration commands / Device resource objects}.

\paragraph{Packet processing order}\label{sec:sec:Device Types / Network Device / Device Operation / Flow filter / Packet processing order}

Note that flow filter rules are applied after MAC/VLAN filtering. Flow filter
rules take precedence over steering: if a flow filter rule results in an action,
the steering configuration does not apply. The steering configuration only applies
to packets for which no flow filter rule action was performed. For example,
incoming packets can be processed in the following order:

\begin{itemize}
\item apply steering configuration received using control virtqueue commands
      VIRTIO_NET_CTRL_RX, VIRTIO_NET_CTRL_MAC and VIRTIO_NET_CTRL_VLAN.
\item apply flow filter rules if any.
\item if no filter rule applied, apply steering configuration received using command
      VIRTIO_NET_CTRL_MQ_RSS_CONFIG or as per automatic receive steering.
\end{itemize}

Some incoming packet processing examples:
\begin{itemize}
\item If the packet is dropped by the flow filter rule, RSS
      steering is ignored for the packet.
\item If the packet is directed to a specific receiveq using flow filter rule,
      the RSS steering is ignored for the packet.
\item If a packet is dropped due to the VIRTIO_NET_CTRL_MAC configuration,
      both flow filter rules and the RSS steering are ignored for the packet.
\item If a packet does not match any flow filter rules,
      the RSS steering is used to select the receiveq for the packet (if enabled).
\item If there are two flow filter groups configured as group_A and group_B
      with respective group priorities as 4, and 5; flow filter rules of
      group_B are applied first having highest group priority, if there is a match,
      the flow filter rules of group_A are ignored; if there is no match for
      the flow filter rules in group_B, the flow filter rules of next level group_A are applied.
\end{itemize}

\paragraph{Device and driver capabilities}
\label{par:Device Types / Network Device / Device Operation / Flow filter / Device and driver capabilities}

\subparagraph{VIRTIO_NET_FF_RESOURCE_CAP}
\label{par:Device Types / Network Device / Device Operation / Flow filter / Device and driver capabilities / VIRTIO-NET-FF-RESOURCE-CAP}

The capability VIRTIO_NET_FF_RESOURCE_CAP indicates the flow filter resource limits.
\field{cap_specific_data} is in the format
\field{struct virtio_net_ff_cap_data}.

\begin{lstlisting}
struct virtio_net_ff_cap_data {
        le32 groups_limit;
        le32 selectors_limit;
        le32 rules_limit;
        le32 rules_per_group_limit;
        u8 last_rule_priority;
        u8 selectors_per_classifier_limit;
};
\end{lstlisting}

\field{groups_limit}, and \field{selectors_limit} represent the maximum
number of flow filter groups and selectors, respectively, that the driver can create.
 \field{rules_limit} is the maximum number of
flow fiilter rules that the driver can create across all the groups.
\field{rules_per_group_limit} is the maximum number of flow filter rules that the driver
can create for each flow filter group.

\field{last_rule_priority} is the highest priority that can be assigned to a
flow filter rule.

\field{selectors_per_classifier_limit} is the maximum number of selectors
that a classifier can have.

\subparagraph{VIRTIO_NET_FF_SELECTOR_CAP}
\label{par:Device Types / Network Device / Device Operation / Flow filter / Device and driver capabilities / VIRTIO-NET-FF-SELECTOR-CAP}

The capability VIRTIO_NET_FF_SELECTOR_CAP lists the supported selectors and the
supported packet header fields for each selector.
\field{cap_specific_data} is in the format \field{struct virtio_net_ff_cap_mask_data}.

\begin{lstlisting}[label={lst:Device Types / Network Device / Device Operation / Flow filter / Device and driver capabilities / VIRTIO-NET-FF-SELECTOR-CAP / virtio-net-ff-selector}]
struct virtio_net_ff_selector {
        u8 type;
        u8 flags;
        u8 reserved[2];
        u8 length;
        u8 reserved1[3];
        u8 mask[];
};

struct virtio_net_ff_cap_mask_data {
        u8 count;
        u8 reserved[7];
        struct virtio_net_ff_selector selectors[];
};

#define VIRTIO_NET_FF_MASK_F_PARTIAL_MASK (1 << 0)
\end{lstlisting}

\field{count} indicates number of valid entries in the \field{selectors} array.
\field{selectors[]} is an array of supported selectors. Within each array entry:
\field{type} specifies the type of the packet header, as defined in table
\ref{table:Device Types / Network Device / Device Operation / Flow filter / Device and driver capabilities / VIRTIO-NET-FF-SELECTOR-CAP / flow filter selector types}. \field{mask} specifies which fields of the
packet header can be matched in a flow filter rule.

Each \field{type} is also listed in table
\ref{table:Device Types / Network Device / Device Operation / Flow filter / Device and driver capabilities / VIRTIO-NET-FF-SELECTOR-CAP / flow filter selector types}. \field{mask} is a byte array
in network byte order. For example, when \field{type} is VIRTIO_NET_FF_MASK_TYPE_IPV6,
the \field{mask} is in the format \hyperref[intro:IPv6-Header-Format]{IPv6 Header Format}.

If partial masking is not set, then all bits in each field have to be either all 0s
to ignore this field or all 1s to match on this field. If partial masking is set,
then any combination of bits can bit set to match on these bits.
For example, when a selector \field{type} is VIRTIO_NET_FF_MASK_TYPE_ETH, if
\field{mask[0-12]} are zero and \field{mask[13-14]} are 0xff (all 1s), it
indicates that matching is only supported for \field{EtherType} of
\field{Ethernet MAC frame}, matching is not supported for
\field{Destination Address} and \field{Source Address}.

The entries in the array \field{selectors} are ordered by
\field{type}, with each \field{type} value only appearing once.

\field{length} is the length of a dynamic array \field{mask} in bytes.
\field{reserved} and \field{reserved1} are reserved and set to zero.

\begin{table}[H]
\caption{Flow filter selector types}
\label{table:Device Types / Network Device / Device Operation / Flow filter / Device and driver capabilities / VIRTIO-NET-FF-SELECTOR-CAP / flow filter selector types}
\begin{tabularx}{\textwidth}{ |l|X|X| }
\hline
Type & Name & Description \\
\hline \hline
0x0 & - & Reserved \\
\hline
0x1 & VIRTIO_NET_FF_MASK_TYPE_ETH & 14 bytes of frame header starting from destination address described in \hyperref[intro:IEEE 802.3-2022]{IEEE 802.3-2022} \\
\hline
0x2 & VIRTIO_NET_FF_MASK_TYPE_IPV4 & 20 bytes of \hyperref[intro:Internet-Header-Format]{IPv4: Internet Header Format} \\
\hline
0x3 & VIRTIO_NET_FF_MASK_TYPE_IPV6 & 40 bytes of \hyperref[intro:IPv6-Header-Format]{IPv6 Header Format} \\
\hline
0x4 & VIRTIO_NET_FF_MASK_TYPE_TCP & 20 bytes of \hyperref[intro:TCP-Header-Format]{TCP Header Format} \\
\hline
0x5 & VIRTIO_NET_FF_MASK_TYPE_UDP & 8 bytes of UDP header described in \hyperref[intro:UDP]{UDP} \\
\hline
0x6 - 0xFF & & Reserved for future \\
\hline
\end{tabularx}
\end{table}

When VIRTIO_NET_FF_MASK_F_PARTIAL_MASK (bit 0) is set, it indicates that
partial masking is supported for all the fields of the selector identified by \field{type}.

For the selector \field{type} VIRTIO_NET_FF_MASK_TYPE_IPV4, if a partial mask is unsupported,
then matching on an individual bit of \field{Flags} in the
\field{IPv4: Internet Header Format} is unsupported. \field{Flags} has to match as a whole
if it is supported.

For the selector \field{type} VIRTIO_NET_FF_MASK_TYPE_IPV4, \field{mask} includes fields
up to the \field{Destination Address}; that is, \field{Options} and
\field{Padding} are excluded.

For the selector \field{type} VIRTIO_NET_FF_MASK_TYPE_IPV6, the \field{Next Header} field
of the \field{mask} corresponds to the \field{Next Header} in the packet
when \field{IPv6 Extension Headers} are not present. When the packet includes
one or more \field{IPv6 Extension Headers}, the \field{Next Header} field of
the \field{mask} corresponds to the \field{Next Header} of the last
\field{IPv6 Extension Header} in the packet.

For the selector \field{type} VIRTIO_NET_FF_MASK_TYPE_TCP, \field{Control bits}
are treated as individual fields for matching; that is, matching individual
\field{Control bits} does not depend on the partial mask support.

\subparagraph{VIRTIO_NET_FF_ACTION_CAP}
\label{par:Device Types / Network Device / Device Operation / Flow filter / Device and driver capabilities / VIRTIO-NET-FF-ACTION-CAP}

The capability VIRTIO_NET_FF_ACTION_CAP lists the supported actions in a rule.
\field{cap_specific_data} is in the format \field{struct virtio_net_ff_cap_actions}.

\begin{lstlisting}
struct virtio_net_ff_actions {
        u8 count;
        u8 reserved[7];
        u8 actions[];
};
\end{lstlisting}

\field{actions} is an array listing all possible actions.
The entries in the array are ordered from the smallest to the largest,
with each supported value appearing exactly once. Each entry can have the
following values:

\begin{table}[H]
\caption{Flow filter rule actions}
\label{table:Device Types / Network Device / Device Operation / Flow filter / Device and driver capabilities / VIRTIO-NET-FF-ACTION-CAP / flow filter rule actions}
\begin{tabularx}{\textwidth}{ |l|X|X| }
\hline
Action & Name & Description \\
\hline \hline
0x0 & - & reserved \\
\hline
0x1 & VIRTIO_NET_FF_ACTION_DROP & Matching packet will be dropped by the device \\
\hline
0x2 & VIRTIO_NET_FF_ACTION_DIRECT_RX_VQ & Matching packet will be directed to a receive queue \\
\hline
0x3 - 0xFF & & Reserved for future \\
\hline
\end{tabularx}
\end{table}

\paragraph{Resource objects}
\label{par:Device Types / Network Device / Device Operation / Flow filter / Resource objects}

\subparagraph{VIRTIO_NET_RESOURCE_OBJ_FF_GROUP}\label{par:Device Types / Network Device / Device Operation / Flow filter / Resource objects / VIRTIO-NET-RESOURCE-OBJ-FF-GROUP}

A flow filter group contains between 0 and \field{rules_limit} rules, as specified by the
capability VIRTIO_NET_FF_RESOURCE_CAP. For the flow filter group object both
\field{resource_obj_specific_data} and
\field{resource_obj_specific_result} are in the format
\field{struct virtio_net_resource_obj_ff_group}.

\begin{lstlisting}
struct virtio_net_resource_obj_ff_group {
        le16 group_priority;
};
\end{lstlisting}

\field{group_priority} specifies the priority for the group. Each group has a
distinct priority. For each incoming packet, the device tries to apply rules
from groups from higher \field{group_priority} value to lower, until either a
rule matches the packet or all groups have been tried.

\subparagraph{VIRTIO_NET_RESOURCE_OBJ_FF_CLASSIFIER}\label{par:Device Types / Network Device / Device Operation / Flow filter / Resource objects / VIRTIO-NET-RESOURCE-OBJ-FF-CLASSIFIER}

A classifier is used to match a flow filter key against a packet. The
classifier defines the desired packet fields to match, and is represented by
the VIRTIO_NET_RESOURCE_OBJ_FF_CLASSIFIER device resource object.

For the flow filter classifier object both \field{resource_obj_specific_data} and
\field{resource_obj_specific_result} are in the format
\field{struct virtio_net_resource_obj_ff_classifier}.

\begin{lstlisting}
struct virtio_net_resource_obj_ff_classifier {
        u8 count;
        u8 reserved[7];
        struct virtio_net_ff_selector selectors[];
};
\end{lstlisting}

A classifier is an array of \field{selectors}. The number of selectors in the
array is indicated by \field{count}. The selector has a type that specifies
the header fields to be matched against, and a mask.
See \ref{lst:Device Types / Network Device / Device Operation / Flow filter / Device and driver capabilities / VIRTIO-NET-FF-SELECTOR-CAP / virtio-net-ff-selector}
for details about selectors.

The first selector is always VIRTIO_NET_FF_MASK_TYPE_ETH. When there are multiple
selectors, a second selector can be either VIRTIO_NET_FF_MASK_TYPE_IPV4
or VIRTIO_NET_FF_MASK_TYPE_IPV6. If the third selector exists, the third
selector can be either VIRTIO_NET_FF_MASK_TYPE_UDP or VIRTIO_NET_FF_MASK_TYPE_TCP.
For example, to match a Ethernet IPv6 UDP packet,
\field{selectors[0].type} is set to VIRTIO_NET_FF_MASK_TYPE_ETH, \field{selectors[1].type}
is set to VIRTIO_NET_FF_MASK_TYPE_IPV6 and \field{selectors[2].type} is
set to VIRTIO_NET_FF_MASK_TYPE_UDP; accordingly, \field{selectors[0].mask[0-13]} is
for Ethernet header fields, \field{selectors[1].mask[0-39]} is set for IPV6 header
and \field{selectors[2].mask[0-7]} is set for UDP header.

When there are multiple selectors, the type of the (N+1)\textsuperscript{th} selector
affects the mask of the (N)\textsuperscript{th} selector. If
\field{count} is 2 or more, all the mask bits within \field{selectors[0]}
corresponding to \field{EtherType} of an Ethernet header are set.

If \field{count} is more than 2:
\begin{itemize}
\item if \field{selector[1].type} is, VIRTIO_NET_FF_MASK_TYPE_IPV4, then, all the mask bits within
\field{selector[1]} for \field{Protocol} is set.
\item if \field{selector[1].type} is, VIRTIO_NET_FF_MASK_TYPE_IPV6, then, all the mask bits within
\field{selector[1]} for \field{Next Header} is set.
\end{itemize}

If for a given packet header field, a subset of bits of a field is to be matched,
and if the partial mask is supported, the flow filter
mask object can specify a mask which has fewer bits set than the packet header
field size. For example, a partial mask for the Ethernet header source mac
address can be of 1-bit for multicast detection instead of 48-bits.

\subparagraph{VIRTIO_NET_RESOURCE_OBJ_FF_RULE}\label{par:Device Types / Network Device / Device Operation / Flow filter / Resource objects / VIRTIO-NET-RESOURCE-OBJ-FF-RULE}

Each flow filter rule resource object comprises a key, a priority, and an action.
For the flow filter rule object,
\field{resource_obj_specific_data} and
\field{resource_obj_specific_result} are in the format
\field{struct virtio_net_resource_obj_ff_rule}.

\begin{lstlisting}
struct virtio_net_resource_obj_ff_rule {
        le32 group_id;
        le32 classifier_id;
        u8 rule_priority;
        u8 key_length; /* length of key in bytes */
        u8 action;
        u8 reserved;
        le16 vq_index;
        u8 reserved1[2];
        u8 keys[][];
};
\end{lstlisting}

\field{group_id} is the resource object ID of the flow filter group to which
this rule belongs. \field{classifier_id} is the resource object ID of the
classifier used to match a packet against the \field{key}.

\field{rule_priority} denotes the priority of the rule within the group
specified by the \field{group_id}.
Rules within the group are applied from the highest to the lowest priority
until a rule matches the packet and an
action is taken. Rules with the same priority can be applied in any order.

\field{reserved} and \field{reserved1} are reserved and set to 0.

\field{keys[][]} is an array of keys to match against packets, using
the classifier specified by \field{classifier_id}. Each entry (key) comprises
a byte array, and they are located one immediately after another.
The size (number of entries) of the array is exactly the same as that of
\field{selectors} in the classifier, or in other words, \field{count}
in the classifier.

\field{key_length} specifies the total length of \field{keys} in bytes.
In other words, it equals the sum total of \field{length} of all
selectors in \field{selectors} in the classifier specified by
\field{classifier_id}.

For example, if a classifier object's \field{selectors[0].type} is
VIRTIO_NET_FF_MASK_TYPE_ETH and \field{selectors[1].type} is
VIRTIO_NET_FF_MASK_TYPE_IPV6,
then selectors[0].length is 14 and selectors[1].length is 40.
Accordingly, the \field{key_length} is set to 54.
This setting indicates that the \field{key} array's length is 54 bytes
comprising a first byte array of 14 bytes for the
Ethernet MAC header in bytes 0-13, immediately followed by 40 bytes for the
IPv6 header in bytes 14-53.

When there are multiple selectors in the classifier object, the key bytes
for (N)\textsuperscript{th} selector are set so that
(N+1)\textsuperscript{th} selector can be matched.

If \field{count} is 2 or more, key bytes of \field{EtherType}
are set according to \hyperref[intro:IEEE 802 Ethertypes]{IEEE 802 Ethertypes}
for VIRTIO_NET_FF_MASK_TYPE_IPV4 or VIRTIO_NET_FF_MASK_TYPE_IPV6 respectively.

If \field{count} is more than 2, when \field{selector[1].type} is
VIRTIO_NET_FF_MASK_TYPE_IPV4 or VIRTIO_NET_FF_MASK_TYPE_IPV6, key
bytes of \field{Protocol} or \field{Next Header} is set as per
\field{Protocol Numbers} defined \hyperref[intro:IANA Protocol Numbers]{IANA Protocol Numbers}
respectively.

\field{action} is the action to take when a packet matches the
\field{key} using the \field{classifier_id}. Supported actions are described in
\ref{table:Device Types / Network Device / Device Operation / Flow filter / Device and driver capabilities / VIRTIO-NET-FF-ACTION-CAP / flow filter rule actions}.

\field{vq_index} specifies a receive virtqueue. When the \field{action} is set
to VIRTIO_NET_FF_ACTION_DIRECT_RX_VQ, and the packet matches the \field{key},
the matching packet is directed to this virtqueue.

Note that at most one action is ever taken for a given packet. If a rule is
applied and an action is taken, the action of other rules is not taken.

\devicenormative{\paragraph}{Flow filter}{Device Types / Network Device / Device Operation / Flow filter}

When the device supports flow filter operations,
\begin{itemize}
\item the device MUST set VIRTIO_NET_FF_RESOURCE_CAP, VIRTIO_NET_FF_SELECTOR_CAP
and VIRTIO_NET_FF_ACTION_CAP capability in the \field{supported_caps} in the
command VIRTIO_ADMIN_CMD_CAP_SUPPORT_QUERY.
\item the device MUST support the administration commands
VIRTIO_ADMIN_CMD_RESOURCE_OBJ_CREATE,
VIRTIO_ADMIN_CMD_RESOURCE_OBJ_MODIFY, VIRTIO_ADMIN_CMD_RESOURCE_OBJ_QUERY,
VIRTIO_ADMIN_CMD_RESOURCE_OBJ_DESTROY for the resource types
VIRTIO_NET_RESOURCE_OBJ_FF_GROUP, VIRTIO_NET_RESOURCE_OBJ_FF_CLASSIFIER and
VIRTIO_NET_RESOURCE_OBJ_FF_RULE.
\end{itemize}

When any of the VIRTIO_NET_FF_RESOURCE_CAP, VIRTIO_NET_FF_SELECTOR_CAP, or
VIRTIO_NET_FF_ACTION_CAP capability is disabled, the device SHOULD set
\field{status} to VIRTIO_ADMIN_STATUS_Q_INVALID_OPCODE for the commands
VIRTIO_ADMIN_CMD_RESOURCE_OBJ_CREATE,
VIRTIO_ADMIN_CMD_RESOURCE_OBJ_MODIFY, VIRTIO_ADMIN_CMD_RESOURCE_OBJ_QUERY,
and VIRTIO_ADMIN_CMD_RESOURCE_OBJ_DESTROY. These commands apply to the resource
\field{type} of VIRTIO_NET_RESOURCE_OBJ_FF_GROUP, VIRTIO_NET_RESOURCE_OBJ_FF_CLASSIFIER, and
VIRTIO_NET_RESOURCE_OBJ_FF_RULE.

The device SHOULD set \field{status} to VIRTIO_ADMIN_STATUS_EINVAL for the
command VIRTIO_ADMIN_CMD_RESOURCE_OBJ_CREATE when the resource \field{type}
is VIRTIO_NET_RESOURCE_OBJ_FF_GROUP, if a flow filter group already exists
with the supplied \field{group_priority}.

The device SHOULD set \field{status} to VIRTIO_ADMIN_STATUS_ENOSPC for the
command VIRTIO_ADMIN_CMD_RESOURCE_OBJ_CREATE when the resource \field{type}
is VIRTIO_NET_RESOURCE_OBJ_FF_GROUP, if the number of flow filter group
objects in the device exceeds the lower of the configured driver
capabilities \field{groups_limit} and \field{rules_per_group_limit}.

The device SHOULD set \field{status} to VIRTIO_ADMIN_STATUS_ENOSPC for the
command VIRTIO_ADMIN_CMD_RESOURCE_OBJ_CREATE when the resource \field{type} is
VIRTIO_NET_RESOURCE_OBJ_FF_CLASSIFIER, if the number of flow filter selector
objects in the device exceeds the configured driver capability
\field{selectors_limit}.

The device SHOULD set \field{status} to VIRTIO_ADMIN_STATUS_EBUSY for the
command VIRTIO_ADMIN_CMD_RESOURCE_OBJ_DESTROY for a flow filter group when
the flow filter group has one or more flow filter rules depending on it.

The device SHOULD set \field{status} to VIRTIO_ADMIN_STATUS_EBUSY for the
command VIRTIO_ADMIN_CMD_RESOURCE_OBJ_DESTROY for a flow filter classifier when
the flow filter classifier has one or more flow filter rules depending on it.

The device SHOULD fail the command VIRTIO_ADMIN_CMD_RESOURCE_OBJ_CREATE for the
flow filter rule resource object if,
\begin{itemize}
\item \field{vq_index} is not a valid receive virtqueue index for
the VIRTIO_NET_FF_ACTION_DIRECT_RX_VQ action,
\item \field{priority} is greater than or equal to
      \field{last_rule_priority},
\item \field{id} is greater than or equal to \field{rules_limit} or
      greater than or equal to \field{rules_per_group_limit}, whichever is lower,
\item the length of \field{keys} and the length of all the mask bytes of
      \field{selectors[].mask} as referred by \field{classifier_id} differs,
\item the supplied \field{action} is not supported in the capability VIRTIO_NET_FF_ACTION_CAP.
\end{itemize}

When the flow filter directs a packet to the virtqueue identified by
\field{vq_index} and if the receive virtqueue is reset, the device
MUST drop such packets.

Upon applying a flow filter rule to a packet, the device MUST STOP any further
application of rules and cease applying any other steering configurations.

For multiple flow filter groups, the device MUST apply the rules from
the group with the highest priority. If any rule from this group is applied,
the device MUST ignore the remaining groups. If none of the rules from the
highest priority group match, the device MUST apply the rules from
the group with the next highest priority, until either a rule matches or
all groups have been attempted.

The device MUST apply the rules within the group from the highest to the
lowest priority until a rule matches the packet, and the device MUST take
the action. If an action is taken, the device MUST not take any other
action for this packet.

The device MAY apply the rules with the same \field{rule_priority} in any
order within the group.

The device MUST process incoming packets in the following order:
\begin{itemize}
\item apply the steering configuration received using control virtqueue
      commands VIRTIO_NET_CTRL_RX, VIRTIO_NET_CTRL_MAC, and
      VIRTIO_NET_CTRL_VLAN.
\item apply flow filter rules if any.
\item if no filter rule is applied, apply the steering configuration
      received using the command VIRTIO_NET_CTRL_MQ_RSS_CONFIG
      or according to automatic receive steering.
\end{itemize}

When processing an incoming packet, if the packet is dropped at any stage, the device
MUST skip further processing.

When the device drops the packet due to the configuration done using the control
virtqueue commands VIRTIO_NET_CTRL_RX or VIRTIO_NET_CTRL_MAC or VIRTIO_NET_CTRL_VLAN,
the device MUST skip flow filter rules for this packet.

When the device performs flow filter match operations and if the operation
result did not have any match in all the groups, the receive packet processing
continues to next level, i.e. to apply configuration done using
VIRTIO_NET_CTRL_MQ_RSS_CONFIG command.

The device MUST support the creation of flow filter classifier objects
using the command VIRTIO_ADMIN_CMD_RESOURCE_OBJ_CREATE with \field{flags}
set to VIRTIO_NET_FF_MASK_F_PARTIAL_MASK;
this support is required even if all the bits of the masks are set for
a field in \field{selectors}, provided that partial masking is supported
for the selectors.

\drivernormative{\paragraph}{Flow filter}{Device Types / Network Device / Device Operation / Flow filter}

The driver MUST enable VIRTIO_NET_FF_RESOURCE_CAP, VIRTIO_NET_FF_SELECTOR_CAP,
and VIRTIO_NET_FF_ACTION_CAP capabilities to use flow filter.

The driver SHOULD NOT remove a flow filter group using the command
VIRTIO_ADMIN_CMD_RESOURCE_OBJ_DESTROY when one or more flow filter rules
depend on that group. The driver SHOULD only destroy the group after
all the associated rules have been destroyed.

The driver SHOULD NOT remove a flow filter classifier using the command
VIRTIO_ADMIN_CMD_RESOURCE_OBJ_DESTROY when one or more flow filter rules
depend on the classifier. The driver SHOULD only destroy the classifier
after all the associated rules have been destroyed.

The driver SHOULD NOT add multiple flow filter rules with the same
\field{rule_priority} within a flow filter group, as these rules MAY match
the same packet. The driver SHOULD assign different \field{rule_priority}
values to different flow filter rules if multiple rules may match a single
packet.

For the command VIRTIO_ADMIN_CMD_RESOURCE_OBJ_CREATE, when creating a resource
of \field{type} VIRTIO_NET_RESOURCE_OBJ_FF_CLASSIFIER, the driver MUST set:
\begin{itemize}
\item \field{selectors[0].type} to VIRTIO_NET_FF_MASK_TYPE_ETH.
\item \field{selectors[1].type} to VIRTIO_NET_FF_MASK_TYPE_IPV4 or
      VIRTIO_NET_FF_MASK_TYPE_IPV6 when \field{count} is more than 1,
\item \field{selectors[2].type} VIRTIO_NET_FF_MASK_TYPE_UDP or
      VIRTIO_NET_FF_MASK_TYPE_TCP when \field{count} is more than 2.
\end{itemize}

For the command VIRTIO_ADMIN_CMD_RESOURCE_OBJ_CREATE, when creating a resource
of \field{type} VIRTIO_NET_RESOURCE_OBJ_FF_CLASSIFIER, the driver MUST set:
\begin{itemize}
\item \field{selectors[0].mask} bytes to all 1s for the \field{EtherType}
       when \field{count} is 2 or more.
\item \field{selectors[1].mask} bytes to all 1s for \field{Protocol} or \field{Next Header}
       when \field{selector[1].type} is VIRTIO_NET_FF_MASK_TYPE_IPV4 or VIRTIO_NET_FF_MASK_TYPE_IPV6,
       and when \field{count} is more than 2.
\end{itemize}

For the command VIRTIO_ADMIN_CMD_RESOURCE_OBJ_CREATE, the resource \field{type}
VIRTIO_NET_RESOURCE_OBJ_FF_RULE, if the corresponding classifier object's
\field{count} is 2 or more, the driver MUST SET the \field{keys} bytes of
\field{EtherType} in accordance with
\hyperref[intro:IEEE 802 Ethertypes]{IEEE 802 Ethertypes}
for either VIRTIO_NET_FF_MASK_TYPE_IPV4 or VIRTIO_NET_FF_MASK_TYPE_IPV6.

For the command VIRTIO_ADMIN_CMD_RESOURCE_OBJ_CREATE, when creating a resource of
\field{type} VIRTIO_NET_RESOURCE_OBJ_FF_RULE, if the corresponding classifier
object's \field{count} is more than 2, and the \field{selector[1].type} is either
VIRTIO_NET_FF_MASK_TYPE_IPV4 or VIRTIO_NET_FF_MASK_TYPE_IPV6, the driver MUST
set the \field{keys} bytes for the \field{Protocol} or \field{Next Header}
according to \hyperref[intro:IANA Protocol Numbers]{IANA Protocol Numbers} respectively.

The driver SHOULD set all the bits for a field in the mask of a selector in both the
capability and the classifier object, unless the VIRTIO_NET_FF_MASK_F_PARTIAL_MASK
is enabled.

\subsubsection{Legacy Interface: Framing Requirements}\label{sec:Device
Types / Network Device / Legacy Interface: Framing Requirements}

When using legacy interfaces, transitional drivers which have not
negotiated VIRTIO_F_ANY_LAYOUT MUST use a single descriptor for the
\field{struct virtio_net_hdr} on both transmit and receive, with the
network data in the following descriptors.

Additionally, when using the control virtqueue (see \ref{sec:Device
Types / Network Device / Device Operation / Control Virtqueue})
, transitional drivers which have not
negotiated VIRTIO_F_ANY_LAYOUT MUST:
\begin{itemize}
\item for all commands, use a single 2-byte descriptor including the first two
fields: \field{class} and \field{command}
\item for all commands except VIRTIO_NET_CTRL_MAC_TABLE_SET
use a single descriptor including command-specific-data
with no padding.
\item for the VIRTIO_NET_CTRL_MAC_TABLE_SET command use exactly
two descriptors including command-specific-data with no padding:
the first of these descriptors MUST include the
virtio_net_ctrl_mac table structure for the unicast addresses with no padding,
the second of these descriptors MUST include the
virtio_net_ctrl_mac table structure for the multicast addresses
with no padding.
\item for all commands, use a single 1-byte descriptor for the
\field{ack} field
\end{itemize}

See \ref{sec:Basic
Facilities of a Virtio Device / Virtqueues / Message Framing}.

\section{Network Device}\label{sec:Device Types / Network Device}

The virtio network device is a virtual network interface controller.
It consists of a virtual Ethernet link which connects the device
to the Ethernet network. The device has transmit and receive
queues. The driver adds empty buffers to the receive virtqueue.
The device receives incoming packets from the link; the device
places these incoming packets in the receive virtqueue buffers.
The driver adds outgoing packets to the transmit virtqueue. The device
removes these packets from the transmit virtqueue and sends them to
the link. The device may have a control virtqueue. The driver
uses the control virtqueue to dynamically manipulate various
features of the initialized device.

\subsection{Device ID}\label{sec:Device Types / Network Device / Device ID}

 1

\subsection{Virtqueues}\label{sec:Device Types / Network Device / Virtqueues}

\begin{description}
\item[0] receiveq1
\item[1] transmitq1
\item[\ldots]
\item[2(N-1)] receiveqN
\item[2(N-1)+1] transmitqN
\item[2N] controlq
\end{description}

 N=1 if neither VIRTIO_NET_F_MQ nor VIRTIO_NET_F_RSS are negotiated, otherwise N is set by
 \field{max_virtqueue_pairs}.

controlq is optional; it only exists if VIRTIO_NET_F_CTRL_VQ is
negotiated.

\subsection{Feature bits}\label{sec:Device Types / Network Device / Feature bits}

\begin{description}
\item[VIRTIO_NET_F_CSUM (0)] Device handles packets with partial checksum offload.

\item[VIRTIO_NET_F_GUEST_CSUM (1)] Driver handles packets with partial checksum.

\item[VIRTIO_NET_F_CTRL_GUEST_OFFLOADS (2)] Control channel offloads
        reconfiguration support.

\item[VIRTIO_NET_F_MTU(3)] Device maximum MTU reporting is supported. If
    offered by the device, device advises driver about the value of
    its maximum MTU. If negotiated, the driver uses \field{mtu} as
    the maximum MTU value.

\item[VIRTIO_NET_F_MAC (5)] Device has given MAC address.

\item[VIRTIO_NET_F_GUEST_TSO4 (7)] Driver can receive TSOv4.

\item[VIRTIO_NET_F_GUEST_TSO6 (8)] Driver can receive TSOv6.

\item[VIRTIO_NET_F_GUEST_ECN (9)] Driver can receive TSO with ECN.

\item[VIRTIO_NET_F_GUEST_UFO (10)] Driver can receive UFO.

\item[VIRTIO_NET_F_HOST_TSO4 (11)] Device can receive TSOv4.

\item[VIRTIO_NET_F_HOST_TSO6 (12)] Device can receive TSOv6.

\item[VIRTIO_NET_F_HOST_ECN (13)] Device can receive TSO with ECN.

\item[VIRTIO_NET_F_HOST_UFO (14)] Device can receive UFO.

\item[VIRTIO_NET_F_MRG_RXBUF (15)] Driver can merge receive buffers.

\item[VIRTIO_NET_F_STATUS (16)] Configuration status field is
    available.

\item[VIRTIO_NET_F_CTRL_VQ (17)] Control channel is available.

\item[VIRTIO_NET_F_CTRL_RX (18)] Control channel RX mode support.

\item[VIRTIO_NET_F_CTRL_VLAN (19)] Control channel VLAN filtering.

\item[VIRTIO_NET_F_CTRL_RX_EXTRA (20)]	Control channel RX extra mode support.

\item[VIRTIO_NET_F_GUEST_ANNOUNCE(21)] Driver can send gratuitous
    packets.

\item[VIRTIO_NET_F_MQ(22)] Device supports multiqueue with automatic
    receive steering.

\item[VIRTIO_NET_F_CTRL_MAC_ADDR(23)] Set MAC address through control
    channel.

\item[VIRTIO_NET_F_DEVICE_STATS(50)] Device can provide device-level statistics
    to the driver through the control virtqueue.

\item[VIRTIO_NET_F_HASH_TUNNEL(51)] Device supports inner header hash for encapsulated packets.

\item[VIRTIO_NET_F_VQ_NOTF_COAL(52)] Device supports virtqueue notification coalescing.

\item[VIRTIO_NET_F_NOTF_COAL(53)] Device supports notifications coalescing.

\item[VIRTIO_NET_F_GUEST_USO4 (54)] Driver can receive USOv4 packets.

\item[VIRTIO_NET_F_GUEST_USO6 (55)] Driver can receive USOv6 packets.

\item[VIRTIO_NET_F_HOST_USO (56)] Device can receive USO packets. Unlike UFO
 (fragmenting the packet) the USO splits large UDP packet
 to several segments when each of these smaller packets has UDP header.

\item[VIRTIO_NET_F_HASH_REPORT(57)] Device can report per-packet hash
    value and a type of calculated hash.

\item[VIRTIO_NET_F_GUEST_HDRLEN(59)] Driver can provide the exact \field{hdr_len}
    value. Device benefits from knowing the exact header length.

\item[VIRTIO_NET_F_RSS(60)] Device supports RSS (receive-side scaling)
    with Toeplitz hash calculation and configurable hash
    parameters for receive steering.

\item[VIRTIO_NET_F_RSC_EXT(61)] Device can process duplicated ACKs
    and report number of coalesced segments and duplicated ACKs.

\item[VIRTIO_NET_F_STANDBY(62)] Device may act as a standby for a primary
    device with the same MAC address.

\item[VIRTIO_NET_F_SPEED_DUPLEX(63)] Device reports speed and duplex.

\item[VIRTIO_NET_F_RSS_CONTEXT(64)] Device supports multiple RSS contexts.

\item[VIRTIO_NET_F_GUEST_UDP_TUNNEL_GSO (65)] Driver can receive GSO packets
  carried by a UDP tunnel.

\item[VIRTIO_NET_F_GUEST_UDP_TUNNEL_GSO_CSUM (66)] Driver handles packets
  carried by a UDP tunnel with partial csum for the outer header.

\item[VIRTIO_NET_F_HOST_UDP_TUNNEL_GSO (67)] Device can receive GSO packets
  carried by a UDP tunnel.

\item[VIRTIO_NET_F_HOST_UDP_TUNNEL_GSO_CSUM (68)] Device handles packets
  carried by a UDP tunnel with partial csum for the outer header.
\end{description}

\subsubsection{Feature bit requirements}\label{sec:Device Types / Network Device / Feature bits / Feature bit requirements}

Some networking feature bits require other networking feature bits
(see \ref{drivernormative:Basic Facilities of a Virtio Device / Feature Bits}):

\begin{description}
\item[VIRTIO_NET_F_GUEST_TSO4] Requires VIRTIO_NET_F_GUEST_CSUM.
\item[VIRTIO_NET_F_GUEST_TSO6] Requires VIRTIO_NET_F_GUEST_CSUM.
\item[VIRTIO_NET_F_GUEST_ECN] Requires VIRTIO_NET_F_GUEST_TSO4 or VIRTIO_NET_F_GUEST_TSO6.
\item[VIRTIO_NET_F_GUEST_UFO] Requires VIRTIO_NET_F_GUEST_CSUM.
\item[VIRTIO_NET_F_GUEST_USO4] Requires VIRTIO_NET_F_GUEST_CSUM.
\item[VIRTIO_NET_F_GUEST_USO6] Requires VIRTIO_NET_F_GUEST_CSUM.
\item[VIRTIO_NET_F_GUEST_UDP_TUNNEL_GSO] Requires VIRTIO_NET_F_GUEST_TSO4, VIRTIO_NET_F_GUEST_TSO6,
   VIRTIO_NET_F_GUEST_USO4 and VIRTIO_NET_F_GUEST_USO6.
\item[VIRTIO_NET_F_GUEST_UDP_TUNNEL_GSO_CSUM] Requires VIRTIO_NET_F_GUEST_UDP_TUNNEL_GSO

\item[VIRTIO_NET_F_HOST_TSO4] Requires VIRTIO_NET_F_CSUM.
\item[VIRTIO_NET_F_HOST_TSO6] Requires VIRTIO_NET_F_CSUM.
\item[VIRTIO_NET_F_HOST_ECN] Requires VIRTIO_NET_F_HOST_TSO4 or VIRTIO_NET_F_HOST_TSO6.
\item[VIRTIO_NET_F_HOST_UFO] Requires VIRTIO_NET_F_CSUM.
\item[VIRTIO_NET_F_HOST_USO] Requires VIRTIO_NET_F_CSUM.
\item[VIRTIO_NET_F_HOST_UDP_TUNNEL_GSO] Requires VIRTIO_NET_F_HOST_TSO4, VIRTIO_NET_F_HOST_TSO6
   and VIRTIO_NET_F_HOST_USO.
\item[VIRTIO_NET_F_HOST_UDP_TUNNEL_GSO_CSUM] Requires VIRTIO_NET_F_HOST_UDP_TUNNEL_GSO

\item[VIRTIO_NET_F_CTRL_RX] Requires VIRTIO_NET_F_CTRL_VQ.
\item[VIRTIO_NET_F_CTRL_VLAN] Requires VIRTIO_NET_F_CTRL_VQ.
\item[VIRTIO_NET_F_GUEST_ANNOUNCE] Requires VIRTIO_NET_F_CTRL_VQ.
\item[VIRTIO_NET_F_MQ] Requires VIRTIO_NET_F_CTRL_VQ.
\item[VIRTIO_NET_F_CTRL_MAC_ADDR] Requires VIRTIO_NET_F_CTRL_VQ.
\item[VIRTIO_NET_F_NOTF_COAL] Requires VIRTIO_NET_F_CTRL_VQ.
\item[VIRTIO_NET_F_RSC_EXT] Requires VIRTIO_NET_F_HOST_TSO4 or VIRTIO_NET_F_HOST_TSO6.
\item[VIRTIO_NET_F_RSS] Requires VIRTIO_NET_F_CTRL_VQ.
\item[VIRTIO_NET_F_VQ_NOTF_COAL] Requires VIRTIO_NET_F_CTRL_VQ.
\item[VIRTIO_NET_F_HASH_TUNNEL] Requires VIRTIO_NET_F_CTRL_VQ along with VIRTIO_NET_F_RSS or VIRTIO_NET_F_HASH_REPORT.
\item[VIRTIO_NET_F_RSS_CONTEXT] Requires VIRTIO_NET_F_CTRL_VQ and VIRTIO_NET_F_RSS.
\end{description}

\begin{note}
The dependency between UDP_TUNNEL_GSO_CSUM and UDP_TUNNEL_GSO is intentionally
in the opposite direction with respect to the plain GSO features and the plain
checksum offload because UDP tunnel checksum offload gives very little gain
for non GSO packets and is quite complex to implement in H/W.
\end{note}

\subsubsection{Legacy Interface: Feature bits}\label{sec:Device Types / Network Device / Feature bits / Legacy Interface: Feature bits}
\begin{description}
\item[VIRTIO_NET_F_GSO (6)] Device handles packets with any GSO type. This was supposed to indicate segmentation offload support, but
upon further investigation it became clear that multiple bits were needed.
\item[VIRTIO_NET_F_GUEST_RSC4 (41)] Device coalesces TCPIP v4 packets. This was implemented by hypervisor patch for certification
purposes and current Windows driver depends on it. It will not function if virtio-net device reports this feature.
\item[VIRTIO_NET_F_GUEST_RSC6 (42)] Device coalesces TCPIP v6 packets. Similar to VIRTIO_NET_F_GUEST_RSC4.
\end{description}

\subsection{Device configuration layout}\label{sec:Device Types / Network Device / Device configuration layout}
\label{sec:Device Types / Block Device / Feature bits / Device configuration layout}

The network device has the following device configuration layout.
All of the device configuration fields are read-only for the driver.

\begin{lstlisting}
struct virtio_net_config {
        u8 mac[6];
        le16 status;
        le16 max_virtqueue_pairs;
        le16 mtu;
        le32 speed;
        u8 duplex;
        u8 rss_max_key_size;
        le16 rss_max_indirection_table_length;
        le32 supported_hash_types;
        le32 supported_tunnel_types;
};
\end{lstlisting}

The \field{mac} address field always exists (although it is only
valid if VIRTIO_NET_F_MAC is set).

The \field{status} only exists if VIRTIO_NET_F_STATUS is set.
Two bits are currently defined for the status field: VIRTIO_NET_S_LINK_UP
and VIRTIO_NET_S_ANNOUNCE.

\begin{lstlisting}
#define VIRTIO_NET_S_LINK_UP     1
#define VIRTIO_NET_S_ANNOUNCE    2
\end{lstlisting}

The following field, \field{max_virtqueue_pairs} only exists if
VIRTIO_NET_F_MQ or VIRTIO_NET_F_RSS is set. This field specifies the maximum number
of each of transmit and receive virtqueues (receiveq1\ldots receiveqN
and transmitq1\ldots transmitqN respectively) that can be configured once at least one of these features
is negotiated.

The following field, \field{mtu} only exists if VIRTIO_NET_F_MTU
is set. This field specifies the maximum MTU for the driver to
use.

The following two fields, \field{speed} and \field{duplex}, only
exist if VIRTIO_NET_F_SPEED_DUPLEX is set.

\field{speed} contains the device speed, in units of 1 MBit per
second, 0 to 0x7fffffff, or 0xffffffff for unknown speed.

\field{duplex} has the values of 0x01 for full duplex, 0x00 for
half duplex and 0xff for unknown duplex state.

Both \field{speed} and \field{duplex} can change, thus the driver
is expected to re-read these values after receiving a
configuration change notification.

The following field, \field{rss_max_key_size} only exists if VIRTIO_NET_F_RSS or VIRTIO_NET_F_HASH_REPORT is set.
It specifies the maximum supported length of RSS key in bytes.

The following field, \field{rss_max_indirection_table_length} only exists if VIRTIO_NET_F_RSS is set.
It specifies the maximum number of 16-bit entries in RSS indirection table.

The next field, \field{supported_hash_types} only exists if the device supports hash calculation,
i.e. if VIRTIO_NET_F_RSS or VIRTIO_NET_F_HASH_REPORT is set.

Field \field{supported_hash_types} contains the bitmask of supported hash types.
See \ref{sec:Device Types / Network Device / Device Operation / Processing of Incoming Packets / Hash calculation for incoming packets / Supported/enabled hash types} for details of supported hash types.

Field \field{supported_tunnel_types} only exists if the device supports inner header hash, i.e. if VIRTIO_NET_F_HASH_TUNNEL is set.

Field \field{supported_tunnel_types} contains the bitmask of encapsulation types supported by the device for inner header hash.
Encapsulation types are defined in \ref{sec:Device Types / Network Device / Device Operation / Processing of Incoming Packets /
Hash calculation for incoming packets / Encapsulation types supported/enabled for inner header hash}.

\devicenormative{\subsubsection}{Device configuration layout}{Device Types / Network Device / Device configuration layout}

The device MUST set \field{max_virtqueue_pairs} to between 1 and 0x8000 inclusive,
if it offers VIRTIO_NET_F_MQ.

The device MUST set \field{mtu} to between 68 and 65535 inclusive,
if it offers VIRTIO_NET_F_MTU.

The device SHOULD set \field{mtu} to at least 1280, if it offers
VIRTIO_NET_F_MTU.

The device MUST NOT modify \field{mtu} once it has been set.

The device MUST NOT pass received packets that exceed \field{mtu} (plus low
level ethernet header length) size with \field{gso_type} NONE or ECN
after VIRTIO_NET_F_MTU has been successfully negotiated.

The device MUST forward transmitted packets of up to \field{mtu} (plus low
level ethernet header length) size with \field{gso_type} NONE or ECN, and do
so without fragmentation, after VIRTIO_NET_F_MTU has been successfully
negotiated.

The device MUST set \field{rss_max_key_size} to at least 40, if it offers
VIRTIO_NET_F_RSS or VIRTIO_NET_F_HASH_REPORT.

The device MUST set \field{rss_max_indirection_table_length} to at least 128, if it offers
VIRTIO_NET_F_RSS.

If the driver negotiates the VIRTIO_NET_F_STANDBY feature, the device MAY act
as a standby device for a primary device with the same MAC address.

If VIRTIO_NET_F_SPEED_DUPLEX has been negotiated, \field{speed}
MUST contain the device speed, in units of 1 MBit per second, 0 to
0x7ffffffff, or 0xfffffffff for unknown.

If VIRTIO_NET_F_SPEED_DUPLEX has been negotiated, \field{duplex}
MUST have the values of 0x00 for full duplex, 0x01 for half
duplex, or 0xff for unknown.

If VIRTIO_NET_F_SPEED_DUPLEX and VIRTIO_NET_F_STATUS have both
been negotiated, the device SHOULD NOT change the \field{speed} and
\field{duplex} fields as long as VIRTIO_NET_S_LINK_UP is set in
the \field{status}.

The device SHOULD NOT offer VIRTIO_NET_F_HASH_REPORT if it
does not offer VIRTIO_NET_F_CTRL_VQ.

The device SHOULD NOT offer VIRTIO_NET_F_CTRL_RX_EXTRA if it
does not offer VIRTIO_NET_F_CTRL_VQ.

\drivernormative{\subsubsection}{Device configuration layout}{Device Types / Network Device / Device configuration layout}

The driver MUST NOT write to any of the device configuration fields.

A driver SHOULD negotiate VIRTIO_NET_F_MAC if the device offers it.
If the driver negotiates the VIRTIO_NET_F_MAC feature, the driver MUST set
the physical address of the NIC to \field{mac}.  Otherwise, it SHOULD
use a locally-administered MAC address (see \hyperref[intro:IEEE 802]{IEEE 802},
``9.2 48-bit universal LAN MAC addresses'').

If the driver does not negotiate the VIRTIO_NET_F_STATUS feature, it SHOULD
assume the link is active, otherwise it SHOULD read the link status from
the bottom bit of \field{status}.

A driver SHOULD negotiate VIRTIO_NET_F_MTU if the device offers it.

If the driver negotiates VIRTIO_NET_F_MTU, it MUST supply enough receive
buffers to receive at least one receive packet of size \field{mtu} (plus low
level ethernet header length) with \field{gso_type} NONE or ECN.

If the driver negotiates VIRTIO_NET_F_MTU, it MUST NOT transmit packets of
size exceeding the value of \field{mtu} (plus low level ethernet header length)
with \field{gso_type} NONE or ECN.

A driver SHOULD negotiate the VIRTIO_NET_F_STANDBY feature if the device offers it.

If VIRTIO_NET_F_SPEED_DUPLEX has been negotiated,
the driver MUST treat any value of \field{speed} above
0x7fffffff as well as any value of \field{duplex} not
matching 0x00 or 0x01 as an unknown value.

If VIRTIO_NET_F_SPEED_DUPLEX has been negotiated, the driver
SHOULD re-read \field{speed} and \field{duplex} after a
configuration change notification.

A driver SHOULD NOT negotiate VIRTIO_NET_F_HASH_REPORT if it
does not negotiate VIRTIO_NET_F_CTRL_VQ.

A driver SHOULD NOT negotiate VIRTIO_NET_F_CTRL_RX_EXTRA if it
does not negotiate VIRTIO_NET_F_CTRL_VQ.

\subsubsection{Legacy Interface: Device configuration layout}\label{sec:Device Types / Network Device / Device configuration layout / Legacy Interface: Device configuration layout}
\label{sec:Device Types / Block Device / Feature bits / Device configuration layout / Legacy Interface: Device configuration layout}
When using the legacy interface, transitional devices and drivers
MUST format \field{status} and
\field{max_virtqueue_pairs} in struct virtio_net_config
according to the native endian of the guest rather than
(necessarily when not using the legacy interface) little-endian.

When using the legacy interface, \field{mac} is driver-writable
which provided a way for drivers to update the MAC without
negotiating VIRTIO_NET_F_CTRL_MAC_ADDR.

\subsection{Device Initialization}\label{sec:Device Types / Network Device / Device Initialization}

A driver would perform a typical initialization routine like so:

\begin{enumerate}
\item Identify and initialize the receive and
  transmission virtqueues, up to N of each kind. If
  VIRTIO_NET_F_MQ feature bit is negotiated,
  N=\field{max_virtqueue_pairs}, otherwise identify N=1.

\item If the VIRTIO_NET_F_CTRL_VQ feature bit is negotiated,
  identify the control virtqueue.

\item Fill the receive queues with buffers: see \ref{sec:Device Types / Network Device / Device Operation / Setting Up Receive Buffers}.

\item Even with VIRTIO_NET_F_MQ, only receiveq1, transmitq1 and
  controlq are used by default.  The driver would send the
  VIRTIO_NET_CTRL_MQ_VQ_PAIRS_SET command specifying the
  number of the transmit and receive queues to use.

\item If the VIRTIO_NET_F_MAC feature bit is set, the configuration
  space \field{mac} entry indicates the ``physical'' address of the
  device, otherwise the driver would typically generate a random
  local MAC address.

\item If the VIRTIO_NET_F_STATUS feature bit is negotiated, the link
  status comes from the bottom bit of \field{status}.
  Otherwise, the driver assumes it's active.

\item A performant driver would indicate that it will generate checksumless
  packets by negotiating the VIRTIO_NET_F_CSUM feature.

\item If that feature is negotiated, a driver can use TCP segmentation or UDP
  segmentation/fragmentation offload by negotiating the VIRTIO_NET_F_HOST_TSO4 (IPv4
  TCP), VIRTIO_NET_F_HOST_TSO6 (IPv6 TCP), VIRTIO_NET_F_HOST_UFO
  (UDP fragmentation) and VIRTIO_NET_F_HOST_USO (UDP segmentation) features.

\item If the VIRTIO_NET_F_HOST_TSO6, VIRTIO_NET_F_HOST_TSO4 and VIRTIO_NET_F_HOST_USO
  segmentation features are negotiated, a driver can
  use TCP segmentation or UDP segmentation on top of UDP encapsulation
  offload, when the outer header does not require checksumming - e.g.
  the outer UDP checksum is zero - by negotiating the
  VIRTIO_NET_F_HOST_UDP_TUNNEL_GSO feature.
  GSO over UDP tunnels packets carry two sets of headers: the outer ones
  and the inner ones. The outer transport protocol is UDP, the inner
  could be either TCP or UDP. Only a single level of encapsulation
  offload is supported.

\item If VIRTIO_NET_F_HOST_UDP_TUNNEL_GSO is negotiated, a driver can
  additionally use TCP segmentation or UDP segmentation on top of UDP
  encapsulation with the outer header requiring checksum offload,
  negotiating the VIRTIO_NET_F_HOST_UDP_TUNNEL_GSO_CSUM feature.

\item The converse features are also available: a driver can save
  the virtual device some work by negotiating these features.\note{For example, a network packet transported between two guests on
the same system might not need checksumming at all, nor segmentation,
if both guests are amenable.}
   The VIRTIO_NET_F_GUEST_CSUM feature indicates that partially
  checksummed packets can be received, and if it can do that then
  the VIRTIO_NET_F_GUEST_TSO4, VIRTIO_NET_F_GUEST_TSO6,
  VIRTIO_NET_F_GUEST_UFO, VIRTIO_NET_F_GUEST_ECN, VIRTIO_NET_F_GUEST_USO4,
  VIRTIO_NET_F_GUEST_USO6 VIRTIO_NET_F_GUEST_UDP_TUNNEL_GSO and
  VIRTIO_NET_F_GUEST_UDP_TUNNEL_GSO_CSUM are the input equivalents of
  the features described above.
  See \ref{sec:Device Types / Network Device / Device Operation /
Setting Up Receive Buffers}~\nameref{sec:Device Types / Network
Device / Device Operation / Setting Up Receive Buffers} and
\ref{sec:Device Types / Network Device / Device Operation /
Processing of Incoming Packets}~\nameref{sec:Device Types /
Network Device / Device Operation / Processing of Incoming Packets} below.
\end{enumerate}

A truly minimal driver would only accept VIRTIO_NET_F_MAC and ignore
everything else.

\subsection{Device and driver capabilities}\label{sec:Device Types / Network Device / Device and driver capabilities}

The network device has the following capabilities.

\begin{tabularx}{\textwidth}{ |l||l|X| }
\hline
Identifier & Name & Description \\
\hline \hline
0x0800 & \hyperref[par:Device Types / Network Device / Device Operation / Flow filter / Device and driver capabilities / VIRTIO-NET-FF-RESOURCE-CAP]{VIRTIO_NET_FF_RESOURCE_CAP} & Flow filter resource capability \\
\hline
0x0801 & \hyperref[par:Device Types / Network Device / Device Operation / Flow filter / Device and driver capabilities / VIRTIO-NET-FF-SELECTOR-CAP]{VIRTIO_NET_FF_SELECTOR_CAP} & Flow filter classifier capability \\
\hline
0x0802 & \hyperref[par:Device Types / Network Device / Device Operation / Flow filter / Device and driver capabilities / VIRTIO-NET-FF-ACTION-CAP]{VIRTIO_NET_FF_ACTION_CAP} & Flow filter action capability \\
\hline
\end{tabularx}

\subsection{Device resource objects}\label{sec:Device Types / Network Device / Device resource objects}

The network device has the following resource objects.

\begin{tabularx}{\textwidth}{ |l||l|X| }
\hline
type & Name & Description \\
\hline \hline
0x0200 & \hyperref[par:Device Types / Network Device / Device Operation / Flow filter / Resource objects / VIRTIO-NET-RESOURCE-OBJ-FF-GROUP]{VIRTIO_NET_RESOURCE_OBJ_FF_GROUP} & Flow filter group resource object \\
\hline
0x0201 & \hyperref[par:Device Types / Network Device / Device Operation / Flow filter / Resource objects / VIRTIO-NET-RESOURCE-OBJ-FF-CLASSIFIER]{VIRTIO_NET_RESOURCE_OBJ_FF_CLASSIFIER} & Flow filter mask object \\
\hline
0x0202 & \hyperref[par:Device Types / Network Device / Device Operation / Flow filter / Resource objects / VIRTIO-NET-RESOURCE-OBJ-FF-RULE]{VIRTIO_NET_RESOURCE_OBJ_FF_RULE} & Flow filter rule object \\
\hline
\end{tabularx}

\subsection{Device parts}\label{sec:Device Types / Network Device / Device parts}

Network device parts represent the configuration done by the driver using control
virtqueue commands. Network device part is in the format of
\field{struct virtio_dev_part}.

\begin{tabularx}{\textwidth}{ |l||l|X| }
\hline
Type & Name & Description \\
\hline \hline
0x200 & VIRTIO_NET_DEV_PART_CVQ_CFG_PART & Represents device configuration done through a control virtqueue command, see \ref{sec:Device Types / Network Device / Device parts / VIRTIO-NET-DEV-PART-CVQ-CFG-PART} \\
\hline
0x201 - 0x5FF & - & reserved for future \\
\hline
\hline
\end{tabularx}

\subsubsection{VIRTIO_NET_DEV_PART_CVQ_CFG_PART}\label{sec:Device Types / Network Device / Device parts / VIRTIO-NET-DEV-PART-CVQ-CFG-PART}

For VIRTIO_NET_DEV_PART_CVQ_CFG_PART, \field{part_type} is set to 0x200. The
VIRTIO_NET_DEV_PART_CVQ_CFG_PART part indicates configuration performed by the
driver using a control virtqueue command.

\begin{lstlisting}
struct virtio_net_dev_part_cvq_selector {
        u8 class;
        u8 command;
        u8 reserved[6];
};
\end{lstlisting}

There is one device part of type VIRTIO_NET_DEV_PART_CVQ_CFG_PART for each
individual configuration. Each part is identified by a unique selector value.
The selector, \field{device_type_raw}, is in the format
\field{struct virtio_net_dev_part_cvq_selector}.

The selector consists of two fields: \field{class} and \field{command}. These
fields correspond to the \field{class} and \field{command} defined in
\field{struct virtio_net_ctrl}, as described in the relevant sections of
\ref{sec:Device Types / Network Device / Device Operation / Control Virtqueue}.

The value corresponding to each part’s selector follows the same format as the
respective \field{command-specific-data} described in the relevant sections of
\ref{sec:Device Types / Network Device / Device Operation / Control Virtqueue}.

For example, when the \field{class} is VIRTIO_NET_CTRL_MAC, the \field{command}
can be either VIRTIO_NET_CTRL_MAC_TABLE_SET or VIRTIO_NET_CTRL_MAC_ADDR_SET;
when \field{command} is set to VIRTIO_NET_CTRL_MAC_TABLE_SET, \field{value}
is in the format of \field{struct virtio_net_ctrl_mac}.

Supported selectors are listed in the table:

\begin{tabularx}{\textwidth}{ |l|X| }
\hline
Class selector & Command selector \\
\hline \hline
VIRTIO_NET_CTRL_RX & VIRTIO_NET_CTRL_RX_PROMISC \\
\hline
VIRTIO_NET_CTRL_RX & VIRTIO_NET_CTRL_RX_ALLMULTI \\
\hline
VIRTIO_NET_CTRL_RX & VIRTIO_NET_CTRL_RX_ALLUNI \\
\hline
VIRTIO_NET_CTRL_RX & VIRTIO_NET_CTRL_RX_NOMULTI \\
\hline
VIRTIO_NET_CTRL_RX & VIRTIO_NET_CTRL_RX_NOUNI \\
\hline
VIRTIO_NET_CTRL_RX & VIRTIO_NET_CTRL_RX_NOBCAST \\
\hline
VIRTIO_NET_CTRL_MAC & VIRTIO_NET_CTRL_MAC_TABLE_SET \\
\hline
VIRTIO_NET_CTRL_MAC & VIRTIO_NET_CTRL_MAC_ADDR_SET \\
\hline
VIRTIO_NET_CTRL_VLAN & VIRTIO_NET_CTRL_VLAN_ADD \\
\hline
VIRTIO_NET_CTRL_ANNOUNCE & VIRTIO_NET_CTRL_ANNOUNCE_ACK \\
\hline
VIRTIO_NET_CTRL_MQ & VIRTIO_NET_CTRL_MQ_VQ_PAIRS_SET \\
\hline
VIRTIO_NET_CTRL_MQ & VIRTIO_NET_CTRL_MQ_RSS_CONFIG \\
\hline
VIRTIO_NET_CTRL_MQ & VIRTIO_NET_CTRL_MQ_HASH_CONFIG \\
\hline
\hline
\end{tabularx}

For command selector VIRTIO_NET_CTRL_VLAN_ADD, device part consists of a whole
VLAN table.

\field{reserved} is reserved and set to zero.

\subsection{Device Operation}\label{sec:Device Types / Network Device / Device Operation}

Packets are transmitted by placing them in the
transmitq1\ldots transmitqN, and buffers for incoming packets are
placed in the receiveq1\ldots receiveqN. In each case, the packet
itself is preceded by a header:

\begin{lstlisting}
struct virtio_net_hdr {
#define VIRTIO_NET_HDR_F_NEEDS_CSUM    1
#define VIRTIO_NET_HDR_F_DATA_VALID    2
#define VIRTIO_NET_HDR_F_RSC_INFO      4
#define VIRTIO_NET_HDR_F_UDP_TUNNEL_CSUM 8
        u8 flags;
#define VIRTIO_NET_HDR_GSO_NONE        0
#define VIRTIO_NET_HDR_GSO_TCPV4       1
#define VIRTIO_NET_HDR_GSO_UDP         3
#define VIRTIO_NET_HDR_GSO_TCPV6       4
#define VIRTIO_NET_HDR_GSO_UDP_L4      5
#define VIRTIO_NET_HDR_GSO_UDP_TUNNEL_IPV4 0x20
#define VIRTIO_NET_HDR_GSO_UDP_TUNNEL_IPV6 0x40
#define VIRTIO_NET_HDR_GSO_ECN      0x80
        u8 gso_type;
        le16 hdr_len;
        le16 gso_size;
        le16 csum_start;
        le16 csum_offset;
        le16 num_buffers;
        le32 hash_value;        (Only if VIRTIO_NET_F_HASH_REPORT negotiated)
        le16 hash_report;       (Only if VIRTIO_NET_F_HASH_REPORT negotiated)
        le16 padding_reserved;  (Only if VIRTIO_NET_F_HASH_REPORT negotiated)
        le16 outer_th_offset    (Only if VIRTIO_NET_F_HOST_UDP_TUNNEL_GSO or VIRTIO_NET_F_GUEST_UDP_TUNNEL_GSO negotiated)
        le16 inner_nh_offset;   (Only if VIRTIO_NET_F_HOST_UDP_TUNNEL_GSO or VIRTIO_NET_F_GUEST_UDP_TUNNEL_GSO negotiated)
};
\end{lstlisting}

The controlq is used to control various device features described further in
section \ref{sec:Device Types / Network Device / Device Operation / Control Virtqueue}.

\subsubsection{Legacy Interface: Device Operation}\label{sec:Device Types / Network Device / Device Operation / Legacy Interface: Device Operation}
When using the legacy interface, transitional devices and drivers
MUST format the fields in \field{struct virtio_net_hdr}
according to the native endian of the guest rather than
(necessarily when not using the legacy interface) little-endian.

The legacy driver only presented \field{num_buffers} in the \field{struct virtio_net_hdr}
when VIRTIO_NET_F_MRG_RXBUF was negotiated; without that feature the
structure was 2 bytes shorter.

When using the legacy interface, the driver SHOULD ignore the
used length for the transmit queues
and the controlq queue.
\begin{note}
Historically, some devices put
the total descriptor length there, even though no data was
actually written.
\end{note}

\subsubsection{Packet Transmission}\label{sec:Device Types / Network Device / Device Operation / Packet Transmission}

Transmitting a single packet is simple, but varies depending on
the different features the driver negotiated.

\begin{enumerate}
\item The driver can send a completely checksummed packet.  In this case,
  \field{flags} will be zero, and \field{gso_type} will be VIRTIO_NET_HDR_GSO_NONE.

\item If the driver negotiated VIRTIO_NET_F_CSUM, it can skip
  checksumming the packet:
  \begin{itemize}
  \item \field{flags} has the VIRTIO_NET_HDR_F_NEEDS_CSUM set,

  \item \field{csum_start} is set to the offset within the packet to begin checksumming,
    and

  \item \field{csum_offset} indicates how many bytes after the csum_start the
    new (16 bit ones' complement) checksum is placed by the device.

  \item The TCP checksum field in the packet is set to the sum
    of the TCP pseudo header, so that replacing it by the ones'
    complement checksum of the TCP header and body will give the
    correct result.
  \end{itemize}

\begin{note}
For example, consider a partially checksummed TCP (IPv4) packet.
It will have a 14 byte ethernet header and 20 byte IP header
followed by the TCP header (with the TCP checksum field 16 bytes
into that header). \field{csum_start} will be 14+20 = 34 (the TCP
checksum includes the header), and \field{csum_offset} will be 16.
If the given packet has the VIRTIO_NET_HDR_GSO_UDP_TUNNEL_IPV4 bit or the
VIRTIO_NET_HDR_GSO_UDP_TUNNEL_IPV6 bit set,
the above checksum fields refer to the inner header checksum, see
the example below.
\end{note}

\item If the driver negotiated
  VIRTIO_NET_F_HOST_TSO4, TSO6, USO or UFO, and the packet requires
  TCP segmentation, UDP segmentation or fragmentation, then \field{gso_type}
  is set to VIRTIO_NET_HDR_GSO_TCPV4, TCPV6, UDP_L4 or UDP.
  (Otherwise, it is set to VIRTIO_NET_HDR_GSO_NONE). In this
  case, packets larger than 1514 bytes can be transmitted: the
  metadata indicates how to replicate the packet header to cut it
  into smaller packets. The other gso fields are set:

  \begin{itemize}
  \item If the VIRTIO_NET_F_GUEST_HDRLEN feature has been negotiated,
    \field{hdr_len} indicates the header length that needs to be replicated
    for each packet. It's the number of bytes from the beginning of the packet
    to the beginning of the transport payload.
    If the \field{gso_type} has the VIRTIO_NET_HDR_GSO_UDP_TUNNEL_IPV4 bit or
    VIRTIO_NET_HDR_GSO_UDP_TUNNEL_IPV6 bit set, \field{hdr_len} accounts for
    all the headers up to and including the inner transport.
    Otherwise, if the VIRTIO_NET_F_GUEST_HDRLEN feature has not been negotiated,
    \field{hdr_len} is a hint to the device as to how much of the header
    needs to be kept to copy into each packet, usually set to the
    length of the headers, including the transport header\footnote{Due to various bugs in implementations, this field is not useful
as a guarantee of the transport header size.
}.

  \begin{note}
  Some devices benefit from knowledge of the exact header length.
  \end{note}

  \item \field{gso_size} is the maximum size of each packet beyond that
    header (ie. MSS).

  \item If the driver negotiated the VIRTIO_NET_F_HOST_ECN feature,
    the VIRTIO_NET_HDR_GSO_ECN bit in \field{gso_type}
    indicates that the TCP packet has the ECN bit set\footnote{This case is not handled by some older hardware, so is called out
specifically in the protocol.}.
   \end{itemize}

\item If the driver negotiated the VIRTIO_NET_F_HOST_UDP_TUNNEL_GSO feature and the
  \field{gso_type} has the VIRTIO_NET_HDR_GSO_UDP_TUNNEL_IPV4 bit or
  VIRTIO_NET_HDR_GSO_UDP_TUNNEL_IPV6 bit set, the GSO protocol is encapsulated
  in a UDP tunnel.
  If the outer UDP header requires checksumming, the driver must have
  additionally negotiated the VIRTIO_NET_F_HOST_UDP_TUNNEL_GSO_CSUM feature
  and offloaded the outer checksum accordingly, otherwise
  the outer UDP header must not require checksum validation, i.e. the outer
  UDP checksum must be positive zero (0x0) as defined in UDP RFC 768.
  The other tunnel-related fields indicate how to replicate the packet
  headers to cut it into smaller packets:

  \begin{itemize}
  \item \field{outer_th_offset} field indicates the outer transport header within
      the packet. This field differs from \field{csum_start} as the latter
      points to the inner transport header within the packet.

  \item \field{inner_nh_offset} field indicates the inner network header within
      the packet.
  \end{itemize}

\begin{note}
For example, consider a partially checksummed TCP (IPv4) packet carried over a
Geneve UDP tunnel (again IPv4) with no tunnel options. The
only relevant variable related to the tunnel type is the tunnel header length.
The packet will have a 14 byte outer ethernet header, 20 byte outer IP header
followed by the 8 byte UDP header (with a 0 checksum value), 8 byte Geneve header,
14 byte inner ethernet header, 20 byte inner IP header
and the TCP header (with the TCP checksum field 16 bytes
into that header). \field{csum_start} will be 14+20+8+8+14+20 = 84 (the TCP
checksum includes the header), \field{csum_offset} will be 16.
\field{inner_nh_offset} will be 14+20+8+8+14 = 62, \field{outer_th_offset} will be
14+20+8 = 42 and \field{gso_type} will be
VIRTIO_NET_HDR_GSO_TCPV4 | VIRTIO_NET_HDR_GSO_UDP_TUNNEL_IPV4 = 0x21
\end{note}

\item If the driver negotiated the VIRTIO_NET_F_HOST_UDP_TUNNEL_GSO_CSUM feature,
  the transmitted packet is a GSO one encapsulated in a UDP tunnel, and
  the outer UDP header requires checksumming, the driver can skip checksumming
  the outer header:

  \begin{itemize}
  \item \field{flags} has the VIRTIO_NET_HDR_F_UDP_TUNNEL_CSUM set,

  \item The outer UDP checksum field in the packet is set to the sum
    of the UDP pseudo header, so that replacing it by the ones'
    complement checksum of the outer UDP header and payload will give the
    correct result.
  \end{itemize}

\item \field{num_buffers} is set to zero.  This field is unused on transmitted packets.

\item The header and packet are added as one output descriptor to the
  transmitq, and the device is notified of the new entry
  (see \ref{sec:Device Types / Network Device / Device Initialization}~\nameref{sec:Device Types / Network Device / Device Initialization}).
\end{enumerate}

\drivernormative{\paragraph}{Packet Transmission}{Device Types / Network Device / Device Operation / Packet Transmission}

For the transmit packet buffer, the driver MUST use the size of the
structure \field{struct virtio_net_hdr} same as the receive packet buffer.

The driver MUST set \field{num_buffers} to zero.

If VIRTIO_NET_F_CSUM is not negotiated, the driver MUST set
\field{flags} to zero and SHOULD supply a fully checksummed
packet to the device.

If VIRTIO_NET_F_HOST_TSO4 is negotiated, the driver MAY set
\field{gso_type} to VIRTIO_NET_HDR_GSO_TCPV4 to request TCPv4
segmentation, otherwise the driver MUST NOT set
\field{gso_type} to VIRTIO_NET_HDR_GSO_TCPV4.

If VIRTIO_NET_F_HOST_TSO6 is negotiated, the driver MAY set
\field{gso_type} to VIRTIO_NET_HDR_GSO_TCPV6 to request TCPv6
segmentation, otherwise the driver MUST NOT set
\field{gso_type} to VIRTIO_NET_HDR_GSO_TCPV6.

If VIRTIO_NET_F_HOST_UFO is negotiated, the driver MAY set
\field{gso_type} to VIRTIO_NET_HDR_GSO_UDP to request UDP
fragmentation, otherwise the driver MUST NOT set
\field{gso_type} to VIRTIO_NET_HDR_GSO_UDP.

If VIRTIO_NET_F_HOST_USO is negotiated, the driver MAY set
\field{gso_type} to VIRTIO_NET_HDR_GSO_UDP_L4 to request UDP
segmentation, otherwise the driver MUST NOT set
\field{gso_type} to VIRTIO_NET_HDR_GSO_UDP_L4.

The driver SHOULD NOT send to the device TCP packets requiring segmentation offload
which have the Explicit Congestion Notification bit set, unless the
VIRTIO_NET_F_HOST_ECN feature is negotiated, in which case the
driver MUST set the VIRTIO_NET_HDR_GSO_ECN bit in
\field{gso_type}.

If VIRTIO_NET_F_HOST_UDP_TUNNEL_GSO is negotiated, the driver MAY set
VIRTIO_NET_HDR_GSO_UDP_TUNNEL_IPV4 bit or the VIRTIO_NET_HDR_GSO_UDP_TUNNEL_IPV6 bit
in \field{gso_type} according to the inner network header protocol type
to request GSO packets over UDPv4 or UDPv6 tunnel segmentation,
otherwise the driver MUST NOT set either the
VIRTIO_NET_HDR_GSO_UDP_TUNNEL_IPV4 bit or the VIRTIO_NET_HDR_GSO_UDP_TUNNEL_IPV6 bit
in \field{gso_type}.

When requesting GSO segmentation over UDP tunnel, the driver MUST SET the
VIRTIO_NET_HDR_GSO_UDP_TUNNEL_IPV4 bit if the inner network header is IPv4, i.e. the
packet is a TCPv4 GSO one, otherwise, if the inner network header is IPv6, the driver
MUST SET the VIRTIO_NET_HDR_GSO_UDP_TUNNEL_IPV6 bit.

The driver MUST NOT send to the device GSO packets over UDP tunnel
requiring segmentation and outer UDP checksum offload, unless both the
VIRTIO_NET_F_HOST_UDP_TUNNEL_GSO and VIRTIO_NET_F_HOST_UDP_TUNNEL_GSO_CSUM features
are negotiated, in which case the driver MUST set either the
VIRTIO_NET_HDR_GSO_UDP_TUNNEL_IPV4 bit or the VIRTIO_NET_HDR_GSO_UDP_TUNNEL_IPV6
bit in the \field{gso_type} and the VIRTIO_NET_HDR_F_UDP_TUNNEL_CSUM bit in
the \field{flags}.

If VIRTIO_NET_F_HOST_UDP_TUNNEL_GSO_CSUM is not negotiated, the driver MUST not set
the VIRTIO_NET_HDR_F_UDP_TUNNEL_CSUM bit in the \field{flags} and
MUST NOT send to the device GSO packets over UDP tunnel
requiring segmentation and outer UDP checksum offload.

The driver MUST NOT set the VIRTIO_NET_HDR_GSO_UDP_TUNNEL_IPV4 bit or the
VIRTIO_NET_HDR_GSO_UDP_TUNNEL_IPV6 bit together with VIRTIO_NET_HDR_GSO_UDP, as the
latter is deprecated in favor of UDP_L4 and no new feature will support it.

The driver MUST NOT set the VIRTIO_NET_HDR_GSO_UDP_TUNNEL_IPV4 bit and the
VIRTIO_NET_HDR_GSO_UDP_TUNNEL_IPV6 bit together.

The driver MUST NOT set the VIRTIO_NET_HDR_F_UDP_TUNNEL_CSUM bit \field{flags}
without setting either the VIRTIO_NET_HDR_GSO_UDP_TUNNEL_IPV4 bit or
the VIRTIO_NET_HDR_GSO_UDP_TUNNEL_IPV6 bit in \field{gso_type}.

If the VIRTIO_NET_F_CSUM feature has been negotiated, the
driver MAY set the VIRTIO_NET_HDR_F_NEEDS_CSUM bit in
\field{flags}, if so:
\begin{enumerate}
\item the driver MUST validate the packet checksum at
	offset \field{csum_offset} from \field{csum_start} as well as all
	preceding offsets;
\begin{note}
If \field{gso_type} differs from VIRTIO_NET_HDR_GSO_NONE and the
VIRTIO_NET_HDR_GSO_UDP_TUNNEL_IPV4 bit or the VIRTIO_NET_HDR_GSO_UDP_TUNNEL_IPV6
bit are not set in \field{gso_type}, \field{csum_offset}
points to the only transport header present in the packet, and there are no
additional preceding checksums validated by VIRTIO_NET_HDR_F_NEEDS_CSUM.
\end{note}
\item the driver MUST set the packet checksum stored in the
	buffer to the TCP/UDP pseudo header;
\item the driver MUST set \field{csum_start} and
	\field{csum_offset} such that calculating a ones'
	complement checksum from \field{csum_start} up until the end of
	the packet and storing the result at offset \field{csum_offset}
	from  \field{csum_start} will result in a fully checksummed
	packet;
\end{enumerate}

If none of the VIRTIO_NET_F_HOST_TSO4, TSO6, USO or UFO options have
been negotiated, the driver MUST set \field{gso_type} to
VIRTIO_NET_HDR_GSO_NONE.

If \field{gso_type} differs from VIRTIO_NET_HDR_GSO_NONE, then
the driver MUST also set the VIRTIO_NET_HDR_F_NEEDS_CSUM bit in
\field{flags} and MUST set \field{gso_size} to indicate the
desired MSS.

If one of the VIRTIO_NET_F_HOST_TSO4, TSO6, USO or UFO options have
been negotiated:
\begin{itemize}
\item If the VIRTIO_NET_F_GUEST_HDRLEN feature has been negotiated,
	and \field{gso_type} differs from VIRTIO_NET_HDR_GSO_NONE,
	the driver MUST set \field{hdr_len} to a value equal to the length
	of the headers, including the transport header. If \field{gso_type}
	has the VIRTIO_NET_HDR_GSO_UDP_TUNNEL_IPV4 bit or the
	VIRTIO_NET_HDR_GSO_UDP_TUNNEL_IPV6 bit set, \field{hdr_len} includes
	the inner transport header.

\item If the VIRTIO_NET_F_GUEST_HDRLEN feature has not been negotiated,
	or \field{gso_type} is VIRTIO_NET_HDR_GSO_NONE,
	the driver SHOULD set \field{hdr_len} to a value
	not less than the length of the headers, including the transport
	header.
\end{itemize}

If the VIRTIO_NET_F_HOST_UDP_TUNNEL_GSO option has been negotiated, the
driver MAY set the VIRTIO_NET_HDR_GSO_UDP_TUNNEL_IPV4 bit or the
VIRTIO_NET_HDR_GSO_UDP_TUNNEL_IPV6 bit in \field{gso_type}, if so:
\begin{itemize}
\item the driver MUST set \field{outer_th_offset} to the outer UDP header
  offset and \field{inner_nh_offset} to the inner network header offset.
  The \field{csum_start} and \field{csum_offset} fields point respectively
  to the inner transport header and inner transport checksum field.
\end{itemize}

If the VIRTIO_NET_F_HOST_UDP_TUNNEL_GSO_CSUM feature has been negotiated,
and the VIRTIO_NET_HDR_GSO_UDP_TUNNEL_IPV4 bit or
VIRTIO_NET_HDR_GSO_UDP_TUNNEL_IPV6 bit in \field{gso_type} are set,
the driver MAY set the VIRTIO_NET_HDR_F_UDP_TUNNEL_CSUM bit in
\field{flags}, if so the driver MUST set the packet outer UDP header checksum
to the outer UDP pseudo header checksum.

\begin{note}
calculating a ones' complement checksum from \field{outer_th_offset}
up until the end of the packet and storing the result at offset 6
from \field{outer_th_offset} will result in a fully checksummed outer UDP packet;
\end{note}

If the VIRTIO_NET_HDR_GSO_UDP_TUNNEL_IPV4 bit or the
VIRTIO_NET_HDR_GSO_UDP_TUNNEL_IPV6 bit in \field{gso_type} are set
and the VIRTIO_NET_F_HOST_UDP_TUNNEL_GSO_CSUM feature has not
been negotiated, the
outer UDP header MUST NOT require checksum validation. That is, the
outer UDP checksum value MUST be 0 or the validated complete checksum
for such header.

\begin{note}
The valid complete checksum of the outer UDP header of individual segments
can be computed by the driver prior to segmentation only if the GSO packet
size is a multiple of \field{gso_size}, because then all segments
have the same size and thus all data included in the outer UDP
checksum is the same for every segment. These pre-computed segment
length and checksum fields are different from those of the GSO
packet.
In this scenario the outer UDP header of the GSO packet must carry the
segmented UDP packet length.
\end{note}

If the VIRTIO_NET_F_HOST_UDP_TUNNEL_GSO option has not
been negotiated, the driver MUST NOT set either the VIRTIO_NET_HDR_F_GSO_UDP_TUNNEL_IPV4
bit or the VIRTIO_NET_HDR_F_GSO_UDP_TUNNEL_IPV6 in \field{gso_type}.

If the VIRTIO_NET_F_HOST_UDP_TUNNEL_GSO_CSUM option has not been negotiated,
the driver MUST NOT set the VIRTIO_NET_HDR_F_UDP_TUNNEL_CSUM bit
in \field{flags}.

The driver SHOULD accept the VIRTIO_NET_F_GUEST_HDRLEN feature if it has
been offered, and if it's able to provide the exact header length.

The driver MUST NOT set the VIRTIO_NET_HDR_F_DATA_VALID and
VIRTIO_NET_HDR_F_RSC_INFO bits in \field{flags}.

The driver MUST NOT set the VIRTIO_NET_HDR_F_DATA_VALID bit in \field{flags}
together with the VIRTIO_NET_HDR_F_GSO_UDP_TUNNEL_IPV4 bit or the
VIRTIO_NET_HDR_F_GSO_UDP_TUNNEL_IPV6 bit in \field{gso_type}.

\devicenormative{\paragraph}{Packet Transmission}{Device Types / Network Device / Device Operation / Packet Transmission}
The device MUST ignore \field{flag} bits that it does not recognize.

If VIRTIO_NET_HDR_F_NEEDS_CSUM bit in \field{flags} is not set, the
device MUST NOT use the \field{csum_start} and \field{csum_offset}.

If one of the VIRTIO_NET_F_HOST_TSO4, TSO6, USO or UFO options have
been negotiated:
\begin{itemize}
\item If the VIRTIO_NET_F_GUEST_HDRLEN feature has been negotiated,
	and \field{gso_type} differs from VIRTIO_NET_HDR_GSO_NONE,
	the device MAY use \field{hdr_len} as the transport header size.

	\begin{note}
	Caution should be taken by the implementation so as to prevent
	a malicious driver from attacking the device by setting an incorrect hdr_len.
	\end{note}

\item If the VIRTIO_NET_F_GUEST_HDRLEN feature has not been negotiated,
	or \field{gso_type} is VIRTIO_NET_HDR_GSO_NONE,
	the device MAY use \field{hdr_len} only as a hint about the
	transport header size.
	The device MUST NOT rely on \field{hdr_len} to be correct.

	\begin{note}
	This is due to various bugs in implementations.
	\end{note}
\end{itemize}

If both the VIRTIO_NET_HDR_GSO_UDP_TUNNEL_IPV4 bit and
the VIRTIO_NET_HDR_GSO_UDP_TUNNEL_IPV6 bit in in \field{gso_type} are set,
the device MUST NOT accept the packet.

If the VIRTIO_NET_HDR_GSO_UDP_TUNNEL_IPV4 bit and the VIRTIO_NET_HDR_GSO_UDP_TUNNEL_IPV6
bit in \field{gso_type} are not set, the device MUST NOT use the
\field{outer_th_offset} and \field{inner_nh_offset}.

If either the VIRTIO_NET_HDR_GSO_UDP_TUNNEL_IPV4 bit or
the VIRTIO_NET_HDR_GSO_UDP_TUNNEL_IPV6 bit in \field{gso_type} are set, and any of
the following is true:
\begin{itemize}
\item the VIRTIO_NET_HDR_F_NEEDS_CSUM is not set in \field{flags}
\item the VIRTIO_NET_HDR_F_DATA_VALID is set in \field{flags}
\item the \field{gso_type} excluding the VIRTIO_NET_HDR_GSO_UDP_TUNNEL_IPV4
bit and the VIRTIO_NET_HDR_GSO_UDP_TUNNEL_IPV6 bit is VIRTIO_NET_HDR_GSO_NONE
\end{itemize}
the device MUST NOT accept the packet.

If the VIRTIO_NET_HDR_F_UDP_TUNNEL_CSUM bit in \field{flags} is set,
and both the bits VIRTIO_NET_HDR_GSO_UDP_TUNNEL_IPV4 and
VIRTIO_NET_HDR_GSO_UDP_TUNNEL_IPV6 in \field{gso_type} are not set,
the device MOST NOT accept the packet.

If VIRTIO_NET_HDR_F_NEEDS_CSUM is not set, the device MUST NOT
rely on the packet checksum being correct.
\paragraph{Packet Transmission Interrupt}\label{sec:Device Types / Network Device / Device Operation / Packet Transmission / Packet Transmission Interrupt}

Often a driver will suppress transmission virtqueue interrupts
and check for used packets in the transmit path of following
packets.

The normal behavior in this interrupt handler is to retrieve
used buffers from the virtqueue and free the corresponding
headers and packets.

\subsubsection{Setting Up Receive Buffers}\label{sec:Device Types / Network Device / Device Operation / Setting Up Receive Buffers}

It is generally a good idea to keep the receive virtqueue as
fully populated as possible: if it runs out, network performance
will suffer.

If the VIRTIO_NET_F_GUEST_TSO4, VIRTIO_NET_F_GUEST_TSO6,
VIRTIO_NET_F_GUEST_UFO, VIRTIO_NET_F_GUEST_USO4 or VIRTIO_NET_F_GUEST_USO6
features are used, the maximum incoming packet
will be 65589 bytes long (14 bytes of Ethernet header, plus 40 bytes of
the IPv6 header, plus 65535 bytes of maximum IPv6 payload including any
extension header), otherwise 1514 bytes.
When VIRTIO_NET_F_HASH_REPORT is not negotiated, the required receive buffer
size is either 65601 or 1526 bytes accounting for 20 bytes of
\field{struct virtio_net_hdr} followed by receive packet.
When VIRTIO_NET_F_HASH_REPORT is negotiated, the required receive buffer
size is either 65609 or 1534 bytes accounting for 12 bytes of
\field{struct virtio_net_hdr} followed by receive packet.

\drivernormative{\paragraph}{Setting Up Receive Buffers}{Device Types / Network Device / Device Operation / Setting Up Receive Buffers}

\begin{itemize}
\item If VIRTIO_NET_F_MRG_RXBUF is not negotiated:
  \begin{itemize}
    \item If VIRTIO_NET_F_GUEST_TSO4, VIRTIO_NET_F_GUEST_TSO6, VIRTIO_NET_F_GUEST_UFO,
	VIRTIO_NET_F_GUEST_USO4 or VIRTIO_NET_F_GUEST_USO6 are negotiated, the driver SHOULD populate
      the receive queue(s) with buffers of at least 65609 bytes if
      VIRTIO_NET_F_HASH_REPORT is negotiated, and of at least 65601 bytes if not.
    \item Otherwise, the driver SHOULD populate the receive queue(s)
      with buffers of at least 1534 bytes if VIRTIO_NET_F_HASH_REPORT
      is negotiated, and of at least 1526 bytes if not.
  \end{itemize}
\item If VIRTIO_NET_F_MRG_RXBUF is negotiated, each buffer MUST be at
least size of \field{struct virtio_net_hdr},
i.e. 20 bytes if VIRTIO_NET_F_HASH_REPORT is negotiated, and 12 bytes if not.
\end{itemize}

\begin{note}
Obviously each buffer can be split across multiple descriptor elements.
\end{note}

When calculating the size of \field{struct virtio_net_hdr}, the driver
MUST consider all the fields inclusive up to \field{padding_reserved},
i.e. 20 bytes if VIRTIO_NET_F_HASH_REPORT is negotiated, and 12 bytes if not.

If VIRTIO_NET_F_MQ is negotiated, each of receiveq1\ldots receiveqN
that will be used SHOULD be populated with receive buffers.

\devicenormative{\paragraph}{Setting Up Receive Buffers}{Device Types / Network Device / Device Operation / Setting Up Receive Buffers}

The device MUST set \field{num_buffers} to the number of descriptors used to
hold the incoming packet.

The device MUST use only a single descriptor if VIRTIO_NET_F_MRG_RXBUF
was not negotiated.
\begin{note}
{This means that \field{num_buffers} will always be 1
if VIRTIO_NET_F_MRG_RXBUF is not negotiated.}
\end{note}

\subsubsection{Processing of Incoming Packets}\label{sec:Device Types / Network Device / Device Operation / Processing of Incoming Packets}
\label{sec:Device Types / Network Device / Device Operation / Processing of Packets}%old label for latexdiff

When a packet is copied into a buffer in the receiveq, the
optimal path is to disable further used buffer notifications for the
receiveq and process packets until no more are found, then re-enable
them.

Processing incoming packets involves:

\begin{enumerate}
\item \field{num_buffers} indicates how many descriptors
  this packet is spread over (including this one): this will
  always be 1 if VIRTIO_NET_F_MRG_RXBUF was not negotiated.
  This allows receipt of large packets without having to allocate large
  buffers: a packet that does not fit in a single buffer can flow
  over to the next buffer, and so on. In this case, there will be
  at least \field{num_buffers} used buffers in the virtqueue, and the device
  chains them together to form a single packet in a way similar to
  how it would store it in a single buffer spread over multiple
  descriptors.
  The other buffers will not begin with a \field{struct virtio_net_hdr}.

\item If
  \field{num_buffers} is one, then the entire packet will be
  contained within this buffer, immediately following the struct
  virtio_net_hdr.
\item If the VIRTIO_NET_F_GUEST_CSUM feature was negotiated, the
  VIRTIO_NET_HDR_F_DATA_VALID bit in \field{flags} can be
  set: if so, device has validated the packet checksum.
  If the VIRTIO_NET_F_GUEST_UDP_TUNNEL_GSO_CSUM feature has been negotiated,
  and the VIRTIO_NET_HDR_F_UDP_TUNNEL_CSUM bit is set in \field{flags},
  both the outer UDP checksum and the inner transport checksum
  have been validated, otherwise only one level of checksums (the outer one
  in case of tunnels) has been validated.
\end{enumerate}

Additionally, VIRTIO_NET_F_GUEST_CSUM, TSO4, TSO6, UDP, UDP_TUNNEL
and ECN features enable receive checksum, large receive offload and ECN
support which are the input equivalents of the transmit checksum,
transmit segmentation offloading and ECN features, as described
in \ref{sec:Device Types / Network Device / Device Operation /
Packet Transmission}:
\begin{enumerate}
\item If the VIRTIO_NET_F_GUEST_TSO4, TSO6, UFO, USO4 or USO6 options were
  negotiated, then \field{gso_type} MAY be something other than
  VIRTIO_NET_HDR_GSO_NONE, and \field{gso_size} field indicates the
  desired MSS (see Packet Transmission point 2).
\item If the VIRTIO_NET_F_RSC_EXT option was negotiated (this
  implies one of VIRTIO_NET_F_GUEST_TSO4, TSO6), the
  device processes also duplicated ACK segments, reports
  number of coalesced TCP segments in \field{csum_start} field and
  number of duplicated ACK segments in \field{csum_offset} field
  and sets bit VIRTIO_NET_HDR_F_RSC_INFO in \field{flags}.
\item If the VIRTIO_NET_F_GUEST_CSUM feature was negotiated, the
  VIRTIO_NET_HDR_F_NEEDS_CSUM bit in \field{flags} can be
  set: if so, the packet checksum at offset \field{csum_offset}
  from \field{csum_start} and any preceding checksums
  have been validated.  The checksum on the packet is incomplete and
  if bit VIRTIO_NET_HDR_F_RSC_INFO is not set in \field{flags},
  then \field{csum_start} and \field{csum_offset} indicate how to calculate it
  (see Packet Transmission point 1).
\begin{note}
If \field{gso_type} differs from VIRTIO_NET_HDR_GSO_NONE and the
VIRTIO_NET_HDR_GSO_UDP_TUNNEL_IPV4 bit or the VIRTIO_NET_HDR_GSO_UDP_TUNNEL_IPV6
bit are not set, \field{csum_offset}
points to the only transport header present in the packet, and there are no
additional preceding checksums validated by VIRTIO_NET_HDR_F_NEEDS_CSUM.
\end{note}
\item If the VIRTIO_NET_F_GUEST_UDP_TUNNEL_GSO option was negotiated and
  \field{gso_type} is not VIRTIO_NET_HDR_GSO_NONE, the
  VIRTIO_NET_HDR_GSO_UDP_TUNNEL_IPV4 bit or the VIRTIO_NET_HDR_GSO_UDP_TUNNEL_IPV6
  bit MAY be set. In such case the \field{outer_th_offset} and
  \field{inner_nh_offset} fields indicate the corresponding
  headers information.
\item If the VIRTIO_NET_F_GUEST_UDP_TUNNEL_GSO_CSUM feature was
negotiated, and
  the VIRTIO_NET_HDR_GSO_UDP_TUNNEL_IPV4 bit or the VIRTIO_NET_HDR_GSO_UDP_TUNNEL_IPV6
  are set in \field{gso_type}, the VIRTIO_NET_HDR_F_UDP_TUNNEL_CSUM bit in the
  \field{flags} can be set: if so, the outer UDP checksum has been validated
  and the UDP header checksum at offset 6 from from \field{outer_th_offset}
  is set to the outer UDP pseudo header checksum.

\begin{note}
If the VIRTIO_NET_HDR_GSO_UDP_TUNNEL_IPV4 bit or VIRTIO_NET_HDR_GSO_UDP_TUNNEL_IPV6
bit are set in \field{gso_type}, the \field{csum_start} field refers to
the inner transport header offset (see Packet Transmission point 1).
If the VIRTIO_NET_HDR_F_UDP_TUNNEL_CSUM bit in \field{flags} is set both
the inner and the outer header checksums have been validated by
VIRTIO_NET_HDR_F_NEEDS_CSUM, otherwise only the inner transport header
checksum has been validated.
\end{note}
\end{enumerate}

If applicable, the device calculates per-packet hash for incoming packets as
defined in \ref{sec:Device Types / Network Device / Device Operation / Processing of Incoming Packets / Hash calculation for incoming packets}.

If applicable, the device reports hash information for incoming packets as
defined in \ref{sec:Device Types / Network Device / Device Operation / Processing of Incoming Packets / Hash reporting for incoming packets}.

\devicenormative{\paragraph}{Processing of Incoming Packets}{Device Types / Network Device / Device Operation / Processing of Incoming Packets}
\label{devicenormative:Device Types / Network Device / Device Operation / Processing of Packets}%old label for latexdiff

If VIRTIO_NET_F_MRG_RXBUF has not been negotiated, the device MUST set
\field{num_buffers} to 1.

If VIRTIO_NET_F_MRG_RXBUF has been negotiated, the device MUST set
\field{num_buffers} to indicate the number of buffers
the packet (including the header) is spread over.

If a receive packet is spread over multiple buffers, the device
MUST use all buffers but the last (i.e. the first \field{num_buffers} -
1 buffers) completely up to the full length of each buffer
supplied by the driver.

The device MUST use all buffers used by a single receive
packet together, such that at least \field{num_buffers} are
observed by driver as used.

If VIRTIO_NET_F_GUEST_CSUM is not negotiated, the device MUST set
\field{flags} to zero and SHOULD supply a fully checksummed
packet to the driver.

If VIRTIO_NET_F_GUEST_TSO4 is not negotiated, the device MUST NOT set
\field{gso_type} to VIRTIO_NET_HDR_GSO_TCPV4.

If VIRTIO_NET_F_GUEST_UDP is not negotiated, the device MUST NOT set
\field{gso_type} to VIRTIO_NET_HDR_GSO_UDP.

If VIRTIO_NET_F_GUEST_TSO6 is not negotiated, the device MUST NOT set
\field{gso_type} to VIRTIO_NET_HDR_GSO_TCPV6.

If none of VIRTIO_NET_F_GUEST_USO4 or VIRTIO_NET_F_GUEST_USO6 have been negotiated,
the device MUST NOT set \field{gso_type} to VIRTIO_NET_HDR_GSO_UDP_L4.

If VIRTIO_NET_F_GUEST_UDP_TUNNEL_GSO is not negotiated, the device MUST NOT set
either the VIRTIO_NET_HDR_GSO_UDP_TUNNEL_IPV4 bit or the
VIRTIO_NET_HDR_GSO_UDP_TUNNEL_IPV6 bit in \field{gso_type}.

If VIRTIO_NET_F_GUEST_UDP_TUNNEL_GSO_CSUM is not negotiated the device MUST NOT set
the VIRTIO_NET_HDR_F_UDP_TUNNEL_CSUM bit in \field{flags}.

The device SHOULD NOT send to the driver TCP packets requiring segmentation offload
which have the Explicit Congestion Notification bit set, unless the
VIRTIO_NET_F_GUEST_ECN feature is negotiated, in which case the
device MUST set the VIRTIO_NET_HDR_GSO_ECN bit in
\field{gso_type}.

If the VIRTIO_NET_F_GUEST_CSUM feature has been negotiated, the
device MAY set the VIRTIO_NET_HDR_F_NEEDS_CSUM bit in
\field{flags}, if so:
\begin{enumerate}
\item the device MUST validate the packet checksum at
	offset \field{csum_offset} from \field{csum_start} as well as all
	preceding offsets;
\item the device MUST set the packet checksum stored in the
	receive buffer to the TCP/UDP pseudo header;
\item the device MUST set \field{csum_start} and
	\field{csum_offset} such that calculating a ones'
	complement checksum from \field{csum_start} up until the
	end of the packet and storing the result at offset
	\field{csum_offset} from  \field{csum_start} will result in a
	fully checksummed packet;
\end{enumerate}

The device MUST NOT send to the driver GSO packets encapsulated in UDP
tunnel and requiring segmentation offload, unless the
VIRTIO_NET_F_GUEST_UDP_TUNNEL_GSO is negotiated, in which case the device MUST set
the VIRTIO_NET_HDR_GSO_UDP_TUNNEL_IPV4 bit or the VIRTIO_NET_HDR_GSO_UDP_TUNNEL_IPV6
bit in \field{gso_type} according to the inner network header protocol type,
MUST set the \field{outer_th_offset} and \field{inner_nh_offset} fields
to the corresponding header information, and the outer UDP header MUST NOT
require checksum offload.

If the VIRTIO_NET_F_GUEST_UDP_TUNNEL_GSO_CSUM feature has not been negotiated,
the device MUST NOT send the driver GSO packets encapsulated in UDP
tunnel and requiring segmentation and outer checksum offload.

If none of the VIRTIO_NET_F_GUEST_TSO4, TSO6, UFO, USO4 or USO6 options have
been negotiated, the device MUST set \field{gso_type} to
VIRTIO_NET_HDR_GSO_NONE.

If \field{gso_type} differs from VIRTIO_NET_HDR_GSO_NONE, then
the device MUST also set the VIRTIO_NET_HDR_F_NEEDS_CSUM bit in
\field{flags} MUST set \field{gso_size} to indicate the desired MSS.
If VIRTIO_NET_F_RSC_EXT was negotiated, the device MUST also
set VIRTIO_NET_HDR_F_RSC_INFO bit in \field{flags},
set \field{csum_start} to number of coalesced TCP segments and
set \field{csum_offset} to number of received duplicated ACK segments.

If VIRTIO_NET_F_RSC_EXT was not negotiated, the device MUST
not set VIRTIO_NET_HDR_F_RSC_INFO bit in \field{flags}.

If one of the VIRTIO_NET_F_GUEST_TSO4, TSO6, UFO, USO4 or USO6 options have
been negotiated, the device SHOULD set \field{hdr_len} to a value
not less than the length of the headers, including the transport
header. If \field{gso_type} has the VIRTIO_NET_HDR_GSO_UDP_TUNNEL_IPV4 bit
or the VIRTIO_NET_HDR_GSO_UDP_TUNNEL_IPV6 bit set, the referenced transport
header is the inner one.

If the VIRTIO_NET_F_GUEST_CSUM feature has been negotiated, the
device MAY set the VIRTIO_NET_HDR_F_DATA_VALID bit in
\field{flags}, if so, the device MUST validate the packet
checksum. If the VIRTIO_NET_F_GUEST_UDP_TUNNEL_GSO_CSUM feature has
been negotiated, and the VIRTIO_NET_HDR_F_UDP_TUNNEL_CSUM bit set in
\field{flags}, both the outer UDP checksum and the inner transport
checksum have been validated.
Otherwise level of checksum is validated: in case of multiple
encapsulated protocols the outermost one.

If either the VIRTIO_NET_HDR_GSO_UDP_TUNNEL_IPV4 bit or the
VIRTIO_NET_HDR_GSO_UDP_TUNNEL_IPV6 bit in \field{gso_type} are set,
the device MUST NOT set the VIRTIO_NET_HDR_F_DATA_VALID bit in
\field{flags}.

If the VIRTIO_NET_F_GUEST_UDP_TUNNEL_GSO_CSUM feature has been negotiated
and either the VIRTIO_NET_HDR_GSO_UDP_TUNNEL_IPV4 bit is set or the
VIRTIO_NET_HDR_GSO_UDP_TUNNEL_IPV6 bit is set in \field{gso_type}, the
device MAY set the VIRTIO_NET_HDR_F_UDP_TUNNEL_CSUM bit in
\field{flags}, if so the device MUST set the packet outer UDP checksum
stored in the receive buffer to the outer UDP pseudo header.

Otherwise, the VIRTIO_NET_F_GUEST_UDP_TUNNEL_GSO_CSUM feature has been
negotiated, either the VIRTIO_NET_HDR_GSO_UDP_TUNNEL_IPV4 bit is set or the
VIRTIO_NET_HDR_GSO_UDP_TUNNEL_IPV6 bit is set in \field{gso_type},
and the bit VIRTIO_NET_HDR_F_UDP_TUNNEL_CSUM is not set in
\field{flags}, the device MUST either provide a zero outer UDP header
checksum or a fully checksummed outer UDP header.

\drivernormative{\paragraph}{Processing of Incoming
Packets}{Device Types / Network Device / Device Operation /
Processing of Incoming Packets}

The driver MUST ignore \field{flag} bits that it does not recognize.

If VIRTIO_NET_HDR_F_NEEDS_CSUM bit in \field{flags} is not set or
if VIRTIO_NET_HDR_F_RSC_INFO bit \field{flags} is set, the
driver MUST NOT use the \field{csum_start} and \field{csum_offset}.

If one of the VIRTIO_NET_F_GUEST_TSO4, TSO6, UFO, USO4 or USO6 options have
been negotiated, the driver MAY use \field{hdr_len} only as a hint about the
transport header size.
The driver MUST NOT rely on \field{hdr_len} to be correct.
\begin{note}
This is due to various bugs in implementations.
\end{note}

If neither VIRTIO_NET_HDR_F_NEEDS_CSUM nor
VIRTIO_NET_HDR_F_DATA_VALID is set, the driver MUST NOT
rely on the packet checksum being correct.

If both the VIRTIO_NET_HDR_GSO_UDP_TUNNEL_IPV4 bit and
the VIRTIO_NET_HDR_GSO_UDP_TUNNEL_IPV6 bit in in \field{gso_type} are set,
the driver MUST NOT accept the packet.

If the VIRTIO_NET_HDR_GSO_UDP_TUNNEL_IPV4 bit or the VIRTIO_NET_HDR_GSO_UDP_TUNNEL_IPV6
bit in \field{gso_type} are not set, the driver MUST NOT use the
\field{outer_th_offset} and \field{inner_nh_offset}.

If either the VIRTIO_NET_HDR_GSO_UDP_TUNNEL_IPV4 bit or
the VIRTIO_NET_HDR_GSO_UDP_TUNNEL_IPV6 bit in \field{gso_type} are set, and any of
the following is true:
\begin{itemize}
\item the VIRTIO_NET_HDR_F_NEEDS_CSUM bit is not set in \field{flags}
\item the VIRTIO_NET_HDR_F_DATA_VALID bit is set in \field{flags}
\item the \field{gso_type} excluding the VIRTIO_NET_HDR_GSO_UDP_TUNNEL_IPV4
bit and the VIRTIO_NET_HDR_GSO_UDP_TUNNEL_IPV6 bit is VIRTIO_NET_HDR_GSO_NONE
\end{itemize}
the driver MUST NOT accept the packet.

If the VIRTIO_NET_HDR_F_UDP_TUNNEL_CSUM bit and the VIRTIO_NET_HDR_F_NEEDS_CSUM
bit in \field{flags} are set,
and both the bits VIRTIO_NET_HDR_GSO_UDP_TUNNEL_IPV4 and
VIRTIO_NET_HDR_GSO_UDP_TUNNEL_IPV6 in \field{gso_type} are not set,
the driver MOST NOT accept the packet.

\paragraph{Hash calculation for incoming packets}
\label{sec:Device Types / Network Device / Device Operation / Processing of Incoming Packets / Hash calculation for incoming packets}

A device attempts to calculate a per-packet hash in the following cases:
\begin{itemize}
\item The feature VIRTIO_NET_F_RSS was negotiated. The device uses the hash to determine the receive virtqueue to place incoming packets.
\item The feature VIRTIO_NET_F_HASH_REPORT was negotiated. The device reports the hash value and the hash type with the packet.
\end{itemize}

If the feature VIRTIO_NET_F_RSS was negotiated:
\begin{itemize}
\item The device uses \field{hash_types} of the virtio_net_rss_config structure as 'Enabled hash types' bitmask.
\item If additionally the feature VIRTIO_NET_F_HASH_TUNNEL was negotiated, the device uses \field{enabled_tunnel_types} of the
      virtnet_hash_tunnel structure as 'Encapsulation types enabled for inner header hash' bitmask.
\item The device uses a key as defined in \field{hash_key_data} and \field{hash_key_length} of the virtio_net_rss_config structure (see
\ref{sec:Device Types / Network Device / Device Operation / Control Virtqueue / Receive-side scaling (RSS) / Setting RSS parameters}).
\end{itemize}

If the feature VIRTIO_NET_F_RSS was not negotiated:
\begin{itemize}
\item The device uses \field{hash_types} of the virtio_net_hash_config structure as 'Enabled hash types' bitmask.
\item If additionally the feature VIRTIO_NET_F_HASH_TUNNEL was negotiated, the device uses \field{enabled_tunnel_types} of the
      virtnet_hash_tunnel structure as 'Encapsulation types enabled for inner header hash' bitmask.
\item The device uses a key as defined in \field{hash_key_data} and \field{hash_key_length} of the virtio_net_hash_config structure (see
\ref{sec:Device Types / Network Device / Device Operation / Control Virtqueue / Automatic receive steering in multiqueue mode / Hash calculation}).
\end{itemize}

Note that if the device offers VIRTIO_NET_F_HASH_REPORT, even if it supports only one pair of virtqueues, it MUST support
at least one of commands of VIRTIO_NET_CTRL_MQ class to configure reported hash parameters:
\begin{itemize}
\item If the device offers VIRTIO_NET_F_RSS, it MUST support VIRTIO_NET_CTRL_MQ_RSS_CONFIG command per
 \ref{sec:Device Types / Network Device / Device Operation / Control Virtqueue / Receive-side scaling (RSS) / Setting RSS parameters}.
\item Otherwise the device MUST support VIRTIO_NET_CTRL_MQ_HASH_CONFIG command per
 \ref{sec:Device Types / Network Device / Device Operation / Control Virtqueue / Automatic receive steering in multiqueue mode / Hash calculation}.
\end{itemize}

The per-packet hash calculation can depend on the IP packet type. See
\hyperref[intro:IP]{[IP]}, \hyperref[intro:UDP]{[UDP]} and \hyperref[intro:TCP]{[TCP]}.

\subparagraph{Supported/enabled hash types}
\label{sec:Device Types / Network Device / Device Operation / Processing of Incoming Packets / Hash calculation for incoming packets / Supported/enabled hash types}
Hash types applicable for IPv4 packets:
\begin{lstlisting}
#define VIRTIO_NET_HASH_TYPE_IPv4              (1 << 0)
#define VIRTIO_NET_HASH_TYPE_TCPv4             (1 << 1)
#define VIRTIO_NET_HASH_TYPE_UDPv4             (1 << 2)
\end{lstlisting}
Hash types applicable for IPv6 packets without extension headers
\begin{lstlisting}
#define VIRTIO_NET_HASH_TYPE_IPv6              (1 << 3)
#define VIRTIO_NET_HASH_TYPE_TCPv6             (1 << 4)
#define VIRTIO_NET_HASH_TYPE_UDPv6             (1 << 5)
\end{lstlisting}
Hash types applicable for IPv6 packets with extension headers
\begin{lstlisting}
#define VIRTIO_NET_HASH_TYPE_IP_EX             (1 << 6)
#define VIRTIO_NET_HASH_TYPE_TCP_EX            (1 << 7)
#define VIRTIO_NET_HASH_TYPE_UDP_EX            (1 << 8)
\end{lstlisting}

\subparagraph{IPv4 packets}
\label{sec:Device Types / Network Device / Device Operation / Processing of Incoming Packets / Hash calculation for incoming packets / IPv4 packets}
The device calculates the hash on IPv4 packets according to 'Enabled hash types' bitmask as follows:
\begin{itemize}
\item If VIRTIO_NET_HASH_TYPE_TCPv4 is set and the packet has
a TCP header, the hash is calculated over the following fields:
\begin{itemize}
\item Source IP address
\item Destination IP address
\item Source TCP port
\item Destination TCP port
\end{itemize}
\item Else if VIRTIO_NET_HASH_TYPE_UDPv4 is set and the
packet has a UDP header, the hash is calculated over the following fields:
\begin{itemize}
\item Source IP address
\item Destination IP address
\item Source UDP port
\item Destination UDP port
\end{itemize}
\item Else if VIRTIO_NET_HASH_TYPE_IPv4 is set, the hash is
calculated over the following fields:
\begin{itemize}
\item Source IP address
\item Destination IP address
\end{itemize}
\item Else the device does not calculate the hash
\end{itemize}

\subparagraph{IPv6 packets without extension header}
\label{sec:Device Types / Network Device / Device Operation / Processing of Incoming Packets / Hash calculation for incoming packets / IPv6 packets without extension header}
The device calculates the hash on IPv6 packets without extension
headers according to 'Enabled hash types' bitmask as follows:
\begin{itemize}
\item If VIRTIO_NET_HASH_TYPE_TCPv6 is set and the packet has
a TCPv6 header, the hash is calculated over the following fields:
\begin{itemize}
\item Source IPv6 address
\item Destination IPv6 address
\item Source TCP port
\item Destination TCP port
\end{itemize}
\item Else if VIRTIO_NET_HASH_TYPE_UDPv6 is set and the
packet has a UDPv6 header, the hash is calculated over the following fields:
\begin{itemize}
\item Source IPv6 address
\item Destination IPv6 address
\item Source UDP port
\item Destination UDP port
\end{itemize}
\item Else if VIRTIO_NET_HASH_TYPE_IPv6 is set, the hash is
calculated over the following fields:
\begin{itemize}
\item Source IPv6 address
\item Destination IPv6 address
\end{itemize}
\item Else the device does not calculate the hash
\end{itemize}

\subparagraph{IPv6 packets with extension header}
\label{sec:Device Types / Network Device / Device Operation / Processing of Incoming Packets / Hash calculation for incoming packets / IPv6 packets with extension header}
The device calculates the hash on IPv6 packets with extension
headers according to 'Enabled hash types' bitmask as follows:
\begin{itemize}
\item If VIRTIO_NET_HASH_TYPE_TCP_EX is set and the packet
has a TCPv6 header, the hash is calculated over the following fields:
\begin{itemize}
\item Home address from the home address option in the IPv6 destination options header. If the extension header is not present, use the Source IPv6 address.
\item IPv6 address that is contained in the Routing-Header-Type-2 from the associated extension header. If the extension header is not present, use the Destination IPv6 address.
\item Source TCP port
\item Destination TCP port
\end{itemize}
\item Else if VIRTIO_NET_HASH_TYPE_UDP_EX is set and the
packet has a UDPv6 header, the hash is calculated over the following fields:
\begin{itemize}
\item Home address from the home address option in the IPv6 destination options header. If the extension header is not present, use the Source IPv6 address.
\item IPv6 address that is contained in the Routing-Header-Type-2 from the associated extension header. If the extension header is not present, use the Destination IPv6 address.
\item Source UDP port
\item Destination UDP port
\end{itemize}
\item Else if VIRTIO_NET_HASH_TYPE_IP_EX is set, the hash is
calculated over the following fields:
\begin{itemize}
\item Home address from the home address option in the IPv6 destination options header. If the extension header is not present, use the Source IPv6 address.
\item IPv6 address that is contained in the Routing-Header-Type-2 from the associated extension header. If the extension header is not present, use the Destination IPv6 address.
\end{itemize}
\item Else skip IPv6 extension headers and calculate the hash as
defined for an IPv6 packet without extension headers
(see \ref{sec:Device Types / Network Device / Device Operation / Processing of Incoming Packets / Hash calculation for incoming packets / IPv6 packets without extension header}).
\end{itemize}

\paragraph{Inner Header Hash}
\label{sec:Device Types / Network Device / Device Operation / Processing of Incoming Packets / Inner Header Hash}

If VIRTIO_NET_F_HASH_TUNNEL has been negotiated, the driver can send the command
VIRTIO_NET_CTRL_HASH_TUNNEL_SET to configure the calculation of the inner header hash.

\begin{lstlisting}
struct virtnet_hash_tunnel {
    le32 enabled_tunnel_types;
};

#define VIRTIO_NET_CTRL_HASH_TUNNEL 7
 #define VIRTIO_NET_CTRL_HASH_TUNNEL_SET 0
\end{lstlisting}

Field \field{enabled_tunnel_types} contains the bitmask of encapsulation types enabled for inner header hash.
See \ref{sec:Device Types / Network Device / Device Operation / Processing of Incoming Packets /
Hash calculation for incoming packets / Encapsulation types supported/enabled for inner header hash}.

The class VIRTIO_NET_CTRL_HASH_TUNNEL has one command:
VIRTIO_NET_CTRL_HASH_TUNNEL_SET sets \field{enabled_tunnel_types} for the device using the
virtnet_hash_tunnel structure, which is read-only for the device.

Inner header hash is disabled by VIRTIO_NET_CTRL_HASH_TUNNEL_SET with \field{enabled_tunnel_types} set to 0.

Initially (before the driver sends any VIRTIO_NET_CTRL_HASH_TUNNEL_SET command) all
encapsulation types are disabled for inner header hash.

\subparagraph{Encapsulated packet}
\label{sec:Device Types / Network Device / Device Operation / Processing of Incoming Packets / Hash calculation for incoming packets / Encapsulated packet}

Multiple tunneling protocols allow encapsulating an inner, payload packet in an outer, encapsulated packet.
The encapsulated packet thus contains an outer header and an inner header, and the device calculates the
hash over either the inner header or the outer header.

If VIRTIO_NET_F_HASH_TUNNEL is negotiated and a received encapsulated packet's outer header matches one of the
encapsulation types enabled in \field{enabled_tunnel_types}, then the device uses the inner header for hash
calculations (only a single level of encapsulation is currently supported).

If VIRTIO_NET_F_HASH_TUNNEL is negotiated and a received packet's (outer) header does not match any encapsulation
types enabled in \field{enabled_tunnel_types}, then the device uses the outer header for hash calculations.

\subparagraph{Encapsulation types supported/enabled for inner header hash}
\label{sec:Device Types / Network Device / Device Operation / Processing of Incoming Packets /
Hash calculation for incoming packets / Encapsulation types supported/enabled for inner header hash}

Encapsulation types applicable for inner header hash:
\begin{lstlisting}[escapechar=|]
#define VIRTIO_NET_HASH_TUNNEL_TYPE_GRE_2784    (1 << 0) /* |\hyperref[intro:rfc2784]{[RFC2784]}| */
#define VIRTIO_NET_HASH_TUNNEL_TYPE_GRE_2890    (1 << 1) /* |\hyperref[intro:rfc2890]{[RFC2890]}| */
#define VIRTIO_NET_HASH_TUNNEL_TYPE_GRE_7676    (1 << 2) /* |\hyperref[intro:rfc7676]{[RFC7676]}| */
#define VIRTIO_NET_HASH_TUNNEL_TYPE_GRE_UDP     (1 << 3) /* |\hyperref[intro:rfc8086]{[GRE-in-UDP]}| */
#define VIRTIO_NET_HASH_TUNNEL_TYPE_VXLAN       (1 << 4) /* |\hyperref[intro:vxlan]{[VXLAN]}| */
#define VIRTIO_NET_HASH_TUNNEL_TYPE_VXLAN_GPE   (1 << 5) /* |\hyperref[intro:vxlan-gpe]{[VXLAN-GPE]}| */
#define VIRTIO_NET_HASH_TUNNEL_TYPE_GENEVE      (1 << 6) /* |\hyperref[intro:geneve]{[GENEVE]}| */
#define VIRTIO_NET_HASH_TUNNEL_TYPE_IPIP        (1 << 7) /* |\hyperref[intro:ipip]{[IPIP]}| */
#define VIRTIO_NET_HASH_TUNNEL_TYPE_NVGRE       (1 << 8) /* |\hyperref[intro:nvgre]{[NVGRE]}| */
\end{lstlisting}

\subparagraph{Advice}
Example uses of the inner header hash:
\begin{itemize}
\item Legacy tunneling protocols, lacking the outer header entropy, can use RSS with the inner header hash to
      distribute flows with identical outer but different inner headers across various queues, improving performance.
\item Identify an inner flow distributed across multiple outer tunnels.
\end{itemize}

As using the inner header hash completely discards the outer header entropy, care must be taken
if the inner header is controlled by an adversary, as the adversary can then intentionally create
configurations with insufficient entropy.

Besides disabling the inner header hash, mitigations would depend on how the hash is used. When the hash
use is limited to the RSS queue selection, the inner header hash may have quality of service (QoS) limitations.

\devicenormative{\subparagraph}{Inner Header Hash}{Device Types / Network Device / Device Operation / Control Virtqueue / Inner Header Hash}

If the (outer) header of the received packet does not match any encapsulation types enabled
in \field{enabled_tunnel_types}, the device MUST calculate the hash on the outer header.

If the device receives any bits in \field{enabled_tunnel_types} which are not set in \field{supported_tunnel_types},
it SHOULD respond to the VIRTIO_NET_CTRL_HASH_TUNNEL_SET command with VIRTIO_NET_ERR.

If the driver sets \field{enabled_tunnel_types} to 0 through VIRTIO_NET_CTRL_HASH_TUNNEL_SET or upon the device reset,
the device MUST disable the inner header hash for all encapsulation types.

\drivernormative{\subparagraph}{Inner Header Hash}{Device Types / Network Device / Device Operation / Control Virtqueue / Inner Header Hash}

The driver MUST have negotiated the VIRTIO_NET_F_HASH_TUNNEL feature when issuing the VIRTIO_NET_CTRL_HASH_TUNNEL_SET command.

The driver MUST NOT set any bits in \field{enabled_tunnel_types} which are not set in \field{supported_tunnel_types}.

The driver MUST ignore bits in \field{supported_tunnel_types} which are not documented in this specification.

\paragraph{Hash reporting for incoming packets}
\label{sec:Device Types / Network Device / Device Operation / Processing of Incoming Packets / Hash reporting for incoming packets}

If VIRTIO_NET_F_HASH_REPORT was negotiated and
 the device has calculated the hash for the packet, the device fills \field{hash_report} with the report type of calculated hash
and \field{hash_value} with the value of calculated hash.

If VIRTIO_NET_F_HASH_REPORT was negotiated but due to any reason the
hash was not calculated, the device sets \field{hash_report} to VIRTIO_NET_HASH_REPORT_NONE.

Possible values that the device can report in \field{hash_report} are defined below.
They correspond to supported hash types defined in
\ref{sec:Device Types / Network Device / Device Operation / Processing of Incoming Packets / Hash calculation for incoming packets / Supported/enabled hash types}
as follows:

VIRTIO_NET_HASH_TYPE_XXX = 1 << (VIRTIO_NET_HASH_REPORT_XXX - 1)

\begin{lstlisting}
#define VIRTIO_NET_HASH_REPORT_NONE            0
#define VIRTIO_NET_HASH_REPORT_IPv4            1
#define VIRTIO_NET_HASH_REPORT_TCPv4           2
#define VIRTIO_NET_HASH_REPORT_UDPv4           3
#define VIRTIO_NET_HASH_REPORT_IPv6            4
#define VIRTIO_NET_HASH_REPORT_TCPv6           5
#define VIRTIO_NET_HASH_REPORT_UDPv6           6
#define VIRTIO_NET_HASH_REPORT_IPv6_EX         7
#define VIRTIO_NET_HASH_REPORT_TCPv6_EX        8
#define VIRTIO_NET_HASH_REPORT_UDPv6_EX        9
\end{lstlisting}

\subsubsection{Control Virtqueue}\label{sec:Device Types / Network Device / Device Operation / Control Virtqueue}

The driver uses the control virtqueue (if VIRTIO_NET_F_CTRL_VQ is
negotiated) to send commands to manipulate various features of
the device which would not easily map into the configuration
space.

All commands are of the following form:

\begin{lstlisting}
struct virtio_net_ctrl {
        u8 class;
        u8 command;
        u8 command-specific-data[];
        u8 ack;
        u8 command-specific-result[];
};

/* ack values */
#define VIRTIO_NET_OK     0
#define VIRTIO_NET_ERR    1
\end{lstlisting}

The \field{class}, \field{command} and command-specific-data are set by the
driver, and the device sets the \field{ack} byte and optionally
\field{command-specific-result}. There is little the driver can
do except issue a diagnostic if \field{ack} is not VIRTIO_NET_OK.

The command VIRTIO_NET_CTRL_STATS_QUERY and VIRTIO_NET_CTRL_STATS_GET contain
\field{command-specific-result}.

\paragraph{Packet Receive Filtering}\label{sec:Device Types / Network Device / Device Operation / Control Virtqueue / Packet Receive Filtering}
\label{sec:Device Types / Network Device / Device Operation / Control Virtqueue / Setting Promiscuous Mode}%old label for latexdiff

If the VIRTIO_NET_F_CTRL_RX and VIRTIO_NET_F_CTRL_RX_EXTRA
features are negotiated, the driver can send control commands for
promiscuous mode, multicast, unicast and broadcast receiving.

\begin{note}
In general, these commands are best-effort: unwanted
packets could still arrive.
\end{note}

\begin{lstlisting}
#define VIRTIO_NET_CTRL_RX    0
 #define VIRTIO_NET_CTRL_RX_PROMISC      0
 #define VIRTIO_NET_CTRL_RX_ALLMULTI     1
 #define VIRTIO_NET_CTRL_RX_ALLUNI       2
 #define VIRTIO_NET_CTRL_RX_NOMULTI      3
 #define VIRTIO_NET_CTRL_RX_NOUNI        4
 #define VIRTIO_NET_CTRL_RX_NOBCAST      5
\end{lstlisting}


\devicenormative{\subparagraph}{Packet Receive Filtering}{Device Types / Network Device / Device Operation / Control Virtqueue / Packet Receive Filtering}

If the VIRTIO_NET_F_CTRL_RX feature has been negotiated,
the device MUST support the following VIRTIO_NET_CTRL_RX class
commands:
\begin{itemize}
\item VIRTIO_NET_CTRL_RX_PROMISC turns promiscuous mode on and
off. The command-specific-data is one byte containing 0 (off) or
1 (on). If promiscuous mode is on, the device SHOULD receive all
incoming packets.
This SHOULD take effect even if one of the other modes set by
a VIRTIO_NET_CTRL_RX class command is on.
\item VIRTIO_NET_CTRL_RX_ALLMULTI turns all-multicast receive on and
off. The command-specific-data is one byte containing 0 (off) or
1 (on). When all-multicast receive is on the device SHOULD allow
all incoming multicast packets.
\end{itemize}

If the VIRTIO_NET_F_CTRL_RX_EXTRA feature has been negotiated,
the device MUST support the following VIRTIO_NET_CTRL_RX class
commands:
\begin{itemize}
\item VIRTIO_NET_CTRL_RX_ALLUNI turns all-unicast receive on and
off. The command-specific-data is one byte containing 0 (off) or
1 (on). When all-unicast receive is on the device SHOULD allow
all incoming unicast packets.
\item VIRTIO_NET_CTRL_RX_NOMULTI suppresses multicast receive.
The command-specific-data is one byte containing 0 (multicast
receive allowed) or 1 (multicast receive suppressed).
When multicast receive is suppressed, the device SHOULD NOT
send multicast packets to the driver.
This SHOULD take effect even if VIRTIO_NET_CTRL_RX_ALLMULTI is on.
This filter SHOULD NOT apply to broadcast packets.
\item VIRTIO_NET_CTRL_RX_NOUNI suppresses unicast receive.
The command-specific-data is one byte containing 0 (unicast
receive allowed) or 1 (unicast receive suppressed).
When unicast receive is suppressed, the device SHOULD NOT
send unicast packets to the driver.
This SHOULD take effect even if VIRTIO_NET_CTRL_RX_ALLUNI is on.
\item VIRTIO_NET_CTRL_RX_NOBCAST suppresses broadcast receive.
The command-specific-data is one byte containing 0 (broadcast
receive allowed) or 1 (broadcast receive suppressed).
When broadcast receive is suppressed, the device SHOULD NOT
send broadcast packets to the driver.
This SHOULD take effect even if VIRTIO_NET_CTRL_RX_ALLMULTI is on.
\end{itemize}

\drivernormative{\subparagraph}{Packet Receive Filtering}{Device Types / Network Device / Device Operation / Control Virtqueue / Packet Receive Filtering}

If the VIRTIO_NET_F_CTRL_RX feature has not been negotiated,
the driver MUST NOT issue commands VIRTIO_NET_CTRL_RX_PROMISC or
VIRTIO_NET_CTRL_RX_ALLMULTI.

If the VIRTIO_NET_F_CTRL_RX_EXTRA feature has not been negotiated,
the driver MUST NOT issue commands
 VIRTIO_NET_CTRL_RX_ALLUNI,
 VIRTIO_NET_CTRL_RX_NOMULTI,
 VIRTIO_NET_CTRL_RX_NOUNI or
 VIRTIO_NET_CTRL_RX_NOBCAST.

\paragraph{Setting MAC Address Filtering}\label{sec:Device Types / Network Device / Device Operation / Control Virtqueue / Setting MAC Address Filtering}

If the VIRTIO_NET_F_CTRL_RX feature is negotiated, the driver can
send control commands for MAC address filtering.

\begin{lstlisting}
struct virtio_net_ctrl_mac {
        le32 entries;
        u8 macs[entries][6];
};

#define VIRTIO_NET_CTRL_MAC    1
 #define VIRTIO_NET_CTRL_MAC_TABLE_SET        0
 #define VIRTIO_NET_CTRL_MAC_ADDR_SET         1
\end{lstlisting}

The device can filter incoming packets by any number of destination
MAC addresses\footnote{Since there are no guarantees, it can use a hash filter or
silently switch to allmulti or promiscuous mode if it is given too
many addresses.
}. This table is set using the class
VIRTIO_NET_CTRL_MAC and the command VIRTIO_NET_CTRL_MAC_TABLE_SET. The
command-specific-data is two variable length tables of 6-byte MAC
addresses (as described in struct virtio_net_ctrl_mac). The first table contains unicast addresses, and the second
contains multicast addresses.

The VIRTIO_NET_CTRL_MAC_ADDR_SET command is used to set the
default MAC address which rx filtering
accepts (and if VIRTIO_NET_F_MAC has been negotiated,
this will be reflected in \field{mac} in config space).

The command-specific-data for VIRTIO_NET_CTRL_MAC_ADDR_SET is
the 6-byte MAC address.

\devicenormative{\subparagraph}{Setting MAC Address Filtering}{Device Types / Network Device / Device Operation / Control Virtqueue / Setting MAC Address Filtering}

The device MUST have an empty MAC filtering table on reset.

The device MUST update the MAC filtering table before it consumes
the VIRTIO_NET_CTRL_MAC_TABLE_SET command.

The device MUST update \field{mac} in config space before it consumes
the VIRTIO_NET_CTRL_MAC_ADDR_SET command, if VIRTIO_NET_F_MAC has
been negotiated.

The device SHOULD drop incoming packets which have a destination MAC which
matches neither the \field{mac} (or that set with VIRTIO_NET_CTRL_MAC_ADDR_SET)
nor the MAC filtering table.

\drivernormative{\subparagraph}{Setting MAC Address Filtering}{Device Types / Network Device / Device Operation / Control Virtqueue / Setting MAC Address Filtering}

If VIRTIO_NET_F_CTRL_RX has not been negotiated,
the driver MUST NOT issue VIRTIO_NET_CTRL_MAC class commands.

If VIRTIO_NET_F_CTRL_RX has been negotiated,
the driver SHOULD issue VIRTIO_NET_CTRL_MAC_ADDR_SET
to set the default mac if it is different from \field{mac}.

The driver MUST follow the VIRTIO_NET_CTRL_MAC_TABLE_SET command
by a le32 number, followed by that number of non-multicast
MAC addresses, followed by another le32 number, followed by
that number of multicast addresses.  Either number MAY be 0.

\subparagraph{Legacy Interface: Setting MAC Address Filtering}\label{sec:Device Types / Network Device / Device Operation / Control Virtqueue / Setting MAC Address Filtering / Legacy Interface: Setting MAC Address Filtering}
When using the legacy interface, transitional devices and drivers
MUST format \field{entries} in struct virtio_net_ctrl_mac
according to the native endian of the guest rather than
(necessarily when not using the legacy interface) little-endian.

Legacy drivers that didn't negotiate VIRTIO_NET_F_CTRL_MAC_ADDR
changed \field{mac} in config space when NIC is accepting
incoming packets. These drivers always wrote the mac value from
first to last byte, therefore after detecting such drivers,
a transitional device MAY defer MAC update, or MAY defer
processing incoming packets until driver writes the last byte
of \field{mac} in the config space.

\paragraph{VLAN Filtering}\label{sec:Device Types / Network Device / Device Operation / Control Virtqueue / VLAN Filtering}

If the driver negotiates the VIRTIO_NET_F_CTRL_VLAN feature, it
can control a VLAN filter table in the device. The VLAN filter
table applies only to VLAN tagged packets.

When VIRTIO_NET_F_CTRL_VLAN is negotiated, the device starts with
an empty VLAN filter table.

\begin{note}
Similar to the MAC address based filtering, the VLAN filtering
is also best-effort: unwanted packets could still arrive.
\end{note}

\begin{lstlisting}
#define VIRTIO_NET_CTRL_VLAN       2
 #define VIRTIO_NET_CTRL_VLAN_ADD             0
 #define VIRTIO_NET_CTRL_VLAN_DEL             1
\end{lstlisting}

Both the VIRTIO_NET_CTRL_VLAN_ADD and VIRTIO_NET_CTRL_VLAN_DEL
command take a little-endian 16-bit VLAN id as the command-specific-data.

VIRTIO_NET_CTRL_VLAN_ADD command adds the specified VLAN to the
VLAN filter table.

VIRTIO_NET_CTRL_VLAN_DEL command removes the specified VLAN from
the VLAN filter table.

\devicenormative{\subparagraph}{VLAN Filtering}{Device Types / Network Device / Device Operation / Control Virtqueue / VLAN Filtering}

When VIRTIO_NET_F_CTRL_VLAN is not negotiated, the device MUST
accept all VLAN tagged packets.

When VIRTIO_NET_F_CTRL_VLAN is negotiated, the device MUST
accept all VLAN tagged packets whose VLAN tag is present in
the VLAN filter table and SHOULD drop all VLAN tagged packets
whose VLAN tag is absent in the VLAN filter table.

\subparagraph{Legacy Interface: VLAN Filtering}\label{sec:Device Types / Network Device / Device Operation / Control Virtqueue / VLAN Filtering / Legacy Interface: VLAN Filtering}
When using the legacy interface, transitional devices and drivers
MUST format the VLAN id
according to the native endian of the guest rather than
(necessarily when not using the legacy interface) little-endian.

\paragraph{Gratuitous Packet Sending}\label{sec:Device Types / Network Device / Device Operation / Control Virtqueue / Gratuitous Packet Sending}

If the driver negotiates the VIRTIO_NET_F_GUEST_ANNOUNCE (depends
on VIRTIO_NET_F_CTRL_VQ), the device can ask the driver to send gratuitous
packets; this is usually done after the guest has been physically
migrated, and needs to announce its presence on the new network
links. (As hypervisor does not have the knowledge of guest
network configuration (eg. tagged vlan) it is simplest to prod
the guest in this way).

\begin{lstlisting}
#define VIRTIO_NET_CTRL_ANNOUNCE       3
 #define VIRTIO_NET_CTRL_ANNOUNCE_ACK             0
\end{lstlisting}

The driver checks VIRTIO_NET_S_ANNOUNCE bit in the device configuration \field{status} field
when it notices the changes of device configuration. The
command VIRTIO_NET_CTRL_ANNOUNCE_ACK is used to indicate that
driver has received the notification and device clears the
VIRTIO_NET_S_ANNOUNCE bit in \field{status}.

Processing this notification involves:

\begin{enumerate}
\item Sending the gratuitous packets (eg. ARP) or marking there are pending
  gratuitous packets to be sent and letting deferred routine to
  send them.

\item Sending VIRTIO_NET_CTRL_ANNOUNCE_ACK command through control
  vq.
\end{enumerate}

\drivernormative{\subparagraph}{Gratuitous Packet Sending}{Device Types / Network Device / Device Operation / Control Virtqueue / Gratuitous Packet Sending}

If the driver negotiates VIRTIO_NET_F_GUEST_ANNOUNCE, it SHOULD notify
network peers of its new location after it sees the VIRTIO_NET_S_ANNOUNCE bit
in \field{status}.  The driver MUST send a command on the command queue
with class VIRTIO_NET_CTRL_ANNOUNCE and command VIRTIO_NET_CTRL_ANNOUNCE_ACK.

\devicenormative{\subparagraph}{Gratuitous Packet Sending}{Device Types / Network Device / Device Operation / Control Virtqueue / Gratuitous Packet Sending}

If VIRTIO_NET_F_GUEST_ANNOUNCE is negotiated, the device MUST clear the
VIRTIO_NET_S_ANNOUNCE bit in \field{status} upon receipt of a command buffer
with class VIRTIO_NET_CTRL_ANNOUNCE and command VIRTIO_NET_CTRL_ANNOUNCE_ACK
before marking the buffer as used.

\paragraph{Device operation in multiqueue mode}\label{sec:Device Types / Network Device / Device Operation / Control Virtqueue / Device operation in multiqueue mode}

This specification defines the following modes that a device MAY implement for operation with multiple transmit/receive virtqueues:
\begin{itemize}
\item Automatic receive steering as defined in \ref{sec:Device Types / Network Device / Device Operation / Control Virtqueue / Automatic receive steering in multiqueue mode}.
 If a device supports this mode, it offers the VIRTIO_NET_F_MQ feature bit.
\item Receive-side scaling as defined in \ref{devicenormative:Device Types / Network Device / Device Operation / Control Virtqueue / Receive-side scaling (RSS) / RSS processing}.
 If a device supports this mode, it offers the VIRTIO_NET_F_RSS feature bit.
\end{itemize}

A device MAY support one of these features or both. The driver MAY negotiate any set of these features that the device supports.

Multiqueue is disabled by default.

The driver enables multiqueue by sending a command using \field{class} VIRTIO_NET_CTRL_MQ. The \field{command} selects the mode of multiqueue operation, as follows:
\begin{lstlisting}
#define VIRTIO_NET_CTRL_MQ    4
 #define VIRTIO_NET_CTRL_MQ_VQ_PAIRS_SET        0 (for automatic receive steering)
 #define VIRTIO_NET_CTRL_MQ_RSS_CONFIG          1 (for configurable receive steering)
 #define VIRTIO_NET_CTRL_MQ_HASH_CONFIG         2 (for configurable hash calculation)
\end{lstlisting}

If more than one multiqueue mode is negotiated, the resulting device configuration is defined by the last command sent by the driver.

\paragraph{Automatic receive steering in multiqueue mode}\label{sec:Device Types / Network Device / Device Operation / Control Virtqueue / Automatic receive steering in multiqueue mode}

If the driver negotiates the VIRTIO_NET_F_MQ feature bit (depends on VIRTIO_NET_F_CTRL_VQ), it MAY transmit outgoing packets on one
of the multiple transmitq1\ldots transmitqN and ask the device to
queue incoming packets into one of the multiple receiveq1\ldots receiveqN
depending on the packet flow.

The driver enables multiqueue by
sending the VIRTIO_NET_CTRL_MQ_VQ_PAIRS_SET command, specifying
the number of the transmit and receive queues to be used up to
\field{max_virtqueue_pairs}; subsequently,
transmitq1\ldots transmitqn and receiveq1\ldots receiveqn where
n=\field{virtqueue_pairs} MAY be used.
\begin{lstlisting}
struct virtio_net_ctrl_mq_pairs_set {
       le16 virtqueue_pairs;
};
#define VIRTIO_NET_CTRL_MQ_VQ_PAIRS_MIN        1
#define VIRTIO_NET_CTRL_MQ_VQ_PAIRS_MAX        0x8000

\end{lstlisting}

When multiqueue is enabled by VIRTIO_NET_CTRL_MQ_VQ_PAIRS_SET command, the device MUST use automatic receive steering
based on packet flow. Programming of the receive steering
classificator is implicit. After the driver transmitted a packet of a
flow on transmitqX, the device SHOULD cause incoming packets for that flow to
be steered to receiveqX. For uni-directional protocols, or where
no packets have been transmitted yet, the device MAY steer a packet
to a random queue out of the specified receiveq1\ldots receiveqn.

Multiqueue is disabled by VIRTIO_NET_CTRL_MQ_VQ_PAIRS_SET with \field{virtqueue_pairs} to 1 (this is
the default) and waiting for the device to use the command buffer.

\drivernormative{\subparagraph}{Automatic receive steering in multiqueue mode}{Device Types / Network Device / Device Operation / Control Virtqueue / Automatic receive steering in multiqueue mode}

The driver MUST configure the virtqueues before enabling them with the
VIRTIO_NET_CTRL_MQ_VQ_PAIRS_SET command.

The driver MUST NOT request a \field{virtqueue_pairs} of 0 or
greater than \field{max_virtqueue_pairs} in the device configuration space.

The driver MUST queue packets only on any transmitq1 before the
VIRTIO_NET_CTRL_MQ_VQ_PAIRS_SET command.

The driver MUST NOT queue packets on transmit queues greater than
\field{virtqueue_pairs} once it has placed the VIRTIO_NET_CTRL_MQ_VQ_PAIRS_SET command in the available ring.

\devicenormative{\subparagraph}{Automatic receive steering in multiqueue mode}{Device Types / Network Device / Device Operation / Control Virtqueue / Automatic receive steering in multiqueue mode}

After initialization of reset, the device MUST queue packets only on receiveq1.

The device MUST NOT queue packets on receive queues greater than
\field{virtqueue_pairs} once it has placed the
VIRTIO_NET_CTRL_MQ_VQ_PAIRS_SET command in a used buffer.

If the destination receive queue is being reset (See \ref{sec:Basic Facilities of a Virtio Device / Virtqueues / Virtqueue Reset}),
the device SHOULD re-select another random queue. If all receive queues are
being reset, the device MUST drop the packet.

\subparagraph{Legacy Interface: Automatic receive steering in multiqueue mode}\label{sec:Device Types / Network Device / Device Operation / Control Virtqueue / Automatic receive steering in multiqueue mode / Legacy Interface: Automatic receive steering in multiqueue mode}
When using the legacy interface, transitional devices and drivers
MUST format \field{virtqueue_pairs}
according to the native endian of the guest rather than
(necessarily when not using the legacy interface) little-endian.

\subparagraph{Hash calculation}\label{sec:Device Types / Network Device / Device Operation / Control Virtqueue / Automatic receive steering in multiqueue mode / Hash calculation}
If VIRTIO_NET_F_HASH_REPORT was negotiated and the device uses automatic receive steering,
the device MUST support a command to configure hash calculation parameters.

The driver provides parameters for hash calculation as follows:

\field{class} VIRTIO_NET_CTRL_MQ, \field{command} VIRTIO_NET_CTRL_MQ_HASH_CONFIG.

The \field{command-specific-data} has following format:
\begin{lstlisting}
struct virtio_net_hash_config {
    le32 hash_types;
    le16 reserved[4];
    u8 hash_key_length;
    u8 hash_key_data[hash_key_length];
};
\end{lstlisting}
Field \field{hash_types} contains a bitmask of allowed hash types as
defined in
\ref{sec:Device Types / Network Device / Device Operation / Processing of Incoming Packets / Hash calculation for incoming packets / Supported/enabled hash types}.
Initially the device has all hash types disabled and reports only VIRTIO_NET_HASH_REPORT_NONE.

Field \field{reserved} MUST contain zeroes. It is defined to make the structure to match the layout of virtio_net_rss_config structure,
defined in \ref{sec:Device Types / Network Device / Device Operation / Control Virtqueue / Receive-side scaling (RSS)}.

Fields \field{hash_key_length} and \field{hash_key_data} define the key to be used in hash calculation.

\paragraph{Receive-side scaling (RSS)}\label{sec:Device Types / Network Device / Device Operation / Control Virtqueue / Receive-side scaling (RSS)}
A device offers the feature VIRTIO_NET_F_RSS if it supports RSS receive steering with Toeplitz hash calculation and configurable parameters.

A driver queries RSS capabilities of the device by reading device configuration as defined in \ref{sec:Device Types / Network Device / Device configuration layout}

\subparagraph{Setting RSS parameters}\label{sec:Device Types / Network Device / Device Operation / Control Virtqueue / Receive-side scaling (RSS) / Setting RSS parameters}

Driver sends a VIRTIO_NET_CTRL_MQ_RSS_CONFIG command using the following format for \field{command-specific-data}:
\begin{lstlisting}
struct rss_rq_id {
   le16 vq_index_1_16: 15; /* Bits 1 to 16 of the virtqueue index */
   le16 reserved: 1; /* Set to zero */
};

struct virtio_net_rss_config {
    le32 hash_types;
    le16 indirection_table_mask;
    struct rss_rq_id unclassified_queue;
    struct rss_rq_id indirection_table[indirection_table_length];
    le16 max_tx_vq;
    u8 hash_key_length;
    u8 hash_key_data[hash_key_length];
};
\end{lstlisting}
Field \field{hash_types} contains a bitmask of allowed hash types as
defined in
\ref{sec:Device Types / Network Device / Device Operation / Processing of Incoming Packets / Hash calculation for incoming packets / Supported/enabled hash types}.

Field \field{indirection_table_mask} is a mask to be applied to
the calculated hash to produce an index in the
\field{indirection_table} array.
Number of entries in \field{indirection_table} is (\field{indirection_table_mask} + 1).

\field{rss_rq_id} is a receive virtqueue id. \field{vq_index_1_16}
consists of bits 1 to 16 of a virtqueue index. For example, a
\field{vq_index_1_16} value of 3 corresponds to virtqueue index 6,
which maps to receiveq4.

Field \field{unclassified_queue} specifies the receive virtqueue id in which to
place unclassified packets.

Field \field{indirection_table} is an array of receive virtqueues ids.

A driver sets \field{max_tx_vq} to inform a device how many transmit virtqueues it may use (transmitq1\ldots transmitq \field{max_tx_vq}).

Fields \field{hash_key_length} and \field{hash_key_data} define the key to be used in hash calculation.

\drivernormative{\subparagraph}{Setting RSS parameters}{Device Types / Network Device / Device Operation / Control Virtqueue / Receive-side scaling (RSS) }

A driver MUST NOT send the VIRTIO_NET_CTRL_MQ_RSS_CONFIG command if the feature VIRTIO_NET_F_RSS has not been negotiated.

A driver MUST fill the \field{indirection_table} array only with
enabled receive virtqueues ids.

The number of entries in \field{indirection_table} (\field{indirection_table_mask} + 1) MUST be a power of two.

A driver MUST use \field{indirection_table_mask} values that are less than \field{rss_max_indirection_table_length} reported by a device.

A driver MUST NOT set any VIRTIO_NET_HASH_TYPE_ flags that are not supported by a device.

\devicenormative{\subparagraph}{RSS processing}{Device Types / Network Device / Device Operation / Control Virtqueue / Receive-side scaling (RSS) / RSS processing}
The device MUST determine the destination queue for a network packet as follows:
\begin{itemize}
\item Calculate the hash of the packet as defined in \ref{sec:Device Types / Network Device / Device Operation / Processing of Incoming Packets / Hash calculation for incoming packets}.
\item If the device did not calculate the hash for the specific packet, the device directs the packet to the receiveq specified by \field{unclassified_queue} of virtio_net_rss_config structure.
\item Apply \field{indirection_table_mask} to the calculated hash
and use the result as the index in the indirection table to get
the destination receive virtqueue id.
\item If the destination receive queue is being reset (See \ref{sec:Basic Facilities of a Virtio Device / Virtqueues / Virtqueue Reset}), the device MUST drop the packet.
\end{itemize}

\paragraph{RSS Context}\label{sec:Device Types / Network Device / Device Operation / Control Virtqueue / RSS Context}

An RSS context consists of configurable parameters specified by \ref{sec:Device Types / Network Device
/ Device Operation / Control Virtqueue / Receive-side scaling (RSS)}.

The RSS configuration supported by VIRTIO_NET_F_RSS is considered the default RSS configuration.

The device offers the feature VIRTIO_NET_F_RSS_CONTEXT if it supports one or multiple RSS contexts
(excluding the default RSS configuration) and configurable parameters.

\subparagraph{Querying RSS Context Capability}\label{sec:Device Types / Network Device / Device Operation / Control Virtqueue / RSS Context / Querying RSS Context Capability}

\begin{lstlisting}
#define VIRTNET_RSS_CTX_CTRL 9
 #define VIRTNET_RSS_CTX_CTRL_CAP_GET  0
 #define VIRTNET_RSS_CTX_CTRL_ADD      1
 #define VIRTNET_RSS_CTX_CTRL_MOD      2
 #define VIRTNET_RSS_CTX_CTRL_DEL      3

struct virtnet_rss_ctx_cap {
    le16 max_rss_contexts;
}
\end{lstlisting}

Field \field{max_rss_contexts} specifies the maximum number of RSS contexts \ref{sec:Device Types / Network Device /
Device Operation / Control Virtqueue / RSS Context} supported by the device.

The driver queries the RSS context capability of the device by sending the command VIRTNET_RSS_CTX_CTRL_CAP_GET
with the structure virtnet_rss_ctx_cap.

For the command VIRTNET_RSS_CTX_CTRL_CAP_GET, the structure virtnet_rss_ctx_cap is write-only for the device.

\subparagraph{Setting RSS Context Parameters}\label{sec:Device Types / Network Device / Device Operation / Control Virtqueue / RSS Context / Setting RSS Context Parameters}

\begin{lstlisting}
struct virtnet_rss_ctx_add_modify {
    le16 rss_ctx_id;
    u8 reserved[6];
    struct virtio_net_rss_config rss;
};

struct virtnet_rss_ctx_del {
    le16 rss_ctx_id;
};
\end{lstlisting}

RSS context parameters:
\begin{itemize}
\item  \field{rss_ctx_id}: ID of the specific RSS context.
\item  \field{rss}: RSS context parameters of the specific RSS context whose id is \field{rss_ctx_id}.
\end{itemize}

\field{reserved} is reserved and it is ignored by the device.

If the feature VIRTIO_NET_F_RSS_CONTEXT has been negotiated, the driver can send the following
VIRTNET_RSS_CTX_CTRL class commands:
\begin{enumerate}
\item VIRTNET_RSS_CTX_CTRL_ADD: use the structure virtnet_rss_ctx_add_modify to
       add an RSS context configured as \field{rss} and id as \field{rss_ctx_id} for the device.
\item VIRTNET_RSS_CTX_CTRL_MOD: use the structure virtnet_rss_ctx_add_modify to
       configure parameters of the RSS context whose id is \field{rss_ctx_id} as \field{rss} for the device.
\item VIRTNET_RSS_CTX_CTRL_DEL: use the structure virtnet_rss_ctx_del to delete
       the RSS context whose id is \field{rss_ctx_id} for the device.
\end{enumerate}

For commands VIRTNET_RSS_CTX_CTRL_ADD and VIRTNET_RSS_CTX_CTRL_MOD, the structure virtnet_rss_ctx_add_modify is read-only for the device.
For the command VIRTNET_RSS_CTX_CTRL_DEL, the structure virtnet_rss_ctx_del is read-only for the device.

\devicenormative{\subparagraph}{RSS Context}{Device Types / Network Device / Device Operation / Control Virtqueue / RSS Context}

The device MUST set \field{max_rss_contexts} to at least 1 if it offers VIRTIO_NET_F_RSS_CONTEXT.

Upon reset, the device MUST clear all previously configured RSS contexts.

\drivernormative{\subparagraph}{RSS Context}{Device Types / Network Device / Device Operation / Control Virtqueue / RSS Context}

The driver MUST have negotiated the VIRTIO_NET_F_RSS_CONTEXT feature when issuing the VIRTNET_RSS_CTX_CTRL class commands.

The driver MUST set \field{rss_ctx_id} to between 1 and \field{max_rss_contexts} inclusive.

The driver MUST NOT send the command VIRTIO_NET_CTRL_MQ_VQ_PAIRS_SET when the device has successfully configured at least one RSS context.

\paragraph{Offloads State Configuration}\label{sec:Device Types / Network Device / Device Operation / Control Virtqueue / Offloads State Configuration}

If the VIRTIO_NET_F_CTRL_GUEST_OFFLOADS feature is negotiated, the driver can
send control commands for dynamic offloads state configuration.

\subparagraph{Setting Offloads State}\label{sec:Device Types / Network Device / Device Operation / Control Virtqueue / Offloads State Configuration / Setting Offloads State}

To configure the offloads, the following layout structure and
definitions are used:

\begin{lstlisting}
le64 offloads;

#define VIRTIO_NET_F_GUEST_CSUM       1
#define VIRTIO_NET_F_GUEST_TSO4       7
#define VIRTIO_NET_F_GUEST_TSO6       8
#define VIRTIO_NET_F_GUEST_ECN        9
#define VIRTIO_NET_F_GUEST_UFO        10
#define VIRTIO_NET_F_GUEST_UDP_TUNNEL_GSO  46
#define VIRTIO_NET_F_GUEST_UDP_TUNNEL_GSO_CSUM 47
#define VIRTIO_NET_F_GUEST_USO4       54
#define VIRTIO_NET_F_GUEST_USO6       55

#define VIRTIO_NET_CTRL_GUEST_OFFLOADS       5
 #define VIRTIO_NET_CTRL_GUEST_OFFLOADS_SET   0
\end{lstlisting}

The class VIRTIO_NET_CTRL_GUEST_OFFLOADS has one command:
VIRTIO_NET_CTRL_GUEST_OFFLOADS_SET applies the new offloads configuration.

le64 value passed as command data is a bitmask, bits set define
offloads to be enabled, bits cleared - offloads to be disabled.

There is a corresponding device feature for each offload. Upon feature
negotiation corresponding offload gets enabled to preserve backward
compatibility.

\drivernormative{\subparagraph}{Setting Offloads State}{Device Types / Network Device / Device Operation / Control Virtqueue / Offloads State Configuration / Setting Offloads State}

A driver MUST NOT enable an offload for which the appropriate feature
has not been negotiated.

\subparagraph{Legacy Interface: Setting Offloads State}\label{sec:Device Types / Network Device / Device Operation / Control Virtqueue / Offloads State Configuration / Setting Offloads State / Legacy Interface: Setting Offloads State}
When using the legacy interface, transitional devices and drivers
MUST format \field{offloads}
according to the native endian of the guest rather than
(necessarily when not using the legacy interface) little-endian.


\paragraph{Notifications Coalescing}\label{sec:Device Types / Network Device / Device Operation / Control Virtqueue / Notifications Coalescing}

If the VIRTIO_NET_F_NOTF_COAL feature is negotiated, the driver can
send commands VIRTIO_NET_CTRL_NOTF_COAL_TX_SET and VIRTIO_NET_CTRL_NOTF_COAL_RX_SET
for notification coalescing.

If the VIRTIO_NET_F_VQ_NOTF_COAL feature is negotiated, the driver can
send commands VIRTIO_NET_CTRL_NOTF_COAL_VQ_SET and VIRTIO_NET_CTRL_NOTF_COAL_VQ_GET
for virtqueue notification coalescing.

\begin{lstlisting}
struct virtio_net_ctrl_coal {
    le32 max_packets;
    le32 max_usecs;
};

struct virtio_net_ctrl_coal_vq {
    le16 vq_index;
    le16 reserved;
    struct virtio_net_ctrl_coal coal;
};

#define VIRTIO_NET_CTRL_NOTF_COAL 6
 #define VIRTIO_NET_CTRL_NOTF_COAL_TX_SET  0
 #define VIRTIO_NET_CTRL_NOTF_COAL_RX_SET 1
 #define VIRTIO_NET_CTRL_NOTF_COAL_VQ_SET 2
 #define VIRTIO_NET_CTRL_NOTF_COAL_VQ_GET 3
\end{lstlisting}

Coalescing parameters:
\begin{itemize}
\item \field{vq_index}: The virtqueue index of an enabled transmit or receive virtqueue.
\item \field{max_usecs} for RX: Maximum number of microseconds to delay a RX notification.
\item \field{max_usecs} for TX: Maximum number of microseconds to delay a TX notification.
\item \field{max_packets} for RX: Maximum number of packets to receive before a RX notification.
\item \field{max_packets} for TX: Maximum number of packets to send before a TX notification.
\end{itemize}

\field{reserved} is reserved and it is ignored by the device.

Read/Write attributes for coalescing parameters:
\begin{itemize}
\item For commands VIRTIO_NET_CTRL_NOTF_COAL_TX_SET and VIRTIO_NET_CTRL_NOTF_COAL_RX_SET, the structure virtio_net_ctrl_coal is write-only for the driver.
\item For the command VIRTIO_NET_CTRL_NOTF_COAL_VQ_SET, the structure virtio_net_ctrl_coal_vq is write-only for the driver.
\item For the command VIRTIO_NET_CTRL_NOTF_COAL_VQ_GET, \field{vq_index} and \field{reserved} are write-only
      for the driver, and the structure virtio_net_ctrl_coal is read-only for the driver.
\end{itemize}

The class VIRTIO_NET_CTRL_NOTF_COAL has the following commands:
\begin{enumerate}
\item VIRTIO_NET_CTRL_NOTF_COAL_TX_SET: use the structure virtio_net_ctrl_coal to set the \field{max_usecs} and \field{max_packets} parameters for all transmit virtqueues.
\item VIRTIO_NET_CTRL_NOTF_COAL_RX_SET: use the structure virtio_net_ctrl_coal to set the \field{max_usecs} and \field{max_packets} parameters for all receive virtqueues.
\item VIRTIO_NET_CTRL_NOTF_COAL_VQ_SET: use the structure virtio_net_ctrl_coal_vq to set the \field{max_usecs} and \field{max_packets} parameters
                                        for an enabled transmit/receive virtqueue whose index is \field{vq_index}.
\item VIRTIO_NET_CTRL_NOTF_COAL_VQ_GET: use the structure virtio_net_ctrl_coal_vq to get the \field{max_usecs} and \field{max_packets} parameters
                                        for an enabled transmit/receive virtqueue whose index is \field{vq_index}.
\end{enumerate}

The device may generate notifications more or less frequently than specified by set commands of the VIRTIO_NET_CTRL_NOTF_COAL class.

If coalescing parameters are being set, the device applies the last coalescing parameters set for a
virtqueue, regardless of the command used to set the parameters. Use the following command sequence
with two pairs of virtqueues as an example:
Each of the following commands sets \field{max_usecs} and \field{max_packets} parameters for virtqueues.
\begin{itemize}
\item Command1: VIRTIO_NET_CTRL_NOTF_COAL_RX_SET sets coalescing parameters for virtqueues having index 0 and index 2. Virtqueues having index 1 and index 3 retain their previous parameters.
\item Command2: VIRTIO_NET_CTRL_NOTF_COAL_VQ_SET with \field{vq_index} = 0 sets coalescing parameters for virtqueue having index 0. Virtqueue having index 2 retains the parameters from command1.
\item Command3: VIRTIO_NET_CTRL_NOTF_COAL_VQ_GET with \field{vq_index} = 0, the device responds with coalescing parameters of vq_index 0 set by command2.
\item Command4: VIRTIO_NET_CTRL_NOTF_COAL_VQ_SET with \field{vq_index} = 1 sets coalescing parameters for virtqueue having index 1. Virtqueue having index 3 retains its previous parameters.
\item Command5: VIRTIO_NET_CTRL_NOTF_COAL_TX_SET sets coalescing parameters for virtqueues having index 1 and index 3, and overrides the parameters set by command4.
\item Command6: VIRTIO_NET_CTRL_NOTF_COAL_VQ_GET with \field{vq_index} = 1, the device responds with coalescing parameters of index 1 set by command5.
\end{itemize}

\subparagraph{Operation}\label{sec:Device Types / Network Device / Device Operation / Control Virtqueue / Notifications Coalescing / Operation}

The device sends a used buffer notification once the notification conditions are met and if the notifications are not suppressed as explained in \ref{sec:Basic Facilities of a Virtio Device / Virtqueues / Used Buffer Notification Suppression}.

When the device has non-zero \field{max_usecs} and non-zero \field{max_packets}, it starts counting microseconds and packets upon receiving/sending a packet.
The device counts packets and microseconds for each receive virtqueue and transmit virtqueue separately.
In this case, the notification conditions are met when \field{max_usecs} microseconds elapse, or upon sending/receiving \field{max_packets} packets, whichever happens first.
Afterwards, the device waits for the next packet and starts counting packets and microseconds again.

When the device has \field{max_usecs} = 0 or \field{max_packets} = 0, the notification conditions are met after every packet received/sent.

\subparagraph{RX Example}\label{sec:Device Types / Network Device / Device Operation / Control Virtqueue / Notifications Coalescing / RX Example}

If, for example:
\begin{itemize}
\item \field{max_usecs} = 10.
\item \field{max_packets} = 15.
\end{itemize}
then each receive virtqueue of a device will operate as follows:
\begin{itemize}
\item The device will count packets received on each virtqueue until it accumulates 15, or until 10 microseconds elapsed since the first one was received.
\item If the notifications are not suppressed by the driver, the device will send an used buffer notification, otherwise, the device will not send an used buffer notification as long as the notifications are suppressed.
\end{itemize}

\subparagraph{TX Example}\label{sec:Device Types / Network Device / Device Operation / Control Virtqueue / Notifications Coalescing / TX Example}

If, for example:
\begin{itemize}
\item \field{max_usecs} = 10.
\item \field{max_packets} = 15.
\end{itemize}
then each transmit virtqueue of a device will operate as follows:
\begin{itemize}
\item The device will count packets sent on each virtqueue until it accumulates 15, or until 10 microseconds elapsed since the first one was sent.
\item If the notifications are not suppressed by the driver, the device will send an used buffer notification, otherwise, the device will not send an used buffer notification as long as the notifications are suppressed.
\end{itemize}

\subparagraph{Notifications When Coalescing Parameters Change}\label{sec:Device Types / Network Device / Device Operation / Control Virtqueue / Notifications Coalescing / Notifications When Coalescing Parameters Change}

When the coalescing parameters of a device change, the device needs to check if the new notification conditions are met and send a used buffer notification if so.

For example, \field{max_packets} = 15 for a device with a single transmit virtqueue: if the device sends 10 packets and afterwards receives a
VIRTIO_NET_CTRL_NOTF_COAL_TX_SET command with \field{max_packets} = 8, then the notification condition is immediately considered to be met;
the device needs to immediately send a used buffer notification, if the notifications are not suppressed by the driver.

\drivernormative{\subparagraph}{Notifications Coalescing}{Device Types / Network Device / Device Operation / Control Virtqueue / Notifications Coalescing}

The driver MUST set \field{vq_index} to the virtqueue index of an enabled transmit or receive virtqueue.

The driver MUST have negotiated the VIRTIO_NET_F_NOTF_COAL feature when issuing commands VIRTIO_NET_CTRL_NOTF_COAL_TX_SET and VIRTIO_NET_CTRL_NOTF_COAL_RX_SET.

The driver MUST have negotiated the VIRTIO_NET_F_VQ_NOTF_COAL feature when issuing commands VIRTIO_NET_CTRL_NOTF_COAL_VQ_SET and VIRTIO_NET_CTRL_NOTF_COAL_VQ_GET.

The driver MUST ignore the values of coalescing parameters received from the VIRTIO_NET_CTRL_NOTF_COAL_VQ_GET command if the device responds with VIRTIO_NET_ERR.

\devicenormative{\subparagraph}{Notifications Coalescing}{Device Types / Network Device / Device Operation / Control Virtqueue / Notifications Coalescing}

The device MUST ignore \field{reserved}.

The device SHOULD respond to VIRTIO_NET_CTRL_NOTF_COAL_TX_SET and VIRTIO_NET_CTRL_NOTF_COAL_RX_SET commands with VIRTIO_NET_ERR if it was not able to change the parameters.

The device MUST respond to the VIRTIO_NET_CTRL_NOTF_COAL_VQ_SET command with VIRTIO_NET_ERR if it was not able to change the parameters.

The device MUST respond to VIRTIO_NET_CTRL_NOTF_COAL_VQ_SET and VIRTIO_NET_CTRL_NOTF_COAL_VQ_GET commands with
VIRTIO_NET_ERR if the designated virtqueue is not an enabled transmit or receive virtqueue.

Upon disabling and re-enabling a transmit virtqueue, the device MUST set the coalescing parameters of the virtqueue
to those configured through the VIRTIO_NET_CTRL_NOTF_COAL_TX_SET command, or, if the driver did not set any TX coalescing parameters, to 0.

Upon disabling and re-enabling a receive virtqueue, the device MUST set the coalescing parameters of the virtqueue
to those configured through the VIRTIO_NET_CTRL_NOTF_COAL_RX_SET command, or, if the driver did not set any RX coalescing parameters, to 0.

The behavior of the device in response to set commands of the VIRTIO_NET_CTRL_NOTF_COAL class is best-effort:
the device MAY generate notifications more or less frequently than specified.

A device SHOULD NOT send used buffer notifications to the driver if the notifications are suppressed, even if the notification conditions are met.

Upon reset, a device MUST initialize all coalescing parameters to 0.

\paragraph{Device Statistics}\label{sec:Device Types / Network Device / Device Operation / Control Virtqueue / Device Statistics}

If the VIRTIO_NET_F_DEVICE_STATS feature is negotiated, the driver can obtain
device statistics from the device by using the following command.

Different types of virtqueues have different statistics. The statistics of the
receiveq are different from those of the transmitq.

The statistics of a certain type of virtqueue are also divided into multiple types
because different types require different features. This enables the expansion
of new statistics.

In one command, the driver can obtain the statistics of one or multiple virtqueues.
Additionally, the driver can obtain multiple type statistics of each virtqueue.

\subparagraph{Query Statistic Capabilities}\label{sec:Device Types / Network Device / Device Operation / Control Virtqueue / Device Statistics / Query Statistic Capabilities}

\begin{lstlisting}
#define VIRTIO_NET_CTRL_STATS         8
#define VIRTIO_NET_CTRL_STATS_QUERY   0
#define VIRTIO_NET_CTRL_STATS_GET     1

struct virtio_net_stats_capabilities {

#define VIRTIO_NET_STATS_TYPE_CVQ       (1 << 32)

#define VIRTIO_NET_STATS_TYPE_RX_BASIC  (1 << 0)
#define VIRTIO_NET_STATS_TYPE_RX_CSUM   (1 << 1)
#define VIRTIO_NET_STATS_TYPE_RX_GSO    (1 << 2)
#define VIRTIO_NET_STATS_TYPE_RX_SPEED  (1 << 3)

#define VIRTIO_NET_STATS_TYPE_TX_BASIC  (1 << 16)
#define VIRTIO_NET_STATS_TYPE_TX_CSUM   (1 << 17)
#define VIRTIO_NET_STATS_TYPE_TX_GSO    (1 << 18)
#define VIRTIO_NET_STATS_TYPE_TX_SPEED  (1 << 19)

    le64 supported_stats_types[1];
}
\end{lstlisting}

To obtain device statistic capability, use the VIRTIO_NET_CTRL_STATS_QUERY
command. When the command completes successfully, \field{command-specific-result}
is in the format of \field{struct virtio_net_stats_capabilities}.

\subparagraph{Get Statistics}\label{sec:Device Types / Network Device / Device Operation / Control Virtqueue / Device Statistics / Get Statistics}

\begin{lstlisting}
struct virtio_net_ctrl_queue_stats {
       struct {
           le16 vq_index;
           le16 reserved[3];
           le64 types_bitmap[1];
       } stats[];
};

struct virtio_net_stats_reply_hdr {
#define VIRTIO_NET_STATS_TYPE_REPLY_CVQ       32

#define VIRTIO_NET_STATS_TYPE_REPLY_RX_BASIC  0
#define VIRTIO_NET_STATS_TYPE_REPLY_RX_CSUM   1
#define VIRTIO_NET_STATS_TYPE_REPLY_RX_GSO    2
#define VIRTIO_NET_STATS_TYPE_REPLY_RX_SPEED  3

#define VIRTIO_NET_STATS_TYPE_REPLY_TX_BASIC  16
#define VIRTIO_NET_STATS_TYPE_REPLY_TX_CSUM   17
#define VIRTIO_NET_STATS_TYPE_REPLY_TX_GSO    18
#define VIRTIO_NET_STATS_TYPE_REPLY_TX_SPEED  19
    u8 type;
    u8 reserved;
    le16 vq_index;
    le16 reserved1;
    le16 size;
}
\end{lstlisting}

To obtain device statistics, use the VIRTIO_NET_CTRL_STATS_GET command with the
\field{command-specific-data} which is in the format of
\field{struct virtio_net_ctrl_queue_stats}. When the command completes
successfully, \field{command-specific-result} contains multiple statistic
results, each statistic result has the \field{struct virtio_net_stats_reply_hdr}
as the header.

The fields of the \field{struct virtio_net_ctrl_queue_stats}:
\begin{description}
    \item [vq_index]
        The index of the virtqueue to obtain the statistics.

    \item [types_bitmap]
        This is a bitmask of the types of statistics to be obtained. Therefore, a
        \field{stats} inside \field{struct virtio_net_ctrl_queue_stats} may
        indicate multiple statistic replies for the virtqueue.
\end{description}

The fields of the \field{struct virtio_net_stats_reply_hdr}:
\begin{description}
    \item [type]
        The type of the reply statistic.

    \item [vq_index]
        The virtqueue index of the reply statistic.

    \item [size]
        The number of bytes for the statistics entry including size of \field{struct virtio_net_stats_reply_hdr}.

\end{description}

\subparagraph{Controlq Statistics}\label{sec:Device Types / Network Device / Device Operation / Control Virtqueue / Device Statistics / Controlq Statistics}

The structure corresponding to the controlq statistics is
\field{struct virtio_net_stats_cvq}. The corresponding type is
VIRTIO_NET_STATS_TYPE_CVQ. This is for the controlq.

\begin{lstlisting}
struct virtio_net_stats_cvq {
    struct virtio_net_stats_reply_hdr hdr;

    le64 command_num;
    le64 ok_num;
};
\end{lstlisting}

\begin{description}
    \item [command_num]
        The number of commands received by the device including the current command.

    \item [ok_num]
        The number of commands completed successfully by the device including the current command.
\end{description}


\subparagraph{Receiveq Basic Statistics}\label{sec:Device Types / Network Device / Device Operation / Control Virtqueue / Device Statistics / Receiveq Basic Statistics}

The structure corresponding to the receiveq basic statistics is
\field{struct virtio_net_stats_rx_basic}. The corresponding type is
VIRTIO_NET_STATS_TYPE_RX_BASIC. This is for the receiveq.

Receiveq basic statistics do not require any feature. As long as the device supports
VIRTIO_NET_F_DEVICE_STATS, the following are the receiveq basic statistics.

\begin{lstlisting}
struct virtio_net_stats_rx_basic {
    struct virtio_net_stats_reply_hdr hdr;

    le64 rx_notifications;

    le64 rx_packets;
    le64 rx_bytes;

    le64 rx_interrupts;

    le64 rx_drops;
    le64 rx_drop_overruns;
};
\end{lstlisting}

The packets described below were all presented on the specified virtqueue.
\begin{description}
    \item [rx_notifications]
        The number of driver notifications received by the device for this
        receiveq.

    \item [rx_packets]
        This is the number of packets passed to the driver by the device.

    \item [rx_bytes]
        This is the bytes of packets passed to the driver by the device.

    \item [rx_interrupts]
        The number of interrupts generated by the device for this receiveq.

    \item [rx_drops]
        This is the number of packets dropped by the device. The count includes
        all types of packets dropped by the device.

    \item [rx_drop_overruns]
        This is the number of packets dropped by the device when no more
        descriptors were available.

\end{description}

\subparagraph{Transmitq Basic Statistics}\label{sec:Device Types / Network Device / Device Operation / Control Virtqueue / Device Statistics / Transmitq Basic Statistics}

The structure corresponding to the transmitq basic statistics is
\field{struct virtio_net_stats_tx_basic}. The corresponding type is
VIRTIO_NET_STATS_TYPE_TX_BASIC. This is for the transmitq.

Transmitq basic statistics do not require any feature. As long as the device supports
VIRTIO_NET_F_DEVICE_STATS, the following are the transmitq basic statistics.

\begin{lstlisting}
struct virtio_net_stats_tx_basic {
    struct virtio_net_stats_reply_hdr hdr;

    le64 tx_notifications;

    le64 tx_packets;
    le64 tx_bytes;

    le64 tx_interrupts;

    le64 tx_drops;
    le64 tx_drop_malformed;
};
\end{lstlisting}

The packets described below are all for a specific virtqueue.
\begin{description}
    \item [tx_notifications]
        The number of driver notifications received by the device for this
        transmitq.

    \item [tx_packets]
        This is the number of packets sent by the device (not the packets
        got from the driver).

    \item [tx_bytes]
        This is the number of bytes sent by the device for all the sent packets
        (not the bytes sent got from the driver).

    \item [tx_interrupts]
        The number of interrupts generated by the device for this transmitq.

    \item [tx_drops]
        The number of packets dropped by the device. The count includes all
        types of packets dropped by the device.

    \item [tx_drop_malformed]
        The number of packets dropped by the device, when the descriptors are
        malformed. For example, the buffer is too short.
\end{description}

\subparagraph{Receiveq CSUM Statistics}\label{sec:Device Types / Network Device / Device Operation / Control Virtqueue / Device Statistics / Receiveq CSUM Statistics}

The structure corresponding to the receiveq checksum statistics is
\field{struct virtio_net_stats_rx_csum}. The corresponding type is
VIRTIO_NET_STATS_TYPE_RX_CSUM. This is for the receiveq.

Only after the VIRTIO_NET_F_GUEST_CSUM is negotiated, the receiveq checksum
statistics can be obtained.

\begin{lstlisting}
struct virtio_net_stats_rx_csum {
    struct virtio_net_stats_reply_hdr hdr;

    le64 rx_csum_valid;
    le64 rx_needs_csum;
    le64 rx_csum_none;
    le64 rx_csum_bad;
};
\end{lstlisting}

The packets described below were all presented on the specified virtqueue.
\begin{description}
    \item [rx_csum_valid]
        The number of packets with VIRTIO_NET_HDR_F_DATA_VALID.

    \item [rx_needs_csum]
        The number of packets with VIRTIO_NET_HDR_F_NEEDS_CSUM.

    \item [rx_csum_none]
        The number of packets without hardware checksum. The packet here refers
        to the non-TCP/UDP packet that the device cannot recognize.

    \item [rx_csum_bad]
        The number of packets with checksum mismatch.

\end{description}

\subparagraph{Transmitq CSUM Statistics}\label{sec:Device Types / Network Device / Device Operation / Control Virtqueue / Device Statistics / Transmitq CSUM Statistics}

The structure corresponding to the transmitq checksum statistics is
\field{struct virtio_net_stats_tx_csum}. The corresponding type is
VIRTIO_NET_STATS_TYPE_TX_CSUM. This is for the transmitq.

Only after the VIRTIO_NET_F_CSUM is negotiated, the transmitq checksum
statistics can be obtained.

The following are the transmitq checksum statistics:

\begin{lstlisting}
struct virtio_net_stats_tx_csum {
    struct virtio_net_stats_reply_hdr hdr;

    le64 tx_csum_none;
    le64 tx_needs_csum;
};
\end{lstlisting}

The packets described below are all for a specific virtqueue.
\begin{description}
    \item [tx_csum_none]
        The number of packets which do not require hardware checksum.

    \item [tx_needs_csum]
        The number of packets which require checksum calculation by the device.

\end{description}

\subparagraph{Receiveq GSO Statistics}\label{sec:Device Types / Network Device / Device Operation / Control Virtqueue / Device Statistics / Receiveq GSO Statistics}

The structure corresponding to the receivq GSO statistics is
\field{struct virtio_net_stats_rx_gso}. The corresponding type is
VIRTIO_NET_STATS_TYPE_RX_GSO. This is for the receiveq.

If one or more of the VIRTIO_NET_F_GUEST_TSO4, VIRTIO_NET_F_GUEST_TSO6
have been negotiated, the receiveq GSO statistics can be obtained.

GSO packets refer to packets passed by the device to the driver where
\field{gso_type} is not VIRTIO_NET_HDR_GSO_NONE.

\begin{lstlisting}
struct virtio_net_stats_rx_gso {
    struct virtio_net_stats_reply_hdr hdr;

    le64 rx_gso_packets;
    le64 rx_gso_bytes;
    le64 rx_gso_packets_coalesced;
    le64 rx_gso_bytes_coalesced;
};
\end{lstlisting}

The packets described below were all presented on the specified virtqueue.
\begin{description}
    \item [rx_gso_packets]
        The number of the GSO packets received by the device.

    \item [rx_gso_bytes]
        The bytes of the GSO packets received by the device.
        This includes the header size of the GSO packet.

    \item [rx_gso_packets_coalesced]
        The number of the GSO packets coalesced by the device.

    \item [rx_gso_bytes_coalesced]
        The bytes of the GSO packets coalesced by the device.
        This includes the header size of the GSO packet.
\end{description}

\subparagraph{Transmitq GSO Statistics}\label{sec:Device Types / Network Device / Device Operation / Control Virtqueue / Device Statistics / Transmitq GSO Statistics}

The structure corresponding to the transmitq GSO statistics is
\field{struct virtio_net_stats_tx_gso}. The corresponding type is
VIRTIO_NET_STATS_TYPE_TX_GSO. This is for the transmitq.

If one or more of the VIRTIO_NET_F_HOST_TSO4, VIRTIO_NET_F_HOST_TSO6,
VIRTIO_NET_F_HOST_USO options have been negotiated, the transmitq GSO statistics
can be obtained.

GSO packets refer to packets passed by the driver to the device where
\field{gso_type} is not VIRTIO_NET_HDR_GSO_NONE.
See more \ref{sec:Device Types / Network Device / Device Operation / Packet
Transmission}.

\begin{lstlisting}
struct virtio_net_stats_tx_gso {
    struct virtio_net_stats_reply_hdr hdr;

    le64 tx_gso_packets;
    le64 tx_gso_bytes;
    le64 tx_gso_segments;
    le64 tx_gso_segments_bytes;
    le64 tx_gso_packets_noseg;
    le64 tx_gso_bytes_noseg;
};
\end{lstlisting}

The packets described below are all for a specific virtqueue.
\begin{description}
    \item [tx_gso_packets]
        The number of the GSO packets sent by the device.

    \item [tx_gso_bytes]
        The bytes of the GSO packets sent by the device.

    \item [tx_gso_segments]
        The number of segments prepared from GSO packets.

    \item [tx_gso_segments_bytes]
        The bytes of segments prepared from GSO packets.

    \item [tx_gso_packets_noseg]
        The number of the GSO packets without segmentation.

    \item [tx_gso_bytes_noseg]
        The bytes of the GSO packets without segmentation.

\end{description}

\subparagraph{Receiveq Speed Statistics}\label{sec:Device Types / Network Device / Device Operation / Control Virtqueue / Device Statistics / Receiveq Speed Statistics}

The structure corresponding to the receiveq speed statistics is
\field{struct virtio_net_stats_rx_speed}. The corresponding type is
VIRTIO_NET_STATS_TYPE_RX_SPEED. This is for the receiveq.

The device has the allowance for the speed. If VIRTIO_NET_F_SPEED_DUPLEX has
been negotiated, the driver can get this by \field{speed}. When the received
packets bitrate exceeds the \field{speed}, some packets may be dropped by the
device.

\begin{lstlisting}
struct virtio_net_stats_rx_speed {
    struct virtio_net_stats_reply_hdr hdr;

    le64 rx_packets_allowance_exceeded;
    le64 rx_bytes_allowance_exceeded;
};
\end{lstlisting}

The packets described below were all presented on the specified virtqueue.
\begin{description}
    \item [rx_packets_allowance_exceeded]
        The number of the packets dropped by the device due to the received
        packets bitrate exceeding the \field{speed}.

    \item [rx_bytes_allowance_exceeded]
        The bytes of the packets dropped by the device due to the received
        packets bitrate exceeding the \field{speed}.

\end{description}

\subparagraph{Transmitq Speed Statistics}\label{sec:Device Types / Network Device / Device Operation / Control Virtqueue / Device Statistics / Transmitq Speed Statistics}

The structure corresponding to the transmitq speed statistics is
\field{struct virtio_net_stats_tx_speed}. The corresponding type is
VIRTIO_NET_STATS_TYPE_TX_SPEED. This is for the transmitq.

The device has the allowance for the speed. If VIRTIO_NET_F_SPEED_DUPLEX has
been negotiated, the driver can get this by \field{speed}. When the transmit
packets bitrate exceeds the \field{speed}, some packets may be dropped by the
device.

\begin{lstlisting}
struct virtio_net_stats_tx_speed {
    struct virtio_net_stats_reply_hdr hdr;

    le64 tx_packets_allowance_exceeded;
    le64 tx_bytes_allowance_exceeded;
};
\end{lstlisting}

The packets described below were all presented on the specified virtqueue.
\begin{description}
    \item [tx_packets_allowance_exceeded]
        The number of the packets dropped by the device due to the transmit packets
        bitrate exceeding the \field{speed}.

    \item [tx_bytes_allowance_exceeded]
        The bytes of the packets dropped by the device due to the transmit packets
        bitrate exceeding the \field{speed}.

\end{description}

\devicenormative{\subparagraph}{Device Statistics}{Device Types / Network Device / Device Operation / Control Virtqueue / Device Statistics}

When the VIRTIO_NET_F_DEVICE_STATS feature is negotiated, the device MUST reply
to the command VIRTIO_NET_CTRL_STATS_QUERY with the
\field{struct virtio_net_stats_capabilities}. \field{supported_stats_types}
includes all the statistic types supported by the device.

If \field{struct virtio_net_ctrl_queue_stats} is incorrect (such as the
following), the device MUST set \field{ack} to VIRTIO_NET_ERR. Even if there is
only one error, the device MUST fail the entire command.
\begin{itemize}
    \item \field{vq_index} exceeds the queue range.
    \item \field{types_bitmap} contains unknown types.
    \item One or more of the bits present in \field{types_bitmap} is not valid
        for the specified virtqueue.
    \item The feature corresponding to the specified \field{types_bitmap} was
        not negotiated.
\end{itemize}

The device MUST set the actual size of the bytes occupied by the reply to the
\field{size} of the \field{hdr}. And the device MUST set the \field{type} and
the \field{vq_index} of the statistic header.

The \field{command-specific-result} buffer allocated by the driver may be
smaller or bigger than all the statistics specified by
\field{struct virtio_net_ctrl_queue_stats}. The device MUST fill up only upto
the valid bytes.

The statistics counter replied by the device MUST wrap around to zero by the
device on the overflow.

\drivernormative{\subparagraph}{Device Statistics}{Device Types / Network Device / Device Operation / Control Virtqueue / Device Statistics}

The types contained in the \field{types_bitmap} MUST be queried from the device
via command VIRTIO_NET_CTRL_STATS_QUERY.

\field{types_bitmap} in \field{struct virtio_net_ctrl_queue_stats} MUST be valid to the
vq specified by \field{vq_index}.

The \field{command-specific-result} buffer allocated by the driver MUST have
enough capacity to store all the statistics reply headers defined in
\field{struct virtio_net_ctrl_queue_stats}. If the
\field{command-specific-result} buffer is fully utilized by the device but some
replies are missed, it is possible that some statistics may exceed the capacity
of the driver's records. In such cases, the driver should allocate additional
space for the \field{command-specific-result} buffer.

\subsubsection{Flow filter}\label{sec:Device Types / Network Device / Device Operation / Flow filter}

A network device can support one or more flow filter rules. Each flow filter rule
is applied by matching a packet and then taking an action, such as directing the packet
to a specific receiveq or dropping the packet. An example of a match is
matching on specific source and destination IP addresses.

A flow filter rule is a device resource object that consists of a key,
a processing priority, and an action to either direct a packet to a
receive queue or drop the packet.

Each rule uses a classifier. The key is matched against the packet using
a classifier, defining which fields in the packet are matched.
A classifier resource object consists of one or more field selectors, each with
a type that specifies the header fields to be matched against, and a mask.
The mask can match whole fields or parts of a field in a header. Each
rule resource object depends on the classifier resource object.

When a packet is received, relevant fields are extracted
(in the same way) from both the packet and the key according to the
classifier. The resulting field contents are then compared -
if they are identical the rule action is taken, if they are not, the rule is ignored.

Multiple flow filter rules are part of a group. The rule resource object
depends on the group. Each rule within a
group has a rule priority, and each group also has a group priority. For a
packet, a group with the highest priority is selected first. Within a group,
rules are applied from highest to lowest priority, until one of the rules
matches the packet and an action is taken. If all the rules within a group
are ignored, the group with the next highest priority is selected, and so on.

The device and the driver indicates flow filter resource limits using the capability
\ref{par:Device Types / Network Device / Device Operation / Flow filter / Device and driver capabilities / VIRTIO-NET-FF-RESOURCE-CAP} specifying the limits on the number of flow filter rule,
group and classifier resource objects. The capability \ref{par:Device Types / Network Device / Device Operation / Flow filter / Device and driver capabilities / VIRTIO-NET-FF-SELECTOR-CAP} specifies which selectors the device supports.
The driver indicates the selectors it is using by setting the flow
filter selector capability, prior to adding any resource objects.

The capability \ref{par:Device Types / Network Device / Device Operation / Flow filter / Device and driver capabilities / VIRTIO-NET-FF-ACTION-CAP} specifies which actions the device supports.

The driver controls the flow filter rule, classifier and group resource objects using
administration commands described in
\ref{sec:Basic Facilities of a Virtio Device / Device groups / Group administration commands / Device resource objects}.

\paragraph{Packet processing order}\label{sec:sec:Device Types / Network Device / Device Operation / Flow filter / Packet processing order}

Note that flow filter rules are applied after MAC/VLAN filtering. Flow filter
rules take precedence over steering: if a flow filter rule results in an action,
the steering configuration does not apply. The steering configuration only applies
to packets for which no flow filter rule action was performed. For example,
incoming packets can be processed in the following order:

\begin{itemize}
\item apply steering configuration received using control virtqueue commands
      VIRTIO_NET_CTRL_RX, VIRTIO_NET_CTRL_MAC and VIRTIO_NET_CTRL_VLAN.
\item apply flow filter rules if any.
\item if no filter rule applied, apply steering configuration received using command
      VIRTIO_NET_CTRL_MQ_RSS_CONFIG or as per automatic receive steering.
\end{itemize}

Some incoming packet processing examples:
\begin{itemize}
\item If the packet is dropped by the flow filter rule, RSS
      steering is ignored for the packet.
\item If the packet is directed to a specific receiveq using flow filter rule,
      the RSS steering is ignored for the packet.
\item If a packet is dropped due to the VIRTIO_NET_CTRL_MAC configuration,
      both flow filter rules and the RSS steering are ignored for the packet.
\item If a packet does not match any flow filter rules,
      the RSS steering is used to select the receiveq for the packet (if enabled).
\item If there are two flow filter groups configured as group_A and group_B
      with respective group priorities as 4, and 5; flow filter rules of
      group_B are applied first having highest group priority, if there is a match,
      the flow filter rules of group_A are ignored; if there is no match for
      the flow filter rules in group_B, the flow filter rules of next level group_A are applied.
\end{itemize}

\paragraph{Device and driver capabilities}
\label{par:Device Types / Network Device / Device Operation / Flow filter / Device and driver capabilities}

\subparagraph{VIRTIO_NET_FF_RESOURCE_CAP}
\label{par:Device Types / Network Device / Device Operation / Flow filter / Device and driver capabilities / VIRTIO-NET-FF-RESOURCE-CAP}

The capability VIRTIO_NET_FF_RESOURCE_CAP indicates the flow filter resource limits.
\field{cap_specific_data} is in the format
\field{struct virtio_net_ff_cap_data}.

\begin{lstlisting}
struct virtio_net_ff_cap_data {
        le32 groups_limit;
        le32 selectors_limit;
        le32 rules_limit;
        le32 rules_per_group_limit;
        u8 last_rule_priority;
        u8 selectors_per_classifier_limit;
};
\end{lstlisting}

\field{groups_limit}, and \field{selectors_limit} represent the maximum
number of flow filter groups and selectors, respectively, that the driver can create.
 \field{rules_limit} is the maximum number of
flow fiilter rules that the driver can create across all the groups.
\field{rules_per_group_limit} is the maximum number of flow filter rules that the driver
can create for each flow filter group.

\field{last_rule_priority} is the highest priority that can be assigned to a
flow filter rule.

\field{selectors_per_classifier_limit} is the maximum number of selectors
that a classifier can have.

\subparagraph{VIRTIO_NET_FF_SELECTOR_CAP}
\label{par:Device Types / Network Device / Device Operation / Flow filter / Device and driver capabilities / VIRTIO-NET-FF-SELECTOR-CAP}

The capability VIRTIO_NET_FF_SELECTOR_CAP lists the supported selectors and the
supported packet header fields for each selector.
\field{cap_specific_data} is in the format \field{struct virtio_net_ff_cap_mask_data}.

\begin{lstlisting}[label={lst:Device Types / Network Device / Device Operation / Flow filter / Device and driver capabilities / VIRTIO-NET-FF-SELECTOR-CAP / virtio-net-ff-selector}]
struct virtio_net_ff_selector {
        u8 type;
        u8 flags;
        u8 reserved[2];
        u8 length;
        u8 reserved1[3];
        u8 mask[];
};

struct virtio_net_ff_cap_mask_data {
        u8 count;
        u8 reserved[7];
        struct virtio_net_ff_selector selectors[];
};

#define VIRTIO_NET_FF_MASK_F_PARTIAL_MASK (1 << 0)
\end{lstlisting}

\field{count} indicates number of valid entries in the \field{selectors} array.
\field{selectors[]} is an array of supported selectors. Within each array entry:
\field{type} specifies the type of the packet header, as defined in table
\ref{table:Device Types / Network Device / Device Operation / Flow filter / Device and driver capabilities / VIRTIO-NET-FF-SELECTOR-CAP / flow filter selector types}. \field{mask} specifies which fields of the
packet header can be matched in a flow filter rule.

Each \field{type} is also listed in table
\ref{table:Device Types / Network Device / Device Operation / Flow filter / Device and driver capabilities / VIRTIO-NET-FF-SELECTOR-CAP / flow filter selector types}. \field{mask} is a byte array
in network byte order. For example, when \field{type} is VIRTIO_NET_FF_MASK_TYPE_IPV6,
the \field{mask} is in the format \hyperref[intro:IPv6-Header-Format]{IPv6 Header Format}.

If partial masking is not set, then all bits in each field have to be either all 0s
to ignore this field or all 1s to match on this field. If partial masking is set,
then any combination of bits can bit set to match on these bits.
For example, when a selector \field{type} is VIRTIO_NET_FF_MASK_TYPE_ETH, if
\field{mask[0-12]} are zero and \field{mask[13-14]} are 0xff (all 1s), it
indicates that matching is only supported for \field{EtherType} of
\field{Ethernet MAC frame}, matching is not supported for
\field{Destination Address} and \field{Source Address}.

The entries in the array \field{selectors} are ordered by
\field{type}, with each \field{type} value only appearing once.

\field{length} is the length of a dynamic array \field{mask} in bytes.
\field{reserved} and \field{reserved1} are reserved and set to zero.

\begin{table}[H]
\caption{Flow filter selector types}
\label{table:Device Types / Network Device / Device Operation / Flow filter / Device and driver capabilities / VIRTIO-NET-FF-SELECTOR-CAP / flow filter selector types}
\begin{tabularx}{\textwidth}{ |l|X|X| }
\hline
Type & Name & Description \\
\hline \hline
0x0 & - & Reserved \\
\hline
0x1 & VIRTIO_NET_FF_MASK_TYPE_ETH & 14 bytes of frame header starting from destination address described in \hyperref[intro:IEEE 802.3-2022]{IEEE 802.3-2022} \\
\hline
0x2 & VIRTIO_NET_FF_MASK_TYPE_IPV4 & 20 bytes of \hyperref[intro:Internet-Header-Format]{IPv4: Internet Header Format} \\
\hline
0x3 & VIRTIO_NET_FF_MASK_TYPE_IPV6 & 40 bytes of \hyperref[intro:IPv6-Header-Format]{IPv6 Header Format} \\
\hline
0x4 & VIRTIO_NET_FF_MASK_TYPE_TCP & 20 bytes of \hyperref[intro:TCP-Header-Format]{TCP Header Format} \\
\hline
0x5 & VIRTIO_NET_FF_MASK_TYPE_UDP & 8 bytes of UDP header described in \hyperref[intro:UDP]{UDP} \\
\hline
0x6 - 0xFF & & Reserved for future \\
\hline
\end{tabularx}
\end{table}

When VIRTIO_NET_FF_MASK_F_PARTIAL_MASK (bit 0) is set, it indicates that
partial masking is supported for all the fields of the selector identified by \field{type}.

For the selector \field{type} VIRTIO_NET_FF_MASK_TYPE_IPV4, if a partial mask is unsupported,
then matching on an individual bit of \field{Flags} in the
\field{IPv4: Internet Header Format} is unsupported. \field{Flags} has to match as a whole
if it is supported.

For the selector \field{type} VIRTIO_NET_FF_MASK_TYPE_IPV4, \field{mask} includes fields
up to the \field{Destination Address}; that is, \field{Options} and
\field{Padding} are excluded.

For the selector \field{type} VIRTIO_NET_FF_MASK_TYPE_IPV6, the \field{Next Header} field
of the \field{mask} corresponds to the \field{Next Header} in the packet
when \field{IPv6 Extension Headers} are not present. When the packet includes
one or more \field{IPv6 Extension Headers}, the \field{Next Header} field of
the \field{mask} corresponds to the \field{Next Header} of the last
\field{IPv6 Extension Header} in the packet.

For the selector \field{type} VIRTIO_NET_FF_MASK_TYPE_TCP, \field{Control bits}
are treated as individual fields for matching; that is, matching individual
\field{Control bits} does not depend on the partial mask support.

\subparagraph{VIRTIO_NET_FF_ACTION_CAP}
\label{par:Device Types / Network Device / Device Operation / Flow filter / Device and driver capabilities / VIRTIO-NET-FF-ACTION-CAP}

The capability VIRTIO_NET_FF_ACTION_CAP lists the supported actions in a rule.
\field{cap_specific_data} is in the format \field{struct virtio_net_ff_cap_actions}.

\begin{lstlisting}
struct virtio_net_ff_actions {
        u8 count;
        u8 reserved[7];
        u8 actions[];
};
\end{lstlisting}

\field{actions} is an array listing all possible actions.
The entries in the array are ordered from the smallest to the largest,
with each supported value appearing exactly once. Each entry can have the
following values:

\begin{table}[H]
\caption{Flow filter rule actions}
\label{table:Device Types / Network Device / Device Operation / Flow filter / Device and driver capabilities / VIRTIO-NET-FF-ACTION-CAP / flow filter rule actions}
\begin{tabularx}{\textwidth}{ |l|X|X| }
\hline
Action & Name & Description \\
\hline \hline
0x0 & - & reserved \\
\hline
0x1 & VIRTIO_NET_FF_ACTION_DROP & Matching packet will be dropped by the device \\
\hline
0x2 & VIRTIO_NET_FF_ACTION_DIRECT_RX_VQ & Matching packet will be directed to a receive queue \\
\hline
0x3 - 0xFF & & Reserved for future \\
\hline
\end{tabularx}
\end{table}

\paragraph{Resource objects}
\label{par:Device Types / Network Device / Device Operation / Flow filter / Resource objects}

\subparagraph{VIRTIO_NET_RESOURCE_OBJ_FF_GROUP}\label{par:Device Types / Network Device / Device Operation / Flow filter / Resource objects / VIRTIO-NET-RESOURCE-OBJ-FF-GROUP}

A flow filter group contains between 0 and \field{rules_limit} rules, as specified by the
capability VIRTIO_NET_FF_RESOURCE_CAP. For the flow filter group object both
\field{resource_obj_specific_data} and
\field{resource_obj_specific_result} are in the format
\field{struct virtio_net_resource_obj_ff_group}.

\begin{lstlisting}
struct virtio_net_resource_obj_ff_group {
        le16 group_priority;
};
\end{lstlisting}

\field{group_priority} specifies the priority for the group. Each group has a
distinct priority. For each incoming packet, the device tries to apply rules
from groups from higher \field{group_priority} value to lower, until either a
rule matches the packet or all groups have been tried.

\subparagraph{VIRTIO_NET_RESOURCE_OBJ_FF_CLASSIFIER}\label{par:Device Types / Network Device / Device Operation / Flow filter / Resource objects / VIRTIO-NET-RESOURCE-OBJ-FF-CLASSIFIER}

A classifier is used to match a flow filter key against a packet. The
classifier defines the desired packet fields to match, and is represented by
the VIRTIO_NET_RESOURCE_OBJ_FF_CLASSIFIER device resource object.

For the flow filter classifier object both \field{resource_obj_specific_data} and
\field{resource_obj_specific_result} are in the format
\field{struct virtio_net_resource_obj_ff_classifier}.

\begin{lstlisting}
struct virtio_net_resource_obj_ff_classifier {
        u8 count;
        u8 reserved[7];
        struct virtio_net_ff_selector selectors[];
};
\end{lstlisting}

A classifier is an array of \field{selectors}. The number of selectors in the
array is indicated by \field{count}. The selector has a type that specifies
the header fields to be matched against, and a mask.
See \ref{lst:Device Types / Network Device / Device Operation / Flow filter / Device and driver capabilities / VIRTIO-NET-FF-SELECTOR-CAP / virtio-net-ff-selector}
for details about selectors.

The first selector is always VIRTIO_NET_FF_MASK_TYPE_ETH. When there are multiple
selectors, a second selector can be either VIRTIO_NET_FF_MASK_TYPE_IPV4
or VIRTIO_NET_FF_MASK_TYPE_IPV6. If the third selector exists, the third
selector can be either VIRTIO_NET_FF_MASK_TYPE_UDP or VIRTIO_NET_FF_MASK_TYPE_TCP.
For example, to match a Ethernet IPv6 UDP packet,
\field{selectors[0].type} is set to VIRTIO_NET_FF_MASK_TYPE_ETH, \field{selectors[1].type}
is set to VIRTIO_NET_FF_MASK_TYPE_IPV6 and \field{selectors[2].type} is
set to VIRTIO_NET_FF_MASK_TYPE_UDP; accordingly, \field{selectors[0].mask[0-13]} is
for Ethernet header fields, \field{selectors[1].mask[0-39]} is set for IPV6 header
and \field{selectors[2].mask[0-7]} is set for UDP header.

When there are multiple selectors, the type of the (N+1)\textsuperscript{th} selector
affects the mask of the (N)\textsuperscript{th} selector. If
\field{count} is 2 or more, all the mask bits within \field{selectors[0]}
corresponding to \field{EtherType} of an Ethernet header are set.

If \field{count} is more than 2:
\begin{itemize}
\item if \field{selector[1].type} is, VIRTIO_NET_FF_MASK_TYPE_IPV4, then, all the mask bits within
\field{selector[1]} for \field{Protocol} is set.
\item if \field{selector[1].type} is, VIRTIO_NET_FF_MASK_TYPE_IPV6, then, all the mask bits within
\field{selector[1]} for \field{Next Header} is set.
\end{itemize}

If for a given packet header field, a subset of bits of a field is to be matched,
and if the partial mask is supported, the flow filter
mask object can specify a mask which has fewer bits set than the packet header
field size. For example, a partial mask for the Ethernet header source mac
address can be of 1-bit for multicast detection instead of 48-bits.

\subparagraph{VIRTIO_NET_RESOURCE_OBJ_FF_RULE}\label{par:Device Types / Network Device / Device Operation / Flow filter / Resource objects / VIRTIO-NET-RESOURCE-OBJ-FF-RULE}

Each flow filter rule resource object comprises a key, a priority, and an action.
For the flow filter rule object,
\field{resource_obj_specific_data} and
\field{resource_obj_specific_result} are in the format
\field{struct virtio_net_resource_obj_ff_rule}.

\begin{lstlisting}
struct virtio_net_resource_obj_ff_rule {
        le32 group_id;
        le32 classifier_id;
        u8 rule_priority;
        u8 key_length; /* length of key in bytes */
        u8 action;
        u8 reserved;
        le16 vq_index;
        u8 reserved1[2];
        u8 keys[][];
};
\end{lstlisting}

\field{group_id} is the resource object ID of the flow filter group to which
this rule belongs. \field{classifier_id} is the resource object ID of the
classifier used to match a packet against the \field{key}.

\field{rule_priority} denotes the priority of the rule within the group
specified by the \field{group_id}.
Rules within the group are applied from the highest to the lowest priority
until a rule matches the packet and an
action is taken. Rules with the same priority can be applied in any order.

\field{reserved} and \field{reserved1} are reserved and set to 0.

\field{keys[][]} is an array of keys to match against packets, using
the classifier specified by \field{classifier_id}. Each entry (key) comprises
a byte array, and they are located one immediately after another.
The size (number of entries) of the array is exactly the same as that of
\field{selectors} in the classifier, or in other words, \field{count}
in the classifier.

\field{key_length} specifies the total length of \field{keys} in bytes.
In other words, it equals the sum total of \field{length} of all
selectors in \field{selectors} in the classifier specified by
\field{classifier_id}.

For example, if a classifier object's \field{selectors[0].type} is
VIRTIO_NET_FF_MASK_TYPE_ETH and \field{selectors[1].type} is
VIRTIO_NET_FF_MASK_TYPE_IPV6,
then selectors[0].length is 14 and selectors[1].length is 40.
Accordingly, the \field{key_length} is set to 54.
This setting indicates that the \field{key} array's length is 54 bytes
comprising a first byte array of 14 bytes for the
Ethernet MAC header in bytes 0-13, immediately followed by 40 bytes for the
IPv6 header in bytes 14-53.

When there are multiple selectors in the classifier object, the key bytes
for (N)\textsuperscript{th} selector are set so that
(N+1)\textsuperscript{th} selector can be matched.

If \field{count} is 2 or more, key bytes of \field{EtherType}
are set according to \hyperref[intro:IEEE 802 Ethertypes]{IEEE 802 Ethertypes}
for VIRTIO_NET_FF_MASK_TYPE_IPV4 or VIRTIO_NET_FF_MASK_TYPE_IPV6 respectively.

If \field{count} is more than 2, when \field{selector[1].type} is
VIRTIO_NET_FF_MASK_TYPE_IPV4 or VIRTIO_NET_FF_MASK_TYPE_IPV6, key
bytes of \field{Protocol} or \field{Next Header} is set as per
\field{Protocol Numbers} defined \hyperref[intro:IANA Protocol Numbers]{IANA Protocol Numbers}
respectively.

\field{action} is the action to take when a packet matches the
\field{key} using the \field{classifier_id}. Supported actions are described in
\ref{table:Device Types / Network Device / Device Operation / Flow filter / Device and driver capabilities / VIRTIO-NET-FF-ACTION-CAP / flow filter rule actions}.

\field{vq_index} specifies a receive virtqueue. When the \field{action} is set
to VIRTIO_NET_FF_ACTION_DIRECT_RX_VQ, and the packet matches the \field{key},
the matching packet is directed to this virtqueue.

Note that at most one action is ever taken for a given packet. If a rule is
applied and an action is taken, the action of other rules is not taken.

\devicenormative{\paragraph}{Flow filter}{Device Types / Network Device / Device Operation / Flow filter}

When the device supports flow filter operations,
\begin{itemize}
\item the device MUST set VIRTIO_NET_FF_RESOURCE_CAP, VIRTIO_NET_FF_SELECTOR_CAP
and VIRTIO_NET_FF_ACTION_CAP capability in the \field{supported_caps} in the
command VIRTIO_ADMIN_CMD_CAP_SUPPORT_QUERY.
\item the device MUST support the administration commands
VIRTIO_ADMIN_CMD_RESOURCE_OBJ_CREATE,
VIRTIO_ADMIN_CMD_RESOURCE_OBJ_MODIFY, VIRTIO_ADMIN_CMD_RESOURCE_OBJ_QUERY,
VIRTIO_ADMIN_CMD_RESOURCE_OBJ_DESTROY for the resource types
VIRTIO_NET_RESOURCE_OBJ_FF_GROUP, VIRTIO_NET_RESOURCE_OBJ_FF_CLASSIFIER and
VIRTIO_NET_RESOURCE_OBJ_FF_RULE.
\end{itemize}

When any of the VIRTIO_NET_FF_RESOURCE_CAP, VIRTIO_NET_FF_SELECTOR_CAP, or
VIRTIO_NET_FF_ACTION_CAP capability is disabled, the device SHOULD set
\field{status} to VIRTIO_ADMIN_STATUS_Q_INVALID_OPCODE for the commands
VIRTIO_ADMIN_CMD_RESOURCE_OBJ_CREATE,
VIRTIO_ADMIN_CMD_RESOURCE_OBJ_MODIFY, VIRTIO_ADMIN_CMD_RESOURCE_OBJ_QUERY,
and VIRTIO_ADMIN_CMD_RESOURCE_OBJ_DESTROY. These commands apply to the resource
\field{type} of VIRTIO_NET_RESOURCE_OBJ_FF_GROUP, VIRTIO_NET_RESOURCE_OBJ_FF_CLASSIFIER, and
VIRTIO_NET_RESOURCE_OBJ_FF_RULE.

The device SHOULD set \field{status} to VIRTIO_ADMIN_STATUS_EINVAL for the
command VIRTIO_ADMIN_CMD_RESOURCE_OBJ_CREATE when the resource \field{type}
is VIRTIO_NET_RESOURCE_OBJ_FF_GROUP, if a flow filter group already exists
with the supplied \field{group_priority}.

The device SHOULD set \field{status} to VIRTIO_ADMIN_STATUS_ENOSPC for the
command VIRTIO_ADMIN_CMD_RESOURCE_OBJ_CREATE when the resource \field{type}
is VIRTIO_NET_RESOURCE_OBJ_FF_GROUP, if the number of flow filter group
objects in the device exceeds the lower of the configured driver
capabilities \field{groups_limit} and \field{rules_per_group_limit}.

The device SHOULD set \field{status} to VIRTIO_ADMIN_STATUS_ENOSPC for the
command VIRTIO_ADMIN_CMD_RESOURCE_OBJ_CREATE when the resource \field{type} is
VIRTIO_NET_RESOURCE_OBJ_FF_CLASSIFIER, if the number of flow filter selector
objects in the device exceeds the configured driver capability
\field{selectors_limit}.

The device SHOULD set \field{status} to VIRTIO_ADMIN_STATUS_EBUSY for the
command VIRTIO_ADMIN_CMD_RESOURCE_OBJ_DESTROY for a flow filter group when
the flow filter group has one or more flow filter rules depending on it.

The device SHOULD set \field{status} to VIRTIO_ADMIN_STATUS_EBUSY for the
command VIRTIO_ADMIN_CMD_RESOURCE_OBJ_DESTROY for a flow filter classifier when
the flow filter classifier has one or more flow filter rules depending on it.

The device SHOULD fail the command VIRTIO_ADMIN_CMD_RESOURCE_OBJ_CREATE for the
flow filter rule resource object if,
\begin{itemize}
\item \field{vq_index} is not a valid receive virtqueue index for
the VIRTIO_NET_FF_ACTION_DIRECT_RX_VQ action,
\item \field{priority} is greater than or equal to
      \field{last_rule_priority},
\item \field{id} is greater than or equal to \field{rules_limit} or
      greater than or equal to \field{rules_per_group_limit}, whichever is lower,
\item the length of \field{keys} and the length of all the mask bytes of
      \field{selectors[].mask} as referred by \field{classifier_id} differs,
\item the supplied \field{action} is not supported in the capability VIRTIO_NET_FF_ACTION_CAP.
\end{itemize}

When the flow filter directs a packet to the virtqueue identified by
\field{vq_index} and if the receive virtqueue is reset, the device
MUST drop such packets.

Upon applying a flow filter rule to a packet, the device MUST STOP any further
application of rules and cease applying any other steering configurations.

For multiple flow filter groups, the device MUST apply the rules from
the group with the highest priority. If any rule from this group is applied,
the device MUST ignore the remaining groups. If none of the rules from the
highest priority group match, the device MUST apply the rules from
the group with the next highest priority, until either a rule matches or
all groups have been attempted.

The device MUST apply the rules within the group from the highest to the
lowest priority until a rule matches the packet, and the device MUST take
the action. If an action is taken, the device MUST not take any other
action for this packet.

The device MAY apply the rules with the same \field{rule_priority} in any
order within the group.

The device MUST process incoming packets in the following order:
\begin{itemize}
\item apply the steering configuration received using control virtqueue
      commands VIRTIO_NET_CTRL_RX, VIRTIO_NET_CTRL_MAC, and
      VIRTIO_NET_CTRL_VLAN.
\item apply flow filter rules if any.
\item if no filter rule is applied, apply the steering configuration
      received using the command VIRTIO_NET_CTRL_MQ_RSS_CONFIG
      or according to automatic receive steering.
\end{itemize}

When processing an incoming packet, if the packet is dropped at any stage, the device
MUST skip further processing.

When the device drops the packet due to the configuration done using the control
virtqueue commands VIRTIO_NET_CTRL_RX or VIRTIO_NET_CTRL_MAC or VIRTIO_NET_CTRL_VLAN,
the device MUST skip flow filter rules for this packet.

When the device performs flow filter match operations and if the operation
result did not have any match in all the groups, the receive packet processing
continues to next level, i.e. to apply configuration done using
VIRTIO_NET_CTRL_MQ_RSS_CONFIG command.

The device MUST support the creation of flow filter classifier objects
using the command VIRTIO_ADMIN_CMD_RESOURCE_OBJ_CREATE with \field{flags}
set to VIRTIO_NET_FF_MASK_F_PARTIAL_MASK;
this support is required even if all the bits of the masks are set for
a field in \field{selectors}, provided that partial masking is supported
for the selectors.

\drivernormative{\paragraph}{Flow filter}{Device Types / Network Device / Device Operation / Flow filter}

The driver MUST enable VIRTIO_NET_FF_RESOURCE_CAP, VIRTIO_NET_FF_SELECTOR_CAP,
and VIRTIO_NET_FF_ACTION_CAP capabilities to use flow filter.

The driver SHOULD NOT remove a flow filter group using the command
VIRTIO_ADMIN_CMD_RESOURCE_OBJ_DESTROY when one or more flow filter rules
depend on that group. The driver SHOULD only destroy the group after
all the associated rules have been destroyed.

The driver SHOULD NOT remove a flow filter classifier using the command
VIRTIO_ADMIN_CMD_RESOURCE_OBJ_DESTROY when one or more flow filter rules
depend on the classifier. The driver SHOULD only destroy the classifier
after all the associated rules have been destroyed.

The driver SHOULD NOT add multiple flow filter rules with the same
\field{rule_priority} within a flow filter group, as these rules MAY match
the same packet. The driver SHOULD assign different \field{rule_priority}
values to different flow filter rules if multiple rules may match a single
packet.

For the command VIRTIO_ADMIN_CMD_RESOURCE_OBJ_CREATE, when creating a resource
of \field{type} VIRTIO_NET_RESOURCE_OBJ_FF_CLASSIFIER, the driver MUST set:
\begin{itemize}
\item \field{selectors[0].type} to VIRTIO_NET_FF_MASK_TYPE_ETH.
\item \field{selectors[1].type} to VIRTIO_NET_FF_MASK_TYPE_IPV4 or
      VIRTIO_NET_FF_MASK_TYPE_IPV6 when \field{count} is more than 1,
\item \field{selectors[2].type} VIRTIO_NET_FF_MASK_TYPE_UDP or
      VIRTIO_NET_FF_MASK_TYPE_TCP when \field{count} is more than 2.
\end{itemize}

For the command VIRTIO_ADMIN_CMD_RESOURCE_OBJ_CREATE, when creating a resource
of \field{type} VIRTIO_NET_RESOURCE_OBJ_FF_CLASSIFIER, the driver MUST set:
\begin{itemize}
\item \field{selectors[0].mask} bytes to all 1s for the \field{EtherType}
       when \field{count} is 2 or more.
\item \field{selectors[1].mask} bytes to all 1s for \field{Protocol} or \field{Next Header}
       when \field{selector[1].type} is VIRTIO_NET_FF_MASK_TYPE_IPV4 or VIRTIO_NET_FF_MASK_TYPE_IPV6,
       and when \field{count} is more than 2.
\end{itemize}

For the command VIRTIO_ADMIN_CMD_RESOURCE_OBJ_CREATE, the resource \field{type}
VIRTIO_NET_RESOURCE_OBJ_FF_RULE, if the corresponding classifier object's
\field{count} is 2 or more, the driver MUST SET the \field{keys} bytes of
\field{EtherType} in accordance with
\hyperref[intro:IEEE 802 Ethertypes]{IEEE 802 Ethertypes}
for either VIRTIO_NET_FF_MASK_TYPE_IPV4 or VIRTIO_NET_FF_MASK_TYPE_IPV6.

For the command VIRTIO_ADMIN_CMD_RESOURCE_OBJ_CREATE, when creating a resource of
\field{type} VIRTIO_NET_RESOURCE_OBJ_FF_RULE, if the corresponding classifier
object's \field{count} is more than 2, and the \field{selector[1].type} is either
VIRTIO_NET_FF_MASK_TYPE_IPV4 or VIRTIO_NET_FF_MASK_TYPE_IPV6, the driver MUST
set the \field{keys} bytes for the \field{Protocol} or \field{Next Header}
according to \hyperref[intro:IANA Protocol Numbers]{IANA Protocol Numbers} respectively.

The driver SHOULD set all the bits for a field in the mask of a selector in both the
capability and the classifier object, unless the VIRTIO_NET_FF_MASK_F_PARTIAL_MASK
is enabled.

\subsubsection{Legacy Interface: Framing Requirements}\label{sec:Device
Types / Network Device / Legacy Interface: Framing Requirements}

When using legacy interfaces, transitional drivers which have not
negotiated VIRTIO_F_ANY_LAYOUT MUST use a single descriptor for the
\field{struct virtio_net_hdr} on both transmit and receive, with the
network data in the following descriptors.

Additionally, when using the control virtqueue (see \ref{sec:Device
Types / Network Device / Device Operation / Control Virtqueue})
, transitional drivers which have not
negotiated VIRTIO_F_ANY_LAYOUT MUST:
\begin{itemize}
\item for all commands, use a single 2-byte descriptor including the first two
fields: \field{class} and \field{command}
\item for all commands except VIRTIO_NET_CTRL_MAC_TABLE_SET
use a single descriptor including command-specific-data
with no padding.
\item for the VIRTIO_NET_CTRL_MAC_TABLE_SET command use exactly
two descriptors including command-specific-data with no padding:
the first of these descriptors MUST include the
virtio_net_ctrl_mac table structure for the unicast addresses with no padding,
the second of these descriptors MUST include the
virtio_net_ctrl_mac table structure for the multicast addresses
with no padding.
\item for all commands, use a single 1-byte descriptor for the
\field{ack} field
\end{itemize}

See \ref{sec:Basic
Facilities of a Virtio Device / Virtqueues / Message Framing}.

\section{Network Device}\label{sec:Device Types / Network Device}

The virtio network device is a virtual network interface controller.
It consists of a virtual Ethernet link which connects the device
to the Ethernet network. The device has transmit and receive
queues. The driver adds empty buffers to the receive virtqueue.
The device receives incoming packets from the link; the device
places these incoming packets in the receive virtqueue buffers.
The driver adds outgoing packets to the transmit virtqueue. The device
removes these packets from the transmit virtqueue and sends them to
the link. The device may have a control virtqueue. The driver
uses the control virtqueue to dynamically manipulate various
features of the initialized device.

\subsection{Device ID}\label{sec:Device Types / Network Device / Device ID}

 1

\subsection{Virtqueues}\label{sec:Device Types / Network Device / Virtqueues}

\begin{description}
\item[0] receiveq1
\item[1] transmitq1
\item[\ldots]
\item[2(N-1)] receiveqN
\item[2(N-1)+1] transmitqN
\item[2N] controlq
\end{description}

 N=1 if neither VIRTIO_NET_F_MQ nor VIRTIO_NET_F_RSS are negotiated, otherwise N is set by
 \field{max_virtqueue_pairs}.

controlq is optional; it only exists if VIRTIO_NET_F_CTRL_VQ is
negotiated.

\subsection{Feature bits}\label{sec:Device Types / Network Device / Feature bits}

\begin{description}
\item[VIRTIO_NET_F_CSUM (0)] Device handles packets with partial checksum offload.

\item[VIRTIO_NET_F_GUEST_CSUM (1)] Driver handles packets with partial checksum.

\item[VIRTIO_NET_F_CTRL_GUEST_OFFLOADS (2)] Control channel offloads
        reconfiguration support.

\item[VIRTIO_NET_F_MTU(3)] Device maximum MTU reporting is supported. If
    offered by the device, device advises driver about the value of
    its maximum MTU. If negotiated, the driver uses \field{mtu} as
    the maximum MTU value.

\item[VIRTIO_NET_F_MAC (5)] Device has given MAC address.

\item[VIRTIO_NET_F_GUEST_TSO4 (7)] Driver can receive TSOv4.

\item[VIRTIO_NET_F_GUEST_TSO6 (8)] Driver can receive TSOv6.

\item[VIRTIO_NET_F_GUEST_ECN (9)] Driver can receive TSO with ECN.

\item[VIRTIO_NET_F_GUEST_UFO (10)] Driver can receive UFO.

\item[VIRTIO_NET_F_HOST_TSO4 (11)] Device can receive TSOv4.

\item[VIRTIO_NET_F_HOST_TSO6 (12)] Device can receive TSOv6.

\item[VIRTIO_NET_F_HOST_ECN (13)] Device can receive TSO with ECN.

\item[VIRTIO_NET_F_HOST_UFO (14)] Device can receive UFO.

\item[VIRTIO_NET_F_MRG_RXBUF (15)] Driver can merge receive buffers.

\item[VIRTIO_NET_F_STATUS (16)] Configuration status field is
    available.

\item[VIRTIO_NET_F_CTRL_VQ (17)] Control channel is available.

\item[VIRTIO_NET_F_CTRL_RX (18)] Control channel RX mode support.

\item[VIRTIO_NET_F_CTRL_VLAN (19)] Control channel VLAN filtering.

\item[VIRTIO_NET_F_CTRL_RX_EXTRA (20)]	Control channel RX extra mode support.

\item[VIRTIO_NET_F_GUEST_ANNOUNCE(21)] Driver can send gratuitous
    packets.

\item[VIRTIO_NET_F_MQ(22)] Device supports multiqueue with automatic
    receive steering.

\item[VIRTIO_NET_F_CTRL_MAC_ADDR(23)] Set MAC address through control
    channel.

\item[VIRTIO_NET_F_DEVICE_STATS(50)] Device can provide device-level statistics
    to the driver through the control virtqueue.

\item[VIRTIO_NET_F_HASH_TUNNEL(51)] Device supports inner header hash for encapsulated packets.

\item[VIRTIO_NET_F_VQ_NOTF_COAL(52)] Device supports virtqueue notification coalescing.

\item[VIRTIO_NET_F_NOTF_COAL(53)] Device supports notifications coalescing.

\item[VIRTIO_NET_F_GUEST_USO4 (54)] Driver can receive USOv4 packets.

\item[VIRTIO_NET_F_GUEST_USO6 (55)] Driver can receive USOv6 packets.

\item[VIRTIO_NET_F_HOST_USO (56)] Device can receive USO packets. Unlike UFO
 (fragmenting the packet) the USO splits large UDP packet
 to several segments when each of these smaller packets has UDP header.

\item[VIRTIO_NET_F_HASH_REPORT(57)] Device can report per-packet hash
    value and a type of calculated hash.

\item[VIRTIO_NET_F_GUEST_HDRLEN(59)] Driver can provide the exact \field{hdr_len}
    value. Device benefits from knowing the exact header length.

\item[VIRTIO_NET_F_RSS(60)] Device supports RSS (receive-side scaling)
    with Toeplitz hash calculation and configurable hash
    parameters for receive steering.

\item[VIRTIO_NET_F_RSC_EXT(61)] Device can process duplicated ACKs
    and report number of coalesced segments and duplicated ACKs.

\item[VIRTIO_NET_F_STANDBY(62)] Device may act as a standby for a primary
    device with the same MAC address.

\item[VIRTIO_NET_F_SPEED_DUPLEX(63)] Device reports speed and duplex.

\item[VIRTIO_NET_F_RSS_CONTEXT(64)] Device supports multiple RSS contexts.

\item[VIRTIO_NET_F_GUEST_UDP_TUNNEL_GSO (65)] Driver can receive GSO packets
  carried by a UDP tunnel.

\item[VIRTIO_NET_F_GUEST_UDP_TUNNEL_GSO_CSUM (66)] Driver handles packets
  carried by a UDP tunnel with partial csum for the outer header.

\item[VIRTIO_NET_F_HOST_UDP_TUNNEL_GSO (67)] Device can receive GSO packets
  carried by a UDP tunnel.

\item[VIRTIO_NET_F_HOST_UDP_TUNNEL_GSO_CSUM (68)] Device handles packets
  carried by a UDP tunnel with partial csum for the outer header.
\end{description}

\subsubsection{Feature bit requirements}\label{sec:Device Types / Network Device / Feature bits / Feature bit requirements}

Some networking feature bits require other networking feature bits
(see \ref{drivernormative:Basic Facilities of a Virtio Device / Feature Bits}):

\begin{description}
\item[VIRTIO_NET_F_GUEST_TSO4] Requires VIRTIO_NET_F_GUEST_CSUM.
\item[VIRTIO_NET_F_GUEST_TSO6] Requires VIRTIO_NET_F_GUEST_CSUM.
\item[VIRTIO_NET_F_GUEST_ECN] Requires VIRTIO_NET_F_GUEST_TSO4 or VIRTIO_NET_F_GUEST_TSO6.
\item[VIRTIO_NET_F_GUEST_UFO] Requires VIRTIO_NET_F_GUEST_CSUM.
\item[VIRTIO_NET_F_GUEST_USO4] Requires VIRTIO_NET_F_GUEST_CSUM.
\item[VIRTIO_NET_F_GUEST_USO6] Requires VIRTIO_NET_F_GUEST_CSUM.
\item[VIRTIO_NET_F_GUEST_UDP_TUNNEL_GSO] Requires VIRTIO_NET_F_GUEST_TSO4, VIRTIO_NET_F_GUEST_TSO6,
   VIRTIO_NET_F_GUEST_USO4 and VIRTIO_NET_F_GUEST_USO6.
\item[VIRTIO_NET_F_GUEST_UDP_TUNNEL_GSO_CSUM] Requires VIRTIO_NET_F_GUEST_UDP_TUNNEL_GSO

\item[VIRTIO_NET_F_HOST_TSO4] Requires VIRTIO_NET_F_CSUM.
\item[VIRTIO_NET_F_HOST_TSO6] Requires VIRTIO_NET_F_CSUM.
\item[VIRTIO_NET_F_HOST_ECN] Requires VIRTIO_NET_F_HOST_TSO4 or VIRTIO_NET_F_HOST_TSO6.
\item[VIRTIO_NET_F_HOST_UFO] Requires VIRTIO_NET_F_CSUM.
\item[VIRTIO_NET_F_HOST_USO] Requires VIRTIO_NET_F_CSUM.
\item[VIRTIO_NET_F_HOST_UDP_TUNNEL_GSO] Requires VIRTIO_NET_F_HOST_TSO4, VIRTIO_NET_F_HOST_TSO6
   and VIRTIO_NET_F_HOST_USO.
\item[VIRTIO_NET_F_HOST_UDP_TUNNEL_GSO_CSUM] Requires VIRTIO_NET_F_HOST_UDP_TUNNEL_GSO

\item[VIRTIO_NET_F_CTRL_RX] Requires VIRTIO_NET_F_CTRL_VQ.
\item[VIRTIO_NET_F_CTRL_VLAN] Requires VIRTIO_NET_F_CTRL_VQ.
\item[VIRTIO_NET_F_GUEST_ANNOUNCE] Requires VIRTIO_NET_F_CTRL_VQ.
\item[VIRTIO_NET_F_MQ] Requires VIRTIO_NET_F_CTRL_VQ.
\item[VIRTIO_NET_F_CTRL_MAC_ADDR] Requires VIRTIO_NET_F_CTRL_VQ.
\item[VIRTIO_NET_F_NOTF_COAL] Requires VIRTIO_NET_F_CTRL_VQ.
\item[VIRTIO_NET_F_RSC_EXT] Requires VIRTIO_NET_F_HOST_TSO4 or VIRTIO_NET_F_HOST_TSO6.
\item[VIRTIO_NET_F_RSS] Requires VIRTIO_NET_F_CTRL_VQ.
\item[VIRTIO_NET_F_VQ_NOTF_COAL] Requires VIRTIO_NET_F_CTRL_VQ.
\item[VIRTIO_NET_F_HASH_TUNNEL] Requires VIRTIO_NET_F_CTRL_VQ along with VIRTIO_NET_F_RSS or VIRTIO_NET_F_HASH_REPORT.
\item[VIRTIO_NET_F_RSS_CONTEXT] Requires VIRTIO_NET_F_CTRL_VQ and VIRTIO_NET_F_RSS.
\end{description}

\begin{note}
The dependency between UDP_TUNNEL_GSO_CSUM and UDP_TUNNEL_GSO is intentionally
in the opposite direction with respect to the plain GSO features and the plain
checksum offload because UDP tunnel checksum offload gives very little gain
for non GSO packets and is quite complex to implement in H/W.
\end{note}

\subsubsection{Legacy Interface: Feature bits}\label{sec:Device Types / Network Device / Feature bits / Legacy Interface: Feature bits}
\begin{description}
\item[VIRTIO_NET_F_GSO (6)] Device handles packets with any GSO type. This was supposed to indicate segmentation offload support, but
upon further investigation it became clear that multiple bits were needed.
\item[VIRTIO_NET_F_GUEST_RSC4 (41)] Device coalesces TCPIP v4 packets. This was implemented by hypervisor patch for certification
purposes and current Windows driver depends on it. It will not function if virtio-net device reports this feature.
\item[VIRTIO_NET_F_GUEST_RSC6 (42)] Device coalesces TCPIP v6 packets. Similar to VIRTIO_NET_F_GUEST_RSC4.
\end{description}

\subsection{Device configuration layout}\label{sec:Device Types / Network Device / Device configuration layout}
\label{sec:Device Types / Block Device / Feature bits / Device configuration layout}

The network device has the following device configuration layout.
All of the device configuration fields are read-only for the driver.

\begin{lstlisting}
struct virtio_net_config {
        u8 mac[6];
        le16 status;
        le16 max_virtqueue_pairs;
        le16 mtu;
        le32 speed;
        u8 duplex;
        u8 rss_max_key_size;
        le16 rss_max_indirection_table_length;
        le32 supported_hash_types;
        le32 supported_tunnel_types;
};
\end{lstlisting}

The \field{mac} address field always exists (although it is only
valid if VIRTIO_NET_F_MAC is set).

The \field{status} only exists if VIRTIO_NET_F_STATUS is set.
Two bits are currently defined for the status field: VIRTIO_NET_S_LINK_UP
and VIRTIO_NET_S_ANNOUNCE.

\begin{lstlisting}
#define VIRTIO_NET_S_LINK_UP     1
#define VIRTIO_NET_S_ANNOUNCE    2
\end{lstlisting}

The following field, \field{max_virtqueue_pairs} only exists if
VIRTIO_NET_F_MQ or VIRTIO_NET_F_RSS is set. This field specifies the maximum number
of each of transmit and receive virtqueues (receiveq1\ldots receiveqN
and transmitq1\ldots transmitqN respectively) that can be configured once at least one of these features
is negotiated.

The following field, \field{mtu} only exists if VIRTIO_NET_F_MTU
is set. This field specifies the maximum MTU for the driver to
use.

The following two fields, \field{speed} and \field{duplex}, only
exist if VIRTIO_NET_F_SPEED_DUPLEX is set.

\field{speed} contains the device speed, in units of 1 MBit per
second, 0 to 0x7fffffff, or 0xffffffff for unknown speed.

\field{duplex} has the values of 0x01 for full duplex, 0x00 for
half duplex and 0xff for unknown duplex state.

Both \field{speed} and \field{duplex} can change, thus the driver
is expected to re-read these values after receiving a
configuration change notification.

The following field, \field{rss_max_key_size} only exists if VIRTIO_NET_F_RSS or VIRTIO_NET_F_HASH_REPORT is set.
It specifies the maximum supported length of RSS key in bytes.

The following field, \field{rss_max_indirection_table_length} only exists if VIRTIO_NET_F_RSS is set.
It specifies the maximum number of 16-bit entries in RSS indirection table.

The next field, \field{supported_hash_types} only exists if the device supports hash calculation,
i.e. if VIRTIO_NET_F_RSS or VIRTIO_NET_F_HASH_REPORT is set.

Field \field{supported_hash_types} contains the bitmask of supported hash types.
See \ref{sec:Device Types / Network Device / Device Operation / Processing of Incoming Packets / Hash calculation for incoming packets / Supported/enabled hash types} for details of supported hash types.

Field \field{supported_tunnel_types} only exists if the device supports inner header hash, i.e. if VIRTIO_NET_F_HASH_TUNNEL is set.

Field \field{supported_tunnel_types} contains the bitmask of encapsulation types supported by the device for inner header hash.
Encapsulation types are defined in \ref{sec:Device Types / Network Device / Device Operation / Processing of Incoming Packets /
Hash calculation for incoming packets / Encapsulation types supported/enabled for inner header hash}.

\devicenormative{\subsubsection}{Device configuration layout}{Device Types / Network Device / Device configuration layout}

The device MUST set \field{max_virtqueue_pairs} to between 1 and 0x8000 inclusive,
if it offers VIRTIO_NET_F_MQ.

The device MUST set \field{mtu} to between 68 and 65535 inclusive,
if it offers VIRTIO_NET_F_MTU.

The device SHOULD set \field{mtu} to at least 1280, if it offers
VIRTIO_NET_F_MTU.

The device MUST NOT modify \field{mtu} once it has been set.

The device MUST NOT pass received packets that exceed \field{mtu} (plus low
level ethernet header length) size with \field{gso_type} NONE or ECN
after VIRTIO_NET_F_MTU has been successfully negotiated.

The device MUST forward transmitted packets of up to \field{mtu} (plus low
level ethernet header length) size with \field{gso_type} NONE or ECN, and do
so without fragmentation, after VIRTIO_NET_F_MTU has been successfully
negotiated.

The device MUST set \field{rss_max_key_size} to at least 40, if it offers
VIRTIO_NET_F_RSS or VIRTIO_NET_F_HASH_REPORT.

The device MUST set \field{rss_max_indirection_table_length} to at least 128, if it offers
VIRTIO_NET_F_RSS.

If the driver negotiates the VIRTIO_NET_F_STANDBY feature, the device MAY act
as a standby device for a primary device with the same MAC address.

If VIRTIO_NET_F_SPEED_DUPLEX has been negotiated, \field{speed}
MUST contain the device speed, in units of 1 MBit per second, 0 to
0x7ffffffff, or 0xfffffffff for unknown.

If VIRTIO_NET_F_SPEED_DUPLEX has been negotiated, \field{duplex}
MUST have the values of 0x00 for full duplex, 0x01 for half
duplex, or 0xff for unknown.

If VIRTIO_NET_F_SPEED_DUPLEX and VIRTIO_NET_F_STATUS have both
been negotiated, the device SHOULD NOT change the \field{speed} and
\field{duplex} fields as long as VIRTIO_NET_S_LINK_UP is set in
the \field{status}.

The device SHOULD NOT offer VIRTIO_NET_F_HASH_REPORT if it
does not offer VIRTIO_NET_F_CTRL_VQ.

The device SHOULD NOT offer VIRTIO_NET_F_CTRL_RX_EXTRA if it
does not offer VIRTIO_NET_F_CTRL_VQ.

\drivernormative{\subsubsection}{Device configuration layout}{Device Types / Network Device / Device configuration layout}

The driver MUST NOT write to any of the device configuration fields.

A driver SHOULD negotiate VIRTIO_NET_F_MAC if the device offers it.
If the driver negotiates the VIRTIO_NET_F_MAC feature, the driver MUST set
the physical address of the NIC to \field{mac}.  Otherwise, it SHOULD
use a locally-administered MAC address (see \hyperref[intro:IEEE 802]{IEEE 802},
``9.2 48-bit universal LAN MAC addresses'').

If the driver does not negotiate the VIRTIO_NET_F_STATUS feature, it SHOULD
assume the link is active, otherwise it SHOULD read the link status from
the bottom bit of \field{status}.

A driver SHOULD negotiate VIRTIO_NET_F_MTU if the device offers it.

If the driver negotiates VIRTIO_NET_F_MTU, it MUST supply enough receive
buffers to receive at least one receive packet of size \field{mtu} (plus low
level ethernet header length) with \field{gso_type} NONE or ECN.

If the driver negotiates VIRTIO_NET_F_MTU, it MUST NOT transmit packets of
size exceeding the value of \field{mtu} (plus low level ethernet header length)
with \field{gso_type} NONE or ECN.

A driver SHOULD negotiate the VIRTIO_NET_F_STANDBY feature if the device offers it.

If VIRTIO_NET_F_SPEED_DUPLEX has been negotiated,
the driver MUST treat any value of \field{speed} above
0x7fffffff as well as any value of \field{duplex} not
matching 0x00 or 0x01 as an unknown value.

If VIRTIO_NET_F_SPEED_DUPLEX has been negotiated, the driver
SHOULD re-read \field{speed} and \field{duplex} after a
configuration change notification.

A driver SHOULD NOT negotiate VIRTIO_NET_F_HASH_REPORT if it
does not negotiate VIRTIO_NET_F_CTRL_VQ.

A driver SHOULD NOT negotiate VIRTIO_NET_F_CTRL_RX_EXTRA if it
does not negotiate VIRTIO_NET_F_CTRL_VQ.

\subsubsection{Legacy Interface: Device configuration layout}\label{sec:Device Types / Network Device / Device configuration layout / Legacy Interface: Device configuration layout}
\label{sec:Device Types / Block Device / Feature bits / Device configuration layout / Legacy Interface: Device configuration layout}
When using the legacy interface, transitional devices and drivers
MUST format \field{status} and
\field{max_virtqueue_pairs} in struct virtio_net_config
according to the native endian of the guest rather than
(necessarily when not using the legacy interface) little-endian.

When using the legacy interface, \field{mac} is driver-writable
which provided a way for drivers to update the MAC without
negotiating VIRTIO_NET_F_CTRL_MAC_ADDR.

\subsection{Device Initialization}\label{sec:Device Types / Network Device / Device Initialization}

A driver would perform a typical initialization routine like so:

\begin{enumerate}
\item Identify and initialize the receive and
  transmission virtqueues, up to N of each kind. If
  VIRTIO_NET_F_MQ feature bit is negotiated,
  N=\field{max_virtqueue_pairs}, otherwise identify N=1.

\item If the VIRTIO_NET_F_CTRL_VQ feature bit is negotiated,
  identify the control virtqueue.

\item Fill the receive queues with buffers: see \ref{sec:Device Types / Network Device / Device Operation / Setting Up Receive Buffers}.

\item Even with VIRTIO_NET_F_MQ, only receiveq1, transmitq1 and
  controlq are used by default.  The driver would send the
  VIRTIO_NET_CTRL_MQ_VQ_PAIRS_SET command specifying the
  number of the transmit and receive queues to use.

\item If the VIRTIO_NET_F_MAC feature bit is set, the configuration
  space \field{mac} entry indicates the ``physical'' address of the
  device, otherwise the driver would typically generate a random
  local MAC address.

\item If the VIRTIO_NET_F_STATUS feature bit is negotiated, the link
  status comes from the bottom bit of \field{status}.
  Otherwise, the driver assumes it's active.

\item A performant driver would indicate that it will generate checksumless
  packets by negotiating the VIRTIO_NET_F_CSUM feature.

\item If that feature is negotiated, a driver can use TCP segmentation or UDP
  segmentation/fragmentation offload by negotiating the VIRTIO_NET_F_HOST_TSO4 (IPv4
  TCP), VIRTIO_NET_F_HOST_TSO6 (IPv6 TCP), VIRTIO_NET_F_HOST_UFO
  (UDP fragmentation) and VIRTIO_NET_F_HOST_USO (UDP segmentation) features.

\item If the VIRTIO_NET_F_HOST_TSO6, VIRTIO_NET_F_HOST_TSO4 and VIRTIO_NET_F_HOST_USO
  segmentation features are negotiated, a driver can
  use TCP segmentation or UDP segmentation on top of UDP encapsulation
  offload, when the outer header does not require checksumming - e.g.
  the outer UDP checksum is zero - by negotiating the
  VIRTIO_NET_F_HOST_UDP_TUNNEL_GSO feature.
  GSO over UDP tunnels packets carry two sets of headers: the outer ones
  and the inner ones. The outer transport protocol is UDP, the inner
  could be either TCP or UDP. Only a single level of encapsulation
  offload is supported.

\item If VIRTIO_NET_F_HOST_UDP_TUNNEL_GSO is negotiated, a driver can
  additionally use TCP segmentation or UDP segmentation on top of UDP
  encapsulation with the outer header requiring checksum offload,
  negotiating the VIRTIO_NET_F_HOST_UDP_TUNNEL_GSO_CSUM feature.

\item The converse features are also available: a driver can save
  the virtual device some work by negotiating these features.\note{For example, a network packet transported between two guests on
the same system might not need checksumming at all, nor segmentation,
if both guests are amenable.}
   The VIRTIO_NET_F_GUEST_CSUM feature indicates that partially
  checksummed packets can be received, and if it can do that then
  the VIRTIO_NET_F_GUEST_TSO4, VIRTIO_NET_F_GUEST_TSO6,
  VIRTIO_NET_F_GUEST_UFO, VIRTIO_NET_F_GUEST_ECN, VIRTIO_NET_F_GUEST_USO4,
  VIRTIO_NET_F_GUEST_USO6 VIRTIO_NET_F_GUEST_UDP_TUNNEL_GSO and
  VIRTIO_NET_F_GUEST_UDP_TUNNEL_GSO_CSUM are the input equivalents of
  the features described above.
  See \ref{sec:Device Types / Network Device / Device Operation /
Setting Up Receive Buffers}~\nameref{sec:Device Types / Network
Device / Device Operation / Setting Up Receive Buffers} and
\ref{sec:Device Types / Network Device / Device Operation /
Processing of Incoming Packets}~\nameref{sec:Device Types /
Network Device / Device Operation / Processing of Incoming Packets} below.
\end{enumerate}

A truly minimal driver would only accept VIRTIO_NET_F_MAC and ignore
everything else.

\subsection{Device and driver capabilities}\label{sec:Device Types / Network Device / Device and driver capabilities}

The network device has the following capabilities.

\begin{tabularx}{\textwidth}{ |l||l|X| }
\hline
Identifier & Name & Description \\
\hline \hline
0x0800 & \hyperref[par:Device Types / Network Device / Device Operation / Flow filter / Device and driver capabilities / VIRTIO-NET-FF-RESOURCE-CAP]{VIRTIO_NET_FF_RESOURCE_CAP} & Flow filter resource capability \\
\hline
0x0801 & \hyperref[par:Device Types / Network Device / Device Operation / Flow filter / Device and driver capabilities / VIRTIO-NET-FF-SELECTOR-CAP]{VIRTIO_NET_FF_SELECTOR_CAP} & Flow filter classifier capability \\
\hline
0x0802 & \hyperref[par:Device Types / Network Device / Device Operation / Flow filter / Device and driver capabilities / VIRTIO-NET-FF-ACTION-CAP]{VIRTIO_NET_FF_ACTION_CAP} & Flow filter action capability \\
\hline
\end{tabularx}

\subsection{Device resource objects}\label{sec:Device Types / Network Device / Device resource objects}

The network device has the following resource objects.

\begin{tabularx}{\textwidth}{ |l||l|X| }
\hline
type & Name & Description \\
\hline \hline
0x0200 & \hyperref[par:Device Types / Network Device / Device Operation / Flow filter / Resource objects / VIRTIO-NET-RESOURCE-OBJ-FF-GROUP]{VIRTIO_NET_RESOURCE_OBJ_FF_GROUP} & Flow filter group resource object \\
\hline
0x0201 & \hyperref[par:Device Types / Network Device / Device Operation / Flow filter / Resource objects / VIRTIO-NET-RESOURCE-OBJ-FF-CLASSIFIER]{VIRTIO_NET_RESOURCE_OBJ_FF_CLASSIFIER} & Flow filter mask object \\
\hline
0x0202 & \hyperref[par:Device Types / Network Device / Device Operation / Flow filter / Resource objects / VIRTIO-NET-RESOURCE-OBJ-FF-RULE]{VIRTIO_NET_RESOURCE_OBJ_FF_RULE} & Flow filter rule object \\
\hline
\end{tabularx}

\subsection{Device parts}\label{sec:Device Types / Network Device / Device parts}

Network device parts represent the configuration done by the driver using control
virtqueue commands. Network device part is in the format of
\field{struct virtio_dev_part}.

\begin{tabularx}{\textwidth}{ |l||l|X| }
\hline
Type & Name & Description \\
\hline \hline
0x200 & VIRTIO_NET_DEV_PART_CVQ_CFG_PART & Represents device configuration done through a control virtqueue command, see \ref{sec:Device Types / Network Device / Device parts / VIRTIO-NET-DEV-PART-CVQ-CFG-PART} \\
\hline
0x201 - 0x5FF & - & reserved for future \\
\hline
\hline
\end{tabularx}

\subsubsection{VIRTIO_NET_DEV_PART_CVQ_CFG_PART}\label{sec:Device Types / Network Device / Device parts / VIRTIO-NET-DEV-PART-CVQ-CFG-PART}

For VIRTIO_NET_DEV_PART_CVQ_CFG_PART, \field{part_type} is set to 0x200. The
VIRTIO_NET_DEV_PART_CVQ_CFG_PART part indicates configuration performed by the
driver using a control virtqueue command.

\begin{lstlisting}
struct virtio_net_dev_part_cvq_selector {
        u8 class;
        u8 command;
        u8 reserved[6];
};
\end{lstlisting}

There is one device part of type VIRTIO_NET_DEV_PART_CVQ_CFG_PART for each
individual configuration. Each part is identified by a unique selector value.
The selector, \field{device_type_raw}, is in the format
\field{struct virtio_net_dev_part_cvq_selector}.

The selector consists of two fields: \field{class} and \field{command}. These
fields correspond to the \field{class} and \field{command} defined in
\field{struct virtio_net_ctrl}, as described in the relevant sections of
\ref{sec:Device Types / Network Device / Device Operation / Control Virtqueue}.

The value corresponding to each part’s selector follows the same format as the
respective \field{command-specific-data} described in the relevant sections of
\ref{sec:Device Types / Network Device / Device Operation / Control Virtqueue}.

For example, when the \field{class} is VIRTIO_NET_CTRL_MAC, the \field{command}
can be either VIRTIO_NET_CTRL_MAC_TABLE_SET or VIRTIO_NET_CTRL_MAC_ADDR_SET;
when \field{command} is set to VIRTIO_NET_CTRL_MAC_TABLE_SET, \field{value}
is in the format of \field{struct virtio_net_ctrl_mac}.

Supported selectors are listed in the table:

\begin{tabularx}{\textwidth}{ |l|X| }
\hline
Class selector & Command selector \\
\hline \hline
VIRTIO_NET_CTRL_RX & VIRTIO_NET_CTRL_RX_PROMISC \\
\hline
VIRTIO_NET_CTRL_RX & VIRTIO_NET_CTRL_RX_ALLMULTI \\
\hline
VIRTIO_NET_CTRL_RX & VIRTIO_NET_CTRL_RX_ALLUNI \\
\hline
VIRTIO_NET_CTRL_RX & VIRTIO_NET_CTRL_RX_NOMULTI \\
\hline
VIRTIO_NET_CTRL_RX & VIRTIO_NET_CTRL_RX_NOUNI \\
\hline
VIRTIO_NET_CTRL_RX & VIRTIO_NET_CTRL_RX_NOBCAST \\
\hline
VIRTIO_NET_CTRL_MAC & VIRTIO_NET_CTRL_MAC_TABLE_SET \\
\hline
VIRTIO_NET_CTRL_MAC & VIRTIO_NET_CTRL_MAC_ADDR_SET \\
\hline
VIRTIO_NET_CTRL_VLAN & VIRTIO_NET_CTRL_VLAN_ADD \\
\hline
VIRTIO_NET_CTRL_ANNOUNCE & VIRTIO_NET_CTRL_ANNOUNCE_ACK \\
\hline
VIRTIO_NET_CTRL_MQ & VIRTIO_NET_CTRL_MQ_VQ_PAIRS_SET \\
\hline
VIRTIO_NET_CTRL_MQ & VIRTIO_NET_CTRL_MQ_RSS_CONFIG \\
\hline
VIRTIO_NET_CTRL_MQ & VIRTIO_NET_CTRL_MQ_HASH_CONFIG \\
\hline
\hline
\end{tabularx}

For command selector VIRTIO_NET_CTRL_VLAN_ADD, device part consists of a whole
VLAN table.

\field{reserved} is reserved and set to zero.

\subsection{Device Operation}\label{sec:Device Types / Network Device / Device Operation}

Packets are transmitted by placing them in the
transmitq1\ldots transmitqN, and buffers for incoming packets are
placed in the receiveq1\ldots receiveqN. In each case, the packet
itself is preceded by a header:

\begin{lstlisting}
struct virtio_net_hdr {
#define VIRTIO_NET_HDR_F_NEEDS_CSUM    1
#define VIRTIO_NET_HDR_F_DATA_VALID    2
#define VIRTIO_NET_HDR_F_RSC_INFO      4
#define VIRTIO_NET_HDR_F_UDP_TUNNEL_CSUM 8
        u8 flags;
#define VIRTIO_NET_HDR_GSO_NONE        0
#define VIRTIO_NET_HDR_GSO_TCPV4       1
#define VIRTIO_NET_HDR_GSO_UDP         3
#define VIRTIO_NET_HDR_GSO_TCPV6       4
#define VIRTIO_NET_HDR_GSO_UDP_L4      5
#define VIRTIO_NET_HDR_GSO_UDP_TUNNEL_IPV4 0x20
#define VIRTIO_NET_HDR_GSO_UDP_TUNNEL_IPV6 0x40
#define VIRTIO_NET_HDR_GSO_ECN      0x80
        u8 gso_type;
        le16 hdr_len;
        le16 gso_size;
        le16 csum_start;
        le16 csum_offset;
        le16 num_buffers;
        le32 hash_value;        (Only if VIRTIO_NET_F_HASH_REPORT negotiated)
        le16 hash_report;       (Only if VIRTIO_NET_F_HASH_REPORT negotiated)
        le16 padding_reserved;  (Only if VIRTIO_NET_F_HASH_REPORT negotiated)
        le16 outer_th_offset    (Only if VIRTIO_NET_F_HOST_UDP_TUNNEL_GSO or VIRTIO_NET_F_GUEST_UDP_TUNNEL_GSO negotiated)
        le16 inner_nh_offset;   (Only if VIRTIO_NET_F_HOST_UDP_TUNNEL_GSO or VIRTIO_NET_F_GUEST_UDP_TUNNEL_GSO negotiated)
};
\end{lstlisting}

The controlq is used to control various device features described further in
section \ref{sec:Device Types / Network Device / Device Operation / Control Virtqueue}.

\subsubsection{Legacy Interface: Device Operation}\label{sec:Device Types / Network Device / Device Operation / Legacy Interface: Device Operation}
When using the legacy interface, transitional devices and drivers
MUST format the fields in \field{struct virtio_net_hdr}
according to the native endian of the guest rather than
(necessarily when not using the legacy interface) little-endian.

The legacy driver only presented \field{num_buffers} in the \field{struct virtio_net_hdr}
when VIRTIO_NET_F_MRG_RXBUF was negotiated; without that feature the
structure was 2 bytes shorter.

When using the legacy interface, the driver SHOULD ignore the
used length for the transmit queues
and the controlq queue.
\begin{note}
Historically, some devices put
the total descriptor length there, even though no data was
actually written.
\end{note}

\subsubsection{Packet Transmission}\label{sec:Device Types / Network Device / Device Operation / Packet Transmission}

Transmitting a single packet is simple, but varies depending on
the different features the driver negotiated.

\begin{enumerate}
\item The driver can send a completely checksummed packet.  In this case,
  \field{flags} will be zero, and \field{gso_type} will be VIRTIO_NET_HDR_GSO_NONE.

\item If the driver negotiated VIRTIO_NET_F_CSUM, it can skip
  checksumming the packet:
  \begin{itemize}
  \item \field{flags} has the VIRTIO_NET_HDR_F_NEEDS_CSUM set,

  \item \field{csum_start} is set to the offset within the packet to begin checksumming,
    and

  \item \field{csum_offset} indicates how many bytes after the csum_start the
    new (16 bit ones' complement) checksum is placed by the device.

  \item The TCP checksum field in the packet is set to the sum
    of the TCP pseudo header, so that replacing it by the ones'
    complement checksum of the TCP header and body will give the
    correct result.
  \end{itemize}

\begin{note}
For example, consider a partially checksummed TCP (IPv4) packet.
It will have a 14 byte ethernet header and 20 byte IP header
followed by the TCP header (with the TCP checksum field 16 bytes
into that header). \field{csum_start} will be 14+20 = 34 (the TCP
checksum includes the header), and \field{csum_offset} will be 16.
If the given packet has the VIRTIO_NET_HDR_GSO_UDP_TUNNEL_IPV4 bit or the
VIRTIO_NET_HDR_GSO_UDP_TUNNEL_IPV6 bit set,
the above checksum fields refer to the inner header checksum, see
the example below.
\end{note}

\item If the driver negotiated
  VIRTIO_NET_F_HOST_TSO4, TSO6, USO or UFO, and the packet requires
  TCP segmentation, UDP segmentation or fragmentation, then \field{gso_type}
  is set to VIRTIO_NET_HDR_GSO_TCPV4, TCPV6, UDP_L4 or UDP.
  (Otherwise, it is set to VIRTIO_NET_HDR_GSO_NONE). In this
  case, packets larger than 1514 bytes can be transmitted: the
  metadata indicates how to replicate the packet header to cut it
  into smaller packets. The other gso fields are set:

  \begin{itemize}
  \item If the VIRTIO_NET_F_GUEST_HDRLEN feature has been negotiated,
    \field{hdr_len} indicates the header length that needs to be replicated
    for each packet. It's the number of bytes from the beginning of the packet
    to the beginning of the transport payload.
    If the \field{gso_type} has the VIRTIO_NET_HDR_GSO_UDP_TUNNEL_IPV4 bit or
    VIRTIO_NET_HDR_GSO_UDP_TUNNEL_IPV6 bit set, \field{hdr_len} accounts for
    all the headers up to and including the inner transport.
    Otherwise, if the VIRTIO_NET_F_GUEST_HDRLEN feature has not been negotiated,
    \field{hdr_len} is a hint to the device as to how much of the header
    needs to be kept to copy into each packet, usually set to the
    length of the headers, including the transport header\footnote{Due to various bugs in implementations, this field is not useful
as a guarantee of the transport header size.
}.

  \begin{note}
  Some devices benefit from knowledge of the exact header length.
  \end{note}

  \item \field{gso_size} is the maximum size of each packet beyond that
    header (ie. MSS).

  \item If the driver negotiated the VIRTIO_NET_F_HOST_ECN feature,
    the VIRTIO_NET_HDR_GSO_ECN bit in \field{gso_type}
    indicates that the TCP packet has the ECN bit set\footnote{This case is not handled by some older hardware, so is called out
specifically in the protocol.}.
   \end{itemize}

\item If the driver negotiated the VIRTIO_NET_F_HOST_UDP_TUNNEL_GSO feature and the
  \field{gso_type} has the VIRTIO_NET_HDR_GSO_UDP_TUNNEL_IPV4 bit or
  VIRTIO_NET_HDR_GSO_UDP_TUNNEL_IPV6 bit set, the GSO protocol is encapsulated
  in a UDP tunnel.
  If the outer UDP header requires checksumming, the driver must have
  additionally negotiated the VIRTIO_NET_F_HOST_UDP_TUNNEL_GSO_CSUM feature
  and offloaded the outer checksum accordingly, otherwise
  the outer UDP header must not require checksum validation, i.e. the outer
  UDP checksum must be positive zero (0x0) as defined in UDP RFC 768.
  The other tunnel-related fields indicate how to replicate the packet
  headers to cut it into smaller packets:

  \begin{itemize}
  \item \field{outer_th_offset} field indicates the outer transport header within
      the packet. This field differs from \field{csum_start} as the latter
      points to the inner transport header within the packet.

  \item \field{inner_nh_offset} field indicates the inner network header within
      the packet.
  \end{itemize}

\begin{note}
For example, consider a partially checksummed TCP (IPv4) packet carried over a
Geneve UDP tunnel (again IPv4) with no tunnel options. The
only relevant variable related to the tunnel type is the tunnel header length.
The packet will have a 14 byte outer ethernet header, 20 byte outer IP header
followed by the 8 byte UDP header (with a 0 checksum value), 8 byte Geneve header,
14 byte inner ethernet header, 20 byte inner IP header
and the TCP header (with the TCP checksum field 16 bytes
into that header). \field{csum_start} will be 14+20+8+8+14+20 = 84 (the TCP
checksum includes the header), \field{csum_offset} will be 16.
\field{inner_nh_offset} will be 14+20+8+8+14 = 62, \field{outer_th_offset} will be
14+20+8 = 42 and \field{gso_type} will be
VIRTIO_NET_HDR_GSO_TCPV4 | VIRTIO_NET_HDR_GSO_UDP_TUNNEL_IPV4 = 0x21
\end{note}

\item If the driver negotiated the VIRTIO_NET_F_HOST_UDP_TUNNEL_GSO_CSUM feature,
  the transmitted packet is a GSO one encapsulated in a UDP tunnel, and
  the outer UDP header requires checksumming, the driver can skip checksumming
  the outer header:

  \begin{itemize}
  \item \field{flags} has the VIRTIO_NET_HDR_F_UDP_TUNNEL_CSUM set,

  \item The outer UDP checksum field in the packet is set to the sum
    of the UDP pseudo header, so that replacing it by the ones'
    complement checksum of the outer UDP header and payload will give the
    correct result.
  \end{itemize}

\item \field{num_buffers} is set to zero.  This field is unused on transmitted packets.

\item The header and packet are added as one output descriptor to the
  transmitq, and the device is notified of the new entry
  (see \ref{sec:Device Types / Network Device / Device Initialization}~\nameref{sec:Device Types / Network Device / Device Initialization}).
\end{enumerate}

\drivernormative{\paragraph}{Packet Transmission}{Device Types / Network Device / Device Operation / Packet Transmission}

For the transmit packet buffer, the driver MUST use the size of the
structure \field{struct virtio_net_hdr} same as the receive packet buffer.

The driver MUST set \field{num_buffers} to zero.

If VIRTIO_NET_F_CSUM is not negotiated, the driver MUST set
\field{flags} to zero and SHOULD supply a fully checksummed
packet to the device.

If VIRTIO_NET_F_HOST_TSO4 is negotiated, the driver MAY set
\field{gso_type} to VIRTIO_NET_HDR_GSO_TCPV4 to request TCPv4
segmentation, otherwise the driver MUST NOT set
\field{gso_type} to VIRTIO_NET_HDR_GSO_TCPV4.

If VIRTIO_NET_F_HOST_TSO6 is negotiated, the driver MAY set
\field{gso_type} to VIRTIO_NET_HDR_GSO_TCPV6 to request TCPv6
segmentation, otherwise the driver MUST NOT set
\field{gso_type} to VIRTIO_NET_HDR_GSO_TCPV6.

If VIRTIO_NET_F_HOST_UFO is negotiated, the driver MAY set
\field{gso_type} to VIRTIO_NET_HDR_GSO_UDP to request UDP
fragmentation, otherwise the driver MUST NOT set
\field{gso_type} to VIRTIO_NET_HDR_GSO_UDP.

If VIRTIO_NET_F_HOST_USO is negotiated, the driver MAY set
\field{gso_type} to VIRTIO_NET_HDR_GSO_UDP_L4 to request UDP
segmentation, otherwise the driver MUST NOT set
\field{gso_type} to VIRTIO_NET_HDR_GSO_UDP_L4.

The driver SHOULD NOT send to the device TCP packets requiring segmentation offload
which have the Explicit Congestion Notification bit set, unless the
VIRTIO_NET_F_HOST_ECN feature is negotiated, in which case the
driver MUST set the VIRTIO_NET_HDR_GSO_ECN bit in
\field{gso_type}.

If VIRTIO_NET_F_HOST_UDP_TUNNEL_GSO is negotiated, the driver MAY set
VIRTIO_NET_HDR_GSO_UDP_TUNNEL_IPV4 bit or the VIRTIO_NET_HDR_GSO_UDP_TUNNEL_IPV6 bit
in \field{gso_type} according to the inner network header protocol type
to request GSO packets over UDPv4 or UDPv6 tunnel segmentation,
otherwise the driver MUST NOT set either the
VIRTIO_NET_HDR_GSO_UDP_TUNNEL_IPV4 bit or the VIRTIO_NET_HDR_GSO_UDP_TUNNEL_IPV6 bit
in \field{gso_type}.

When requesting GSO segmentation over UDP tunnel, the driver MUST SET the
VIRTIO_NET_HDR_GSO_UDP_TUNNEL_IPV4 bit if the inner network header is IPv4, i.e. the
packet is a TCPv4 GSO one, otherwise, if the inner network header is IPv6, the driver
MUST SET the VIRTIO_NET_HDR_GSO_UDP_TUNNEL_IPV6 bit.

The driver MUST NOT send to the device GSO packets over UDP tunnel
requiring segmentation and outer UDP checksum offload, unless both the
VIRTIO_NET_F_HOST_UDP_TUNNEL_GSO and VIRTIO_NET_F_HOST_UDP_TUNNEL_GSO_CSUM features
are negotiated, in which case the driver MUST set either the
VIRTIO_NET_HDR_GSO_UDP_TUNNEL_IPV4 bit or the VIRTIO_NET_HDR_GSO_UDP_TUNNEL_IPV6
bit in the \field{gso_type} and the VIRTIO_NET_HDR_F_UDP_TUNNEL_CSUM bit in
the \field{flags}.

If VIRTIO_NET_F_HOST_UDP_TUNNEL_GSO_CSUM is not negotiated, the driver MUST not set
the VIRTIO_NET_HDR_F_UDP_TUNNEL_CSUM bit in the \field{flags} and
MUST NOT send to the device GSO packets over UDP tunnel
requiring segmentation and outer UDP checksum offload.

The driver MUST NOT set the VIRTIO_NET_HDR_GSO_UDP_TUNNEL_IPV4 bit or the
VIRTIO_NET_HDR_GSO_UDP_TUNNEL_IPV6 bit together with VIRTIO_NET_HDR_GSO_UDP, as the
latter is deprecated in favor of UDP_L4 and no new feature will support it.

The driver MUST NOT set the VIRTIO_NET_HDR_GSO_UDP_TUNNEL_IPV4 bit and the
VIRTIO_NET_HDR_GSO_UDP_TUNNEL_IPV6 bit together.

The driver MUST NOT set the VIRTIO_NET_HDR_F_UDP_TUNNEL_CSUM bit \field{flags}
without setting either the VIRTIO_NET_HDR_GSO_UDP_TUNNEL_IPV4 bit or
the VIRTIO_NET_HDR_GSO_UDP_TUNNEL_IPV6 bit in \field{gso_type}.

If the VIRTIO_NET_F_CSUM feature has been negotiated, the
driver MAY set the VIRTIO_NET_HDR_F_NEEDS_CSUM bit in
\field{flags}, if so:
\begin{enumerate}
\item the driver MUST validate the packet checksum at
	offset \field{csum_offset} from \field{csum_start} as well as all
	preceding offsets;
\begin{note}
If \field{gso_type} differs from VIRTIO_NET_HDR_GSO_NONE and the
VIRTIO_NET_HDR_GSO_UDP_TUNNEL_IPV4 bit or the VIRTIO_NET_HDR_GSO_UDP_TUNNEL_IPV6
bit are not set in \field{gso_type}, \field{csum_offset}
points to the only transport header present in the packet, and there are no
additional preceding checksums validated by VIRTIO_NET_HDR_F_NEEDS_CSUM.
\end{note}
\item the driver MUST set the packet checksum stored in the
	buffer to the TCP/UDP pseudo header;
\item the driver MUST set \field{csum_start} and
	\field{csum_offset} such that calculating a ones'
	complement checksum from \field{csum_start} up until the end of
	the packet and storing the result at offset \field{csum_offset}
	from  \field{csum_start} will result in a fully checksummed
	packet;
\end{enumerate}

If none of the VIRTIO_NET_F_HOST_TSO4, TSO6, USO or UFO options have
been negotiated, the driver MUST set \field{gso_type} to
VIRTIO_NET_HDR_GSO_NONE.

If \field{gso_type} differs from VIRTIO_NET_HDR_GSO_NONE, then
the driver MUST also set the VIRTIO_NET_HDR_F_NEEDS_CSUM bit in
\field{flags} and MUST set \field{gso_size} to indicate the
desired MSS.

If one of the VIRTIO_NET_F_HOST_TSO4, TSO6, USO or UFO options have
been negotiated:
\begin{itemize}
\item If the VIRTIO_NET_F_GUEST_HDRLEN feature has been negotiated,
	and \field{gso_type} differs from VIRTIO_NET_HDR_GSO_NONE,
	the driver MUST set \field{hdr_len} to a value equal to the length
	of the headers, including the transport header. If \field{gso_type}
	has the VIRTIO_NET_HDR_GSO_UDP_TUNNEL_IPV4 bit or the
	VIRTIO_NET_HDR_GSO_UDP_TUNNEL_IPV6 bit set, \field{hdr_len} includes
	the inner transport header.

\item If the VIRTIO_NET_F_GUEST_HDRLEN feature has not been negotiated,
	or \field{gso_type} is VIRTIO_NET_HDR_GSO_NONE,
	the driver SHOULD set \field{hdr_len} to a value
	not less than the length of the headers, including the transport
	header.
\end{itemize}

If the VIRTIO_NET_F_HOST_UDP_TUNNEL_GSO option has been negotiated, the
driver MAY set the VIRTIO_NET_HDR_GSO_UDP_TUNNEL_IPV4 bit or the
VIRTIO_NET_HDR_GSO_UDP_TUNNEL_IPV6 bit in \field{gso_type}, if so:
\begin{itemize}
\item the driver MUST set \field{outer_th_offset} to the outer UDP header
  offset and \field{inner_nh_offset} to the inner network header offset.
  The \field{csum_start} and \field{csum_offset} fields point respectively
  to the inner transport header and inner transport checksum field.
\end{itemize}

If the VIRTIO_NET_F_HOST_UDP_TUNNEL_GSO_CSUM feature has been negotiated,
and the VIRTIO_NET_HDR_GSO_UDP_TUNNEL_IPV4 bit or
VIRTIO_NET_HDR_GSO_UDP_TUNNEL_IPV6 bit in \field{gso_type} are set,
the driver MAY set the VIRTIO_NET_HDR_F_UDP_TUNNEL_CSUM bit in
\field{flags}, if so the driver MUST set the packet outer UDP header checksum
to the outer UDP pseudo header checksum.

\begin{note}
calculating a ones' complement checksum from \field{outer_th_offset}
up until the end of the packet and storing the result at offset 6
from \field{outer_th_offset} will result in a fully checksummed outer UDP packet;
\end{note}

If the VIRTIO_NET_HDR_GSO_UDP_TUNNEL_IPV4 bit or the
VIRTIO_NET_HDR_GSO_UDP_TUNNEL_IPV6 bit in \field{gso_type} are set
and the VIRTIO_NET_F_HOST_UDP_TUNNEL_GSO_CSUM feature has not
been negotiated, the
outer UDP header MUST NOT require checksum validation. That is, the
outer UDP checksum value MUST be 0 or the validated complete checksum
for such header.

\begin{note}
The valid complete checksum of the outer UDP header of individual segments
can be computed by the driver prior to segmentation only if the GSO packet
size is a multiple of \field{gso_size}, because then all segments
have the same size and thus all data included in the outer UDP
checksum is the same for every segment. These pre-computed segment
length and checksum fields are different from those of the GSO
packet.
In this scenario the outer UDP header of the GSO packet must carry the
segmented UDP packet length.
\end{note}

If the VIRTIO_NET_F_HOST_UDP_TUNNEL_GSO option has not
been negotiated, the driver MUST NOT set either the VIRTIO_NET_HDR_F_GSO_UDP_TUNNEL_IPV4
bit or the VIRTIO_NET_HDR_F_GSO_UDP_TUNNEL_IPV6 in \field{gso_type}.

If the VIRTIO_NET_F_HOST_UDP_TUNNEL_GSO_CSUM option has not been negotiated,
the driver MUST NOT set the VIRTIO_NET_HDR_F_UDP_TUNNEL_CSUM bit
in \field{flags}.

The driver SHOULD accept the VIRTIO_NET_F_GUEST_HDRLEN feature if it has
been offered, and if it's able to provide the exact header length.

The driver MUST NOT set the VIRTIO_NET_HDR_F_DATA_VALID and
VIRTIO_NET_HDR_F_RSC_INFO bits in \field{flags}.

The driver MUST NOT set the VIRTIO_NET_HDR_F_DATA_VALID bit in \field{flags}
together with the VIRTIO_NET_HDR_F_GSO_UDP_TUNNEL_IPV4 bit or the
VIRTIO_NET_HDR_F_GSO_UDP_TUNNEL_IPV6 bit in \field{gso_type}.

\devicenormative{\paragraph}{Packet Transmission}{Device Types / Network Device / Device Operation / Packet Transmission}
The device MUST ignore \field{flag} bits that it does not recognize.

If VIRTIO_NET_HDR_F_NEEDS_CSUM bit in \field{flags} is not set, the
device MUST NOT use the \field{csum_start} and \field{csum_offset}.

If one of the VIRTIO_NET_F_HOST_TSO4, TSO6, USO or UFO options have
been negotiated:
\begin{itemize}
\item If the VIRTIO_NET_F_GUEST_HDRLEN feature has been negotiated,
	and \field{gso_type} differs from VIRTIO_NET_HDR_GSO_NONE,
	the device MAY use \field{hdr_len} as the transport header size.

	\begin{note}
	Caution should be taken by the implementation so as to prevent
	a malicious driver from attacking the device by setting an incorrect hdr_len.
	\end{note}

\item If the VIRTIO_NET_F_GUEST_HDRLEN feature has not been negotiated,
	or \field{gso_type} is VIRTIO_NET_HDR_GSO_NONE,
	the device MAY use \field{hdr_len} only as a hint about the
	transport header size.
	The device MUST NOT rely on \field{hdr_len} to be correct.

	\begin{note}
	This is due to various bugs in implementations.
	\end{note}
\end{itemize}

If both the VIRTIO_NET_HDR_GSO_UDP_TUNNEL_IPV4 bit and
the VIRTIO_NET_HDR_GSO_UDP_TUNNEL_IPV6 bit in in \field{gso_type} are set,
the device MUST NOT accept the packet.

If the VIRTIO_NET_HDR_GSO_UDP_TUNNEL_IPV4 bit and the VIRTIO_NET_HDR_GSO_UDP_TUNNEL_IPV6
bit in \field{gso_type} are not set, the device MUST NOT use the
\field{outer_th_offset} and \field{inner_nh_offset}.

If either the VIRTIO_NET_HDR_GSO_UDP_TUNNEL_IPV4 bit or
the VIRTIO_NET_HDR_GSO_UDP_TUNNEL_IPV6 bit in \field{gso_type} are set, and any of
the following is true:
\begin{itemize}
\item the VIRTIO_NET_HDR_F_NEEDS_CSUM is not set in \field{flags}
\item the VIRTIO_NET_HDR_F_DATA_VALID is set in \field{flags}
\item the \field{gso_type} excluding the VIRTIO_NET_HDR_GSO_UDP_TUNNEL_IPV4
bit and the VIRTIO_NET_HDR_GSO_UDP_TUNNEL_IPV6 bit is VIRTIO_NET_HDR_GSO_NONE
\end{itemize}
the device MUST NOT accept the packet.

If the VIRTIO_NET_HDR_F_UDP_TUNNEL_CSUM bit in \field{flags} is set,
and both the bits VIRTIO_NET_HDR_GSO_UDP_TUNNEL_IPV4 and
VIRTIO_NET_HDR_GSO_UDP_TUNNEL_IPV6 in \field{gso_type} are not set,
the device MOST NOT accept the packet.

If VIRTIO_NET_HDR_F_NEEDS_CSUM is not set, the device MUST NOT
rely on the packet checksum being correct.
\paragraph{Packet Transmission Interrupt}\label{sec:Device Types / Network Device / Device Operation / Packet Transmission / Packet Transmission Interrupt}

Often a driver will suppress transmission virtqueue interrupts
and check for used packets in the transmit path of following
packets.

The normal behavior in this interrupt handler is to retrieve
used buffers from the virtqueue and free the corresponding
headers and packets.

\subsubsection{Setting Up Receive Buffers}\label{sec:Device Types / Network Device / Device Operation / Setting Up Receive Buffers}

It is generally a good idea to keep the receive virtqueue as
fully populated as possible: if it runs out, network performance
will suffer.

If the VIRTIO_NET_F_GUEST_TSO4, VIRTIO_NET_F_GUEST_TSO6,
VIRTIO_NET_F_GUEST_UFO, VIRTIO_NET_F_GUEST_USO4 or VIRTIO_NET_F_GUEST_USO6
features are used, the maximum incoming packet
will be 65589 bytes long (14 bytes of Ethernet header, plus 40 bytes of
the IPv6 header, plus 65535 bytes of maximum IPv6 payload including any
extension header), otherwise 1514 bytes.
When VIRTIO_NET_F_HASH_REPORT is not negotiated, the required receive buffer
size is either 65601 or 1526 bytes accounting for 20 bytes of
\field{struct virtio_net_hdr} followed by receive packet.
When VIRTIO_NET_F_HASH_REPORT is negotiated, the required receive buffer
size is either 65609 or 1534 bytes accounting for 12 bytes of
\field{struct virtio_net_hdr} followed by receive packet.

\drivernormative{\paragraph}{Setting Up Receive Buffers}{Device Types / Network Device / Device Operation / Setting Up Receive Buffers}

\begin{itemize}
\item If VIRTIO_NET_F_MRG_RXBUF is not negotiated:
  \begin{itemize}
    \item If VIRTIO_NET_F_GUEST_TSO4, VIRTIO_NET_F_GUEST_TSO6, VIRTIO_NET_F_GUEST_UFO,
	VIRTIO_NET_F_GUEST_USO4 or VIRTIO_NET_F_GUEST_USO6 are negotiated, the driver SHOULD populate
      the receive queue(s) with buffers of at least 65609 bytes if
      VIRTIO_NET_F_HASH_REPORT is negotiated, and of at least 65601 bytes if not.
    \item Otherwise, the driver SHOULD populate the receive queue(s)
      with buffers of at least 1534 bytes if VIRTIO_NET_F_HASH_REPORT
      is negotiated, and of at least 1526 bytes if not.
  \end{itemize}
\item If VIRTIO_NET_F_MRG_RXBUF is negotiated, each buffer MUST be at
least size of \field{struct virtio_net_hdr},
i.e. 20 bytes if VIRTIO_NET_F_HASH_REPORT is negotiated, and 12 bytes if not.
\end{itemize}

\begin{note}
Obviously each buffer can be split across multiple descriptor elements.
\end{note}

When calculating the size of \field{struct virtio_net_hdr}, the driver
MUST consider all the fields inclusive up to \field{padding_reserved},
i.e. 20 bytes if VIRTIO_NET_F_HASH_REPORT is negotiated, and 12 bytes if not.

If VIRTIO_NET_F_MQ is negotiated, each of receiveq1\ldots receiveqN
that will be used SHOULD be populated with receive buffers.

\devicenormative{\paragraph}{Setting Up Receive Buffers}{Device Types / Network Device / Device Operation / Setting Up Receive Buffers}

The device MUST set \field{num_buffers} to the number of descriptors used to
hold the incoming packet.

The device MUST use only a single descriptor if VIRTIO_NET_F_MRG_RXBUF
was not negotiated.
\begin{note}
{This means that \field{num_buffers} will always be 1
if VIRTIO_NET_F_MRG_RXBUF is not negotiated.}
\end{note}

\subsubsection{Processing of Incoming Packets}\label{sec:Device Types / Network Device / Device Operation / Processing of Incoming Packets}
\label{sec:Device Types / Network Device / Device Operation / Processing of Packets}%old label for latexdiff

When a packet is copied into a buffer in the receiveq, the
optimal path is to disable further used buffer notifications for the
receiveq and process packets until no more are found, then re-enable
them.

Processing incoming packets involves:

\begin{enumerate}
\item \field{num_buffers} indicates how many descriptors
  this packet is spread over (including this one): this will
  always be 1 if VIRTIO_NET_F_MRG_RXBUF was not negotiated.
  This allows receipt of large packets without having to allocate large
  buffers: a packet that does not fit in a single buffer can flow
  over to the next buffer, and so on. In this case, there will be
  at least \field{num_buffers} used buffers in the virtqueue, and the device
  chains them together to form a single packet in a way similar to
  how it would store it in a single buffer spread over multiple
  descriptors.
  The other buffers will not begin with a \field{struct virtio_net_hdr}.

\item If
  \field{num_buffers} is one, then the entire packet will be
  contained within this buffer, immediately following the struct
  virtio_net_hdr.
\item If the VIRTIO_NET_F_GUEST_CSUM feature was negotiated, the
  VIRTIO_NET_HDR_F_DATA_VALID bit in \field{flags} can be
  set: if so, device has validated the packet checksum.
  If the VIRTIO_NET_F_GUEST_UDP_TUNNEL_GSO_CSUM feature has been negotiated,
  and the VIRTIO_NET_HDR_F_UDP_TUNNEL_CSUM bit is set in \field{flags},
  both the outer UDP checksum and the inner transport checksum
  have been validated, otherwise only one level of checksums (the outer one
  in case of tunnels) has been validated.
\end{enumerate}

Additionally, VIRTIO_NET_F_GUEST_CSUM, TSO4, TSO6, UDP, UDP_TUNNEL
and ECN features enable receive checksum, large receive offload and ECN
support which are the input equivalents of the transmit checksum,
transmit segmentation offloading and ECN features, as described
in \ref{sec:Device Types / Network Device / Device Operation /
Packet Transmission}:
\begin{enumerate}
\item If the VIRTIO_NET_F_GUEST_TSO4, TSO6, UFO, USO4 or USO6 options were
  negotiated, then \field{gso_type} MAY be something other than
  VIRTIO_NET_HDR_GSO_NONE, and \field{gso_size} field indicates the
  desired MSS (see Packet Transmission point 2).
\item If the VIRTIO_NET_F_RSC_EXT option was negotiated (this
  implies one of VIRTIO_NET_F_GUEST_TSO4, TSO6), the
  device processes also duplicated ACK segments, reports
  number of coalesced TCP segments in \field{csum_start} field and
  number of duplicated ACK segments in \field{csum_offset} field
  and sets bit VIRTIO_NET_HDR_F_RSC_INFO in \field{flags}.
\item If the VIRTIO_NET_F_GUEST_CSUM feature was negotiated, the
  VIRTIO_NET_HDR_F_NEEDS_CSUM bit in \field{flags} can be
  set: if so, the packet checksum at offset \field{csum_offset}
  from \field{csum_start} and any preceding checksums
  have been validated.  The checksum on the packet is incomplete and
  if bit VIRTIO_NET_HDR_F_RSC_INFO is not set in \field{flags},
  then \field{csum_start} and \field{csum_offset} indicate how to calculate it
  (see Packet Transmission point 1).
\begin{note}
If \field{gso_type} differs from VIRTIO_NET_HDR_GSO_NONE and the
VIRTIO_NET_HDR_GSO_UDP_TUNNEL_IPV4 bit or the VIRTIO_NET_HDR_GSO_UDP_TUNNEL_IPV6
bit are not set, \field{csum_offset}
points to the only transport header present in the packet, and there are no
additional preceding checksums validated by VIRTIO_NET_HDR_F_NEEDS_CSUM.
\end{note}
\item If the VIRTIO_NET_F_GUEST_UDP_TUNNEL_GSO option was negotiated and
  \field{gso_type} is not VIRTIO_NET_HDR_GSO_NONE, the
  VIRTIO_NET_HDR_GSO_UDP_TUNNEL_IPV4 bit or the VIRTIO_NET_HDR_GSO_UDP_TUNNEL_IPV6
  bit MAY be set. In such case the \field{outer_th_offset} and
  \field{inner_nh_offset} fields indicate the corresponding
  headers information.
\item If the VIRTIO_NET_F_GUEST_UDP_TUNNEL_GSO_CSUM feature was
negotiated, and
  the VIRTIO_NET_HDR_GSO_UDP_TUNNEL_IPV4 bit or the VIRTIO_NET_HDR_GSO_UDP_TUNNEL_IPV6
  are set in \field{gso_type}, the VIRTIO_NET_HDR_F_UDP_TUNNEL_CSUM bit in the
  \field{flags} can be set: if so, the outer UDP checksum has been validated
  and the UDP header checksum at offset 6 from from \field{outer_th_offset}
  is set to the outer UDP pseudo header checksum.

\begin{note}
If the VIRTIO_NET_HDR_GSO_UDP_TUNNEL_IPV4 bit or VIRTIO_NET_HDR_GSO_UDP_TUNNEL_IPV6
bit are set in \field{gso_type}, the \field{csum_start} field refers to
the inner transport header offset (see Packet Transmission point 1).
If the VIRTIO_NET_HDR_F_UDP_TUNNEL_CSUM bit in \field{flags} is set both
the inner and the outer header checksums have been validated by
VIRTIO_NET_HDR_F_NEEDS_CSUM, otherwise only the inner transport header
checksum has been validated.
\end{note}
\end{enumerate}

If applicable, the device calculates per-packet hash for incoming packets as
defined in \ref{sec:Device Types / Network Device / Device Operation / Processing of Incoming Packets / Hash calculation for incoming packets}.

If applicable, the device reports hash information for incoming packets as
defined in \ref{sec:Device Types / Network Device / Device Operation / Processing of Incoming Packets / Hash reporting for incoming packets}.

\devicenormative{\paragraph}{Processing of Incoming Packets}{Device Types / Network Device / Device Operation / Processing of Incoming Packets}
\label{devicenormative:Device Types / Network Device / Device Operation / Processing of Packets}%old label for latexdiff

If VIRTIO_NET_F_MRG_RXBUF has not been negotiated, the device MUST set
\field{num_buffers} to 1.

If VIRTIO_NET_F_MRG_RXBUF has been negotiated, the device MUST set
\field{num_buffers} to indicate the number of buffers
the packet (including the header) is spread over.

If a receive packet is spread over multiple buffers, the device
MUST use all buffers but the last (i.e. the first \field{num_buffers} -
1 buffers) completely up to the full length of each buffer
supplied by the driver.

The device MUST use all buffers used by a single receive
packet together, such that at least \field{num_buffers} are
observed by driver as used.

If VIRTIO_NET_F_GUEST_CSUM is not negotiated, the device MUST set
\field{flags} to zero and SHOULD supply a fully checksummed
packet to the driver.

If VIRTIO_NET_F_GUEST_TSO4 is not negotiated, the device MUST NOT set
\field{gso_type} to VIRTIO_NET_HDR_GSO_TCPV4.

If VIRTIO_NET_F_GUEST_UDP is not negotiated, the device MUST NOT set
\field{gso_type} to VIRTIO_NET_HDR_GSO_UDP.

If VIRTIO_NET_F_GUEST_TSO6 is not negotiated, the device MUST NOT set
\field{gso_type} to VIRTIO_NET_HDR_GSO_TCPV6.

If none of VIRTIO_NET_F_GUEST_USO4 or VIRTIO_NET_F_GUEST_USO6 have been negotiated,
the device MUST NOT set \field{gso_type} to VIRTIO_NET_HDR_GSO_UDP_L4.

If VIRTIO_NET_F_GUEST_UDP_TUNNEL_GSO is not negotiated, the device MUST NOT set
either the VIRTIO_NET_HDR_GSO_UDP_TUNNEL_IPV4 bit or the
VIRTIO_NET_HDR_GSO_UDP_TUNNEL_IPV6 bit in \field{gso_type}.

If VIRTIO_NET_F_GUEST_UDP_TUNNEL_GSO_CSUM is not negotiated the device MUST NOT set
the VIRTIO_NET_HDR_F_UDP_TUNNEL_CSUM bit in \field{flags}.

The device SHOULD NOT send to the driver TCP packets requiring segmentation offload
which have the Explicit Congestion Notification bit set, unless the
VIRTIO_NET_F_GUEST_ECN feature is negotiated, in which case the
device MUST set the VIRTIO_NET_HDR_GSO_ECN bit in
\field{gso_type}.

If the VIRTIO_NET_F_GUEST_CSUM feature has been negotiated, the
device MAY set the VIRTIO_NET_HDR_F_NEEDS_CSUM bit in
\field{flags}, if so:
\begin{enumerate}
\item the device MUST validate the packet checksum at
	offset \field{csum_offset} from \field{csum_start} as well as all
	preceding offsets;
\item the device MUST set the packet checksum stored in the
	receive buffer to the TCP/UDP pseudo header;
\item the device MUST set \field{csum_start} and
	\field{csum_offset} such that calculating a ones'
	complement checksum from \field{csum_start} up until the
	end of the packet and storing the result at offset
	\field{csum_offset} from  \field{csum_start} will result in a
	fully checksummed packet;
\end{enumerate}

The device MUST NOT send to the driver GSO packets encapsulated in UDP
tunnel and requiring segmentation offload, unless the
VIRTIO_NET_F_GUEST_UDP_TUNNEL_GSO is negotiated, in which case the device MUST set
the VIRTIO_NET_HDR_GSO_UDP_TUNNEL_IPV4 bit or the VIRTIO_NET_HDR_GSO_UDP_TUNNEL_IPV6
bit in \field{gso_type} according to the inner network header protocol type,
MUST set the \field{outer_th_offset} and \field{inner_nh_offset} fields
to the corresponding header information, and the outer UDP header MUST NOT
require checksum offload.

If the VIRTIO_NET_F_GUEST_UDP_TUNNEL_GSO_CSUM feature has not been negotiated,
the device MUST NOT send the driver GSO packets encapsulated in UDP
tunnel and requiring segmentation and outer checksum offload.

If none of the VIRTIO_NET_F_GUEST_TSO4, TSO6, UFO, USO4 or USO6 options have
been negotiated, the device MUST set \field{gso_type} to
VIRTIO_NET_HDR_GSO_NONE.

If \field{gso_type} differs from VIRTIO_NET_HDR_GSO_NONE, then
the device MUST also set the VIRTIO_NET_HDR_F_NEEDS_CSUM bit in
\field{flags} MUST set \field{gso_size} to indicate the desired MSS.
If VIRTIO_NET_F_RSC_EXT was negotiated, the device MUST also
set VIRTIO_NET_HDR_F_RSC_INFO bit in \field{flags},
set \field{csum_start} to number of coalesced TCP segments and
set \field{csum_offset} to number of received duplicated ACK segments.

If VIRTIO_NET_F_RSC_EXT was not negotiated, the device MUST
not set VIRTIO_NET_HDR_F_RSC_INFO bit in \field{flags}.

If one of the VIRTIO_NET_F_GUEST_TSO4, TSO6, UFO, USO4 or USO6 options have
been negotiated, the device SHOULD set \field{hdr_len} to a value
not less than the length of the headers, including the transport
header. If \field{gso_type} has the VIRTIO_NET_HDR_GSO_UDP_TUNNEL_IPV4 bit
or the VIRTIO_NET_HDR_GSO_UDP_TUNNEL_IPV6 bit set, the referenced transport
header is the inner one.

If the VIRTIO_NET_F_GUEST_CSUM feature has been negotiated, the
device MAY set the VIRTIO_NET_HDR_F_DATA_VALID bit in
\field{flags}, if so, the device MUST validate the packet
checksum. If the VIRTIO_NET_F_GUEST_UDP_TUNNEL_GSO_CSUM feature has
been negotiated, and the VIRTIO_NET_HDR_F_UDP_TUNNEL_CSUM bit set in
\field{flags}, both the outer UDP checksum and the inner transport
checksum have been validated.
Otherwise level of checksum is validated: in case of multiple
encapsulated protocols the outermost one.

If either the VIRTIO_NET_HDR_GSO_UDP_TUNNEL_IPV4 bit or the
VIRTIO_NET_HDR_GSO_UDP_TUNNEL_IPV6 bit in \field{gso_type} are set,
the device MUST NOT set the VIRTIO_NET_HDR_F_DATA_VALID bit in
\field{flags}.

If the VIRTIO_NET_F_GUEST_UDP_TUNNEL_GSO_CSUM feature has been negotiated
and either the VIRTIO_NET_HDR_GSO_UDP_TUNNEL_IPV4 bit is set or the
VIRTIO_NET_HDR_GSO_UDP_TUNNEL_IPV6 bit is set in \field{gso_type}, the
device MAY set the VIRTIO_NET_HDR_F_UDP_TUNNEL_CSUM bit in
\field{flags}, if so the device MUST set the packet outer UDP checksum
stored in the receive buffer to the outer UDP pseudo header.

Otherwise, the VIRTIO_NET_F_GUEST_UDP_TUNNEL_GSO_CSUM feature has been
negotiated, either the VIRTIO_NET_HDR_GSO_UDP_TUNNEL_IPV4 bit is set or the
VIRTIO_NET_HDR_GSO_UDP_TUNNEL_IPV6 bit is set in \field{gso_type},
and the bit VIRTIO_NET_HDR_F_UDP_TUNNEL_CSUM is not set in
\field{flags}, the device MUST either provide a zero outer UDP header
checksum or a fully checksummed outer UDP header.

\drivernormative{\paragraph}{Processing of Incoming
Packets}{Device Types / Network Device / Device Operation /
Processing of Incoming Packets}

The driver MUST ignore \field{flag} bits that it does not recognize.

If VIRTIO_NET_HDR_F_NEEDS_CSUM bit in \field{flags} is not set or
if VIRTIO_NET_HDR_F_RSC_INFO bit \field{flags} is set, the
driver MUST NOT use the \field{csum_start} and \field{csum_offset}.

If one of the VIRTIO_NET_F_GUEST_TSO4, TSO6, UFO, USO4 or USO6 options have
been negotiated, the driver MAY use \field{hdr_len} only as a hint about the
transport header size.
The driver MUST NOT rely on \field{hdr_len} to be correct.
\begin{note}
This is due to various bugs in implementations.
\end{note}

If neither VIRTIO_NET_HDR_F_NEEDS_CSUM nor
VIRTIO_NET_HDR_F_DATA_VALID is set, the driver MUST NOT
rely on the packet checksum being correct.

If both the VIRTIO_NET_HDR_GSO_UDP_TUNNEL_IPV4 bit and
the VIRTIO_NET_HDR_GSO_UDP_TUNNEL_IPV6 bit in in \field{gso_type} are set,
the driver MUST NOT accept the packet.

If the VIRTIO_NET_HDR_GSO_UDP_TUNNEL_IPV4 bit or the VIRTIO_NET_HDR_GSO_UDP_TUNNEL_IPV6
bit in \field{gso_type} are not set, the driver MUST NOT use the
\field{outer_th_offset} and \field{inner_nh_offset}.

If either the VIRTIO_NET_HDR_GSO_UDP_TUNNEL_IPV4 bit or
the VIRTIO_NET_HDR_GSO_UDP_TUNNEL_IPV6 bit in \field{gso_type} are set, and any of
the following is true:
\begin{itemize}
\item the VIRTIO_NET_HDR_F_NEEDS_CSUM bit is not set in \field{flags}
\item the VIRTIO_NET_HDR_F_DATA_VALID bit is set in \field{flags}
\item the \field{gso_type} excluding the VIRTIO_NET_HDR_GSO_UDP_TUNNEL_IPV4
bit and the VIRTIO_NET_HDR_GSO_UDP_TUNNEL_IPV6 bit is VIRTIO_NET_HDR_GSO_NONE
\end{itemize}
the driver MUST NOT accept the packet.

If the VIRTIO_NET_HDR_F_UDP_TUNNEL_CSUM bit and the VIRTIO_NET_HDR_F_NEEDS_CSUM
bit in \field{flags} are set,
and both the bits VIRTIO_NET_HDR_GSO_UDP_TUNNEL_IPV4 and
VIRTIO_NET_HDR_GSO_UDP_TUNNEL_IPV6 in \field{gso_type} are not set,
the driver MOST NOT accept the packet.

\paragraph{Hash calculation for incoming packets}
\label{sec:Device Types / Network Device / Device Operation / Processing of Incoming Packets / Hash calculation for incoming packets}

A device attempts to calculate a per-packet hash in the following cases:
\begin{itemize}
\item The feature VIRTIO_NET_F_RSS was negotiated. The device uses the hash to determine the receive virtqueue to place incoming packets.
\item The feature VIRTIO_NET_F_HASH_REPORT was negotiated. The device reports the hash value and the hash type with the packet.
\end{itemize}

If the feature VIRTIO_NET_F_RSS was negotiated:
\begin{itemize}
\item The device uses \field{hash_types} of the virtio_net_rss_config structure as 'Enabled hash types' bitmask.
\item If additionally the feature VIRTIO_NET_F_HASH_TUNNEL was negotiated, the device uses \field{enabled_tunnel_types} of the
      virtnet_hash_tunnel structure as 'Encapsulation types enabled for inner header hash' bitmask.
\item The device uses a key as defined in \field{hash_key_data} and \field{hash_key_length} of the virtio_net_rss_config structure (see
\ref{sec:Device Types / Network Device / Device Operation / Control Virtqueue / Receive-side scaling (RSS) / Setting RSS parameters}).
\end{itemize}

If the feature VIRTIO_NET_F_RSS was not negotiated:
\begin{itemize}
\item The device uses \field{hash_types} of the virtio_net_hash_config structure as 'Enabled hash types' bitmask.
\item If additionally the feature VIRTIO_NET_F_HASH_TUNNEL was negotiated, the device uses \field{enabled_tunnel_types} of the
      virtnet_hash_tunnel structure as 'Encapsulation types enabled for inner header hash' bitmask.
\item The device uses a key as defined in \field{hash_key_data} and \field{hash_key_length} of the virtio_net_hash_config structure (see
\ref{sec:Device Types / Network Device / Device Operation / Control Virtqueue / Automatic receive steering in multiqueue mode / Hash calculation}).
\end{itemize}

Note that if the device offers VIRTIO_NET_F_HASH_REPORT, even if it supports only one pair of virtqueues, it MUST support
at least one of commands of VIRTIO_NET_CTRL_MQ class to configure reported hash parameters:
\begin{itemize}
\item If the device offers VIRTIO_NET_F_RSS, it MUST support VIRTIO_NET_CTRL_MQ_RSS_CONFIG command per
 \ref{sec:Device Types / Network Device / Device Operation / Control Virtqueue / Receive-side scaling (RSS) / Setting RSS parameters}.
\item Otherwise the device MUST support VIRTIO_NET_CTRL_MQ_HASH_CONFIG command per
 \ref{sec:Device Types / Network Device / Device Operation / Control Virtqueue / Automatic receive steering in multiqueue mode / Hash calculation}.
\end{itemize}

The per-packet hash calculation can depend on the IP packet type. See
\hyperref[intro:IP]{[IP]}, \hyperref[intro:UDP]{[UDP]} and \hyperref[intro:TCP]{[TCP]}.

\subparagraph{Supported/enabled hash types}
\label{sec:Device Types / Network Device / Device Operation / Processing of Incoming Packets / Hash calculation for incoming packets / Supported/enabled hash types}
Hash types applicable for IPv4 packets:
\begin{lstlisting}
#define VIRTIO_NET_HASH_TYPE_IPv4              (1 << 0)
#define VIRTIO_NET_HASH_TYPE_TCPv4             (1 << 1)
#define VIRTIO_NET_HASH_TYPE_UDPv4             (1 << 2)
\end{lstlisting}
Hash types applicable for IPv6 packets without extension headers
\begin{lstlisting}
#define VIRTIO_NET_HASH_TYPE_IPv6              (1 << 3)
#define VIRTIO_NET_HASH_TYPE_TCPv6             (1 << 4)
#define VIRTIO_NET_HASH_TYPE_UDPv6             (1 << 5)
\end{lstlisting}
Hash types applicable for IPv6 packets with extension headers
\begin{lstlisting}
#define VIRTIO_NET_HASH_TYPE_IP_EX             (1 << 6)
#define VIRTIO_NET_HASH_TYPE_TCP_EX            (1 << 7)
#define VIRTIO_NET_HASH_TYPE_UDP_EX            (1 << 8)
\end{lstlisting}

\subparagraph{IPv4 packets}
\label{sec:Device Types / Network Device / Device Operation / Processing of Incoming Packets / Hash calculation for incoming packets / IPv4 packets}
The device calculates the hash on IPv4 packets according to 'Enabled hash types' bitmask as follows:
\begin{itemize}
\item If VIRTIO_NET_HASH_TYPE_TCPv4 is set and the packet has
a TCP header, the hash is calculated over the following fields:
\begin{itemize}
\item Source IP address
\item Destination IP address
\item Source TCP port
\item Destination TCP port
\end{itemize}
\item Else if VIRTIO_NET_HASH_TYPE_UDPv4 is set and the
packet has a UDP header, the hash is calculated over the following fields:
\begin{itemize}
\item Source IP address
\item Destination IP address
\item Source UDP port
\item Destination UDP port
\end{itemize}
\item Else if VIRTIO_NET_HASH_TYPE_IPv4 is set, the hash is
calculated over the following fields:
\begin{itemize}
\item Source IP address
\item Destination IP address
\end{itemize}
\item Else the device does not calculate the hash
\end{itemize}

\subparagraph{IPv6 packets without extension header}
\label{sec:Device Types / Network Device / Device Operation / Processing of Incoming Packets / Hash calculation for incoming packets / IPv6 packets without extension header}
The device calculates the hash on IPv6 packets without extension
headers according to 'Enabled hash types' bitmask as follows:
\begin{itemize}
\item If VIRTIO_NET_HASH_TYPE_TCPv6 is set and the packet has
a TCPv6 header, the hash is calculated over the following fields:
\begin{itemize}
\item Source IPv6 address
\item Destination IPv6 address
\item Source TCP port
\item Destination TCP port
\end{itemize}
\item Else if VIRTIO_NET_HASH_TYPE_UDPv6 is set and the
packet has a UDPv6 header, the hash is calculated over the following fields:
\begin{itemize}
\item Source IPv6 address
\item Destination IPv6 address
\item Source UDP port
\item Destination UDP port
\end{itemize}
\item Else if VIRTIO_NET_HASH_TYPE_IPv6 is set, the hash is
calculated over the following fields:
\begin{itemize}
\item Source IPv6 address
\item Destination IPv6 address
\end{itemize}
\item Else the device does not calculate the hash
\end{itemize}

\subparagraph{IPv6 packets with extension header}
\label{sec:Device Types / Network Device / Device Operation / Processing of Incoming Packets / Hash calculation for incoming packets / IPv6 packets with extension header}
The device calculates the hash on IPv6 packets with extension
headers according to 'Enabled hash types' bitmask as follows:
\begin{itemize}
\item If VIRTIO_NET_HASH_TYPE_TCP_EX is set and the packet
has a TCPv6 header, the hash is calculated over the following fields:
\begin{itemize}
\item Home address from the home address option in the IPv6 destination options header. If the extension header is not present, use the Source IPv6 address.
\item IPv6 address that is contained in the Routing-Header-Type-2 from the associated extension header. If the extension header is not present, use the Destination IPv6 address.
\item Source TCP port
\item Destination TCP port
\end{itemize}
\item Else if VIRTIO_NET_HASH_TYPE_UDP_EX is set and the
packet has a UDPv6 header, the hash is calculated over the following fields:
\begin{itemize}
\item Home address from the home address option in the IPv6 destination options header. If the extension header is not present, use the Source IPv6 address.
\item IPv6 address that is contained in the Routing-Header-Type-2 from the associated extension header. If the extension header is not present, use the Destination IPv6 address.
\item Source UDP port
\item Destination UDP port
\end{itemize}
\item Else if VIRTIO_NET_HASH_TYPE_IP_EX is set, the hash is
calculated over the following fields:
\begin{itemize}
\item Home address from the home address option in the IPv6 destination options header. If the extension header is not present, use the Source IPv6 address.
\item IPv6 address that is contained in the Routing-Header-Type-2 from the associated extension header. If the extension header is not present, use the Destination IPv6 address.
\end{itemize}
\item Else skip IPv6 extension headers and calculate the hash as
defined for an IPv6 packet without extension headers
(see \ref{sec:Device Types / Network Device / Device Operation / Processing of Incoming Packets / Hash calculation for incoming packets / IPv6 packets without extension header}).
\end{itemize}

\paragraph{Inner Header Hash}
\label{sec:Device Types / Network Device / Device Operation / Processing of Incoming Packets / Inner Header Hash}

If VIRTIO_NET_F_HASH_TUNNEL has been negotiated, the driver can send the command
VIRTIO_NET_CTRL_HASH_TUNNEL_SET to configure the calculation of the inner header hash.

\begin{lstlisting}
struct virtnet_hash_tunnel {
    le32 enabled_tunnel_types;
};

#define VIRTIO_NET_CTRL_HASH_TUNNEL 7
 #define VIRTIO_NET_CTRL_HASH_TUNNEL_SET 0
\end{lstlisting}

Field \field{enabled_tunnel_types} contains the bitmask of encapsulation types enabled for inner header hash.
See \ref{sec:Device Types / Network Device / Device Operation / Processing of Incoming Packets /
Hash calculation for incoming packets / Encapsulation types supported/enabled for inner header hash}.

The class VIRTIO_NET_CTRL_HASH_TUNNEL has one command:
VIRTIO_NET_CTRL_HASH_TUNNEL_SET sets \field{enabled_tunnel_types} for the device using the
virtnet_hash_tunnel structure, which is read-only for the device.

Inner header hash is disabled by VIRTIO_NET_CTRL_HASH_TUNNEL_SET with \field{enabled_tunnel_types} set to 0.

Initially (before the driver sends any VIRTIO_NET_CTRL_HASH_TUNNEL_SET command) all
encapsulation types are disabled for inner header hash.

\subparagraph{Encapsulated packet}
\label{sec:Device Types / Network Device / Device Operation / Processing of Incoming Packets / Hash calculation for incoming packets / Encapsulated packet}

Multiple tunneling protocols allow encapsulating an inner, payload packet in an outer, encapsulated packet.
The encapsulated packet thus contains an outer header and an inner header, and the device calculates the
hash over either the inner header or the outer header.

If VIRTIO_NET_F_HASH_TUNNEL is negotiated and a received encapsulated packet's outer header matches one of the
encapsulation types enabled in \field{enabled_tunnel_types}, then the device uses the inner header for hash
calculations (only a single level of encapsulation is currently supported).

If VIRTIO_NET_F_HASH_TUNNEL is negotiated and a received packet's (outer) header does not match any encapsulation
types enabled in \field{enabled_tunnel_types}, then the device uses the outer header for hash calculations.

\subparagraph{Encapsulation types supported/enabled for inner header hash}
\label{sec:Device Types / Network Device / Device Operation / Processing of Incoming Packets /
Hash calculation for incoming packets / Encapsulation types supported/enabled for inner header hash}

Encapsulation types applicable for inner header hash:
\begin{lstlisting}[escapechar=|]
#define VIRTIO_NET_HASH_TUNNEL_TYPE_GRE_2784    (1 << 0) /* |\hyperref[intro:rfc2784]{[RFC2784]}| */
#define VIRTIO_NET_HASH_TUNNEL_TYPE_GRE_2890    (1 << 1) /* |\hyperref[intro:rfc2890]{[RFC2890]}| */
#define VIRTIO_NET_HASH_TUNNEL_TYPE_GRE_7676    (1 << 2) /* |\hyperref[intro:rfc7676]{[RFC7676]}| */
#define VIRTIO_NET_HASH_TUNNEL_TYPE_GRE_UDP     (1 << 3) /* |\hyperref[intro:rfc8086]{[GRE-in-UDP]}| */
#define VIRTIO_NET_HASH_TUNNEL_TYPE_VXLAN       (1 << 4) /* |\hyperref[intro:vxlan]{[VXLAN]}| */
#define VIRTIO_NET_HASH_TUNNEL_TYPE_VXLAN_GPE   (1 << 5) /* |\hyperref[intro:vxlan-gpe]{[VXLAN-GPE]}| */
#define VIRTIO_NET_HASH_TUNNEL_TYPE_GENEVE      (1 << 6) /* |\hyperref[intro:geneve]{[GENEVE]}| */
#define VIRTIO_NET_HASH_TUNNEL_TYPE_IPIP        (1 << 7) /* |\hyperref[intro:ipip]{[IPIP]}| */
#define VIRTIO_NET_HASH_TUNNEL_TYPE_NVGRE       (1 << 8) /* |\hyperref[intro:nvgre]{[NVGRE]}| */
\end{lstlisting}

\subparagraph{Advice}
Example uses of the inner header hash:
\begin{itemize}
\item Legacy tunneling protocols, lacking the outer header entropy, can use RSS with the inner header hash to
      distribute flows with identical outer but different inner headers across various queues, improving performance.
\item Identify an inner flow distributed across multiple outer tunnels.
\end{itemize}

As using the inner header hash completely discards the outer header entropy, care must be taken
if the inner header is controlled by an adversary, as the adversary can then intentionally create
configurations with insufficient entropy.

Besides disabling the inner header hash, mitigations would depend on how the hash is used. When the hash
use is limited to the RSS queue selection, the inner header hash may have quality of service (QoS) limitations.

\devicenormative{\subparagraph}{Inner Header Hash}{Device Types / Network Device / Device Operation / Control Virtqueue / Inner Header Hash}

If the (outer) header of the received packet does not match any encapsulation types enabled
in \field{enabled_tunnel_types}, the device MUST calculate the hash on the outer header.

If the device receives any bits in \field{enabled_tunnel_types} which are not set in \field{supported_tunnel_types},
it SHOULD respond to the VIRTIO_NET_CTRL_HASH_TUNNEL_SET command with VIRTIO_NET_ERR.

If the driver sets \field{enabled_tunnel_types} to 0 through VIRTIO_NET_CTRL_HASH_TUNNEL_SET or upon the device reset,
the device MUST disable the inner header hash for all encapsulation types.

\drivernormative{\subparagraph}{Inner Header Hash}{Device Types / Network Device / Device Operation / Control Virtqueue / Inner Header Hash}

The driver MUST have negotiated the VIRTIO_NET_F_HASH_TUNNEL feature when issuing the VIRTIO_NET_CTRL_HASH_TUNNEL_SET command.

The driver MUST NOT set any bits in \field{enabled_tunnel_types} which are not set in \field{supported_tunnel_types}.

The driver MUST ignore bits in \field{supported_tunnel_types} which are not documented in this specification.

\paragraph{Hash reporting for incoming packets}
\label{sec:Device Types / Network Device / Device Operation / Processing of Incoming Packets / Hash reporting for incoming packets}

If VIRTIO_NET_F_HASH_REPORT was negotiated and
 the device has calculated the hash for the packet, the device fills \field{hash_report} with the report type of calculated hash
and \field{hash_value} with the value of calculated hash.

If VIRTIO_NET_F_HASH_REPORT was negotiated but due to any reason the
hash was not calculated, the device sets \field{hash_report} to VIRTIO_NET_HASH_REPORT_NONE.

Possible values that the device can report in \field{hash_report} are defined below.
They correspond to supported hash types defined in
\ref{sec:Device Types / Network Device / Device Operation / Processing of Incoming Packets / Hash calculation for incoming packets / Supported/enabled hash types}
as follows:

VIRTIO_NET_HASH_TYPE_XXX = 1 << (VIRTIO_NET_HASH_REPORT_XXX - 1)

\begin{lstlisting}
#define VIRTIO_NET_HASH_REPORT_NONE            0
#define VIRTIO_NET_HASH_REPORT_IPv4            1
#define VIRTIO_NET_HASH_REPORT_TCPv4           2
#define VIRTIO_NET_HASH_REPORT_UDPv4           3
#define VIRTIO_NET_HASH_REPORT_IPv6            4
#define VIRTIO_NET_HASH_REPORT_TCPv6           5
#define VIRTIO_NET_HASH_REPORT_UDPv6           6
#define VIRTIO_NET_HASH_REPORT_IPv6_EX         7
#define VIRTIO_NET_HASH_REPORT_TCPv6_EX        8
#define VIRTIO_NET_HASH_REPORT_UDPv6_EX        9
\end{lstlisting}

\subsubsection{Control Virtqueue}\label{sec:Device Types / Network Device / Device Operation / Control Virtqueue}

The driver uses the control virtqueue (if VIRTIO_NET_F_CTRL_VQ is
negotiated) to send commands to manipulate various features of
the device which would not easily map into the configuration
space.

All commands are of the following form:

\begin{lstlisting}
struct virtio_net_ctrl {
        u8 class;
        u8 command;
        u8 command-specific-data[];
        u8 ack;
        u8 command-specific-result[];
};

/* ack values */
#define VIRTIO_NET_OK     0
#define VIRTIO_NET_ERR    1
\end{lstlisting}

The \field{class}, \field{command} and command-specific-data are set by the
driver, and the device sets the \field{ack} byte and optionally
\field{command-specific-result}. There is little the driver can
do except issue a diagnostic if \field{ack} is not VIRTIO_NET_OK.

The command VIRTIO_NET_CTRL_STATS_QUERY and VIRTIO_NET_CTRL_STATS_GET contain
\field{command-specific-result}.

\paragraph{Packet Receive Filtering}\label{sec:Device Types / Network Device / Device Operation / Control Virtqueue / Packet Receive Filtering}
\label{sec:Device Types / Network Device / Device Operation / Control Virtqueue / Setting Promiscuous Mode}%old label for latexdiff

If the VIRTIO_NET_F_CTRL_RX and VIRTIO_NET_F_CTRL_RX_EXTRA
features are negotiated, the driver can send control commands for
promiscuous mode, multicast, unicast and broadcast receiving.

\begin{note}
In general, these commands are best-effort: unwanted
packets could still arrive.
\end{note}

\begin{lstlisting}
#define VIRTIO_NET_CTRL_RX    0
 #define VIRTIO_NET_CTRL_RX_PROMISC      0
 #define VIRTIO_NET_CTRL_RX_ALLMULTI     1
 #define VIRTIO_NET_CTRL_RX_ALLUNI       2
 #define VIRTIO_NET_CTRL_RX_NOMULTI      3
 #define VIRTIO_NET_CTRL_RX_NOUNI        4
 #define VIRTIO_NET_CTRL_RX_NOBCAST      5
\end{lstlisting}


\devicenormative{\subparagraph}{Packet Receive Filtering}{Device Types / Network Device / Device Operation / Control Virtqueue / Packet Receive Filtering}

If the VIRTIO_NET_F_CTRL_RX feature has been negotiated,
the device MUST support the following VIRTIO_NET_CTRL_RX class
commands:
\begin{itemize}
\item VIRTIO_NET_CTRL_RX_PROMISC turns promiscuous mode on and
off. The command-specific-data is one byte containing 0 (off) or
1 (on). If promiscuous mode is on, the device SHOULD receive all
incoming packets.
This SHOULD take effect even if one of the other modes set by
a VIRTIO_NET_CTRL_RX class command is on.
\item VIRTIO_NET_CTRL_RX_ALLMULTI turns all-multicast receive on and
off. The command-specific-data is one byte containing 0 (off) or
1 (on). When all-multicast receive is on the device SHOULD allow
all incoming multicast packets.
\end{itemize}

If the VIRTIO_NET_F_CTRL_RX_EXTRA feature has been negotiated,
the device MUST support the following VIRTIO_NET_CTRL_RX class
commands:
\begin{itemize}
\item VIRTIO_NET_CTRL_RX_ALLUNI turns all-unicast receive on and
off. The command-specific-data is one byte containing 0 (off) or
1 (on). When all-unicast receive is on the device SHOULD allow
all incoming unicast packets.
\item VIRTIO_NET_CTRL_RX_NOMULTI suppresses multicast receive.
The command-specific-data is one byte containing 0 (multicast
receive allowed) or 1 (multicast receive suppressed).
When multicast receive is suppressed, the device SHOULD NOT
send multicast packets to the driver.
This SHOULD take effect even if VIRTIO_NET_CTRL_RX_ALLMULTI is on.
This filter SHOULD NOT apply to broadcast packets.
\item VIRTIO_NET_CTRL_RX_NOUNI suppresses unicast receive.
The command-specific-data is one byte containing 0 (unicast
receive allowed) or 1 (unicast receive suppressed).
When unicast receive is suppressed, the device SHOULD NOT
send unicast packets to the driver.
This SHOULD take effect even if VIRTIO_NET_CTRL_RX_ALLUNI is on.
\item VIRTIO_NET_CTRL_RX_NOBCAST suppresses broadcast receive.
The command-specific-data is one byte containing 0 (broadcast
receive allowed) or 1 (broadcast receive suppressed).
When broadcast receive is suppressed, the device SHOULD NOT
send broadcast packets to the driver.
This SHOULD take effect even if VIRTIO_NET_CTRL_RX_ALLMULTI is on.
\end{itemize}

\drivernormative{\subparagraph}{Packet Receive Filtering}{Device Types / Network Device / Device Operation / Control Virtqueue / Packet Receive Filtering}

If the VIRTIO_NET_F_CTRL_RX feature has not been negotiated,
the driver MUST NOT issue commands VIRTIO_NET_CTRL_RX_PROMISC or
VIRTIO_NET_CTRL_RX_ALLMULTI.

If the VIRTIO_NET_F_CTRL_RX_EXTRA feature has not been negotiated,
the driver MUST NOT issue commands
 VIRTIO_NET_CTRL_RX_ALLUNI,
 VIRTIO_NET_CTRL_RX_NOMULTI,
 VIRTIO_NET_CTRL_RX_NOUNI or
 VIRTIO_NET_CTRL_RX_NOBCAST.

\paragraph{Setting MAC Address Filtering}\label{sec:Device Types / Network Device / Device Operation / Control Virtqueue / Setting MAC Address Filtering}

If the VIRTIO_NET_F_CTRL_RX feature is negotiated, the driver can
send control commands for MAC address filtering.

\begin{lstlisting}
struct virtio_net_ctrl_mac {
        le32 entries;
        u8 macs[entries][6];
};

#define VIRTIO_NET_CTRL_MAC    1
 #define VIRTIO_NET_CTRL_MAC_TABLE_SET        0
 #define VIRTIO_NET_CTRL_MAC_ADDR_SET         1
\end{lstlisting}

The device can filter incoming packets by any number of destination
MAC addresses\footnote{Since there are no guarantees, it can use a hash filter or
silently switch to allmulti or promiscuous mode if it is given too
many addresses.
}. This table is set using the class
VIRTIO_NET_CTRL_MAC and the command VIRTIO_NET_CTRL_MAC_TABLE_SET. The
command-specific-data is two variable length tables of 6-byte MAC
addresses (as described in struct virtio_net_ctrl_mac). The first table contains unicast addresses, and the second
contains multicast addresses.

The VIRTIO_NET_CTRL_MAC_ADDR_SET command is used to set the
default MAC address which rx filtering
accepts (and if VIRTIO_NET_F_MAC has been negotiated,
this will be reflected in \field{mac} in config space).

The command-specific-data for VIRTIO_NET_CTRL_MAC_ADDR_SET is
the 6-byte MAC address.

\devicenormative{\subparagraph}{Setting MAC Address Filtering}{Device Types / Network Device / Device Operation / Control Virtqueue / Setting MAC Address Filtering}

The device MUST have an empty MAC filtering table on reset.

The device MUST update the MAC filtering table before it consumes
the VIRTIO_NET_CTRL_MAC_TABLE_SET command.

The device MUST update \field{mac} in config space before it consumes
the VIRTIO_NET_CTRL_MAC_ADDR_SET command, if VIRTIO_NET_F_MAC has
been negotiated.

The device SHOULD drop incoming packets which have a destination MAC which
matches neither the \field{mac} (or that set with VIRTIO_NET_CTRL_MAC_ADDR_SET)
nor the MAC filtering table.

\drivernormative{\subparagraph}{Setting MAC Address Filtering}{Device Types / Network Device / Device Operation / Control Virtqueue / Setting MAC Address Filtering}

If VIRTIO_NET_F_CTRL_RX has not been negotiated,
the driver MUST NOT issue VIRTIO_NET_CTRL_MAC class commands.

If VIRTIO_NET_F_CTRL_RX has been negotiated,
the driver SHOULD issue VIRTIO_NET_CTRL_MAC_ADDR_SET
to set the default mac if it is different from \field{mac}.

The driver MUST follow the VIRTIO_NET_CTRL_MAC_TABLE_SET command
by a le32 number, followed by that number of non-multicast
MAC addresses, followed by another le32 number, followed by
that number of multicast addresses.  Either number MAY be 0.

\subparagraph{Legacy Interface: Setting MAC Address Filtering}\label{sec:Device Types / Network Device / Device Operation / Control Virtqueue / Setting MAC Address Filtering / Legacy Interface: Setting MAC Address Filtering}
When using the legacy interface, transitional devices and drivers
MUST format \field{entries} in struct virtio_net_ctrl_mac
according to the native endian of the guest rather than
(necessarily when not using the legacy interface) little-endian.

Legacy drivers that didn't negotiate VIRTIO_NET_F_CTRL_MAC_ADDR
changed \field{mac} in config space when NIC is accepting
incoming packets. These drivers always wrote the mac value from
first to last byte, therefore after detecting such drivers,
a transitional device MAY defer MAC update, or MAY defer
processing incoming packets until driver writes the last byte
of \field{mac} in the config space.

\paragraph{VLAN Filtering}\label{sec:Device Types / Network Device / Device Operation / Control Virtqueue / VLAN Filtering}

If the driver negotiates the VIRTIO_NET_F_CTRL_VLAN feature, it
can control a VLAN filter table in the device. The VLAN filter
table applies only to VLAN tagged packets.

When VIRTIO_NET_F_CTRL_VLAN is negotiated, the device starts with
an empty VLAN filter table.

\begin{note}
Similar to the MAC address based filtering, the VLAN filtering
is also best-effort: unwanted packets could still arrive.
\end{note}

\begin{lstlisting}
#define VIRTIO_NET_CTRL_VLAN       2
 #define VIRTIO_NET_CTRL_VLAN_ADD             0
 #define VIRTIO_NET_CTRL_VLAN_DEL             1
\end{lstlisting}

Both the VIRTIO_NET_CTRL_VLAN_ADD and VIRTIO_NET_CTRL_VLAN_DEL
command take a little-endian 16-bit VLAN id as the command-specific-data.

VIRTIO_NET_CTRL_VLAN_ADD command adds the specified VLAN to the
VLAN filter table.

VIRTIO_NET_CTRL_VLAN_DEL command removes the specified VLAN from
the VLAN filter table.

\devicenormative{\subparagraph}{VLAN Filtering}{Device Types / Network Device / Device Operation / Control Virtqueue / VLAN Filtering}

When VIRTIO_NET_F_CTRL_VLAN is not negotiated, the device MUST
accept all VLAN tagged packets.

When VIRTIO_NET_F_CTRL_VLAN is negotiated, the device MUST
accept all VLAN tagged packets whose VLAN tag is present in
the VLAN filter table and SHOULD drop all VLAN tagged packets
whose VLAN tag is absent in the VLAN filter table.

\subparagraph{Legacy Interface: VLAN Filtering}\label{sec:Device Types / Network Device / Device Operation / Control Virtqueue / VLAN Filtering / Legacy Interface: VLAN Filtering}
When using the legacy interface, transitional devices and drivers
MUST format the VLAN id
according to the native endian of the guest rather than
(necessarily when not using the legacy interface) little-endian.

\paragraph{Gratuitous Packet Sending}\label{sec:Device Types / Network Device / Device Operation / Control Virtqueue / Gratuitous Packet Sending}

If the driver negotiates the VIRTIO_NET_F_GUEST_ANNOUNCE (depends
on VIRTIO_NET_F_CTRL_VQ), the device can ask the driver to send gratuitous
packets; this is usually done after the guest has been physically
migrated, and needs to announce its presence on the new network
links. (As hypervisor does not have the knowledge of guest
network configuration (eg. tagged vlan) it is simplest to prod
the guest in this way).

\begin{lstlisting}
#define VIRTIO_NET_CTRL_ANNOUNCE       3
 #define VIRTIO_NET_CTRL_ANNOUNCE_ACK             0
\end{lstlisting}

The driver checks VIRTIO_NET_S_ANNOUNCE bit in the device configuration \field{status} field
when it notices the changes of device configuration. The
command VIRTIO_NET_CTRL_ANNOUNCE_ACK is used to indicate that
driver has received the notification and device clears the
VIRTIO_NET_S_ANNOUNCE bit in \field{status}.

Processing this notification involves:

\begin{enumerate}
\item Sending the gratuitous packets (eg. ARP) or marking there are pending
  gratuitous packets to be sent and letting deferred routine to
  send them.

\item Sending VIRTIO_NET_CTRL_ANNOUNCE_ACK command through control
  vq.
\end{enumerate}

\drivernormative{\subparagraph}{Gratuitous Packet Sending}{Device Types / Network Device / Device Operation / Control Virtqueue / Gratuitous Packet Sending}

If the driver negotiates VIRTIO_NET_F_GUEST_ANNOUNCE, it SHOULD notify
network peers of its new location after it sees the VIRTIO_NET_S_ANNOUNCE bit
in \field{status}.  The driver MUST send a command on the command queue
with class VIRTIO_NET_CTRL_ANNOUNCE and command VIRTIO_NET_CTRL_ANNOUNCE_ACK.

\devicenormative{\subparagraph}{Gratuitous Packet Sending}{Device Types / Network Device / Device Operation / Control Virtqueue / Gratuitous Packet Sending}

If VIRTIO_NET_F_GUEST_ANNOUNCE is negotiated, the device MUST clear the
VIRTIO_NET_S_ANNOUNCE bit in \field{status} upon receipt of a command buffer
with class VIRTIO_NET_CTRL_ANNOUNCE and command VIRTIO_NET_CTRL_ANNOUNCE_ACK
before marking the buffer as used.

\paragraph{Device operation in multiqueue mode}\label{sec:Device Types / Network Device / Device Operation / Control Virtqueue / Device operation in multiqueue mode}

This specification defines the following modes that a device MAY implement for operation with multiple transmit/receive virtqueues:
\begin{itemize}
\item Automatic receive steering as defined in \ref{sec:Device Types / Network Device / Device Operation / Control Virtqueue / Automatic receive steering in multiqueue mode}.
 If a device supports this mode, it offers the VIRTIO_NET_F_MQ feature bit.
\item Receive-side scaling as defined in \ref{devicenormative:Device Types / Network Device / Device Operation / Control Virtqueue / Receive-side scaling (RSS) / RSS processing}.
 If a device supports this mode, it offers the VIRTIO_NET_F_RSS feature bit.
\end{itemize}

A device MAY support one of these features or both. The driver MAY negotiate any set of these features that the device supports.

Multiqueue is disabled by default.

The driver enables multiqueue by sending a command using \field{class} VIRTIO_NET_CTRL_MQ. The \field{command} selects the mode of multiqueue operation, as follows:
\begin{lstlisting}
#define VIRTIO_NET_CTRL_MQ    4
 #define VIRTIO_NET_CTRL_MQ_VQ_PAIRS_SET        0 (for automatic receive steering)
 #define VIRTIO_NET_CTRL_MQ_RSS_CONFIG          1 (for configurable receive steering)
 #define VIRTIO_NET_CTRL_MQ_HASH_CONFIG         2 (for configurable hash calculation)
\end{lstlisting}

If more than one multiqueue mode is negotiated, the resulting device configuration is defined by the last command sent by the driver.

\paragraph{Automatic receive steering in multiqueue mode}\label{sec:Device Types / Network Device / Device Operation / Control Virtqueue / Automatic receive steering in multiqueue mode}

If the driver negotiates the VIRTIO_NET_F_MQ feature bit (depends on VIRTIO_NET_F_CTRL_VQ), it MAY transmit outgoing packets on one
of the multiple transmitq1\ldots transmitqN and ask the device to
queue incoming packets into one of the multiple receiveq1\ldots receiveqN
depending on the packet flow.

The driver enables multiqueue by
sending the VIRTIO_NET_CTRL_MQ_VQ_PAIRS_SET command, specifying
the number of the transmit and receive queues to be used up to
\field{max_virtqueue_pairs}; subsequently,
transmitq1\ldots transmitqn and receiveq1\ldots receiveqn where
n=\field{virtqueue_pairs} MAY be used.
\begin{lstlisting}
struct virtio_net_ctrl_mq_pairs_set {
       le16 virtqueue_pairs;
};
#define VIRTIO_NET_CTRL_MQ_VQ_PAIRS_MIN        1
#define VIRTIO_NET_CTRL_MQ_VQ_PAIRS_MAX        0x8000

\end{lstlisting}

When multiqueue is enabled by VIRTIO_NET_CTRL_MQ_VQ_PAIRS_SET command, the device MUST use automatic receive steering
based on packet flow. Programming of the receive steering
classificator is implicit. After the driver transmitted a packet of a
flow on transmitqX, the device SHOULD cause incoming packets for that flow to
be steered to receiveqX. For uni-directional protocols, or where
no packets have been transmitted yet, the device MAY steer a packet
to a random queue out of the specified receiveq1\ldots receiveqn.

Multiqueue is disabled by VIRTIO_NET_CTRL_MQ_VQ_PAIRS_SET with \field{virtqueue_pairs} to 1 (this is
the default) and waiting for the device to use the command buffer.

\drivernormative{\subparagraph}{Automatic receive steering in multiqueue mode}{Device Types / Network Device / Device Operation / Control Virtqueue / Automatic receive steering in multiqueue mode}

The driver MUST configure the virtqueues before enabling them with the
VIRTIO_NET_CTRL_MQ_VQ_PAIRS_SET command.

The driver MUST NOT request a \field{virtqueue_pairs} of 0 or
greater than \field{max_virtqueue_pairs} in the device configuration space.

The driver MUST queue packets only on any transmitq1 before the
VIRTIO_NET_CTRL_MQ_VQ_PAIRS_SET command.

The driver MUST NOT queue packets on transmit queues greater than
\field{virtqueue_pairs} once it has placed the VIRTIO_NET_CTRL_MQ_VQ_PAIRS_SET command in the available ring.

\devicenormative{\subparagraph}{Automatic receive steering in multiqueue mode}{Device Types / Network Device / Device Operation / Control Virtqueue / Automatic receive steering in multiqueue mode}

After initialization of reset, the device MUST queue packets only on receiveq1.

The device MUST NOT queue packets on receive queues greater than
\field{virtqueue_pairs} once it has placed the
VIRTIO_NET_CTRL_MQ_VQ_PAIRS_SET command in a used buffer.

If the destination receive queue is being reset (See \ref{sec:Basic Facilities of a Virtio Device / Virtqueues / Virtqueue Reset}),
the device SHOULD re-select another random queue. If all receive queues are
being reset, the device MUST drop the packet.

\subparagraph{Legacy Interface: Automatic receive steering in multiqueue mode}\label{sec:Device Types / Network Device / Device Operation / Control Virtqueue / Automatic receive steering in multiqueue mode / Legacy Interface: Automatic receive steering in multiqueue mode}
When using the legacy interface, transitional devices and drivers
MUST format \field{virtqueue_pairs}
according to the native endian of the guest rather than
(necessarily when not using the legacy interface) little-endian.

\subparagraph{Hash calculation}\label{sec:Device Types / Network Device / Device Operation / Control Virtqueue / Automatic receive steering in multiqueue mode / Hash calculation}
If VIRTIO_NET_F_HASH_REPORT was negotiated and the device uses automatic receive steering,
the device MUST support a command to configure hash calculation parameters.

The driver provides parameters for hash calculation as follows:

\field{class} VIRTIO_NET_CTRL_MQ, \field{command} VIRTIO_NET_CTRL_MQ_HASH_CONFIG.

The \field{command-specific-data} has following format:
\begin{lstlisting}
struct virtio_net_hash_config {
    le32 hash_types;
    le16 reserved[4];
    u8 hash_key_length;
    u8 hash_key_data[hash_key_length];
};
\end{lstlisting}
Field \field{hash_types} contains a bitmask of allowed hash types as
defined in
\ref{sec:Device Types / Network Device / Device Operation / Processing of Incoming Packets / Hash calculation for incoming packets / Supported/enabled hash types}.
Initially the device has all hash types disabled and reports only VIRTIO_NET_HASH_REPORT_NONE.

Field \field{reserved} MUST contain zeroes. It is defined to make the structure to match the layout of virtio_net_rss_config structure,
defined in \ref{sec:Device Types / Network Device / Device Operation / Control Virtqueue / Receive-side scaling (RSS)}.

Fields \field{hash_key_length} and \field{hash_key_data} define the key to be used in hash calculation.

\paragraph{Receive-side scaling (RSS)}\label{sec:Device Types / Network Device / Device Operation / Control Virtqueue / Receive-side scaling (RSS)}
A device offers the feature VIRTIO_NET_F_RSS if it supports RSS receive steering with Toeplitz hash calculation and configurable parameters.

A driver queries RSS capabilities of the device by reading device configuration as defined in \ref{sec:Device Types / Network Device / Device configuration layout}

\subparagraph{Setting RSS parameters}\label{sec:Device Types / Network Device / Device Operation / Control Virtqueue / Receive-side scaling (RSS) / Setting RSS parameters}

Driver sends a VIRTIO_NET_CTRL_MQ_RSS_CONFIG command using the following format for \field{command-specific-data}:
\begin{lstlisting}
struct rss_rq_id {
   le16 vq_index_1_16: 15; /* Bits 1 to 16 of the virtqueue index */
   le16 reserved: 1; /* Set to zero */
};

struct virtio_net_rss_config {
    le32 hash_types;
    le16 indirection_table_mask;
    struct rss_rq_id unclassified_queue;
    struct rss_rq_id indirection_table[indirection_table_length];
    le16 max_tx_vq;
    u8 hash_key_length;
    u8 hash_key_data[hash_key_length];
};
\end{lstlisting}
Field \field{hash_types} contains a bitmask of allowed hash types as
defined in
\ref{sec:Device Types / Network Device / Device Operation / Processing of Incoming Packets / Hash calculation for incoming packets / Supported/enabled hash types}.

Field \field{indirection_table_mask} is a mask to be applied to
the calculated hash to produce an index in the
\field{indirection_table} array.
Number of entries in \field{indirection_table} is (\field{indirection_table_mask} + 1).

\field{rss_rq_id} is a receive virtqueue id. \field{vq_index_1_16}
consists of bits 1 to 16 of a virtqueue index. For example, a
\field{vq_index_1_16} value of 3 corresponds to virtqueue index 6,
which maps to receiveq4.

Field \field{unclassified_queue} specifies the receive virtqueue id in which to
place unclassified packets.

Field \field{indirection_table} is an array of receive virtqueues ids.

A driver sets \field{max_tx_vq} to inform a device how many transmit virtqueues it may use (transmitq1\ldots transmitq \field{max_tx_vq}).

Fields \field{hash_key_length} and \field{hash_key_data} define the key to be used in hash calculation.

\drivernormative{\subparagraph}{Setting RSS parameters}{Device Types / Network Device / Device Operation / Control Virtqueue / Receive-side scaling (RSS) }

A driver MUST NOT send the VIRTIO_NET_CTRL_MQ_RSS_CONFIG command if the feature VIRTIO_NET_F_RSS has not been negotiated.

A driver MUST fill the \field{indirection_table} array only with
enabled receive virtqueues ids.

The number of entries in \field{indirection_table} (\field{indirection_table_mask} + 1) MUST be a power of two.

A driver MUST use \field{indirection_table_mask} values that are less than \field{rss_max_indirection_table_length} reported by a device.

A driver MUST NOT set any VIRTIO_NET_HASH_TYPE_ flags that are not supported by a device.

\devicenormative{\subparagraph}{RSS processing}{Device Types / Network Device / Device Operation / Control Virtqueue / Receive-side scaling (RSS) / RSS processing}
The device MUST determine the destination queue for a network packet as follows:
\begin{itemize}
\item Calculate the hash of the packet as defined in \ref{sec:Device Types / Network Device / Device Operation / Processing of Incoming Packets / Hash calculation for incoming packets}.
\item If the device did not calculate the hash for the specific packet, the device directs the packet to the receiveq specified by \field{unclassified_queue} of virtio_net_rss_config structure.
\item Apply \field{indirection_table_mask} to the calculated hash
and use the result as the index in the indirection table to get
the destination receive virtqueue id.
\item If the destination receive queue is being reset (See \ref{sec:Basic Facilities of a Virtio Device / Virtqueues / Virtqueue Reset}), the device MUST drop the packet.
\end{itemize}

\paragraph{RSS Context}\label{sec:Device Types / Network Device / Device Operation / Control Virtqueue / RSS Context}

An RSS context consists of configurable parameters specified by \ref{sec:Device Types / Network Device
/ Device Operation / Control Virtqueue / Receive-side scaling (RSS)}.

The RSS configuration supported by VIRTIO_NET_F_RSS is considered the default RSS configuration.

The device offers the feature VIRTIO_NET_F_RSS_CONTEXT if it supports one or multiple RSS contexts
(excluding the default RSS configuration) and configurable parameters.

\subparagraph{Querying RSS Context Capability}\label{sec:Device Types / Network Device / Device Operation / Control Virtqueue / RSS Context / Querying RSS Context Capability}

\begin{lstlisting}
#define VIRTNET_RSS_CTX_CTRL 9
 #define VIRTNET_RSS_CTX_CTRL_CAP_GET  0
 #define VIRTNET_RSS_CTX_CTRL_ADD      1
 #define VIRTNET_RSS_CTX_CTRL_MOD      2
 #define VIRTNET_RSS_CTX_CTRL_DEL      3

struct virtnet_rss_ctx_cap {
    le16 max_rss_contexts;
}
\end{lstlisting}

Field \field{max_rss_contexts} specifies the maximum number of RSS contexts \ref{sec:Device Types / Network Device /
Device Operation / Control Virtqueue / RSS Context} supported by the device.

The driver queries the RSS context capability of the device by sending the command VIRTNET_RSS_CTX_CTRL_CAP_GET
with the structure virtnet_rss_ctx_cap.

For the command VIRTNET_RSS_CTX_CTRL_CAP_GET, the structure virtnet_rss_ctx_cap is write-only for the device.

\subparagraph{Setting RSS Context Parameters}\label{sec:Device Types / Network Device / Device Operation / Control Virtqueue / RSS Context / Setting RSS Context Parameters}

\begin{lstlisting}
struct virtnet_rss_ctx_add_modify {
    le16 rss_ctx_id;
    u8 reserved[6];
    struct virtio_net_rss_config rss;
};

struct virtnet_rss_ctx_del {
    le16 rss_ctx_id;
};
\end{lstlisting}

RSS context parameters:
\begin{itemize}
\item  \field{rss_ctx_id}: ID of the specific RSS context.
\item  \field{rss}: RSS context parameters of the specific RSS context whose id is \field{rss_ctx_id}.
\end{itemize}

\field{reserved} is reserved and it is ignored by the device.

If the feature VIRTIO_NET_F_RSS_CONTEXT has been negotiated, the driver can send the following
VIRTNET_RSS_CTX_CTRL class commands:
\begin{enumerate}
\item VIRTNET_RSS_CTX_CTRL_ADD: use the structure virtnet_rss_ctx_add_modify to
       add an RSS context configured as \field{rss} and id as \field{rss_ctx_id} for the device.
\item VIRTNET_RSS_CTX_CTRL_MOD: use the structure virtnet_rss_ctx_add_modify to
       configure parameters of the RSS context whose id is \field{rss_ctx_id} as \field{rss} for the device.
\item VIRTNET_RSS_CTX_CTRL_DEL: use the structure virtnet_rss_ctx_del to delete
       the RSS context whose id is \field{rss_ctx_id} for the device.
\end{enumerate}

For commands VIRTNET_RSS_CTX_CTRL_ADD and VIRTNET_RSS_CTX_CTRL_MOD, the structure virtnet_rss_ctx_add_modify is read-only for the device.
For the command VIRTNET_RSS_CTX_CTRL_DEL, the structure virtnet_rss_ctx_del is read-only for the device.

\devicenormative{\subparagraph}{RSS Context}{Device Types / Network Device / Device Operation / Control Virtqueue / RSS Context}

The device MUST set \field{max_rss_contexts} to at least 1 if it offers VIRTIO_NET_F_RSS_CONTEXT.

Upon reset, the device MUST clear all previously configured RSS contexts.

\drivernormative{\subparagraph}{RSS Context}{Device Types / Network Device / Device Operation / Control Virtqueue / RSS Context}

The driver MUST have negotiated the VIRTIO_NET_F_RSS_CONTEXT feature when issuing the VIRTNET_RSS_CTX_CTRL class commands.

The driver MUST set \field{rss_ctx_id} to between 1 and \field{max_rss_contexts} inclusive.

The driver MUST NOT send the command VIRTIO_NET_CTRL_MQ_VQ_PAIRS_SET when the device has successfully configured at least one RSS context.

\paragraph{Offloads State Configuration}\label{sec:Device Types / Network Device / Device Operation / Control Virtqueue / Offloads State Configuration}

If the VIRTIO_NET_F_CTRL_GUEST_OFFLOADS feature is negotiated, the driver can
send control commands for dynamic offloads state configuration.

\subparagraph{Setting Offloads State}\label{sec:Device Types / Network Device / Device Operation / Control Virtqueue / Offloads State Configuration / Setting Offloads State}

To configure the offloads, the following layout structure and
definitions are used:

\begin{lstlisting}
le64 offloads;

#define VIRTIO_NET_F_GUEST_CSUM       1
#define VIRTIO_NET_F_GUEST_TSO4       7
#define VIRTIO_NET_F_GUEST_TSO6       8
#define VIRTIO_NET_F_GUEST_ECN        9
#define VIRTIO_NET_F_GUEST_UFO        10
#define VIRTIO_NET_F_GUEST_UDP_TUNNEL_GSO  46
#define VIRTIO_NET_F_GUEST_UDP_TUNNEL_GSO_CSUM 47
#define VIRTIO_NET_F_GUEST_USO4       54
#define VIRTIO_NET_F_GUEST_USO6       55

#define VIRTIO_NET_CTRL_GUEST_OFFLOADS       5
 #define VIRTIO_NET_CTRL_GUEST_OFFLOADS_SET   0
\end{lstlisting}

The class VIRTIO_NET_CTRL_GUEST_OFFLOADS has one command:
VIRTIO_NET_CTRL_GUEST_OFFLOADS_SET applies the new offloads configuration.

le64 value passed as command data is a bitmask, bits set define
offloads to be enabled, bits cleared - offloads to be disabled.

There is a corresponding device feature for each offload. Upon feature
negotiation corresponding offload gets enabled to preserve backward
compatibility.

\drivernormative{\subparagraph}{Setting Offloads State}{Device Types / Network Device / Device Operation / Control Virtqueue / Offloads State Configuration / Setting Offloads State}

A driver MUST NOT enable an offload for which the appropriate feature
has not been negotiated.

\subparagraph{Legacy Interface: Setting Offloads State}\label{sec:Device Types / Network Device / Device Operation / Control Virtqueue / Offloads State Configuration / Setting Offloads State / Legacy Interface: Setting Offloads State}
When using the legacy interface, transitional devices and drivers
MUST format \field{offloads}
according to the native endian of the guest rather than
(necessarily when not using the legacy interface) little-endian.


\paragraph{Notifications Coalescing}\label{sec:Device Types / Network Device / Device Operation / Control Virtqueue / Notifications Coalescing}

If the VIRTIO_NET_F_NOTF_COAL feature is negotiated, the driver can
send commands VIRTIO_NET_CTRL_NOTF_COAL_TX_SET and VIRTIO_NET_CTRL_NOTF_COAL_RX_SET
for notification coalescing.

If the VIRTIO_NET_F_VQ_NOTF_COAL feature is negotiated, the driver can
send commands VIRTIO_NET_CTRL_NOTF_COAL_VQ_SET and VIRTIO_NET_CTRL_NOTF_COAL_VQ_GET
for virtqueue notification coalescing.

\begin{lstlisting}
struct virtio_net_ctrl_coal {
    le32 max_packets;
    le32 max_usecs;
};

struct virtio_net_ctrl_coal_vq {
    le16 vq_index;
    le16 reserved;
    struct virtio_net_ctrl_coal coal;
};

#define VIRTIO_NET_CTRL_NOTF_COAL 6
 #define VIRTIO_NET_CTRL_NOTF_COAL_TX_SET  0
 #define VIRTIO_NET_CTRL_NOTF_COAL_RX_SET 1
 #define VIRTIO_NET_CTRL_NOTF_COAL_VQ_SET 2
 #define VIRTIO_NET_CTRL_NOTF_COAL_VQ_GET 3
\end{lstlisting}

Coalescing parameters:
\begin{itemize}
\item \field{vq_index}: The virtqueue index of an enabled transmit or receive virtqueue.
\item \field{max_usecs} for RX: Maximum number of microseconds to delay a RX notification.
\item \field{max_usecs} for TX: Maximum number of microseconds to delay a TX notification.
\item \field{max_packets} for RX: Maximum number of packets to receive before a RX notification.
\item \field{max_packets} for TX: Maximum number of packets to send before a TX notification.
\end{itemize}

\field{reserved} is reserved and it is ignored by the device.

Read/Write attributes for coalescing parameters:
\begin{itemize}
\item For commands VIRTIO_NET_CTRL_NOTF_COAL_TX_SET and VIRTIO_NET_CTRL_NOTF_COAL_RX_SET, the structure virtio_net_ctrl_coal is write-only for the driver.
\item For the command VIRTIO_NET_CTRL_NOTF_COAL_VQ_SET, the structure virtio_net_ctrl_coal_vq is write-only for the driver.
\item For the command VIRTIO_NET_CTRL_NOTF_COAL_VQ_GET, \field{vq_index} and \field{reserved} are write-only
      for the driver, and the structure virtio_net_ctrl_coal is read-only for the driver.
\end{itemize}

The class VIRTIO_NET_CTRL_NOTF_COAL has the following commands:
\begin{enumerate}
\item VIRTIO_NET_CTRL_NOTF_COAL_TX_SET: use the structure virtio_net_ctrl_coal to set the \field{max_usecs} and \field{max_packets} parameters for all transmit virtqueues.
\item VIRTIO_NET_CTRL_NOTF_COAL_RX_SET: use the structure virtio_net_ctrl_coal to set the \field{max_usecs} and \field{max_packets} parameters for all receive virtqueues.
\item VIRTIO_NET_CTRL_NOTF_COAL_VQ_SET: use the structure virtio_net_ctrl_coal_vq to set the \field{max_usecs} and \field{max_packets} parameters
                                        for an enabled transmit/receive virtqueue whose index is \field{vq_index}.
\item VIRTIO_NET_CTRL_NOTF_COAL_VQ_GET: use the structure virtio_net_ctrl_coal_vq to get the \field{max_usecs} and \field{max_packets} parameters
                                        for an enabled transmit/receive virtqueue whose index is \field{vq_index}.
\end{enumerate}

The device may generate notifications more or less frequently than specified by set commands of the VIRTIO_NET_CTRL_NOTF_COAL class.

If coalescing parameters are being set, the device applies the last coalescing parameters set for a
virtqueue, regardless of the command used to set the parameters. Use the following command sequence
with two pairs of virtqueues as an example:
Each of the following commands sets \field{max_usecs} and \field{max_packets} parameters for virtqueues.
\begin{itemize}
\item Command1: VIRTIO_NET_CTRL_NOTF_COAL_RX_SET sets coalescing parameters for virtqueues having index 0 and index 2. Virtqueues having index 1 and index 3 retain their previous parameters.
\item Command2: VIRTIO_NET_CTRL_NOTF_COAL_VQ_SET with \field{vq_index} = 0 sets coalescing parameters for virtqueue having index 0. Virtqueue having index 2 retains the parameters from command1.
\item Command3: VIRTIO_NET_CTRL_NOTF_COAL_VQ_GET with \field{vq_index} = 0, the device responds with coalescing parameters of vq_index 0 set by command2.
\item Command4: VIRTIO_NET_CTRL_NOTF_COAL_VQ_SET with \field{vq_index} = 1 sets coalescing parameters for virtqueue having index 1. Virtqueue having index 3 retains its previous parameters.
\item Command5: VIRTIO_NET_CTRL_NOTF_COAL_TX_SET sets coalescing parameters for virtqueues having index 1 and index 3, and overrides the parameters set by command4.
\item Command6: VIRTIO_NET_CTRL_NOTF_COAL_VQ_GET with \field{vq_index} = 1, the device responds with coalescing parameters of index 1 set by command5.
\end{itemize}

\subparagraph{Operation}\label{sec:Device Types / Network Device / Device Operation / Control Virtqueue / Notifications Coalescing / Operation}

The device sends a used buffer notification once the notification conditions are met and if the notifications are not suppressed as explained in \ref{sec:Basic Facilities of a Virtio Device / Virtqueues / Used Buffer Notification Suppression}.

When the device has non-zero \field{max_usecs} and non-zero \field{max_packets}, it starts counting microseconds and packets upon receiving/sending a packet.
The device counts packets and microseconds for each receive virtqueue and transmit virtqueue separately.
In this case, the notification conditions are met when \field{max_usecs} microseconds elapse, or upon sending/receiving \field{max_packets} packets, whichever happens first.
Afterwards, the device waits for the next packet and starts counting packets and microseconds again.

When the device has \field{max_usecs} = 0 or \field{max_packets} = 0, the notification conditions are met after every packet received/sent.

\subparagraph{RX Example}\label{sec:Device Types / Network Device / Device Operation / Control Virtqueue / Notifications Coalescing / RX Example}

If, for example:
\begin{itemize}
\item \field{max_usecs} = 10.
\item \field{max_packets} = 15.
\end{itemize}
then each receive virtqueue of a device will operate as follows:
\begin{itemize}
\item The device will count packets received on each virtqueue until it accumulates 15, or until 10 microseconds elapsed since the first one was received.
\item If the notifications are not suppressed by the driver, the device will send an used buffer notification, otherwise, the device will not send an used buffer notification as long as the notifications are suppressed.
\end{itemize}

\subparagraph{TX Example}\label{sec:Device Types / Network Device / Device Operation / Control Virtqueue / Notifications Coalescing / TX Example}

If, for example:
\begin{itemize}
\item \field{max_usecs} = 10.
\item \field{max_packets} = 15.
\end{itemize}
then each transmit virtqueue of a device will operate as follows:
\begin{itemize}
\item The device will count packets sent on each virtqueue until it accumulates 15, or until 10 microseconds elapsed since the first one was sent.
\item If the notifications are not suppressed by the driver, the device will send an used buffer notification, otherwise, the device will not send an used buffer notification as long as the notifications are suppressed.
\end{itemize}

\subparagraph{Notifications When Coalescing Parameters Change}\label{sec:Device Types / Network Device / Device Operation / Control Virtqueue / Notifications Coalescing / Notifications When Coalescing Parameters Change}

When the coalescing parameters of a device change, the device needs to check if the new notification conditions are met and send a used buffer notification if so.

For example, \field{max_packets} = 15 for a device with a single transmit virtqueue: if the device sends 10 packets and afterwards receives a
VIRTIO_NET_CTRL_NOTF_COAL_TX_SET command with \field{max_packets} = 8, then the notification condition is immediately considered to be met;
the device needs to immediately send a used buffer notification, if the notifications are not suppressed by the driver.

\drivernormative{\subparagraph}{Notifications Coalescing}{Device Types / Network Device / Device Operation / Control Virtqueue / Notifications Coalescing}

The driver MUST set \field{vq_index} to the virtqueue index of an enabled transmit or receive virtqueue.

The driver MUST have negotiated the VIRTIO_NET_F_NOTF_COAL feature when issuing commands VIRTIO_NET_CTRL_NOTF_COAL_TX_SET and VIRTIO_NET_CTRL_NOTF_COAL_RX_SET.

The driver MUST have negotiated the VIRTIO_NET_F_VQ_NOTF_COAL feature when issuing commands VIRTIO_NET_CTRL_NOTF_COAL_VQ_SET and VIRTIO_NET_CTRL_NOTF_COAL_VQ_GET.

The driver MUST ignore the values of coalescing parameters received from the VIRTIO_NET_CTRL_NOTF_COAL_VQ_GET command if the device responds with VIRTIO_NET_ERR.

\devicenormative{\subparagraph}{Notifications Coalescing}{Device Types / Network Device / Device Operation / Control Virtqueue / Notifications Coalescing}

The device MUST ignore \field{reserved}.

The device SHOULD respond to VIRTIO_NET_CTRL_NOTF_COAL_TX_SET and VIRTIO_NET_CTRL_NOTF_COAL_RX_SET commands with VIRTIO_NET_ERR if it was not able to change the parameters.

The device MUST respond to the VIRTIO_NET_CTRL_NOTF_COAL_VQ_SET command with VIRTIO_NET_ERR if it was not able to change the parameters.

The device MUST respond to VIRTIO_NET_CTRL_NOTF_COAL_VQ_SET and VIRTIO_NET_CTRL_NOTF_COAL_VQ_GET commands with
VIRTIO_NET_ERR if the designated virtqueue is not an enabled transmit or receive virtqueue.

Upon disabling and re-enabling a transmit virtqueue, the device MUST set the coalescing parameters of the virtqueue
to those configured through the VIRTIO_NET_CTRL_NOTF_COAL_TX_SET command, or, if the driver did not set any TX coalescing parameters, to 0.

Upon disabling and re-enabling a receive virtqueue, the device MUST set the coalescing parameters of the virtqueue
to those configured through the VIRTIO_NET_CTRL_NOTF_COAL_RX_SET command, or, if the driver did not set any RX coalescing parameters, to 0.

The behavior of the device in response to set commands of the VIRTIO_NET_CTRL_NOTF_COAL class is best-effort:
the device MAY generate notifications more or less frequently than specified.

A device SHOULD NOT send used buffer notifications to the driver if the notifications are suppressed, even if the notification conditions are met.

Upon reset, a device MUST initialize all coalescing parameters to 0.

\paragraph{Device Statistics}\label{sec:Device Types / Network Device / Device Operation / Control Virtqueue / Device Statistics}

If the VIRTIO_NET_F_DEVICE_STATS feature is negotiated, the driver can obtain
device statistics from the device by using the following command.

Different types of virtqueues have different statistics. The statistics of the
receiveq are different from those of the transmitq.

The statistics of a certain type of virtqueue are also divided into multiple types
because different types require different features. This enables the expansion
of new statistics.

In one command, the driver can obtain the statistics of one or multiple virtqueues.
Additionally, the driver can obtain multiple type statistics of each virtqueue.

\subparagraph{Query Statistic Capabilities}\label{sec:Device Types / Network Device / Device Operation / Control Virtqueue / Device Statistics / Query Statistic Capabilities}

\begin{lstlisting}
#define VIRTIO_NET_CTRL_STATS         8
#define VIRTIO_NET_CTRL_STATS_QUERY   0
#define VIRTIO_NET_CTRL_STATS_GET     1

struct virtio_net_stats_capabilities {

#define VIRTIO_NET_STATS_TYPE_CVQ       (1 << 32)

#define VIRTIO_NET_STATS_TYPE_RX_BASIC  (1 << 0)
#define VIRTIO_NET_STATS_TYPE_RX_CSUM   (1 << 1)
#define VIRTIO_NET_STATS_TYPE_RX_GSO    (1 << 2)
#define VIRTIO_NET_STATS_TYPE_RX_SPEED  (1 << 3)

#define VIRTIO_NET_STATS_TYPE_TX_BASIC  (1 << 16)
#define VIRTIO_NET_STATS_TYPE_TX_CSUM   (1 << 17)
#define VIRTIO_NET_STATS_TYPE_TX_GSO    (1 << 18)
#define VIRTIO_NET_STATS_TYPE_TX_SPEED  (1 << 19)

    le64 supported_stats_types[1];
}
\end{lstlisting}

To obtain device statistic capability, use the VIRTIO_NET_CTRL_STATS_QUERY
command. When the command completes successfully, \field{command-specific-result}
is in the format of \field{struct virtio_net_stats_capabilities}.

\subparagraph{Get Statistics}\label{sec:Device Types / Network Device / Device Operation / Control Virtqueue / Device Statistics / Get Statistics}

\begin{lstlisting}
struct virtio_net_ctrl_queue_stats {
       struct {
           le16 vq_index;
           le16 reserved[3];
           le64 types_bitmap[1];
       } stats[];
};

struct virtio_net_stats_reply_hdr {
#define VIRTIO_NET_STATS_TYPE_REPLY_CVQ       32

#define VIRTIO_NET_STATS_TYPE_REPLY_RX_BASIC  0
#define VIRTIO_NET_STATS_TYPE_REPLY_RX_CSUM   1
#define VIRTIO_NET_STATS_TYPE_REPLY_RX_GSO    2
#define VIRTIO_NET_STATS_TYPE_REPLY_RX_SPEED  3

#define VIRTIO_NET_STATS_TYPE_REPLY_TX_BASIC  16
#define VIRTIO_NET_STATS_TYPE_REPLY_TX_CSUM   17
#define VIRTIO_NET_STATS_TYPE_REPLY_TX_GSO    18
#define VIRTIO_NET_STATS_TYPE_REPLY_TX_SPEED  19
    u8 type;
    u8 reserved;
    le16 vq_index;
    le16 reserved1;
    le16 size;
}
\end{lstlisting}

To obtain device statistics, use the VIRTIO_NET_CTRL_STATS_GET command with the
\field{command-specific-data} which is in the format of
\field{struct virtio_net_ctrl_queue_stats}. When the command completes
successfully, \field{command-specific-result} contains multiple statistic
results, each statistic result has the \field{struct virtio_net_stats_reply_hdr}
as the header.

The fields of the \field{struct virtio_net_ctrl_queue_stats}:
\begin{description}
    \item [vq_index]
        The index of the virtqueue to obtain the statistics.

    \item [types_bitmap]
        This is a bitmask of the types of statistics to be obtained. Therefore, a
        \field{stats} inside \field{struct virtio_net_ctrl_queue_stats} may
        indicate multiple statistic replies for the virtqueue.
\end{description}

The fields of the \field{struct virtio_net_stats_reply_hdr}:
\begin{description}
    \item [type]
        The type of the reply statistic.

    \item [vq_index]
        The virtqueue index of the reply statistic.

    \item [size]
        The number of bytes for the statistics entry including size of \field{struct virtio_net_stats_reply_hdr}.

\end{description}

\subparagraph{Controlq Statistics}\label{sec:Device Types / Network Device / Device Operation / Control Virtqueue / Device Statistics / Controlq Statistics}

The structure corresponding to the controlq statistics is
\field{struct virtio_net_stats_cvq}. The corresponding type is
VIRTIO_NET_STATS_TYPE_CVQ. This is for the controlq.

\begin{lstlisting}
struct virtio_net_stats_cvq {
    struct virtio_net_stats_reply_hdr hdr;

    le64 command_num;
    le64 ok_num;
};
\end{lstlisting}

\begin{description}
    \item [command_num]
        The number of commands received by the device including the current command.

    \item [ok_num]
        The number of commands completed successfully by the device including the current command.
\end{description}


\subparagraph{Receiveq Basic Statistics}\label{sec:Device Types / Network Device / Device Operation / Control Virtqueue / Device Statistics / Receiveq Basic Statistics}

The structure corresponding to the receiveq basic statistics is
\field{struct virtio_net_stats_rx_basic}. The corresponding type is
VIRTIO_NET_STATS_TYPE_RX_BASIC. This is for the receiveq.

Receiveq basic statistics do not require any feature. As long as the device supports
VIRTIO_NET_F_DEVICE_STATS, the following are the receiveq basic statistics.

\begin{lstlisting}
struct virtio_net_stats_rx_basic {
    struct virtio_net_stats_reply_hdr hdr;

    le64 rx_notifications;

    le64 rx_packets;
    le64 rx_bytes;

    le64 rx_interrupts;

    le64 rx_drops;
    le64 rx_drop_overruns;
};
\end{lstlisting}

The packets described below were all presented on the specified virtqueue.
\begin{description}
    \item [rx_notifications]
        The number of driver notifications received by the device for this
        receiveq.

    \item [rx_packets]
        This is the number of packets passed to the driver by the device.

    \item [rx_bytes]
        This is the bytes of packets passed to the driver by the device.

    \item [rx_interrupts]
        The number of interrupts generated by the device for this receiveq.

    \item [rx_drops]
        This is the number of packets dropped by the device. The count includes
        all types of packets dropped by the device.

    \item [rx_drop_overruns]
        This is the number of packets dropped by the device when no more
        descriptors were available.

\end{description}

\subparagraph{Transmitq Basic Statistics}\label{sec:Device Types / Network Device / Device Operation / Control Virtqueue / Device Statistics / Transmitq Basic Statistics}

The structure corresponding to the transmitq basic statistics is
\field{struct virtio_net_stats_tx_basic}. The corresponding type is
VIRTIO_NET_STATS_TYPE_TX_BASIC. This is for the transmitq.

Transmitq basic statistics do not require any feature. As long as the device supports
VIRTIO_NET_F_DEVICE_STATS, the following are the transmitq basic statistics.

\begin{lstlisting}
struct virtio_net_stats_tx_basic {
    struct virtio_net_stats_reply_hdr hdr;

    le64 tx_notifications;

    le64 tx_packets;
    le64 tx_bytes;

    le64 tx_interrupts;

    le64 tx_drops;
    le64 tx_drop_malformed;
};
\end{lstlisting}

The packets described below are all for a specific virtqueue.
\begin{description}
    \item [tx_notifications]
        The number of driver notifications received by the device for this
        transmitq.

    \item [tx_packets]
        This is the number of packets sent by the device (not the packets
        got from the driver).

    \item [tx_bytes]
        This is the number of bytes sent by the device for all the sent packets
        (not the bytes sent got from the driver).

    \item [tx_interrupts]
        The number of interrupts generated by the device for this transmitq.

    \item [tx_drops]
        The number of packets dropped by the device. The count includes all
        types of packets dropped by the device.

    \item [tx_drop_malformed]
        The number of packets dropped by the device, when the descriptors are
        malformed. For example, the buffer is too short.
\end{description}

\subparagraph{Receiveq CSUM Statistics}\label{sec:Device Types / Network Device / Device Operation / Control Virtqueue / Device Statistics / Receiveq CSUM Statistics}

The structure corresponding to the receiveq checksum statistics is
\field{struct virtio_net_stats_rx_csum}. The corresponding type is
VIRTIO_NET_STATS_TYPE_RX_CSUM. This is for the receiveq.

Only after the VIRTIO_NET_F_GUEST_CSUM is negotiated, the receiveq checksum
statistics can be obtained.

\begin{lstlisting}
struct virtio_net_stats_rx_csum {
    struct virtio_net_stats_reply_hdr hdr;

    le64 rx_csum_valid;
    le64 rx_needs_csum;
    le64 rx_csum_none;
    le64 rx_csum_bad;
};
\end{lstlisting}

The packets described below were all presented on the specified virtqueue.
\begin{description}
    \item [rx_csum_valid]
        The number of packets with VIRTIO_NET_HDR_F_DATA_VALID.

    \item [rx_needs_csum]
        The number of packets with VIRTIO_NET_HDR_F_NEEDS_CSUM.

    \item [rx_csum_none]
        The number of packets without hardware checksum. The packet here refers
        to the non-TCP/UDP packet that the device cannot recognize.

    \item [rx_csum_bad]
        The number of packets with checksum mismatch.

\end{description}

\subparagraph{Transmitq CSUM Statistics}\label{sec:Device Types / Network Device / Device Operation / Control Virtqueue / Device Statistics / Transmitq CSUM Statistics}

The structure corresponding to the transmitq checksum statistics is
\field{struct virtio_net_stats_tx_csum}. The corresponding type is
VIRTIO_NET_STATS_TYPE_TX_CSUM. This is for the transmitq.

Only after the VIRTIO_NET_F_CSUM is negotiated, the transmitq checksum
statistics can be obtained.

The following are the transmitq checksum statistics:

\begin{lstlisting}
struct virtio_net_stats_tx_csum {
    struct virtio_net_stats_reply_hdr hdr;

    le64 tx_csum_none;
    le64 tx_needs_csum;
};
\end{lstlisting}

The packets described below are all for a specific virtqueue.
\begin{description}
    \item [tx_csum_none]
        The number of packets which do not require hardware checksum.

    \item [tx_needs_csum]
        The number of packets which require checksum calculation by the device.

\end{description}

\subparagraph{Receiveq GSO Statistics}\label{sec:Device Types / Network Device / Device Operation / Control Virtqueue / Device Statistics / Receiveq GSO Statistics}

The structure corresponding to the receivq GSO statistics is
\field{struct virtio_net_stats_rx_gso}. The corresponding type is
VIRTIO_NET_STATS_TYPE_RX_GSO. This is for the receiveq.

If one or more of the VIRTIO_NET_F_GUEST_TSO4, VIRTIO_NET_F_GUEST_TSO6
have been negotiated, the receiveq GSO statistics can be obtained.

GSO packets refer to packets passed by the device to the driver where
\field{gso_type} is not VIRTIO_NET_HDR_GSO_NONE.

\begin{lstlisting}
struct virtio_net_stats_rx_gso {
    struct virtio_net_stats_reply_hdr hdr;

    le64 rx_gso_packets;
    le64 rx_gso_bytes;
    le64 rx_gso_packets_coalesced;
    le64 rx_gso_bytes_coalesced;
};
\end{lstlisting}

The packets described below were all presented on the specified virtqueue.
\begin{description}
    \item [rx_gso_packets]
        The number of the GSO packets received by the device.

    \item [rx_gso_bytes]
        The bytes of the GSO packets received by the device.
        This includes the header size of the GSO packet.

    \item [rx_gso_packets_coalesced]
        The number of the GSO packets coalesced by the device.

    \item [rx_gso_bytes_coalesced]
        The bytes of the GSO packets coalesced by the device.
        This includes the header size of the GSO packet.
\end{description}

\subparagraph{Transmitq GSO Statistics}\label{sec:Device Types / Network Device / Device Operation / Control Virtqueue / Device Statistics / Transmitq GSO Statistics}

The structure corresponding to the transmitq GSO statistics is
\field{struct virtio_net_stats_tx_gso}. The corresponding type is
VIRTIO_NET_STATS_TYPE_TX_GSO. This is for the transmitq.

If one or more of the VIRTIO_NET_F_HOST_TSO4, VIRTIO_NET_F_HOST_TSO6,
VIRTIO_NET_F_HOST_USO options have been negotiated, the transmitq GSO statistics
can be obtained.

GSO packets refer to packets passed by the driver to the device where
\field{gso_type} is not VIRTIO_NET_HDR_GSO_NONE.
See more \ref{sec:Device Types / Network Device / Device Operation / Packet
Transmission}.

\begin{lstlisting}
struct virtio_net_stats_tx_gso {
    struct virtio_net_stats_reply_hdr hdr;

    le64 tx_gso_packets;
    le64 tx_gso_bytes;
    le64 tx_gso_segments;
    le64 tx_gso_segments_bytes;
    le64 tx_gso_packets_noseg;
    le64 tx_gso_bytes_noseg;
};
\end{lstlisting}

The packets described below are all for a specific virtqueue.
\begin{description}
    \item [tx_gso_packets]
        The number of the GSO packets sent by the device.

    \item [tx_gso_bytes]
        The bytes of the GSO packets sent by the device.

    \item [tx_gso_segments]
        The number of segments prepared from GSO packets.

    \item [tx_gso_segments_bytes]
        The bytes of segments prepared from GSO packets.

    \item [tx_gso_packets_noseg]
        The number of the GSO packets without segmentation.

    \item [tx_gso_bytes_noseg]
        The bytes of the GSO packets without segmentation.

\end{description}

\subparagraph{Receiveq Speed Statistics}\label{sec:Device Types / Network Device / Device Operation / Control Virtqueue / Device Statistics / Receiveq Speed Statistics}

The structure corresponding to the receiveq speed statistics is
\field{struct virtio_net_stats_rx_speed}. The corresponding type is
VIRTIO_NET_STATS_TYPE_RX_SPEED. This is for the receiveq.

The device has the allowance for the speed. If VIRTIO_NET_F_SPEED_DUPLEX has
been negotiated, the driver can get this by \field{speed}. When the received
packets bitrate exceeds the \field{speed}, some packets may be dropped by the
device.

\begin{lstlisting}
struct virtio_net_stats_rx_speed {
    struct virtio_net_stats_reply_hdr hdr;

    le64 rx_packets_allowance_exceeded;
    le64 rx_bytes_allowance_exceeded;
};
\end{lstlisting}

The packets described below were all presented on the specified virtqueue.
\begin{description}
    \item [rx_packets_allowance_exceeded]
        The number of the packets dropped by the device due to the received
        packets bitrate exceeding the \field{speed}.

    \item [rx_bytes_allowance_exceeded]
        The bytes of the packets dropped by the device due to the received
        packets bitrate exceeding the \field{speed}.

\end{description}

\subparagraph{Transmitq Speed Statistics}\label{sec:Device Types / Network Device / Device Operation / Control Virtqueue / Device Statistics / Transmitq Speed Statistics}

The structure corresponding to the transmitq speed statistics is
\field{struct virtio_net_stats_tx_speed}. The corresponding type is
VIRTIO_NET_STATS_TYPE_TX_SPEED. This is for the transmitq.

The device has the allowance for the speed. If VIRTIO_NET_F_SPEED_DUPLEX has
been negotiated, the driver can get this by \field{speed}. When the transmit
packets bitrate exceeds the \field{speed}, some packets may be dropped by the
device.

\begin{lstlisting}
struct virtio_net_stats_tx_speed {
    struct virtio_net_stats_reply_hdr hdr;

    le64 tx_packets_allowance_exceeded;
    le64 tx_bytes_allowance_exceeded;
};
\end{lstlisting}

The packets described below were all presented on the specified virtqueue.
\begin{description}
    \item [tx_packets_allowance_exceeded]
        The number of the packets dropped by the device due to the transmit packets
        bitrate exceeding the \field{speed}.

    \item [tx_bytes_allowance_exceeded]
        The bytes of the packets dropped by the device due to the transmit packets
        bitrate exceeding the \field{speed}.

\end{description}

\devicenormative{\subparagraph}{Device Statistics}{Device Types / Network Device / Device Operation / Control Virtqueue / Device Statistics}

When the VIRTIO_NET_F_DEVICE_STATS feature is negotiated, the device MUST reply
to the command VIRTIO_NET_CTRL_STATS_QUERY with the
\field{struct virtio_net_stats_capabilities}. \field{supported_stats_types}
includes all the statistic types supported by the device.

If \field{struct virtio_net_ctrl_queue_stats} is incorrect (such as the
following), the device MUST set \field{ack} to VIRTIO_NET_ERR. Even if there is
only one error, the device MUST fail the entire command.
\begin{itemize}
    \item \field{vq_index} exceeds the queue range.
    \item \field{types_bitmap} contains unknown types.
    \item One or more of the bits present in \field{types_bitmap} is not valid
        for the specified virtqueue.
    \item The feature corresponding to the specified \field{types_bitmap} was
        not negotiated.
\end{itemize}

The device MUST set the actual size of the bytes occupied by the reply to the
\field{size} of the \field{hdr}. And the device MUST set the \field{type} and
the \field{vq_index} of the statistic header.

The \field{command-specific-result} buffer allocated by the driver may be
smaller or bigger than all the statistics specified by
\field{struct virtio_net_ctrl_queue_stats}. The device MUST fill up only upto
the valid bytes.

The statistics counter replied by the device MUST wrap around to zero by the
device on the overflow.

\drivernormative{\subparagraph}{Device Statistics}{Device Types / Network Device / Device Operation / Control Virtqueue / Device Statistics}

The types contained in the \field{types_bitmap} MUST be queried from the device
via command VIRTIO_NET_CTRL_STATS_QUERY.

\field{types_bitmap} in \field{struct virtio_net_ctrl_queue_stats} MUST be valid to the
vq specified by \field{vq_index}.

The \field{command-specific-result} buffer allocated by the driver MUST have
enough capacity to store all the statistics reply headers defined in
\field{struct virtio_net_ctrl_queue_stats}. If the
\field{command-specific-result} buffer is fully utilized by the device but some
replies are missed, it is possible that some statistics may exceed the capacity
of the driver's records. In such cases, the driver should allocate additional
space for the \field{command-specific-result} buffer.

\subsubsection{Flow filter}\label{sec:Device Types / Network Device / Device Operation / Flow filter}

A network device can support one or more flow filter rules. Each flow filter rule
is applied by matching a packet and then taking an action, such as directing the packet
to a specific receiveq or dropping the packet. An example of a match is
matching on specific source and destination IP addresses.

A flow filter rule is a device resource object that consists of a key,
a processing priority, and an action to either direct a packet to a
receive queue or drop the packet.

Each rule uses a classifier. The key is matched against the packet using
a classifier, defining which fields in the packet are matched.
A classifier resource object consists of one or more field selectors, each with
a type that specifies the header fields to be matched against, and a mask.
The mask can match whole fields or parts of a field in a header. Each
rule resource object depends on the classifier resource object.

When a packet is received, relevant fields are extracted
(in the same way) from both the packet and the key according to the
classifier. The resulting field contents are then compared -
if they are identical the rule action is taken, if they are not, the rule is ignored.

Multiple flow filter rules are part of a group. The rule resource object
depends on the group. Each rule within a
group has a rule priority, and each group also has a group priority. For a
packet, a group with the highest priority is selected first. Within a group,
rules are applied from highest to lowest priority, until one of the rules
matches the packet and an action is taken. If all the rules within a group
are ignored, the group with the next highest priority is selected, and so on.

The device and the driver indicates flow filter resource limits using the capability
\ref{par:Device Types / Network Device / Device Operation / Flow filter / Device and driver capabilities / VIRTIO-NET-FF-RESOURCE-CAP} specifying the limits on the number of flow filter rule,
group and classifier resource objects. The capability \ref{par:Device Types / Network Device / Device Operation / Flow filter / Device and driver capabilities / VIRTIO-NET-FF-SELECTOR-CAP} specifies which selectors the device supports.
The driver indicates the selectors it is using by setting the flow
filter selector capability, prior to adding any resource objects.

The capability \ref{par:Device Types / Network Device / Device Operation / Flow filter / Device and driver capabilities / VIRTIO-NET-FF-ACTION-CAP} specifies which actions the device supports.

The driver controls the flow filter rule, classifier and group resource objects using
administration commands described in
\ref{sec:Basic Facilities of a Virtio Device / Device groups / Group administration commands / Device resource objects}.

\paragraph{Packet processing order}\label{sec:sec:Device Types / Network Device / Device Operation / Flow filter / Packet processing order}

Note that flow filter rules are applied after MAC/VLAN filtering. Flow filter
rules take precedence over steering: if a flow filter rule results in an action,
the steering configuration does not apply. The steering configuration only applies
to packets for which no flow filter rule action was performed. For example,
incoming packets can be processed in the following order:

\begin{itemize}
\item apply steering configuration received using control virtqueue commands
      VIRTIO_NET_CTRL_RX, VIRTIO_NET_CTRL_MAC and VIRTIO_NET_CTRL_VLAN.
\item apply flow filter rules if any.
\item if no filter rule applied, apply steering configuration received using command
      VIRTIO_NET_CTRL_MQ_RSS_CONFIG or as per automatic receive steering.
\end{itemize}

Some incoming packet processing examples:
\begin{itemize}
\item If the packet is dropped by the flow filter rule, RSS
      steering is ignored for the packet.
\item If the packet is directed to a specific receiveq using flow filter rule,
      the RSS steering is ignored for the packet.
\item If a packet is dropped due to the VIRTIO_NET_CTRL_MAC configuration,
      both flow filter rules and the RSS steering are ignored for the packet.
\item If a packet does not match any flow filter rules,
      the RSS steering is used to select the receiveq for the packet (if enabled).
\item If there are two flow filter groups configured as group_A and group_B
      with respective group priorities as 4, and 5; flow filter rules of
      group_B are applied first having highest group priority, if there is a match,
      the flow filter rules of group_A are ignored; if there is no match for
      the flow filter rules in group_B, the flow filter rules of next level group_A are applied.
\end{itemize}

\paragraph{Device and driver capabilities}
\label{par:Device Types / Network Device / Device Operation / Flow filter / Device and driver capabilities}

\subparagraph{VIRTIO_NET_FF_RESOURCE_CAP}
\label{par:Device Types / Network Device / Device Operation / Flow filter / Device and driver capabilities / VIRTIO-NET-FF-RESOURCE-CAP}

The capability VIRTIO_NET_FF_RESOURCE_CAP indicates the flow filter resource limits.
\field{cap_specific_data} is in the format
\field{struct virtio_net_ff_cap_data}.

\begin{lstlisting}
struct virtio_net_ff_cap_data {
        le32 groups_limit;
        le32 selectors_limit;
        le32 rules_limit;
        le32 rules_per_group_limit;
        u8 last_rule_priority;
        u8 selectors_per_classifier_limit;
};
\end{lstlisting}

\field{groups_limit}, and \field{selectors_limit} represent the maximum
number of flow filter groups and selectors, respectively, that the driver can create.
 \field{rules_limit} is the maximum number of
flow fiilter rules that the driver can create across all the groups.
\field{rules_per_group_limit} is the maximum number of flow filter rules that the driver
can create for each flow filter group.

\field{last_rule_priority} is the highest priority that can be assigned to a
flow filter rule.

\field{selectors_per_classifier_limit} is the maximum number of selectors
that a classifier can have.

\subparagraph{VIRTIO_NET_FF_SELECTOR_CAP}
\label{par:Device Types / Network Device / Device Operation / Flow filter / Device and driver capabilities / VIRTIO-NET-FF-SELECTOR-CAP}

The capability VIRTIO_NET_FF_SELECTOR_CAP lists the supported selectors and the
supported packet header fields for each selector.
\field{cap_specific_data} is in the format \field{struct virtio_net_ff_cap_mask_data}.

\begin{lstlisting}[label={lst:Device Types / Network Device / Device Operation / Flow filter / Device and driver capabilities / VIRTIO-NET-FF-SELECTOR-CAP / virtio-net-ff-selector}]
struct virtio_net_ff_selector {
        u8 type;
        u8 flags;
        u8 reserved[2];
        u8 length;
        u8 reserved1[3];
        u8 mask[];
};

struct virtio_net_ff_cap_mask_data {
        u8 count;
        u8 reserved[7];
        struct virtio_net_ff_selector selectors[];
};

#define VIRTIO_NET_FF_MASK_F_PARTIAL_MASK (1 << 0)
\end{lstlisting}

\field{count} indicates number of valid entries in the \field{selectors} array.
\field{selectors[]} is an array of supported selectors. Within each array entry:
\field{type} specifies the type of the packet header, as defined in table
\ref{table:Device Types / Network Device / Device Operation / Flow filter / Device and driver capabilities / VIRTIO-NET-FF-SELECTOR-CAP / flow filter selector types}. \field{mask} specifies which fields of the
packet header can be matched in a flow filter rule.

Each \field{type} is also listed in table
\ref{table:Device Types / Network Device / Device Operation / Flow filter / Device and driver capabilities / VIRTIO-NET-FF-SELECTOR-CAP / flow filter selector types}. \field{mask} is a byte array
in network byte order. For example, when \field{type} is VIRTIO_NET_FF_MASK_TYPE_IPV6,
the \field{mask} is in the format \hyperref[intro:IPv6-Header-Format]{IPv6 Header Format}.

If partial masking is not set, then all bits in each field have to be either all 0s
to ignore this field or all 1s to match on this field. If partial masking is set,
then any combination of bits can bit set to match on these bits.
For example, when a selector \field{type} is VIRTIO_NET_FF_MASK_TYPE_ETH, if
\field{mask[0-12]} are zero and \field{mask[13-14]} are 0xff (all 1s), it
indicates that matching is only supported for \field{EtherType} of
\field{Ethernet MAC frame}, matching is not supported for
\field{Destination Address} and \field{Source Address}.

The entries in the array \field{selectors} are ordered by
\field{type}, with each \field{type} value only appearing once.

\field{length} is the length of a dynamic array \field{mask} in bytes.
\field{reserved} and \field{reserved1} are reserved and set to zero.

\begin{table}[H]
\caption{Flow filter selector types}
\label{table:Device Types / Network Device / Device Operation / Flow filter / Device and driver capabilities / VIRTIO-NET-FF-SELECTOR-CAP / flow filter selector types}
\begin{tabularx}{\textwidth}{ |l|X|X| }
\hline
Type & Name & Description \\
\hline \hline
0x0 & - & Reserved \\
\hline
0x1 & VIRTIO_NET_FF_MASK_TYPE_ETH & 14 bytes of frame header starting from destination address described in \hyperref[intro:IEEE 802.3-2022]{IEEE 802.3-2022} \\
\hline
0x2 & VIRTIO_NET_FF_MASK_TYPE_IPV4 & 20 bytes of \hyperref[intro:Internet-Header-Format]{IPv4: Internet Header Format} \\
\hline
0x3 & VIRTIO_NET_FF_MASK_TYPE_IPV6 & 40 bytes of \hyperref[intro:IPv6-Header-Format]{IPv6 Header Format} \\
\hline
0x4 & VIRTIO_NET_FF_MASK_TYPE_TCP & 20 bytes of \hyperref[intro:TCP-Header-Format]{TCP Header Format} \\
\hline
0x5 & VIRTIO_NET_FF_MASK_TYPE_UDP & 8 bytes of UDP header described in \hyperref[intro:UDP]{UDP} \\
\hline
0x6 - 0xFF & & Reserved for future \\
\hline
\end{tabularx}
\end{table}

When VIRTIO_NET_FF_MASK_F_PARTIAL_MASK (bit 0) is set, it indicates that
partial masking is supported for all the fields of the selector identified by \field{type}.

For the selector \field{type} VIRTIO_NET_FF_MASK_TYPE_IPV4, if a partial mask is unsupported,
then matching on an individual bit of \field{Flags} in the
\field{IPv4: Internet Header Format} is unsupported. \field{Flags} has to match as a whole
if it is supported.

For the selector \field{type} VIRTIO_NET_FF_MASK_TYPE_IPV4, \field{mask} includes fields
up to the \field{Destination Address}; that is, \field{Options} and
\field{Padding} are excluded.

For the selector \field{type} VIRTIO_NET_FF_MASK_TYPE_IPV6, the \field{Next Header} field
of the \field{mask} corresponds to the \field{Next Header} in the packet
when \field{IPv6 Extension Headers} are not present. When the packet includes
one or more \field{IPv6 Extension Headers}, the \field{Next Header} field of
the \field{mask} corresponds to the \field{Next Header} of the last
\field{IPv6 Extension Header} in the packet.

For the selector \field{type} VIRTIO_NET_FF_MASK_TYPE_TCP, \field{Control bits}
are treated as individual fields for matching; that is, matching individual
\field{Control bits} does not depend on the partial mask support.

\subparagraph{VIRTIO_NET_FF_ACTION_CAP}
\label{par:Device Types / Network Device / Device Operation / Flow filter / Device and driver capabilities / VIRTIO-NET-FF-ACTION-CAP}

The capability VIRTIO_NET_FF_ACTION_CAP lists the supported actions in a rule.
\field{cap_specific_data} is in the format \field{struct virtio_net_ff_cap_actions}.

\begin{lstlisting}
struct virtio_net_ff_actions {
        u8 count;
        u8 reserved[7];
        u8 actions[];
};
\end{lstlisting}

\field{actions} is an array listing all possible actions.
The entries in the array are ordered from the smallest to the largest,
with each supported value appearing exactly once. Each entry can have the
following values:

\begin{table}[H]
\caption{Flow filter rule actions}
\label{table:Device Types / Network Device / Device Operation / Flow filter / Device and driver capabilities / VIRTIO-NET-FF-ACTION-CAP / flow filter rule actions}
\begin{tabularx}{\textwidth}{ |l|X|X| }
\hline
Action & Name & Description \\
\hline \hline
0x0 & - & reserved \\
\hline
0x1 & VIRTIO_NET_FF_ACTION_DROP & Matching packet will be dropped by the device \\
\hline
0x2 & VIRTIO_NET_FF_ACTION_DIRECT_RX_VQ & Matching packet will be directed to a receive queue \\
\hline
0x3 - 0xFF & & Reserved for future \\
\hline
\end{tabularx}
\end{table}

\paragraph{Resource objects}
\label{par:Device Types / Network Device / Device Operation / Flow filter / Resource objects}

\subparagraph{VIRTIO_NET_RESOURCE_OBJ_FF_GROUP}\label{par:Device Types / Network Device / Device Operation / Flow filter / Resource objects / VIRTIO-NET-RESOURCE-OBJ-FF-GROUP}

A flow filter group contains between 0 and \field{rules_limit} rules, as specified by the
capability VIRTIO_NET_FF_RESOURCE_CAP. For the flow filter group object both
\field{resource_obj_specific_data} and
\field{resource_obj_specific_result} are in the format
\field{struct virtio_net_resource_obj_ff_group}.

\begin{lstlisting}
struct virtio_net_resource_obj_ff_group {
        le16 group_priority;
};
\end{lstlisting}

\field{group_priority} specifies the priority for the group. Each group has a
distinct priority. For each incoming packet, the device tries to apply rules
from groups from higher \field{group_priority} value to lower, until either a
rule matches the packet or all groups have been tried.

\subparagraph{VIRTIO_NET_RESOURCE_OBJ_FF_CLASSIFIER}\label{par:Device Types / Network Device / Device Operation / Flow filter / Resource objects / VIRTIO-NET-RESOURCE-OBJ-FF-CLASSIFIER}

A classifier is used to match a flow filter key against a packet. The
classifier defines the desired packet fields to match, and is represented by
the VIRTIO_NET_RESOURCE_OBJ_FF_CLASSIFIER device resource object.

For the flow filter classifier object both \field{resource_obj_specific_data} and
\field{resource_obj_specific_result} are in the format
\field{struct virtio_net_resource_obj_ff_classifier}.

\begin{lstlisting}
struct virtio_net_resource_obj_ff_classifier {
        u8 count;
        u8 reserved[7];
        struct virtio_net_ff_selector selectors[];
};
\end{lstlisting}

A classifier is an array of \field{selectors}. The number of selectors in the
array is indicated by \field{count}. The selector has a type that specifies
the header fields to be matched against, and a mask.
See \ref{lst:Device Types / Network Device / Device Operation / Flow filter / Device and driver capabilities / VIRTIO-NET-FF-SELECTOR-CAP / virtio-net-ff-selector}
for details about selectors.

The first selector is always VIRTIO_NET_FF_MASK_TYPE_ETH. When there are multiple
selectors, a second selector can be either VIRTIO_NET_FF_MASK_TYPE_IPV4
or VIRTIO_NET_FF_MASK_TYPE_IPV6. If the third selector exists, the third
selector can be either VIRTIO_NET_FF_MASK_TYPE_UDP or VIRTIO_NET_FF_MASK_TYPE_TCP.
For example, to match a Ethernet IPv6 UDP packet,
\field{selectors[0].type} is set to VIRTIO_NET_FF_MASK_TYPE_ETH, \field{selectors[1].type}
is set to VIRTIO_NET_FF_MASK_TYPE_IPV6 and \field{selectors[2].type} is
set to VIRTIO_NET_FF_MASK_TYPE_UDP; accordingly, \field{selectors[0].mask[0-13]} is
for Ethernet header fields, \field{selectors[1].mask[0-39]} is set for IPV6 header
and \field{selectors[2].mask[0-7]} is set for UDP header.

When there are multiple selectors, the type of the (N+1)\textsuperscript{th} selector
affects the mask of the (N)\textsuperscript{th} selector. If
\field{count} is 2 or more, all the mask bits within \field{selectors[0]}
corresponding to \field{EtherType} of an Ethernet header are set.

If \field{count} is more than 2:
\begin{itemize}
\item if \field{selector[1].type} is, VIRTIO_NET_FF_MASK_TYPE_IPV4, then, all the mask bits within
\field{selector[1]} for \field{Protocol} is set.
\item if \field{selector[1].type} is, VIRTIO_NET_FF_MASK_TYPE_IPV6, then, all the mask bits within
\field{selector[1]} for \field{Next Header} is set.
\end{itemize}

If for a given packet header field, a subset of bits of a field is to be matched,
and if the partial mask is supported, the flow filter
mask object can specify a mask which has fewer bits set than the packet header
field size. For example, a partial mask for the Ethernet header source mac
address can be of 1-bit for multicast detection instead of 48-bits.

\subparagraph{VIRTIO_NET_RESOURCE_OBJ_FF_RULE}\label{par:Device Types / Network Device / Device Operation / Flow filter / Resource objects / VIRTIO-NET-RESOURCE-OBJ-FF-RULE}

Each flow filter rule resource object comprises a key, a priority, and an action.
For the flow filter rule object,
\field{resource_obj_specific_data} and
\field{resource_obj_specific_result} are in the format
\field{struct virtio_net_resource_obj_ff_rule}.

\begin{lstlisting}
struct virtio_net_resource_obj_ff_rule {
        le32 group_id;
        le32 classifier_id;
        u8 rule_priority;
        u8 key_length; /* length of key in bytes */
        u8 action;
        u8 reserved;
        le16 vq_index;
        u8 reserved1[2];
        u8 keys[][];
};
\end{lstlisting}

\field{group_id} is the resource object ID of the flow filter group to which
this rule belongs. \field{classifier_id} is the resource object ID of the
classifier used to match a packet against the \field{key}.

\field{rule_priority} denotes the priority of the rule within the group
specified by the \field{group_id}.
Rules within the group are applied from the highest to the lowest priority
until a rule matches the packet and an
action is taken. Rules with the same priority can be applied in any order.

\field{reserved} and \field{reserved1} are reserved and set to 0.

\field{keys[][]} is an array of keys to match against packets, using
the classifier specified by \field{classifier_id}. Each entry (key) comprises
a byte array, and they are located one immediately after another.
The size (number of entries) of the array is exactly the same as that of
\field{selectors} in the classifier, or in other words, \field{count}
in the classifier.

\field{key_length} specifies the total length of \field{keys} in bytes.
In other words, it equals the sum total of \field{length} of all
selectors in \field{selectors} in the classifier specified by
\field{classifier_id}.

For example, if a classifier object's \field{selectors[0].type} is
VIRTIO_NET_FF_MASK_TYPE_ETH and \field{selectors[1].type} is
VIRTIO_NET_FF_MASK_TYPE_IPV6,
then selectors[0].length is 14 and selectors[1].length is 40.
Accordingly, the \field{key_length} is set to 54.
This setting indicates that the \field{key} array's length is 54 bytes
comprising a first byte array of 14 bytes for the
Ethernet MAC header in bytes 0-13, immediately followed by 40 bytes for the
IPv6 header in bytes 14-53.

When there are multiple selectors in the classifier object, the key bytes
for (N)\textsuperscript{th} selector are set so that
(N+1)\textsuperscript{th} selector can be matched.

If \field{count} is 2 or more, key bytes of \field{EtherType}
are set according to \hyperref[intro:IEEE 802 Ethertypes]{IEEE 802 Ethertypes}
for VIRTIO_NET_FF_MASK_TYPE_IPV4 or VIRTIO_NET_FF_MASK_TYPE_IPV6 respectively.

If \field{count} is more than 2, when \field{selector[1].type} is
VIRTIO_NET_FF_MASK_TYPE_IPV4 or VIRTIO_NET_FF_MASK_TYPE_IPV6, key
bytes of \field{Protocol} or \field{Next Header} is set as per
\field{Protocol Numbers} defined \hyperref[intro:IANA Protocol Numbers]{IANA Protocol Numbers}
respectively.

\field{action} is the action to take when a packet matches the
\field{key} using the \field{classifier_id}. Supported actions are described in
\ref{table:Device Types / Network Device / Device Operation / Flow filter / Device and driver capabilities / VIRTIO-NET-FF-ACTION-CAP / flow filter rule actions}.

\field{vq_index} specifies a receive virtqueue. When the \field{action} is set
to VIRTIO_NET_FF_ACTION_DIRECT_RX_VQ, and the packet matches the \field{key},
the matching packet is directed to this virtqueue.

Note that at most one action is ever taken for a given packet. If a rule is
applied and an action is taken, the action of other rules is not taken.

\devicenormative{\paragraph}{Flow filter}{Device Types / Network Device / Device Operation / Flow filter}

When the device supports flow filter operations,
\begin{itemize}
\item the device MUST set VIRTIO_NET_FF_RESOURCE_CAP, VIRTIO_NET_FF_SELECTOR_CAP
and VIRTIO_NET_FF_ACTION_CAP capability in the \field{supported_caps} in the
command VIRTIO_ADMIN_CMD_CAP_SUPPORT_QUERY.
\item the device MUST support the administration commands
VIRTIO_ADMIN_CMD_RESOURCE_OBJ_CREATE,
VIRTIO_ADMIN_CMD_RESOURCE_OBJ_MODIFY, VIRTIO_ADMIN_CMD_RESOURCE_OBJ_QUERY,
VIRTIO_ADMIN_CMD_RESOURCE_OBJ_DESTROY for the resource types
VIRTIO_NET_RESOURCE_OBJ_FF_GROUP, VIRTIO_NET_RESOURCE_OBJ_FF_CLASSIFIER and
VIRTIO_NET_RESOURCE_OBJ_FF_RULE.
\end{itemize}

When any of the VIRTIO_NET_FF_RESOURCE_CAP, VIRTIO_NET_FF_SELECTOR_CAP, or
VIRTIO_NET_FF_ACTION_CAP capability is disabled, the device SHOULD set
\field{status} to VIRTIO_ADMIN_STATUS_Q_INVALID_OPCODE for the commands
VIRTIO_ADMIN_CMD_RESOURCE_OBJ_CREATE,
VIRTIO_ADMIN_CMD_RESOURCE_OBJ_MODIFY, VIRTIO_ADMIN_CMD_RESOURCE_OBJ_QUERY,
and VIRTIO_ADMIN_CMD_RESOURCE_OBJ_DESTROY. These commands apply to the resource
\field{type} of VIRTIO_NET_RESOURCE_OBJ_FF_GROUP, VIRTIO_NET_RESOURCE_OBJ_FF_CLASSIFIER, and
VIRTIO_NET_RESOURCE_OBJ_FF_RULE.

The device SHOULD set \field{status} to VIRTIO_ADMIN_STATUS_EINVAL for the
command VIRTIO_ADMIN_CMD_RESOURCE_OBJ_CREATE when the resource \field{type}
is VIRTIO_NET_RESOURCE_OBJ_FF_GROUP, if a flow filter group already exists
with the supplied \field{group_priority}.

The device SHOULD set \field{status} to VIRTIO_ADMIN_STATUS_ENOSPC for the
command VIRTIO_ADMIN_CMD_RESOURCE_OBJ_CREATE when the resource \field{type}
is VIRTIO_NET_RESOURCE_OBJ_FF_GROUP, if the number of flow filter group
objects in the device exceeds the lower of the configured driver
capabilities \field{groups_limit} and \field{rules_per_group_limit}.

The device SHOULD set \field{status} to VIRTIO_ADMIN_STATUS_ENOSPC for the
command VIRTIO_ADMIN_CMD_RESOURCE_OBJ_CREATE when the resource \field{type} is
VIRTIO_NET_RESOURCE_OBJ_FF_CLASSIFIER, if the number of flow filter selector
objects in the device exceeds the configured driver capability
\field{selectors_limit}.

The device SHOULD set \field{status} to VIRTIO_ADMIN_STATUS_EBUSY for the
command VIRTIO_ADMIN_CMD_RESOURCE_OBJ_DESTROY for a flow filter group when
the flow filter group has one or more flow filter rules depending on it.

The device SHOULD set \field{status} to VIRTIO_ADMIN_STATUS_EBUSY for the
command VIRTIO_ADMIN_CMD_RESOURCE_OBJ_DESTROY for a flow filter classifier when
the flow filter classifier has one or more flow filter rules depending on it.

The device SHOULD fail the command VIRTIO_ADMIN_CMD_RESOURCE_OBJ_CREATE for the
flow filter rule resource object if,
\begin{itemize}
\item \field{vq_index} is not a valid receive virtqueue index for
the VIRTIO_NET_FF_ACTION_DIRECT_RX_VQ action,
\item \field{priority} is greater than or equal to
      \field{last_rule_priority},
\item \field{id} is greater than or equal to \field{rules_limit} or
      greater than or equal to \field{rules_per_group_limit}, whichever is lower,
\item the length of \field{keys} and the length of all the mask bytes of
      \field{selectors[].mask} as referred by \field{classifier_id} differs,
\item the supplied \field{action} is not supported in the capability VIRTIO_NET_FF_ACTION_CAP.
\end{itemize}

When the flow filter directs a packet to the virtqueue identified by
\field{vq_index} and if the receive virtqueue is reset, the device
MUST drop such packets.

Upon applying a flow filter rule to a packet, the device MUST STOP any further
application of rules and cease applying any other steering configurations.

For multiple flow filter groups, the device MUST apply the rules from
the group with the highest priority. If any rule from this group is applied,
the device MUST ignore the remaining groups. If none of the rules from the
highest priority group match, the device MUST apply the rules from
the group with the next highest priority, until either a rule matches or
all groups have been attempted.

The device MUST apply the rules within the group from the highest to the
lowest priority until a rule matches the packet, and the device MUST take
the action. If an action is taken, the device MUST not take any other
action for this packet.

The device MAY apply the rules with the same \field{rule_priority} in any
order within the group.

The device MUST process incoming packets in the following order:
\begin{itemize}
\item apply the steering configuration received using control virtqueue
      commands VIRTIO_NET_CTRL_RX, VIRTIO_NET_CTRL_MAC, and
      VIRTIO_NET_CTRL_VLAN.
\item apply flow filter rules if any.
\item if no filter rule is applied, apply the steering configuration
      received using the command VIRTIO_NET_CTRL_MQ_RSS_CONFIG
      or according to automatic receive steering.
\end{itemize}

When processing an incoming packet, if the packet is dropped at any stage, the device
MUST skip further processing.

When the device drops the packet due to the configuration done using the control
virtqueue commands VIRTIO_NET_CTRL_RX or VIRTIO_NET_CTRL_MAC or VIRTIO_NET_CTRL_VLAN,
the device MUST skip flow filter rules for this packet.

When the device performs flow filter match operations and if the operation
result did not have any match in all the groups, the receive packet processing
continues to next level, i.e. to apply configuration done using
VIRTIO_NET_CTRL_MQ_RSS_CONFIG command.

The device MUST support the creation of flow filter classifier objects
using the command VIRTIO_ADMIN_CMD_RESOURCE_OBJ_CREATE with \field{flags}
set to VIRTIO_NET_FF_MASK_F_PARTIAL_MASK;
this support is required even if all the bits of the masks are set for
a field in \field{selectors}, provided that partial masking is supported
for the selectors.

\drivernormative{\paragraph}{Flow filter}{Device Types / Network Device / Device Operation / Flow filter}

The driver MUST enable VIRTIO_NET_FF_RESOURCE_CAP, VIRTIO_NET_FF_SELECTOR_CAP,
and VIRTIO_NET_FF_ACTION_CAP capabilities to use flow filter.

The driver SHOULD NOT remove a flow filter group using the command
VIRTIO_ADMIN_CMD_RESOURCE_OBJ_DESTROY when one or more flow filter rules
depend on that group. The driver SHOULD only destroy the group after
all the associated rules have been destroyed.

The driver SHOULD NOT remove a flow filter classifier using the command
VIRTIO_ADMIN_CMD_RESOURCE_OBJ_DESTROY when one or more flow filter rules
depend on the classifier. The driver SHOULD only destroy the classifier
after all the associated rules have been destroyed.

The driver SHOULD NOT add multiple flow filter rules with the same
\field{rule_priority} within a flow filter group, as these rules MAY match
the same packet. The driver SHOULD assign different \field{rule_priority}
values to different flow filter rules if multiple rules may match a single
packet.

For the command VIRTIO_ADMIN_CMD_RESOURCE_OBJ_CREATE, when creating a resource
of \field{type} VIRTIO_NET_RESOURCE_OBJ_FF_CLASSIFIER, the driver MUST set:
\begin{itemize}
\item \field{selectors[0].type} to VIRTIO_NET_FF_MASK_TYPE_ETH.
\item \field{selectors[1].type} to VIRTIO_NET_FF_MASK_TYPE_IPV4 or
      VIRTIO_NET_FF_MASK_TYPE_IPV6 when \field{count} is more than 1,
\item \field{selectors[2].type} VIRTIO_NET_FF_MASK_TYPE_UDP or
      VIRTIO_NET_FF_MASK_TYPE_TCP when \field{count} is more than 2.
\end{itemize}

For the command VIRTIO_ADMIN_CMD_RESOURCE_OBJ_CREATE, when creating a resource
of \field{type} VIRTIO_NET_RESOURCE_OBJ_FF_CLASSIFIER, the driver MUST set:
\begin{itemize}
\item \field{selectors[0].mask} bytes to all 1s for the \field{EtherType}
       when \field{count} is 2 or more.
\item \field{selectors[1].mask} bytes to all 1s for \field{Protocol} or \field{Next Header}
       when \field{selector[1].type} is VIRTIO_NET_FF_MASK_TYPE_IPV4 or VIRTIO_NET_FF_MASK_TYPE_IPV6,
       and when \field{count} is more than 2.
\end{itemize}

For the command VIRTIO_ADMIN_CMD_RESOURCE_OBJ_CREATE, the resource \field{type}
VIRTIO_NET_RESOURCE_OBJ_FF_RULE, if the corresponding classifier object's
\field{count} is 2 or more, the driver MUST SET the \field{keys} bytes of
\field{EtherType} in accordance with
\hyperref[intro:IEEE 802 Ethertypes]{IEEE 802 Ethertypes}
for either VIRTIO_NET_FF_MASK_TYPE_IPV4 or VIRTIO_NET_FF_MASK_TYPE_IPV6.

For the command VIRTIO_ADMIN_CMD_RESOURCE_OBJ_CREATE, when creating a resource of
\field{type} VIRTIO_NET_RESOURCE_OBJ_FF_RULE, if the corresponding classifier
object's \field{count} is more than 2, and the \field{selector[1].type} is either
VIRTIO_NET_FF_MASK_TYPE_IPV4 or VIRTIO_NET_FF_MASK_TYPE_IPV6, the driver MUST
set the \field{keys} bytes for the \field{Protocol} or \field{Next Header}
according to \hyperref[intro:IANA Protocol Numbers]{IANA Protocol Numbers} respectively.

The driver SHOULD set all the bits for a field in the mask of a selector in both the
capability and the classifier object, unless the VIRTIO_NET_FF_MASK_F_PARTIAL_MASK
is enabled.

\subsubsection{Legacy Interface: Framing Requirements}\label{sec:Device
Types / Network Device / Legacy Interface: Framing Requirements}

When using legacy interfaces, transitional drivers which have not
negotiated VIRTIO_F_ANY_LAYOUT MUST use a single descriptor for the
\field{struct virtio_net_hdr} on both transmit and receive, with the
network data in the following descriptors.

Additionally, when using the control virtqueue (see \ref{sec:Device
Types / Network Device / Device Operation / Control Virtqueue})
, transitional drivers which have not
negotiated VIRTIO_F_ANY_LAYOUT MUST:
\begin{itemize}
\item for all commands, use a single 2-byte descriptor including the first two
fields: \field{class} and \field{command}
\item for all commands except VIRTIO_NET_CTRL_MAC_TABLE_SET
use a single descriptor including command-specific-data
with no padding.
\item for the VIRTIO_NET_CTRL_MAC_TABLE_SET command use exactly
two descriptors including command-specific-data with no padding:
the first of these descriptors MUST include the
virtio_net_ctrl_mac table structure for the unicast addresses with no padding,
the second of these descriptors MUST include the
virtio_net_ctrl_mac table structure for the multicast addresses
with no padding.
\item for all commands, use a single 1-byte descriptor for the
\field{ack} field
\end{itemize}

See \ref{sec:Basic
Facilities of a Virtio Device / Virtqueues / Message Framing}.

\section{Network Device}\label{sec:Device Types / Network Device}

The virtio network device is a virtual network interface controller.
It consists of a virtual Ethernet link which connects the device
to the Ethernet network. The device has transmit and receive
queues. The driver adds empty buffers to the receive virtqueue.
The device receives incoming packets from the link; the device
places these incoming packets in the receive virtqueue buffers.
The driver adds outgoing packets to the transmit virtqueue. The device
removes these packets from the transmit virtqueue and sends them to
the link. The device may have a control virtqueue. The driver
uses the control virtqueue to dynamically manipulate various
features of the initialized device.

\subsection{Device ID}\label{sec:Device Types / Network Device / Device ID}

 1

\subsection{Virtqueues}\label{sec:Device Types / Network Device / Virtqueues}

\begin{description}
\item[0] receiveq1
\item[1] transmitq1
\item[\ldots]
\item[2(N-1)] receiveqN
\item[2(N-1)+1] transmitqN
\item[2N] controlq
\end{description}

 N=1 if neither VIRTIO_NET_F_MQ nor VIRTIO_NET_F_RSS are negotiated, otherwise N is set by
 \field{max_virtqueue_pairs}.

controlq is optional; it only exists if VIRTIO_NET_F_CTRL_VQ is
negotiated.

\subsection{Feature bits}\label{sec:Device Types / Network Device / Feature bits}

\begin{description}
\item[VIRTIO_NET_F_CSUM (0)] Device handles packets with partial checksum offload.

\item[VIRTIO_NET_F_GUEST_CSUM (1)] Driver handles packets with partial checksum.

\item[VIRTIO_NET_F_CTRL_GUEST_OFFLOADS (2)] Control channel offloads
        reconfiguration support.

\item[VIRTIO_NET_F_MTU(3)] Device maximum MTU reporting is supported. If
    offered by the device, device advises driver about the value of
    its maximum MTU. If negotiated, the driver uses \field{mtu} as
    the maximum MTU value.

\item[VIRTIO_NET_F_MAC (5)] Device has given MAC address.

\item[VIRTIO_NET_F_GUEST_TSO4 (7)] Driver can receive TSOv4.

\item[VIRTIO_NET_F_GUEST_TSO6 (8)] Driver can receive TSOv6.

\item[VIRTIO_NET_F_GUEST_ECN (9)] Driver can receive TSO with ECN.

\item[VIRTIO_NET_F_GUEST_UFO (10)] Driver can receive UFO.

\item[VIRTIO_NET_F_HOST_TSO4 (11)] Device can receive TSOv4.

\item[VIRTIO_NET_F_HOST_TSO6 (12)] Device can receive TSOv6.

\item[VIRTIO_NET_F_HOST_ECN (13)] Device can receive TSO with ECN.

\item[VIRTIO_NET_F_HOST_UFO (14)] Device can receive UFO.

\item[VIRTIO_NET_F_MRG_RXBUF (15)] Driver can merge receive buffers.

\item[VIRTIO_NET_F_STATUS (16)] Configuration status field is
    available.

\item[VIRTIO_NET_F_CTRL_VQ (17)] Control channel is available.

\item[VIRTIO_NET_F_CTRL_RX (18)] Control channel RX mode support.

\item[VIRTIO_NET_F_CTRL_VLAN (19)] Control channel VLAN filtering.

\item[VIRTIO_NET_F_CTRL_RX_EXTRA (20)]	Control channel RX extra mode support.

\item[VIRTIO_NET_F_GUEST_ANNOUNCE(21)] Driver can send gratuitous
    packets.

\item[VIRTIO_NET_F_MQ(22)] Device supports multiqueue with automatic
    receive steering.

\item[VIRTIO_NET_F_CTRL_MAC_ADDR(23)] Set MAC address through control
    channel.

\item[VIRTIO_NET_F_DEVICE_STATS(50)] Device can provide device-level statistics
    to the driver through the control virtqueue.

\item[VIRTIO_NET_F_HASH_TUNNEL(51)] Device supports inner header hash for encapsulated packets.

\item[VIRTIO_NET_F_VQ_NOTF_COAL(52)] Device supports virtqueue notification coalescing.

\item[VIRTIO_NET_F_NOTF_COAL(53)] Device supports notifications coalescing.

\item[VIRTIO_NET_F_GUEST_USO4 (54)] Driver can receive USOv4 packets.

\item[VIRTIO_NET_F_GUEST_USO6 (55)] Driver can receive USOv6 packets.

\item[VIRTIO_NET_F_HOST_USO (56)] Device can receive USO packets. Unlike UFO
 (fragmenting the packet) the USO splits large UDP packet
 to several segments when each of these smaller packets has UDP header.

\item[VIRTIO_NET_F_HASH_REPORT(57)] Device can report per-packet hash
    value and a type of calculated hash.

\item[VIRTIO_NET_F_GUEST_HDRLEN(59)] Driver can provide the exact \field{hdr_len}
    value. Device benefits from knowing the exact header length.

\item[VIRTIO_NET_F_RSS(60)] Device supports RSS (receive-side scaling)
    with Toeplitz hash calculation and configurable hash
    parameters for receive steering.

\item[VIRTIO_NET_F_RSC_EXT(61)] Device can process duplicated ACKs
    and report number of coalesced segments and duplicated ACKs.

\item[VIRTIO_NET_F_STANDBY(62)] Device may act as a standby for a primary
    device with the same MAC address.

\item[VIRTIO_NET_F_SPEED_DUPLEX(63)] Device reports speed and duplex.

\item[VIRTIO_NET_F_RSS_CONTEXT(64)] Device supports multiple RSS contexts.

\item[VIRTIO_NET_F_GUEST_UDP_TUNNEL_GSO (65)] Driver can receive GSO packets
  carried by a UDP tunnel.

\item[VIRTIO_NET_F_GUEST_UDP_TUNNEL_GSO_CSUM (66)] Driver handles packets
  carried by a UDP tunnel with partial csum for the outer header.

\item[VIRTIO_NET_F_HOST_UDP_TUNNEL_GSO (67)] Device can receive GSO packets
  carried by a UDP tunnel.

\item[VIRTIO_NET_F_HOST_UDP_TUNNEL_GSO_CSUM (68)] Device handles packets
  carried by a UDP tunnel with partial csum for the outer header.
\end{description}

\subsubsection{Feature bit requirements}\label{sec:Device Types / Network Device / Feature bits / Feature bit requirements}

Some networking feature bits require other networking feature bits
(see \ref{drivernormative:Basic Facilities of a Virtio Device / Feature Bits}):

\begin{description}
\item[VIRTIO_NET_F_GUEST_TSO4] Requires VIRTIO_NET_F_GUEST_CSUM.
\item[VIRTIO_NET_F_GUEST_TSO6] Requires VIRTIO_NET_F_GUEST_CSUM.
\item[VIRTIO_NET_F_GUEST_ECN] Requires VIRTIO_NET_F_GUEST_TSO4 or VIRTIO_NET_F_GUEST_TSO6.
\item[VIRTIO_NET_F_GUEST_UFO] Requires VIRTIO_NET_F_GUEST_CSUM.
\item[VIRTIO_NET_F_GUEST_USO4] Requires VIRTIO_NET_F_GUEST_CSUM.
\item[VIRTIO_NET_F_GUEST_USO6] Requires VIRTIO_NET_F_GUEST_CSUM.
\item[VIRTIO_NET_F_GUEST_UDP_TUNNEL_GSO] Requires VIRTIO_NET_F_GUEST_TSO4, VIRTIO_NET_F_GUEST_TSO6,
   VIRTIO_NET_F_GUEST_USO4 and VIRTIO_NET_F_GUEST_USO6.
\item[VIRTIO_NET_F_GUEST_UDP_TUNNEL_GSO_CSUM] Requires VIRTIO_NET_F_GUEST_UDP_TUNNEL_GSO

\item[VIRTIO_NET_F_HOST_TSO4] Requires VIRTIO_NET_F_CSUM.
\item[VIRTIO_NET_F_HOST_TSO6] Requires VIRTIO_NET_F_CSUM.
\item[VIRTIO_NET_F_HOST_ECN] Requires VIRTIO_NET_F_HOST_TSO4 or VIRTIO_NET_F_HOST_TSO6.
\item[VIRTIO_NET_F_HOST_UFO] Requires VIRTIO_NET_F_CSUM.
\item[VIRTIO_NET_F_HOST_USO] Requires VIRTIO_NET_F_CSUM.
\item[VIRTIO_NET_F_HOST_UDP_TUNNEL_GSO] Requires VIRTIO_NET_F_HOST_TSO4, VIRTIO_NET_F_HOST_TSO6
   and VIRTIO_NET_F_HOST_USO.
\item[VIRTIO_NET_F_HOST_UDP_TUNNEL_GSO_CSUM] Requires VIRTIO_NET_F_HOST_UDP_TUNNEL_GSO

\item[VIRTIO_NET_F_CTRL_RX] Requires VIRTIO_NET_F_CTRL_VQ.
\item[VIRTIO_NET_F_CTRL_VLAN] Requires VIRTIO_NET_F_CTRL_VQ.
\item[VIRTIO_NET_F_GUEST_ANNOUNCE] Requires VIRTIO_NET_F_CTRL_VQ.
\item[VIRTIO_NET_F_MQ] Requires VIRTIO_NET_F_CTRL_VQ.
\item[VIRTIO_NET_F_CTRL_MAC_ADDR] Requires VIRTIO_NET_F_CTRL_VQ.
\item[VIRTIO_NET_F_NOTF_COAL] Requires VIRTIO_NET_F_CTRL_VQ.
\item[VIRTIO_NET_F_RSC_EXT] Requires VIRTIO_NET_F_HOST_TSO4 or VIRTIO_NET_F_HOST_TSO6.
\item[VIRTIO_NET_F_RSS] Requires VIRTIO_NET_F_CTRL_VQ.
\item[VIRTIO_NET_F_VQ_NOTF_COAL] Requires VIRTIO_NET_F_CTRL_VQ.
\item[VIRTIO_NET_F_HASH_TUNNEL] Requires VIRTIO_NET_F_CTRL_VQ along with VIRTIO_NET_F_RSS or VIRTIO_NET_F_HASH_REPORT.
\item[VIRTIO_NET_F_RSS_CONTEXT] Requires VIRTIO_NET_F_CTRL_VQ and VIRTIO_NET_F_RSS.
\end{description}

\begin{note}
The dependency between UDP_TUNNEL_GSO_CSUM and UDP_TUNNEL_GSO is intentionally
in the opposite direction with respect to the plain GSO features and the plain
checksum offload because UDP tunnel checksum offload gives very little gain
for non GSO packets and is quite complex to implement in H/W.
\end{note}

\subsubsection{Legacy Interface: Feature bits}\label{sec:Device Types / Network Device / Feature bits / Legacy Interface: Feature bits}
\begin{description}
\item[VIRTIO_NET_F_GSO (6)] Device handles packets with any GSO type. This was supposed to indicate segmentation offload support, but
upon further investigation it became clear that multiple bits were needed.
\item[VIRTIO_NET_F_GUEST_RSC4 (41)] Device coalesces TCPIP v4 packets. This was implemented by hypervisor patch for certification
purposes and current Windows driver depends on it. It will not function if virtio-net device reports this feature.
\item[VIRTIO_NET_F_GUEST_RSC6 (42)] Device coalesces TCPIP v6 packets. Similar to VIRTIO_NET_F_GUEST_RSC4.
\end{description}

\subsection{Device configuration layout}\label{sec:Device Types / Network Device / Device configuration layout}
\label{sec:Device Types / Block Device / Feature bits / Device configuration layout}

The network device has the following device configuration layout.
All of the device configuration fields are read-only for the driver.

\begin{lstlisting}
struct virtio_net_config {
        u8 mac[6];
        le16 status;
        le16 max_virtqueue_pairs;
        le16 mtu;
        le32 speed;
        u8 duplex;
        u8 rss_max_key_size;
        le16 rss_max_indirection_table_length;
        le32 supported_hash_types;
        le32 supported_tunnel_types;
};
\end{lstlisting}

The \field{mac} address field always exists (although it is only
valid if VIRTIO_NET_F_MAC is set).

The \field{status} only exists if VIRTIO_NET_F_STATUS is set.
Two bits are currently defined for the status field: VIRTIO_NET_S_LINK_UP
and VIRTIO_NET_S_ANNOUNCE.

\begin{lstlisting}
#define VIRTIO_NET_S_LINK_UP     1
#define VIRTIO_NET_S_ANNOUNCE    2
\end{lstlisting}

The following field, \field{max_virtqueue_pairs} only exists if
VIRTIO_NET_F_MQ or VIRTIO_NET_F_RSS is set. This field specifies the maximum number
of each of transmit and receive virtqueues (receiveq1\ldots receiveqN
and transmitq1\ldots transmitqN respectively) that can be configured once at least one of these features
is negotiated.

The following field, \field{mtu} only exists if VIRTIO_NET_F_MTU
is set. This field specifies the maximum MTU for the driver to
use.

The following two fields, \field{speed} and \field{duplex}, only
exist if VIRTIO_NET_F_SPEED_DUPLEX is set.

\field{speed} contains the device speed, in units of 1 MBit per
second, 0 to 0x7fffffff, or 0xffffffff for unknown speed.

\field{duplex} has the values of 0x01 for full duplex, 0x00 for
half duplex and 0xff for unknown duplex state.

Both \field{speed} and \field{duplex} can change, thus the driver
is expected to re-read these values after receiving a
configuration change notification.

The following field, \field{rss_max_key_size} only exists if VIRTIO_NET_F_RSS or VIRTIO_NET_F_HASH_REPORT is set.
It specifies the maximum supported length of RSS key in bytes.

The following field, \field{rss_max_indirection_table_length} only exists if VIRTIO_NET_F_RSS is set.
It specifies the maximum number of 16-bit entries in RSS indirection table.

The next field, \field{supported_hash_types} only exists if the device supports hash calculation,
i.e. if VIRTIO_NET_F_RSS or VIRTIO_NET_F_HASH_REPORT is set.

Field \field{supported_hash_types} contains the bitmask of supported hash types.
See \ref{sec:Device Types / Network Device / Device Operation / Processing of Incoming Packets / Hash calculation for incoming packets / Supported/enabled hash types} for details of supported hash types.

Field \field{supported_tunnel_types} only exists if the device supports inner header hash, i.e. if VIRTIO_NET_F_HASH_TUNNEL is set.

Field \field{supported_tunnel_types} contains the bitmask of encapsulation types supported by the device for inner header hash.
Encapsulation types are defined in \ref{sec:Device Types / Network Device / Device Operation / Processing of Incoming Packets /
Hash calculation for incoming packets / Encapsulation types supported/enabled for inner header hash}.

\devicenormative{\subsubsection}{Device configuration layout}{Device Types / Network Device / Device configuration layout}

The device MUST set \field{max_virtqueue_pairs} to between 1 and 0x8000 inclusive,
if it offers VIRTIO_NET_F_MQ.

The device MUST set \field{mtu} to between 68 and 65535 inclusive,
if it offers VIRTIO_NET_F_MTU.

The device SHOULD set \field{mtu} to at least 1280, if it offers
VIRTIO_NET_F_MTU.

The device MUST NOT modify \field{mtu} once it has been set.

The device MUST NOT pass received packets that exceed \field{mtu} (plus low
level ethernet header length) size with \field{gso_type} NONE or ECN
after VIRTIO_NET_F_MTU has been successfully negotiated.

The device MUST forward transmitted packets of up to \field{mtu} (plus low
level ethernet header length) size with \field{gso_type} NONE or ECN, and do
so without fragmentation, after VIRTIO_NET_F_MTU has been successfully
negotiated.

The device MUST set \field{rss_max_key_size} to at least 40, if it offers
VIRTIO_NET_F_RSS or VIRTIO_NET_F_HASH_REPORT.

The device MUST set \field{rss_max_indirection_table_length} to at least 128, if it offers
VIRTIO_NET_F_RSS.

If the driver negotiates the VIRTIO_NET_F_STANDBY feature, the device MAY act
as a standby device for a primary device with the same MAC address.

If VIRTIO_NET_F_SPEED_DUPLEX has been negotiated, \field{speed}
MUST contain the device speed, in units of 1 MBit per second, 0 to
0x7ffffffff, or 0xfffffffff for unknown.

If VIRTIO_NET_F_SPEED_DUPLEX has been negotiated, \field{duplex}
MUST have the values of 0x00 for full duplex, 0x01 for half
duplex, or 0xff for unknown.

If VIRTIO_NET_F_SPEED_DUPLEX and VIRTIO_NET_F_STATUS have both
been negotiated, the device SHOULD NOT change the \field{speed} and
\field{duplex} fields as long as VIRTIO_NET_S_LINK_UP is set in
the \field{status}.

The device SHOULD NOT offer VIRTIO_NET_F_HASH_REPORT if it
does not offer VIRTIO_NET_F_CTRL_VQ.

The device SHOULD NOT offer VIRTIO_NET_F_CTRL_RX_EXTRA if it
does not offer VIRTIO_NET_F_CTRL_VQ.

\drivernormative{\subsubsection}{Device configuration layout}{Device Types / Network Device / Device configuration layout}

The driver MUST NOT write to any of the device configuration fields.

A driver SHOULD negotiate VIRTIO_NET_F_MAC if the device offers it.
If the driver negotiates the VIRTIO_NET_F_MAC feature, the driver MUST set
the physical address of the NIC to \field{mac}.  Otherwise, it SHOULD
use a locally-administered MAC address (see \hyperref[intro:IEEE 802]{IEEE 802},
``9.2 48-bit universal LAN MAC addresses'').

If the driver does not negotiate the VIRTIO_NET_F_STATUS feature, it SHOULD
assume the link is active, otherwise it SHOULD read the link status from
the bottom bit of \field{status}.

A driver SHOULD negotiate VIRTIO_NET_F_MTU if the device offers it.

If the driver negotiates VIRTIO_NET_F_MTU, it MUST supply enough receive
buffers to receive at least one receive packet of size \field{mtu} (plus low
level ethernet header length) with \field{gso_type} NONE or ECN.

If the driver negotiates VIRTIO_NET_F_MTU, it MUST NOT transmit packets of
size exceeding the value of \field{mtu} (plus low level ethernet header length)
with \field{gso_type} NONE or ECN.

A driver SHOULD negotiate the VIRTIO_NET_F_STANDBY feature if the device offers it.

If VIRTIO_NET_F_SPEED_DUPLEX has been negotiated,
the driver MUST treat any value of \field{speed} above
0x7fffffff as well as any value of \field{duplex} not
matching 0x00 or 0x01 as an unknown value.

If VIRTIO_NET_F_SPEED_DUPLEX has been negotiated, the driver
SHOULD re-read \field{speed} and \field{duplex} after a
configuration change notification.

A driver SHOULD NOT negotiate VIRTIO_NET_F_HASH_REPORT if it
does not negotiate VIRTIO_NET_F_CTRL_VQ.

A driver SHOULD NOT negotiate VIRTIO_NET_F_CTRL_RX_EXTRA if it
does not negotiate VIRTIO_NET_F_CTRL_VQ.

\subsubsection{Legacy Interface: Device configuration layout}\label{sec:Device Types / Network Device / Device configuration layout / Legacy Interface: Device configuration layout}
\label{sec:Device Types / Block Device / Feature bits / Device configuration layout / Legacy Interface: Device configuration layout}
When using the legacy interface, transitional devices and drivers
MUST format \field{status} and
\field{max_virtqueue_pairs} in struct virtio_net_config
according to the native endian of the guest rather than
(necessarily when not using the legacy interface) little-endian.

When using the legacy interface, \field{mac} is driver-writable
which provided a way for drivers to update the MAC without
negotiating VIRTIO_NET_F_CTRL_MAC_ADDR.

\subsection{Device Initialization}\label{sec:Device Types / Network Device / Device Initialization}

A driver would perform a typical initialization routine like so:

\begin{enumerate}
\item Identify and initialize the receive and
  transmission virtqueues, up to N of each kind. If
  VIRTIO_NET_F_MQ feature bit is negotiated,
  N=\field{max_virtqueue_pairs}, otherwise identify N=1.

\item If the VIRTIO_NET_F_CTRL_VQ feature bit is negotiated,
  identify the control virtqueue.

\item Fill the receive queues with buffers: see \ref{sec:Device Types / Network Device / Device Operation / Setting Up Receive Buffers}.

\item Even with VIRTIO_NET_F_MQ, only receiveq1, transmitq1 and
  controlq are used by default.  The driver would send the
  VIRTIO_NET_CTRL_MQ_VQ_PAIRS_SET command specifying the
  number of the transmit and receive queues to use.

\item If the VIRTIO_NET_F_MAC feature bit is set, the configuration
  space \field{mac} entry indicates the ``physical'' address of the
  device, otherwise the driver would typically generate a random
  local MAC address.

\item If the VIRTIO_NET_F_STATUS feature bit is negotiated, the link
  status comes from the bottom bit of \field{status}.
  Otherwise, the driver assumes it's active.

\item A performant driver would indicate that it will generate checksumless
  packets by negotiating the VIRTIO_NET_F_CSUM feature.

\item If that feature is negotiated, a driver can use TCP segmentation or UDP
  segmentation/fragmentation offload by negotiating the VIRTIO_NET_F_HOST_TSO4 (IPv4
  TCP), VIRTIO_NET_F_HOST_TSO6 (IPv6 TCP), VIRTIO_NET_F_HOST_UFO
  (UDP fragmentation) and VIRTIO_NET_F_HOST_USO (UDP segmentation) features.

\item If the VIRTIO_NET_F_HOST_TSO6, VIRTIO_NET_F_HOST_TSO4 and VIRTIO_NET_F_HOST_USO
  segmentation features are negotiated, a driver can
  use TCP segmentation or UDP segmentation on top of UDP encapsulation
  offload, when the outer header does not require checksumming - e.g.
  the outer UDP checksum is zero - by negotiating the
  VIRTIO_NET_F_HOST_UDP_TUNNEL_GSO feature.
  GSO over UDP tunnels packets carry two sets of headers: the outer ones
  and the inner ones. The outer transport protocol is UDP, the inner
  could be either TCP or UDP. Only a single level of encapsulation
  offload is supported.

\item If VIRTIO_NET_F_HOST_UDP_TUNNEL_GSO is negotiated, a driver can
  additionally use TCP segmentation or UDP segmentation on top of UDP
  encapsulation with the outer header requiring checksum offload,
  negotiating the VIRTIO_NET_F_HOST_UDP_TUNNEL_GSO_CSUM feature.

\item The converse features are also available: a driver can save
  the virtual device some work by negotiating these features.\note{For example, a network packet transported between two guests on
the same system might not need checksumming at all, nor segmentation,
if both guests are amenable.}
   The VIRTIO_NET_F_GUEST_CSUM feature indicates that partially
  checksummed packets can be received, and if it can do that then
  the VIRTIO_NET_F_GUEST_TSO4, VIRTIO_NET_F_GUEST_TSO6,
  VIRTIO_NET_F_GUEST_UFO, VIRTIO_NET_F_GUEST_ECN, VIRTIO_NET_F_GUEST_USO4,
  VIRTIO_NET_F_GUEST_USO6 VIRTIO_NET_F_GUEST_UDP_TUNNEL_GSO and
  VIRTIO_NET_F_GUEST_UDP_TUNNEL_GSO_CSUM are the input equivalents of
  the features described above.
  See \ref{sec:Device Types / Network Device / Device Operation /
Setting Up Receive Buffers}~\nameref{sec:Device Types / Network
Device / Device Operation / Setting Up Receive Buffers} and
\ref{sec:Device Types / Network Device / Device Operation /
Processing of Incoming Packets}~\nameref{sec:Device Types /
Network Device / Device Operation / Processing of Incoming Packets} below.
\end{enumerate}

A truly minimal driver would only accept VIRTIO_NET_F_MAC and ignore
everything else.

\subsection{Device and driver capabilities}\label{sec:Device Types / Network Device / Device and driver capabilities}

The network device has the following capabilities.

\begin{tabularx}{\textwidth}{ |l||l|X| }
\hline
Identifier & Name & Description \\
\hline \hline
0x0800 & \hyperref[par:Device Types / Network Device / Device Operation / Flow filter / Device and driver capabilities / VIRTIO-NET-FF-RESOURCE-CAP]{VIRTIO_NET_FF_RESOURCE_CAP} & Flow filter resource capability \\
\hline
0x0801 & \hyperref[par:Device Types / Network Device / Device Operation / Flow filter / Device and driver capabilities / VIRTIO-NET-FF-SELECTOR-CAP]{VIRTIO_NET_FF_SELECTOR_CAP} & Flow filter classifier capability \\
\hline
0x0802 & \hyperref[par:Device Types / Network Device / Device Operation / Flow filter / Device and driver capabilities / VIRTIO-NET-FF-ACTION-CAP]{VIRTIO_NET_FF_ACTION_CAP} & Flow filter action capability \\
\hline
\end{tabularx}

\subsection{Device resource objects}\label{sec:Device Types / Network Device / Device resource objects}

The network device has the following resource objects.

\begin{tabularx}{\textwidth}{ |l||l|X| }
\hline
type & Name & Description \\
\hline \hline
0x0200 & \hyperref[par:Device Types / Network Device / Device Operation / Flow filter / Resource objects / VIRTIO-NET-RESOURCE-OBJ-FF-GROUP]{VIRTIO_NET_RESOURCE_OBJ_FF_GROUP} & Flow filter group resource object \\
\hline
0x0201 & \hyperref[par:Device Types / Network Device / Device Operation / Flow filter / Resource objects / VIRTIO-NET-RESOURCE-OBJ-FF-CLASSIFIER]{VIRTIO_NET_RESOURCE_OBJ_FF_CLASSIFIER} & Flow filter mask object \\
\hline
0x0202 & \hyperref[par:Device Types / Network Device / Device Operation / Flow filter / Resource objects / VIRTIO-NET-RESOURCE-OBJ-FF-RULE]{VIRTIO_NET_RESOURCE_OBJ_FF_RULE} & Flow filter rule object \\
\hline
\end{tabularx}

\subsection{Device parts}\label{sec:Device Types / Network Device / Device parts}

Network device parts represent the configuration done by the driver using control
virtqueue commands. Network device part is in the format of
\field{struct virtio_dev_part}.

\begin{tabularx}{\textwidth}{ |l||l|X| }
\hline
Type & Name & Description \\
\hline \hline
0x200 & VIRTIO_NET_DEV_PART_CVQ_CFG_PART & Represents device configuration done through a control virtqueue command, see \ref{sec:Device Types / Network Device / Device parts / VIRTIO-NET-DEV-PART-CVQ-CFG-PART} \\
\hline
0x201 - 0x5FF & - & reserved for future \\
\hline
\hline
\end{tabularx}

\subsubsection{VIRTIO_NET_DEV_PART_CVQ_CFG_PART}\label{sec:Device Types / Network Device / Device parts / VIRTIO-NET-DEV-PART-CVQ-CFG-PART}

For VIRTIO_NET_DEV_PART_CVQ_CFG_PART, \field{part_type} is set to 0x200. The
VIRTIO_NET_DEV_PART_CVQ_CFG_PART part indicates configuration performed by the
driver using a control virtqueue command.

\begin{lstlisting}
struct virtio_net_dev_part_cvq_selector {
        u8 class;
        u8 command;
        u8 reserved[6];
};
\end{lstlisting}

There is one device part of type VIRTIO_NET_DEV_PART_CVQ_CFG_PART for each
individual configuration. Each part is identified by a unique selector value.
The selector, \field{device_type_raw}, is in the format
\field{struct virtio_net_dev_part_cvq_selector}.

The selector consists of two fields: \field{class} and \field{command}. These
fields correspond to the \field{class} and \field{command} defined in
\field{struct virtio_net_ctrl}, as described in the relevant sections of
\ref{sec:Device Types / Network Device / Device Operation / Control Virtqueue}.

The value corresponding to each part’s selector follows the same format as the
respective \field{command-specific-data} described in the relevant sections of
\ref{sec:Device Types / Network Device / Device Operation / Control Virtqueue}.

For example, when the \field{class} is VIRTIO_NET_CTRL_MAC, the \field{command}
can be either VIRTIO_NET_CTRL_MAC_TABLE_SET or VIRTIO_NET_CTRL_MAC_ADDR_SET;
when \field{command} is set to VIRTIO_NET_CTRL_MAC_TABLE_SET, \field{value}
is in the format of \field{struct virtio_net_ctrl_mac}.

Supported selectors are listed in the table:

\begin{tabularx}{\textwidth}{ |l|X| }
\hline
Class selector & Command selector \\
\hline \hline
VIRTIO_NET_CTRL_RX & VIRTIO_NET_CTRL_RX_PROMISC \\
\hline
VIRTIO_NET_CTRL_RX & VIRTIO_NET_CTRL_RX_ALLMULTI \\
\hline
VIRTIO_NET_CTRL_RX & VIRTIO_NET_CTRL_RX_ALLUNI \\
\hline
VIRTIO_NET_CTRL_RX & VIRTIO_NET_CTRL_RX_NOMULTI \\
\hline
VIRTIO_NET_CTRL_RX & VIRTIO_NET_CTRL_RX_NOUNI \\
\hline
VIRTIO_NET_CTRL_RX & VIRTIO_NET_CTRL_RX_NOBCAST \\
\hline
VIRTIO_NET_CTRL_MAC & VIRTIO_NET_CTRL_MAC_TABLE_SET \\
\hline
VIRTIO_NET_CTRL_MAC & VIRTIO_NET_CTRL_MAC_ADDR_SET \\
\hline
VIRTIO_NET_CTRL_VLAN & VIRTIO_NET_CTRL_VLAN_ADD \\
\hline
VIRTIO_NET_CTRL_ANNOUNCE & VIRTIO_NET_CTRL_ANNOUNCE_ACK \\
\hline
VIRTIO_NET_CTRL_MQ & VIRTIO_NET_CTRL_MQ_VQ_PAIRS_SET \\
\hline
VIRTIO_NET_CTRL_MQ & VIRTIO_NET_CTRL_MQ_RSS_CONFIG \\
\hline
VIRTIO_NET_CTRL_MQ & VIRTIO_NET_CTRL_MQ_HASH_CONFIG \\
\hline
\hline
\end{tabularx}

For command selector VIRTIO_NET_CTRL_VLAN_ADD, device part consists of a whole
VLAN table.

\field{reserved} is reserved and set to zero.

\subsection{Device Operation}\label{sec:Device Types / Network Device / Device Operation}

Packets are transmitted by placing them in the
transmitq1\ldots transmitqN, and buffers for incoming packets are
placed in the receiveq1\ldots receiveqN. In each case, the packet
itself is preceded by a header:

\begin{lstlisting}
struct virtio_net_hdr {
#define VIRTIO_NET_HDR_F_NEEDS_CSUM    1
#define VIRTIO_NET_HDR_F_DATA_VALID    2
#define VIRTIO_NET_HDR_F_RSC_INFO      4
#define VIRTIO_NET_HDR_F_UDP_TUNNEL_CSUM 8
        u8 flags;
#define VIRTIO_NET_HDR_GSO_NONE        0
#define VIRTIO_NET_HDR_GSO_TCPV4       1
#define VIRTIO_NET_HDR_GSO_UDP         3
#define VIRTIO_NET_HDR_GSO_TCPV6       4
#define VIRTIO_NET_HDR_GSO_UDP_L4      5
#define VIRTIO_NET_HDR_GSO_UDP_TUNNEL_IPV4 0x20
#define VIRTIO_NET_HDR_GSO_UDP_TUNNEL_IPV6 0x40
#define VIRTIO_NET_HDR_GSO_ECN      0x80
        u8 gso_type;
        le16 hdr_len;
        le16 gso_size;
        le16 csum_start;
        le16 csum_offset;
        le16 num_buffers;
        le32 hash_value;        (Only if VIRTIO_NET_F_HASH_REPORT negotiated)
        le16 hash_report;       (Only if VIRTIO_NET_F_HASH_REPORT negotiated)
        le16 padding_reserved;  (Only if VIRTIO_NET_F_HASH_REPORT negotiated)
        le16 outer_th_offset    (Only if VIRTIO_NET_F_HOST_UDP_TUNNEL_GSO or VIRTIO_NET_F_GUEST_UDP_TUNNEL_GSO negotiated)
        le16 inner_nh_offset;   (Only if VIRTIO_NET_F_HOST_UDP_TUNNEL_GSO or VIRTIO_NET_F_GUEST_UDP_TUNNEL_GSO negotiated)
};
\end{lstlisting}

The controlq is used to control various device features described further in
section \ref{sec:Device Types / Network Device / Device Operation / Control Virtqueue}.

\subsubsection{Legacy Interface: Device Operation}\label{sec:Device Types / Network Device / Device Operation / Legacy Interface: Device Operation}
When using the legacy interface, transitional devices and drivers
MUST format the fields in \field{struct virtio_net_hdr}
according to the native endian of the guest rather than
(necessarily when not using the legacy interface) little-endian.

The legacy driver only presented \field{num_buffers} in the \field{struct virtio_net_hdr}
when VIRTIO_NET_F_MRG_RXBUF was negotiated; without that feature the
structure was 2 bytes shorter.

When using the legacy interface, the driver SHOULD ignore the
used length for the transmit queues
and the controlq queue.
\begin{note}
Historically, some devices put
the total descriptor length there, even though no data was
actually written.
\end{note}

\subsubsection{Packet Transmission}\label{sec:Device Types / Network Device / Device Operation / Packet Transmission}

Transmitting a single packet is simple, but varies depending on
the different features the driver negotiated.

\begin{enumerate}
\item The driver can send a completely checksummed packet.  In this case,
  \field{flags} will be zero, and \field{gso_type} will be VIRTIO_NET_HDR_GSO_NONE.

\item If the driver negotiated VIRTIO_NET_F_CSUM, it can skip
  checksumming the packet:
  \begin{itemize}
  \item \field{flags} has the VIRTIO_NET_HDR_F_NEEDS_CSUM set,

  \item \field{csum_start} is set to the offset within the packet to begin checksumming,
    and

  \item \field{csum_offset} indicates how many bytes after the csum_start the
    new (16 bit ones' complement) checksum is placed by the device.

  \item The TCP checksum field in the packet is set to the sum
    of the TCP pseudo header, so that replacing it by the ones'
    complement checksum of the TCP header and body will give the
    correct result.
  \end{itemize}

\begin{note}
For example, consider a partially checksummed TCP (IPv4) packet.
It will have a 14 byte ethernet header and 20 byte IP header
followed by the TCP header (with the TCP checksum field 16 bytes
into that header). \field{csum_start} will be 14+20 = 34 (the TCP
checksum includes the header), and \field{csum_offset} will be 16.
If the given packet has the VIRTIO_NET_HDR_GSO_UDP_TUNNEL_IPV4 bit or the
VIRTIO_NET_HDR_GSO_UDP_TUNNEL_IPV6 bit set,
the above checksum fields refer to the inner header checksum, see
the example below.
\end{note}

\item If the driver negotiated
  VIRTIO_NET_F_HOST_TSO4, TSO6, USO or UFO, and the packet requires
  TCP segmentation, UDP segmentation or fragmentation, then \field{gso_type}
  is set to VIRTIO_NET_HDR_GSO_TCPV4, TCPV6, UDP_L4 or UDP.
  (Otherwise, it is set to VIRTIO_NET_HDR_GSO_NONE). In this
  case, packets larger than 1514 bytes can be transmitted: the
  metadata indicates how to replicate the packet header to cut it
  into smaller packets. The other gso fields are set:

  \begin{itemize}
  \item If the VIRTIO_NET_F_GUEST_HDRLEN feature has been negotiated,
    \field{hdr_len} indicates the header length that needs to be replicated
    for each packet. It's the number of bytes from the beginning of the packet
    to the beginning of the transport payload.
    If the \field{gso_type} has the VIRTIO_NET_HDR_GSO_UDP_TUNNEL_IPV4 bit or
    VIRTIO_NET_HDR_GSO_UDP_TUNNEL_IPV6 bit set, \field{hdr_len} accounts for
    all the headers up to and including the inner transport.
    Otherwise, if the VIRTIO_NET_F_GUEST_HDRLEN feature has not been negotiated,
    \field{hdr_len} is a hint to the device as to how much of the header
    needs to be kept to copy into each packet, usually set to the
    length of the headers, including the transport header\footnote{Due to various bugs in implementations, this field is not useful
as a guarantee of the transport header size.
}.

  \begin{note}
  Some devices benefit from knowledge of the exact header length.
  \end{note}

  \item \field{gso_size} is the maximum size of each packet beyond that
    header (ie. MSS).

  \item If the driver negotiated the VIRTIO_NET_F_HOST_ECN feature,
    the VIRTIO_NET_HDR_GSO_ECN bit in \field{gso_type}
    indicates that the TCP packet has the ECN bit set\footnote{This case is not handled by some older hardware, so is called out
specifically in the protocol.}.
   \end{itemize}

\item If the driver negotiated the VIRTIO_NET_F_HOST_UDP_TUNNEL_GSO feature and the
  \field{gso_type} has the VIRTIO_NET_HDR_GSO_UDP_TUNNEL_IPV4 bit or
  VIRTIO_NET_HDR_GSO_UDP_TUNNEL_IPV6 bit set, the GSO protocol is encapsulated
  in a UDP tunnel.
  If the outer UDP header requires checksumming, the driver must have
  additionally negotiated the VIRTIO_NET_F_HOST_UDP_TUNNEL_GSO_CSUM feature
  and offloaded the outer checksum accordingly, otherwise
  the outer UDP header must not require checksum validation, i.e. the outer
  UDP checksum must be positive zero (0x0) as defined in UDP RFC 768.
  The other tunnel-related fields indicate how to replicate the packet
  headers to cut it into smaller packets:

  \begin{itemize}
  \item \field{outer_th_offset} field indicates the outer transport header within
      the packet. This field differs from \field{csum_start} as the latter
      points to the inner transport header within the packet.

  \item \field{inner_nh_offset} field indicates the inner network header within
      the packet.
  \end{itemize}

\begin{note}
For example, consider a partially checksummed TCP (IPv4) packet carried over a
Geneve UDP tunnel (again IPv4) with no tunnel options. The
only relevant variable related to the tunnel type is the tunnel header length.
The packet will have a 14 byte outer ethernet header, 20 byte outer IP header
followed by the 8 byte UDP header (with a 0 checksum value), 8 byte Geneve header,
14 byte inner ethernet header, 20 byte inner IP header
and the TCP header (with the TCP checksum field 16 bytes
into that header). \field{csum_start} will be 14+20+8+8+14+20 = 84 (the TCP
checksum includes the header), \field{csum_offset} will be 16.
\field{inner_nh_offset} will be 14+20+8+8+14 = 62, \field{outer_th_offset} will be
14+20+8 = 42 and \field{gso_type} will be
VIRTIO_NET_HDR_GSO_TCPV4 | VIRTIO_NET_HDR_GSO_UDP_TUNNEL_IPV4 = 0x21
\end{note}

\item If the driver negotiated the VIRTIO_NET_F_HOST_UDP_TUNNEL_GSO_CSUM feature,
  the transmitted packet is a GSO one encapsulated in a UDP tunnel, and
  the outer UDP header requires checksumming, the driver can skip checksumming
  the outer header:

  \begin{itemize}
  \item \field{flags} has the VIRTIO_NET_HDR_F_UDP_TUNNEL_CSUM set,

  \item The outer UDP checksum field in the packet is set to the sum
    of the UDP pseudo header, so that replacing it by the ones'
    complement checksum of the outer UDP header and payload will give the
    correct result.
  \end{itemize}

\item \field{num_buffers} is set to zero.  This field is unused on transmitted packets.

\item The header and packet are added as one output descriptor to the
  transmitq, and the device is notified of the new entry
  (see \ref{sec:Device Types / Network Device / Device Initialization}~\nameref{sec:Device Types / Network Device / Device Initialization}).
\end{enumerate}

\drivernormative{\paragraph}{Packet Transmission}{Device Types / Network Device / Device Operation / Packet Transmission}

For the transmit packet buffer, the driver MUST use the size of the
structure \field{struct virtio_net_hdr} same as the receive packet buffer.

The driver MUST set \field{num_buffers} to zero.

If VIRTIO_NET_F_CSUM is not negotiated, the driver MUST set
\field{flags} to zero and SHOULD supply a fully checksummed
packet to the device.

If VIRTIO_NET_F_HOST_TSO4 is negotiated, the driver MAY set
\field{gso_type} to VIRTIO_NET_HDR_GSO_TCPV4 to request TCPv4
segmentation, otherwise the driver MUST NOT set
\field{gso_type} to VIRTIO_NET_HDR_GSO_TCPV4.

If VIRTIO_NET_F_HOST_TSO6 is negotiated, the driver MAY set
\field{gso_type} to VIRTIO_NET_HDR_GSO_TCPV6 to request TCPv6
segmentation, otherwise the driver MUST NOT set
\field{gso_type} to VIRTIO_NET_HDR_GSO_TCPV6.

If VIRTIO_NET_F_HOST_UFO is negotiated, the driver MAY set
\field{gso_type} to VIRTIO_NET_HDR_GSO_UDP to request UDP
fragmentation, otherwise the driver MUST NOT set
\field{gso_type} to VIRTIO_NET_HDR_GSO_UDP.

If VIRTIO_NET_F_HOST_USO is negotiated, the driver MAY set
\field{gso_type} to VIRTIO_NET_HDR_GSO_UDP_L4 to request UDP
segmentation, otherwise the driver MUST NOT set
\field{gso_type} to VIRTIO_NET_HDR_GSO_UDP_L4.

The driver SHOULD NOT send to the device TCP packets requiring segmentation offload
which have the Explicit Congestion Notification bit set, unless the
VIRTIO_NET_F_HOST_ECN feature is negotiated, in which case the
driver MUST set the VIRTIO_NET_HDR_GSO_ECN bit in
\field{gso_type}.

If VIRTIO_NET_F_HOST_UDP_TUNNEL_GSO is negotiated, the driver MAY set
VIRTIO_NET_HDR_GSO_UDP_TUNNEL_IPV4 bit or the VIRTIO_NET_HDR_GSO_UDP_TUNNEL_IPV6 bit
in \field{gso_type} according to the inner network header protocol type
to request GSO packets over UDPv4 or UDPv6 tunnel segmentation,
otherwise the driver MUST NOT set either the
VIRTIO_NET_HDR_GSO_UDP_TUNNEL_IPV4 bit or the VIRTIO_NET_HDR_GSO_UDP_TUNNEL_IPV6 bit
in \field{gso_type}.

When requesting GSO segmentation over UDP tunnel, the driver MUST SET the
VIRTIO_NET_HDR_GSO_UDP_TUNNEL_IPV4 bit if the inner network header is IPv4, i.e. the
packet is a TCPv4 GSO one, otherwise, if the inner network header is IPv6, the driver
MUST SET the VIRTIO_NET_HDR_GSO_UDP_TUNNEL_IPV6 bit.

The driver MUST NOT send to the device GSO packets over UDP tunnel
requiring segmentation and outer UDP checksum offload, unless both the
VIRTIO_NET_F_HOST_UDP_TUNNEL_GSO and VIRTIO_NET_F_HOST_UDP_TUNNEL_GSO_CSUM features
are negotiated, in which case the driver MUST set either the
VIRTIO_NET_HDR_GSO_UDP_TUNNEL_IPV4 bit or the VIRTIO_NET_HDR_GSO_UDP_TUNNEL_IPV6
bit in the \field{gso_type} and the VIRTIO_NET_HDR_F_UDP_TUNNEL_CSUM bit in
the \field{flags}.

If VIRTIO_NET_F_HOST_UDP_TUNNEL_GSO_CSUM is not negotiated, the driver MUST not set
the VIRTIO_NET_HDR_F_UDP_TUNNEL_CSUM bit in the \field{flags} and
MUST NOT send to the device GSO packets over UDP tunnel
requiring segmentation and outer UDP checksum offload.

The driver MUST NOT set the VIRTIO_NET_HDR_GSO_UDP_TUNNEL_IPV4 bit or the
VIRTIO_NET_HDR_GSO_UDP_TUNNEL_IPV6 bit together with VIRTIO_NET_HDR_GSO_UDP, as the
latter is deprecated in favor of UDP_L4 and no new feature will support it.

The driver MUST NOT set the VIRTIO_NET_HDR_GSO_UDP_TUNNEL_IPV4 bit and the
VIRTIO_NET_HDR_GSO_UDP_TUNNEL_IPV6 bit together.

The driver MUST NOT set the VIRTIO_NET_HDR_F_UDP_TUNNEL_CSUM bit \field{flags}
without setting either the VIRTIO_NET_HDR_GSO_UDP_TUNNEL_IPV4 bit or
the VIRTIO_NET_HDR_GSO_UDP_TUNNEL_IPV6 bit in \field{gso_type}.

If the VIRTIO_NET_F_CSUM feature has been negotiated, the
driver MAY set the VIRTIO_NET_HDR_F_NEEDS_CSUM bit in
\field{flags}, if so:
\begin{enumerate}
\item the driver MUST validate the packet checksum at
	offset \field{csum_offset} from \field{csum_start} as well as all
	preceding offsets;
\begin{note}
If \field{gso_type} differs from VIRTIO_NET_HDR_GSO_NONE and the
VIRTIO_NET_HDR_GSO_UDP_TUNNEL_IPV4 bit or the VIRTIO_NET_HDR_GSO_UDP_TUNNEL_IPV6
bit are not set in \field{gso_type}, \field{csum_offset}
points to the only transport header present in the packet, and there are no
additional preceding checksums validated by VIRTIO_NET_HDR_F_NEEDS_CSUM.
\end{note}
\item the driver MUST set the packet checksum stored in the
	buffer to the TCP/UDP pseudo header;
\item the driver MUST set \field{csum_start} and
	\field{csum_offset} such that calculating a ones'
	complement checksum from \field{csum_start} up until the end of
	the packet and storing the result at offset \field{csum_offset}
	from  \field{csum_start} will result in a fully checksummed
	packet;
\end{enumerate}

If none of the VIRTIO_NET_F_HOST_TSO4, TSO6, USO or UFO options have
been negotiated, the driver MUST set \field{gso_type} to
VIRTIO_NET_HDR_GSO_NONE.

If \field{gso_type} differs from VIRTIO_NET_HDR_GSO_NONE, then
the driver MUST also set the VIRTIO_NET_HDR_F_NEEDS_CSUM bit in
\field{flags} and MUST set \field{gso_size} to indicate the
desired MSS.

If one of the VIRTIO_NET_F_HOST_TSO4, TSO6, USO or UFO options have
been negotiated:
\begin{itemize}
\item If the VIRTIO_NET_F_GUEST_HDRLEN feature has been negotiated,
	and \field{gso_type} differs from VIRTIO_NET_HDR_GSO_NONE,
	the driver MUST set \field{hdr_len} to a value equal to the length
	of the headers, including the transport header. If \field{gso_type}
	has the VIRTIO_NET_HDR_GSO_UDP_TUNNEL_IPV4 bit or the
	VIRTIO_NET_HDR_GSO_UDP_TUNNEL_IPV6 bit set, \field{hdr_len} includes
	the inner transport header.

\item If the VIRTIO_NET_F_GUEST_HDRLEN feature has not been negotiated,
	or \field{gso_type} is VIRTIO_NET_HDR_GSO_NONE,
	the driver SHOULD set \field{hdr_len} to a value
	not less than the length of the headers, including the transport
	header.
\end{itemize}

If the VIRTIO_NET_F_HOST_UDP_TUNNEL_GSO option has been negotiated, the
driver MAY set the VIRTIO_NET_HDR_GSO_UDP_TUNNEL_IPV4 bit or the
VIRTIO_NET_HDR_GSO_UDP_TUNNEL_IPV6 bit in \field{gso_type}, if so:
\begin{itemize}
\item the driver MUST set \field{outer_th_offset} to the outer UDP header
  offset and \field{inner_nh_offset} to the inner network header offset.
  The \field{csum_start} and \field{csum_offset} fields point respectively
  to the inner transport header and inner transport checksum field.
\end{itemize}

If the VIRTIO_NET_F_HOST_UDP_TUNNEL_GSO_CSUM feature has been negotiated,
and the VIRTIO_NET_HDR_GSO_UDP_TUNNEL_IPV4 bit or
VIRTIO_NET_HDR_GSO_UDP_TUNNEL_IPV6 bit in \field{gso_type} are set,
the driver MAY set the VIRTIO_NET_HDR_F_UDP_TUNNEL_CSUM bit in
\field{flags}, if so the driver MUST set the packet outer UDP header checksum
to the outer UDP pseudo header checksum.

\begin{note}
calculating a ones' complement checksum from \field{outer_th_offset}
up until the end of the packet and storing the result at offset 6
from \field{outer_th_offset} will result in a fully checksummed outer UDP packet;
\end{note}

If the VIRTIO_NET_HDR_GSO_UDP_TUNNEL_IPV4 bit or the
VIRTIO_NET_HDR_GSO_UDP_TUNNEL_IPV6 bit in \field{gso_type} are set
and the VIRTIO_NET_F_HOST_UDP_TUNNEL_GSO_CSUM feature has not
been negotiated, the
outer UDP header MUST NOT require checksum validation. That is, the
outer UDP checksum value MUST be 0 or the validated complete checksum
for such header.

\begin{note}
The valid complete checksum of the outer UDP header of individual segments
can be computed by the driver prior to segmentation only if the GSO packet
size is a multiple of \field{gso_size}, because then all segments
have the same size and thus all data included in the outer UDP
checksum is the same for every segment. These pre-computed segment
length and checksum fields are different from those of the GSO
packet.
In this scenario the outer UDP header of the GSO packet must carry the
segmented UDP packet length.
\end{note}

If the VIRTIO_NET_F_HOST_UDP_TUNNEL_GSO option has not
been negotiated, the driver MUST NOT set either the VIRTIO_NET_HDR_F_GSO_UDP_TUNNEL_IPV4
bit or the VIRTIO_NET_HDR_F_GSO_UDP_TUNNEL_IPV6 in \field{gso_type}.

If the VIRTIO_NET_F_HOST_UDP_TUNNEL_GSO_CSUM option has not been negotiated,
the driver MUST NOT set the VIRTIO_NET_HDR_F_UDP_TUNNEL_CSUM bit
in \field{flags}.

The driver SHOULD accept the VIRTIO_NET_F_GUEST_HDRLEN feature if it has
been offered, and if it's able to provide the exact header length.

The driver MUST NOT set the VIRTIO_NET_HDR_F_DATA_VALID and
VIRTIO_NET_HDR_F_RSC_INFO bits in \field{flags}.

The driver MUST NOT set the VIRTIO_NET_HDR_F_DATA_VALID bit in \field{flags}
together with the VIRTIO_NET_HDR_F_GSO_UDP_TUNNEL_IPV4 bit or the
VIRTIO_NET_HDR_F_GSO_UDP_TUNNEL_IPV6 bit in \field{gso_type}.

\devicenormative{\paragraph}{Packet Transmission}{Device Types / Network Device / Device Operation / Packet Transmission}
The device MUST ignore \field{flag} bits that it does not recognize.

If VIRTIO_NET_HDR_F_NEEDS_CSUM bit in \field{flags} is not set, the
device MUST NOT use the \field{csum_start} and \field{csum_offset}.

If one of the VIRTIO_NET_F_HOST_TSO4, TSO6, USO or UFO options have
been negotiated:
\begin{itemize}
\item If the VIRTIO_NET_F_GUEST_HDRLEN feature has been negotiated,
	and \field{gso_type} differs from VIRTIO_NET_HDR_GSO_NONE,
	the device MAY use \field{hdr_len} as the transport header size.

	\begin{note}
	Caution should be taken by the implementation so as to prevent
	a malicious driver from attacking the device by setting an incorrect hdr_len.
	\end{note}

\item If the VIRTIO_NET_F_GUEST_HDRLEN feature has not been negotiated,
	or \field{gso_type} is VIRTIO_NET_HDR_GSO_NONE,
	the device MAY use \field{hdr_len} only as a hint about the
	transport header size.
	The device MUST NOT rely on \field{hdr_len} to be correct.

	\begin{note}
	This is due to various bugs in implementations.
	\end{note}
\end{itemize}

If both the VIRTIO_NET_HDR_GSO_UDP_TUNNEL_IPV4 bit and
the VIRTIO_NET_HDR_GSO_UDP_TUNNEL_IPV6 bit in in \field{gso_type} are set,
the device MUST NOT accept the packet.

If the VIRTIO_NET_HDR_GSO_UDP_TUNNEL_IPV4 bit and the VIRTIO_NET_HDR_GSO_UDP_TUNNEL_IPV6
bit in \field{gso_type} are not set, the device MUST NOT use the
\field{outer_th_offset} and \field{inner_nh_offset}.

If either the VIRTIO_NET_HDR_GSO_UDP_TUNNEL_IPV4 bit or
the VIRTIO_NET_HDR_GSO_UDP_TUNNEL_IPV6 bit in \field{gso_type} are set, and any of
the following is true:
\begin{itemize}
\item the VIRTIO_NET_HDR_F_NEEDS_CSUM is not set in \field{flags}
\item the VIRTIO_NET_HDR_F_DATA_VALID is set in \field{flags}
\item the \field{gso_type} excluding the VIRTIO_NET_HDR_GSO_UDP_TUNNEL_IPV4
bit and the VIRTIO_NET_HDR_GSO_UDP_TUNNEL_IPV6 bit is VIRTIO_NET_HDR_GSO_NONE
\end{itemize}
the device MUST NOT accept the packet.

If the VIRTIO_NET_HDR_F_UDP_TUNNEL_CSUM bit in \field{flags} is set,
and both the bits VIRTIO_NET_HDR_GSO_UDP_TUNNEL_IPV4 and
VIRTIO_NET_HDR_GSO_UDP_TUNNEL_IPV6 in \field{gso_type} are not set,
the device MOST NOT accept the packet.

If VIRTIO_NET_HDR_F_NEEDS_CSUM is not set, the device MUST NOT
rely on the packet checksum being correct.
\paragraph{Packet Transmission Interrupt}\label{sec:Device Types / Network Device / Device Operation / Packet Transmission / Packet Transmission Interrupt}

Often a driver will suppress transmission virtqueue interrupts
and check for used packets in the transmit path of following
packets.

The normal behavior in this interrupt handler is to retrieve
used buffers from the virtqueue and free the corresponding
headers and packets.

\subsubsection{Setting Up Receive Buffers}\label{sec:Device Types / Network Device / Device Operation / Setting Up Receive Buffers}

It is generally a good idea to keep the receive virtqueue as
fully populated as possible: if it runs out, network performance
will suffer.

If the VIRTIO_NET_F_GUEST_TSO4, VIRTIO_NET_F_GUEST_TSO6,
VIRTIO_NET_F_GUEST_UFO, VIRTIO_NET_F_GUEST_USO4 or VIRTIO_NET_F_GUEST_USO6
features are used, the maximum incoming packet
will be 65589 bytes long (14 bytes of Ethernet header, plus 40 bytes of
the IPv6 header, plus 65535 bytes of maximum IPv6 payload including any
extension header), otherwise 1514 bytes.
When VIRTIO_NET_F_HASH_REPORT is not negotiated, the required receive buffer
size is either 65601 or 1526 bytes accounting for 20 bytes of
\field{struct virtio_net_hdr} followed by receive packet.
When VIRTIO_NET_F_HASH_REPORT is negotiated, the required receive buffer
size is either 65609 or 1534 bytes accounting for 12 bytes of
\field{struct virtio_net_hdr} followed by receive packet.

\drivernormative{\paragraph}{Setting Up Receive Buffers}{Device Types / Network Device / Device Operation / Setting Up Receive Buffers}

\begin{itemize}
\item If VIRTIO_NET_F_MRG_RXBUF is not negotiated:
  \begin{itemize}
    \item If VIRTIO_NET_F_GUEST_TSO4, VIRTIO_NET_F_GUEST_TSO6, VIRTIO_NET_F_GUEST_UFO,
	VIRTIO_NET_F_GUEST_USO4 or VIRTIO_NET_F_GUEST_USO6 are negotiated, the driver SHOULD populate
      the receive queue(s) with buffers of at least 65609 bytes if
      VIRTIO_NET_F_HASH_REPORT is negotiated, and of at least 65601 bytes if not.
    \item Otherwise, the driver SHOULD populate the receive queue(s)
      with buffers of at least 1534 bytes if VIRTIO_NET_F_HASH_REPORT
      is negotiated, and of at least 1526 bytes if not.
  \end{itemize}
\item If VIRTIO_NET_F_MRG_RXBUF is negotiated, each buffer MUST be at
least size of \field{struct virtio_net_hdr},
i.e. 20 bytes if VIRTIO_NET_F_HASH_REPORT is negotiated, and 12 bytes if not.
\end{itemize}

\begin{note}
Obviously each buffer can be split across multiple descriptor elements.
\end{note}

When calculating the size of \field{struct virtio_net_hdr}, the driver
MUST consider all the fields inclusive up to \field{padding_reserved},
i.e. 20 bytes if VIRTIO_NET_F_HASH_REPORT is negotiated, and 12 bytes if not.

If VIRTIO_NET_F_MQ is negotiated, each of receiveq1\ldots receiveqN
that will be used SHOULD be populated with receive buffers.

\devicenormative{\paragraph}{Setting Up Receive Buffers}{Device Types / Network Device / Device Operation / Setting Up Receive Buffers}

The device MUST set \field{num_buffers} to the number of descriptors used to
hold the incoming packet.

The device MUST use only a single descriptor if VIRTIO_NET_F_MRG_RXBUF
was not negotiated.
\begin{note}
{This means that \field{num_buffers} will always be 1
if VIRTIO_NET_F_MRG_RXBUF is not negotiated.}
\end{note}

\subsubsection{Processing of Incoming Packets}\label{sec:Device Types / Network Device / Device Operation / Processing of Incoming Packets}
\label{sec:Device Types / Network Device / Device Operation / Processing of Packets}%old label for latexdiff

When a packet is copied into a buffer in the receiveq, the
optimal path is to disable further used buffer notifications for the
receiveq and process packets until no more are found, then re-enable
them.

Processing incoming packets involves:

\begin{enumerate}
\item \field{num_buffers} indicates how many descriptors
  this packet is spread over (including this one): this will
  always be 1 if VIRTIO_NET_F_MRG_RXBUF was not negotiated.
  This allows receipt of large packets without having to allocate large
  buffers: a packet that does not fit in a single buffer can flow
  over to the next buffer, and so on. In this case, there will be
  at least \field{num_buffers} used buffers in the virtqueue, and the device
  chains them together to form a single packet in a way similar to
  how it would store it in a single buffer spread over multiple
  descriptors.
  The other buffers will not begin with a \field{struct virtio_net_hdr}.

\item If
  \field{num_buffers} is one, then the entire packet will be
  contained within this buffer, immediately following the struct
  virtio_net_hdr.
\item If the VIRTIO_NET_F_GUEST_CSUM feature was negotiated, the
  VIRTIO_NET_HDR_F_DATA_VALID bit in \field{flags} can be
  set: if so, device has validated the packet checksum.
  If the VIRTIO_NET_F_GUEST_UDP_TUNNEL_GSO_CSUM feature has been negotiated,
  and the VIRTIO_NET_HDR_F_UDP_TUNNEL_CSUM bit is set in \field{flags},
  both the outer UDP checksum and the inner transport checksum
  have been validated, otherwise only one level of checksums (the outer one
  in case of tunnels) has been validated.
\end{enumerate}

Additionally, VIRTIO_NET_F_GUEST_CSUM, TSO4, TSO6, UDP, UDP_TUNNEL
and ECN features enable receive checksum, large receive offload and ECN
support which are the input equivalents of the transmit checksum,
transmit segmentation offloading and ECN features, as described
in \ref{sec:Device Types / Network Device / Device Operation /
Packet Transmission}:
\begin{enumerate}
\item If the VIRTIO_NET_F_GUEST_TSO4, TSO6, UFO, USO4 or USO6 options were
  negotiated, then \field{gso_type} MAY be something other than
  VIRTIO_NET_HDR_GSO_NONE, and \field{gso_size} field indicates the
  desired MSS (see Packet Transmission point 2).
\item If the VIRTIO_NET_F_RSC_EXT option was negotiated (this
  implies one of VIRTIO_NET_F_GUEST_TSO4, TSO6), the
  device processes also duplicated ACK segments, reports
  number of coalesced TCP segments in \field{csum_start} field and
  number of duplicated ACK segments in \field{csum_offset} field
  and sets bit VIRTIO_NET_HDR_F_RSC_INFO in \field{flags}.
\item If the VIRTIO_NET_F_GUEST_CSUM feature was negotiated, the
  VIRTIO_NET_HDR_F_NEEDS_CSUM bit in \field{flags} can be
  set: if so, the packet checksum at offset \field{csum_offset}
  from \field{csum_start} and any preceding checksums
  have been validated.  The checksum on the packet is incomplete and
  if bit VIRTIO_NET_HDR_F_RSC_INFO is not set in \field{flags},
  then \field{csum_start} and \field{csum_offset} indicate how to calculate it
  (see Packet Transmission point 1).
\begin{note}
If \field{gso_type} differs from VIRTIO_NET_HDR_GSO_NONE and the
VIRTIO_NET_HDR_GSO_UDP_TUNNEL_IPV4 bit or the VIRTIO_NET_HDR_GSO_UDP_TUNNEL_IPV6
bit are not set, \field{csum_offset}
points to the only transport header present in the packet, and there are no
additional preceding checksums validated by VIRTIO_NET_HDR_F_NEEDS_CSUM.
\end{note}
\item If the VIRTIO_NET_F_GUEST_UDP_TUNNEL_GSO option was negotiated and
  \field{gso_type} is not VIRTIO_NET_HDR_GSO_NONE, the
  VIRTIO_NET_HDR_GSO_UDP_TUNNEL_IPV4 bit or the VIRTIO_NET_HDR_GSO_UDP_TUNNEL_IPV6
  bit MAY be set. In such case the \field{outer_th_offset} and
  \field{inner_nh_offset} fields indicate the corresponding
  headers information.
\item If the VIRTIO_NET_F_GUEST_UDP_TUNNEL_GSO_CSUM feature was
negotiated, and
  the VIRTIO_NET_HDR_GSO_UDP_TUNNEL_IPV4 bit or the VIRTIO_NET_HDR_GSO_UDP_TUNNEL_IPV6
  are set in \field{gso_type}, the VIRTIO_NET_HDR_F_UDP_TUNNEL_CSUM bit in the
  \field{flags} can be set: if so, the outer UDP checksum has been validated
  and the UDP header checksum at offset 6 from from \field{outer_th_offset}
  is set to the outer UDP pseudo header checksum.

\begin{note}
If the VIRTIO_NET_HDR_GSO_UDP_TUNNEL_IPV4 bit or VIRTIO_NET_HDR_GSO_UDP_TUNNEL_IPV6
bit are set in \field{gso_type}, the \field{csum_start} field refers to
the inner transport header offset (see Packet Transmission point 1).
If the VIRTIO_NET_HDR_F_UDP_TUNNEL_CSUM bit in \field{flags} is set both
the inner and the outer header checksums have been validated by
VIRTIO_NET_HDR_F_NEEDS_CSUM, otherwise only the inner transport header
checksum has been validated.
\end{note}
\end{enumerate}

If applicable, the device calculates per-packet hash for incoming packets as
defined in \ref{sec:Device Types / Network Device / Device Operation / Processing of Incoming Packets / Hash calculation for incoming packets}.

If applicable, the device reports hash information for incoming packets as
defined in \ref{sec:Device Types / Network Device / Device Operation / Processing of Incoming Packets / Hash reporting for incoming packets}.

\devicenormative{\paragraph}{Processing of Incoming Packets}{Device Types / Network Device / Device Operation / Processing of Incoming Packets}
\label{devicenormative:Device Types / Network Device / Device Operation / Processing of Packets}%old label for latexdiff

If VIRTIO_NET_F_MRG_RXBUF has not been negotiated, the device MUST set
\field{num_buffers} to 1.

If VIRTIO_NET_F_MRG_RXBUF has been negotiated, the device MUST set
\field{num_buffers} to indicate the number of buffers
the packet (including the header) is spread over.

If a receive packet is spread over multiple buffers, the device
MUST use all buffers but the last (i.e. the first \field{num_buffers} -
1 buffers) completely up to the full length of each buffer
supplied by the driver.

The device MUST use all buffers used by a single receive
packet together, such that at least \field{num_buffers} are
observed by driver as used.

If VIRTIO_NET_F_GUEST_CSUM is not negotiated, the device MUST set
\field{flags} to zero and SHOULD supply a fully checksummed
packet to the driver.

If VIRTIO_NET_F_GUEST_TSO4 is not negotiated, the device MUST NOT set
\field{gso_type} to VIRTIO_NET_HDR_GSO_TCPV4.

If VIRTIO_NET_F_GUEST_UDP is not negotiated, the device MUST NOT set
\field{gso_type} to VIRTIO_NET_HDR_GSO_UDP.

If VIRTIO_NET_F_GUEST_TSO6 is not negotiated, the device MUST NOT set
\field{gso_type} to VIRTIO_NET_HDR_GSO_TCPV6.

If none of VIRTIO_NET_F_GUEST_USO4 or VIRTIO_NET_F_GUEST_USO6 have been negotiated,
the device MUST NOT set \field{gso_type} to VIRTIO_NET_HDR_GSO_UDP_L4.

If VIRTIO_NET_F_GUEST_UDP_TUNNEL_GSO is not negotiated, the device MUST NOT set
either the VIRTIO_NET_HDR_GSO_UDP_TUNNEL_IPV4 bit or the
VIRTIO_NET_HDR_GSO_UDP_TUNNEL_IPV6 bit in \field{gso_type}.

If VIRTIO_NET_F_GUEST_UDP_TUNNEL_GSO_CSUM is not negotiated the device MUST NOT set
the VIRTIO_NET_HDR_F_UDP_TUNNEL_CSUM bit in \field{flags}.

The device SHOULD NOT send to the driver TCP packets requiring segmentation offload
which have the Explicit Congestion Notification bit set, unless the
VIRTIO_NET_F_GUEST_ECN feature is negotiated, in which case the
device MUST set the VIRTIO_NET_HDR_GSO_ECN bit in
\field{gso_type}.

If the VIRTIO_NET_F_GUEST_CSUM feature has been negotiated, the
device MAY set the VIRTIO_NET_HDR_F_NEEDS_CSUM bit in
\field{flags}, if so:
\begin{enumerate}
\item the device MUST validate the packet checksum at
	offset \field{csum_offset} from \field{csum_start} as well as all
	preceding offsets;
\item the device MUST set the packet checksum stored in the
	receive buffer to the TCP/UDP pseudo header;
\item the device MUST set \field{csum_start} and
	\field{csum_offset} such that calculating a ones'
	complement checksum from \field{csum_start} up until the
	end of the packet and storing the result at offset
	\field{csum_offset} from  \field{csum_start} will result in a
	fully checksummed packet;
\end{enumerate}

The device MUST NOT send to the driver GSO packets encapsulated in UDP
tunnel and requiring segmentation offload, unless the
VIRTIO_NET_F_GUEST_UDP_TUNNEL_GSO is negotiated, in which case the device MUST set
the VIRTIO_NET_HDR_GSO_UDP_TUNNEL_IPV4 bit or the VIRTIO_NET_HDR_GSO_UDP_TUNNEL_IPV6
bit in \field{gso_type} according to the inner network header protocol type,
MUST set the \field{outer_th_offset} and \field{inner_nh_offset} fields
to the corresponding header information, and the outer UDP header MUST NOT
require checksum offload.

If the VIRTIO_NET_F_GUEST_UDP_TUNNEL_GSO_CSUM feature has not been negotiated,
the device MUST NOT send the driver GSO packets encapsulated in UDP
tunnel and requiring segmentation and outer checksum offload.

If none of the VIRTIO_NET_F_GUEST_TSO4, TSO6, UFO, USO4 or USO6 options have
been negotiated, the device MUST set \field{gso_type} to
VIRTIO_NET_HDR_GSO_NONE.

If \field{gso_type} differs from VIRTIO_NET_HDR_GSO_NONE, then
the device MUST also set the VIRTIO_NET_HDR_F_NEEDS_CSUM bit in
\field{flags} MUST set \field{gso_size} to indicate the desired MSS.
If VIRTIO_NET_F_RSC_EXT was negotiated, the device MUST also
set VIRTIO_NET_HDR_F_RSC_INFO bit in \field{flags},
set \field{csum_start} to number of coalesced TCP segments and
set \field{csum_offset} to number of received duplicated ACK segments.

If VIRTIO_NET_F_RSC_EXT was not negotiated, the device MUST
not set VIRTIO_NET_HDR_F_RSC_INFO bit in \field{flags}.

If one of the VIRTIO_NET_F_GUEST_TSO4, TSO6, UFO, USO4 or USO6 options have
been negotiated, the device SHOULD set \field{hdr_len} to a value
not less than the length of the headers, including the transport
header. If \field{gso_type} has the VIRTIO_NET_HDR_GSO_UDP_TUNNEL_IPV4 bit
or the VIRTIO_NET_HDR_GSO_UDP_TUNNEL_IPV6 bit set, the referenced transport
header is the inner one.

If the VIRTIO_NET_F_GUEST_CSUM feature has been negotiated, the
device MAY set the VIRTIO_NET_HDR_F_DATA_VALID bit in
\field{flags}, if so, the device MUST validate the packet
checksum. If the VIRTIO_NET_F_GUEST_UDP_TUNNEL_GSO_CSUM feature has
been negotiated, and the VIRTIO_NET_HDR_F_UDP_TUNNEL_CSUM bit set in
\field{flags}, both the outer UDP checksum and the inner transport
checksum have been validated.
Otherwise level of checksum is validated: in case of multiple
encapsulated protocols the outermost one.

If either the VIRTIO_NET_HDR_GSO_UDP_TUNNEL_IPV4 bit or the
VIRTIO_NET_HDR_GSO_UDP_TUNNEL_IPV6 bit in \field{gso_type} are set,
the device MUST NOT set the VIRTIO_NET_HDR_F_DATA_VALID bit in
\field{flags}.

If the VIRTIO_NET_F_GUEST_UDP_TUNNEL_GSO_CSUM feature has been negotiated
and either the VIRTIO_NET_HDR_GSO_UDP_TUNNEL_IPV4 bit is set or the
VIRTIO_NET_HDR_GSO_UDP_TUNNEL_IPV6 bit is set in \field{gso_type}, the
device MAY set the VIRTIO_NET_HDR_F_UDP_TUNNEL_CSUM bit in
\field{flags}, if so the device MUST set the packet outer UDP checksum
stored in the receive buffer to the outer UDP pseudo header.

Otherwise, the VIRTIO_NET_F_GUEST_UDP_TUNNEL_GSO_CSUM feature has been
negotiated, either the VIRTIO_NET_HDR_GSO_UDP_TUNNEL_IPV4 bit is set or the
VIRTIO_NET_HDR_GSO_UDP_TUNNEL_IPV6 bit is set in \field{gso_type},
and the bit VIRTIO_NET_HDR_F_UDP_TUNNEL_CSUM is not set in
\field{flags}, the device MUST either provide a zero outer UDP header
checksum or a fully checksummed outer UDP header.

\drivernormative{\paragraph}{Processing of Incoming
Packets}{Device Types / Network Device / Device Operation /
Processing of Incoming Packets}

The driver MUST ignore \field{flag} bits that it does not recognize.

If VIRTIO_NET_HDR_F_NEEDS_CSUM bit in \field{flags} is not set or
if VIRTIO_NET_HDR_F_RSC_INFO bit \field{flags} is set, the
driver MUST NOT use the \field{csum_start} and \field{csum_offset}.

If one of the VIRTIO_NET_F_GUEST_TSO4, TSO6, UFO, USO4 or USO6 options have
been negotiated, the driver MAY use \field{hdr_len} only as a hint about the
transport header size.
The driver MUST NOT rely on \field{hdr_len} to be correct.
\begin{note}
This is due to various bugs in implementations.
\end{note}

If neither VIRTIO_NET_HDR_F_NEEDS_CSUM nor
VIRTIO_NET_HDR_F_DATA_VALID is set, the driver MUST NOT
rely on the packet checksum being correct.

If both the VIRTIO_NET_HDR_GSO_UDP_TUNNEL_IPV4 bit and
the VIRTIO_NET_HDR_GSO_UDP_TUNNEL_IPV6 bit in in \field{gso_type} are set,
the driver MUST NOT accept the packet.

If the VIRTIO_NET_HDR_GSO_UDP_TUNNEL_IPV4 bit or the VIRTIO_NET_HDR_GSO_UDP_TUNNEL_IPV6
bit in \field{gso_type} are not set, the driver MUST NOT use the
\field{outer_th_offset} and \field{inner_nh_offset}.

If either the VIRTIO_NET_HDR_GSO_UDP_TUNNEL_IPV4 bit or
the VIRTIO_NET_HDR_GSO_UDP_TUNNEL_IPV6 bit in \field{gso_type} are set, and any of
the following is true:
\begin{itemize}
\item the VIRTIO_NET_HDR_F_NEEDS_CSUM bit is not set in \field{flags}
\item the VIRTIO_NET_HDR_F_DATA_VALID bit is set in \field{flags}
\item the \field{gso_type} excluding the VIRTIO_NET_HDR_GSO_UDP_TUNNEL_IPV4
bit and the VIRTIO_NET_HDR_GSO_UDP_TUNNEL_IPV6 bit is VIRTIO_NET_HDR_GSO_NONE
\end{itemize}
the driver MUST NOT accept the packet.

If the VIRTIO_NET_HDR_F_UDP_TUNNEL_CSUM bit and the VIRTIO_NET_HDR_F_NEEDS_CSUM
bit in \field{flags} are set,
and both the bits VIRTIO_NET_HDR_GSO_UDP_TUNNEL_IPV4 and
VIRTIO_NET_HDR_GSO_UDP_TUNNEL_IPV6 in \field{gso_type} are not set,
the driver MOST NOT accept the packet.

\paragraph{Hash calculation for incoming packets}
\label{sec:Device Types / Network Device / Device Operation / Processing of Incoming Packets / Hash calculation for incoming packets}

A device attempts to calculate a per-packet hash in the following cases:
\begin{itemize}
\item The feature VIRTIO_NET_F_RSS was negotiated. The device uses the hash to determine the receive virtqueue to place incoming packets.
\item The feature VIRTIO_NET_F_HASH_REPORT was negotiated. The device reports the hash value and the hash type with the packet.
\end{itemize}

If the feature VIRTIO_NET_F_RSS was negotiated:
\begin{itemize}
\item The device uses \field{hash_types} of the virtio_net_rss_config structure as 'Enabled hash types' bitmask.
\item If additionally the feature VIRTIO_NET_F_HASH_TUNNEL was negotiated, the device uses \field{enabled_tunnel_types} of the
      virtnet_hash_tunnel structure as 'Encapsulation types enabled for inner header hash' bitmask.
\item The device uses a key as defined in \field{hash_key_data} and \field{hash_key_length} of the virtio_net_rss_config structure (see
\ref{sec:Device Types / Network Device / Device Operation / Control Virtqueue / Receive-side scaling (RSS) / Setting RSS parameters}).
\end{itemize}

If the feature VIRTIO_NET_F_RSS was not negotiated:
\begin{itemize}
\item The device uses \field{hash_types} of the virtio_net_hash_config structure as 'Enabled hash types' bitmask.
\item If additionally the feature VIRTIO_NET_F_HASH_TUNNEL was negotiated, the device uses \field{enabled_tunnel_types} of the
      virtnet_hash_tunnel structure as 'Encapsulation types enabled for inner header hash' bitmask.
\item The device uses a key as defined in \field{hash_key_data} and \field{hash_key_length} of the virtio_net_hash_config structure (see
\ref{sec:Device Types / Network Device / Device Operation / Control Virtqueue / Automatic receive steering in multiqueue mode / Hash calculation}).
\end{itemize}

Note that if the device offers VIRTIO_NET_F_HASH_REPORT, even if it supports only one pair of virtqueues, it MUST support
at least one of commands of VIRTIO_NET_CTRL_MQ class to configure reported hash parameters:
\begin{itemize}
\item If the device offers VIRTIO_NET_F_RSS, it MUST support VIRTIO_NET_CTRL_MQ_RSS_CONFIG command per
 \ref{sec:Device Types / Network Device / Device Operation / Control Virtqueue / Receive-side scaling (RSS) / Setting RSS parameters}.
\item Otherwise the device MUST support VIRTIO_NET_CTRL_MQ_HASH_CONFIG command per
 \ref{sec:Device Types / Network Device / Device Operation / Control Virtqueue / Automatic receive steering in multiqueue mode / Hash calculation}.
\end{itemize}

The per-packet hash calculation can depend on the IP packet type. See
\hyperref[intro:IP]{[IP]}, \hyperref[intro:UDP]{[UDP]} and \hyperref[intro:TCP]{[TCP]}.

\subparagraph{Supported/enabled hash types}
\label{sec:Device Types / Network Device / Device Operation / Processing of Incoming Packets / Hash calculation for incoming packets / Supported/enabled hash types}
Hash types applicable for IPv4 packets:
\begin{lstlisting}
#define VIRTIO_NET_HASH_TYPE_IPv4              (1 << 0)
#define VIRTIO_NET_HASH_TYPE_TCPv4             (1 << 1)
#define VIRTIO_NET_HASH_TYPE_UDPv4             (1 << 2)
\end{lstlisting}
Hash types applicable for IPv6 packets without extension headers
\begin{lstlisting}
#define VIRTIO_NET_HASH_TYPE_IPv6              (1 << 3)
#define VIRTIO_NET_HASH_TYPE_TCPv6             (1 << 4)
#define VIRTIO_NET_HASH_TYPE_UDPv6             (1 << 5)
\end{lstlisting}
Hash types applicable for IPv6 packets with extension headers
\begin{lstlisting}
#define VIRTIO_NET_HASH_TYPE_IP_EX             (1 << 6)
#define VIRTIO_NET_HASH_TYPE_TCP_EX            (1 << 7)
#define VIRTIO_NET_HASH_TYPE_UDP_EX            (1 << 8)
\end{lstlisting}

\subparagraph{IPv4 packets}
\label{sec:Device Types / Network Device / Device Operation / Processing of Incoming Packets / Hash calculation for incoming packets / IPv4 packets}
The device calculates the hash on IPv4 packets according to 'Enabled hash types' bitmask as follows:
\begin{itemize}
\item If VIRTIO_NET_HASH_TYPE_TCPv4 is set and the packet has
a TCP header, the hash is calculated over the following fields:
\begin{itemize}
\item Source IP address
\item Destination IP address
\item Source TCP port
\item Destination TCP port
\end{itemize}
\item Else if VIRTIO_NET_HASH_TYPE_UDPv4 is set and the
packet has a UDP header, the hash is calculated over the following fields:
\begin{itemize}
\item Source IP address
\item Destination IP address
\item Source UDP port
\item Destination UDP port
\end{itemize}
\item Else if VIRTIO_NET_HASH_TYPE_IPv4 is set, the hash is
calculated over the following fields:
\begin{itemize}
\item Source IP address
\item Destination IP address
\end{itemize}
\item Else the device does not calculate the hash
\end{itemize}

\subparagraph{IPv6 packets without extension header}
\label{sec:Device Types / Network Device / Device Operation / Processing of Incoming Packets / Hash calculation for incoming packets / IPv6 packets without extension header}
The device calculates the hash on IPv6 packets without extension
headers according to 'Enabled hash types' bitmask as follows:
\begin{itemize}
\item If VIRTIO_NET_HASH_TYPE_TCPv6 is set and the packet has
a TCPv6 header, the hash is calculated over the following fields:
\begin{itemize}
\item Source IPv6 address
\item Destination IPv6 address
\item Source TCP port
\item Destination TCP port
\end{itemize}
\item Else if VIRTIO_NET_HASH_TYPE_UDPv6 is set and the
packet has a UDPv6 header, the hash is calculated over the following fields:
\begin{itemize}
\item Source IPv6 address
\item Destination IPv6 address
\item Source UDP port
\item Destination UDP port
\end{itemize}
\item Else if VIRTIO_NET_HASH_TYPE_IPv6 is set, the hash is
calculated over the following fields:
\begin{itemize}
\item Source IPv6 address
\item Destination IPv6 address
\end{itemize}
\item Else the device does not calculate the hash
\end{itemize}

\subparagraph{IPv6 packets with extension header}
\label{sec:Device Types / Network Device / Device Operation / Processing of Incoming Packets / Hash calculation for incoming packets / IPv6 packets with extension header}
The device calculates the hash on IPv6 packets with extension
headers according to 'Enabled hash types' bitmask as follows:
\begin{itemize}
\item If VIRTIO_NET_HASH_TYPE_TCP_EX is set and the packet
has a TCPv6 header, the hash is calculated over the following fields:
\begin{itemize}
\item Home address from the home address option in the IPv6 destination options header. If the extension header is not present, use the Source IPv6 address.
\item IPv6 address that is contained in the Routing-Header-Type-2 from the associated extension header. If the extension header is not present, use the Destination IPv6 address.
\item Source TCP port
\item Destination TCP port
\end{itemize}
\item Else if VIRTIO_NET_HASH_TYPE_UDP_EX is set and the
packet has a UDPv6 header, the hash is calculated over the following fields:
\begin{itemize}
\item Home address from the home address option in the IPv6 destination options header. If the extension header is not present, use the Source IPv6 address.
\item IPv6 address that is contained in the Routing-Header-Type-2 from the associated extension header. If the extension header is not present, use the Destination IPv6 address.
\item Source UDP port
\item Destination UDP port
\end{itemize}
\item Else if VIRTIO_NET_HASH_TYPE_IP_EX is set, the hash is
calculated over the following fields:
\begin{itemize}
\item Home address from the home address option in the IPv6 destination options header. If the extension header is not present, use the Source IPv6 address.
\item IPv6 address that is contained in the Routing-Header-Type-2 from the associated extension header. If the extension header is not present, use the Destination IPv6 address.
\end{itemize}
\item Else skip IPv6 extension headers and calculate the hash as
defined for an IPv6 packet without extension headers
(see \ref{sec:Device Types / Network Device / Device Operation / Processing of Incoming Packets / Hash calculation for incoming packets / IPv6 packets without extension header}).
\end{itemize}

\paragraph{Inner Header Hash}
\label{sec:Device Types / Network Device / Device Operation / Processing of Incoming Packets / Inner Header Hash}

If VIRTIO_NET_F_HASH_TUNNEL has been negotiated, the driver can send the command
VIRTIO_NET_CTRL_HASH_TUNNEL_SET to configure the calculation of the inner header hash.

\begin{lstlisting}
struct virtnet_hash_tunnel {
    le32 enabled_tunnel_types;
};

#define VIRTIO_NET_CTRL_HASH_TUNNEL 7
 #define VIRTIO_NET_CTRL_HASH_TUNNEL_SET 0
\end{lstlisting}

Field \field{enabled_tunnel_types} contains the bitmask of encapsulation types enabled for inner header hash.
See \ref{sec:Device Types / Network Device / Device Operation / Processing of Incoming Packets /
Hash calculation for incoming packets / Encapsulation types supported/enabled for inner header hash}.

The class VIRTIO_NET_CTRL_HASH_TUNNEL has one command:
VIRTIO_NET_CTRL_HASH_TUNNEL_SET sets \field{enabled_tunnel_types} for the device using the
virtnet_hash_tunnel structure, which is read-only for the device.

Inner header hash is disabled by VIRTIO_NET_CTRL_HASH_TUNNEL_SET with \field{enabled_tunnel_types} set to 0.

Initially (before the driver sends any VIRTIO_NET_CTRL_HASH_TUNNEL_SET command) all
encapsulation types are disabled for inner header hash.

\subparagraph{Encapsulated packet}
\label{sec:Device Types / Network Device / Device Operation / Processing of Incoming Packets / Hash calculation for incoming packets / Encapsulated packet}

Multiple tunneling protocols allow encapsulating an inner, payload packet in an outer, encapsulated packet.
The encapsulated packet thus contains an outer header and an inner header, and the device calculates the
hash over either the inner header or the outer header.

If VIRTIO_NET_F_HASH_TUNNEL is negotiated and a received encapsulated packet's outer header matches one of the
encapsulation types enabled in \field{enabled_tunnel_types}, then the device uses the inner header for hash
calculations (only a single level of encapsulation is currently supported).

If VIRTIO_NET_F_HASH_TUNNEL is negotiated and a received packet's (outer) header does not match any encapsulation
types enabled in \field{enabled_tunnel_types}, then the device uses the outer header for hash calculations.

\subparagraph{Encapsulation types supported/enabled for inner header hash}
\label{sec:Device Types / Network Device / Device Operation / Processing of Incoming Packets /
Hash calculation for incoming packets / Encapsulation types supported/enabled for inner header hash}

Encapsulation types applicable for inner header hash:
\begin{lstlisting}[escapechar=|]
#define VIRTIO_NET_HASH_TUNNEL_TYPE_GRE_2784    (1 << 0) /* |\hyperref[intro:rfc2784]{[RFC2784]}| */
#define VIRTIO_NET_HASH_TUNNEL_TYPE_GRE_2890    (1 << 1) /* |\hyperref[intro:rfc2890]{[RFC2890]}| */
#define VIRTIO_NET_HASH_TUNNEL_TYPE_GRE_7676    (1 << 2) /* |\hyperref[intro:rfc7676]{[RFC7676]}| */
#define VIRTIO_NET_HASH_TUNNEL_TYPE_GRE_UDP     (1 << 3) /* |\hyperref[intro:rfc8086]{[GRE-in-UDP]}| */
#define VIRTIO_NET_HASH_TUNNEL_TYPE_VXLAN       (1 << 4) /* |\hyperref[intro:vxlan]{[VXLAN]}| */
#define VIRTIO_NET_HASH_TUNNEL_TYPE_VXLAN_GPE   (1 << 5) /* |\hyperref[intro:vxlan-gpe]{[VXLAN-GPE]}| */
#define VIRTIO_NET_HASH_TUNNEL_TYPE_GENEVE      (1 << 6) /* |\hyperref[intro:geneve]{[GENEVE]}| */
#define VIRTIO_NET_HASH_TUNNEL_TYPE_IPIP        (1 << 7) /* |\hyperref[intro:ipip]{[IPIP]}| */
#define VIRTIO_NET_HASH_TUNNEL_TYPE_NVGRE       (1 << 8) /* |\hyperref[intro:nvgre]{[NVGRE]}| */
\end{lstlisting}

\subparagraph{Advice}
Example uses of the inner header hash:
\begin{itemize}
\item Legacy tunneling protocols, lacking the outer header entropy, can use RSS with the inner header hash to
      distribute flows with identical outer but different inner headers across various queues, improving performance.
\item Identify an inner flow distributed across multiple outer tunnels.
\end{itemize}

As using the inner header hash completely discards the outer header entropy, care must be taken
if the inner header is controlled by an adversary, as the adversary can then intentionally create
configurations with insufficient entropy.

Besides disabling the inner header hash, mitigations would depend on how the hash is used. When the hash
use is limited to the RSS queue selection, the inner header hash may have quality of service (QoS) limitations.

\devicenormative{\subparagraph}{Inner Header Hash}{Device Types / Network Device / Device Operation / Control Virtqueue / Inner Header Hash}

If the (outer) header of the received packet does not match any encapsulation types enabled
in \field{enabled_tunnel_types}, the device MUST calculate the hash on the outer header.

If the device receives any bits in \field{enabled_tunnel_types} which are not set in \field{supported_tunnel_types},
it SHOULD respond to the VIRTIO_NET_CTRL_HASH_TUNNEL_SET command with VIRTIO_NET_ERR.

If the driver sets \field{enabled_tunnel_types} to 0 through VIRTIO_NET_CTRL_HASH_TUNNEL_SET or upon the device reset,
the device MUST disable the inner header hash for all encapsulation types.

\drivernormative{\subparagraph}{Inner Header Hash}{Device Types / Network Device / Device Operation / Control Virtqueue / Inner Header Hash}

The driver MUST have negotiated the VIRTIO_NET_F_HASH_TUNNEL feature when issuing the VIRTIO_NET_CTRL_HASH_TUNNEL_SET command.

The driver MUST NOT set any bits in \field{enabled_tunnel_types} which are not set in \field{supported_tunnel_types}.

The driver MUST ignore bits in \field{supported_tunnel_types} which are not documented in this specification.

\paragraph{Hash reporting for incoming packets}
\label{sec:Device Types / Network Device / Device Operation / Processing of Incoming Packets / Hash reporting for incoming packets}

If VIRTIO_NET_F_HASH_REPORT was negotiated and
 the device has calculated the hash for the packet, the device fills \field{hash_report} with the report type of calculated hash
and \field{hash_value} with the value of calculated hash.

If VIRTIO_NET_F_HASH_REPORT was negotiated but due to any reason the
hash was not calculated, the device sets \field{hash_report} to VIRTIO_NET_HASH_REPORT_NONE.

Possible values that the device can report in \field{hash_report} are defined below.
They correspond to supported hash types defined in
\ref{sec:Device Types / Network Device / Device Operation / Processing of Incoming Packets / Hash calculation for incoming packets / Supported/enabled hash types}
as follows:

VIRTIO_NET_HASH_TYPE_XXX = 1 << (VIRTIO_NET_HASH_REPORT_XXX - 1)

\begin{lstlisting}
#define VIRTIO_NET_HASH_REPORT_NONE            0
#define VIRTIO_NET_HASH_REPORT_IPv4            1
#define VIRTIO_NET_HASH_REPORT_TCPv4           2
#define VIRTIO_NET_HASH_REPORT_UDPv4           3
#define VIRTIO_NET_HASH_REPORT_IPv6            4
#define VIRTIO_NET_HASH_REPORT_TCPv6           5
#define VIRTIO_NET_HASH_REPORT_UDPv6           6
#define VIRTIO_NET_HASH_REPORT_IPv6_EX         7
#define VIRTIO_NET_HASH_REPORT_TCPv6_EX        8
#define VIRTIO_NET_HASH_REPORT_UDPv6_EX        9
\end{lstlisting}

\subsubsection{Control Virtqueue}\label{sec:Device Types / Network Device / Device Operation / Control Virtqueue}

The driver uses the control virtqueue (if VIRTIO_NET_F_CTRL_VQ is
negotiated) to send commands to manipulate various features of
the device which would not easily map into the configuration
space.

All commands are of the following form:

\begin{lstlisting}
struct virtio_net_ctrl {
        u8 class;
        u8 command;
        u8 command-specific-data[];
        u8 ack;
        u8 command-specific-result[];
};

/* ack values */
#define VIRTIO_NET_OK     0
#define VIRTIO_NET_ERR    1
\end{lstlisting}

The \field{class}, \field{command} and command-specific-data are set by the
driver, and the device sets the \field{ack} byte and optionally
\field{command-specific-result}. There is little the driver can
do except issue a diagnostic if \field{ack} is not VIRTIO_NET_OK.

The command VIRTIO_NET_CTRL_STATS_QUERY and VIRTIO_NET_CTRL_STATS_GET contain
\field{command-specific-result}.

\paragraph{Packet Receive Filtering}\label{sec:Device Types / Network Device / Device Operation / Control Virtqueue / Packet Receive Filtering}
\label{sec:Device Types / Network Device / Device Operation / Control Virtqueue / Setting Promiscuous Mode}%old label for latexdiff

If the VIRTIO_NET_F_CTRL_RX and VIRTIO_NET_F_CTRL_RX_EXTRA
features are negotiated, the driver can send control commands for
promiscuous mode, multicast, unicast and broadcast receiving.

\begin{note}
In general, these commands are best-effort: unwanted
packets could still arrive.
\end{note}

\begin{lstlisting}
#define VIRTIO_NET_CTRL_RX    0
 #define VIRTIO_NET_CTRL_RX_PROMISC      0
 #define VIRTIO_NET_CTRL_RX_ALLMULTI     1
 #define VIRTIO_NET_CTRL_RX_ALLUNI       2
 #define VIRTIO_NET_CTRL_RX_NOMULTI      3
 #define VIRTIO_NET_CTRL_RX_NOUNI        4
 #define VIRTIO_NET_CTRL_RX_NOBCAST      5
\end{lstlisting}


\devicenormative{\subparagraph}{Packet Receive Filtering}{Device Types / Network Device / Device Operation / Control Virtqueue / Packet Receive Filtering}

If the VIRTIO_NET_F_CTRL_RX feature has been negotiated,
the device MUST support the following VIRTIO_NET_CTRL_RX class
commands:
\begin{itemize}
\item VIRTIO_NET_CTRL_RX_PROMISC turns promiscuous mode on and
off. The command-specific-data is one byte containing 0 (off) or
1 (on). If promiscuous mode is on, the device SHOULD receive all
incoming packets.
This SHOULD take effect even if one of the other modes set by
a VIRTIO_NET_CTRL_RX class command is on.
\item VIRTIO_NET_CTRL_RX_ALLMULTI turns all-multicast receive on and
off. The command-specific-data is one byte containing 0 (off) or
1 (on). When all-multicast receive is on the device SHOULD allow
all incoming multicast packets.
\end{itemize}

If the VIRTIO_NET_F_CTRL_RX_EXTRA feature has been negotiated,
the device MUST support the following VIRTIO_NET_CTRL_RX class
commands:
\begin{itemize}
\item VIRTIO_NET_CTRL_RX_ALLUNI turns all-unicast receive on and
off. The command-specific-data is one byte containing 0 (off) or
1 (on). When all-unicast receive is on the device SHOULD allow
all incoming unicast packets.
\item VIRTIO_NET_CTRL_RX_NOMULTI suppresses multicast receive.
The command-specific-data is one byte containing 0 (multicast
receive allowed) or 1 (multicast receive suppressed).
When multicast receive is suppressed, the device SHOULD NOT
send multicast packets to the driver.
This SHOULD take effect even if VIRTIO_NET_CTRL_RX_ALLMULTI is on.
This filter SHOULD NOT apply to broadcast packets.
\item VIRTIO_NET_CTRL_RX_NOUNI suppresses unicast receive.
The command-specific-data is one byte containing 0 (unicast
receive allowed) or 1 (unicast receive suppressed).
When unicast receive is suppressed, the device SHOULD NOT
send unicast packets to the driver.
This SHOULD take effect even if VIRTIO_NET_CTRL_RX_ALLUNI is on.
\item VIRTIO_NET_CTRL_RX_NOBCAST suppresses broadcast receive.
The command-specific-data is one byte containing 0 (broadcast
receive allowed) or 1 (broadcast receive suppressed).
When broadcast receive is suppressed, the device SHOULD NOT
send broadcast packets to the driver.
This SHOULD take effect even if VIRTIO_NET_CTRL_RX_ALLMULTI is on.
\end{itemize}

\drivernormative{\subparagraph}{Packet Receive Filtering}{Device Types / Network Device / Device Operation / Control Virtqueue / Packet Receive Filtering}

If the VIRTIO_NET_F_CTRL_RX feature has not been negotiated,
the driver MUST NOT issue commands VIRTIO_NET_CTRL_RX_PROMISC or
VIRTIO_NET_CTRL_RX_ALLMULTI.

If the VIRTIO_NET_F_CTRL_RX_EXTRA feature has not been negotiated,
the driver MUST NOT issue commands
 VIRTIO_NET_CTRL_RX_ALLUNI,
 VIRTIO_NET_CTRL_RX_NOMULTI,
 VIRTIO_NET_CTRL_RX_NOUNI or
 VIRTIO_NET_CTRL_RX_NOBCAST.

\paragraph{Setting MAC Address Filtering}\label{sec:Device Types / Network Device / Device Operation / Control Virtqueue / Setting MAC Address Filtering}

If the VIRTIO_NET_F_CTRL_RX feature is negotiated, the driver can
send control commands for MAC address filtering.

\begin{lstlisting}
struct virtio_net_ctrl_mac {
        le32 entries;
        u8 macs[entries][6];
};

#define VIRTIO_NET_CTRL_MAC    1
 #define VIRTIO_NET_CTRL_MAC_TABLE_SET        0
 #define VIRTIO_NET_CTRL_MAC_ADDR_SET         1
\end{lstlisting}

The device can filter incoming packets by any number of destination
MAC addresses\footnote{Since there are no guarantees, it can use a hash filter or
silently switch to allmulti or promiscuous mode if it is given too
many addresses.
}. This table is set using the class
VIRTIO_NET_CTRL_MAC and the command VIRTIO_NET_CTRL_MAC_TABLE_SET. The
command-specific-data is two variable length tables of 6-byte MAC
addresses (as described in struct virtio_net_ctrl_mac). The first table contains unicast addresses, and the second
contains multicast addresses.

The VIRTIO_NET_CTRL_MAC_ADDR_SET command is used to set the
default MAC address which rx filtering
accepts (and if VIRTIO_NET_F_MAC has been negotiated,
this will be reflected in \field{mac} in config space).

The command-specific-data for VIRTIO_NET_CTRL_MAC_ADDR_SET is
the 6-byte MAC address.

\devicenormative{\subparagraph}{Setting MAC Address Filtering}{Device Types / Network Device / Device Operation / Control Virtqueue / Setting MAC Address Filtering}

The device MUST have an empty MAC filtering table on reset.

The device MUST update the MAC filtering table before it consumes
the VIRTIO_NET_CTRL_MAC_TABLE_SET command.

The device MUST update \field{mac} in config space before it consumes
the VIRTIO_NET_CTRL_MAC_ADDR_SET command, if VIRTIO_NET_F_MAC has
been negotiated.

The device SHOULD drop incoming packets which have a destination MAC which
matches neither the \field{mac} (or that set with VIRTIO_NET_CTRL_MAC_ADDR_SET)
nor the MAC filtering table.

\drivernormative{\subparagraph}{Setting MAC Address Filtering}{Device Types / Network Device / Device Operation / Control Virtqueue / Setting MAC Address Filtering}

If VIRTIO_NET_F_CTRL_RX has not been negotiated,
the driver MUST NOT issue VIRTIO_NET_CTRL_MAC class commands.

If VIRTIO_NET_F_CTRL_RX has been negotiated,
the driver SHOULD issue VIRTIO_NET_CTRL_MAC_ADDR_SET
to set the default mac if it is different from \field{mac}.

The driver MUST follow the VIRTIO_NET_CTRL_MAC_TABLE_SET command
by a le32 number, followed by that number of non-multicast
MAC addresses, followed by another le32 number, followed by
that number of multicast addresses.  Either number MAY be 0.

\subparagraph{Legacy Interface: Setting MAC Address Filtering}\label{sec:Device Types / Network Device / Device Operation / Control Virtqueue / Setting MAC Address Filtering / Legacy Interface: Setting MAC Address Filtering}
When using the legacy interface, transitional devices and drivers
MUST format \field{entries} in struct virtio_net_ctrl_mac
according to the native endian of the guest rather than
(necessarily when not using the legacy interface) little-endian.

Legacy drivers that didn't negotiate VIRTIO_NET_F_CTRL_MAC_ADDR
changed \field{mac} in config space when NIC is accepting
incoming packets. These drivers always wrote the mac value from
first to last byte, therefore after detecting such drivers,
a transitional device MAY defer MAC update, or MAY defer
processing incoming packets until driver writes the last byte
of \field{mac} in the config space.

\paragraph{VLAN Filtering}\label{sec:Device Types / Network Device / Device Operation / Control Virtqueue / VLAN Filtering}

If the driver negotiates the VIRTIO_NET_F_CTRL_VLAN feature, it
can control a VLAN filter table in the device. The VLAN filter
table applies only to VLAN tagged packets.

When VIRTIO_NET_F_CTRL_VLAN is negotiated, the device starts with
an empty VLAN filter table.

\begin{note}
Similar to the MAC address based filtering, the VLAN filtering
is also best-effort: unwanted packets could still arrive.
\end{note}

\begin{lstlisting}
#define VIRTIO_NET_CTRL_VLAN       2
 #define VIRTIO_NET_CTRL_VLAN_ADD             0
 #define VIRTIO_NET_CTRL_VLAN_DEL             1
\end{lstlisting}

Both the VIRTIO_NET_CTRL_VLAN_ADD and VIRTIO_NET_CTRL_VLAN_DEL
command take a little-endian 16-bit VLAN id as the command-specific-data.

VIRTIO_NET_CTRL_VLAN_ADD command adds the specified VLAN to the
VLAN filter table.

VIRTIO_NET_CTRL_VLAN_DEL command removes the specified VLAN from
the VLAN filter table.

\devicenormative{\subparagraph}{VLAN Filtering}{Device Types / Network Device / Device Operation / Control Virtqueue / VLAN Filtering}

When VIRTIO_NET_F_CTRL_VLAN is not negotiated, the device MUST
accept all VLAN tagged packets.

When VIRTIO_NET_F_CTRL_VLAN is negotiated, the device MUST
accept all VLAN tagged packets whose VLAN tag is present in
the VLAN filter table and SHOULD drop all VLAN tagged packets
whose VLAN tag is absent in the VLAN filter table.

\subparagraph{Legacy Interface: VLAN Filtering}\label{sec:Device Types / Network Device / Device Operation / Control Virtqueue / VLAN Filtering / Legacy Interface: VLAN Filtering}
When using the legacy interface, transitional devices and drivers
MUST format the VLAN id
according to the native endian of the guest rather than
(necessarily when not using the legacy interface) little-endian.

\paragraph{Gratuitous Packet Sending}\label{sec:Device Types / Network Device / Device Operation / Control Virtqueue / Gratuitous Packet Sending}

If the driver negotiates the VIRTIO_NET_F_GUEST_ANNOUNCE (depends
on VIRTIO_NET_F_CTRL_VQ), the device can ask the driver to send gratuitous
packets; this is usually done after the guest has been physically
migrated, and needs to announce its presence on the new network
links. (As hypervisor does not have the knowledge of guest
network configuration (eg. tagged vlan) it is simplest to prod
the guest in this way).

\begin{lstlisting}
#define VIRTIO_NET_CTRL_ANNOUNCE       3
 #define VIRTIO_NET_CTRL_ANNOUNCE_ACK             0
\end{lstlisting}

The driver checks VIRTIO_NET_S_ANNOUNCE bit in the device configuration \field{status} field
when it notices the changes of device configuration. The
command VIRTIO_NET_CTRL_ANNOUNCE_ACK is used to indicate that
driver has received the notification and device clears the
VIRTIO_NET_S_ANNOUNCE bit in \field{status}.

Processing this notification involves:

\begin{enumerate}
\item Sending the gratuitous packets (eg. ARP) or marking there are pending
  gratuitous packets to be sent and letting deferred routine to
  send them.

\item Sending VIRTIO_NET_CTRL_ANNOUNCE_ACK command through control
  vq.
\end{enumerate}

\drivernormative{\subparagraph}{Gratuitous Packet Sending}{Device Types / Network Device / Device Operation / Control Virtqueue / Gratuitous Packet Sending}

If the driver negotiates VIRTIO_NET_F_GUEST_ANNOUNCE, it SHOULD notify
network peers of its new location after it sees the VIRTIO_NET_S_ANNOUNCE bit
in \field{status}.  The driver MUST send a command on the command queue
with class VIRTIO_NET_CTRL_ANNOUNCE and command VIRTIO_NET_CTRL_ANNOUNCE_ACK.

\devicenormative{\subparagraph}{Gratuitous Packet Sending}{Device Types / Network Device / Device Operation / Control Virtqueue / Gratuitous Packet Sending}

If VIRTIO_NET_F_GUEST_ANNOUNCE is negotiated, the device MUST clear the
VIRTIO_NET_S_ANNOUNCE bit in \field{status} upon receipt of a command buffer
with class VIRTIO_NET_CTRL_ANNOUNCE and command VIRTIO_NET_CTRL_ANNOUNCE_ACK
before marking the buffer as used.

\paragraph{Device operation in multiqueue mode}\label{sec:Device Types / Network Device / Device Operation / Control Virtqueue / Device operation in multiqueue mode}

This specification defines the following modes that a device MAY implement for operation with multiple transmit/receive virtqueues:
\begin{itemize}
\item Automatic receive steering as defined in \ref{sec:Device Types / Network Device / Device Operation / Control Virtqueue / Automatic receive steering in multiqueue mode}.
 If a device supports this mode, it offers the VIRTIO_NET_F_MQ feature bit.
\item Receive-side scaling as defined in \ref{devicenormative:Device Types / Network Device / Device Operation / Control Virtqueue / Receive-side scaling (RSS) / RSS processing}.
 If a device supports this mode, it offers the VIRTIO_NET_F_RSS feature bit.
\end{itemize}

A device MAY support one of these features or both. The driver MAY negotiate any set of these features that the device supports.

Multiqueue is disabled by default.

The driver enables multiqueue by sending a command using \field{class} VIRTIO_NET_CTRL_MQ. The \field{command} selects the mode of multiqueue operation, as follows:
\begin{lstlisting}
#define VIRTIO_NET_CTRL_MQ    4
 #define VIRTIO_NET_CTRL_MQ_VQ_PAIRS_SET        0 (for automatic receive steering)
 #define VIRTIO_NET_CTRL_MQ_RSS_CONFIG          1 (for configurable receive steering)
 #define VIRTIO_NET_CTRL_MQ_HASH_CONFIG         2 (for configurable hash calculation)
\end{lstlisting}

If more than one multiqueue mode is negotiated, the resulting device configuration is defined by the last command sent by the driver.

\paragraph{Automatic receive steering in multiqueue mode}\label{sec:Device Types / Network Device / Device Operation / Control Virtqueue / Automatic receive steering in multiqueue mode}

If the driver negotiates the VIRTIO_NET_F_MQ feature bit (depends on VIRTIO_NET_F_CTRL_VQ), it MAY transmit outgoing packets on one
of the multiple transmitq1\ldots transmitqN and ask the device to
queue incoming packets into one of the multiple receiveq1\ldots receiveqN
depending on the packet flow.

The driver enables multiqueue by
sending the VIRTIO_NET_CTRL_MQ_VQ_PAIRS_SET command, specifying
the number of the transmit and receive queues to be used up to
\field{max_virtqueue_pairs}; subsequently,
transmitq1\ldots transmitqn and receiveq1\ldots receiveqn where
n=\field{virtqueue_pairs} MAY be used.
\begin{lstlisting}
struct virtio_net_ctrl_mq_pairs_set {
       le16 virtqueue_pairs;
};
#define VIRTIO_NET_CTRL_MQ_VQ_PAIRS_MIN        1
#define VIRTIO_NET_CTRL_MQ_VQ_PAIRS_MAX        0x8000

\end{lstlisting}

When multiqueue is enabled by VIRTIO_NET_CTRL_MQ_VQ_PAIRS_SET command, the device MUST use automatic receive steering
based on packet flow. Programming of the receive steering
classificator is implicit. After the driver transmitted a packet of a
flow on transmitqX, the device SHOULD cause incoming packets for that flow to
be steered to receiveqX. For uni-directional protocols, or where
no packets have been transmitted yet, the device MAY steer a packet
to a random queue out of the specified receiveq1\ldots receiveqn.

Multiqueue is disabled by VIRTIO_NET_CTRL_MQ_VQ_PAIRS_SET with \field{virtqueue_pairs} to 1 (this is
the default) and waiting for the device to use the command buffer.

\drivernormative{\subparagraph}{Automatic receive steering in multiqueue mode}{Device Types / Network Device / Device Operation / Control Virtqueue / Automatic receive steering in multiqueue mode}

The driver MUST configure the virtqueues before enabling them with the
VIRTIO_NET_CTRL_MQ_VQ_PAIRS_SET command.

The driver MUST NOT request a \field{virtqueue_pairs} of 0 or
greater than \field{max_virtqueue_pairs} in the device configuration space.

The driver MUST queue packets only on any transmitq1 before the
VIRTIO_NET_CTRL_MQ_VQ_PAIRS_SET command.

The driver MUST NOT queue packets on transmit queues greater than
\field{virtqueue_pairs} once it has placed the VIRTIO_NET_CTRL_MQ_VQ_PAIRS_SET command in the available ring.

\devicenormative{\subparagraph}{Automatic receive steering in multiqueue mode}{Device Types / Network Device / Device Operation / Control Virtqueue / Automatic receive steering in multiqueue mode}

After initialization of reset, the device MUST queue packets only on receiveq1.

The device MUST NOT queue packets on receive queues greater than
\field{virtqueue_pairs} once it has placed the
VIRTIO_NET_CTRL_MQ_VQ_PAIRS_SET command in a used buffer.

If the destination receive queue is being reset (See \ref{sec:Basic Facilities of a Virtio Device / Virtqueues / Virtqueue Reset}),
the device SHOULD re-select another random queue. If all receive queues are
being reset, the device MUST drop the packet.

\subparagraph{Legacy Interface: Automatic receive steering in multiqueue mode}\label{sec:Device Types / Network Device / Device Operation / Control Virtqueue / Automatic receive steering in multiqueue mode / Legacy Interface: Automatic receive steering in multiqueue mode}
When using the legacy interface, transitional devices and drivers
MUST format \field{virtqueue_pairs}
according to the native endian of the guest rather than
(necessarily when not using the legacy interface) little-endian.

\subparagraph{Hash calculation}\label{sec:Device Types / Network Device / Device Operation / Control Virtqueue / Automatic receive steering in multiqueue mode / Hash calculation}
If VIRTIO_NET_F_HASH_REPORT was negotiated and the device uses automatic receive steering,
the device MUST support a command to configure hash calculation parameters.

The driver provides parameters for hash calculation as follows:

\field{class} VIRTIO_NET_CTRL_MQ, \field{command} VIRTIO_NET_CTRL_MQ_HASH_CONFIG.

The \field{command-specific-data} has following format:
\begin{lstlisting}
struct virtio_net_hash_config {
    le32 hash_types;
    le16 reserved[4];
    u8 hash_key_length;
    u8 hash_key_data[hash_key_length];
};
\end{lstlisting}
Field \field{hash_types} contains a bitmask of allowed hash types as
defined in
\ref{sec:Device Types / Network Device / Device Operation / Processing of Incoming Packets / Hash calculation for incoming packets / Supported/enabled hash types}.
Initially the device has all hash types disabled and reports only VIRTIO_NET_HASH_REPORT_NONE.

Field \field{reserved} MUST contain zeroes. It is defined to make the structure to match the layout of virtio_net_rss_config structure,
defined in \ref{sec:Device Types / Network Device / Device Operation / Control Virtqueue / Receive-side scaling (RSS)}.

Fields \field{hash_key_length} and \field{hash_key_data} define the key to be used in hash calculation.

\paragraph{Receive-side scaling (RSS)}\label{sec:Device Types / Network Device / Device Operation / Control Virtqueue / Receive-side scaling (RSS)}
A device offers the feature VIRTIO_NET_F_RSS if it supports RSS receive steering with Toeplitz hash calculation and configurable parameters.

A driver queries RSS capabilities of the device by reading device configuration as defined in \ref{sec:Device Types / Network Device / Device configuration layout}

\subparagraph{Setting RSS parameters}\label{sec:Device Types / Network Device / Device Operation / Control Virtqueue / Receive-side scaling (RSS) / Setting RSS parameters}

Driver sends a VIRTIO_NET_CTRL_MQ_RSS_CONFIG command using the following format for \field{command-specific-data}:
\begin{lstlisting}
struct rss_rq_id {
   le16 vq_index_1_16: 15; /* Bits 1 to 16 of the virtqueue index */
   le16 reserved: 1; /* Set to zero */
};

struct virtio_net_rss_config {
    le32 hash_types;
    le16 indirection_table_mask;
    struct rss_rq_id unclassified_queue;
    struct rss_rq_id indirection_table[indirection_table_length];
    le16 max_tx_vq;
    u8 hash_key_length;
    u8 hash_key_data[hash_key_length];
};
\end{lstlisting}
Field \field{hash_types} contains a bitmask of allowed hash types as
defined in
\ref{sec:Device Types / Network Device / Device Operation / Processing of Incoming Packets / Hash calculation for incoming packets / Supported/enabled hash types}.

Field \field{indirection_table_mask} is a mask to be applied to
the calculated hash to produce an index in the
\field{indirection_table} array.
Number of entries in \field{indirection_table} is (\field{indirection_table_mask} + 1).

\field{rss_rq_id} is a receive virtqueue id. \field{vq_index_1_16}
consists of bits 1 to 16 of a virtqueue index. For example, a
\field{vq_index_1_16} value of 3 corresponds to virtqueue index 6,
which maps to receiveq4.

Field \field{unclassified_queue} specifies the receive virtqueue id in which to
place unclassified packets.

Field \field{indirection_table} is an array of receive virtqueues ids.

A driver sets \field{max_tx_vq} to inform a device how many transmit virtqueues it may use (transmitq1\ldots transmitq \field{max_tx_vq}).

Fields \field{hash_key_length} and \field{hash_key_data} define the key to be used in hash calculation.

\drivernormative{\subparagraph}{Setting RSS parameters}{Device Types / Network Device / Device Operation / Control Virtqueue / Receive-side scaling (RSS) }

A driver MUST NOT send the VIRTIO_NET_CTRL_MQ_RSS_CONFIG command if the feature VIRTIO_NET_F_RSS has not been negotiated.

A driver MUST fill the \field{indirection_table} array only with
enabled receive virtqueues ids.

The number of entries in \field{indirection_table} (\field{indirection_table_mask} + 1) MUST be a power of two.

A driver MUST use \field{indirection_table_mask} values that are less than \field{rss_max_indirection_table_length} reported by a device.

A driver MUST NOT set any VIRTIO_NET_HASH_TYPE_ flags that are not supported by a device.

\devicenormative{\subparagraph}{RSS processing}{Device Types / Network Device / Device Operation / Control Virtqueue / Receive-side scaling (RSS) / RSS processing}
The device MUST determine the destination queue for a network packet as follows:
\begin{itemize}
\item Calculate the hash of the packet as defined in \ref{sec:Device Types / Network Device / Device Operation / Processing of Incoming Packets / Hash calculation for incoming packets}.
\item If the device did not calculate the hash for the specific packet, the device directs the packet to the receiveq specified by \field{unclassified_queue} of virtio_net_rss_config structure.
\item Apply \field{indirection_table_mask} to the calculated hash
and use the result as the index in the indirection table to get
the destination receive virtqueue id.
\item If the destination receive queue is being reset (See \ref{sec:Basic Facilities of a Virtio Device / Virtqueues / Virtqueue Reset}), the device MUST drop the packet.
\end{itemize}

\paragraph{RSS Context}\label{sec:Device Types / Network Device / Device Operation / Control Virtqueue / RSS Context}

An RSS context consists of configurable parameters specified by \ref{sec:Device Types / Network Device
/ Device Operation / Control Virtqueue / Receive-side scaling (RSS)}.

The RSS configuration supported by VIRTIO_NET_F_RSS is considered the default RSS configuration.

The device offers the feature VIRTIO_NET_F_RSS_CONTEXT if it supports one or multiple RSS contexts
(excluding the default RSS configuration) and configurable parameters.

\subparagraph{Querying RSS Context Capability}\label{sec:Device Types / Network Device / Device Operation / Control Virtqueue / RSS Context / Querying RSS Context Capability}

\begin{lstlisting}
#define VIRTNET_RSS_CTX_CTRL 9
 #define VIRTNET_RSS_CTX_CTRL_CAP_GET  0
 #define VIRTNET_RSS_CTX_CTRL_ADD      1
 #define VIRTNET_RSS_CTX_CTRL_MOD      2
 #define VIRTNET_RSS_CTX_CTRL_DEL      3

struct virtnet_rss_ctx_cap {
    le16 max_rss_contexts;
}
\end{lstlisting}

Field \field{max_rss_contexts} specifies the maximum number of RSS contexts \ref{sec:Device Types / Network Device /
Device Operation / Control Virtqueue / RSS Context} supported by the device.

The driver queries the RSS context capability of the device by sending the command VIRTNET_RSS_CTX_CTRL_CAP_GET
with the structure virtnet_rss_ctx_cap.

For the command VIRTNET_RSS_CTX_CTRL_CAP_GET, the structure virtnet_rss_ctx_cap is write-only for the device.

\subparagraph{Setting RSS Context Parameters}\label{sec:Device Types / Network Device / Device Operation / Control Virtqueue / RSS Context / Setting RSS Context Parameters}

\begin{lstlisting}
struct virtnet_rss_ctx_add_modify {
    le16 rss_ctx_id;
    u8 reserved[6];
    struct virtio_net_rss_config rss;
};

struct virtnet_rss_ctx_del {
    le16 rss_ctx_id;
};
\end{lstlisting}

RSS context parameters:
\begin{itemize}
\item  \field{rss_ctx_id}: ID of the specific RSS context.
\item  \field{rss}: RSS context parameters of the specific RSS context whose id is \field{rss_ctx_id}.
\end{itemize}

\field{reserved} is reserved and it is ignored by the device.

If the feature VIRTIO_NET_F_RSS_CONTEXT has been negotiated, the driver can send the following
VIRTNET_RSS_CTX_CTRL class commands:
\begin{enumerate}
\item VIRTNET_RSS_CTX_CTRL_ADD: use the structure virtnet_rss_ctx_add_modify to
       add an RSS context configured as \field{rss} and id as \field{rss_ctx_id} for the device.
\item VIRTNET_RSS_CTX_CTRL_MOD: use the structure virtnet_rss_ctx_add_modify to
       configure parameters of the RSS context whose id is \field{rss_ctx_id} as \field{rss} for the device.
\item VIRTNET_RSS_CTX_CTRL_DEL: use the structure virtnet_rss_ctx_del to delete
       the RSS context whose id is \field{rss_ctx_id} for the device.
\end{enumerate}

For commands VIRTNET_RSS_CTX_CTRL_ADD and VIRTNET_RSS_CTX_CTRL_MOD, the structure virtnet_rss_ctx_add_modify is read-only for the device.
For the command VIRTNET_RSS_CTX_CTRL_DEL, the structure virtnet_rss_ctx_del is read-only for the device.

\devicenormative{\subparagraph}{RSS Context}{Device Types / Network Device / Device Operation / Control Virtqueue / RSS Context}

The device MUST set \field{max_rss_contexts} to at least 1 if it offers VIRTIO_NET_F_RSS_CONTEXT.

Upon reset, the device MUST clear all previously configured RSS contexts.

\drivernormative{\subparagraph}{RSS Context}{Device Types / Network Device / Device Operation / Control Virtqueue / RSS Context}

The driver MUST have negotiated the VIRTIO_NET_F_RSS_CONTEXT feature when issuing the VIRTNET_RSS_CTX_CTRL class commands.

The driver MUST set \field{rss_ctx_id} to between 1 and \field{max_rss_contexts} inclusive.

The driver MUST NOT send the command VIRTIO_NET_CTRL_MQ_VQ_PAIRS_SET when the device has successfully configured at least one RSS context.

\paragraph{Offloads State Configuration}\label{sec:Device Types / Network Device / Device Operation / Control Virtqueue / Offloads State Configuration}

If the VIRTIO_NET_F_CTRL_GUEST_OFFLOADS feature is negotiated, the driver can
send control commands for dynamic offloads state configuration.

\subparagraph{Setting Offloads State}\label{sec:Device Types / Network Device / Device Operation / Control Virtqueue / Offloads State Configuration / Setting Offloads State}

To configure the offloads, the following layout structure and
definitions are used:

\begin{lstlisting}
le64 offloads;

#define VIRTIO_NET_F_GUEST_CSUM       1
#define VIRTIO_NET_F_GUEST_TSO4       7
#define VIRTIO_NET_F_GUEST_TSO6       8
#define VIRTIO_NET_F_GUEST_ECN        9
#define VIRTIO_NET_F_GUEST_UFO        10
#define VIRTIO_NET_F_GUEST_UDP_TUNNEL_GSO  46
#define VIRTIO_NET_F_GUEST_UDP_TUNNEL_GSO_CSUM 47
#define VIRTIO_NET_F_GUEST_USO4       54
#define VIRTIO_NET_F_GUEST_USO6       55

#define VIRTIO_NET_CTRL_GUEST_OFFLOADS       5
 #define VIRTIO_NET_CTRL_GUEST_OFFLOADS_SET   0
\end{lstlisting}

The class VIRTIO_NET_CTRL_GUEST_OFFLOADS has one command:
VIRTIO_NET_CTRL_GUEST_OFFLOADS_SET applies the new offloads configuration.

le64 value passed as command data is a bitmask, bits set define
offloads to be enabled, bits cleared - offloads to be disabled.

There is a corresponding device feature for each offload. Upon feature
negotiation corresponding offload gets enabled to preserve backward
compatibility.

\drivernormative{\subparagraph}{Setting Offloads State}{Device Types / Network Device / Device Operation / Control Virtqueue / Offloads State Configuration / Setting Offloads State}

A driver MUST NOT enable an offload for which the appropriate feature
has not been negotiated.

\subparagraph{Legacy Interface: Setting Offloads State}\label{sec:Device Types / Network Device / Device Operation / Control Virtqueue / Offloads State Configuration / Setting Offloads State / Legacy Interface: Setting Offloads State}
When using the legacy interface, transitional devices and drivers
MUST format \field{offloads}
according to the native endian of the guest rather than
(necessarily when not using the legacy interface) little-endian.


\paragraph{Notifications Coalescing}\label{sec:Device Types / Network Device / Device Operation / Control Virtqueue / Notifications Coalescing}

If the VIRTIO_NET_F_NOTF_COAL feature is negotiated, the driver can
send commands VIRTIO_NET_CTRL_NOTF_COAL_TX_SET and VIRTIO_NET_CTRL_NOTF_COAL_RX_SET
for notification coalescing.

If the VIRTIO_NET_F_VQ_NOTF_COAL feature is negotiated, the driver can
send commands VIRTIO_NET_CTRL_NOTF_COAL_VQ_SET and VIRTIO_NET_CTRL_NOTF_COAL_VQ_GET
for virtqueue notification coalescing.

\begin{lstlisting}
struct virtio_net_ctrl_coal {
    le32 max_packets;
    le32 max_usecs;
};

struct virtio_net_ctrl_coal_vq {
    le16 vq_index;
    le16 reserved;
    struct virtio_net_ctrl_coal coal;
};

#define VIRTIO_NET_CTRL_NOTF_COAL 6
 #define VIRTIO_NET_CTRL_NOTF_COAL_TX_SET  0
 #define VIRTIO_NET_CTRL_NOTF_COAL_RX_SET 1
 #define VIRTIO_NET_CTRL_NOTF_COAL_VQ_SET 2
 #define VIRTIO_NET_CTRL_NOTF_COAL_VQ_GET 3
\end{lstlisting}

Coalescing parameters:
\begin{itemize}
\item \field{vq_index}: The virtqueue index of an enabled transmit or receive virtqueue.
\item \field{max_usecs} for RX: Maximum number of microseconds to delay a RX notification.
\item \field{max_usecs} for TX: Maximum number of microseconds to delay a TX notification.
\item \field{max_packets} for RX: Maximum number of packets to receive before a RX notification.
\item \field{max_packets} for TX: Maximum number of packets to send before a TX notification.
\end{itemize}

\field{reserved} is reserved and it is ignored by the device.

Read/Write attributes for coalescing parameters:
\begin{itemize}
\item For commands VIRTIO_NET_CTRL_NOTF_COAL_TX_SET and VIRTIO_NET_CTRL_NOTF_COAL_RX_SET, the structure virtio_net_ctrl_coal is write-only for the driver.
\item For the command VIRTIO_NET_CTRL_NOTF_COAL_VQ_SET, the structure virtio_net_ctrl_coal_vq is write-only for the driver.
\item For the command VIRTIO_NET_CTRL_NOTF_COAL_VQ_GET, \field{vq_index} and \field{reserved} are write-only
      for the driver, and the structure virtio_net_ctrl_coal is read-only for the driver.
\end{itemize}

The class VIRTIO_NET_CTRL_NOTF_COAL has the following commands:
\begin{enumerate}
\item VIRTIO_NET_CTRL_NOTF_COAL_TX_SET: use the structure virtio_net_ctrl_coal to set the \field{max_usecs} and \field{max_packets} parameters for all transmit virtqueues.
\item VIRTIO_NET_CTRL_NOTF_COAL_RX_SET: use the structure virtio_net_ctrl_coal to set the \field{max_usecs} and \field{max_packets} parameters for all receive virtqueues.
\item VIRTIO_NET_CTRL_NOTF_COAL_VQ_SET: use the structure virtio_net_ctrl_coal_vq to set the \field{max_usecs} and \field{max_packets} parameters
                                        for an enabled transmit/receive virtqueue whose index is \field{vq_index}.
\item VIRTIO_NET_CTRL_NOTF_COAL_VQ_GET: use the structure virtio_net_ctrl_coal_vq to get the \field{max_usecs} and \field{max_packets} parameters
                                        for an enabled transmit/receive virtqueue whose index is \field{vq_index}.
\end{enumerate}

The device may generate notifications more or less frequently than specified by set commands of the VIRTIO_NET_CTRL_NOTF_COAL class.

If coalescing parameters are being set, the device applies the last coalescing parameters set for a
virtqueue, regardless of the command used to set the parameters. Use the following command sequence
with two pairs of virtqueues as an example:
Each of the following commands sets \field{max_usecs} and \field{max_packets} parameters for virtqueues.
\begin{itemize}
\item Command1: VIRTIO_NET_CTRL_NOTF_COAL_RX_SET sets coalescing parameters for virtqueues having index 0 and index 2. Virtqueues having index 1 and index 3 retain their previous parameters.
\item Command2: VIRTIO_NET_CTRL_NOTF_COAL_VQ_SET with \field{vq_index} = 0 sets coalescing parameters for virtqueue having index 0. Virtqueue having index 2 retains the parameters from command1.
\item Command3: VIRTIO_NET_CTRL_NOTF_COAL_VQ_GET with \field{vq_index} = 0, the device responds with coalescing parameters of vq_index 0 set by command2.
\item Command4: VIRTIO_NET_CTRL_NOTF_COAL_VQ_SET with \field{vq_index} = 1 sets coalescing parameters for virtqueue having index 1. Virtqueue having index 3 retains its previous parameters.
\item Command5: VIRTIO_NET_CTRL_NOTF_COAL_TX_SET sets coalescing parameters for virtqueues having index 1 and index 3, and overrides the parameters set by command4.
\item Command6: VIRTIO_NET_CTRL_NOTF_COAL_VQ_GET with \field{vq_index} = 1, the device responds with coalescing parameters of index 1 set by command5.
\end{itemize}

\subparagraph{Operation}\label{sec:Device Types / Network Device / Device Operation / Control Virtqueue / Notifications Coalescing / Operation}

The device sends a used buffer notification once the notification conditions are met and if the notifications are not suppressed as explained in \ref{sec:Basic Facilities of a Virtio Device / Virtqueues / Used Buffer Notification Suppression}.

When the device has non-zero \field{max_usecs} and non-zero \field{max_packets}, it starts counting microseconds and packets upon receiving/sending a packet.
The device counts packets and microseconds for each receive virtqueue and transmit virtqueue separately.
In this case, the notification conditions are met when \field{max_usecs} microseconds elapse, or upon sending/receiving \field{max_packets} packets, whichever happens first.
Afterwards, the device waits for the next packet and starts counting packets and microseconds again.

When the device has \field{max_usecs} = 0 or \field{max_packets} = 0, the notification conditions are met after every packet received/sent.

\subparagraph{RX Example}\label{sec:Device Types / Network Device / Device Operation / Control Virtqueue / Notifications Coalescing / RX Example}

If, for example:
\begin{itemize}
\item \field{max_usecs} = 10.
\item \field{max_packets} = 15.
\end{itemize}
then each receive virtqueue of a device will operate as follows:
\begin{itemize}
\item The device will count packets received on each virtqueue until it accumulates 15, or until 10 microseconds elapsed since the first one was received.
\item If the notifications are not suppressed by the driver, the device will send an used buffer notification, otherwise, the device will not send an used buffer notification as long as the notifications are suppressed.
\end{itemize}

\subparagraph{TX Example}\label{sec:Device Types / Network Device / Device Operation / Control Virtqueue / Notifications Coalescing / TX Example}

If, for example:
\begin{itemize}
\item \field{max_usecs} = 10.
\item \field{max_packets} = 15.
\end{itemize}
then each transmit virtqueue of a device will operate as follows:
\begin{itemize}
\item The device will count packets sent on each virtqueue until it accumulates 15, or until 10 microseconds elapsed since the first one was sent.
\item If the notifications are not suppressed by the driver, the device will send an used buffer notification, otherwise, the device will not send an used buffer notification as long as the notifications are suppressed.
\end{itemize}

\subparagraph{Notifications When Coalescing Parameters Change}\label{sec:Device Types / Network Device / Device Operation / Control Virtqueue / Notifications Coalescing / Notifications When Coalescing Parameters Change}

When the coalescing parameters of a device change, the device needs to check if the new notification conditions are met and send a used buffer notification if so.

For example, \field{max_packets} = 15 for a device with a single transmit virtqueue: if the device sends 10 packets and afterwards receives a
VIRTIO_NET_CTRL_NOTF_COAL_TX_SET command with \field{max_packets} = 8, then the notification condition is immediately considered to be met;
the device needs to immediately send a used buffer notification, if the notifications are not suppressed by the driver.

\drivernormative{\subparagraph}{Notifications Coalescing}{Device Types / Network Device / Device Operation / Control Virtqueue / Notifications Coalescing}

The driver MUST set \field{vq_index} to the virtqueue index of an enabled transmit or receive virtqueue.

The driver MUST have negotiated the VIRTIO_NET_F_NOTF_COAL feature when issuing commands VIRTIO_NET_CTRL_NOTF_COAL_TX_SET and VIRTIO_NET_CTRL_NOTF_COAL_RX_SET.

The driver MUST have negotiated the VIRTIO_NET_F_VQ_NOTF_COAL feature when issuing commands VIRTIO_NET_CTRL_NOTF_COAL_VQ_SET and VIRTIO_NET_CTRL_NOTF_COAL_VQ_GET.

The driver MUST ignore the values of coalescing parameters received from the VIRTIO_NET_CTRL_NOTF_COAL_VQ_GET command if the device responds with VIRTIO_NET_ERR.

\devicenormative{\subparagraph}{Notifications Coalescing}{Device Types / Network Device / Device Operation / Control Virtqueue / Notifications Coalescing}

The device MUST ignore \field{reserved}.

The device SHOULD respond to VIRTIO_NET_CTRL_NOTF_COAL_TX_SET and VIRTIO_NET_CTRL_NOTF_COAL_RX_SET commands with VIRTIO_NET_ERR if it was not able to change the parameters.

The device MUST respond to the VIRTIO_NET_CTRL_NOTF_COAL_VQ_SET command with VIRTIO_NET_ERR if it was not able to change the parameters.

The device MUST respond to VIRTIO_NET_CTRL_NOTF_COAL_VQ_SET and VIRTIO_NET_CTRL_NOTF_COAL_VQ_GET commands with
VIRTIO_NET_ERR if the designated virtqueue is not an enabled transmit or receive virtqueue.

Upon disabling and re-enabling a transmit virtqueue, the device MUST set the coalescing parameters of the virtqueue
to those configured through the VIRTIO_NET_CTRL_NOTF_COAL_TX_SET command, or, if the driver did not set any TX coalescing parameters, to 0.

Upon disabling and re-enabling a receive virtqueue, the device MUST set the coalescing parameters of the virtqueue
to those configured through the VIRTIO_NET_CTRL_NOTF_COAL_RX_SET command, or, if the driver did not set any RX coalescing parameters, to 0.

The behavior of the device in response to set commands of the VIRTIO_NET_CTRL_NOTF_COAL class is best-effort:
the device MAY generate notifications more or less frequently than specified.

A device SHOULD NOT send used buffer notifications to the driver if the notifications are suppressed, even if the notification conditions are met.

Upon reset, a device MUST initialize all coalescing parameters to 0.

\paragraph{Device Statistics}\label{sec:Device Types / Network Device / Device Operation / Control Virtqueue / Device Statistics}

If the VIRTIO_NET_F_DEVICE_STATS feature is negotiated, the driver can obtain
device statistics from the device by using the following command.

Different types of virtqueues have different statistics. The statistics of the
receiveq are different from those of the transmitq.

The statistics of a certain type of virtqueue are also divided into multiple types
because different types require different features. This enables the expansion
of new statistics.

In one command, the driver can obtain the statistics of one or multiple virtqueues.
Additionally, the driver can obtain multiple type statistics of each virtqueue.

\subparagraph{Query Statistic Capabilities}\label{sec:Device Types / Network Device / Device Operation / Control Virtqueue / Device Statistics / Query Statistic Capabilities}

\begin{lstlisting}
#define VIRTIO_NET_CTRL_STATS         8
#define VIRTIO_NET_CTRL_STATS_QUERY   0
#define VIRTIO_NET_CTRL_STATS_GET     1

struct virtio_net_stats_capabilities {

#define VIRTIO_NET_STATS_TYPE_CVQ       (1 << 32)

#define VIRTIO_NET_STATS_TYPE_RX_BASIC  (1 << 0)
#define VIRTIO_NET_STATS_TYPE_RX_CSUM   (1 << 1)
#define VIRTIO_NET_STATS_TYPE_RX_GSO    (1 << 2)
#define VIRTIO_NET_STATS_TYPE_RX_SPEED  (1 << 3)

#define VIRTIO_NET_STATS_TYPE_TX_BASIC  (1 << 16)
#define VIRTIO_NET_STATS_TYPE_TX_CSUM   (1 << 17)
#define VIRTIO_NET_STATS_TYPE_TX_GSO    (1 << 18)
#define VIRTIO_NET_STATS_TYPE_TX_SPEED  (1 << 19)

    le64 supported_stats_types[1];
}
\end{lstlisting}

To obtain device statistic capability, use the VIRTIO_NET_CTRL_STATS_QUERY
command. When the command completes successfully, \field{command-specific-result}
is in the format of \field{struct virtio_net_stats_capabilities}.

\subparagraph{Get Statistics}\label{sec:Device Types / Network Device / Device Operation / Control Virtqueue / Device Statistics / Get Statistics}

\begin{lstlisting}
struct virtio_net_ctrl_queue_stats {
       struct {
           le16 vq_index;
           le16 reserved[3];
           le64 types_bitmap[1];
       } stats[];
};

struct virtio_net_stats_reply_hdr {
#define VIRTIO_NET_STATS_TYPE_REPLY_CVQ       32

#define VIRTIO_NET_STATS_TYPE_REPLY_RX_BASIC  0
#define VIRTIO_NET_STATS_TYPE_REPLY_RX_CSUM   1
#define VIRTIO_NET_STATS_TYPE_REPLY_RX_GSO    2
#define VIRTIO_NET_STATS_TYPE_REPLY_RX_SPEED  3

#define VIRTIO_NET_STATS_TYPE_REPLY_TX_BASIC  16
#define VIRTIO_NET_STATS_TYPE_REPLY_TX_CSUM   17
#define VIRTIO_NET_STATS_TYPE_REPLY_TX_GSO    18
#define VIRTIO_NET_STATS_TYPE_REPLY_TX_SPEED  19
    u8 type;
    u8 reserved;
    le16 vq_index;
    le16 reserved1;
    le16 size;
}
\end{lstlisting}

To obtain device statistics, use the VIRTIO_NET_CTRL_STATS_GET command with the
\field{command-specific-data} which is in the format of
\field{struct virtio_net_ctrl_queue_stats}. When the command completes
successfully, \field{command-specific-result} contains multiple statistic
results, each statistic result has the \field{struct virtio_net_stats_reply_hdr}
as the header.

The fields of the \field{struct virtio_net_ctrl_queue_stats}:
\begin{description}
    \item [vq_index]
        The index of the virtqueue to obtain the statistics.

    \item [types_bitmap]
        This is a bitmask of the types of statistics to be obtained. Therefore, a
        \field{stats} inside \field{struct virtio_net_ctrl_queue_stats} may
        indicate multiple statistic replies for the virtqueue.
\end{description}

The fields of the \field{struct virtio_net_stats_reply_hdr}:
\begin{description}
    \item [type]
        The type of the reply statistic.

    \item [vq_index]
        The virtqueue index of the reply statistic.

    \item [size]
        The number of bytes for the statistics entry including size of \field{struct virtio_net_stats_reply_hdr}.

\end{description}

\subparagraph{Controlq Statistics}\label{sec:Device Types / Network Device / Device Operation / Control Virtqueue / Device Statistics / Controlq Statistics}

The structure corresponding to the controlq statistics is
\field{struct virtio_net_stats_cvq}. The corresponding type is
VIRTIO_NET_STATS_TYPE_CVQ. This is for the controlq.

\begin{lstlisting}
struct virtio_net_stats_cvq {
    struct virtio_net_stats_reply_hdr hdr;

    le64 command_num;
    le64 ok_num;
};
\end{lstlisting}

\begin{description}
    \item [command_num]
        The number of commands received by the device including the current command.

    \item [ok_num]
        The number of commands completed successfully by the device including the current command.
\end{description}


\subparagraph{Receiveq Basic Statistics}\label{sec:Device Types / Network Device / Device Operation / Control Virtqueue / Device Statistics / Receiveq Basic Statistics}

The structure corresponding to the receiveq basic statistics is
\field{struct virtio_net_stats_rx_basic}. The corresponding type is
VIRTIO_NET_STATS_TYPE_RX_BASIC. This is for the receiveq.

Receiveq basic statistics do not require any feature. As long as the device supports
VIRTIO_NET_F_DEVICE_STATS, the following are the receiveq basic statistics.

\begin{lstlisting}
struct virtio_net_stats_rx_basic {
    struct virtio_net_stats_reply_hdr hdr;

    le64 rx_notifications;

    le64 rx_packets;
    le64 rx_bytes;

    le64 rx_interrupts;

    le64 rx_drops;
    le64 rx_drop_overruns;
};
\end{lstlisting}

The packets described below were all presented on the specified virtqueue.
\begin{description}
    \item [rx_notifications]
        The number of driver notifications received by the device for this
        receiveq.

    \item [rx_packets]
        This is the number of packets passed to the driver by the device.

    \item [rx_bytes]
        This is the bytes of packets passed to the driver by the device.

    \item [rx_interrupts]
        The number of interrupts generated by the device for this receiveq.

    \item [rx_drops]
        This is the number of packets dropped by the device. The count includes
        all types of packets dropped by the device.

    \item [rx_drop_overruns]
        This is the number of packets dropped by the device when no more
        descriptors were available.

\end{description}

\subparagraph{Transmitq Basic Statistics}\label{sec:Device Types / Network Device / Device Operation / Control Virtqueue / Device Statistics / Transmitq Basic Statistics}

The structure corresponding to the transmitq basic statistics is
\field{struct virtio_net_stats_tx_basic}. The corresponding type is
VIRTIO_NET_STATS_TYPE_TX_BASIC. This is for the transmitq.

Transmitq basic statistics do not require any feature. As long as the device supports
VIRTIO_NET_F_DEVICE_STATS, the following are the transmitq basic statistics.

\begin{lstlisting}
struct virtio_net_stats_tx_basic {
    struct virtio_net_stats_reply_hdr hdr;

    le64 tx_notifications;

    le64 tx_packets;
    le64 tx_bytes;

    le64 tx_interrupts;

    le64 tx_drops;
    le64 tx_drop_malformed;
};
\end{lstlisting}

The packets described below are all for a specific virtqueue.
\begin{description}
    \item [tx_notifications]
        The number of driver notifications received by the device for this
        transmitq.

    \item [tx_packets]
        This is the number of packets sent by the device (not the packets
        got from the driver).

    \item [tx_bytes]
        This is the number of bytes sent by the device for all the sent packets
        (not the bytes sent got from the driver).

    \item [tx_interrupts]
        The number of interrupts generated by the device for this transmitq.

    \item [tx_drops]
        The number of packets dropped by the device. The count includes all
        types of packets dropped by the device.

    \item [tx_drop_malformed]
        The number of packets dropped by the device, when the descriptors are
        malformed. For example, the buffer is too short.
\end{description}

\subparagraph{Receiveq CSUM Statistics}\label{sec:Device Types / Network Device / Device Operation / Control Virtqueue / Device Statistics / Receiveq CSUM Statistics}

The structure corresponding to the receiveq checksum statistics is
\field{struct virtio_net_stats_rx_csum}. The corresponding type is
VIRTIO_NET_STATS_TYPE_RX_CSUM. This is for the receiveq.

Only after the VIRTIO_NET_F_GUEST_CSUM is negotiated, the receiveq checksum
statistics can be obtained.

\begin{lstlisting}
struct virtio_net_stats_rx_csum {
    struct virtio_net_stats_reply_hdr hdr;

    le64 rx_csum_valid;
    le64 rx_needs_csum;
    le64 rx_csum_none;
    le64 rx_csum_bad;
};
\end{lstlisting}

The packets described below were all presented on the specified virtqueue.
\begin{description}
    \item [rx_csum_valid]
        The number of packets with VIRTIO_NET_HDR_F_DATA_VALID.

    \item [rx_needs_csum]
        The number of packets with VIRTIO_NET_HDR_F_NEEDS_CSUM.

    \item [rx_csum_none]
        The number of packets without hardware checksum. The packet here refers
        to the non-TCP/UDP packet that the device cannot recognize.

    \item [rx_csum_bad]
        The number of packets with checksum mismatch.

\end{description}

\subparagraph{Transmitq CSUM Statistics}\label{sec:Device Types / Network Device / Device Operation / Control Virtqueue / Device Statistics / Transmitq CSUM Statistics}

The structure corresponding to the transmitq checksum statistics is
\field{struct virtio_net_stats_tx_csum}. The corresponding type is
VIRTIO_NET_STATS_TYPE_TX_CSUM. This is for the transmitq.

Only after the VIRTIO_NET_F_CSUM is negotiated, the transmitq checksum
statistics can be obtained.

The following are the transmitq checksum statistics:

\begin{lstlisting}
struct virtio_net_stats_tx_csum {
    struct virtio_net_stats_reply_hdr hdr;

    le64 tx_csum_none;
    le64 tx_needs_csum;
};
\end{lstlisting}

The packets described below are all for a specific virtqueue.
\begin{description}
    \item [tx_csum_none]
        The number of packets which do not require hardware checksum.

    \item [tx_needs_csum]
        The number of packets which require checksum calculation by the device.

\end{description}

\subparagraph{Receiveq GSO Statistics}\label{sec:Device Types / Network Device / Device Operation / Control Virtqueue / Device Statistics / Receiveq GSO Statistics}

The structure corresponding to the receivq GSO statistics is
\field{struct virtio_net_stats_rx_gso}. The corresponding type is
VIRTIO_NET_STATS_TYPE_RX_GSO. This is for the receiveq.

If one or more of the VIRTIO_NET_F_GUEST_TSO4, VIRTIO_NET_F_GUEST_TSO6
have been negotiated, the receiveq GSO statistics can be obtained.

GSO packets refer to packets passed by the device to the driver where
\field{gso_type} is not VIRTIO_NET_HDR_GSO_NONE.

\begin{lstlisting}
struct virtio_net_stats_rx_gso {
    struct virtio_net_stats_reply_hdr hdr;

    le64 rx_gso_packets;
    le64 rx_gso_bytes;
    le64 rx_gso_packets_coalesced;
    le64 rx_gso_bytes_coalesced;
};
\end{lstlisting}

The packets described below were all presented on the specified virtqueue.
\begin{description}
    \item [rx_gso_packets]
        The number of the GSO packets received by the device.

    \item [rx_gso_bytes]
        The bytes of the GSO packets received by the device.
        This includes the header size of the GSO packet.

    \item [rx_gso_packets_coalesced]
        The number of the GSO packets coalesced by the device.

    \item [rx_gso_bytes_coalesced]
        The bytes of the GSO packets coalesced by the device.
        This includes the header size of the GSO packet.
\end{description}

\subparagraph{Transmitq GSO Statistics}\label{sec:Device Types / Network Device / Device Operation / Control Virtqueue / Device Statistics / Transmitq GSO Statistics}

The structure corresponding to the transmitq GSO statistics is
\field{struct virtio_net_stats_tx_gso}. The corresponding type is
VIRTIO_NET_STATS_TYPE_TX_GSO. This is for the transmitq.

If one or more of the VIRTIO_NET_F_HOST_TSO4, VIRTIO_NET_F_HOST_TSO6,
VIRTIO_NET_F_HOST_USO options have been negotiated, the transmitq GSO statistics
can be obtained.

GSO packets refer to packets passed by the driver to the device where
\field{gso_type} is not VIRTIO_NET_HDR_GSO_NONE.
See more \ref{sec:Device Types / Network Device / Device Operation / Packet
Transmission}.

\begin{lstlisting}
struct virtio_net_stats_tx_gso {
    struct virtio_net_stats_reply_hdr hdr;

    le64 tx_gso_packets;
    le64 tx_gso_bytes;
    le64 tx_gso_segments;
    le64 tx_gso_segments_bytes;
    le64 tx_gso_packets_noseg;
    le64 tx_gso_bytes_noseg;
};
\end{lstlisting}

The packets described below are all for a specific virtqueue.
\begin{description}
    \item [tx_gso_packets]
        The number of the GSO packets sent by the device.

    \item [tx_gso_bytes]
        The bytes of the GSO packets sent by the device.

    \item [tx_gso_segments]
        The number of segments prepared from GSO packets.

    \item [tx_gso_segments_bytes]
        The bytes of segments prepared from GSO packets.

    \item [tx_gso_packets_noseg]
        The number of the GSO packets without segmentation.

    \item [tx_gso_bytes_noseg]
        The bytes of the GSO packets without segmentation.

\end{description}

\subparagraph{Receiveq Speed Statistics}\label{sec:Device Types / Network Device / Device Operation / Control Virtqueue / Device Statistics / Receiveq Speed Statistics}

The structure corresponding to the receiveq speed statistics is
\field{struct virtio_net_stats_rx_speed}. The corresponding type is
VIRTIO_NET_STATS_TYPE_RX_SPEED. This is for the receiveq.

The device has the allowance for the speed. If VIRTIO_NET_F_SPEED_DUPLEX has
been negotiated, the driver can get this by \field{speed}. When the received
packets bitrate exceeds the \field{speed}, some packets may be dropped by the
device.

\begin{lstlisting}
struct virtio_net_stats_rx_speed {
    struct virtio_net_stats_reply_hdr hdr;

    le64 rx_packets_allowance_exceeded;
    le64 rx_bytes_allowance_exceeded;
};
\end{lstlisting}

The packets described below were all presented on the specified virtqueue.
\begin{description}
    \item [rx_packets_allowance_exceeded]
        The number of the packets dropped by the device due to the received
        packets bitrate exceeding the \field{speed}.

    \item [rx_bytes_allowance_exceeded]
        The bytes of the packets dropped by the device due to the received
        packets bitrate exceeding the \field{speed}.

\end{description}

\subparagraph{Transmitq Speed Statistics}\label{sec:Device Types / Network Device / Device Operation / Control Virtqueue / Device Statistics / Transmitq Speed Statistics}

The structure corresponding to the transmitq speed statistics is
\field{struct virtio_net_stats_tx_speed}. The corresponding type is
VIRTIO_NET_STATS_TYPE_TX_SPEED. This is for the transmitq.

The device has the allowance for the speed. If VIRTIO_NET_F_SPEED_DUPLEX has
been negotiated, the driver can get this by \field{speed}. When the transmit
packets bitrate exceeds the \field{speed}, some packets may be dropped by the
device.

\begin{lstlisting}
struct virtio_net_stats_tx_speed {
    struct virtio_net_stats_reply_hdr hdr;

    le64 tx_packets_allowance_exceeded;
    le64 tx_bytes_allowance_exceeded;
};
\end{lstlisting}

The packets described below were all presented on the specified virtqueue.
\begin{description}
    \item [tx_packets_allowance_exceeded]
        The number of the packets dropped by the device due to the transmit packets
        bitrate exceeding the \field{speed}.

    \item [tx_bytes_allowance_exceeded]
        The bytes of the packets dropped by the device due to the transmit packets
        bitrate exceeding the \field{speed}.

\end{description}

\devicenormative{\subparagraph}{Device Statistics}{Device Types / Network Device / Device Operation / Control Virtqueue / Device Statistics}

When the VIRTIO_NET_F_DEVICE_STATS feature is negotiated, the device MUST reply
to the command VIRTIO_NET_CTRL_STATS_QUERY with the
\field{struct virtio_net_stats_capabilities}. \field{supported_stats_types}
includes all the statistic types supported by the device.

If \field{struct virtio_net_ctrl_queue_stats} is incorrect (such as the
following), the device MUST set \field{ack} to VIRTIO_NET_ERR. Even if there is
only one error, the device MUST fail the entire command.
\begin{itemize}
    \item \field{vq_index} exceeds the queue range.
    \item \field{types_bitmap} contains unknown types.
    \item One or more of the bits present in \field{types_bitmap} is not valid
        for the specified virtqueue.
    \item The feature corresponding to the specified \field{types_bitmap} was
        not negotiated.
\end{itemize}

The device MUST set the actual size of the bytes occupied by the reply to the
\field{size} of the \field{hdr}. And the device MUST set the \field{type} and
the \field{vq_index} of the statistic header.

The \field{command-specific-result} buffer allocated by the driver may be
smaller or bigger than all the statistics specified by
\field{struct virtio_net_ctrl_queue_stats}. The device MUST fill up only upto
the valid bytes.

The statistics counter replied by the device MUST wrap around to zero by the
device on the overflow.

\drivernormative{\subparagraph}{Device Statistics}{Device Types / Network Device / Device Operation / Control Virtqueue / Device Statistics}

The types contained in the \field{types_bitmap} MUST be queried from the device
via command VIRTIO_NET_CTRL_STATS_QUERY.

\field{types_bitmap} in \field{struct virtio_net_ctrl_queue_stats} MUST be valid to the
vq specified by \field{vq_index}.

The \field{command-specific-result} buffer allocated by the driver MUST have
enough capacity to store all the statistics reply headers defined in
\field{struct virtio_net_ctrl_queue_stats}. If the
\field{command-specific-result} buffer is fully utilized by the device but some
replies are missed, it is possible that some statistics may exceed the capacity
of the driver's records. In such cases, the driver should allocate additional
space for the \field{command-specific-result} buffer.

\subsubsection{Flow filter}\label{sec:Device Types / Network Device / Device Operation / Flow filter}

A network device can support one or more flow filter rules. Each flow filter rule
is applied by matching a packet and then taking an action, such as directing the packet
to a specific receiveq or dropping the packet. An example of a match is
matching on specific source and destination IP addresses.

A flow filter rule is a device resource object that consists of a key,
a processing priority, and an action to either direct a packet to a
receive queue or drop the packet.

Each rule uses a classifier. The key is matched against the packet using
a classifier, defining which fields in the packet are matched.
A classifier resource object consists of one or more field selectors, each with
a type that specifies the header fields to be matched against, and a mask.
The mask can match whole fields or parts of a field in a header. Each
rule resource object depends on the classifier resource object.

When a packet is received, relevant fields are extracted
(in the same way) from both the packet and the key according to the
classifier. The resulting field contents are then compared -
if they are identical the rule action is taken, if they are not, the rule is ignored.

Multiple flow filter rules are part of a group. The rule resource object
depends on the group. Each rule within a
group has a rule priority, and each group also has a group priority. For a
packet, a group with the highest priority is selected first. Within a group,
rules are applied from highest to lowest priority, until one of the rules
matches the packet and an action is taken. If all the rules within a group
are ignored, the group with the next highest priority is selected, and so on.

The device and the driver indicates flow filter resource limits using the capability
\ref{par:Device Types / Network Device / Device Operation / Flow filter / Device and driver capabilities / VIRTIO-NET-FF-RESOURCE-CAP} specifying the limits on the number of flow filter rule,
group and classifier resource objects. The capability \ref{par:Device Types / Network Device / Device Operation / Flow filter / Device and driver capabilities / VIRTIO-NET-FF-SELECTOR-CAP} specifies which selectors the device supports.
The driver indicates the selectors it is using by setting the flow
filter selector capability, prior to adding any resource objects.

The capability \ref{par:Device Types / Network Device / Device Operation / Flow filter / Device and driver capabilities / VIRTIO-NET-FF-ACTION-CAP} specifies which actions the device supports.

The driver controls the flow filter rule, classifier and group resource objects using
administration commands described in
\ref{sec:Basic Facilities of a Virtio Device / Device groups / Group administration commands / Device resource objects}.

\paragraph{Packet processing order}\label{sec:sec:Device Types / Network Device / Device Operation / Flow filter / Packet processing order}

Note that flow filter rules are applied after MAC/VLAN filtering. Flow filter
rules take precedence over steering: if a flow filter rule results in an action,
the steering configuration does not apply. The steering configuration only applies
to packets for which no flow filter rule action was performed. For example,
incoming packets can be processed in the following order:

\begin{itemize}
\item apply steering configuration received using control virtqueue commands
      VIRTIO_NET_CTRL_RX, VIRTIO_NET_CTRL_MAC and VIRTIO_NET_CTRL_VLAN.
\item apply flow filter rules if any.
\item if no filter rule applied, apply steering configuration received using command
      VIRTIO_NET_CTRL_MQ_RSS_CONFIG or as per automatic receive steering.
\end{itemize}

Some incoming packet processing examples:
\begin{itemize}
\item If the packet is dropped by the flow filter rule, RSS
      steering is ignored for the packet.
\item If the packet is directed to a specific receiveq using flow filter rule,
      the RSS steering is ignored for the packet.
\item If a packet is dropped due to the VIRTIO_NET_CTRL_MAC configuration,
      both flow filter rules and the RSS steering are ignored for the packet.
\item If a packet does not match any flow filter rules,
      the RSS steering is used to select the receiveq for the packet (if enabled).
\item If there are two flow filter groups configured as group_A and group_B
      with respective group priorities as 4, and 5; flow filter rules of
      group_B are applied first having highest group priority, if there is a match,
      the flow filter rules of group_A are ignored; if there is no match for
      the flow filter rules in group_B, the flow filter rules of next level group_A are applied.
\end{itemize}

\paragraph{Device and driver capabilities}
\label{par:Device Types / Network Device / Device Operation / Flow filter / Device and driver capabilities}

\subparagraph{VIRTIO_NET_FF_RESOURCE_CAP}
\label{par:Device Types / Network Device / Device Operation / Flow filter / Device and driver capabilities / VIRTIO-NET-FF-RESOURCE-CAP}

The capability VIRTIO_NET_FF_RESOURCE_CAP indicates the flow filter resource limits.
\field{cap_specific_data} is in the format
\field{struct virtio_net_ff_cap_data}.

\begin{lstlisting}
struct virtio_net_ff_cap_data {
        le32 groups_limit;
        le32 selectors_limit;
        le32 rules_limit;
        le32 rules_per_group_limit;
        u8 last_rule_priority;
        u8 selectors_per_classifier_limit;
};
\end{lstlisting}

\field{groups_limit}, and \field{selectors_limit} represent the maximum
number of flow filter groups and selectors, respectively, that the driver can create.
 \field{rules_limit} is the maximum number of
flow fiilter rules that the driver can create across all the groups.
\field{rules_per_group_limit} is the maximum number of flow filter rules that the driver
can create for each flow filter group.

\field{last_rule_priority} is the highest priority that can be assigned to a
flow filter rule.

\field{selectors_per_classifier_limit} is the maximum number of selectors
that a classifier can have.

\subparagraph{VIRTIO_NET_FF_SELECTOR_CAP}
\label{par:Device Types / Network Device / Device Operation / Flow filter / Device and driver capabilities / VIRTIO-NET-FF-SELECTOR-CAP}

The capability VIRTIO_NET_FF_SELECTOR_CAP lists the supported selectors and the
supported packet header fields for each selector.
\field{cap_specific_data} is in the format \field{struct virtio_net_ff_cap_mask_data}.

\begin{lstlisting}[label={lst:Device Types / Network Device / Device Operation / Flow filter / Device and driver capabilities / VIRTIO-NET-FF-SELECTOR-CAP / virtio-net-ff-selector}]
struct virtio_net_ff_selector {
        u8 type;
        u8 flags;
        u8 reserved[2];
        u8 length;
        u8 reserved1[3];
        u8 mask[];
};

struct virtio_net_ff_cap_mask_data {
        u8 count;
        u8 reserved[7];
        struct virtio_net_ff_selector selectors[];
};

#define VIRTIO_NET_FF_MASK_F_PARTIAL_MASK (1 << 0)
\end{lstlisting}

\field{count} indicates number of valid entries in the \field{selectors} array.
\field{selectors[]} is an array of supported selectors. Within each array entry:
\field{type} specifies the type of the packet header, as defined in table
\ref{table:Device Types / Network Device / Device Operation / Flow filter / Device and driver capabilities / VIRTIO-NET-FF-SELECTOR-CAP / flow filter selector types}. \field{mask} specifies which fields of the
packet header can be matched in a flow filter rule.

Each \field{type} is also listed in table
\ref{table:Device Types / Network Device / Device Operation / Flow filter / Device and driver capabilities / VIRTIO-NET-FF-SELECTOR-CAP / flow filter selector types}. \field{mask} is a byte array
in network byte order. For example, when \field{type} is VIRTIO_NET_FF_MASK_TYPE_IPV6,
the \field{mask} is in the format \hyperref[intro:IPv6-Header-Format]{IPv6 Header Format}.

If partial masking is not set, then all bits in each field have to be either all 0s
to ignore this field or all 1s to match on this field. If partial masking is set,
then any combination of bits can bit set to match on these bits.
For example, when a selector \field{type} is VIRTIO_NET_FF_MASK_TYPE_ETH, if
\field{mask[0-12]} are zero and \field{mask[13-14]} are 0xff (all 1s), it
indicates that matching is only supported for \field{EtherType} of
\field{Ethernet MAC frame}, matching is not supported for
\field{Destination Address} and \field{Source Address}.

The entries in the array \field{selectors} are ordered by
\field{type}, with each \field{type} value only appearing once.

\field{length} is the length of a dynamic array \field{mask} in bytes.
\field{reserved} and \field{reserved1} are reserved and set to zero.

\begin{table}[H]
\caption{Flow filter selector types}
\label{table:Device Types / Network Device / Device Operation / Flow filter / Device and driver capabilities / VIRTIO-NET-FF-SELECTOR-CAP / flow filter selector types}
\begin{tabularx}{\textwidth}{ |l|X|X| }
\hline
Type & Name & Description \\
\hline \hline
0x0 & - & Reserved \\
\hline
0x1 & VIRTIO_NET_FF_MASK_TYPE_ETH & 14 bytes of frame header starting from destination address described in \hyperref[intro:IEEE 802.3-2022]{IEEE 802.3-2022} \\
\hline
0x2 & VIRTIO_NET_FF_MASK_TYPE_IPV4 & 20 bytes of \hyperref[intro:Internet-Header-Format]{IPv4: Internet Header Format} \\
\hline
0x3 & VIRTIO_NET_FF_MASK_TYPE_IPV6 & 40 bytes of \hyperref[intro:IPv6-Header-Format]{IPv6 Header Format} \\
\hline
0x4 & VIRTIO_NET_FF_MASK_TYPE_TCP & 20 bytes of \hyperref[intro:TCP-Header-Format]{TCP Header Format} \\
\hline
0x5 & VIRTIO_NET_FF_MASK_TYPE_UDP & 8 bytes of UDP header described in \hyperref[intro:UDP]{UDP} \\
\hline
0x6 - 0xFF & & Reserved for future \\
\hline
\end{tabularx}
\end{table}

When VIRTIO_NET_FF_MASK_F_PARTIAL_MASK (bit 0) is set, it indicates that
partial masking is supported for all the fields of the selector identified by \field{type}.

For the selector \field{type} VIRTIO_NET_FF_MASK_TYPE_IPV4, if a partial mask is unsupported,
then matching on an individual bit of \field{Flags} in the
\field{IPv4: Internet Header Format} is unsupported. \field{Flags} has to match as a whole
if it is supported.

For the selector \field{type} VIRTIO_NET_FF_MASK_TYPE_IPV4, \field{mask} includes fields
up to the \field{Destination Address}; that is, \field{Options} and
\field{Padding} are excluded.

For the selector \field{type} VIRTIO_NET_FF_MASK_TYPE_IPV6, the \field{Next Header} field
of the \field{mask} corresponds to the \field{Next Header} in the packet
when \field{IPv6 Extension Headers} are not present. When the packet includes
one or more \field{IPv6 Extension Headers}, the \field{Next Header} field of
the \field{mask} corresponds to the \field{Next Header} of the last
\field{IPv6 Extension Header} in the packet.

For the selector \field{type} VIRTIO_NET_FF_MASK_TYPE_TCP, \field{Control bits}
are treated as individual fields for matching; that is, matching individual
\field{Control bits} does not depend on the partial mask support.

\subparagraph{VIRTIO_NET_FF_ACTION_CAP}
\label{par:Device Types / Network Device / Device Operation / Flow filter / Device and driver capabilities / VIRTIO-NET-FF-ACTION-CAP}

The capability VIRTIO_NET_FF_ACTION_CAP lists the supported actions in a rule.
\field{cap_specific_data} is in the format \field{struct virtio_net_ff_cap_actions}.

\begin{lstlisting}
struct virtio_net_ff_actions {
        u8 count;
        u8 reserved[7];
        u8 actions[];
};
\end{lstlisting}

\field{actions} is an array listing all possible actions.
The entries in the array are ordered from the smallest to the largest,
with each supported value appearing exactly once. Each entry can have the
following values:

\begin{table}[H]
\caption{Flow filter rule actions}
\label{table:Device Types / Network Device / Device Operation / Flow filter / Device and driver capabilities / VIRTIO-NET-FF-ACTION-CAP / flow filter rule actions}
\begin{tabularx}{\textwidth}{ |l|X|X| }
\hline
Action & Name & Description \\
\hline \hline
0x0 & - & reserved \\
\hline
0x1 & VIRTIO_NET_FF_ACTION_DROP & Matching packet will be dropped by the device \\
\hline
0x2 & VIRTIO_NET_FF_ACTION_DIRECT_RX_VQ & Matching packet will be directed to a receive queue \\
\hline
0x3 - 0xFF & & Reserved for future \\
\hline
\end{tabularx}
\end{table}

\paragraph{Resource objects}
\label{par:Device Types / Network Device / Device Operation / Flow filter / Resource objects}

\subparagraph{VIRTIO_NET_RESOURCE_OBJ_FF_GROUP}\label{par:Device Types / Network Device / Device Operation / Flow filter / Resource objects / VIRTIO-NET-RESOURCE-OBJ-FF-GROUP}

A flow filter group contains between 0 and \field{rules_limit} rules, as specified by the
capability VIRTIO_NET_FF_RESOURCE_CAP. For the flow filter group object both
\field{resource_obj_specific_data} and
\field{resource_obj_specific_result} are in the format
\field{struct virtio_net_resource_obj_ff_group}.

\begin{lstlisting}
struct virtio_net_resource_obj_ff_group {
        le16 group_priority;
};
\end{lstlisting}

\field{group_priority} specifies the priority for the group. Each group has a
distinct priority. For each incoming packet, the device tries to apply rules
from groups from higher \field{group_priority} value to lower, until either a
rule matches the packet or all groups have been tried.

\subparagraph{VIRTIO_NET_RESOURCE_OBJ_FF_CLASSIFIER}\label{par:Device Types / Network Device / Device Operation / Flow filter / Resource objects / VIRTIO-NET-RESOURCE-OBJ-FF-CLASSIFIER}

A classifier is used to match a flow filter key against a packet. The
classifier defines the desired packet fields to match, and is represented by
the VIRTIO_NET_RESOURCE_OBJ_FF_CLASSIFIER device resource object.

For the flow filter classifier object both \field{resource_obj_specific_data} and
\field{resource_obj_specific_result} are in the format
\field{struct virtio_net_resource_obj_ff_classifier}.

\begin{lstlisting}
struct virtio_net_resource_obj_ff_classifier {
        u8 count;
        u8 reserved[7];
        struct virtio_net_ff_selector selectors[];
};
\end{lstlisting}

A classifier is an array of \field{selectors}. The number of selectors in the
array is indicated by \field{count}. The selector has a type that specifies
the header fields to be matched against, and a mask.
See \ref{lst:Device Types / Network Device / Device Operation / Flow filter / Device and driver capabilities / VIRTIO-NET-FF-SELECTOR-CAP / virtio-net-ff-selector}
for details about selectors.

The first selector is always VIRTIO_NET_FF_MASK_TYPE_ETH. When there are multiple
selectors, a second selector can be either VIRTIO_NET_FF_MASK_TYPE_IPV4
or VIRTIO_NET_FF_MASK_TYPE_IPV6. If the third selector exists, the third
selector can be either VIRTIO_NET_FF_MASK_TYPE_UDP or VIRTIO_NET_FF_MASK_TYPE_TCP.
For example, to match a Ethernet IPv6 UDP packet,
\field{selectors[0].type} is set to VIRTIO_NET_FF_MASK_TYPE_ETH, \field{selectors[1].type}
is set to VIRTIO_NET_FF_MASK_TYPE_IPV6 and \field{selectors[2].type} is
set to VIRTIO_NET_FF_MASK_TYPE_UDP; accordingly, \field{selectors[0].mask[0-13]} is
for Ethernet header fields, \field{selectors[1].mask[0-39]} is set for IPV6 header
and \field{selectors[2].mask[0-7]} is set for UDP header.

When there are multiple selectors, the type of the (N+1)\textsuperscript{th} selector
affects the mask of the (N)\textsuperscript{th} selector. If
\field{count} is 2 or more, all the mask bits within \field{selectors[0]}
corresponding to \field{EtherType} of an Ethernet header are set.

If \field{count} is more than 2:
\begin{itemize}
\item if \field{selector[1].type} is, VIRTIO_NET_FF_MASK_TYPE_IPV4, then, all the mask bits within
\field{selector[1]} for \field{Protocol} is set.
\item if \field{selector[1].type} is, VIRTIO_NET_FF_MASK_TYPE_IPV6, then, all the mask bits within
\field{selector[1]} for \field{Next Header} is set.
\end{itemize}

If for a given packet header field, a subset of bits of a field is to be matched,
and if the partial mask is supported, the flow filter
mask object can specify a mask which has fewer bits set than the packet header
field size. For example, a partial mask for the Ethernet header source mac
address can be of 1-bit for multicast detection instead of 48-bits.

\subparagraph{VIRTIO_NET_RESOURCE_OBJ_FF_RULE}\label{par:Device Types / Network Device / Device Operation / Flow filter / Resource objects / VIRTIO-NET-RESOURCE-OBJ-FF-RULE}

Each flow filter rule resource object comprises a key, a priority, and an action.
For the flow filter rule object,
\field{resource_obj_specific_data} and
\field{resource_obj_specific_result} are in the format
\field{struct virtio_net_resource_obj_ff_rule}.

\begin{lstlisting}
struct virtio_net_resource_obj_ff_rule {
        le32 group_id;
        le32 classifier_id;
        u8 rule_priority;
        u8 key_length; /* length of key in bytes */
        u8 action;
        u8 reserved;
        le16 vq_index;
        u8 reserved1[2];
        u8 keys[][];
};
\end{lstlisting}

\field{group_id} is the resource object ID of the flow filter group to which
this rule belongs. \field{classifier_id} is the resource object ID of the
classifier used to match a packet against the \field{key}.

\field{rule_priority} denotes the priority of the rule within the group
specified by the \field{group_id}.
Rules within the group are applied from the highest to the lowest priority
until a rule matches the packet and an
action is taken. Rules with the same priority can be applied in any order.

\field{reserved} and \field{reserved1} are reserved and set to 0.

\field{keys[][]} is an array of keys to match against packets, using
the classifier specified by \field{classifier_id}. Each entry (key) comprises
a byte array, and they are located one immediately after another.
The size (number of entries) of the array is exactly the same as that of
\field{selectors} in the classifier, or in other words, \field{count}
in the classifier.

\field{key_length} specifies the total length of \field{keys} in bytes.
In other words, it equals the sum total of \field{length} of all
selectors in \field{selectors} in the classifier specified by
\field{classifier_id}.

For example, if a classifier object's \field{selectors[0].type} is
VIRTIO_NET_FF_MASK_TYPE_ETH and \field{selectors[1].type} is
VIRTIO_NET_FF_MASK_TYPE_IPV6,
then selectors[0].length is 14 and selectors[1].length is 40.
Accordingly, the \field{key_length} is set to 54.
This setting indicates that the \field{key} array's length is 54 bytes
comprising a first byte array of 14 bytes for the
Ethernet MAC header in bytes 0-13, immediately followed by 40 bytes for the
IPv6 header in bytes 14-53.

When there are multiple selectors in the classifier object, the key bytes
for (N)\textsuperscript{th} selector are set so that
(N+1)\textsuperscript{th} selector can be matched.

If \field{count} is 2 or more, key bytes of \field{EtherType}
are set according to \hyperref[intro:IEEE 802 Ethertypes]{IEEE 802 Ethertypes}
for VIRTIO_NET_FF_MASK_TYPE_IPV4 or VIRTIO_NET_FF_MASK_TYPE_IPV6 respectively.

If \field{count} is more than 2, when \field{selector[1].type} is
VIRTIO_NET_FF_MASK_TYPE_IPV4 or VIRTIO_NET_FF_MASK_TYPE_IPV6, key
bytes of \field{Protocol} or \field{Next Header} is set as per
\field{Protocol Numbers} defined \hyperref[intro:IANA Protocol Numbers]{IANA Protocol Numbers}
respectively.

\field{action} is the action to take when a packet matches the
\field{key} using the \field{classifier_id}. Supported actions are described in
\ref{table:Device Types / Network Device / Device Operation / Flow filter / Device and driver capabilities / VIRTIO-NET-FF-ACTION-CAP / flow filter rule actions}.

\field{vq_index} specifies a receive virtqueue. When the \field{action} is set
to VIRTIO_NET_FF_ACTION_DIRECT_RX_VQ, and the packet matches the \field{key},
the matching packet is directed to this virtqueue.

Note that at most one action is ever taken for a given packet. If a rule is
applied and an action is taken, the action of other rules is not taken.

\devicenormative{\paragraph}{Flow filter}{Device Types / Network Device / Device Operation / Flow filter}

When the device supports flow filter operations,
\begin{itemize}
\item the device MUST set VIRTIO_NET_FF_RESOURCE_CAP, VIRTIO_NET_FF_SELECTOR_CAP
and VIRTIO_NET_FF_ACTION_CAP capability in the \field{supported_caps} in the
command VIRTIO_ADMIN_CMD_CAP_SUPPORT_QUERY.
\item the device MUST support the administration commands
VIRTIO_ADMIN_CMD_RESOURCE_OBJ_CREATE,
VIRTIO_ADMIN_CMD_RESOURCE_OBJ_MODIFY, VIRTIO_ADMIN_CMD_RESOURCE_OBJ_QUERY,
VIRTIO_ADMIN_CMD_RESOURCE_OBJ_DESTROY for the resource types
VIRTIO_NET_RESOURCE_OBJ_FF_GROUP, VIRTIO_NET_RESOURCE_OBJ_FF_CLASSIFIER and
VIRTIO_NET_RESOURCE_OBJ_FF_RULE.
\end{itemize}

When any of the VIRTIO_NET_FF_RESOURCE_CAP, VIRTIO_NET_FF_SELECTOR_CAP, or
VIRTIO_NET_FF_ACTION_CAP capability is disabled, the device SHOULD set
\field{status} to VIRTIO_ADMIN_STATUS_Q_INVALID_OPCODE for the commands
VIRTIO_ADMIN_CMD_RESOURCE_OBJ_CREATE,
VIRTIO_ADMIN_CMD_RESOURCE_OBJ_MODIFY, VIRTIO_ADMIN_CMD_RESOURCE_OBJ_QUERY,
and VIRTIO_ADMIN_CMD_RESOURCE_OBJ_DESTROY. These commands apply to the resource
\field{type} of VIRTIO_NET_RESOURCE_OBJ_FF_GROUP, VIRTIO_NET_RESOURCE_OBJ_FF_CLASSIFIER, and
VIRTIO_NET_RESOURCE_OBJ_FF_RULE.

The device SHOULD set \field{status} to VIRTIO_ADMIN_STATUS_EINVAL for the
command VIRTIO_ADMIN_CMD_RESOURCE_OBJ_CREATE when the resource \field{type}
is VIRTIO_NET_RESOURCE_OBJ_FF_GROUP, if a flow filter group already exists
with the supplied \field{group_priority}.

The device SHOULD set \field{status} to VIRTIO_ADMIN_STATUS_ENOSPC for the
command VIRTIO_ADMIN_CMD_RESOURCE_OBJ_CREATE when the resource \field{type}
is VIRTIO_NET_RESOURCE_OBJ_FF_GROUP, if the number of flow filter group
objects in the device exceeds the lower of the configured driver
capabilities \field{groups_limit} and \field{rules_per_group_limit}.

The device SHOULD set \field{status} to VIRTIO_ADMIN_STATUS_ENOSPC for the
command VIRTIO_ADMIN_CMD_RESOURCE_OBJ_CREATE when the resource \field{type} is
VIRTIO_NET_RESOURCE_OBJ_FF_CLASSIFIER, if the number of flow filter selector
objects in the device exceeds the configured driver capability
\field{selectors_limit}.

The device SHOULD set \field{status} to VIRTIO_ADMIN_STATUS_EBUSY for the
command VIRTIO_ADMIN_CMD_RESOURCE_OBJ_DESTROY for a flow filter group when
the flow filter group has one or more flow filter rules depending on it.

The device SHOULD set \field{status} to VIRTIO_ADMIN_STATUS_EBUSY for the
command VIRTIO_ADMIN_CMD_RESOURCE_OBJ_DESTROY for a flow filter classifier when
the flow filter classifier has one or more flow filter rules depending on it.

The device SHOULD fail the command VIRTIO_ADMIN_CMD_RESOURCE_OBJ_CREATE for the
flow filter rule resource object if,
\begin{itemize}
\item \field{vq_index} is not a valid receive virtqueue index for
the VIRTIO_NET_FF_ACTION_DIRECT_RX_VQ action,
\item \field{priority} is greater than or equal to
      \field{last_rule_priority},
\item \field{id} is greater than or equal to \field{rules_limit} or
      greater than or equal to \field{rules_per_group_limit}, whichever is lower,
\item the length of \field{keys} and the length of all the mask bytes of
      \field{selectors[].mask} as referred by \field{classifier_id} differs,
\item the supplied \field{action} is not supported in the capability VIRTIO_NET_FF_ACTION_CAP.
\end{itemize}

When the flow filter directs a packet to the virtqueue identified by
\field{vq_index} and if the receive virtqueue is reset, the device
MUST drop such packets.

Upon applying a flow filter rule to a packet, the device MUST STOP any further
application of rules and cease applying any other steering configurations.

For multiple flow filter groups, the device MUST apply the rules from
the group with the highest priority. If any rule from this group is applied,
the device MUST ignore the remaining groups. If none of the rules from the
highest priority group match, the device MUST apply the rules from
the group with the next highest priority, until either a rule matches or
all groups have been attempted.

The device MUST apply the rules within the group from the highest to the
lowest priority until a rule matches the packet, and the device MUST take
the action. If an action is taken, the device MUST not take any other
action for this packet.

The device MAY apply the rules with the same \field{rule_priority} in any
order within the group.

The device MUST process incoming packets in the following order:
\begin{itemize}
\item apply the steering configuration received using control virtqueue
      commands VIRTIO_NET_CTRL_RX, VIRTIO_NET_CTRL_MAC, and
      VIRTIO_NET_CTRL_VLAN.
\item apply flow filter rules if any.
\item if no filter rule is applied, apply the steering configuration
      received using the command VIRTIO_NET_CTRL_MQ_RSS_CONFIG
      or according to automatic receive steering.
\end{itemize}

When processing an incoming packet, if the packet is dropped at any stage, the device
MUST skip further processing.

When the device drops the packet due to the configuration done using the control
virtqueue commands VIRTIO_NET_CTRL_RX or VIRTIO_NET_CTRL_MAC or VIRTIO_NET_CTRL_VLAN,
the device MUST skip flow filter rules for this packet.

When the device performs flow filter match operations and if the operation
result did not have any match in all the groups, the receive packet processing
continues to next level, i.e. to apply configuration done using
VIRTIO_NET_CTRL_MQ_RSS_CONFIG command.

The device MUST support the creation of flow filter classifier objects
using the command VIRTIO_ADMIN_CMD_RESOURCE_OBJ_CREATE with \field{flags}
set to VIRTIO_NET_FF_MASK_F_PARTIAL_MASK;
this support is required even if all the bits of the masks are set for
a field in \field{selectors}, provided that partial masking is supported
for the selectors.

\drivernormative{\paragraph}{Flow filter}{Device Types / Network Device / Device Operation / Flow filter}

The driver MUST enable VIRTIO_NET_FF_RESOURCE_CAP, VIRTIO_NET_FF_SELECTOR_CAP,
and VIRTIO_NET_FF_ACTION_CAP capabilities to use flow filter.

The driver SHOULD NOT remove a flow filter group using the command
VIRTIO_ADMIN_CMD_RESOURCE_OBJ_DESTROY when one or more flow filter rules
depend on that group. The driver SHOULD only destroy the group after
all the associated rules have been destroyed.

The driver SHOULD NOT remove a flow filter classifier using the command
VIRTIO_ADMIN_CMD_RESOURCE_OBJ_DESTROY when one or more flow filter rules
depend on the classifier. The driver SHOULD only destroy the classifier
after all the associated rules have been destroyed.

The driver SHOULD NOT add multiple flow filter rules with the same
\field{rule_priority} within a flow filter group, as these rules MAY match
the same packet. The driver SHOULD assign different \field{rule_priority}
values to different flow filter rules if multiple rules may match a single
packet.

For the command VIRTIO_ADMIN_CMD_RESOURCE_OBJ_CREATE, when creating a resource
of \field{type} VIRTIO_NET_RESOURCE_OBJ_FF_CLASSIFIER, the driver MUST set:
\begin{itemize}
\item \field{selectors[0].type} to VIRTIO_NET_FF_MASK_TYPE_ETH.
\item \field{selectors[1].type} to VIRTIO_NET_FF_MASK_TYPE_IPV4 or
      VIRTIO_NET_FF_MASK_TYPE_IPV6 when \field{count} is more than 1,
\item \field{selectors[2].type} VIRTIO_NET_FF_MASK_TYPE_UDP or
      VIRTIO_NET_FF_MASK_TYPE_TCP when \field{count} is more than 2.
\end{itemize}

For the command VIRTIO_ADMIN_CMD_RESOURCE_OBJ_CREATE, when creating a resource
of \field{type} VIRTIO_NET_RESOURCE_OBJ_FF_CLASSIFIER, the driver MUST set:
\begin{itemize}
\item \field{selectors[0].mask} bytes to all 1s for the \field{EtherType}
       when \field{count} is 2 or more.
\item \field{selectors[1].mask} bytes to all 1s for \field{Protocol} or \field{Next Header}
       when \field{selector[1].type} is VIRTIO_NET_FF_MASK_TYPE_IPV4 or VIRTIO_NET_FF_MASK_TYPE_IPV6,
       and when \field{count} is more than 2.
\end{itemize}

For the command VIRTIO_ADMIN_CMD_RESOURCE_OBJ_CREATE, the resource \field{type}
VIRTIO_NET_RESOURCE_OBJ_FF_RULE, if the corresponding classifier object's
\field{count} is 2 or more, the driver MUST SET the \field{keys} bytes of
\field{EtherType} in accordance with
\hyperref[intro:IEEE 802 Ethertypes]{IEEE 802 Ethertypes}
for either VIRTIO_NET_FF_MASK_TYPE_IPV4 or VIRTIO_NET_FF_MASK_TYPE_IPV6.

For the command VIRTIO_ADMIN_CMD_RESOURCE_OBJ_CREATE, when creating a resource of
\field{type} VIRTIO_NET_RESOURCE_OBJ_FF_RULE, if the corresponding classifier
object's \field{count} is more than 2, and the \field{selector[1].type} is either
VIRTIO_NET_FF_MASK_TYPE_IPV4 or VIRTIO_NET_FF_MASK_TYPE_IPV6, the driver MUST
set the \field{keys} bytes for the \field{Protocol} or \field{Next Header}
according to \hyperref[intro:IANA Protocol Numbers]{IANA Protocol Numbers} respectively.

The driver SHOULD set all the bits for a field in the mask of a selector in both the
capability and the classifier object, unless the VIRTIO_NET_FF_MASK_F_PARTIAL_MASK
is enabled.

\subsubsection{Legacy Interface: Framing Requirements}\label{sec:Device
Types / Network Device / Legacy Interface: Framing Requirements}

When using legacy interfaces, transitional drivers which have not
negotiated VIRTIO_F_ANY_LAYOUT MUST use a single descriptor for the
\field{struct virtio_net_hdr} on both transmit and receive, with the
network data in the following descriptors.

Additionally, when using the control virtqueue (see \ref{sec:Device
Types / Network Device / Device Operation / Control Virtqueue})
, transitional drivers which have not
negotiated VIRTIO_F_ANY_LAYOUT MUST:
\begin{itemize}
\item for all commands, use a single 2-byte descriptor including the first two
fields: \field{class} and \field{command}
\item for all commands except VIRTIO_NET_CTRL_MAC_TABLE_SET
use a single descriptor including command-specific-data
with no padding.
\item for the VIRTIO_NET_CTRL_MAC_TABLE_SET command use exactly
two descriptors including command-specific-data with no padding:
the first of these descriptors MUST include the
virtio_net_ctrl_mac table structure for the unicast addresses with no padding,
the second of these descriptors MUST include the
virtio_net_ctrl_mac table structure for the multicast addresses
with no padding.
\item for all commands, use a single 1-byte descriptor for the
\field{ack} field
\end{itemize}

See \ref{sec:Basic
Facilities of a Virtio Device / Virtqueues / Message Framing}.

\section{Network Device}\label{sec:Device Types / Network Device}

The virtio network device is a virtual network interface controller.
It consists of a virtual Ethernet link which connects the device
to the Ethernet network. The device has transmit and receive
queues. The driver adds empty buffers to the receive virtqueue.
The device receives incoming packets from the link; the device
places these incoming packets in the receive virtqueue buffers.
The driver adds outgoing packets to the transmit virtqueue. The device
removes these packets from the transmit virtqueue and sends them to
the link. The device may have a control virtqueue. The driver
uses the control virtqueue to dynamically manipulate various
features of the initialized device.

\subsection{Device ID}\label{sec:Device Types / Network Device / Device ID}

 1

\subsection{Virtqueues}\label{sec:Device Types / Network Device / Virtqueues}

\begin{description}
\item[0] receiveq1
\item[1] transmitq1
\item[\ldots]
\item[2(N-1)] receiveqN
\item[2(N-1)+1] transmitqN
\item[2N] controlq
\end{description}

 N=1 if neither VIRTIO_NET_F_MQ nor VIRTIO_NET_F_RSS are negotiated, otherwise N is set by
 \field{max_virtqueue_pairs}.

controlq is optional; it only exists if VIRTIO_NET_F_CTRL_VQ is
negotiated.

\subsection{Feature bits}\label{sec:Device Types / Network Device / Feature bits}

\begin{description}
\item[VIRTIO_NET_F_CSUM (0)] Device handles packets with partial checksum offload.

\item[VIRTIO_NET_F_GUEST_CSUM (1)] Driver handles packets with partial checksum.

\item[VIRTIO_NET_F_CTRL_GUEST_OFFLOADS (2)] Control channel offloads
        reconfiguration support.

\item[VIRTIO_NET_F_MTU(3)] Device maximum MTU reporting is supported. If
    offered by the device, device advises driver about the value of
    its maximum MTU. If negotiated, the driver uses \field{mtu} as
    the maximum MTU value.

\item[VIRTIO_NET_F_MAC (5)] Device has given MAC address.

\item[VIRTIO_NET_F_GUEST_TSO4 (7)] Driver can receive TSOv4.

\item[VIRTIO_NET_F_GUEST_TSO6 (8)] Driver can receive TSOv6.

\item[VIRTIO_NET_F_GUEST_ECN (9)] Driver can receive TSO with ECN.

\item[VIRTIO_NET_F_GUEST_UFO (10)] Driver can receive UFO.

\item[VIRTIO_NET_F_HOST_TSO4 (11)] Device can receive TSOv4.

\item[VIRTIO_NET_F_HOST_TSO6 (12)] Device can receive TSOv6.

\item[VIRTIO_NET_F_HOST_ECN (13)] Device can receive TSO with ECN.

\item[VIRTIO_NET_F_HOST_UFO (14)] Device can receive UFO.

\item[VIRTIO_NET_F_MRG_RXBUF (15)] Driver can merge receive buffers.

\item[VIRTIO_NET_F_STATUS (16)] Configuration status field is
    available.

\item[VIRTIO_NET_F_CTRL_VQ (17)] Control channel is available.

\item[VIRTIO_NET_F_CTRL_RX (18)] Control channel RX mode support.

\item[VIRTIO_NET_F_CTRL_VLAN (19)] Control channel VLAN filtering.

\item[VIRTIO_NET_F_CTRL_RX_EXTRA (20)]	Control channel RX extra mode support.

\item[VIRTIO_NET_F_GUEST_ANNOUNCE(21)] Driver can send gratuitous
    packets.

\item[VIRTIO_NET_F_MQ(22)] Device supports multiqueue with automatic
    receive steering.

\item[VIRTIO_NET_F_CTRL_MAC_ADDR(23)] Set MAC address through control
    channel.

\item[VIRTIO_NET_F_DEVICE_STATS(50)] Device can provide device-level statistics
    to the driver through the control virtqueue.

\item[VIRTIO_NET_F_HASH_TUNNEL(51)] Device supports inner header hash for encapsulated packets.

\item[VIRTIO_NET_F_VQ_NOTF_COAL(52)] Device supports virtqueue notification coalescing.

\item[VIRTIO_NET_F_NOTF_COAL(53)] Device supports notifications coalescing.

\item[VIRTIO_NET_F_GUEST_USO4 (54)] Driver can receive USOv4 packets.

\item[VIRTIO_NET_F_GUEST_USO6 (55)] Driver can receive USOv6 packets.

\item[VIRTIO_NET_F_HOST_USO (56)] Device can receive USO packets. Unlike UFO
 (fragmenting the packet) the USO splits large UDP packet
 to several segments when each of these smaller packets has UDP header.

\item[VIRTIO_NET_F_HASH_REPORT(57)] Device can report per-packet hash
    value and a type of calculated hash.

\item[VIRTIO_NET_F_GUEST_HDRLEN(59)] Driver can provide the exact \field{hdr_len}
    value. Device benefits from knowing the exact header length.

\item[VIRTIO_NET_F_RSS(60)] Device supports RSS (receive-side scaling)
    with Toeplitz hash calculation and configurable hash
    parameters for receive steering.

\item[VIRTIO_NET_F_RSC_EXT(61)] Device can process duplicated ACKs
    and report number of coalesced segments and duplicated ACKs.

\item[VIRTIO_NET_F_STANDBY(62)] Device may act as a standby for a primary
    device with the same MAC address.

\item[VIRTIO_NET_F_SPEED_DUPLEX(63)] Device reports speed and duplex.

\item[VIRTIO_NET_F_RSS_CONTEXT(64)] Device supports multiple RSS contexts.

\item[VIRTIO_NET_F_GUEST_UDP_TUNNEL_GSO (65)] Driver can receive GSO packets
  carried by a UDP tunnel.

\item[VIRTIO_NET_F_GUEST_UDP_TUNNEL_GSO_CSUM (66)] Driver handles packets
  carried by a UDP tunnel with partial csum for the outer header.

\item[VIRTIO_NET_F_HOST_UDP_TUNNEL_GSO (67)] Device can receive GSO packets
  carried by a UDP tunnel.

\item[VIRTIO_NET_F_HOST_UDP_TUNNEL_GSO_CSUM (68)] Device handles packets
  carried by a UDP tunnel with partial csum for the outer header.
\end{description}

\subsubsection{Feature bit requirements}\label{sec:Device Types / Network Device / Feature bits / Feature bit requirements}

Some networking feature bits require other networking feature bits
(see \ref{drivernormative:Basic Facilities of a Virtio Device / Feature Bits}):

\begin{description}
\item[VIRTIO_NET_F_GUEST_TSO4] Requires VIRTIO_NET_F_GUEST_CSUM.
\item[VIRTIO_NET_F_GUEST_TSO6] Requires VIRTIO_NET_F_GUEST_CSUM.
\item[VIRTIO_NET_F_GUEST_ECN] Requires VIRTIO_NET_F_GUEST_TSO4 or VIRTIO_NET_F_GUEST_TSO6.
\item[VIRTIO_NET_F_GUEST_UFO] Requires VIRTIO_NET_F_GUEST_CSUM.
\item[VIRTIO_NET_F_GUEST_USO4] Requires VIRTIO_NET_F_GUEST_CSUM.
\item[VIRTIO_NET_F_GUEST_USO6] Requires VIRTIO_NET_F_GUEST_CSUM.
\item[VIRTIO_NET_F_GUEST_UDP_TUNNEL_GSO] Requires VIRTIO_NET_F_GUEST_TSO4, VIRTIO_NET_F_GUEST_TSO6,
   VIRTIO_NET_F_GUEST_USO4 and VIRTIO_NET_F_GUEST_USO6.
\item[VIRTIO_NET_F_GUEST_UDP_TUNNEL_GSO_CSUM] Requires VIRTIO_NET_F_GUEST_UDP_TUNNEL_GSO

\item[VIRTIO_NET_F_HOST_TSO4] Requires VIRTIO_NET_F_CSUM.
\item[VIRTIO_NET_F_HOST_TSO6] Requires VIRTIO_NET_F_CSUM.
\item[VIRTIO_NET_F_HOST_ECN] Requires VIRTIO_NET_F_HOST_TSO4 or VIRTIO_NET_F_HOST_TSO6.
\item[VIRTIO_NET_F_HOST_UFO] Requires VIRTIO_NET_F_CSUM.
\item[VIRTIO_NET_F_HOST_USO] Requires VIRTIO_NET_F_CSUM.
\item[VIRTIO_NET_F_HOST_UDP_TUNNEL_GSO] Requires VIRTIO_NET_F_HOST_TSO4, VIRTIO_NET_F_HOST_TSO6
   and VIRTIO_NET_F_HOST_USO.
\item[VIRTIO_NET_F_HOST_UDP_TUNNEL_GSO_CSUM] Requires VIRTIO_NET_F_HOST_UDP_TUNNEL_GSO

\item[VIRTIO_NET_F_CTRL_RX] Requires VIRTIO_NET_F_CTRL_VQ.
\item[VIRTIO_NET_F_CTRL_VLAN] Requires VIRTIO_NET_F_CTRL_VQ.
\item[VIRTIO_NET_F_GUEST_ANNOUNCE] Requires VIRTIO_NET_F_CTRL_VQ.
\item[VIRTIO_NET_F_MQ] Requires VIRTIO_NET_F_CTRL_VQ.
\item[VIRTIO_NET_F_CTRL_MAC_ADDR] Requires VIRTIO_NET_F_CTRL_VQ.
\item[VIRTIO_NET_F_NOTF_COAL] Requires VIRTIO_NET_F_CTRL_VQ.
\item[VIRTIO_NET_F_RSC_EXT] Requires VIRTIO_NET_F_HOST_TSO4 or VIRTIO_NET_F_HOST_TSO6.
\item[VIRTIO_NET_F_RSS] Requires VIRTIO_NET_F_CTRL_VQ.
\item[VIRTIO_NET_F_VQ_NOTF_COAL] Requires VIRTIO_NET_F_CTRL_VQ.
\item[VIRTIO_NET_F_HASH_TUNNEL] Requires VIRTIO_NET_F_CTRL_VQ along with VIRTIO_NET_F_RSS or VIRTIO_NET_F_HASH_REPORT.
\item[VIRTIO_NET_F_RSS_CONTEXT] Requires VIRTIO_NET_F_CTRL_VQ and VIRTIO_NET_F_RSS.
\end{description}

\begin{note}
The dependency between UDP_TUNNEL_GSO_CSUM and UDP_TUNNEL_GSO is intentionally
in the opposite direction with respect to the plain GSO features and the plain
checksum offload because UDP tunnel checksum offload gives very little gain
for non GSO packets and is quite complex to implement in H/W.
\end{note}

\subsubsection{Legacy Interface: Feature bits}\label{sec:Device Types / Network Device / Feature bits / Legacy Interface: Feature bits}
\begin{description}
\item[VIRTIO_NET_F_GSO (6)] Device handles packets with any GSO type. This was supposed to indicate segmentation offload support, but
upon further investigation it became clear that multiple bits were needed.
\item[VIRTIO_NET_F_GUEST_RSC4 (41)] Device coalesces TCPIP v4 packets. This was implemented by hypervisor patch for certification
purposes and current Windows driver depends on it. It will not function if virtio-net device reports this feature.
\item[VIRTIO_NET_F_GUEST_RSC6 (42)] Device coalesces TCPIP v6 packets. Similar to VIRTIO_NET_F_GUEST_RSC4.
\end{description}

\subsection{Device configuration layout}\label{sec:Device Types / Network Device / Device configuration layout}
\label{sec:Device Types / Block Device / Feature bits / Device configuration layout}

The network device has the following device configuration layout.
All of the device configuration fields are read-only for the driver.

\begin{lstlisting}
struct virtio_net_config {
        u8 mac[6];
        le16 status;
        le16 max_virtqueue_pairs;
        le16 mtu;
        le32 speed;
        u8 duplex;
        u8 rss_max_key_size;
        le16 rss_max_indirection_table_length;
        le32 supported_hash_types;
        le32 supported_tunnel_types;
};
\end{lstlisting}

The \field{mac} address field always exists (although it is only
valid if VIRTIO_NET_F_MAC is set).

The \field{status} only exists if VIRTIO_NET_F_STATUS is set.
Two bits are currently defined for the status field: VIRTIO_NET_S_LINK_UP
and VIRTIO_NET_S_ANNOUNCE.

\begin{lstlisting}
#define VIRTIO_NET_S_LINK_UP     1
#define VIRTIO_NET_S_ANNOUNCE    2
\end{lstlisting}

The following field, \field{max_virtqueue_pairs} only exists if
VIRTIO_NET_F_MQ or VIRTIO_NET_F_RSS is set. This field specifies the maximum number
of each of transmit and receive virtqueues (receiveq1\ldots receiveqN
and transmitq1\ldots transmitqN respectively) that can be configured once at least one of these features
is negotiated.

The following field, \field{mtu} only exists if VIRTIO_NET_F_MTU
is set. This field specifies the maximum MTU for the driver to
use.

The following two fields, \field{speed} and \field{duplex}, only
exist if VIRTIO_NET_F_SPEED_DUPLEX is set.

\field{speed} contains the device speed, in units of 1 MBit per
second, 0 to 0x7fffffff, or 0xffffffff for unknown speed.

\field{duplex} has the values of 0x01 for full duplex, 0x00 for
half duplex and 0xff for unknown duplex state.

Both \field{speed} and \field{duplex} can change, thus the driver
is expected to re-read these values after receiving a
configuration change notification.

The following field, \field{rss_max_key_size} only exists if VIRTIO_NET_F_RSS or VIRTIO_NET_F_HASH_REPORT is set.
It specifies the maximum supported length of RSS key in bytes.

The following field, \field{rss_max_indirection_table_length} only exists if VIRTIO_NET_F_RSS is set.
It specifies the maximum number of 16-bit entries in RSS indirection table.

The next field, \field{supported_hash_types} only exists if the device supports hash calculation,
i.e. if VIRTIO_NET_F_RSS or VIRTIO_NET_F_HASH_REPORT is set.

Field \field{supported_hash_types} contains the bitmask of supported hash types.
See \ref{sec:Device Types / Network Device / Device Operation / Processing of Incoming Packets / Hash calculation for incoming packets / Supported/enabled hash types} for details of supported hash types.

Field \field{supported_tunnel_types} only exists if the device supports inner header hash, i.e. if VIRTIO_NET_F_HASH_TUNNEL is set.

Field \field{supported_tunnel_types} contains the bitmask of encapsulation types supported by the device for inner header hash.
Encapsulation types are defined in \ref{sec:Device Types / Network Device / Device Operation / Processing of Incoming Packets /
Hash calculation for incoming packets / Encapsulation types supported/enabled for inner header hash}.

\devicenormative{\subsubsection}{Device configuration layout}{Device Types / Network Device / Device configuration layout}

The device MUST set \field{max_virtqueue_pairs} to between 1 and 0x8000 inclusive,
if it offers VIRTIO_NET_F_MQ.

The device MUST set \field{mtu} to between 68 and 65535 inclusive,
if it offers VIRTIO_NET_F_MTU.

The device SHOULD set \field{mtu} to at least 1280, if it offers
VIRTIO_NET_F_MTU.

The device MUST NOT modify \field{mtu} once it has been set.

The device MUST NOT pass received packets that exceed \field{mtu} (plus low
level ethernet header length) size with \field{gso_type} NONE or ECN
after VIRTIO_NET_F_MTU has been successfully negotiated.

The device MUST forward transmitted packets of up to \field{mtu} (plus low
level ethernet header length) size with \field{gso_type} NONE or ECN, and do
so without fragmentation, after VIRTIO_NET_F_MTU has been successfully
negotiated.

The device MUST set \field{rss_max_key_size} to at least 40, if it offers
VIRTIO_NET_F_RSS or VIRTIO_NET_F_HASH_REPORT.

The device MUST set \field{rss_max_indirection_table_length} to at least 128, if it offers
VIRTIO_NET_F_RSS.

If the driver negotiates the VIRTIO_NET_F_STANDBY feature, the device MAY act
as a standby device for a primary device with the same MAC address.

If VIRTIO_NET_F_SPEED_DUPLEX has been negotiated, \field{speed}
MUST contain the device speed, in units of 1 MBit per second, 0 to
0x7ffffffff, or 0xfffffffff for unknown.

If VIRTIO_NET_F_SPEED_DUPLEX has been negotiated, \field{duplex}
MUST have the values of 0x00 for full duplex, 0x01 for half
duplex, or 0xff for unknown.

If VIRTIO_NET_F_SPEED_DUPLEX and VIRTIO_NET_F_STATUS have both
been negotiated, the device SHOULD NOT change the \field{speed} and
\field{duplex} fields as long as VIRTIO_NET_S_LINK_UP is set in
the \field{status}.

The device SHOULD NOT offer VIRTIO_NET_F_HASH_REPORT if it
does not offer VIRTIO_NET_F_CTRL_VQ.

The device SHOULD NOT offer VIRTIO_NET_F_CTRL_RX_EXTRA if it
does not offer VIRTIO_NET_F_CTRL_VQ.

\drivernormative{\subsubsection}{Device configuration layout}{Device Types / Network Device / Device configuration layout}

The driver MUST NOT write to any of the device configuration fields.

A driver SHOULD negotiate VIRTIO_NET_F_MAC if the device offers it.
If the driver negotiates the VIRTIO_NET_F_MAC feature, the driver MUST set
the physical address of the NIC to \field{mac}.  Otherwise, it SHOULD
use a locally-administered MAC address (see \hyperref[intro:IEEE 802]{IEEE 802},
``9.2 48-bit universal LAN MAC addresses'').

If the driver does not negotiate the VIRTIO_NET_F_STATUS feature, it SHOULD
assume the link is active, otherwise it SHOULD read the link status from
the bottom bit of \field{status}.

A driver SHOULD negotiate VIRTIO_NET_F_MTU if the device offers it.

If the driver negotiates VIRTIO_NET_F_MTU, it MUST supply enough receive
buffers to receive at least one receive packet of size \field{mtu} (plus low
level ethernet header length) with \field{gso_type} NONE or ECN.

If the driver negotiates VIRTIO_NET_F_MTU, it MUST NOT transmit packets of
size exceeding the value of \field{mtu} (plus low level ethernet header length)
with \field{gso_type} NONE or ECN.

A driver SHOULD negotiate the VIRTIO_NET_F_STANDBY feature if the device offers it.

If VIRTIO_NET_F_SPEED_DUPLEX has been negotiated,
the driver MUST treat any value of \field{speed} above
0x7fffffff as well as any value of \field{duplex} not
matching 0x00 or 0x01 as an unknown value.

If VIRTIO_NET_F_SPEED_DUPLEX has been negotiated, the driver
SHOULD re-read \field{speed} and \field{duplex} after a
configuration change notification.

A driver SHOULD NOT negotiate VIRTIO_NET_F_HASH_REPORT if it
does not negotiate VIRTIO_NET_F_CTRL_VQ.

A driver SHOULD NOT negotiate VIRTIO_NET_F_CTRL_RX_EXTRA if it
does not negotiate VIRTIO_NET_F_CTRL_VQ.

\subsubsection{Legacy Interface: Device configuration layout}\label{sec:Device Types / Network Device / Device configuration layout / Legacy Interface: Device configuration layout}
\label{sec:Device Types / Block Device / Feature bits / Device configuration layout / Legacy Interface: Device configuration layout}
When using the legacy interface, transitional devices and drivers
MUST format \field{status} and
\field{max_virtqueue_pairs} in struct virtio_net_config
according to the native endian of the guest rather than
(necessarily when not using the legacy interface) little-endian.

When using the legacy interface, \field{mac} is driver-writable
which provided a way for drivers to update the MAC without
negotiating VIRTIO_NET_F_CTRL_MAC_ADDR.

\subsection{Device Initialization}\label{sec:Device Types / Network Device / Device Initialization}

A driver would perform a typical initialization routine like so:

\begin{enumerate}
\item Identify and initialize the receive and
  transmission virtqueues, up to N of each kind. If
  VIRTIO_NET_F_MQ feature bit is negotiated,
  N=\field{max_virtqueue_pairs}, otherwise identify N=1.

\item If the VIRTIO_NET_F_CTRL_VQ feature bit is negotiated,
  identify the control virtqueue.

\item Fill the receive queues with buffers: see \ref{sec:Device Types / Network Device / Device Operation / Setting Up Receive Buffers}.

\item Even with VIRTIO_NET_F_MQ, only receiveq1, transmitq1 and
  controlq are used by default.  The driver would send the
  VIRTIO_NET_CTRL_MQ_VQ_PAIRS_SET command specifying the
  number of the transmit and receive queues to use.

\item If the VIRTIO_NET_F_MAC feature bit is set, the configuration
  space \field{mac} entry indicates the ``physical'' address of the
  device, otherwise the driver would typically generate a random
  local MAC address.

\item If the VIRTIO_NET_F_STATUS feature bit is negotiated, the link
  status comes from the bottom bit of \field{status}.
  Otherwise, the driver assumes it's active.

\item A performant driver would indicate that it will generate checksumless
  packets by negotiating the VIRTIO_NET_F_CSUM feature.

\item If that feature is negotiated, a driver can use TCP segmentation or UDP
  segmentation/fragmentation offload by negotiating the VIRTIO_NET_F_HOST_TSO4 (IPv4
  TCP), VIRTIO_NET_F_HOST_TSO6 (IPv6 TCP), VIRTIO_NET_F_HOST_UFO
  (UDP fragmentation) and VIRTIO_NET_F_HOST_USO (UDP segmentation) features.

\item If the VIRTIO_NET_F_HOST_TSO6, VIRTIO_NET_F_HOST_TSO4 and VIRTIO_NET_F_HOST_USO
  segmentation features are negotiated, a driver can
  use TCP segmentation or UDP segmentation on top of UDP encapsulation
  offload, when the outer header does not require checksumming - e.g.
  the outer UDP checksum is zero - by negotiating the
  VIRTIO_NET_F_HOST_UDP_TUNNEL_GSO feature.
  GSO over UDP tunnels packets carry two sets of headers: the outer ones
  and the inner ones. The outer transport protocol is UDP, the inner
  could be either TCP or UDP. Only a single level of encapsulation
  offload is supported.

\item If VIRTIO_NET_F_HOST_UDP_TUNNEL_GSO is negotiated, a driver can
  additionally use TCP segmentation or UDP segmentation on top of UDP
  encapsulation with the outer header requiring checksum offload,
  negotiating the VIRTIO_NET_F_HOST_UDP_TUNNEL_GSO_CSUM feature.

\item The converse features are also available: a driver can save
  the virtual device some work by negotiating these features.\note{For example, a network packet transported between two guests on
the same system might not need checksumming at all, nor segmentation,
if both guests are amenable.}
   The VIRTIO_NET_F_GUEST_CSUM feature indicates that partially
  checksummed packets can be received, and if it can do that then
  the VIRTIO_NET_F_GUEST_TSO4, VIRTIO_NET_F_GUEST_TSO6,
  VIRTIO_NET_F_GUEST_UFO, VIRTIO_NET_F_GUEST_ECN, VIRTIO_NET_F_GUEST_USO4,
  VIRTIO_NET_F_GUEST_USO6 VIRTIO_NET_F_GUEST_UDP_TUNNEL_GSO and
  VIRTIO_NET_F_GUEST_UDP_TUNNEL_GSO_CSUM are the input equivalents of
  the features described above.
  See \ref{sec:Device Types / Network Device / Device Operation /
Setting Up Receive Buffers}~\nameref{sec:Device Types / Network
Device / Device Operation / Setting Up Receive Buffers} and
\ref{sec:Device Types / Network Device / Device Operation /
Processing of Incoming Packets}~\nameref{sec:Device Types /
Network Device / Device Operation / Processing of Incoming Packets} below.
\end{enumerate}

A truly minimal driver would only accept VIRTIO_NET_F_MAC and ignore
everything else.

\subsection{Device and driver capabilities}\label{sec:Device Types / Network Device / Device and driver capabilities}

The network device has the following capabilities.

\begin{tabularx}{\textwidth}{ |l||l|X| }
\hline
Identifier & Name & Description \\
\hline \hline
0x0800 & \hyperref[par:Device Types / Network Device / Device Operation / Flow filter / Device and driver capabilities / VIRTIO-NET-FF-RESOURCE-CAP]{VIRTIO_NET_FF_RESOURCE_CAP} & Flow filter resource capability \\
\hline
0x0801 & \hyperref[par:Device Types / Network Device / Device Operation / Flow filter / Device and driver capabilities / VIRTIO-NET-FF-SELECTOR-CAP]{VIRTIO_NET_FF_SELECTOR_CAP} & Flow filter classifier capability \\
\hline
0x0802 & \hyperref[par:Device Types / Network Device / Device Operation / Flow filter / Device and driver capabilities / VIRTIO-NET-FF-ACTION-CAP]{VIRTIO_NET_FF_ACTION_CAP} & Flow filter action capability \\
\hline
\end{tabularx}

\subsection{Device resource objects}\label{sec:Device Types / Network Device / Device resource objects}

The network device has the following resource objects.

\begin{tabularx}{\textwidth}{ |l||l|X| }
\hline
type & Name & Description \\
\hline \hline
0x0200 & \hyperref[par:Device Types / Network Device / Device Operation / Flow filter / Resource objects / VIRTIO-NET-RESOURCE-OBJ-FF-GROUP]{VIRTIO_NET_RESOURCE_OBJ_FF_GROUP} & Flow filter group resource object \\
\hline
0x0201 & \hyperref[par:Device Types / Network Device / Device Operation / Flow filter / Resource objects / VIRTIO-NET-RESOURCE-OBJ-FF-CLASSIFIER]{VIRTIO_NET_RESOURCE_OBJ_FF_CLASSIFIER} & Flow filter mask object \\
\hline
0x0202 & \hyperref[par:Device Types / Network Device / Device Operation / Flow filter / Resource objects / VIRTIO-NET-RESOURCE-OBJ-FF-RULE]{VIRTIO_NET_RESOURCE_OBJ_FF_RULE} & Flow filter rule object \\
\hline
\end{tabularx}

\subsection{Device parts}\label{sec:Device Types / Network Device / Device parts}

Network device parts represent the configuration done by the driver using control
virtqueue commands. Network device part is in the format of
\field{struct virtio_dev_part}.

\begin{tabularx}{\textwidth}{ |l||l|X| }
\hline
Type & Name & Description \\
\hline \hline
0x200 & VIRTIO_NET_DEV_PART_CVQ_CFG_PART & Represents device configuration done through a control virtqueue command, see \ref{sec:Device Types / Network Device / Device parts / VIRTIO-NET-DEV-PART-CVQ-CFG-PART} \\
\hline
0x201 - 0x5FF & - & reserved for future \\
\hline
\hline
\end{tabularx}

\subsubsection{VIRTIO_NET_DEV_PART_CVQ_CFG_PART}\label{sec:Device Types / Network Device / Device parts / VIRTIO-NET-DEV-PART-CVQ-CFG-PART}

For VIRTIO_NET_DEV_PART_CVQ_CFG_PART, \field{part_type} is set to 0x200. The
VIRTIO_NET_DEV_PART_CVQ_CFG_PART part indicates configuration performed by the
driver using a control virtqueue command.

\begin{lstlisting}
struct virtio_net_dev_part_cvq_selector {
        u8 class;
        u8 command;
        u8 reserved[6];
};
\end{lstlisting}

There is one device part of type VIRTIO_NET_DEV_PART_CVQ_CFG_PART for each
individual configuration. Each part is identified by a unique selector value.
The selector, \field{device_type_raw}, is in the format
\field{struct virtio_net_dev_part_cvq_selector}.

The selector consists of two fields: \field{class} and \field{command}. These
fields correspond to the \field{class} and \field{command} defined in
\field{struct virtio_net_ctrl}, as described in the relevant sections of
\ref{sec:Device Types / Network Device / Device Operation / Control Virtqueue}.

The value corresponding to each part’s selector follows the same format as the
respective \field{command-specific-data} described in the relevant sections of
\ref{sec:Device Types / Network Device / Device Operation / Control Virtqueue}.

For example, when the \field{class} is VIRTIO_NET_CTRL_MAC, the \field{command}
can be either VIRTIO_NET_CTRL_MAC_TABLE_SET or VIRTIO_NET_CTRL_MAC_ADDR_SET;
when \field{command} is set to VIRTIO_NET_CTRL_MAC_TABLE_SET, \field{value}
is in the format of \field{struct virtio_net_ctrl_mac}.

Supported selectors are listed in the table:

\begin{tabularx}{\textwidth}{ |l|X| }
\hline
Class selector & Command selector \\
\hline \hline
VIRTIO_NET_CTRL_RX & VIRTIO_NET_CTRL_RX_PROMISC \\
\hline
VIRTIO_NET_CTRL_RX & VIRTIO_NET_CTRL_RX_ALLMULTI \\
\hline
VIRTIO_NET_CTRL_RX & VIRTIO_NET_CTRL_RX_ALLUNI \\
\hline
VIRTIO_NET_CTRL_RX & VIRTIO_NET_CTRL_RX_NOMULTI \\
\hline
VIRTIO_NET_CTRL_RX & VIRTIO_NET_CTRL_RX_NOUNI \\
\hline
VIRTIO_NET_CTRL_RX & VIRTIO_NET_CTRL_RX_NOBCAST \\
\hline
VIRTIO_NET_CTRL_MAC & VIRTIO_NET_CTRL_MAC_TABLE_SET \\
\hline
VIRTIO_NET_CTRL_MAC & VIRTIO_NET_CTRL_MAC_ADDR_SET \\
\hline
VIRTIO_NET_CTRL_VLAN & VIRTIO_NET_CTRL_VLAN_ADD \\
\hline
VIRTIO_NET_CTRL_ANNOUNCE & VIRTIO_NET_CTRL_ANNOUNCE_ACK \\
\hline
VIRTIO_NET_CTRL_MQ & VIRTIO_NET_CTRL_MQ_VQ_PAIRS_SET \\
\hline
VIRTIO_NET_CTRL_MQ & VIRTIO_NET_CTRL_MQ_RSS_CONFIG \\
\hline
VIRTIO_NET_CTRL_MQ & VIRTIO_NET_CTRL_MQ_HASH_CONFIG \\
\hline
\hline
\end{tabularx}

For command selector VIRTIO_NET_CTRL_VLAN_ADD, device part consists of a whole
VLAN table.

\field{reserved} is reserved and set to zero.

\subsection{Device Operation}\label{sec:Device Types / Network Device / Device Operation}

Packets are transmitted by placing them in the
transmitq1\ldots transmitqN, and buffers for incoming packets are
placed in the receiveq1\ldots receiveqN. In each case, the packet
itself is preceded by a header:

\begin{lstlisting}
struct virtio_net_hdr {
#define VIRTIO_NET_HDR_F_NEEDS_CSUM    1
#define VIRTIO_NET_HDR_F_DATA_VALID    2
#define VIRTIO_NET_HDR_F_RSC_INFO      4
#define VIRTIO_NET_HDR_F_UDP_TUNNEL_CSUM 8
        u8 flags;
#define VIRTIO_NET_HDR_GSO_NONE        0
#define VIRTIO_NET_HDR_GSO_TCPV4       1
#define VIRTIO_NET_HDR_GSO_UDP         3
#define VIRTIO_NET_HDR_GSO_TCPV6       4
#define VIRTIO_NET_HDR_GSO_UDP_L4      5
#define VIRTIO_NET_HDR_GSO_UDP_TUNNEL_IPV4 0x20
#define VIRTIO_NET_HDR_GSO_UDP_TUNNEL_IPV6 0x40
#define VIRTIO_NET_HDR_GSO_ECN      0x80
        u8 gso_type;
        le16 hdr_len;
        le16 gso_size;
        le16 csum_start;
        le16 csum_offset;
        le16 num_buffers;
        le32 hash_value;        (Only if VIRTIO_NET_F_HASH_REPORT negotiated)
        le16 hash_report;       (Only if VIRTIO_NET_F_HASH_REPORT negotiated)
        le16 padding_reserved;  (Only if VIRTIO_NET_F_HASH_REPORT negotiated)
        le16 outer_th_offset    (Only if VIRTIO_NET_F_HOST_UDP_TUNNEL_GSO or VIRTIO_NET_F_GUEST_UDP_TUNNEL_GSO negotiated)
        le16 inner_nh_offset;   (Only if VIRTIO_NET_F_HOST_UDP_TUNNEL_GSO or VIRTIO_NET_F_GUEST_UDP_TUNNEL_GSO negotiated)
};
\end{lstlisting}

The controlq is used to control various device features described further in
section \ref{sec:Device Types / Network Device / Device Operation / Control Virtqueue}.

\subsubsection{Legacy Interface: Device Operation}\label{sec:Device Types / Network Device / Device Operation / Legacy Interface: Device Operation}
When using the legacy interface, transitional devices and drivers
MUST format the fields in \field{struct virtio_net_hdr}
according to the native endian of the guest rather than
(necessarily when not using the legacy interface) little-endian.

The legacy driver only presented \field{num_buffers} in the \field{struct virtio_net_hdr}
when VIRTIO_NET_F_MRG_RXBUF was negotiated; without that feature the
structure was 2 bytes shorter.

When using the legacy interface, the driver SHOULD ignore the
used length for the transmit queues
and the controlq queue.
\begin{note}
Historically, some devices put
the total descriptor length there, even though no data was
actually written.
\end{note}

\subsubsection{Packet Transmission}\label{sec:Device Types / Network Device / Device Operation / Packet Transmission}

Transmitting a single packet is simple, but varies depending on
the different features the driver negotiated.

\begin{enumerate}
\item The driver can send a completely checksummed packet.  In this case,
  \field{flags} will be zero, and \field{gso_type} will be VIRTIO_NET_HDR_GSO_NONE.

\item If the driver negotiated VIRTIO_NET_F_CSUM, it can skip
  checksumming the packet:
  \begin{itemize}
  \item \field{flags} has the VIRTIO_NET_HDR_F_NEEDS_CSUM set,

  \item \field{csum_start} is set to the offset within the packet to begin checksumming,
    and

  \item \field{csum_offset} indicates how many bytes after the csum_start the
    new (16 bit ones' complement) checksum is placed by the device.

  \item The TCP checksum field in the packet is set to the sum
    of the TCP pseudo header, so that replacing it by the ones'
    complement checksum of the TCP header and body will give the
    correct result.
  \end{itemize}

\begin{note}
For example, consider a partially checksummed TCP (IPv4) packet.
It will have a 14 byte ethernet header and 20 byte IP header
followed by the TCP header (with the TCP checksum field 16 bytes
into that header). \field{csum_start} will be 14+20 = 34 (the TCP
checksum includes the header), and \field{csum_offset} will be 16.
If the given packet has the VIRTIO_NET_HDR_GSO_UDP_TUNNEL_IPV4 bit or the
VIRTIO_NET_HDR_GSO_UDP_TUNNEL_IPV6 bit set,
the above checksum fields refer to the inner header checksum, see
the example below.
\end{note}

\item If the driver negotiated
  VIRTIO_NET_F_HOST_TSO4, TSO6, USO or UFO, and the packet requires
  TCP segmentation, UDP segmentation or fragmentation, then \field{gso_type}
  is set to VIRTIO_NET_HDR_GSO_TCPV4, TCPV6, UDP_L4 or UDP.
  (Otherwise, it is set to VIRTIO_NET_HDR_GSO_NONE). In this
  case, packets larger than 1514 bytes can be transmitted: the
  metadata indicates how to replicate the packet header to cut it
  into smaller packets. The other gso fields are set:

  \begin{itemize}
  \item If the VIRTIO_NET_F_GUEST_HDRLEN feature has been negotiated,
    \field{hdr_len} indicates the header length that needs to be replicated
    for each packet. It's the number of bytes from the beginning of the packet
    to the beginning of the transport payload.
    If the \field{gso_type} has the VIRTIO_NET_HDR_GSO_UDP_TUNNEL_IPV4 bit or
    VIRTIO_NET_HDR_GSO_UDP_TUNNEL_IPV6 bit set, \field{hdr_len} accounts for
    all the headers up to and including the inner transport.
    Otherwise, if the VIRTIO_NET_F_GUEST_HDRLEN feature has not been negotiated,
    \field{hdr_len} is a hint to the device as to how much of the header
    needs to be kept to copy into each packet, usually set to the
    length of the headers, including the transport header\footnote{Due to various bugs in implementations, this field is not useful
as a guarantee of the transport header size.
}.

  \begin{note}
  Some devices benefit from knowledge of the exact header length.
  \end{note}

  \item \field{gso_size} is the maximum size of each packet beyond that
    header (ie. MSS).

  \item If the driver negotiated the VIRTIO_NET_F_HOST_ECN feature,
    the VIRTIO_NET_HDR_GSO_ECN bit in \field{gso_type}
    indicates that the TCP packet has the ECN bit set\footnote{This case is not handled by some older hardware, so is called out
specifically in the protocol.}.
   \end{itemize}

\item If the driver negotiated the VIRTIO_NET_F_HOST_UDP_TUNNEL_GSO feature and the
  \field{gso_type} has the VIRTIO_NET_HDR_GSO_UDP_TUNNEL_IPV4 bit or
  VIRTIO_NET_HDR_GSO_UDP_TUNNEL_IPV6 bit set, the GSO protocol is encapsulated
  in a UDP tunnel.
  If the outer UDP header requires checksumming, the driver must have
  additionally negotiated the VIRTIO_NET_F_HOST_UDP_TUNNEL_GSO_CSUM feature
  and offloaded the outer checksum accordingly, otherwise
  the outer UDP header must not require checksum validation, i.e. the outer
  UDP checksum must be positive zero (0x0) as defined in UDP RFC 768.
  The other tunnel-related fields indicate how to replicate the packet
  headers to cut it into smaller packets:

  \begin{itemize}
  \item \field{outer_th_offset} field indicates the outer transport header within
      the packet. This field differs from \field{csum_start} as the latter
      points to the inner transport header within the packet.

  \item \field{inner_nh_offset} field indicates the inner network header within
      the packet.
  \end{itemize}

\begin{note}
For example, consider a partially checksummed TCP (IPv4) packet carried over a
Geneve UDP tunnel (again IPv4) with no tunnel options. The
only relevant variable related to the tunnel type is the tunnel header length.
The packet will have a 14 byte outer ethernet header, 20 byte outer IP header
followed by the 8 byte UDP header (with a 0 checksum value), 8 byte Geneve header,
14 byte inner ethernet header, 20 byte inner IP header
and the TCP header (with the TCP checksum field 16 bytes
into that header). \field{csum_start} will be 14+20+8+8+14+20 = 84 (the TCP
checksum includes the header), \field{csum_offset} will be 16.
\field{inner_nh_offset} will be 14+20+8+8+14 = 62, \field{outer_th_offset} will be
14+20+8 = 42 and \field{gso_type} will be
VIRTIO_NET_HDR_GSO_TCPV4 | VIRTIO_NET_HDR_GSO_UDP_TUNNEL_IPV4 = 0x21
\end{note}

\item If the driver negotiated the VIRTIO_NET_F_HOST_UDP_TUNNEL_GSO_CSUM feature,
  the transmitted packet is a GSO one encapsulated in a UDP tunnel, and
  the outer UDP header requires checksumming, the driver can skip checksumming
  the outer header:

  \begin{itemize}
  \item \field{flags} has the VIRTIO_NET_HDR_F_UDP_TUNNEL_CSUM set,

  \item The outer UDP checksum field in the packet is set to the sum
    of the UDP pseudo header, so that replacing it by the ones'
    complement checksum of the outer UDP header and payload will give the
    correct result.
  \end{itemize}

\item \field{num_buffers} is set to zero.  This field is unused on transmitted packets.

\item The header and packet are added as one output descriptor to the
  transmitq, and the device is notified of the new entry
  (see \ref{sec:Device Types / Network Device / Device Initialization}~\nameref{sec:Device Types / Network Device / Device Initialization}).
\end{enumerate}

\drivernormative{\paragraph}{Packet Transmission}{Device Types / Network Device / Device Operation / Packet Transmission}

For the transmit packet buffer, the driver MUST use the size of the
structure \field{struct virtio_net_hdr} same as the receive packet buffer.

The driver MUST set \field{num_buffers} to zero.

If VIRTIO_NET_F_CSUM is not negotiated, the driver MUST set
\field{flags} to zero and SHOULD supply a fully checksummed
packet to the device.

If VIRTIO_NET_F_HOST_TSO4 is negotiated, the driver MAY set
\field{gso_type} to VIRTIO_NET_HDR_GSO_TCPV4 to request TCPv4
segmentation, otherwise the driver MUST NOT set
\field{gso_type} to VIRTIO_NET_HDR_GSO_TCPV4.

If VIRTIO_NET_F_HOST_TSO6 is negotiated, the driver MAY set
\field{gso_type} to VIRTIO_NET_HDR_GSO_TCPV6 to request TCPv6
segmentation, otherwise the driver MUST NOT set
\field{gso_type} to VIRTIO_NET_HDR_GSO_TCPV6.

If VIRTIO_NET_F_HOST_UFO is negotiated, the driver MAY set
\field{gso_type} to VIRTIO_NET_HDR_GSO_UDP to request UDP
fragmentation, otherwise the driver MUST NOT set
\field{gso_type} to VIRTIO_NET_HDR_GSO_UDP.

If VIRTIO_NET_F_HOST_USO is negotiated, the driver MAY set
\field{gso_type} to VIRTIO_NET_HDR_GSO_UDP_L4 to request UDP
segmentation, otherwise the driver MUST NOT set
\field{gso_type} to VIRTIO_NET_HDR_GSO_UDP_L4.

The driver SHOULD NOT send to the device TCP packets requiring segmentation offload
which have the Explicit Congestion Notification bit set, unless the
VIRTIO_NET_F_HOST_ECN feature is negotiated, in which case the
driver MUST set the VIRTIO_NET_HDR_GSO_ECN bit in
\field{gso_type}.

If VIRTIO_NET_F_HOST_UDP_TUNNEL_GSO is negotiated, the driver MAY set
VIRTIO_NET_HDR_GSO_UDP_TUNNEL_IPV4 bit or the VIRTIO_NET_HDR_GSO_UDP_TUNNEL_IPV6 bit
in \field{gso_type} according to the inner network header protocol type
to request GSO packets over UDPv4 or UDPv6 tunnel segmentation,
otherwise the driver MUST NOT set either the
VIRTIO_NET_HDR_GSO_UDP_TUNNEL_IPV4 bit or the VIRTIO_NET_HDR_GSO_UDP_TUNNEL_IPV6 bit
in \field{gso_type}.

When requesting GSO segmentation over UDP tunnel, the driver MUST SET the
VIRTIO_NET_HDR_GSO_UDP_TUNNEL_IPV4 bit if the inner network header is IPv4, i.e. the
packet is a TCPv4 GSO one, otherwise, if the inner network header is IPv6, the driver
MUST SET the VIRTIO_NET_HDR_GSO_UDP_TUNNEL_IPV6 bit.

The driver MUST NOT send to the device GSO packets over UDP tunnel
requiring segmentation and outer UDP checksum offload, unless both the
VIRTIO_NET_F_HOST_UDP_TUNNEL_GSO and VIRTIO_NET_F_HOST_UDP_TUNNEL_GSO_CSUM features
are negotiated, in which case the driver MUST set either the
VIRTIO_NET_HDR_GSO_UDP_TUNNEL_IPV4 bit or the VIRTIO_NET_HDR_GSO_UDP_TUNNEL_IPV6
bit in the \field{gso_type} and the VIRTIO_NET_HDR_F_UDP_TUNNEL_CSUM bit in
the \field{flags}.

If VIRTIO_NET_F_HOST_UDP_TUNNEL_GSO_CSUM is not negotiated, the driver MUST not set
the VIRTIO_NET_HDR_F_UDP_TUNNEL_CSUM bit in the \field{flags} and
MUST NOT send to the device GSO packets over UDP tunnel
requiring segmentation and outer UDP checksum offload.

The driver MUST NOT set the VIRTIO_NET_HDR_GSO_UDP_TUNNEL_IPV4 bit or the
VIRTIO_NET_HDR_GSO_UDP_TUNNEL_IPV6 bit together with VIRTIO_NET_HDR_GSO_UDP, as the
latter is deprecated in favor of UDP_L4 and no new feature will support it.

The driver MUST NOT set the VIRTIO_NET_HDR_GSO_UDP_TUNNEL_IPV4 bit and the
VIRTIO_NET_HDR_GSO_UDP_TUNNEL_IPV6 bit together.

The driver MUST NOT set the VIRTIO_NET_HDR_F_UDP_TUNNEL_CSUM bit \field{flags}
without setting either the VIRTIO_NET_HDR_GSO_UDP_TUNNEL_IPV4 bit or
the VIRTIO_NET_HDR_GSO_UDP_TUNNEL_IPV6 bit in \field{gso_type}.

If the VIRTIO_NET_F_CSUM feature has been negotiated, the
driver MAY set the VIRTIO_NET_HDR_F_NEEDS_CSUM bit in
\field{flags}, if so:
\begin{enumerate}
\item the driver MUST validate the packet checksum at
	offset \field{csum_offset} from \field{csum_start} as well as all
	preceding offsets;
\begin{note}
If \field{gso_type} differs from VIRTIO_NET_HDR_GSO_NONE and the
VIRTIO_NET_HDR_GSO_UDP_TUNNEL_IPV4 bit or the VIRTIO_NET_HDR_GSO_UDP_TUNNEL_IPV6
bit are not set in \field{gso_type}, \field{csum_offset}
points to the only transport header present in the packet, and there are no
additional preceding checksums validated by VIRTIO_NET_HDR_F_NEEDS_CSUM.
\end{note}
\item the driver MUST set the packet checksum stored in the
	buffer to the TCP/UDP pseudo header;
\item the driver MUST set \field{csum_start} and
	\field{csum_offset} such that calculating a ones'
	complement checksum from \field{csum_start} up until the end of
	the packet and storing the result at offset \field{csum_offset}
	from  \field{csum_start} will result in a fully checksummed
	packet;
\end{enumerate}

If none of the VIRTIO_NET_F_HOST_TSO4, TSO6, USO or UFO options have
been negotiated, the driver MUST set \field{gso_type} to
VIRTIO_NET_HDR_GSO_NONE.

If \field{gso_type} differs from VIRTIO_NET_HDR_GSO_NONE, then
the driver MUST also set the VIRTIO_NET_HDR_F_NEEDS_CSUM bit in
\field{flags} and MUST set \field{gso_size} to indicate the
desired MSS.

If one of the VIRTIO_NET_F_HOST_TSO4, TSO6, USO or UFO options have
been negotiated:
\begin{itemize}
\item If the VIRTIO_NET_F_GUEST_HDRLEN feature has been negotiated,
	and \field{gso_type} differs from VIRTIO_NET_HDR_GSO_NONE,
	the driver MUST set \field{hdr_len} to a value equal to the length
	of the headers, including the transport header. If \field{gso_type}
	has the VIRTIO_NET_HDR_GSO_UDP_TUNNEL_IPV4 bit or the
	VIRTIO_NET_HDR_GSO_UDP_TUNNEL_IPV6 bit set, \field{hdr_len} includes
	the inner transport header.

\item If the VIRTIO_NET_F_GUEST_HDRLEN feature has not been negotiated,
	or \field{gso_type} is VIRTIO_NET_HDR_GSO_NONE,
	the driver SHOULD set \field{hdr_len} to a value
	not less than the length of the headers, including the transport
	header.
\end{itemize}

If the VIRTIO_NET_F_HOST_UDP_TUNNEL_GSO option has been negotiated, the
driver MAY set the VIRTIO_NET_HDR_GSO_UDP_TUNNEL_IPV4 bit or the
VIRTIO_NET_HDR_GSO_UDP_TUNNEL_IPV6 bit in \field{gso_type}, if so:
\begin{itemize}
\item the driver MUST set \field{outer_th_offset} to the outer UDP header
  offset and \field{inner_nh_offset} to the inner network header offset.
  The \field{csum_start} and \field{csum_offset} fields point respectively
  to the inner transport header and inner transport checksum field.
\end{itemize}

If the VIRTIO_NET_F_HOST_UDP_TUNNEL_GSO_CSUM feature has been negotiated,
and the VIRTIO_NET_HDR_GSO_UDP_TUNNEL_IPV4 bit or
VIRTIO_NET_HDR_GSO_UDP_TUNNEL_IPV6 bit in \field{gso_type} are set,
the driver MAY set the VIRTIO_NET_HDR_F_UDP_TUNNEL_CSUM bit in
\field{flags}, if so the driver MUST set the packet outer UDP header checksum
to the outer UDP pseudo header checksum.

\begin{note}
calculating a ones' complement checksum from \field{outer_th_offset}
up until the end of the packet and storing the result at offset 6
from \field{outer_th_offset} will result in a fully checksummed outer UDP packet;
\end{note}

If the VIRTIO_NET_HDR_GSO_UDP_TUNNEL_IPV4 bit or the
VIRTIO_NET_HDR_GSO_UDP_TUNNEL_IPV6 bit in \field{gso_type} are set
and the VIRTIO_NET_F_HOST_UDP_TUNNEL_GSO_CSUM feature has not
been negotiated, the
outer UDP header MUST NOT require checksum validation. That is, the
outer UDP checksum value MUST be 0 or the validated complete checksum
for such header.

\begin{note}
The valid complete checksum of the outer UDP header of individual segments
can be computed by the driver prior to segmentation only if the GSO packet
size is a multiple of \field{gso_size}, because then all segments
have the same size and thus all data included in the outer UDP
checksum is the same for every segment. These pre-computed segment
length and checksum fields are different from those of the GSO
packet.
In this scenario the outer UDP header of the GSO packet must carry the
segmented UDP packet length.
\end{note}

If the VIRTIO_NET_F_HOST_UDP_TUNNEL_GSO option has not
been negotiated, the driver MUST NOT set either the VIRTIO_NET_HDR_F_GSO_UDP_TUNNEL_IPV4
bit or the VIRTIO_NET_HDR_F_GSO_UDP_TUNNEL_IPV6 in \field{gso_type}.

If the VIRTIO_NET_F_HOST_UDP_TUNNEL_GSO_CSUM option has not been negotiated,
the driver MUST NOT set the VIRTIO_NET_HDR_F_UDP_TUNNEL_CSUM bit
in \field{flags}.

The driver SHOULD accept the VIRTIO_NET_F_GUEST_HDRLEN feature if it has
been offered, and if it's able to provide the exact header length.

The driver MUST NOT set the VIRTIO_NET_HDR_F_DATA_VALID and
VIRTIO_NET_HDR_F_RSC_INFO bits in \field{flags}.

The driver MUST NOT set the VIRTIO_NET_HDR_F_DATA_VALID bit in \field{flags}
together with the VIRTIO_NET_HDR_F_GSO_UDP_TUNNEL_IPV4 bit or the
VIRTIO_NET_HDR_F_GSO_UDP_TUNNEL_IPV6 bit in \field{gso_type}.

\devicenormative{\paragraph}{Packet Transmission}{Device Types / Network Device / Device Operation / Packet Transmission}
The device MUST ignore \field{flag} bits that it does not recognize.

If VIRTIO_NET_HDR_F_NEEDS_CSUM bit in \field{flags} is not set, the
device MUST NOT use the \field{csum_start} and \field{csum_offset}.

If one of the VIRTIO_NET_F_HOST_TSO4, TSO6, USO or UFO options have
been negotiated:
\begin{itemize}
\item If the VIRTIO_NET_F_GUEST_HDRLEN feature has been negotiated,
	and \field{gso_type} differs from VIRTIO_NET_HDR_GSO_NONE,
	the device MAY use \field{hdr_len} as the transport header size.

	\begin{note}
	Caution should be taken by the implementation so as to prevent
	a malicious driver from attacking the device by setting an incorrect hdr_len.
	\end{note}

\item If the VIRTIO_NET_F_GUEST_HDRLEN feature has not been negotiated,
	or \field{gso_type} is VIRTIO_NET_HDR_GSO_NONE,
	the device MAY use \field{hdr_len} only as a hint about the
	transport header size.
	The device MUST NOT rely on \field{hdr_len} to be correct.

	\begin{note}
	This is due to various bugs in implementations.
	\end{note}
\end{itemize}

If both the VIRTIO_NET_HDR_GSO_UDP_TUNNEL_IPV4 bit and
the VIRTIO_NET_HDR_GSO_UDP_TUNNEL_IPV6 bit in in \field{gso_type} are set,
the device MUST NOT accept the packet.

If the VIRTIO_NET_HDR_GSO_UDP_TUNNEL_IPV4 bit and the VIRTIO_NET_HDR_GSO_UDP_TUNNEL_IPV6
bit in \field{gso_type} are not set, the device MUST NOT use the
\field{outer_th_offset} and \field{inner_nh_offset}.

If either the VIRTIO_NET_HDR_GSO_UDP_TUNNEL_IPV4 bit or
the VIRTIO_NET_HDR_GSO_UDP_TUNNEL_IPV6 bit in \field{gso_type} are set, and any of
the following is true:
\begin{itemize}
\item the VIRTIO_NET_HDR_F_NEEDS_CSUM is not set in \field{flags}
\item the VIRTIO_NET_HDR_F_DATA_VALID is set in \field{flags}
\item the \field{gso_type} excluding the VIRTIO_NET_HDR_GSO_UDP_TUNNEL_IPV4
bit and the VIRTIO_NET_HDR_GSO_UDP_TUNNEL_IPV6 bit is VIRTIO_NET_HDR_GSO_NONE
\end{itemize}
the device MUST NOT accept the packet.

If the VIRTIO_NET_HDR_F_UDP_TUNNEL_CSUM bit in \field{flags} is set,
and both the bits VIRTIO_NET_HDR_GSO_UDP_TUNNEL_IPV4 and
VIRTIO_NET_HDR_GSO_UDP_TUNNEL_IPV6 in \field{gso_type} are not set,
the device MOST NOT accept the packet.

If VIRTIO_NET_HDR_F_NEEDS_CSUM is not set, the device MUST NOT
rely on the packet checksum being correct.
\paragraph{Packet Transmission Interrupt}\label{sec:Device Types / Network Device / Device Operation / Packet Transmission / Packet Transmission Interrupt}

Often a driver will suppress transmission virtqueue interrupts
and check for used packets in the transmit path of following
packets.

The normal behavior in this interrupt handler is to retrieve
used buffers from the virtqueue and free the corresponding
headers and packets.

\subsubsection{Setting Up Receive Buffers}\label{sec:Device Types / Network Device / Device Operation / Setting Up Receive Buffers}

It is generally a good idea to keep the receive virtqueue as
fully populated as possible: if it runs out, network performance
will suffer.

If the VIRTIO_NET_F_GUEST_TSO4, VIRTIO_NET_F_GUEST_TSO6,
VIRTIO_NET_F_GUEST_UFO, VIRTIO_NET_F_GUEST_USO4 or VIRTIO_NET_F_GUEST_USO6
features are used, the maximum incoming packet
will be 65589 bytes long (14 bytes of Ethernet header, plus 40 bytes of
the IPv6 header, plus 65535 bytes of maximum IPv6 payload including any
extension header), otherwise 1514 bytes.
When VIRTIO_NET_F_HASH_REPORT is not negotiated, the required receive buffer
size is either 65601 or 1526 bytes accounting for 20 bytes of
\field{struct virtio_net_hdr} followed by receive packet.
When VIRTIO_NET_F_HASH_REPORT is negotiated, the required receive buffer
size is either 65609 or 1534 bytes accounting for 12 bytes of
\field{struct virtio_net_hdr} followed by receive packet.

\drivernormative{\paragraph}{Setting Up Receive Buffers}{Device Types / Network Device / Device Operation / Setting Up Receive Buffers}

\begin{itemize}
\item If VIRTIO_NET_F_MRG_RXBUF is not negotiated:
  \begin{itemize}
    \item If VIRTIO_NET_F_GUEST_TSO4, VIRTIO_NET_F_GUEST_TSO6, VIRTIO_NET_F_GUEST_UFO,
	VIRTIO_NET_F_GUEST_USO4 or VIRTIO_NET_F_GUEST_USO6 are negotiated, the driver SHOULD populate
      the receive queue(s) with buffers of at least 65609 bytes if
      VIRTIO_NET_F_HASH_REPORT is negotiated, and of at least 65601 bytes if not.
    \item Otherwise, the driver SHOULD populate the receive queue(s)
      with buffers of at least 1534 bytes if VIRTIO_NET_F_HASH_REPORT
      is negotiated, and of at least 1526 bytes if not.
  \end{itemize}
\item If VIRTIO_NET_F_MRG_RXBUF is negotiated, each buffer MUST be at
least size of \field{struct virtio_net_hdr},
i.e. 20 bytes if VIRTIO_NET_F_HASH_REPORT is negotiated, and 12 bytes if not.
\end{itemize}

\begin{note}
Obviously each buffer can be split across multiple descriptor elements.
\end{note}

When calculating the size of \field{struct virtio_net_hdr}, the driver
MUST consider all the fields inclusive up to \field{padding_reserved},
i.e. 20 bytes if VIRTIO_NET_F_HASH_REPORT is negotiated, and 12 bytes if not.

If VIRTIO_NET_F_MQ is negotiated, each of receiveq1\ldots receiveqN
that will be used SHOULD be populated with receive buffers.

\devicenormative{\paragraph}{Setting Up Receive Buffers}{Device Types / Network Device / Device Operation / Setting Up Receive Buffers}

The device MUST set \field{num_buffers} to the number of descriptors used to
hold the incoming packet.

The device MUST use only a single descriptor if VIRTIO_NET_F_MRG_RXBUF
was not negotiated.
\begin{note}
{This means that \field{num_buffers} will always be 1
if VIRTIO_NET_F_MRG_RXBUF is not negotiated.}
\end{note}

\subsubsection{Processing of Incoming Packets}\label{sec:Device Types / Network Device / Device Operation / Processing of Incoming Packets}
\label{sec:Device Types / Network Device / Device Operation / Processing of Packets}%old label for latexdiff

When a packet is copied into a buffer in the receiveq, the
optimal path is to disable further used buffer notifications for the
receiveq and process packets until no more are found, then re-enable
them.

Processing incoming packets involves:

\begin{enumerate}
\item \field{num_buffers} indicates how many descriptors
  this packet is spread over (including this one): this will
  always be 1 if VIRTIO_NET_F_MRG_RXBUF was not negotiated.
  This allows receipt of large packets without having to allocate large
  buffers: a packet that does not fit in a single buffer can flow
  over to the next buffer, and so on. In this case, there will be
  at least \field{num_buffers} used buffers in the virtqueue, and the device
  chains them together to form a single packet in a way similar to
  how it would store it in a single buffer spread over multiple
  descriptors.
  The other buffers will not begin with a \field{struct virtio_net_hdr}.

\item If
  \field{num_buffers} is one, then the entire packet will be
  contained within this buffer, immediately following the struct
  virtio_net_hdr.
\item If the VIRTIO_NET_F_GUEST_CSUM feature was negotiated, the
  VIRTIO_NET_HDR_F_DATA_VALID bit in \field{flags} can be
  set: if so, device has validated the packet checksum.
  If the VIRTIO_NET_F_GUEST_UDP_TUNNEL_GSO_CSUM feature has been negotiated,
  and the VIRTIO_NET_HDR_F_UDP_TUNNEL_CSUM bit is set in \field{flags},
  both the outer UDP checksum and the inner transport checksum
  have been validated, otherwise only one level of checksums (the outer one
  in case of tunnels) has been validated.
\end{enumerate}

Additionally, VIRTIO_NET_F_GUEST_CSUM, TSO4, TSO6, UDP, UDP_TUNNEL
and ECN features enable receive checksum, large receive offload and ECN
support which are the input equivalents of the transmit checksum,
transmit segmentation offloading and ECN features, as described
in \ref{sec:Device Types / Network Device / Device Operation /
Packet Transmission}:
\begin{enumerate}
\item If the VIRTIO_NET_F_GUEST_TSO4, TSO6, UFO, USO4 or USO6 options were
  negotiated, then \field{gso_type} MAY be something other than
  VIRTIO_NET_HDR_GSO_NONE, and \field{gso_size} field indicates the
  desired MSS (see Packet Transmission point 2).
\item If the VIRTIO_NET_F_RSC_EXT option was negotiated (this
  implies one of VIRTIO_NET_F_GUEST_TSO4, TSO6), the
  device processes also duplicated ACK segments, reports
  number of coalesced TCP segments in \field{csum_start} field and
  number of duplicated ACK segments in \field{csum_offset} field
  and sets bit VIRTIO_NET_HDR_F_RSC_INFO in \field{flags}.
\item If the VIRTIO_NET_F_GUEST_CSUM feature was negotiated, the
  VIRTIO_NET_HDR_F_NEEDS_CSUM bit in \field{flags} can be
  set: if so, the packet checksum at offset \field{csum_offset}
  from \field{csum_start} and any preceding checksums
  have been validated.  The checksum on the packet is incomplete and
  if bit VIRTIO_NET_HDR_F_RSC_INFO is not set in \field{flags},
  then \field{csum_start} and \field{csum_offset} indicate how to calculate it
  (see Packet Transmission point 1).
\begin{note}
If \field{gso_type} differs from VIRTIO_NET_HDR_GSO_NONE and the
VIRTIO_NET_HDR_GSO_UDP_TUNNEL_IPV4 bit or the VIRTIO_NET_HDR_GSO_UDP_TUNNEL_IPV6
bit are not set, \field{csum_offset}
points to the only transport header present in the packet, and there are no
additional preceding checksums validated by VIRTIO_NET_HDR_F_NEEDS_CSUM.
\end{note}
\item If the VIRTIO_NET_F_GUEST_UDP_TUNNEL_GSO option was negotiated and
  \field{gso_type} is not VIRTIO_NET_HDR_GSO_NONE, the
  VIRTIO_NET_HDR_GSO_UDP_TUNNEL_IPV4 bit or the VIRTIO_NET_HDR_GSO_UDP_TUNNEL_IPV6
  bit MAY be set. In such case the \field{outer_th_offset} and
  \field{inner_nh_offset} fields indicate the corresponding
  headers information.
\item If the VIRTIO_NET_F_GUEST_UDP_TUNNEL_GSO_CSUM feature was
negotiated, and
  the VIRTIO_NET_HDR_GSO_UDP_TUNNEL_IPV4 bit or the VIRTIO_NET_HDR_GSO_UDP_TUNNEL_IPV6
  are set in \field{gso_type}, the VIRTIO_NET_HDR_F_UDP_TUNNEL_CSUM bit in the
  \field{flags} can be set: if so, the outer UDP checksum has been validated
  and the UDP header checksum at offset 6 from from \field{outer_th_offset}
  is set to the outer UDP pseudo header checksum.

\begin{note}
If the VIRTIO_NET_HDR_GSO_UDP_TUNNEL_IPV4 bit or VIRTIO_NET_HDR_GSO_UDP_TUNNEL_IPV6
bit are set in \field{gso_type}, the \field{csum_start} field refers to
the inner transport header offset (see Packet Transmission point 1).
If the VIRTIO_NET_HDR_F_UDP_TUNNEL_CSUM bit in \field{flags} is set both
the inner and the outer header checksums have been validated by
VIRTIO_NET_HDR_F_NEEDS_CSUM, otherwise only the inner transport header
checksum has been validated.
\end{note}
\end{enumerate}

If applicable, the device calculates per-packet hash for incoming packets as
defined in \ref{sec:Device Types / Network Device / Device Operation / Processing of Incoming Packets / Hash calculation for incoming packets}.

If applicable, the device reports hash information for incoming packets as
defined in \ref{sec:Device Types / Network Device / Device Operation / Processing of Incoming Packets / Hash reporting for incoming packets}.

\devicenormative{\paragraph}{Processing of Incoming Packets}{Device Types / Network Device / Device Operation / Processing of Incoming Packets}
\label{devicenormative:Device Types / Network Device / Device Operation / Processing of Packets}%old label for latexdiff

If VIRTIO_NET_F_MRG_RXBUF has not been negotiated, the device MUST set
\field{num_buffers} to 1.

If VIRTIO_NET_F_MRG_RXBUF has been negotiated, the device MUST set
\field{num_buffers} to indicate the number of buffers
the packet (including the header) is spread over.

If a receive packet is spread over multiple buffers, the device
MUST use all buffers but the last (i.e. the first \field{num_buffers} -
1 buffers) completely up to the full length of each buffer
supplied by the driver.

The device MUST use all buffers used by a single receive
packet together, such that at least \field{num_buffers} are
observed by driver as used.

If VIRTIO_NET_F_GUEST_CSUM is not negotiated, the device MUST set
\field{flags} to zero and SHOULD supply a fully checksummed
packet to the driver.

If VIRTIO_NET_F_GUEST_TSO4 is not negotiated, the device MUST NOT set
\field{gso_type} to VIRTIO_NET_HDR_GSO_TCPV4.

If VIRTIO_NET_F_GUEST_UDP is not negotiated, the device MUST NOT set
\field{gso_type} to VIRTIO_NET_HDR_GSO_UDP.

If VIRTIO_NET_F_GUEST_TSO6 is not negotiated, the device MUST NOT set
\field{gso_type} to VIRTIO_NET_HDR_GSO_TCPV6.

If none of VIRTIO_NET_F_GUEST_USO4 or VIRTIO_NET_F_GUEST_USO6 have been negotiated,
the device MUST NOT set \field{gso_type} to VIRTIO_NET_HDR_GSO_UDP_L4.

If VIRTIO_NET_F_GUEST_UDP_TUNNEL_GSO is not negotiated, the device MUST NOT set
either the VIRTIO_NET_HDR_GSO_UDP_TUNNEL_IPV4 bit or the
VIRTIO_NET_HDR_GSO_UDP_TUNNEL_IPV6 bit in \field{gso_type}.

If VIRTIO_NET_F_GUEST_UDP_TUNNEL_GSO_CSUM is not negotiated the device MUST NOT set
the VIRTIO_NET_HDR_F_UDP_TUNNEL_CSUM bit in \field{flags}.

The device SHOULD NOT send to the driver TCP packets requiring segmentation offload
which have the Explicit Congestion Notification bit set, unless the
VIRTIO_NET_F_GUEST_ECN feature is negotiated, in which case the
device MUST set the VIRTIO_NET_HDR_GSO_ECN bit in
\field{gso_type}.

If the VIRTIO_NET_F_GUEST_CSUM feature has been negotiated, the
device MAY set the VIRTIO_NET_HDR_F_NEEDS_CSUM bit in
\field{flags}, if so:
\begin{enumerate}
\item the device MUST validate the packet checksum at
	offset \field{csum_offset} from \field{csum_start} as well as all
	preceding offsets;
\item the device MUST set the packet checksum stored in the
	receive buffer to the TCP/UDP pseudo header;
\item the device MUST set \field{csum_start} and
	\field{csum_offset} such that calculating a ones'
	complement checksum from \field{csum_start} up until the
	end of the packet and storing the result at offset
	\field{csum_offset} from  \field{csum_start} will result in a
	fully checksummed packet;
\end{enumerate}

The device MUST NOT send to the driver GSO packets encapsulated in UDP
tunnel and requiring segmentation offload, unless the
VIRTIO_NET_F_GUEST_UDP_TUNNEL_GSO is negotiated, in which case the device MUST set
the VIRTIO_NET_HDR_GSO_UDP_TUNNEL_IPV4 bit or the VIRTIO_NET_HDR_GSO_UDP_TUNNEL_IPV6
bit in \field{gso_type} according to the inner network header protocol type,
MUST set the \field{outer_th_offset} and \field{inner_nh_offset} fields
to the corresponding header information, and the outer UDP header MUST NOT
require checksum offload.

If the VIRTIO_NET_F_GUEST_UDP_TUNNEL_GSO_CSUM feature has not been negotiated,
the device MUST NOT send the driver GSO packets encapsulated in UDP
tunnel and requiring segmentation and outer checksum offload.

If none of the VIRTIO_NET_F_GUEST_TSO4, TSO6, UFO, USO4 or USO6 options have
been negotiated, the device MUST set \field{gso_type} to
VIRTIO_NET_HDR_GSO_NONE.

If \field{gso_type} differs from VIRTIO_NET_HDR_GSO_NONE, then
the device MUST also set the VIRTIO_NET_HDR_F_NEEDS_CSUM bit in
\field{flags} MUST set \field{gso_size} to indicate the desired MSS.
If VIRTIO_NET_F_RSC_EXT was negotiated, the device MUST also
set VIRTIO_NET_HDR_F_RSC_INFO bit in \field{flags},
set \field{csum_start} to number of coalesced TCP segments and
set \field{csum_offset} to number of received duplicated ACK segments.

If VIRTIO_NET_F_RSC_EXT was not negotiated, the device MUST
not set VIRTIO_NET_HDR_F_RSC_INFO bit in \field{flags}.

If one of the VIRTIO_NET_F_GUEST_TSO4, TSO6, UFO, USO4 or USO6 options have
been negotiated, the device SHOULD set \field{hdr_len} to a value
not less than the length of the headers, including the transport
header. If \field{gso_type} has the VIRTIO_NET_HDR_GSO_UDP_TUNNEL_IPV4 bit
or the VIRTIO_NET_HDR_GSO_UDP_TUNNEL_IPV6 bit set, the referenced transport
header is the inner one.

If the VIRTIO_NET_F_GUEST_CSUM feature has been negotiated, the
device MAY set the VIRTIO_NET_HDR_F_DATA_VALID bit in
\field{flags}, if so, the device MUST validate the packet
checksum. If the VIRTIO_NET_F_GUEST_UDP_TUNNEL_GSO_CSUM feature has
been negotiated, and the VIRTIO_NET_HDR_F_UDP_TUNNEL_CSUM bit set in
\field{flags}, both the outer UDP checksum and the inner transport
checksum have been validated.
Otherwise level of checksum is validated: in case of multiple
encapsulated protocols the outermost one.

If either the VIRTIO_NET_HDR_GSO_UDP_TUNNEL_IPV4 bit or the
VIRTIO_NET_HDR_GSO_UDP_TUNNEL_IPV6 bit in \field{gso_type} are set,
the device MUST NOT set the VIRTIO_NET_HDR_F_DATA_VALID bit in
\field{flags}.

If the VIRTIO_NET_F_GUEST_UDP_TUNNEL_GSO_CSUM feature has been negotiated
and either the VIRTIO_NET_HDR_GSO_UDP_TUNNEL_IPV4 bit is set or the
VIRTIO_NET_HDR_GSO_UDP_TUNNEL_IPV6 bit is set in \field{gso_type}, the
device MAY set the VIRTIO_NET_HDR_F_UDP_TUNNEL_CSUM bit in
\field{flags}, if so the device MUST set the packet outer UDP checksum
stored in the receive buffer to the outer UDP pseudo header.

Otherwise, the VIRTIO_NET_F_GUEST_UDP_TUNNEL_GSO_CSUM feature has been
negotiated, either the VIRTIO_NET_HDR_GSO_UDP_TUNNEL_IPV4 bit is set or the
VIRTIO_NET_HDR_GSO_UDP_TUNNEL_IPV6 bit is set in \field{gso_type},
and the bit VIRTIO_NET_HDR_F_UDP_TUNNEL_CSUM is not set in
\field{flags}, the device MUST either provide a zero outer UDP header
checksum or a fully checksummed outer UDP header.

\drivernormative{\paragraph}{Processing of Incoming
Packets}{Device Types / Network Device / Device Operation /
Processing of Incoming Packets}

The driver MUST ignore \field{flag} bits that it does not recognize.

If VIRTIO_NET_HDR_F_NEEDS_CSUM bit in \field{flags} is not set or
if VIRTIO_NET_HDR_F_RSC_INFO bit \field{flags} is set, the
driver MUST NOT use the \field{csum_start} and \field{csum_offset}.

If one of the VIRTIO_NET_F_GUEST_TSO4, TSO6, UFO, USO4 or USO6 options have
been negotiated, the driver MAY use \field{hdr_len} only as a hint about the
transport header size.
The driver MUST NOT rely on \field{hdr_len} to be correct.
\begin{note}
This is due to various bugs in implementations.
\end{note}

If neither VIRTIO_NET_HDR_F_NEEDS_CSUM nor
VIRTIO_NET_HDR_F_DATA_VALID is set, the driver MUST NOT
rely on the packet checksum being correct.

If both the VIRTIO_NET_HDR_GSO_UDP_TUNNEL_IPV4 bit and
the VIRTIO_NET_HDR_GSO_UDP_TUNNEL_IPV6 bit in in \field{gso_type} are set,
the driver MUST NOT accept the packet.

If the VIRTIO_NET_HDR_GSO_UDP_TUNNEL_IPV4 bit or the VIRTIO_NET_HDR_GSO_UDP_TUNNEL_IPV6
bit in \field{gso_type} are not set, the driver MUST NOT use the
\field{outer_th_offset} and \field{inner_nh_offset}.

If either the VIRTIO_NET_HDR_GSO_UDP_TUNNEL_IPV4 bit or
the VIRTIO_NET_HDR_GSO_UDP_TUNNEL_IPV6 bit in \field{gso_type} are set, and any of
the following is true:
\begin{itemize}
\item the VIRTIO_NET_HDR_F_NEEDS_CSUM bit is not set in \field{flags}
\item the VIRTIO_NET_HDR_F_DATA_VALID bit is set in \field{flags}
\item the \field{gso_type} excluding the VIRTIO_NET_HDR_GSO_UDP_TUNNEL_IPV4
bit and the VIRTIO_NET_HDR_GSO_UDP_TUNNEL_IPV6 bit is VIRTIO_NET_HDR_GSO_NONE
\end{itemize}
the driver MUST NOT accept the packet.

If the VIRTIO_NET_HDR_F_UDP_TUNNEL_CSUM bit and the VIRTIO_NET_HDR_F_NEEDS_CSUM
bit in \field{flags} are set,
and both the bits VIRTIO_NET_HDR_GSO_UDP_TUNNEL_IPV4 and
VIRTIO_NET_HDR_GSO_UDP_TUNNEL_IPV6 in \field{gso_type} are not set,
the driver MOST NOT accept the packet.

\paragraph{Hash calculation for incoming packets}
\label{sec:Device Types / Network Device / Device Operation / Processing of Incoming Packets / Hash calculation for incoming packets}

A device attempts to calculate a per-packet hash in the following cases:
\begin{itemize}
\item The feature VIRTIO_NET_F_RSS was negotiated. The device uses the hash to determine the receive virtqueue to place incoming packets.
\item The feature VIRTIO_NET_F_HASH_REPORT was negotiated. The device reports the hash value and the hash type with the packet.
\end{itemize}

If the feature VIRTIO_NET_F_RSS was negotiated:
\begin{itemize}
\item The device uses \field{hash_types} of the virtio_net_rss_config structure as 'Enabled hash types' bitmask.
\item If additionally the feature VIRTIO_NET_F_HASH_TUNNEL was negotiated, the device uses \field{enabled_tunnel_types} of the
      virtnet_hash_tunnel structure as 'Encapsulation types enabled for inner header hash' bitmask.
\item The device uses a key as defined in \field{hash_key_data} and \field{hash_key_length} of the virtio_net_rss_config structure (see
\ref{sec:Device Types / Network Device / Device Operation / Control Virtqueue / Receive-side scaling (RSS) / Setting RSS parameters}).
\end{itemize}

If the feature VIRTIO_NET_F_RSS was not negotiated:
\begin{itemize}
\item The device uses \field{hash_types} of the virtio_net_hash_config structure as 'Enabled hash types' bitmask.
\item If additionally the feature VIRTIO_NET_F_HASH_TUNNEL was negotiated, the device uses \field{enabled_tunnel_types} of the
      virtnet_hash_tunnel structure as 'Encapsulation types enabled for inner header hash' bitmask.
\item The device uses a key as defined in \field{hash_key_data} and \field{hash_key_length} of the virtio_net_hash_config structure (see
\ref{sec:Device Types / Network Device / Device Operation / Control Virtqueue / Automatic receive steering in multiqueue mode / Hash calculation}).
\end{itemize}

Note that if the device offers VIRTIO_NET_F_HASH_REPORT, even if it supports only one pair of virtqueues, it MUST support
at least one of commands of VIRTIO_NET_CTRL_MQ class to configure reported hash parameters:
\begin{itemize}
\item If the device offers VIRTIO_NET_F_RSS, it MUST support VIRTIO_NET_CTRL_MQ_RSS_CONFIG command per
 \ref{sec:Device Types / Network Device / Device Operation / Control Virtqueue / Receive-side scaling (RSS) / Setting RSS parameters}.
\item Otherwise the device MUST support VIRTIO_NET_CTRL_MQ_HASH_CONFIG command per
 \ref{sec:Device Types / Network Device / Device Operation / Control Virtqueue / Automatic receive steering in multiqueue mode / Hash calculation}.
\end{itemize}

The per-packet hash calculation can depend on the IP packet type. See
\hyperref[intro:IP]{[IP]}, \hyperref[intro:UDP]{[UDP]} and \hyperref[intro:TCP]{[TCP]}.

\subparagraph{Supported/enabled hash types}
\label{sec:Device Types / Network Device / Device Operation / Processing of Incoming Packets / Hash calculation for incoming packets / Supported/enabled hash types}
Hash types applicable for IPv4 packets:
\begin{lstlisting}
#define VIRTIO_NET_HASH_TYPE_IPv4              (1 << 0)
#define VIRTIO_NET_HASH_TYPE_TCPv4             (1 << 1)
#define VIRTIO_NET_HASH_TYPE_UDPv4             (1 << 2)
\end{lstlisting}
Hash types applicable for IPv6 packets without extension headers
\begin{lstlisting}
#define VIRTIO_NET_HASH_TYPE_IPv6              (1 << 3)
#define VIRTIO_NET_HASH_TYPE_TCPv6             (1 << 4)
#define VIRTIO_NET_HASH_TYPE_UDPv6             (1 << 5)
\end{lstlisting}
Hash types applicable for IPv6 packets with extension headers
\begin{lstlisting}
#define VIRTIO_NET_HASH_TYPE_IP_EX             (1 << 6)
#define VIRTIO_NET_HASH_TYPE_TCP_EX            (1 << 7)
#define VIRTIO_NET_HASH_TYPE_UDP_EX            (1 << 8)
\end{lstlisting}

\subparagraph{IPv4 packets}
\label{sec:Device Types / Network Device / Device Operation / Processing of Incoming Packets / Hash calculation for incoming packets / IPv4 packets}
The device calculates the hash on IPv4 packets according to 'Enabled hash types' bitmask as follows:
\begin{itemize}
\item If VIRTIO_NET_HASH_TYPE_TCPv4 is set and the packet has
a TCP header, the hash is calculated over the following fields:
\begin{itemize}
\item Source IP address
\item Destination IP address
\item Source TCP port
\item Destination TCP port
\end{itemize}
\item Else if VIRTIO_NET_HASH_TYPE_UDPv4 is set and the
packet has a UDP header, the hash is calculated over the following fields:
\begin{itemize}
\item Source IP address
\item Destination IP address
\item Source UDP port
\item Destination UDP port
\end{itemize}
\item Else if VIRTIO_NET_HASH_TYPE_IPv4 is set, the hash is
calculated over the following fields:
\begin{itemize}
\item Source IP address
\item Destination IP address
\end{itemize}
\item Else the device does not calculate the hash
\end{itemize}

\subparagraph{IPv6 packets without extension header}
\label{sec:Device Types / Network Device / Device Operation / Processing of Incoming Packets / Hash calculation for incoming packets / IPv6 packets without extension header}
The device calculates the hash on IPv6 packets without extension
headers according to 'Enabled hash types' bitmask as follows:
\begin{itemize}
\item If VIRTIO_NET_HASH_TYPE_TCPv6 is set and the packet has
a TCPv6 header, the hash is calculated over the following fields:
\begin{itemize}
\item Source IPv6 address
\item Destination IPv6 address
\item Source TCP port
\item Destination TCP port
\end{itemize}
\item Else if VIRTIO_NET_HASH_TYPE_UDPv6 is set and the
packet has a UDPv6 header, the hash is calculated over the following fields:
\begin{itemize}
\item Source IPv6 address
\item Destination IPv6 address
\item Source UDP port
\item Destination UDP port
\end{itemize}
\item Else if VIRTIO_NET_HASH_TYPE_IPv6 is set, the hash is
calculated over the following fields:
\begin{itemize}
\item Source IPv6 address
\item Destination IPv6 address
\end{itemize}
\item Else the device does not calculate the hash
\end{itemize}

\subparagraph{IPv6 packets with extension header}
\label{sec:Device Types / Network Device / Device Operation / Processing of Incoming Packets / Hash calculation for incoming packets / IPv6 packets with extension header}
The device calculates the hash on IPv6 packets with extension
headers according to 'Enabled hash types' bitmask as follows:
\begin{itemize}
\item If VIRTIO_NET_HASH_TYPE_TCP_EX is set and the packet
has a TCPv6 header, the hash is calculated over the following fields:
\begin{itemize}
\item Home address from the home address option in the IPv6 destination options header. If the extension header is not present, use the Source IPv6 address.
\item IPv6 address that is contained in the Routing-Header-Type-2 from the associated extension header. If the extension header is not present, use the Destination IPv6 address.
\item Source TCP port
\item Destination TCP port
\end{itemize}
\item Else if VIRTIO_NET_HASH_TYPE_UDP_EX is set and the
packet has a UDPv6 header, the hash is calculated over the following fields:
\begin{itemize}
\item Home address from the home address option in the IPv6 destination options header. If the extension header is not present, use the Source IPv6 address.
\item IPv6 address that is contained in the Routing-Header-Type-2 from the associated extension header. If the extension header is not present, use the Destination IPv6 address.
\item Source UDP port
\item Destination UDP port
\end{itemize}
\item Else if VIRTIO_NET_HASH_TYPE_IP_EX is set, the hash is
calculated over the following fields:
\begin{itemize}
\item Home address from the home address option in the IPv6 destination options header. If the extension header is not present, use the Source IPv6 address.
\item IPv6 address that is contained in the Routing-Header-Type-2 from the associated extension header. If the extension header is not present, use the Destination IPv6 address.
\end{itemize}
\item Else skip IPv6 extension headers and calculate the hash as
defined for an IPv6 packet without extension headers
(see \ref{sec:Device Types / Network Device / Device Operation / Processing of Incoming Packets / Hash calculation for incoming packets / IPv6 packets without extension header}).
\end{itemize}

\paragraph{Inner Header Hash}
\label{sec:Device Types / Network Device / Device Operation / Processing of Incoming Packets / Inner Header Hash}

If VIRTIO_NET_F_HASH_TUNNEL has been negotiated, the driver can send the command
VIRTIO_NET_CTRL_HASH_TUNNEL_SET to configure the calculation of the inner header hash.

\begin{lstlisting}
struct virtnet_hash_tunnel {
    le32 enabled_tunnel_types;
};

#define VIRTIO_NET_CTRL_HASH_TUNNEL 7
 #define VIRTIO_NET_CTRL_HASH_TUNNEL_SET 0
\end{lstlisting}

Field \field{enabled_tunnel_types} contains the bitmask of encapsulation types enabled for inner header hash.
See \ref{sec:Device Types / Network Device / Device Operation / Processing of Incoming Packets /
Hash calculation for incoming packets / Encapsulation types supported/enabled for inner header hash}.

The class VIRTIO_NET_CTRL_HASH_TUNNEL has one command:
VIRTIO_NET_CTRL_HASH_TUNNEL_SET sets \field{enabled_tunnel_types} for the device using the
virtnet_hash_tunnel structure, which is read-only for the device.

Inner header hash is disabled by VIRTIO_NET_CTRL_HASH_TUNNEL_SET with \field{enabled_tunnel_types} set to 0.

Initially (before the driver sends any VIRTIO_NET_CTRL_HASH_TUNNEL_SET command) all
encapsulation types are disabled for inner header hash.

\subparagraph{Encapsulated packet}
\label{sec:Device Types / Network Device / Device Operation / Processing of Incoming Packets / Hash calculation for incoming packets / Encapsulated packet}

Multiple tunneling protocols allow encapsulating an inner, payload packet in an outer, encapsulated packet.
The encapsulated packet thus contains an outer header and an inner header, and the device calculates the
hash over either the inner header or the outer header.

If VIRTIO_NET_F_HASH_TUNNEL is negotiated and a received encapsulated packet's outer header matches one of the
encapsulation types enabled in \field{enabled_tunnel_types}, then the device uses the inner header for hash
calculations (only a single level of encapsulation is currently supported).

If VIRTIO_NET_F_HASH_TUNNEL is negotiated and a received packet's (outer) header does not match any encapsulation
types enabled in \field{enabled_tunnel_types}, then the device uses the outer header for hash calculations.

\subparagraph{Encapsulation types supported/enabled for inner header hash}
\label{sec:Device Types / Network Device / Device Operation / Processing of Incoming Packets /
Hash calculation for incoming packets / Encapsulation types supported/enabled for inner header hash}

Encapsulation types applicable for inner header hash:
\begin{lstlisting}[escapechar=|]
#define VIRTIO_NET_HASH_TUNNEL_TYPE_GRE_2784    (1 << 0) /* |\hyperref[intro:rfc2784]{[RFC2784]}| */
#define VIRTIO_NET_HASH_TUNNEL_TYPE_GRE_2890    (1 << 1) /* |\hyperref[intro:rfc2890]{[RFC2890]}| */
#define VIRTIO_NET_HASH_TUNNEL_TYPE_GRE_7676    (1 << 2) /* |\hyperref[intro:rfc7676]{[RFC7676]}| */
#define VIRTIO_NET_HASH_TUNNEL_TYPE_GRE_UDP     (1 << 3) /* |\hyperref[intro:rfc8086]{[GRE-in-UDP]}| */
#define VIRTIO_NET_HASH_TUNNEL_TYPE_VXLAN       (1 << 4) /* |\hyperref[intro:vxlan]{[VXLAN]}| */
#define VIRTIO_NET_HASH_TUNNEL_TYPE_VXLAN_GPE   (1 << 5) /* |\hyperref[intro:vxlan-gpe]{[VXLAN-GPE]}| */
#define VIRTIO_NET_HASH_TUNNEL_TYPE_GENEVE      (1 << 6) /* |\hyperref[intro:geneve]{[GENEVE]}| */
#define VIRTIO_NET_HASH_TUNNEL_TYPE_IPIP        (1 << 7) /* |\hyperref[intro:ipip]{[IPIP]}| */
#define VIRTIO_NET_HASH_TUNNEL_TYPE_NVGRE       (1 << 8) /* |\hyperref[intro:nvgre]{[NVGRE]}| */
\end{lstlisting}

\subparagraph{Advice}
Example uses of the inner header hash:
\begin{itemize}
\item Legacy tunneling protocols, lacking the outer header entropy, can use RSS with the inner header hash to
      distribute flows with identical outer but different inner headers across various queues, improving performance.
\item Identify an inner flow distributed across multiple outer tunnels.
\end{itemize}

As using the inner header hash completely discards the outer header entropy, care must be taken
if the inner header is controlled by an adversary, as the adversary can then intentionally create
configurations with insufficient entropy.

Besides disabling the inner header hash, mitigations would depend on how the hash is used. When the hash
use is limited to the RSS queue selection, the inner header hash may have quality of service (QoS) limitations.

\devicenormative{\subparagraph}{Inner Header Hash}{Device Types / Network Device / Device Operation / Control Virtqueue / Inner Header Hash}

If the (outer) header of the received packet does not match any encapsulation types enabled
in \field{enabled_tunnel_types}, the device MUST calculate the hash on the outer header.

If the device receives any bits in \field{enabled_tunnel_types} which are not set in \field{supported_tunnel_types},
it SHOULD respond to the VIRTIO_NET_CTRL_HASH_TUNNEL_SET command with VIRTIO_NET_ERR.

If the driver sets \field{enabled_tunnel_types} to 0 through VIRTIO_NET_CTRL_HASH_TUNNEL_SET or upon the device reset,
the device MUST disable the inner header hash for all encapsulation types.

\drivernormative{\subparagraph}{Inner Header Hash}{Device Types / Network Device / Device Operation / Control Virtqueue / Inner Header Hash}

The driver MUST have negotiated the VIRTIO_NET_F_HASH_TUNNEL feature when issuing the VIRTIO_NET_CTRL_HASH_TUNNEL_SET command.

The driver MUST NOT set any bits in \field{enabled_tunnel_types} which are not set in \field{supported_tunnel_types}.

The driver MUST ignore bits in \field{supported_tunnel_types} which are not documented in this specification.

\paragraph{Hash reporting for incoming packets}
\label{sec:Device Types / Network Device / Device Operation / Processing of Incoming Packets / Hash reporting for incoming packets}

If VIRTIO_NET_F_HASH_REPORT was negotiated and
 the device has calculated the hash for the packet, the device fills \field{hash_report} with the report type of calculated hash
and \field{hash_value} with the value of calculated hash.

If VIRTIO_NET_F_HASH_REPORT was negotiated but due to any reason the
hash was not calculated, the device sets \field{hash_report} to VIRTIO_NET_HASH_REPORT_NONE.

Possible values that the device can report in \field{hash_report} are defined below.
They correspond to supported hash types defined in
\ref{sec:Device Types / Network Device / Device Operation / Processing of Incoming Packets / Hash calculation for incoming packets / Supported/enabled hash types}
as follows:

VIRTIO_NET_HASH_TYPE_XXX = 1 << (VIRTIO_NET_HASH_REPORT_XXX - 1)

\begin{lstlisting}
#define VIRTIO_NET_HASH_REPORT_NONE            0
#define VIRTIO_NET_HASH_REPORT_IPv4            1
#define VIRTIO_NET_HASH_REPORT_TCPv4           2
#define VIRTIO_NET_HASH_REPORT_UDPv4           3
#define VIRTIO_NET_HASH_REPORT_IPv6            4
#define VIRTIO_NET_HASH_REPORT_TCPv6           5
#define VIRTIO_NET_HASH_REPORT_UDPv6           6
#define VIRTIO_NET_HASH_REPORT_IPv6_EX         7
#define VIRTIO_NET_HASH_REPORT_TCPv6_EX        8
#define VIRTIO_NET_HASH_REPORT_UDPv6_EX        9
\end{lstlisting}

\subsubsection{Control Virtqueue}\label{sec:Device Types / Network Device / Device Operation / Control Virtqueue}

The driver uses the control virtqueue (if VIRTIO_NET_F_CTRL_VQ is
negotiated) to send commands to manipulate various features of
the device which would not easily map into the configuration
space.

All commands are of the following form:

\begin{lstlisting}
struct virtio_net_ctrl {
        u8 class;
        u8 command;
        u8 command-specific-data[];
        u8 ack;
        u8 command-specific-result[];
};

/* ack values */
#define VIRTIO_NET_OK     0
#define VIRTIO_NET_ERR    1
\end{lstlisting}

The \field{class}, \field{command} and command-specific-data are set by the
driver, and the device sets the \field{ack} byte and optionally
\field{command-specific-result}. There is little the driver can
do except issue a diagnostic if \field{ack} is not VIRTIO_NET_OK.

The command VIRTIO_NET_CTRL_STATS_QUERY and VIRTIO_NET_CTRL_STATS_GET contain
\field{command-specific-result}.

\paragraph{Packet Receive Filtering}\label{sec:Device Types / Network Device / Device Operation / Control Virtqueue / Packet Receive Filtering}
\label{sec:Device Types / Network Device / Device Operation / Control Virtqueue / Setting Promiscuous Mode}%old label for latexdiff

If the VIRTIO_NET_F_CTRL_RX and VIRTIO_NET_F_CTRL_RX_EXTRA
features are negotiated, the driver can send control commands for
promiscuous mode, multicast, unicast and broadcast receiving.

\begin{note}
In general, these commands are best-effort: unwanted
packets could still arrive.
\end{note}

\begin{lstlisting}
#define VIRTIO_NET_CTRL_RX    0
 #define VIRTIO_NET_CTRL_RX_PROMISC      0
 #define VIRTIO_NET_CTRL_RX_ALLMULTI     1
 #define VIRTIO_NET_CTRL_RX_ALLUNI       2
 #define VIRTIO_NET_CTRL_RX_NOMULTI      3
 #define VIRTIO_NET_CTRL_RX_NOUNI        4
 #define VIRTIO_NET_CTRL_RX_NOBCAST      5
\end{lstlisting}


\devicenormative{\subparagraph}{Packet Receive Filtering}{Device Types / Network Device / Device Operation / Control Virtqueue / Packet Receive Filtering}

If the VIRTIO_NET_F_CTRL_RX feature has been negotiated,
the device MUST support the following VIRTIO_NET_CTRL_RX class
commands:
\begin{itemize}
\item VIRTIO_NET_CTRL_RX_PROMISC turns promiscuous mode on and
off. The command-specific-data is one byte containing 0 (off) or
1 (on). If promiscuous mode is on, the device SHOULD receive all
incoming packets.
This SHOULD take effect even if one of the other modes set by
a VIRTIO_NET_CTRL_RX class command is on.
\item VIRTIO_NET_CTRL_RX_ALLMULTI turns all-multicast receive on and
off. The command-specific-data is one byte containing 0 (off) or
1 (on). When all-multicast receive is on the device SHOULD allow
all incoming multicast packets.
\end{itemize}

If the VIRTIO_NET_F_CTRL_RX_EXTRA feature has been negotiated,
the device MUST support the following VIRTIO_NET_CTRL_RX class
commands:
\begin{itemize}
\item VIRTIO_NET_CTRL_RX_ALLUNI turns all-unicast receive on and
off. The command-specific-data is one byte containing 0 (off) or
1 (on). When all-unicast receive is on the device SHOULD allow
all incoming unicast packets.
\item VIRTIO_NET_CTRL_RX_NOMULTI suppresses multicast receive.
The command-specific-data is one byte containing 0 (multicast
receive allowed) or 1 (multicast receive suppressed).
When multicast receive is suppressed, the device SHOULD NOT
send multicast packets to the driver.
This SHOULD take effect even if VIRTIO_NET_CTRL_RX_ALLMULTI is on.
This filter SHOULD NOT apply to broadcast packets.
\item VIRTIO_NET_CTRL_RX_NOUNI suppresses unicast receive.
The command-specific-data is one byte containing 0 (unicast
receive allowed) or 1 (unicast receive suppressed).
When unicast receive is suppressed, the device SHOULD NOT
send unicast packets to the driver.
This SHOULD take effect even if VIRTIO_NET_CTRL_RX_ALLUNI is on.
\item VIRTIO_NET_CTRL_RX_NOBCAST suppresses broadcast receive.
The command-specific-data is one byte containing 0 (broadcast
receive allowed) or 1 (broadcast receive suppressed).
When broadcast receive is suppressed, the device SHOULD NOT
send broadcast packets to the driver.
This SHOULD take effect even if VIRTIO_NET_CTRL_RX_ALLMULTI is on.
\end{itemize}

\drivernormative{\subparagraph}{Packet Receive Filtering}{Device Types / Network Device / Device Operation / Control Virtqueue / Packet Receive Filtering}

If the VIRTIO_NET_F_CTRL_RX feature has not been negotiated,
the driver MUST NOT issue commands VIRTIO_NET_CTRL_RX_PROMISC or
VIRTIO_NET_CTRL_RX_ALLMULTI.

If the VIRTIO_NET_F_CTRL_RX_EXTRA feature has not been negotiated,
the driver MUST NOT issue commands
 VIRTIO_NET_CTRL_RX_ALLUNI,
 VIRTIO_NET_CTRL_RX_NOMULTI,
 VIRTIO_NET_CTRL_RX_NOUNI or
 VIRTIO_NET_CTRL_RX_NOBCAST.

\paragraph{Setting MAC Address Filtering}\label{sec:Device Types / Network Device / Device Operation / Control Virtqueue / Setting MAC Address Filtering}

If the VIRTIO_NET_F_CTRL_RX feature is negotiated, the driver can
send control commands for MAC address filtering.

\begin{lstlisting}
struct virtio_net_ctrl_mac {
        le32 entries;
        u8 macs[entries][6];
};

#define VIRTIO_NET_CTRL_MAC    1
 #define VIRTIO_NET_CTRL_MAC_TABLE_SET        0
 #define VIRTIO_NET_CTRL_MAC_ADDR_SET         1
\end{lstlisting}

The device can filter incoming packets by any number of destination
MAC addresses\footnote{Since there are no guarantees, it can use a hash filter or
silently switch to allmulti or promiscuous mode if it is given too
many addresses.
}. This table is set using the class
VIRTIO_NET_CTRL_MAC and the command VIRTIO_NET_CTRL_MAC_TABLE_SET. The
command-specific-data is two variable length tables of 6-byte MAC
addresses (as described in struct virtio_net_ctrl_mac). The first table contains unicast addresses, and the second
contains multicast addresses.

The VIRTIO_NET_CTRL_MAC_ADDR_SET command is used to set the
default MAC address which rx filtering
accepts (and if VIRTIO_NET_F_MAC has been negotiated,
this will be reflected in \field{mac} in config space).

The command-specific-data for VIRTIO_NET_CTRL_MAC_ADDR_SET is
the 6-byte MAC address.

\devicenormative{\subparagraph}{Setting MAC Address Filtering}{Device Types / Network Device / Device Operation / Control Virtqueue / Setting MAC Address Filtering}

The device MUST have an empty MAC filtering table on reset.

The device MUST update the MAC filtering table before it consumes
the VIRTIO_NET_CTRL_MAC_TABLE_SET command.

The device MUST update \field{mac} in config space before it consumes
the VIRTIO_NET_CTRL_MAC_ADDR_SET command, if VIRTIO_NET_F_MAC has
been negotiated.

The device SHOULD drop incoming packets which have a destination MAC which
matches neither the \field{mac} (or that set with VIRTIO_NET_CTRL_MAC_ADDR_SET)
nor the MAC filtering table.

\drivernormative{\subparagraph}{Setting MAC Address Filtering}{Device Types / Network Device / Device Operation / Control Virtqueue / Setting MAC Address Filtering}

If VIRTIO_NET_F_CTRL_RX has not been negotiated,
the driver MUST NOT issue VIRTIO_NET_CTRL_MAC class commands.

If VIRTIO_NET_F_CTRL_RX has been negotiated,
the driver SHOULD issue VIRTIO_NET_CTRL_MAC_ADDR_SET
to set the default mac if it is different from \field{mac}.

The driver MUST follow the VIRTIO_NET_CTRL_MAC_TABLE_SET command
by a le32 number, followed by that number of non-multicast
MAC addresses, followed by another le32 number, followed by
that number of multicast addresses.  Either number MAY be 0.

\subparagraph{Legacy Interface: Setting MAC Address Filtering}\label{sec:Device Types / Network Device / Device Operation / Control Virtqueue / Setting MAC Address Filtering / Legacy Interface: Setting MAC Address Filtering}
When using the legacy interface, transitional devices and drivers
MUST format \field{entries} in struct virtio_net_ctrl_mac
according to the native endian of the guest rather than
(necessarily when not using the legacy interface) little-endian.

Legacy drivers that didn't negotiate VIRTIO_NET_F_CTRL_MAC_ADDR
changed \field{mac} in config space when NIC is accepting
incoming packets. These drivers always wrote the mac value from
first to last byte, therefore after detecting such drivers,
a transitional device MAY defer MAC update, or MAY defer
processing incoming packets until driver writes the last byte
of \field{mac} in the config space.

\paragraph{VLAN Filtering}\label{sec:Device Types / Network Device / Device Operation / Control Virtqueue / VLAN Filtering}

If the driver negotiates the VIRTIO_NET_F_CTRL_VLAN feature, it
can control a VLAN filter table in the device. The VLAN filter
table applies only to VLAN tagged packets.

When VIRTIO_NET_F_CTRL_VLAN is negotiated, the device starts with
an empty VLAN filter table.

\begin{note}
Similar to the MAC address based filtering, the VLAN filtering
is also best-effort: unwanted packets could still arrive.
\end{note}

\begin{lstlisting}
#define VIRTIO_NET_CTRL_VLAN       2
 #define VIRTIO_NET_CTRL_VLAN_ADD             0
 #define VIRTIO_NET_CTRL_VLAN_DEL             1
\end{lstlisting}

Both the VIRTIO_NET_CTRL_VLAN_ADD and VIRTIO_NET_CTRL_VLAN_DEL
command take a little-endian 16-bit VLAN id as the command-specific-data.

VIRTIO_NET_CTRL_VLAN_ADD command adds the specified VLAN to the
VLAN filter table.

VIRTIO_NET_CTRL_VLAN_DEL command removes the specified VLAN from
the VLAN filter table.

\devicenormative{\subparagraph}{VLAN Filtering}{Device Types / Network Device / Device Operation / Control Virtqueue / VLAN Filtering}

When VIRTIO_NET_F_CTRL_VLAN is not negotiated, the device MUST
accept all VLAN tagged packets.

When VIRTIO_NET_F_CTRL_VLAN is negotiated, the device MUST
accept all VLAN tagged packets whose VLAN tag is present in
the VLAN filter table and SHOULD drop all VLAN tagged packets
whose VLAN tag is absent in the VLAN filter table.

\subparagraph{Legacy Interface: VLAN Filtering}\label{sec:Device Types / Network Device / Device Operation / Control Virtqueue / VLAN Filtering / Legacy Interface: VLAN Filtering}
When using the legacy interface, transitional devices and drivers
MUST format the VLAN id
according to the native endian of the guest rather than
(necessarily when not using the legacy interface) little-endian.

\paragraph{Gratuitous Packet Sending}\label{sec:Device Types / Network Device / Device Operation / Control Virtqueue / Gratuitous Packet Sending}

If the driver negotiates the VIRTIO_NET_F_GUEST_ANNOUNCE (depends
on VIRTIO_NET_F_CTRL_VQ), the device can ask the driver to send gratuitous
packets; this is usually done after the guest has been physically
migrated, and needs to announce its presence on the new network
links. (As hypervisor does not have the knowledge of guest
network configuration (eg. tagged vlan) it is simplest to prod
the guest in this way).

\begin{lstlisting}
#define VIRTIO_NET_CTRL_ANNOUNCE       3
 #define VIRTIO_NET_CTRL_ANNOUNCE_ACK             0
\end{lstlisting}

The driver checks VIRTIO_NET_S_ANNOUNCE bit in the device configuration \field{status} field
when it notices the changes of device configuration. The
command VIRTIO_NET_CTRL_ANNOUNCE_ACK is used to indicate that
driver has received the notification and device clears the
VIRTIO_NET_S_ANNOUNCE bit in \field{status}.

Processing this notification involves:

\begin{enumerate}
\item Sending the gratuitous packets (eg. ARP) or marking there are pending
  gratuitous packets to be sent and letting deferred routine to
  send them.

\item Sending VIRTIO_NET_CTRL_ANNOUNCE_ACK command through control
  vq.
\end{enumerate}

\drivernormative{\subparagraph}{Gratuitous Packet Sending}{Device Types / Network Device / Device Operation / Control Virtqueue / Gratuitous Packet Sending}

If the driver negotiates VIRTIO_NET_F_GUEST_ANNOUNCE, it SHOULD notify
network peers of its new location after it sees the VIRTIO_NET_S_ANNOUNCE bit
in \field{status}.  The driver MUST send a command on the command queue
with class VIRTIO_NET_CTRL_ANNOUNCE and command VIRTIO_NET_CTRL_ANNOUNCE_ACK.

\devicenormative{\subparagraph}{Gratuitous Packet Sending}{Device Types / Network Device / Device Operation / Control Virtqueue / Gratuitous Packet Sending}

If VIRTIO_NET_F_GUEST_ANNOUNCE is negotiated, the device MUST clear the
VIRTIO_NET_S_ANNOUNCE bit in \field{status} upon receipt of a command buffer
with class VIRTIO_NET_CTRL_ANNOUNCE and command VIRTIO_NET_CTRL_ANNOUNCE_ACK
before marking the buffer as used.

\paragraph{Device operation in multiqueue mode}\label{sec:Device Types / Network Device / Device Operation / Control Virtqueue / Device operation in multiqueue mode}

This specification defines the following modes that a device MAY implement for operation with multiple transmit/receive virtqueues:
\begin{itemize}
\item Automatic receive steering as defined in \ref{sec:Device Types / Network Device / Device Operation / Control Virtqueue / Automatic receive steering in multiqueue mode}.
 If a device supports this mode, it offers the VIRTIO_NET_F_MQ feature bit.
\item Receive-side scaling as defined in \ref{devicenormative:Device Types / Network Device / Device Operation / Control Virtqueue / Receive-side scaling (RSS) / RSS processing}.
 If a device supports this mode, it offers the VIRTIO_NET_F_RSS feature bit.
\end{itemize}

A device MAY support one of these features or both. The driver MAY negotiate any set of these features that the device supports.

Multiqueue is disabled by default.

The driver enables multiqueue by sending a command using \field{class} VIRTIO_NET_CTRL_MQ. The \field{command} selects the mode of multiqueue operation, as follows:
\begin{lstlisting}
#define VIRTIO_NET_CTRL_MQ    4
 #define VIRTIO_NET_CTRL_MQ_VQ_PAIRS_SET        0 (for automatic receive steering)
 #define VIRTIO_NET_CTRL_MQ_RSS_CONFIG          1 (for configurable receive steering)
 #define VIRTIO_NET_CTRL_MQ_HASH_CONFIG         2 (for configurable hash calculation)
\end{lstlisting}

If more than one multiqueue mode is negotiated, the resulting device configuration is defined by the last command sent by the driver.

\paragraph{Automatic receive steering in multiqueue mode}\label{sec:Device Types / Network Device / Device Operation / Control Virtqueue / Automatic receive steering in multiqueue mode}

If the driver negotiates the VIRTIO_NET_F_MQ feature bit (depends on VIRTIO_NET_F_CTRL_VQ), it MAY transmit outgoing packets on one
of the multiple transmitq1\ldots transmitqN and ask the device to
queue incoming packets into one of the multiple receiveq1\ldots receiveqN
depending on the packet flow.

The driver enables multiqueue by
sending the VIRTIO_NET_CTRL_MQ_VQ_PAIRS_SET command, specifying
the number of the transmit and receive queues to be used up to
\field{max_virtqueue_pairs}; subsequently,
transmitq1\ldots transmitqn and receiveq1\ldots receiveqn where
n=\field{virtqueue_pairs} MAY be used.
\begin{lstlisting}
struct virtio_net_ctrl_mq_pairs_set {
       le16 virtqueue_pairs;
};
#define VIRTIO_NET_CTRL_MQ_VQ_PAIRS_MIN        1
#define VIRTIO_NET_CTRL_MQ_VQ_PAIRS_MAX        0x8000

\end{lstlisting}

When multiqueue is enabled by VIRTIO_NET_CTRL_MQ_VQ_PAIRS_SET command, the device MUST use automatic receive steering
based on packet flow. Programming of the receive steering
classificator is implicit. After the driver transmitted a packet of a
flow on transmitqX, the device SHOULD cause incoming packets for that flow to
be steered to receiveqX. For uni-directional protocols, or where
no packets have been transmitted yet, the device MAY steer a packet
to a random queue out of the specified receiveq1\ldots receiveqn.

Multiqueue is disabled by VIRTIO_NET_CTRL_MQ_VQ_PAIRS_SET with \field{virtqueue_pairs} to 1 (this is
the default) and waiting for the device to use the command buffer.

\drivernormative{\subparagraph}{Automatic receive steering in multiqueue mode}{Device Types / Network Device / Device Operation / Control Virtqueue / Automatic receive steering in multiqueue mode}

The driver MUST configure the virtqueues before enabling them with the
VIRTIO_NET_CTRL_MQ_VQ_PAIRS_SET command.

The driver MUST NOT request a \field{virtqueue_pairs} of 0 or
greater than \field{max_virtqueue_pairs} in the device configuration space.

The driver MUST queue packets only on any transmitq1 before the
VIRTIO_NET_CTRL_MQ_VQ_PAIRS_SET command.

The driver MUST NOT queue packets on transmit queues greater than
\field{virtqueue_pairs} once it has placed the VIRTIO_NET_CTRL_MQ_VQ_PAIRS_SET command in the available ring.

\devicenormative{\subparagraph}{Automatic receive steering in multiqueue mode}{Device Types / Network Device / Device Operation / Control Virtqueue / Automatic receive steering in multiqueue mode}

After initialization of reset, the device MUST queue packets only on receiveq1.

The device MUST NOT queue packets on receive queues greater than
\field{virtqueue_pairs} once it has placed the
VIRTIO_NET_CTRL_MQ_VQ_PAIRS_SET command in a used buffer.

If the destination receive queue is being reset (See \ref{sec:Basic Facilities of a Virtio Device / Virtqueues / Virtqueue Reset}),
the device SHOULD re-select another random queue. If all receive queues are
being reset, the device MUST drop the packet.

\subparagraph{Legacy Interface: Automatic receive steering in multiqueue mode}\label{sec:Device Types / Network Device / Device Operation / Control Virtqueue / Automatic receive steering in multiqueue mode / Legacy Interface: Automatic receive steering in multiqueue mode}
When using the legacy interface, transitional devices and drivers
MUST format \field{virtqueue_pairs}
according to the native endian of the guest rather than
(necessarily when not using the legacy interface) little-endian.

\subparagraph{Hash calculation}\label{sec:Device Types / Network Device / Device Operation / Control Virtqueue / Automatic receive steering in multiqueue mode / Hash calculation}
If VIRTIO_NET_F_HASH_REPORT was negotiated and the device uses automatic receive steering,
the device MUST support a command to configure hash calculation parameters.

The driver provides parameters for hash calculation as follows:

\field{class} VIRTIO_NET_CTRL_MQ, \field{command} VIRTIO_NET_CTRL_MQ_HASH_CONFIG.

The \field{command-specific-data} has following format:
\begin{lstlisting}
struct virtio_net_hash_config {
    le32 hash_types;
    le16 reserved[4];
    u8 hash_key_length;
    u8 hash_key_data[hash_key_length];
};
\end{lstlisting}
Field \field{hash_types} contains a bitmask of allowed hash types as
defined in
\ref{sec:Device Types / Network Device / Device Operation / Processing of Incoming Packets / Hash calculation for incoming packets / Supported/enabled hash types}.
Initially the device has all hash types disabled and reports only VIRTIO_NET_HASH_REPORT_NONE.

Field \field{reserved} MUST contain zeroes. It is defined to make the structure to match the layout of virtio_net_rss_config structure,
defined in \ref{sec:Device Types / Network Device / Device Operation / Control Virtqueue / Receive-side scaling (RSS)}.

Fields \field{hash_key_length} and \field{hash_key_data} define the key to be used in hash calculation.

\paragraph{Receive-side scaling (RSS)}\label{sec:Device Types / Network Device / Device Operation / Control Virtqueue / Receive-side scaling (RSS)}
A device offers the feature VIRTIO_NET_F_RSS if it supports RSS receive steering with Toeplitz hash calculation and configurable parameters.

A driver queries RSS capabilities of the device by reading device configuration as defined in \ref{sec:Device Types / Network Device / Device configuration layout}

\subparagraph{Setting RSS parameters}\label{sec:Device Types / Network Device / Device Operation / Control Virtqueue / Receive-side scaling (RSS) / Setting RSS parameters}

Driver sends a VIRTIO_NET_CTRL_MQ_RSS_CONFIG command using the following format for \field{command-specific-data}:
\begin{lstlisting}
struct rss_rq_id {
   le16 vq_index_1_16: 15; /* Bits 1 to 16 of the virtqueue index */
   le16 reserved: 1; /* Set to zero */
};

struct virtio_net_rss_config {
    le32 hash_types;
    le16 indirection_table_mask;
    struct rss_rq_id unclassified_queue;
    struct rss_rq_id indirection_table[indirection_table_length];
    le16 max_tx_vq;
    u8 hash_key_length;
    u8 hash_key_data[hash_key_length];
};
\end{lstlisting}
Field \field{hash_types} contains a bitmask of allowed hash types as
defined in
\ref{sec:Device Types / Network Device / Device Operation / Processing of Incoming Packets / Hash calculation for incoming packets / Supported/enabled hash types}.

Field \field{indirection_table_mask} is a mask to be applied to
the calculated hash to produce an index in the
\field{indirection_table} array.
Number of entries in \field{indirection_table} is (\field{indirection_table_mask} + 1).

\field{rss_rq_id} is a receive virtqueue id. \field{vq_index_1_16}
consists of bits 1 to 16 of a virtqueue index. For example, a
\field{vq_index_1_16} value of 3 corresponds to virtqueue index 6,
which maps to receiveq4.

Field \field{unclassified_queue} specifies the receive virtqueue id in which to
place unclassified packets.

Field \field{indirection_table} is an array of receive virtqueues ids.

A driver sets \field{max_tx_vq} to inform a device how many transmit virtqueues it may use (transmitq1\ldots transmitq \field{max_tx_vq}).

Fields \field{hash_key_length} and \field{hash_key_data} define the key to be used in hash calculation.

\drivernormative{\subparagraph}{Setting RSS parameters}{Device Types / Network Device / Device Operation / Control Virtqueue / Receive-side scaling (RSS) }

A driver MUST NOT send the VIRTIO_NET_CTRL_MQ_RSS_CONFIG command if the feature VIRTIO_NET_F_RSS has not been negotiated.

A driver MUST fill the \field{indirection_table} array only with
enabled receive virtqueues ids.

The number of entries in \field{indirection_table} (\field{indirection_table_mask} + 1) MUST be a power of two.

A driver MUST use \field{indirection_table_mask} values that are less than \field{rss_max_indirection_table_length} reported by a device.

A driver MUST NOT set any VIRTIO_NET_HASH_TYPE_ flags that are not supported by a device.

\devicenormative{\subparagraph}{RSS processing}{Device Types / Network Device / Device Operation / Control Virtqueue / Receive-side scaling (RSS) / RSS processing}
The device MUST determine the destination queue for a network packet as follows:
\begin{itemize}
\item Calculate the hash of the packet as defined in \ref{sec:Device Types / Network Device / Device Operation / Processing of Incoming Packets / Hash calculation for incoming packets}.
\item If the device did not calculate the hash for the specific packet, the device directs the packet to the receiveq specified by \field{unclassified_queue} of virtio_net_rss_config structure.
\item Apply \field{indirection_table_mask} to the calculated hash
and use the result as the index in the indirection table to get
the destination receive virtqueue id.
\item If the destination receive queue is being reset (See \ref{sec:Basic Facilities of a Virtio Device / Virtqueues / Virtqueue Reset}), the device MUST drop the packet.
\end{itemize}

\paragraph{RSS Context}\label{sec:Device Types / Network Device / Device Operation / Control Virtqueue / RSS Context}

An RSS context consists of configurable parameters specified by \ref{sec:Device Types / Network Device
/ Device Operation / Control Virtqueue / Receive-side scaling (RSS)}.

The RSS configuration supported by VIRTIO_NET_F_RSS is considered the default RSS configuration.

The device offers the feature VIRTIO_NET_F_RSS_CONTEXT if it supports one or multiple RSS contexts
(excluding the default RSS configuration) and configurable parameters.

\subparagraph{Querying RSS Context Capability}\label{sec:Device Types / Network Device / Device Operation / Control Virtqueue / RSS Context / Querying RSS Context Capability}

\begin{lstlisting}
#define VIRTNET_RSS_CTX_CTRL 9
 #define VIRTNET_RSS_CTX_CTRL_CAP_GET  0
 #define VIRTNET_RSS_CTX_CTRL_ADD      1
 #define VIRTNET_RSS_CTX_CTRL_MOD      2
 #define VIRTNET_RSS_CTX_CTRL_DEL      3

struct virtnet_rss_ctx_cap {
    le16 max_rss_contexts;
}
\end{lstlisting}

Field \field{max_rss_contexts} specifies the maximum number of RSS contexts \ref{sec:Device Types / Network Device /
Device Operation / Control Virtqueue / RSS Context} supported by the device.

The driver queries the RSS context capability of the device by sending the command VIRTNET_RSS_CTX_CTRL_CAP_GET
with the structure virtnet_rss_ctx_cap.

For the command VIRTNET_RSS_CTX_CTRL_CAP_GET, the structure virtnet_rss_ctx_cap is write-only for the device.

\subparagraph{Setting RSS Context Parameters}\label{sec:Device Types / Network Device / Device Operation / Control Virtqueue / RSS Context / Setting RSS Context Parameters}

\begin{lstlisting}
struct virtnet_rss_ctx_add_modify {
    le16 rss_ctx_id;
    u8 reserved[6];
    struct virtio_net_rss_config rss;
};

struct virtnet_rss_ctx_del {
    le16 rss_ctx_id;
};
\end{lstlisting}

RSS context parameters:
\begin{itemize}
\item  \field{rss_ctx_id}: ID of the specific RSS context.
\item  \field{rss}: RSS context parameters of the specific RSS context whose id is \field{rss_ctx_id}.
\end{itemize}

\field{reserved} is reserved and it is ignored by the device.

If the feature VIRTIO_NET_F_RSS_CONTEXT has been negotiated, the driver can send the following
VIRTNET_RSS_CTX_CTRL class commands:
\begin{enumerate}
\item VIRTNET_RSS_CTX_CTRL_ADD: use the structure virtnet_rss_ctx_add_modify to
       add an RSS context configured as \field{rss} and id as \field{rss_ctx_id} for the device.
\item VIRTNET_RSS_CTX_CTRL_MOD: use the structure virtnet_rss_ctx_add_modify to
       configure parameters of the RSS context whose id is \field{rss_ctx_id} as \field{rss} for the device.
\item VIRTNET_RSS_CTX_CTRL_DEL: use the structure virtnet_rss_ctx_del to delete
       the RSS context whose id is \field{rss_ctx_id} for the device.
\end{enumerate}

For commands VIRTNET_RSS_CTX_CTRL_ADD and VIRTNET_RSS_CTX_CTRL_MOD, the structure virtnet_rss_ctx_add_modify is read-only for the device.
For the command VIRTNET_RSS_CTX_CTRL_DEL, the structure virtnet_rss_ctx_del is read-only for the device.

\devicenormative{\subparagraph}{RSS Context}{Device Types / Network Device / Device Operation / Control Virtqueue / RSS Context}

The device MUST set \field{max_rss_contexts} to at least 1 if it offers VIRTIO_NET_F_RSS_CONTEXT.

Upon reset, the device MUST clear all previously configured RSS contexts.

\drivernormative{\subparagraph}{RSS Context}{Device Types / Network Device / Device Operation / Control Virtqueue / RSS Context}

The driver MUST have negotiated the VIRTIO_NET_F_RSS_CONTEXT feature when issuing the VIRTNET_RSS_CTX_CTRL class commands.

The driver MUST set \field{rss_ctx_id} to between 1 and \field{max_rss_contexts} inclusive.

The driver MUST NOT send the command VIRTIO_NET_CTRL_MQ_VQ_PAIRS_SET when the device has successfully configured at least one RSS context.

\paragraph{Offloads State Configuration}\label{sec:Device Types / Network Device / Device Operation / Control Virtqueue / Offloads State Configuration}

If the VIRTIO_NET_F_CTRL_GUEST_OFFLOADS feature is negotiated, the driver can
send control commands for dynamic offloads state configuration.

\subparagraph{Setting Offloads State}\label{sec:Device Types / Network Device / Device Operation / Control Virtqueue / Offloads State Configuration / Setting Offloads State}

To configure the offloads, the following layout structure and
definitions are used:

\begin{lstlisting}
le64 offloads;

#define VIRTIO_NET_F_GUEST_CSUM       1
#define VIRTIO_NET_F_GUEST_TSO4       7
#define VIRTIO_NET_F_GUEST_TSO6       8
#define VIRTIO_NET_F_GUEST_ECN        9
#define VIRTIO_NET_F_GUEST_UFO        10
#define VIRTIO_NET_F_GUEST_UDP_TUNNEL_GSO  46
#define VIRTIO_NET_F_GUEST_UDP_TUNNEL_GSO_CSUM 47
#define VIRTIO_NET_F_GUEST_USO4       54
#define VIRTIO_NET_F_GUEST_USO6       55

#define VIRTIO_NET_CTRL_GUEST_OFFLOADS       5
 #define VIRTIO_NET_CTRL_GUEST_OFFLOADS_SET   0
\end{lstlisting}

The class VIRTIO_NET_CTRL_GUEST_OFFLOADS has one command:
VIRTIO_NET_CTRL_GUEST_OFFLOADS_SET applies the new offloads configuration.

le64 value passed as command data is a bitmask, bits set define
offloads to be enabled, bits cleared - offloads to be disabled.

There is a corresponding device feature for each offload. Upon feature
negotiation corresponding offload gets enabled to preserve backward
compatibility.

\drivernormative{\subparagraph}{Setting Offloads State}{Device Types / Network Device / Device Operation / Control Virtqueue / Offloads State Configuration / Setting Offloads State}

A driver MUST NOT enable an offload for which the appropriate feature
has not been negotiated.

\subparagraph{Legacy Interface: Setting Offloads State}\label{sec:Device Types / Network Device / Device Operation / Control Virtqueue / Offloads State Configuration / Setting Offloads State / Legacy Interface: Setting Offloads State}
When using the legacy interface, transitional devices and drivers
MUST format \field{offloads}
according to the native endian of the guest rather than
(necessarily when not using the legacy interface) little-endian.


\paragraph{Notifications Coalescing}\label{sec:Device Types / Network Device / Device Operation / Control Virtqueue / Notifications Coalescing}

If the VIRTIO_NET_F_NOTF_COAL feature is negotiated, the driver can
send commands VIRTIO_NET_CTRL_NOTF_COAL_TX_SET and VIRTIO_NET_CTRL_NOTF_COAL_RX_SET
for notification coalescing.

If the VIRTIO_NET_F_VQ_NOTF_COAL feature is negotiated, the driver can
send commands VIRTIO_NET_CTRL_NOTF_COAL_VQ_SET and VIRTIO_NET_CTRL_NOTF_COAL_VQ_GET
for virtqueue notification coalescing.

\begin{lstlisting}
struct virtio_net_ctrl_coal {
    le32 max_packets;
    le32 max_usecs;
};

struct virtio_net_ctrl_coal_vq {
    le16 vq_index;
    le16 reserved;
    struct virtio_net_ctrl_coal coal;
};

#define VIRTIO_NET_CTRL_NOTF_COAL 6
 #define VIRTIO_NET_CTRL_NOTF_COAL_TX_SET  0
 #define VIRTIO_NET_CTRL_NOTF_COAL_RX_SET 1
 #define VIRTIO_NET_CTRL_NOTF_COAL_VQ_SET 2
 #define VIRTIO_NET_CTRL_NOTF_COAL_VQ_GET 3
\end{lstlisting}

Coalescing parameters:
\begin{itemize}
\item \field{vq_index}: The virtqueue index of an enabled transmit or receive virtqueue.
\item \field{max_usecs} for RX: Maximum number of microseconds to delay a RX notification.
\item \field{max_usecs} for TX: Maximum number of microseconds to delay a TX notification.
\item \field{max_packets} for RX: Maximum number of packets to receive before a RX notification.
\item \field{max_packets} for TX: Maximum number of packets to send before a TX notification.
\end{itemize}

\field{reserved} is reserved and it is ignored by the device.

Read/Write attributes for coalescing parameters:
\begin{itemize}
\item For commands VIRTIO_NET_CTRL_NOTF_COAL_TX_SET and VIRTIO_NET_CTRL_NOTF_COAL_RX_SET, the structure virtio_net_ctrl_coal is write-only for the driver.
\item For the command VIRTIO_NET_CTRL_NOTF_COAL_VQ_SET, the structure virtio_net_ctrl_coal_vq is write-only for the driver.
\item For the command VIRTIO_NET_CTRL_NOTF_COAL_VQ_GET, \field{vq_index} and \field{reserved} are write-only
      for the driver, and the structure virtio_net_ctrl_coal is read-only for the driver.
\end{itemize}

The class VIRTIO_NET_CTRL_NOTF_COAL has the following commands:
\begin{enumerate}
\item VIRTIO_NET_CTRL_NOTF_COAL_TX_SET: use the structure virtio_net_ctrl_coal to set the \field{max_usecs} and \field{max_packets} parameters for all transmit virtqueues.
\item VIRTIO_NET_CTRL_NOTF_COAL_RX_SET: use the structure virtio_net_ctrl_coal to set the \field{max_usecs} and \field{max_packets} parameters for all receive virtqueues.
\item VIRTIO_NET_CTRL_NOTF_COAL_VQ_SET: use the structure virtio_net_ctrl_coal_vq to set the \field{max_usecs} and \field{max_packets} parameters
                                        for an enabled transmit/receive virtqueue whose index is \field{vq_index}.
\item VIRTIO_NET_CTRL_NOTF_COAL_VQ_GET: use the structure virtio_net_ctrl_coal_vq to get the \field{max_usecs} and \field{max_packets} parameters
                                        for an enabled transmit/receive virtqueue whose index is \field{vq_index}.
\end{enumerate}

The device may generate notifications more or less frequently than specified by set commands of the VIRTIO_NET_CTRL_NOTF_COAL class.

If coalescing parameters are being set, the device applies the last coalescing parameters set for a
virtqueue, regardless of the command used to set the parameters. Use the following command sequence
with two pairs of virtqueues as an example:
Each of the following commands sets \field{max_usecs} and \field{max_packets} parameters for virtqueues.
\begin{itemize}
\item Command1: VIRTIO_NET_CTRL_NOTF_COAL_RX_SET sets coalescing parameters for virtqueues having index 0 and index 2. Virtqueues having index 1 and index 3 retain their previous parameters.
\item Command2: VIRTIO_NET_CTRL_NOTF_COAL_VQ_SET with \field{vq_index} = 0 sets coalescing parameters for virtqueue having index 0. Virtqueue having index 2 retains the parameters from command1.
\item Command3: VIRTIO_NET_CTRL_NOTF_COAL_VQ_GET with \field{vq_index} = 0, the device responds with coalescing parameters of vq_index 0 set by command2.
\item Command4: VIRTIO_NET_CTRL_NOTF_COAL_VQ_SET with \field{vq_index} = 1 sets coalescing parameters for virtqueue having index 1. Virtqueue having index 3 retains its previous parameters.
\item Command5: VIRTIO_NET_CTRL_NOTF_COAL_TX_SET sets coalescing parameters for virtqueues having index 1 and index 3, and overrides the parameters set by command4.
\item Command6: VIRTIO_NET_CTRL_NOTF_COAL_VQ_GET with \field{vq_index} = 1, the device responds with coalescing parameters of index 1 set by command5.
\end{itemize}

\subparagraph{Operation}\label{sec:Device Types / Network Device / Device Operation / Control Virtqueue / Notifications Coalescing / Operation}

The device sends a used buffer notification once the notification conditions are met and if the notifications are not suppressed as explained in \ref{sec:Basic Facilities of a Virtio Device / Virtqueues / Used Buffer Notification Suppression}.

When the device has non-zero \field{max_usecs} and non-zero \field{max_packets}, it starts counting microseconds and packets upon receiving/sending a packet.
The device counts packets and microseconds for each receive virtqueue and transmit virtqueue separately.
In this case, the notification conditions are met when \field{max_usecs} microseconds elapse, or upon sending/receiving \field{max_packets} packets, whichever happens first.
Afterwards, the device waits for the next packet and starts counting packets and microseconds again.

When the device has \field{max_usecs} = 0 or \field{max_packets} = 0, the notification conditions are met after every packet received/sent.

\subparagraph{RX Example}\label{sec:Device Types / Network Device / Device Operation / Control Virtqueue / Notifications Coalescing / RX Example}

If, for example:
\begin{itemize}
\item \field{max_usecs} = 10.
\item \field{max_packets} = 15.
\end{itemize}
then each receive virtqueue of a device will operate as follows:
\begin{itemize}
\item The device will count packets received on each virtqueue until it accumulates 15, or until 10 microseconds elapsed since the first one was received.
\item If the notifications are not suppressed by the driver, the device will send an used buffer notification, otherwise, the device will not send an used buffer notification as long as the notifications are suppressed.
\end{itemize}

\subparagraph{TX Example}\label{sec:Device Types / Network Device / Device Operation / Control Virtqueue / Notifications Coalescing / TX Example}

If, for example:
\begin{itemize}
\item \field{max_usecs} = 10.
\item \field{max_packets} = 15.
\end{itemize}
then each transmit virtqueue of a device will operate as follows:
\begin{itemize}
\item The device will count packets sent on each virtqueue until it accumulates 15, or until 10 microseconds elapsed since the first one was sent.
\item If the notifications are not suppressed by the driver, the device will send an used buffer notification, otherwise, the device will not send an used buffer notification as long as the notifications are suppressed.
\end{itemize}

\subparagraph{Notifications When Coalescing Parameters Change}\label{sec:Device Types / Network Device / Device Operation / Control Virtqueue / Notifications Coalescing / Notifications When Coalescing Parameters Change}

When the coalescing parameters of a device change, the device needs to check if the new notification conditions are met and send a used buffer notification if so.

For example, \field{max_packets} = 15 for a device with a single transmit virtqueue: if the device sends 10 packets and afterwards receives a
VIRTIO_NET_CTRL_NOTF_COAL_TX_SET command with \field{max_packets} = 8, then the notification condition is immediately considered to be met;
the device needs to immediately send a used buffer notification, if the notifications are not suppressed by the driver.

\drivernormative{\subparagraph}{Notifications Coalescing}{Device Types / Network Device / Device Operation / Control Virtqueue / Notifications Coalescing}

The driver MUST set \field{vq_index} to the virtqueue index of an enabled transmit or receive virtqueue.

The driver MUST have negotiated the VIRTIO_NET_F_NOTF_COAL feature when issuing commands VIRTIO_NET_CTRL_NOTF_COAL_TX_SET and VIRTIO_NET_CTRL_NOTF_COAL_RX_SET.

The driver MUST have negotiated the VIRTIO_NET_F_VQ_NOTF_COAL feature when issuing commands VIRTIO_NET_CTRL_NOTF_COAL_VQ_SET and VIRTIO_NET_CTRL_NOTF_COAL_VQ_GET.

The driver MUST ignore the values of coalescing parameters received from the VIRTIO_NET_CTRL_NOTF_COAL_VQ_GET command if the device responds with VIRTIO_NET_ERR.

\devicenormative{\subparagraph}{Notifications Coalescing}{Device Types / Network Device / Device Operation / Control Virtqueue / Notifications Coalescing}

The device MUST ignore \field{reserved}.

The device SHOULD respond to VIRTIO_NET_CTRL_NOTF_COAL_TX_SET and VIRTIO_NET_CTRL_NOTF_COAL_RX_SET commands with VIRTIO_NET_ERR if it was not able to change the parameters.

The device MUST respond to the VIRTIO_NET_CTRL_NOTF_COAL_VQ_SET command with VIRTIO_NET_ERR if it was not able to change the parameters.

The device MUST respond to VIRTIO_NET_CTRL_NOTF_COAL_VQ_SET and VIRTIO_NET_CTRL_NOTF_COAL_VQ_GET commands with
VIRTIO_NET_ERR if the designated virtqueue is not an enabled transmit or receive virtqueue.

Upon disabling and re-enabling a transmit virtqueue, the device MUST set the coalescing parameters of the virtqueue
to those configured through the VIRTIO_NET_CTRL_NOTF_COAL_TX_SET command, or, if the driver did not set any TX coalescing parameters, to 0.

Upon disabling and re-enabling a receive virtqueue, the device MUST set the coalescing parameters of the virtqueue
to those configured through the VIRTIO_NET_CTRL_NOTF_COAL_RX_SET command, or, if the driver did not set any RX coalescing parameters, to 0.

The behavior of the device in response to set commands of the VIRTIO_NET_CTRL_NOTF_COAL class is best-effort:
the device MAY generate notifications more or less frequently than specified.

A device SHOULD NOT send used buffer notifications to the driver if the notifications are suppressed, even if the notification conditions are met.

Upon reset, a device MUST initialize all coalescing parameters to 0.

\paragraph{Device Statistics}\label{sec:Device Types / Network Device / Device Operation / Control Virtqueue / Device Statistics}

If the VIRTIO_NET_F_DEVICE_STATS feature is negotiated, the driver can obtain
device statistics from the device by using the following command.

Different types of virtqueues have different statistics. The statistics of the
receiveq are different from those of the transmitq.

The statistics of a certain type of virtqueue are also divided into multiple types
because different types require different features. This enables the expansion
of new statistics.

In one command, the driver can obtain the statistics of one or multiple virtqueues.
Additionally, the driver can obtain multiple type statistics of each virtqueue.

\subparagraph{Query Statistic Capabilities}\label{sec:Device Types / Network Device / Device Operation / Control Virtqueue / Device Statistics / Query Statistic Capabilities}

\begin{lstlisting}
#define VIRTIO_NET_CTRL_STATS         8
#define VIRTIO_NET_CTRL_STATS_QUERY   0
#define VIRTIO_NET_CTRL_STATS_GET     1

struct virtio_net_stats_capabilities {

#define VIRTIO_NET_STATS_TYPE_CVQ       (1 << 32)

#define VIRTIO_NET_STATS_TYPE_RX_BASIC  (1 << 0)
#define VIRTIO_NET_STATS_TYPE_RX_CSUM   (1 << 1)
#define VIRTIO_NET_STATS_TYPE_RX_GSO    (1 << 2)
#define VIRTIO_NET_STATS_TYPE_RX_SPEED  (1 << 3)

#define VIRTIO_NET_STATS_TYPE_TX_BASIC  (1 << 16)
#define VIRTIO_NET_STATS_TYPE_TX_CSUM   (1 << 17)
#define VIRTIO_NET_STATS_TYPE_TX_GSO    (1 << 18)
#define VIRTIO_NET_STATS_TYPE_TX_SPEED  (1 << 19)

    le64 supported_stats_types[1];
}
\end{lstlisting}

To obtain device statistic capability, use the VIRTIO_NET_CTRL_STATS_QUERY
command. When the command completes successfully, \field{command-specific-result}
is in the format of \field{struct virtio_net_stats_capabilities}.

\subparagraph{Get Statistics}\label{sec:Device Types / Network Device / Device Operation / Control Virtqueue / Device Statistics / Get Statistics}

\begin{lstlisting}
struct virtio_net_ctrl_queue_stats {
       struct {
           le16 vq_index;
           le16 reserved[3];
           le64 types_bitmap[1];
       } stats[];
};

struct virtio_net_stats_reply_hdr {
#define VIRTIO_NET_STATS_TYPE_REPLY_CVQ       32

#define VIRTIO_NET_STATS_TYPE_REPLY_RX_BASIC  0
#define VIRTIO_NET_STATS_TYPE_REPLY_RX_CSUM   1
#define VIRTIO_NET_STATS_TYPE_REPLY_RX_GSO    2
#define VIRTIO_NET_STATS_TYPE_REPLY_RX_SPEED  3

#define VIRTIO_NET_STATS_TYPE_REPLY_TX_BASIC  16
#define VIRTIO_NET_STATS_TYPE_REPLY_TX_CSUM   17
#define VIRTIO_NET_STATS_TYPE_REPLY_TX_GSO    18
#define VIRTIO_NET_STATS_TYPE_REPLY_TX_SPEED  19
    u8 type;
    u8 reserved;
    le16 vq_index;
    le16 reserved1;
    le16 size;
}
\end{lstlisting}

To obtain device statistics, use the VIRTIO_NET_CTRL_STATS_GET command with the
\field{command-specific-data} which is in the format of
\field{struct virtio_net_ctrl_queue_stats}. When the command completes
successfully, \field{command-specific-result} contains multiple statistic
results, each statistic result has the \field{struct virtio_net_stats_reply_hdr}
as the header.

The fields of the \field{struct virtio_net_ctrl_queue_stats}:
\begin{description}
    \item [vq_index]
        The index of the virtqueue to obtain the statistics.

    \item [types_bitmap]
        This is a bitmask of the types of statistics to be obtained. Therefore, a
        \field{stats} inside \field{struct virtio_net_ctrl_queue_stats} may
        indicate multiple statistic replies for the virtqueue.
\end{description}

The fields of the \field{struct virtio_net_stats_reply_hdr}:
\begin{description}
    \item [type]
        The type of the reply statistic.

    \item [vq_index]
        The virtqueue index of the reply statistic.

    \item [size]
        The number of bytes for the statistics entry including size of \field{struct virtio_net_stats_reply_hdr}.

\end{description}

\subparagraph{Controlq Statistics}\label{sec:Device Types / Network Device / Device Operation / Control Virtqueue / Device Statistics / Controlq Statistics}

The structure corresponding to the controlq statistics is
\field{struct virtio_net_stats_cvq}. The corresponding type is
VIRTIO_NET_STATS_TYPE_CVQ. This is for the controlq.

\begin{lstlisting}
struct virtio_net_stats_cvq {
    struct virtio_net_stats_reply_hdr hdr;

    le64 command_num;
    le64 ok_num;
};
\end{lstlisting}

\begin{description}
    \item [command_num]
        The number of commands received by the device including the current command.

    \item [ok_num]
        The number of commands completed successfully by the device including the current command.
\end{description}


\subparagraph{Receiveq Basic Statistics}\label{sec:Device Types / Network Device / Device Operation / Control Virtqueue / Device Statistics / Receiveq Basic Statistics}

The structure corresponding to the receiveq basic statistics is
\field{struct virtio_net_stats_rx_basic}. The corresponding type is
VIRTIO_NET_STATS_TYPE_RX_BASIC. This is for the receiveq.

Receiveq basic statistics do not require any feature. As long as the device supports
VIRTIO_NET_F_DEVICE_STATS, the following are the receiveq basic statistics.

\begin{lstlisting}
struct virtio_net_stats_rx_basic {
    struct virtio_net_stats_reply_hdr hdr;

    le64 rx_notifications;

    le64 rx_packets;
    le64 rx_bytes;

    le64 rx_interrupts;

    le64 rx_drops;
    le64 rx_drop_overruns;
};
\end{lstlisting}

The packets described below were all presented on the specified virtqueue.
\begin{description}
    \item [rx_notifications]
        The number of driver notifications received by the device for this
        receiveq.

    \item [rx_packets]
        This is the number of packets passed to the driver by the device.

    \item [rx_bytes]
        This is the bytes of packets passed to the driver by the device.

    \item [rx_interrupts]
        The number of interrupts generated by the device for this receiveq.

    \item [rx_drops]
        This is the number of packets dropped by the device. The count includes
        all types of packets dropped by the device.

    \item [rx_drop_overruns]
        This is the number of packets dropped by the device when no more
        descriptors were available.

\end{description}

\subparagraph{Transmitq Basic Statistics}\label{sec:Device Types / Network Device / Device Operation / Control Virtqueue / Device Statistics / Transmitq Basic Statistics}

The structure corresponding to the transmitq basic statistics is
\field{struct virtio_net_stats_tx_basic}. The corresponding type is
VIRTIO_NET_STATS_TYPE_TX_BASIC. This is for the transmitq.

Transmitq basic statistics do not require any feature. As long as the device supports
VIRTIO_NET_F_DEVICE_STATS, the following are the transmitq basic statistics.

\begin{lstlisting}
struct virtio_net_stats_tx_basic {
    struct virtio_net_stats_reply_hdr hdr;

    le64 tx_notifications;

    le64 tx_packets;
    le64 tx_bytes;

    le64 tx_interrupts;

    le64 tx_drops;
    le64 tx_drop_malformed;
};
\end{lstlisting}

The packets described below are all for a specific virtqueue.
\begin{description}
    \item [tx_notifications]
        The number of driver notifications received by the device for this
        transmitq.

    \item [tx_packets]
        This is the number of packets sent by the device (not the packets
        got from the driver).

    \item [tx_bytes]
        This is the number of bytes sent by the device for all the sent packets
        (not the bytes sent got from the driver).

    \item [tx_interrupts]
        The number of interrupts generated by the device for this transmitq.

    \item [tx_drops]
        The number of packets dropped by the device. The count includes all
        types of packets dropped by the device.

    \item [tx_drop_malformed]
        The number of packets dropped by the device, when the descriptors are
        malformed. For example, the buffer is too short.
\end{description}

\subparagraph{Receiveq CSUM Statistics}\label{sec:Device Types / Network Device / Device Operation / Control Virtqueue / Device Statistics / Receiveq CSUM Statistics}

The structure corresponding to the receiveq checksum statistics is
\field{struct virtio_net_stats_rx_csum}. The corresponding type is
VIRTIO_NET_STATS_TYPE_RX_CSUM. This is for the receiveq.

Only after the VIRTIO_NET_F_GUEST_CSUM is negotiated, the receiveq checksum
statistics can be obtained.

\begin{lstlisting}
struct virtio_net_stats_rx_csum {
    struct virtio_net_stats_reply_hdr hdr;

    le64 rx_csum_valid;
    le64 rx_needs_csum;
    le64 rx_csum_none;
    le64 rx_csum_bad;
};
\end{lstlisting}

The packets described below were all presented on the specified virtqueue.
\begin{description}
    \item [rx_csum_valid]
        The number of packets with VIRTIO_NET_HDR_F_DATA_VALID.

    \item [rx_needs_csum]
        The number of packets with VIRTIO_NET_HDR_F_NEEDS_CSUM.

    \item [rx_csum_none]
        The number of packets without hardware checksum. The packet here refers
        to the non-TCP/UDP packet that the device cannot recognize.

    \item [rx_csum_bad]
        The number of packets with checksum mismatch.

\end{description}

\subparagraph{Transmitq CSUM Statistics}\label{sec:Device Types / Network Device / Device Operation / Control Virtqueue / Device Statistics / Transmitq CSUM Statistics}

The structure corresponding to the transmitq checksum statistics is
\field{struct virtio_net_stats_tx_csum}. The corresponding type is
VIRTIO_NET_STATS_TYPE_TX_CSUM. This is for the transmitq.

Only after the VIRTIO_NET_F_CSUM is negotiated, the transmitq checksum
statistics can be obtained.

The following are the transmitq checksum statistics:

\begin{lstlisting}
struct virtio_net_stats_tx_csum {
    struct virtio_net_stats_reply_hdr hdr;

    le64 tx_csum_none;
    le64 tx_needs_csum;
};
\end{lstlisting}

The packets described below are all for a specific virtqueue.
\begin{description}
    \item [tx_csum_none]
        The number of packets which do not require hardware checksum.

    \item [tx_needs_csum]
        The number of packets which require checksum calculation by the device.

\end{description}

\subparagraph{Receiveq GSO Statistics}\label{sec:Device Types / Network Device / Device Operation / Control Virtqueue / Device Statistics / Receiveq GSO Statistics}

The structure corresponding to the receivq GSO statistics is
\field{struct virtio_net_stats_rx_gso}. The corresponding type is
VIRTIO_NET_STATS_TYPE_RX_GSO. This is for the receiveq.

If one or more of the VIRTIO_NET_F_GUEST_TSO4, VIRTIO_NET_F_GUEST_TSO6
have been negotiated, the receiveq GSO statistics can be obtained.

GSO packets refer to packets passed by the device to the driver where
\field{gso_type} is not VIRTIO_NET_HDR_GSO_NONE.

\begin{lstlisting}
struct virtio_net_stats_rx_gso {
    struct virtio_net_stats_reply_hdr hdr;

    le64 rx_gso_packets;
    le64 rx_gso_bytes;
    le64 rx_gso_packets_coalesced;
    le64 rx_gso_bytes_coalesced;
};
\end{lstlisting}

The packets described below were all presented on the specified virtqueue.
\begin{description}
    \item [rx_gso_packets]
        The number of the GSO packets received by the device.

    \item [rx_gso_bytes]
        The bytes of the GSO packets received by the device.
        This includes the header size of the GSO packet.

    \item [rx_gso_packets_coalesced]
        The number of the GSO packets coalesced by the device.

    \item [rx_gso_bytes_coalesced]
        The bytes of the GSO packets coalesced by the device.
        This includes the header size of the GSO packet.
\end{description}

\subparagraph{Transmitq GSO Statistics}\label{sec:Device Types / Network Device / Device Operation / Control Virtqueue / Device Statistics / Transmitq GSO Statistics}

The structure corresponding to the transmitq GSO statistics is
\field{struct virtio_net_stats_tx_gso}. The corresponding type is
VIRTIO_NET_STATS_TYPE_TX_GSO. This is for the transmitq.

If one or more of the VIRTIO_NET_F_HOST_TSO4, VIRTIO_NET_F_HOST_TSO6,
VIRTIO_NET_F_HOST_USO options have been negotiated, the transmitq GSO statistics
can be obtained.

GSO packets refer to packets passed by the driver to the device where
\field{gso_type} is not VIRTIO_NET_HDR_GSO_NONE.
See more \ref{sec:Device Types / Network Device / Device Operation / Packet
Transmission}.

\begin{lstlisting}
struct virtio_net_stats_tx_gso {
    struct virtio_net_stats_reply_hdr hdr;

    le64 tx_gso_packets;
    le64 tx_gso_bytes;
    le64 tx_gso_segments;
    le64 tx_gso_segments_bytes;
    le64 tx_gso_packets_noseg;
    le64 tx_gso_bytes_noseg;
};
\end{lstlisting}

The packets described below are all for a specific virtqueue.
\begin{description}
    \item [tx_gso_packets]
        The number of the GSO packets sent by the device.

    \item [tx_gso_bytes]
        The bytes of the GSO packets sent by the device.

    \item [tx_gso_segments]
        The number of segments prepared from GSO packets.

    \item [tx_gso_segments_bytes]
        The bytes of segments prepared from GSO packets.

    \item [tx_gso_packets_noseg]
        The number of the GSO packets without segmentation.

    \item [tx_gso_bytes_noseg]
        The bytes of the GSO packets without segmentation.

\end{description}

\subparagraph{Receiveq Speed Statistics}\label{sec:Device Types / Network Device / Device Operation / Control Virtqueue / Device Statistics / Receiveq Speed Statistics}

The structure corresponding to the receiveq speed statistics is
\field{struct virtio_net_stats_rx_speed}. The corresponding type is
VIRTIO_NET_STATS_TYPE_RX_SPEED. This is for the receiveq.

The device has the allowance for the speed. If VIRTIO_NET_F_SPEED_DUPLEX has
been negotiated, the driver can get this by \field{speed}. When the received
packets bitrate exceeds the \field{speed}, some packets may be dropped by the
device.

\begin{lstlisting}
struct virtio_net_stats_rx_speed {
    struct virtio_net_stats_reply_hdr hdr;

    le64 rx_packets_allowance_exceeded;
    le64 rx_bytes_allowance_exceeded;
};
\end{lstlisting}

The packets described below were all presented on the specified virtqueue.
\begin{description}
    \item [rx_packets_allowance_exceeded]
        The number of the packets dropped by the device due to the received
        packets bitrate exceeding the \field{speed}.

    \item [rx_bytes_allowance_exceeded]
        The bytes of the packets dropped by the device due to the received
        packets bitrate exceeding the \field{speed}.

\end{description}

\subparagraph{Transmitq Speed Statistics}\label{sec:Device Types / Network Device / Device Operation / Control Virtqueue / Device Statistics / Transmitq Speed Statistics}

The structure corresponding to the transmitq speed statistics is
\field{struct virtio_net_stats_tx_speed}. The corresponding type is
VIRTIO_NET_STATS_TYPE_TX_SPEED. This is for the transmitq.

The device has the allowance for the speed. If VIRTIO_NET_F_SPEED_DUPLEX has
been negotiated, the driver can get this by \field{speed}. When the transmit
packets bitrate exceeds the \field{speed}, some packets may be dropped by the
device.

\begin{lstlisting}
struct virtio_net_stats_tx_speed {
    struct virtio_net_stats_reply_hdr hdr;

    le64 tx_packets_allowance_exceeded;
    le64 tx_bytes_allowance_exceeded;
};
\end{lstlisting}

The packets described below were all presented on the specified virtqueue.
\begin{description}
    \item [tx_packets_allowance_exceeded]
        The number of the packets dropped by the device due to the transmit packets
        bitrate exceeding the \field{speed}.

    \item [tx_bytes_allowance_exceeded]
        The bytes of the packets dropped by the device due to the transmit packets
        bitrate exceeding the \field{speed}.

\end{description}

\devicenormative{\subparagraph}{Device Statistics}{Device Types / Network Device / Device Operation / Control Virtqueue / Device Statistics}

When the VIRTIO_NET_F_DEVICE_STATS feature is negotiated, the device MUST reply
to the command VIRTIO_NET_CTRL_STATS_QUERY with the
\field{struct virtio_net_stats_capabilities}. \field{supported_stats_types}
includes all the statistic types supported by the device.

If \field{struct virtio_net_ctrl_queue_stats} is incorrect (such as the
following), the device MUST set \field{ack} to VIRTIO_NET_ERR. Even if there is
only one error, the device MUST fail the entire command.
\begin{itemize}
    \item \field{vq_index} exceeds the queue range.
    \item \field{types_bitmap} contains unknown types.
    \item One or more of the bits present in \field{types_bitmap} is not valid
        for the specified virtqueue.
    \item The feature corresponding to the specified \field{types_bitmap} was
        not negotiated.
\end{itemize}

The device MUST set the actual size of the bytes occupied by the reply to the
\field{size} of the \field{hdr}. And the device MUST set the \field{type} and
the \field{vq_index} of the statistic header.

The \field{command-specific-result} buffer allocated by the driver may be
smaller or bigger than all the statistics specified by
\field{struct virtio_net_ctrl_queue_stats}. The device MUST fill up only upto
the valid bytes.

The statistics counter replied by the device MUST wrap around to zero by the
device on the overflow.

\drivernormative{\subparagraph}{Device Statistics}{Device Types / Network Device / Device Operation / Control Virtqueue / Device Statistics}

The types contained in the \field{types_bitmap} MUST be queried from the device
via command VIRTIO_NET_CTRL_STATS_QUERY.

\field{types_bitmap} in \field{struct virtio_net_ctrl_queue_stats} MUST be valid to the
vq specified by \field{vq_index}.

The \field{command-specific-result} buffer allocated by the driver MUST have
enough capacity to store all the statistics reply headers defined in
\field{struct virtio_net_ctrl_queue_stats}. If the
\field{command-specific-result} buffer is fully utilized by the device but some
replies are missed, it is possible that some statistics may exceed the capacity
of the driver's records. In such cases, the driver should allocate additional
space for the \field{command-specific-result} buffer.

\subsubsection{Flow filter}\label{sec:Device Types / Network Device / Device Operation / Flow filter}

A network device can support one or more flow filter rules. Each flow filter rule
is applied by matching a packet and then taking an action, such as directing the packet
to a specific receiveq or dropping the packet. An example of a match is
matching on specific source and destination IP addresses.

A flow filter rule is a device resource object that consists of a key,
a processing priority, and an action to either direct a packet to a
receive queue or drop the packet.

Each rule uses a classifier. The key is matched against the packet using
a classifier, defining which fields in the packet are matched.
A classifier resource object consists of one or more field selectors, each with
a type that specifies the header fields to be matched against, and a mask.
The mask can match whole fields or parts of a field in a header. Each
rule resource object depends on the classifier resource object.

When a packet is received, relevant fields are extracted
(in the same way) from both the packet and the key according to the
classifier. The resulting field contents are then compared -
if they are identical the rule action is taken, if they are not, the rule is ignored.

Multiple flow filter rules are part of a group. The rule resource object
depends on the group. Each rule within a
group has a rule priority, and each group also has a group priority. For a
packet, a group with the highest priority is selected first. Within a group,
rules are applied from highest to lowest priority, until one of the rules
matches the packet and an action is taken. If all the rules within a group
are ignored, the group with the next highest priority is selected, and so on.

The device and the driver indicates flow filter resource limits using the capability
\ref{par:Device Types / Network Device / Device Operation / Flow filter / Device and driver capabilities / VIRTIO-NET-FF-RESOURCE-CAP} specifying the limits on the number of flow filter rule,
group and classifier resource objects. The capability \ref{par:Device Types / Network Device / Device Operation / Flow filter / Device and driver capabilities / VIRTIO-NET-FF-SELECTOR-CAP} specifies which selectors the device supports.
The driver indicates the selectors it is using by setting the flow
filter selector capability, prior to adding any resource objects.

The capability \ref{par:Device Types / Network Device / Device Operation / Flow filter / Device and driver capabilities / VIRTIO-NET-FF-ACTION-CAP} specifies which actions the device supports.

The driver controls the flow filter rule, classifier and group resource objects using
administration commands described in
\ref{sec:Basic Facilities of a Virtio Device / Device groups / Group administration commands / Device resource objects}.

\paragraph{Packet processing order}\label{sec:sec:Device Types / Network Device / Device Operation / Flow filter / Packet processing order}

Note that flow filter rules are applied after MAC/VLAN filtering. Flow filter
rules take precedence over steering: if a flow filter rule results in an action,
the steering configuration does not apply. The steering configuration only applies
to packets for which no flow filter rule action was performed. For example,
incoming packets can be processed in the following order:

\begin{itemize}
\item apply steering configuration received using control virtqueue commands
      VIRTIO_NET_CTRL_RX, VIRTIO_NET_CTRL_MAC and VIRTIO_NET_CTRL_VLAN.
\item apply flow filter rules if any.
\item if no filter rule applied, apply steering configuration received using command
      VIRTIO_NET_CTRL_MQ_RSS_CONFIG or as per automatic receive steering.
\end{itemize}

Some incoming packet processing examples:
\begin{itemize}
\item If the packet is dropped by the flow filter rule, RSS
      steering is ignored for the packet.
\item If the packet is directed to a specific receiveq using flow filter rule,
      the RSS steering is ignored for the packet.
\item If a packet is dropped due to the VIRTIO_NET_CTRL_MAC configuration,
      both flow filter rules and the RSS steering are ignored for the packet.
\item If a packet does not match any flow filter rules,
      the RSS steering is used to select the receiveq for the packet (if enabled).
\item If there are two flow filter groups configured as group_A and group_B
      with respective group priorities as 4, and 5; flow filter rules of
      group_B are applied first having highest group priority, if there is a match,
      the flow filter rules of group_A are ignored; if there is no match for
      the flow filter rules in group_B, the flow filter rules of next level group_A are applied.
\end{itemize}

\paragraph{Device and driver capabilities}
\label{par:Device Types / Network Device / Device Operation / Flow filter / Device and driver capabilities}

\subparagraph{VIRTIO_NET_FF_RESOURCE_CAP}
\label{par:Device Types / Network Device / Device Operation / Flow filter / Device and driver capabilities / VIRTIO-NET-FF-RESOURCE-CAP}

The capability VIRTIO_NET_FF_RESOURCE_CAP indicates the flow filter resource limits.
\field{cap_specific_data} is in the format
\field{struct virtio_net_ff_cap_data}.

\begin{lstlisting}
struct virtio_net_ff_cap_data {
        le32 groups_limit;
        le32 selectors_limit;
        le32 rules_limit;
        le32 rules_per_group_limit;
        u8 last_rule_priority;
        u8 selectors_per_classifier_limit;
};
\end{lstlisting}

\field{groups_limit}, and \field{selectors_limit} represent the maximum
number of flow filter groups and selectors, respectively, that the driver can create.
 \field{rules_limit} is the maximum number of
flow fiilter rules that the driver can create across all the groups.
\field{rules_per_group_limit} is the maximum number of flow filter rules that the driver
can create for each flow filter group.

\field{last_rule_priority} is the highest priority that can be assigned to a
flow filter rule.

\field{selectors_per_classifier_limit} is the maximum number of selectors
that a classifier can have.

\subparagraph{VIRTIO_NET_FF_SELECTOR_CAP}
\label{par:Device Types / Network Device / Device Operation / Flow filter / Device and driver capabilities / VIRTIO-NET-FF-SELECTOR-CAP}

The capability VIRTIO_NET_FF_SELECTOR_CAP lists the supported selectors and the
supported packet header fields for each selector.
\field{cap_specific_data} is in the format \field{struct virtio_net_ff_cap_mask_data}.

\begin{lstlisting}[label={lst:Device Types / Network Device / Device Operation / Flow filter / Device and driver capabilities / VIRTIO-NET-FF-SELECTOR-CAP / virtio-net-ff-selector}]
struct virtio_net_ff_selector {
        u8 type;
        u8 flags;
        u8 reserved[2];
        u8 length;
        u8 reserved1[3];
        u8 mask[];
};

struct virtio_net_ff_cap_mask_data {
        u8 count;
        u8 reserved[7];
        struct virtio_net_ff_selector selectors[];
};

#define VIRTIO_NET_FF_MASK_F_PARTIAL_MASK (1 << 0)
\end{lstlisting}

\field{count} indicates number of valid entries in the \field{selectors} array.
\field{selectors[]} is an array of supported selectors. Within each array entry:
\field{type} specifies the type of the packet header, as defined in table
\ref{table:Device Types / Network Device / Device Operation / Flow filter / Device and driver capabilities / VIRTIO-NET-FF-SELECTOR-CAP / flow filter selector types}. \field{mask} specifies which fields of the
packet header can be matched in a flow filter rule.

Each \field{type} is also listed in table
\ref{table:Device Types / Network Device / Device Operation / Flow filter / Device and driver capabilities / VIRTIO-NET-FF-SELECTOR-CAP / flow filter selector types}. \field{mask} is a byte array
in network byte order. For example, when \field{type} is VIRTIO_NET_FF_MASK_TYPE_IPV6,
the \field{mask} is in the format \hyperref[intro:IPv6-Header-Format]{IPv6 Header Format}.

If partial masking is not set, then all bits in each field have to be either all 0s
to ignore this field or all 1s to match on this field. If partial masking is set,
then any combination of bits can bit set to match on these bits.
For example, when a selector \field{type} is VIRTIO_NET_FF_MASK_TYPE_ETH, if
\field{mask[0-12]} are zero and \field{mask[13-14]} are 0xff (all 1s), it
indicates that matching is only supported for \field{EtherType} of
\field{Ethernet MAC frame}, matching is not supported for
\field{Destination Address} and \field{Source Address}.

The entries in the array \field{selectors} are ordered by
\field{type}, with each \field{type} value only appearing once.

\field{length} is the length of a dynamic array \field{mask} in bytes.
\field{reserved} and \field{reserved1} are reserved and set to zero.

\begin{table}[H]
\caption{Flow filter selector types}
\label{table:Device Types / Network Device / Device Operation / Flow filter / Device and driver capabilities / VIRTIO-NET-FF-SELECTOR-CAP / flow filter selector types}
\begin{tabularx}{\textwidth}{ |l|X|X| }
\hline
Type & Name & Description \\
\hline \hline
0x0 & - & Reserved \\
\hline
0x1 & VIRTIO_NET_FF_MASK_TYPE_ETH & 14 bytes of frame header starting from destination address described in \hyperref[intro:IEEE 802.3-2022]{IEEE 802.3-2022} \\
\hline
0x2 & VIRTIO_NET_FF_MASK_TYPE_IPV4 & 20 bytes of \hyperref[intro:Internet-Header-Format]{IPv4: Internet Header Format} \\
\hline
0x3 & VIRTIO_NET_FF_MASK_TYPE_IPV6 & 40 bytes of \hyperref[intro:IPv6-Header-Format]{IPv6 Header Format} \\
\hline
0x4 & VIRTIO_NET_FF_MASK_TYPE_TCP & 20 bytes of \hyperref[intro:TCP-Header-Format]{TCP Header Format} \\
\hline
0x5 & VIRTIO_NET_FF_MASK_TYPE_UDP & 8 bytes of UDP header described in \hyperref[intro:UDP]{UDP} \\
\hline
0x6 - 0xFF & & Reserved for future \\
\hline
\end{tabularx}
\end{table}

When VIRTIO_NET_FF_MASK_F_PARTIAL_MASK (bit 0) is set, it indicates that
partial masking is supported for all the fields of the selector identified by \field{type}.

For the selector \field{type} VIRTIO_NET_FF_MASK_TYPE_IPV4, if a partial mask is unsupported,
then matching on an individual bit of \field{Flags} in the
\field{IPv4: Internet Header Format} is unsupported. \field{Flags} has to match as a whole
if it is supported.

For the selector \field{type} VIRTIO_NET_FF_MASK_TYPE_IPV4, \field{mask} includes fields
up to the \field{Destination Address}; that is, \field{Options} and
\field{Padding} are excluded.

For the selector \field{type} VIRTIO_NET_FF_MASK_TYPE_IPV6, the \field{Next Header} field
of the \field{mask} corresponds to the \field{Next Header} in the packet
when \field{IPv6 Extension Headers} are not present. When the packet includes
one or more \field{IPv6 Extension Headers}, the \field{Next Header} field of
the \field{mask} corresponds to the \field{Next Header} of the last
\field{IPv6 Extension Header} in the packet.

For the selector \field{type} VIRTIO_NET_FF_MASK_TYPE_TCP, \field{Control bits}
are treated as individual fields for matching; that is, matching individual
\field{Control bits} does not depend on the partial mask support.

\subparagraph{VIRTIO_NET_FF_ACTION_CAP}
\label{par:Device Types / Network Device / Device Operation / Flow filter / Device and driver capabilities / VIRTIO-NET-FF-ACTION-CAP}

The capability VIRTIO_NET_FF_ACTION_CAP lists the supported actions in a rule.
\field{cap_specific_data} is in the format \field{struct virtio_net_ff_cap_actions}.

\begin{lstlisting}
struct virtio_net_ff_actions {
        u8 count;
        u8 reserved[7];
        u8 actions[];
};
\end{lstlisting}

\field{actions} is an array listing all possible actions.
The entries in the array are ordered from the smallest to the largest,
with each supported value appearing exactly once. Each entry can have the
following values:

\begin{table}[H]
\caption{Flow filter rule actions}
\label{table:Device Types / Network Device / Device Operation / Flow filter / Device and driver capabilities / VIRTIO-NET-FF-ACTION-CAP / flow filter rule actions}
\begin{tabularx}{\textwidth}{ |l|X|X| }
\hline
Action & Name & Description \\
\hline \hline
0x0 & - & reserved \\
\hline
0x1 & VIRTIO_NET_FF_ACTION_DROP & Matching packet will be dropped by the device \\
\hline
0x2 & VIRTIO_NET_FF_ACTION_DIRECT_RX_VQ & Matching packet will be directed to a receive queue \\
\hline
0x3 - 0xFF & & Reserved for future \\
\hline
\end{tabularx}
\end{table}

\paragraph{Resource objects}
\label{par:Device Types / Network Device / Device Operation / Flow filter / Resource objects}

\subparagraph{VIRTIO_NET_RESOURCE_OBJ_FF_GROUP}\label{par:Device Types / Network Device / Device Operation / Flow filter / Resource objects / VIRTIO-NET-RESOURCE-OBJ-FF-GROUP}

A flow filter group contains between 0 and \field{rules_limit} rules, as specified by the
capability VIRTIO_NET_FF_RESOURCE_CAP. For the flow filter group object both
\field{resource_obj_specific_data} and
\field{resource_obj_specific_result} are in the format
\field{struct virtio_net_resource_obj_ff_group}.

\begin{lstlisting}
struct virtio_net_resource_obj_ff_group {
        le16 group_priority;
};
\end{lstlisting}

\field{group_priority} specifies the priority for the group. Each group has a
distinct priority. For each incoming packet, the device tries to apply rules
from groups from higher \field{group_priority} value to lower, until either a
rule matches the packet or all groups have been tried.

\subparagraph{VIRTIO_NET_RESOURCE_OBJ_FF_CLASSIFIER}\label{par:Device Types / Network Device / Device Operation / Flow filter / Resource objects / VIRTIO-NET-RESOURCE-OBJ-FF-CLASSIFIER}

A classifier is used to match a flow filter key against a packet. The
classifier defines the desired packet fields to match, and is represented by
the VIRTIO_NET_RESOURCE_OBJ_FF_CLASSIFIER device resource object.

For the flow filter classifier object both \field{resource_obj_specific_data} and
\field{resource_obj_specific_result} are in the format
\field{struct virtio_net_resource_obj_ff_classifier}.

\begin{lstlisting}
struct virtio_net_resource_obj_ff_classifier {
        u8 count;
        u8 reserved[7];
        struct virtio_net_ff_selector selectors[];
};
\end{lstlisting}

A classifier is an array of \field{selectors}. The number of selectors in the
array is indicated by \field{count}. The selector has a type that specifies
the header fields to be matched against, and a mask.
See \ref{lst:Device Types / Network Device / Device Operation / Flow filter / Device and driver capabilities / VIRTIO-NET-FF-SELECTOR-CAP / virtio-net-ff-selector}
for details about selectors.

The first selector is always VIRTIO_NET_FF_MASK_TYPE_ETH. When there are multiple
selectors, a second selector can be either VIRTIO_NET_FF_MASK_TYPE_IPV4
or VIRTIO_NET_FF_MASK_TYPE_IPV6. If the third selector exists, the third
selector can be either VIRTIO_NET_FF_MASK_TYPE_UDP or VIRTIO_NET_FF_MASK_TYPE_TCP.
For example, to match a Ethernet IPv6 UDP packet,
\field{selectors[0].type} is set to VIRTIO_NET_FF_MASK_TYPE_ETH, \field{selectors[1].type}
is set to VIRTIO_NET_FF_MASK_TYPE_IPV6 and \field{selectors[2].type} is
set to VIRTIO_NET_FF_MASK_TYPE_UDP; accordingly, \field{selectors[0].mask[0-13]} is
for Ethernet header fields, \field{selectors[1].mask[0-39]} is set for IPV6 header
and \field{selectors[2].mask[0-7]} is set for UDP header.

When there are multiple selectors, the type of the (N+1)\textsuperscript{th} selector
affects the mask of the (N)\textsuperscript{th} selector. If
\field{count} is 2 or more, all the mask bits within \field{selectors[0]}
corresponding to \field{EtherType} of an Ethernet header are set.

If \field{count} is more than 2:
\begin{itemize}
\item if \field{selector[1].type} is, VIRTIO_NET_FF_MASK_TYPE_IPV4, then, all the mask bits within
\field{selector[1]} for \field{Protocol} is set.
\item if \field{selector[1].type} is, VIRTIO_NET_FF_MASK_TYPE_IPV6, then, all the mask bits within
\field{selector[1]} for \field{Next Header} is set.
\end{itemize}

If for a given packet header field, a subset of bits of a field is to be matched,
and if the partial mask is supported, the flow filter
mask object can specify a mask which has fewer bits set than the packet header
field size. For example, a partial mask for the Ethernet header source mac
address can be of 1-bit for multicast detection instead of 48-bits.

\subparagraph{VIRTIO_NET_RESOURCE_OBJ_FF_RULE}\label{par:Device Types / Network Device / Device Operation / Flow filter / Resource objects / VIRTIO-NET-RESOURCE-OBJ-FF-RULE}

Each flow filter rule resource object comprises a key, a priority, and an action.
For the flow filter rule object,
\field{resource_obj_specific_data} and
\field{resource_obj_specific_result} are in the format
\field{struct virtio_net_resource_obj_ff_rule}.

\begin{lstlisting}
struct virtio_net_resource_obj_ff_rule {
        le32 group_id;
        le32 classifier_id;
        u8 rule_priority;
        u8 key_length; /* length of key in bytes */
        u8 action;
        u8 reserved;
        le16 vq_index;
        u8 reserved1[2];
        u8 keys[][];
};
\end{lstlisting}

\field{group_id} is the resource object ID of the flow filter group to which
this rule belongs. \field{classifier_id} is the resource object ID of the
classifier used to match a packet against the \field{key}.

\field{rule_priority} denotes the priority of the rule within the group
specified by the \field{group_id}.
Rules within the group are applied from the highest to the lowest priority
until a rule matches the packet and an
action is taken. Rules with the same priority can be applied in any order.

\field{reserved} and \field{reserved1} are reserved and set to 0.

\field{keys[][]} is an array of keys to match against packets, using
the classifier specified by \field{classifier_id}. Each entry (key) comprises
a byte array, and they are located one immediately after another.
The size (number of entries) of the array is exactly the same as that of
\field{selectors} in the classifier, or in other words, \field{count}
in the classifier.

\field{key_length} specifies the total length of \field{keys} in bytes.
In other words, it equals the sum total of \field{length} of all
selectors in \field{selectors} in the classifier specified by
\field{classifier_id}.

For example, if a classifier object's \field{selectors[0].type} is
VIRTIO_NET_FF_MASK_TYPE_ETH and \field{selectors[1].type} is
VIRTIO_NET_FF_MASK_TYPE_IPV6,
then selectors[0].length is 14 and selectors[1].length is 40.
Accordingly, the \field{key_length} is set to 54.
This setting indicates that the \field{key} array's length is 54 bytes
comprising a first byte array of 14 bytes for the
Ethernet MAC header in bytes 0-13, immediately followed by 40 bytes for the
IPv6 header in bytes 14-53.

When there are multiple selectors in the classifier object, the key bytes
for (N)\textsuperscript{th} selector are set so that
(N+1)\textsuperscript{th} selector can be matched.

If \field{count} is 2 or more, key bytes of \field{EtherType}
are set according to \hyperref[intro:IEEE 802 Ethertypes]{IEEE 802 Ethertypes}
for VIRTIO_NET_FF_MASK_TYPE_IPV4 or VIRTIO_NET_FF_MASK_TYPE_IPV6 respectively.

If \field{count} is more than 2, when \field{selector[1].type} is
VIRTIO_NET_FF_MASK_TYPE_IPV4 or VIRTIO_NET_FF_MASK_TYPE_IPV6, key
bytes of \field{Protocol} or \field{Next Header} is set as per
\field{Protocol Numbers} defined \hyperref[intro:IANA Protocol Numbers]{IANA Protocol Numbers}
respectively.

\field{action} is the action to take when a packet matches the
\field{key} using the \field{classifier_id}. Supported actions are described in
\ref{table:Device Types / Network Device / Device Operation / Flow filter / Device and driver capabilities / VIRTIO-NET-FF-ACTION-CAP / flow filter rule actions}.

\field{vq_index} specifies a receive virtqueue. When the \field{action} is set
to VIRTIO_NET_FF_ACTION_DIRECT_RX_VQ, and the packet matches the \field{key},
the matching packet is directed to this virtqueue.

Note that at most one action is ever taken for a given packet. If a rule is
applied and an action is taken, the action of other rules is not taken.

\devicenormative{\paragraph}{Flow filter}{Device Types / Network Device / Device Operation / Flow filter}

When the device supports flow filter operations,
\begin{itemize}
\item the device MUST set VIRTIO_NET_FF_RESOURCE_CAP, VIRTIO_NET_FF_SELECTOR_CAP
and VIRTIO_NET_FF_ACTION_CAP capability in the \field{supported_caps} in the
command VIRTIO_ADMIN_CMD_CAP_SUPPORT_QUERY.
\item the device MUST support the administration commands
VIRTIO_ADMIN_CMD_RESOURCE_OBJ_CREATE,
VIRTIO_ADMIN_CMD_RESOURCE_OBJ_MODIFY, VIRTIO_ADMIN_CMD_RESOURCE_OBJ_QUERY,
VIRTIO_ADMIN_CMD_RESOURCE_OBJ_DESTROY for the resource types
VIRTIO_NET_RESOURCE_OBJ_FF_GROUP, VIRTIO_NET_RESOURCE_OBJ_FF_CLASSIFIER and
VIRTIO_NET_RESOURCE_OBJ_FF_RULE.
\end{itemize}

When any of the VIRTIO_NET_FF_RESOURCE_CAP, VIRTIO_NET_FF_SELECTOR_CAP, or
VIRTIO_NET_FF_ACTION_CAP capability is disabled, the device SHOULD set
\field{status} to VIRTIO_ADMIN_STATUS_Q_INVALID_OPCODE for the commands
VIRTIO_ADMIN_CMD_RESOURCE_OBJ_CREATE,
VIRTIO_ADMIN_CMD_RESOURCE_OBJ_MODIFY, VIRTIO_ADMIN_CMD_RESOURCE_OBJ_QUERY,
and VIRTIO_ADMIN_CMD_RESOURCE_OBJ_DESTROY. These commands apply to the resource
\field{type} of VIRTIO_NET_RESOURCE_OBJ_FF_GROUP, VIRTIO_NET_RESOURCE_OBJ_FF_CLASSIFIER, and
VIRTIO_NET_RESOURCE_OBJ_FF_RULE.

The device SHOULD set \field{status} to VIRTIO_ADMIN_STATUS_EINVAL for the
command VIRTIO_ADMIN_CMD_RESOURCE_OBJ_CREATE when the resource \field{type}
is VIRTIO_NET_RESOURCE_OBJ_FF_GROUP, if a flow filter group already exists
with the supplied \field{group_priority}.

The device SHOULD set \field{status} to VIRTIO_ADMIN_STATUS_ENOSPC for the
command VIRTIO_ADMIN_CMD_RESOURCE_OBJ_CREATE when the resource \field{type}
is VIRTIO_NET_RESOURCE_OBJ_FF_GROUP, if the number of flow filter group
objects in the device exceeds the lower of the configured driver
capabilities \field{groups_limit} and \field{rules_per_group_limit}.

The device SHOULD set \field{status} to VIRTIO_ADMIN_STATUS_ENOSPC for the
command VIRTIO_ADMIN_CMD_RESOURCE_OBJ_CREATE when the resource \field{type} is
VIRTIO_NET_RESOURCE_OBJ_FF_CLASSIFIER, if the number of flow filter selector
objects in the device exceeds the configured driver capability
\field{selectors_limit}.

The device SHOULD set \field{status} to VIRTIO_ADMIN_STATUS_EBUSY for the
command VIRTIO_ADMIN_CMD_RESOURCE_OBJ_DESTROY for a flow filter group when
the flow filter group has one or more flow filter rules depending on it.

The device SHOULD set \field{status} to VIRTIO_ADMIN_STATUS_EBUSY for the
command VIRTIO_ADMIN_CMD_RESOURCE_OBJ_DESTROY for a flow filter classifier when
the flow filter classifier has one or more flow filter rules depending on it.

The device SHOULD fail the command VIRTIO_ADMIN_CMD_RESOURCE_OBJ_CREATE for the
flow filter rule resource object if,
\begin{itemize}
\item \field{vq_index} is not a valid receive virtqueue index for
the VIRTIO_NET_FF_ACTION_DIRECT_RX_VQ action,
\item \field{priority} is greater than or equal to
      \field{last_rule_priority},
\item \field{id} is greater than or equal to \field{rules_limit} or
      greater than or equal to \field{rules_per_group_limit}, whichever is lower,
\item the length of \field{keys} and the length of all the mask bytes of
      \field{selectors[].mask} as referred by \field{classifier_id} differs,
\item the supplied \field{action} is not supported in the capability VIRTIO_NET_FF_ACTION_CAP.
\end{itemize}

When the flow filter directs a packet to the virtqueue identified by
\field{vq_index} and if the receive virtqueue is reset, the device
MUST drop such packets.

Upon applying a flow filter rule to a packet, the device MUST STOP any further
application of rules and cease applying any other steering configurations.

For multiple flow filter groups, the device MUST apply the rules from
the group with the highest priority. If any rule from this group is applied,
the device MUST ignore the remaining groups. If none of the rules from the
highest priority group match, the device MUST apply the rules from
the group with the next highest priority, until either a rule matches or
all groups have been attempted.

The device MUST apply the rules within the group from the highest to the
lowest priority until a rule matches the packet, and the device MUST take
the action. If an action is taken, the device MUST not take any other
action for this packet.

The device MAY apply the rules with the same \field{rule_priority} in any
order within the group.

The device MUST process incoming packets in the following order:
\begin{itemize}
\item apply the steering configuration received using control virtqueue
      commands VIRTIO_NET_CTRL_RX, VIRTIO_NET_CTRL_MAC, and
      VIRTIO_NET_CTRL_VLAN.
\item apply flow filter rules if any.
\item if no filter rule is applied, apply the steering configuration
      received using the command VIRTIO_NET_CTRL_MQ_RSS_CONFIG
      or according to automatic receive steering.
\end{itemize}

When processing an incoming packet, if the packet is dropped at any stage, the device
MUST skip further processing.

When the device drops the packet due to the configuration done using the control
virtqueue commands VIRTIO_NET_CTRL_RX or VIRTIO_NET_CTRL_MAC or VIRTIO_NET_CTRL_VLAN,
the device MUST skip flow filter rules for this packet.

When the device performs flow filter match operations and if the operation
result did not have any match in all the groups, the receive packet processing
continues to next level, i.e. to apply configuration done using
VIRTIO_NET_CTRL_MQ_RSS_CONFIG command.

The device MUST support the creation of flow filter classifier objects
using the command VIRTIO_ADMIN_CMD_RESOURCE_OBJ_CREATE with \field{flags}
set to VIRTIO_NET_FF_MASK_F_PARTIAL_MASK;
this support is required even if all the bits of the masks are set for
a field in \field{selectors}, provided that partial masking is supported
for the selectors.

\drivernormative{\paragraph}{Flow filter}{Device Types / Network Device / Device Operation / Flow filter}

The driver MUST enable VIRTIO_NET_FF_RESOURCE_CAP, VIRTIO_NET_FF_SELECTOR_CAP,
and VIRTIO_NET_FF_ACTION_CAP capabilities to use flow filter.

The driver SHOULD NOT remove a flow filter group using the command
VIRTIO_ADMIN_CMD_RESOURCE_OBJ_DESTROY when one or more flow filter rules
depend on that group. The driver SHOULD only destroy the group after
all the associated rules have been destroyed.

The driver SHOULD NOT remove a flow filter classifier using the command
VIRTIO_ADMIN_CMD_RESOURCE_OBJ_DESTROY when one or more flow filter rules
depend on the classifier. The driver SHOULD only destroy the classifier
after all the associated rules have been destroyed.

The driver SHOULD NOT add multiple flow filter rules with the same
\field{rule_priority} within a flow filter group, as these rules MAY match
the same packet. The driver SHOULD assign different \field{rule_priority}
values to different flow filter rules if multiple rules may match a single
packet.

For the command VIRTIO_ADMIN_CMD_RESOURCE_OBJ_CREATE, when creating a resource
of \field{type} VIRTIO_NET_RESOURCE_OBJ_FF_CLASSIFIER, the driver MUST set:
\begin{itemize}
\item \field{selectors[0].type} to VIRTIO_NET_FF_MASK_TYPE_ETH.
\item \field{selectors[1].type} to VIRTIO_NET_FF_MASK_TYPE_IPV4 or
      VIRTIO_NET_FF_MASK_TYPE_IPV6 when \field{count} is more than 1,
\item \field{selectors[2].type} VIRTIO_NET_FF_MASK_TYPE_UDP or
      VIRTIO_NET_FF_MASK_TYPE_TCP when \field{count} is more than 2.
\end{itemize}

For the command VIRTIO_ADMIN_CMD_RESOURCE_OBJ_CREATE, when creating a resource
of \field{type} VIRTIO_NET_RESOURCE_OBJ_FF_CLASSIFIER, the driver MUST set:
\begin{itemize}
\item \field{selectors[0].mask} bytes to all 1s for the \field{EtherType}
       when \field{count} is 2 or more.
\item \field{selectors[1].mask} bytes to all 1s for \field{Protocol} or \field{Next Header}
       when \field{selector[1].type} is VIRTIO_NET_FF_MASK_TYPE_IPV4 or VIRTIO_NET_FF_MASK_TYPE_IPV6,
       and when \field{count} is more than 2.
\end{itemize}

For the command VIRTIO_ADMIN_CMD_RESOURCE_OBJ_CREATE, the resource \field{type}
VIRTIO_NET_RESOURCE_OBJ_FF_RULE, if the corresponding classifier object's
\field{count} is 2 or more, the driver MUST SET the \field{keys} bytes of
\field{EtherType} in accordance with
\hyperref[intro:IEEE 802 Ethertypes]{IEEE 802 Ethertypes}
for either VIRTIO_NET_FF_MASK_TYPE_IPV4 or VIRTIO_NET_FF_MASK_TYPE_IPV6.

For the command VIRTIO_ADMIN_CMD_RESOURCE_OBJ_CREATE, when creating a resource of
\field{type} VIRTIO_NET_RESOURCE_OBJ_FF_RULE, if the corresponding classifier
object's \field{count} is more than 2, and the \field{selector[1].type} is either
VIRTIO_NET_FF_MASK_TYPE_IPV4 or VIRTIO_NET_FF_MASK_TYPE_IPV6, the driver MUST
set the \field{keys} bytes for the \field{Protocol} or \field{Next Header}
according to \hyperref[intro:IANA Protocol Numbers]{IANA Protocol Numbers} respectively.

The driver SHOULD set all the bits for a field in the mask of a selector in both the
capability and the classifier object, unless the VIRTIO_NET_FF_MASK_F_PARTIAL_MASK
is enabled.

\subsubsection{Legacy Interface: Framing Requirements}\label{sec:Device
Types / Network Device / Legacy Interface: Framing Requirements}

When using legacy interfaces, transitional drivers which have not
negotiated VIRTIO_F_ANY_LAYOUT MUST use a single descriptor for the
\field{struct virtio_net_hdr} on both transmit and receive, with the
network data in the following descriptors.

Additionally, when using the control virtqueue (see \ref{sec:Device
Types / Network Device / Device Operation / Control Virtqueue})
, transitional drivers which have not
negotiated VIRTIO_F_ANY_LAYOUT MUST:
\begin{itemize}
\item for all commands, use a single 2-byte descriptor including the first two
fields: \field{class} and \field{command}
\item for all commands except VIRTIO_NET_CTRL_MAC_TABLE_SET
use a single descriptor including command-specific-data
with no padding.
\item for the VIRTIO_NET_CTRL_MAC_TABLE_SET command use exactly
two descriptors including command-specific-data with no padding:
the first of these descriptors MUST include the
virtio_net_ctrl_mac table structure for the unicast addresses with no padding,
the second of these descriptors MUST include the
virtio_net_ctrl_mac table structure for the multicast addresses
with no padding.
\item for all commands, use a single 1-byte descriptor for the
\field{ack} field
\end{itemize}

See \ref{sec:Basic
Facilities of a Virtio Device / Virtqueues / Message Framing}.

\section{Network Device}\label{sec:Device Types / Network Device}

The virtio network device is a virtual network interface controller.
It consists of a virtual Ethernet link which connects the device
to the Ethernet network. The device has transmit and receive
queues. The driver adds empty buffers to the receive virtqueue.
The device receives incoming packets from the link; the device
places these incoming packets in the receive virtqueue buffers.
The driver adds outgoing packets to the transmit virtqueue. The device
removes these packets from the transmit virtqueue and sends them to
the link. The device may have a control virtqueue. The driver
uses the control virtqueue to dynamically manipulate various
features of the initialized device.

\subsection{Device ID}\label{sec:Device Types / Network Device / Device ID}

 1

\subsection{Virtqueues}\label{sec:Device Types / Network Device / Virtqueues}

\begin{description}
\item[0] receiveq1
\item[1] transmitq1
\item[\ldots]
\item[2(N-1)] receiveqN
\item[2(N-1)+1] transmitqN
\item[2N] controlq
\end{description}

 N=1 if neither VIRTIO_NET_F_MQ nor VIRTIO_NET_F_RSS are negotiated, otherwise N is set by
 \field{max_virtqueue_pairs}.

controlq is optional; it only exists if VIRTIO_NET_F_CTRL_VQ is
negotiated.

\subsection{Feature bits}\label{sec:Device Types / Network Device / Feature bits}

\begin{description}
\item[VIRTIO_NET_F_CSUM (0)] Device handles packets with partial checksum offload.

\item[VIRTIO_NET_F_GUEST_CSUM (1)] Driver handles packets with partial checksum.

\item[VIRTIO_NET_F_CTRL_GUEST_OFFLOADS (2)] Control channel offloads
        reconfiguration support.

\item[VIRTIO_NET_F_MTU(3)] Device maximum MTU reporting is supported. If
    offered by the device, device advises driver about the value of
    its maximum MTU. If negotiated, the driver uses \field{mtu} as
    the maximum MTU value.

\item[VIRTIO_NET_F_MAC (5)] Device has given MAC address.

\item[VIRTIO_NET_F_GUEST_TSO4 (7)] Driver can receive TSOv4.

\item[VIRTIO_NET_F_GUEST_TSO6 (8)] Driver can receive TSOv6.

\item[VIRTIO_NET_F_GUEST_ECN (9)] Driver can receive TSO with ECN.

\item[VIRTIO_NET_F_GUEST_UFO (10)] Driver can receive UFO.

\item[VIRTIO_NET_F_HOST_TSO4 (11)] Device can receive TSOv4.

\item[VIRTIO_NET_F_HOST_TSO6 (12)] Device can receive TSOv6.

\item[VIRTIO_NET_F_HOST_ECN (13)] Device can receive TSO with ECN.

\item[VIRTIO_NET_F_HOST_UFO (14)] Device can receive UFO.

\item[VIRTIO_NET_F_MRG_RXBUF (15)] Driver can merge receive buffers.

\item[VIRTIO_NET_F_STATUS (16)] Configuration status field is
    available.

\item[VIRTIO_NET_F_CTRL_VQ (17)] Control channel is available.

\item[VIRTIO_NET_F_CTRL_RX (18)] Control channel RX mode support.

\item[VIRTIO_NET_F_CTRL_VLAN (19)] Control channel VLAN filtering.

\item[VIRTIO_NET_F_CTRL_RX_EXTRA (20)]	Control channel RX extra mode support.

\item[VIRTIO_NET_F_GUEST_ANNOUNCE(21)] Driver can send gratuitous
    packets.

\item[VIRTIO_NET_F_MQ(22)] Device supports multiqueue with automatic
    receive steering.

\item[VIRTIO_NET_F_CTRL_MAC_ADDR(23)] Set MAC address through control
    channel.

\item[VIRTIO_NET_F_DEVICE_STATS(50)] Device can provide device-level statistics
    to the driver through the control virtqueue.

\item[VIRTIO_NET_F_HASH_TUNNEL(51)] Device supports inner header hash for encapsulated packets.

\item[VIRTIO_NET_F_VQ_NOTF_COAL(52)] Device supports virtqueue notification coalescing.

\item[VIRTIO_NET_F_NOTF_COAL(53)] Device supports notifications coalescing.

\item[VIRTIO_NET_F_GUEST_USO4 (54)] Driver can receive USOv4 packets.

\item[VIRTIO_NET_F_GUEST_USO6 (55)] Driver can receive USOv6 packets.

\item[VIRTIO_NET_F_HOST_USO (56)] Device can receive USO packets. Unlike UFO
 (fragmenting the packet) the USO splits large UDP packet
 to several segments when each of these smaller packets has UDP header.

\item[VIRTIO_NET_F_HASH_REPORT(57)] Device can report per-packet hash
    value and a type of calculated hash.

\item[VIRTIO_NET_F_GUEST_HDRLEN(59)] Driver can provide the exact \field{hdr_len}
    value. Device benefits from knowing the exact header length.

\item[VIRTIO_NET_F_RSS(60)] Device supports RSS (receive-side scaling)
    with Toeplitz hash calculation and configurable hash
    parameters for receive steering.

\item[VIRTIO_NET_F_RSC_EXT(61)] Device can process duplicated ACKs
    and report number of coalesced segments and duplicated ACKs.

\item[VIRTIO_NET_F_STANDBY(62)] Device may act as a standby for a primary
    device with the same MAC address.

\item[VIRTIO_NET_F_SPEED_DUPLEX(63)] Device reports speed and duplex.

\item[VIRTIO_NET_F_RSS_CONTEXT(64)] Device supports multiple RSS contexts.

\item[VIRTIO_NET_F_GUEST_UDP_TUNNEL_GSO (65)] Driver can receive GSO packets
  carried by a UDP tunnel.

\item[VIRTIO_NET_F_GUEST_UDP_TUNNEL_GSO_CSUM (66)] Driver handles packets
  carried by a UDP tunnel with partial csum for the outer header.

\item[VIRTIO_NET_F_HOST_UDP_TUNNEL_GSO (67)] Device can receive GSO packets
  carried by a UDP tunnel.

\item[VIRTIO_NET_F_HOST_UDP_TUNNEL_GSO_CSUM (68)] Device handles packets
  carried by a UDP tunnel with partial csum for the outer header.
\end{description}

\subsubsection{Feature bit requirements}\label{sec:Device Types / Network Device / Feature bits / Feature bit requirements}

Some networking feature bits require other networking feature bits
(see \ref{drivernormative:Basic Facilities of a Virtio Device / Feature Bits}):

\begin{description}
\item[VIRTIO_NET_F_GUEST_TSO4] Requires VIRTIO_NET_F_GUEST_CSUM.
\item[VIRTIO_NET_F_GUEST_TSO6] Requires VIRTIO_NET_F_GUEST_CSUM.
\item[VIRTIO_NET_F_GUEST_ECN] Requires VIRTIO_NET_F_GUEST_TSO4 or VIRTIO_NET_F_GUEST_TSO6.
\item[VIRTIO_NET_F_GUEST_UFO] Requires VIRTIO_NET_F_GUEST_CSUM.
\item[VIRTIO_NET_F_GUEST_USO4] Requires VIRTIO_NET_F_GUEST_CSUM.
\item[VIRTIO_NET_F_GUEST_USO6] Requires VIRTIO_NET_F_GUEST_CSUM.
\item[VIRTIO_NET_F_GUEST_UDP_TUNNEL_GSO] Requires VIRTIO_NET_F_GUEST_TSO4, VIRTIO_NET_F_GUEST_TSO6,
   VIRTIO_NET_F_GUEST_USO4 and VIRTIO_NET_F_GUEST_USO6.
\item[VIRTIO_NET_F_GUEST_UDP_TUNNEL_GSO_CSUM] Requires VIRTIO_NET_F_GUEST_UDP_TUNNEL_GSO

\item[VIRTIO_NET_F_HOST_TSO4] Requires VIRTIO_NET_F_CSUM.
\item[VIRTIO_NET_F_HOST_TSO6] Requires VIRTIO_NET_F_CSUM.
\item[VIRTIO_NET_F_HOST_ECN] Requires VIRTIO_NET_F_HOST_TSO4 or VIRTIO_NET_F_HOST_TSO6.
\item[VIRTIO_NET_F_HOST_UFO] Requires VIRTIO_NET_F_CSUM.
\item[VIRTIO_NET_F_HOST_USO] Requires VIRTIO_NET_F_CSUM.
\item[VIRTIO_NET_F_HOST_UDP_TUNNEL_GSO] Requires VIRTIO_NET_F_HOST_TSO4, VIRTIO_NET_F_HOST_TSO6
   and VIRTIO_NET_F_HOST_USO.
\item[VIRTIO_NET_F_HOST_UDP_TUNNEL_GSO_CSUM] Requires VIRTIO_NET_F_HOST_UDP_TUNNEL_GSO

\item[VIRTIO_NET_F_CTRL_RX] Requires VIRTIO_NET_F_CTRL_VQ.
\item[VIRTIO_NET_F_CTRL_VLAN] Requires VIRTIO_NET_F_CTRL_VQ.
\item[VIRTIO_NET_F_GUEST_ANNOUNCE] Requires VIRTIO_NET_F_CTRL_VQ.
\item[VIRTIO_NET_F_MQ] Requires VIRTIO_NET_F_CTRL_VQ.
\item[VIRTIO_NET_F_CTRL_MAC_ADDR] Requires VIRTIO_NET_F_CTRL_VQ.
\item[VIRTIO_NET_F_NOTF_COAL] Requires VIRTIO_NET_F_CTRL_VQ.
\item[VIRTIO_NET_F_RSC_EXT] Requires VIRTIO_NET_F_HOST_TSO4 or VIRTIO_NET_F_HOST_TSO6.
\item[VIRTIO_NET_F_RSS] Requires VIRTIO_NET_F_CTRL_VQ.
\item[VIRTIO_NET_F_VQ_NOTF_COAL] Requires VIRTIO_NET_F_CTRL_VQ.
\item[VIRTIO_NET_F_HASH_TUNNEL] Requires VIRTIO_NET_F_CTRL_VQ along with VIRTIO_NET_F_RSS or VIRTIO_NET_F_HASH_REPORT.
\item[VIRTIO_NET_F_RSS_CONTEXT] Requires VIRTIO_NET_F_CTRL_VQ and VIRTIO_NET_F_RSS.
\end{description}

\begin{note}
The dependency between UDP_TUNNEL_GSO_CSUM and UDP_TUNNEL_GSO is intentionally
in the opposite direction with respect to the plain GSO features and the plain
checksum offload because UDP tunnel checksum offload gives very little gain
for non GSO packets and is quite complex to implement in H/W.
\end{note}

\subsubsection{Legacy Interface: Feature bits}\label{sec:Device Types / Network Device / Feature bits / Legacy Interface: Feature bits}
\begin{description}
\item[VIRTIO_NET_F_GSO (6)] Device handles packets with any GSO type. This was supposed to indicate segmentation offload support, but
upon further investigation it became clear that multiple bits were needed.
\item[VIRTIO_NET_F_GUEST_RSC4 (41)] Device coalesces TCPIP v4 packets. This was implemented by hypervisor patch for certification
purposes and current Windows driver depends on it. It will not function if virtio-net device reports this feature.
\item[VIRTIO_NET_F_GUEST_RSC6 (42)] Device coalesces TCPIP v6 packets. Similar to VIRTIO_NET_F_GUEST_RSC4.
\end{description}

\subsection{Device configuration layout}\label{sec:Device Types / Network Device / Device configuration layout}
\label{sec:Device Types / Block Device / Feature bits / Device configuration layout}

The network device has the following device configuration layout.
All of the device configuration fields are read-only for the driver.

\begin{lstlisting}
struct virtio_net_config {
        u8 mac[6];
        le16 status;
        le16 max_virtqueue_pairs;
        le16 mtu;
        le32 speed;
        u8 duplex;
        u8 rss_max_key_size;
        le16 rss_max_indirection_table_length;
        le32 supported_hash_types;
        le32 supported_tunnel_types;
};
\end{lstlisting}

The \field{mac} address field always exists (although it is only
valid if VIRTIO_NET_F_MAC is set).

The \field{status} only exists if VIRTIO_NET_F_STATUS is set.
Two bits are currently defined for the status field: VIRTIO_NET_S_LINK_UP
and VIRTIO_NET_S_ANNOUNCE.

\begin{lstlisting}
#define VIRTIO_NET_S_LINK_UP     1
#define VIRTIO_NET_S_ANNOUNCE    2
\end{lstlisting}

The following field, \field{max_virtqueue_pairs} only exists if
VIRTIO_NET_F_MQ or VIRTIO_NET_F_RSS is set. This field specifies the maximum number
of each of transmit and receive virtqueues (receiveq1\ldots receiveqN
and transmitq1\ldots transmitqN respectively) that can be configured once at least one of these features
is negotiated.

The following field, \field{mtu} only exists if VIRTIO_NET_F_MTU
is set. This field specifies the maximum MTU for the driver to
use.

The following two fields, \field{speed} and \field{duplex}, only
exist if VIRTIO_NET_F_SPEED_DUPLEX is set.

\field{speed} contains the device speed, in units of 1 MBit per
second, 0 to 0x7fffffff, or 0xffffffff for unknown speed.

\field{duplex} has the values of 0x01 for full duplex, 0x00 for
half duplex and 0xff for unknown duplex state.

Both \field{speed} and \field{duplex} can change, thus the driver
is expected to re-read these values after receiving a
configuration change notification.

The following field, \field{rss_max_key_size} only exists if VIRTIO_NET_F_RSS or VIRTIO_NET_F_HASH_REPORT is set.
It specifies the maximum supported length of RSS key in bytes.

The following field, \field{rss_max_indirection_table_length} only exists if VIRTIO_NET_F_RSS is set.
It specifies the maximum number of 16-bit entries in RSS indirection table.

The next field, \field{supported_hash_types} only exists if the device supports hash calculation,
i.e. if VIRTIO_NET_F_RSS or VIRTIO_NET_F_HASH_REPORT is set.

Field \field{supported_hash_types} contains the bitmask of supported hash types.
See \ref{sec:Device Types / Network Device / Device Operation / Processing of Incoming Packets / Hash calculation for incoming packets / Supported/enabled hash types} for details of supported hash types.

Field \field{supported_tunnel_types} only exists if the device supports inner header hash, i.e. if VIRTIO_NET_F_HASH_TUNNEL is set.

Field \field{supported_tunnel_types} contains the bitmask of encapsulation types supported by the device for inner header hash.
Encapsulation types are defined in \ref{sec:Device Types / Network Device / Device Operation / Processing of Incoming Packets /
Hash calculation for incoming packets / Encapsulation types supported/enabled for inner header hash}.

\devicenormative{\subsubsection}{Device configuration layout}{Device Types / Network Device / Device configuration layout}

The device MUST set \field{max_virtqueue_pairs} to between 1 and 0x8000 inclusive,
if it offers VIRTIO_NET_F_MQ.

The device MUST set \field{mtu} to between 68 and 65535 inclusive,
if it offers VIRTIO_NET_F_MTU.

The device SHOULD set \field{mtu} to at least 1280, if it offers
VIRTIO_NET_F_MTU.

The device MUST NOT modify \field{mtu} once it has been set.

The device MUST NOT pass received packets that exceed \field{mtu} (plus low
level ethernet header length) size with \field{gso_type} NONE or ECN
after VIRTIO_NET_F_MTU has been successfully negotiated.

The device MUST forward transmitted packets of up to \field{mtu} (plus low
level ethernet header length) size with \field{gso_type} NONE or ECN, and do
so without fragmentation, after VIRTIO_NET_F_MTU has been successfully
negotiated.

The device MUST set \field{rss_max_key_size} to at least 40, if it offers
VIRTIO_NET_F_RSS or VIRTIO_NET_F_HASH_REPORT.

The device MUST set \field{rss_max_indirection_table_length} to at least 128, if it offers
VIRTIO_NET_F_RSS.

If the driver negotiates the VIRTIO_NET_F_STANDBY feature, the device MAY act
as a standby device for a primary device with the same MAC address.

If VIRTIO_NET_F_SPEED_DUPLEX has been negotiated, \field{speed}
MUST contain the device speed, in units of 1 MBit per second, 0 to
0x7ffffffff, or 0xfffffffff for unknown.

If VIRTIO_NET_F_SPEED_DUPLEX has been negotiated, \field{duplex}
MUST have the values of 0x00 for full duplex, 0x01 for half
duplex, or 0xff for unknown.

If VIRTIO_NET_F_SPEED_DUPLEX and VIRTIO_NET_F_STATUS have both
been negotiated, the device SHOULD NOT change the \field{speed} and
\field{duplex} fields as long as VIRTIO_NET_S_LINK_UP is set in
the \field{status}.

The device SHOULD NOT offer VIRTIO_NET_F_HASH_REPORT if it
does not offer VIRTIO_NET_F_CTRL_VQ.

The device SHOULD NOT offer VIRTIO_NET_F_CTRL_RX_EXTRA if it
does not offer VIRTIO_NET_F_CTRL_VQ.

\drivernormative{\subsubsection}{Device configuration layout}{Device Types / Network Device / Device configuration layout}

The driver MUST NOT write to any of the device configuration fields.

A driver SHOULD negotiate VIRTIO_NET_F_MAC if the device offers it.
If the driver negotiates the VIRTIO_NET_F_MAC feature, the driver MUST set
the physical address of the NIC to \field{mac}.  Otherwise, it SHOULD
use a locally-administered MAC address (see \hyperref[intro:IEEE 802]{IEEE 802},
``9.2 48-bit universal LAN MAC addresses'').

If the driver does not negotiate the VIRTIO_NET_F_STATUS feature, it SHOULD
assume the link is active, otherwise it SHOULD read the link status from
the bottom bit of \field{status}.

A driver SHOULD negotiate VIRTIO_NET_F_MTU if the device offers it.

If the driver negotiates VIRTIO_NET_F_MTU, it MUST supply enough receive
buffers to receive at least one receive packet of size \field{mtu} (plus low
level ethernet header length) with \field{gso_type} NONE or ECN.

If the driver negotiates VIRTIO_NET_F_MTU, it MUST NOT transmit packets of
size exceeding the value of \field{mtu} (plus low level ethernet header length)
with \field{gso_type} NONE or ECN.

A driver SHOULD negotiate the VIRTIO_NET_F_STANDBY feature if the device offers it.

If VIRTIO_NET_F_SPEED_DUPLEX has been negotiated,
the driver MUST treat any value of \field{speed} above
0x7fffffff as well as any value of \field{duplex} not
matching 0x00 or 0x01 as an unknown value.

If VIRTIO_NET_F_SPEED_DUPLEX has been negotiated, the driver
SHOULD re-read \field{speed} and \field{duplex} after a
configuration change notification.

A driver SHOULD NOT negotiate VIRTIO_NET_F_HASH_REPORT if it
does not negotiate VIRTIO_NET_F_CTRL_VQ.

A driver SHOULD NOT negotiate VIRTIO_NET_F_CTRL_RX_EXTRA if it
does not negotiate VIRTIO_NET_F_CTRL_VQ.

\subsubsection{Legacy Interface: Device configuration layout}\label{sec:Device Types / Network Device / Device configuration layout / Legacy Interface: Device configuration layout}
\label{sec:Device Types / Block Device / Feature bits / Device configuration layout / Legacy Interface: Device configuration layout}
When using the legacy interface, transitional devices and drivers
MUST format \field{status} and
\field{max_virtqueue_pairs} in struct virtio_net_config
according to the native endian of the guest rather than
(necessarily when not using the legacy interface) little-endian.

When using the legacy interface, \field{mac} is driver-writable
which provided a way for drivers to update the MAC without
negotiating VIRTIO_NET_F_CTRL_MAC_ADDR.

\subsection{Device Initialization}\label{sec:Device Types / Network Device / Device Initialization}

A driver would perform a typical initialization routine like so:

\begin{enumerate}
\item Identify and initialize the receive and
  transmission virtqueues, up to N of each kind. If
  VIRTIO_NET_F_MQ feature bit is negotiated,
  N=\field{max_virtqueue_pairs}, otherwise identify N=1.

\item If the VIRTIO_NET_F_CTRL_VQ feature bit is negotiated,
  identify the control virtqueue.

\item Fill the receive queues with buffers: see \ref{sec:Device Types / Network Device / Device Operation / Setting Up Receive Buffers}.

\item Even with VIRTIO_NET_F_MQ, only receiveq1, transmitq1 and
  controlq are used by default.  The driver would send the
  VIRTIO_NET_CTRL_MQ_VQ_PAIRS_SET command specifying the
  number of the transmit and receive queues to use.

\item If the VIRTIO_NET_F_MAC feature bit is set, the configuration
  space \field{mac} entry indicates the ``physical'' address of the
  device, otherwise the driver would typically generate a random
  local MAC address.

\item If the VIRTIO_NET_F_STATUS feature bit is negotiated, the link
  status comes from the bottom bit of \field{status}.
  Otherwise, the driver assumes it's active.

\item A performant driver would indicate that it will generate checksumless
  packets by negotiating the VIRTIO_NET_F_CSUM feature.

\item If that feature is negotiated, a driver can use TCP segmentation or UDP
  segmentation/fragmentation offload by negotiating the VIRTIO_NET_F_HOST_TSO4 (IPv4
  TCP), VIRTIO_NET_F_HOST_TSO6 (IPv6 TCP), VIRTIO_NET_F_HOST_UFO
  (UDP fragmentation) and VIRTIO_NET_F_HOST_USO (UDP segmentation) features.

\item If the VIRTIO_NET_F_HOST_TSO6, VIRTIO_NET_F_HOST_TSO4 and VIRTIO_NET_F_HOST_USO
  segmentation features are negotiated, a driver can
  use TCP segmentation or UDP segmentation on top of UDP encapsulation
  offload, when the outer header does not require checksumming - e.g.
  the outer UDP checksum is zero - by negotiating the
  VIRTIO_NET_F_HOST_UDP_TUNNEL_GSO feature.
  GSO over UDP tunnels packets carry two sets of headers: the outer ones
  and the inner ones. The outer transport protocol is UDP, the inner
  could be either TCP or UDP. Only a single level of encapsulation
  offload is supported.

\item If VIRTIO_NET_F_HOST_UDP_TUNNEL_GSO is negotiated, a driver can
  additionally use TCP segmentation or UDP segmentation on top of UDP
  encapsulation with the outer header requiring checksum offload,
  negotiating the VIRTIO_NET_F_HOST_UDP_TUNNEL_GSO_CSUM feature.

\item The converse features are also available: a driver can save
  the virtual device some work by negotiating these features.\note{For example, a network packet transported between two guests on
the same system might not need checksumming at all, nor segmentation,
if both guests are amenable.}
   The VIRTIO_NET_F_GUEST_CSUM feature indicates that partially
  checksummed packets can be received, and if it can do that then
  the VIRTIO_NET_F_GUEST_TSO4, VIRTIO_NET_F_GUEST_TSO6,
  VIRTIO_NET_F_GUEST_UFO, VIRTIO_NET_F_GUEST_ECN, VIRTIO_NET_F_GUEST_USO4,
  VIRTIO_NET_F_GUEST_USO6 VIRTIO_NET_F_GUEST_UDP_TUNNEL_GSO and
  VIRTIO_NET_F_GUEST_UDP_TUNNEL_GSO_CSUM are the input equivalents of
  the features described above.
  See \ref{sec:Device Types / Network Device / Device Operation /
Setting Up Receive Buffers}~\nameref{sec:Device Types / Network
Device / Device Operation / Setting Up Receive Buffers} and
\ref{sec:Device Types / Network Device / Device Operation /
Processing of Incoming Packets}~\nameref{sec:Device Types /
Network Device / Device Operation / Processing of Incoming Packets} below.
\end{enumerate}

A truly minimal driver would only accept VIRTIO_NET_F_MAC and ignore
everything else.

\subsection{Device and driver capabilities}\label{sec:Device Types / Network Device / Device and driver capabilities}

The network device has the following capabilities.

\begin{tabularx}{\textwidth}{ |l||l|X| }
\hline
Identifier & Name & Description \\
\hline \hline
0x0800 & \hyperref[par:Device Types / Network Device / Device Operation / Flow filter / Device and driver capabilities / VIRTIO-NET-FF-RESOURCE-CAP]{VIRTIO_NET_FF_RESOURCE_CAP} & Flow filter resource capability \\
\hline
0x0801 & \hyperref[par:Device Types / Network Device / Device Operation / Flow filter / Device and driver capabilities / VIRTIO-NET-FF-SELECTOR-CAP]{VIRTIO_NET_FF_SELECTOR_CAP} & Flow filter classifier capability \\
\hline
0x0802 & \hyperref[par:Device Types / Network Device / Device Operation / Flow filter / Device and driver capabilities / VIRTIO-NET-FF-ACTION-CAP]{VIRTIO_NET_FF_ACTION_CAP} & Flow filter action capability \\
\hline
\end{tabularx}

\subsection{Device resource objects}\label{sec:Device Types / Network Device / Device resource objects}

The network device has the following resource objects.

\begin{tabularx}{\textwidth}{ |l||l|X| }
\hline
type & Name & Description \\
\hline \hline
0x0200 & \hyperref[par:Device Types / Network Device / Device Operation / Flow filter / Resource objects / VIRTIO-NET-RESOURCE-OBJ-FF-GROUP]{VIRTIO_NET_RESOURCE_OBJ_FF_GROUP} & Flow filter group resource object \\
\hline
0x0201 & \hyperref[par:Device Types / Network Device / Device Operation / Flow filter / Resource objects / VIRTIO-NET-RESOURCE-OBJ-FF-CLASSIFIER]{VIRTIO_NET_RESOURCE_OBJ_FF_CLASSIFIER} & Flow filter mask object \\
\hline
0x0202 & \hyperref[par:Device Types / Network Device / Device Operation / Flow filter / Resource objects / VIRTIO-NET-RESOURCE-OBJ-FF-RULE]{VIRTIO_NET_RESOURCE_OBJ_FF_RULE} & Flow filter rule object \\
\hline
\end{tabularx}

\subsection{Device parts}\label{sec:Device Types / Network Device / Device parts}

Network device parts represent the configuration done by the driver using control
virtqueue commands. Network device part is in the format of
\field{struct virtio_dev_part}.

\begin{tabularx}{\textwidth}{ |l||l|X| }
\hline
Type & Name & Description \\
\hline \hline
0x200 & VIRTIO_NET_DEV_PART_CVQ_CFG_PART & Represents device configuration done through a control virtqueue command, see \ref{sec:Device Types / Network Device / Device parts / VIRTIO-NET-DEV-PART-CVQ-CFG-PART} \\
\hline
0x201 - 0x5FF & - & reserved for future \\
\hline
\hline
\end{tabularx}

\subsubsection{VIRTIO_NET_DEV_PART_CVQ_CFG_PART}\label{sec:Device Types / Network Device / Device parts / VIRTIO-NET-DEV-PART-CVQ-CFG-PART}

For VIRTIO_NET_DEV_PART_CVQ_CFG_PART, \field{part_type} is set to 0x200. The
VIRTIO_NET_DEV_PART_CVQ_CFG_PART part indicates configuration performed by the
driver using a control virtqueue command.

\begin{lstlisting}
struct virtio_net_dev_part_cvq_selector {
        u8 class;
        u8 command;
        u8 reserved[6];
};
\end{lstlisting}

There is one device part of type VIRTIO_NET_DEV_PART_CVQ_CFG_PART for each
individual configuration. Each part is identified by a unique selector value.
The selector, \field{device_type_raw}, is in the format
\field{struct virtio_net_dev_part_cvq_selector}.

The selector consists of two fields: \field{class} and \field{command}. These
fields correspond to the \field{class} and \field{command} defined in
\field{struct virtio_net_ctrl}, as described in the relevant sections of
\ref{sec:Device Types / Network Device / Device Operation / Control Virtqueue}.

The value corresponding to each part’s selector follows the same format as the
respective \field{command-specific-data} described in the relevant sections of
\ref{sec:Device Types / Network Device / Device Operation / Control Virtqueue}.

For example, when the \field{class} is VIRTIO_NET_CTRL_MAC, the \field{command}
can be either VIRTIO_NET_CTRL_MAC_TABLE_SET or VIRTIO_NET_CTRL_MAC_ADDR_SET;
when \field{command} is set to VIRTIO_NET_CTRL_MAC_TABLE_SET, \field{value}
is in the format of \field{struct virtio_net_ctrl_mac}.

Supported selectors are listed in the table:

\begin{tabularx}{\textwidth}{ |l|X| }
\hline
Class selector & Command selector \\
\hline \hline
VIRTIO_NET_CTRL_RX & VIRTIO_NET_CTRL_RX_PROMISC \\
\hline
VIRTIO_NET_CTRL_RX & VIRTIO_NET_CTRL_RX_ALLMULTI \\
\hline
VIRTIO_NET_CTRL_RX & VIRTIO_NET_CTRL_RX_ALLUNI \\
\hline
VIRTIO_NET_CTRL_RX & VIRTIO_NET_CTRL_RX_NOMULTI \\
\hline
VIRTIO_NET_CTRL_RX & VIRTIO_NET_CTRL_RX_NOUNI \\
\hline
VIRTIO_NET_CTRL_RX & VIRTIO_NET_CTRL_RX_NOBCAST \\
\hline
VIRTIO_NET_CTRL_MAC & VIRTIO_NET_CTRL_MAC_TABLE_SET \\
\hline
VIRTIO_NET_CTRL_MAC & VIRTIO_NET_CTRL_MAC_ADDR_SET \\
\hline
VIRTIO_NET_CTRL_VLAN & VIRTIO_NET_CTRL_VLAN_ADD \\
\hline
VIRTIO_NET_CTRL_ANNOUNCE & VIRTIO_NET_CTRL_ANNOUNCE_ACK \\
\hline
VIRTIO_NET_CTRL_MQ & VIRTIO_NET_CTRL_MQ_VQ_PAIRS_SET \\
\hline
VIRTIO_NET_CTRL_MQ & VIRTIO_NET_CTRL_MQ_RSS_CONFIG \\
\hline
VIRTIO_NET_CTRL_MQ & VIRTIO_NET_CTRL_MQ_HASH_CONFIG \\
\hline
\hline
\end{tabularx}

For command selector VIRTIO_NET_CTRL_VLAN_ADD, device part consists of a whole
VLAN table.

\field{reserved} is reserved and set to zero.

\subsection{Device Operation}\label{sec:Device Types / Network Device / Device Operation}

Packets are transmitted by placing them in the
transmitq1\ldots transmitqN, and buffers for incoming packets are
placed in the receiveq1\ldots receiveqN. In each case, the packet
itself is preceded by a header:

\begin{lstlisting}
struct virtio_net_hdr {
#define VIRTIO_NET_HDR_F_NEEDS_CSUM    1
#define VIRTIO_NET_HDR_F_DATA_VALID    2
#define VIRTIO_NET_HDR_F_RSC_INFO      4
#define VIRTIO_NET_HDR_F_UDP_TUNNEL_CSUM 8
        u8 flags;
#define VIRTIO_NET_HDR_GSO_NONE        0
#define VIRTIO_NET_HDR_GSO_TCPV4       1
#define VIRTIO_NET_HDR_GSO_UDP         3
#define VIRTIO_NET_HDR_GSO_TCPV6       4
#define VIRTIO_NET_HDR_GSO_UDP_L4      5
#define VIRTIO_NET_HDR_GSO_UDP_TUNNEL_IPV4 0x20
#define VIRTIO_NET_HDR_GSO_UDP_TUNNEL_IPV6 0x40
#define VIRTIO_NET_HDR_GSO_ECN      0x80
        u8 gso_type;
        le16 hdr_len;
        le16 gso_size;
        le16 csum_start;
        le16 csum_offset;
        le16 num_buffers;
        le32 hash_value;        (Only if VIRTIO_NET_F_HASH_REPORT negotiated)
        le16 hash_report;       (Only if VIRTIO_NET_F_HASH_REPORT negotiated)
        le16 padding_reserved;  (Only if VIRTIO_NET_F_HASH_REPORT negotiated)
        le16 outer_th_offset    (Only if VIRTIO_NET_F_HOST_UDP_TUNNEL_GSO or VIRTIO_NET_F_GUEST_UDP_TUNNEL_GSO negotiated)
        le16 inner_nh_offset;   (Only if VIRTIO_NET_F_HOST_UDP_TUNNEL_GSO or VIRTIO_NET_F_GUEST_UDP_TUNNEL_GSO negotiated)
};
\end{lstlisting}

The controlq is used to control various device features described further in
section \ref{sec:Device Types / Network Device / Device Operation / Control Virtqueue}.

\subsubsection{Legacy Interface: Device Operation}\label{sec:Device Types / Network Device / Device Operation / Legacy Interface: Device Operation}
When using the legacy interface, transitional devices and drivers
MUST format the fields in \field{struct virtio_net_hdr}
according to the native endian of the guest rather than
(necessarily when not using the legacy interface) little-endian.

The legacy driver only presented \field{num_buffers} in the \field{struct virtio_net_hdr}
when VIRTIO_NET_F_MRG_RXBUF was negotiated; without that feature the
structure was 2 bytes shorter.

When using the legacy interface, the driver SHOULD ignore the
used length for the transmit queues
and the controlq queue.
\begin{note}
Historically, some devices put
the total descriptor length there, even though no data was
actually written.
\end{note}

\subsubsection{Packet Transmission}\label{sec:Device Types / Network Device / Device Operation / Packet Transmission}

Transmitting a single packet is simple, but varies depending on
the different features the driver negotiated.

\begin{enumerate}
\item The driver can send a completely checksummed packet.  In this case,
  \field{flags} will be zero, and \field{gso_type} will be VIRTIO_NET_HDR_GSO_NONE.

\item If the driver negotiated VIRTIO_NET_F_CSUM, it can skip
  checksumming the packet:
  \begin{itemize}
  \item \field{flags} has the VIRTIO_NET_HDR_F_NEEDS_CSUM set,

  \item \field{csum_start} is set to the offset within the packet to begin checksumming,
    and

  \item \field{csum_offset} indicates how many bytes after the csum_start the
    new (16 bit ones' complement) checksum is placed by the device.

  \item The TCP checksum field in the packet is set to the sum
    of the TCP pseudo header, so that replacing it by the ones'
    complement checksum of the TCP header and body will give the
    correct result.
  \end{itemize}

\begin{note}
For example, consider a partially checksummed TCP (IPv4) packet.
It will have a 14 byte ethernet header and 20 byte IP header
followed by the TCP header (with the TCP checksum field 16 bytes
into that header). \field{csum_start} will be 14+20 = 34 (the TCP
checksum includes the header), and \field{csum_offset} will be 16.
If the given packet has the VIRTIO_NET_HDR_GSO_UDP_TUNNEL_IPV4 bit or the
VIRTIO_NET_HDR_GSO_UDP_TUNNEL_IPV6 bit set,
the above checksum fields refer to the inner header checksum, see
the example below.
\end{note}

\item If the driver negotiated
  VIRTIO_NET_F_HOST_TSO4, TSO6, USO or UFO, and the packet requires
  TCP segmentation, UDP segmentation or fragmentation, then \field{gso_type}
  is set to VIRTIO_NET_HDR_GSO_TCPV4, TCPV6, UDP_L4 or UDP.
  (Otherwise, it is set to VIRTIO_NET_HDR_GSO_NONE). In this
  case, packets larger than 1514 bytes can be transmitted: the
  metadata indicates how to replicate the packet header to cut it
  into smaller packets. The other gso fields are set:

  \begin{itemize}
  \item If the VIRTIO_NET_F_GUEST_HDRLEN feature has been negotiated,
    \field{hdr_len} indicates the header length that needs to be replicated
    for each packet. It's the number of bytes from the beginning of the packet
    to the beginning of the transport payload.
    If the \field{gso_type} has the VIRTIO_NET_HDR_GSO_UDP_TUNNEL_IPV4 bit or
    VIRTIO_NET_HDR_GSO_UDP_TUNNEL_IPV6 bit set, \field{hdr_len} accounts for
    all the headers up to and including the inner transport.
    Otherwise, if the VIRTIO_NET_F_GUEST_HDRLEN feature has not been negotiated,
    \field{hdr_len} is a hint to the device as to how much of the header
    needs to be kept to copy into each packet, usually set to the
    length of the headers, including the transport header\footnote{Due to various bugs in implementations, this field is not useful
as a guarantee of the transport header size.
}.

  \begin{note}
  Some devices benefit from knowledge of the exact header length.
  \end{note}

  \item \field{gso_size} is the maximum size of each packet beyond that
    header (ie. MSS).

  \item If the driver negotiated the VIRTIO_NET_F_HOST_ECN feature,
    the VIRTIO_NET_HDR_GSO_ECN bit in \field{gso_type}
    indicates that the TCP packet has the ECN bit set\footnote{This case is not handled by some older hardware, so is called out
specifically in the protocol.}.
   \end{itemize}

\item If the driver negotiated the VIRTIO_NET_F_HOST_UDP_TUNNEL_GSO feature and the
  \field{gso_type} has the VIRTIO_NET_HDR_GSO_UDP_TUNNEL_IPV4 bit or
  VIRTIO_NET_HDR_GSO_UDP_TUNNEL_IPV6 bit set, the GSO protocol is encapsulated
  in a UDP tunnel.
  If the outer UDP header requires checksumming, the driver must have
  additionally negotiated the VIRTIO_NET_F_HOST_UDP_TUNNEL_GSO_CSUM feature
  and offloaded the outer checksum accordingly, otherwise
  the outer UDP header must not require checksum validation, i.e. the outer
  UDP checksum must be positive zero (0x0) as defined in UDP RFC 768.
  The other tunnel-related fields indicate how to replicate the packet
  headers to cut it into smaller packets:

  \begin{itemize}
  \item \field{outer_th_offset} field indicates the outer transport header within
      the packet. This field differs from \field{csum_start} as the latter
      points to the inner transport header within the packet.

  \item \field{inner_nh_offset} field indicates the inner network header within
      the packet.
  \end{itemize}

\begin{note}
For example, consider a partially checksummed TCP (IPv4) packet carried over a
Geneve UDP tunnel (again IPv4) with no tunnel options. The
only relevant variable related to the tunnel type is the tunnel header length.
The packet will have a 14 byte outer ethernet header, 20 byte outer IP header
followed by the 8 byte UDP header (with a 0 checksum value), 8 byte Geneve header,
14 byte inner ethernet header, 20 byte inner IP header
and the TCP header (with the TCP checksum field 16 bytes
into that header). \field{csum_start} will be 14+20+8+8+14+20 = 84 (the TCP
checksum includes the header), \field{csum_offset} will be 16.
\field{inner_nh_offset} will be 14+20+8+8+14 = 62, \field{outer_th_offset} will be
14+20+8 = 42 and \field{gso_type} will be
VIRTIO_NET_HDR_GSO_TCPV4 | VIRTIO_NET_HDR_GSO_UDP_TUNNEL_IPV4 = 0x21
\end{note}

\item If the driver negotiated the VIRTIO_NET_F_HOST_UDP_TUNNEL_GSO_CSUM feature,
  the transmitted packet is a GSO one encapsulated in a UDP tunnel, and
  the outer UDP header requires checksumming, the driver can skip checksumming
  the outer header:

  \begin{itemize}
  \item \field{flags} has the VIRTIO_NET_HDR_F_UDP_TUNNEL_CSUM set,

  \item The outer UDP checksum field in the packet is set to the sum
    of the UDP pseudo header, so that replacing it by the ones'
    complement checksum of the outer UDP header and payload will give the
    correct result.
  \end{itemize}

\item \field{num_buffers} is set to zero.  This field is unused on transmitted packets.

\item The header and packet are added as one output descriptor to the
  transmitq, and the device is notified of the new entry
  (see \ref{sec:Device Types / Network Device / Device Initialization}~\nameref{sec:Device Types / Network Device / Device Initialization}).
\end{enumerate}

\drivernormative{\paragraph}{Packet Transmission}{Device Types / Network Device / Device Operation / Packet Transmission}

For the transmit packet buffer, the driver MUST use the size of the
structure \field{struct virtio_net_hdr} same as the receive packet buffer.

The driver MUST set \field{num_buffers} to zero.

If VIRTIO_NET_F_CSUM is not negotiated, the driver MUST set
\field{flags} to zero and SHOULD supply a fully checksummed
packet to the device.

If VIRTIO_NET_F_HOST_TSO4 is negotiated, the driver MAY set
\field{gso_type} to VIRTIO_NET_HDR_GSO_TCPV4 to request TCPv4
segmentation, otherwise the driver MUST NOT set
\field{gso_type} to VIRTIO_NET_HDR_GSO_TCPV4.

If VIRTIO_NET_F_HOST_TSO6 is negotiated, the driver MAY set
\field{gso_type} to VIRTIO_NET_HDR_GSO_TCPV6 to request TCPv6
segmentation, otherwise the driver MUST NOT set
\field{gso_type} to VIRTIO_NET_HDR_GSO_TCPV6.

If VIRTIO_NET_F_HOST_UFO is negotiated, the driver MAY set
\field{gso_type} to VIRTIO_NET_HDR_GSO_UDP to request UDP
fragmentation, otherwise the driver MUST NOT set
\field{gso_type} to VIRTIO_NET_HDR_GSO_UDP.

If VIRTIO_NET_F_HOST_USO is negotiated, the driver MAY set
\field{gso_type} to VIRTIO_NET_HDR_GSO_UDP_L4 to request UDP
segmentation, otherwise the driver MUST NOT set
\field{gso_type} to VIRTIO_NET_HDR_GSO_UDP_L4.

The driver SHOULD NOT send to the device TCP packets requiring segmentation offload
which have the Explicit Congestion Notification bit set, unless the
VIRTIO_NET_F_HOST_ECN feature is negotiated, in which case the
driver MUST set the VIRTIO_NET_HDR_GSO_ECN bit in
\field{gso_type}.

If VIRTIO_NET_F_HOST_UDP_TUNNEL_GSO is negotiated, the driver MAY set
VIRTIO_NET_HDR_GSO_UDP_TUNNEL_IPV4 bit or the VIRTIO_NET_HDR_GSO_UDP_TUNNEL_IPV6 bit
in \field{gso_type} according to the inner network header protocol type
to request GSO packets over UDPv4 or UDPv6 tunnel segmentation,
otherwise the driver MUST NOT set either the
VIRTIO_NET_HDR_GSO_UDP_TUNNEL_IPV4 bit or the VIRTIO_NET_HDR_GSO_UDP_TUNNEL_IPV6 bit
in \field{gso_type}.

When requesting GSO segmentation over UDP tunnel, the driver MUST SET the
VIRTIO_NET_HDR_GSO_UDP_TUNNEL_IPV4 bit if the inner network header is IPv4, i.e. the
packet is a TCPv4 GSO one, otherwise, if the inner network header is IPv6, the driver
MUST SET the VIRTIO_NET_HDR_GSO_UDP_TUNNEL_IPV6 bit.

The driver MUST NOT send to the device GSO packets over UDP tunnel
requiring segmentation and outer UDP checksum offload, unless both the
VIRTIO_NET_F_HOST_UDP_TUNNEL_GSO and VIRTIO_NET_F_HOST_UDP_TUNNEL_GSO_CSUM features
are negotiated, in which case the driver MUST set either the
VIRTIO_NET_HDR_GSO_UDP_TUNNEL_IPV4 bit or the VIRTIO_NET_HDR_GSO_UDP_TUNNEL_IPV6
bit in the \field{gso_type} and the VIRTIO_NET_HDR_F_UDP_TUNNEL_CSUM bit in
the \field{flags}.

If VIRTIO_NET_F_HOST_UDP_TUNNEL_GSO_CSUM is not negotiated, the driver MUST not set
the VIRTIO_NET_HDR_F_UDP_TUNNEL_CSUM bit in the \field{flags} and
MUST NOT send to the device GSO packets over UDP tunnel
requiring segmentation and outer UDP checksum offload.

The driver MUST NOT set the VIRTIO_NET_HDR_GSO_UDP_TUNNEL_IPV4 bit or the
VIRTIO_NET_HDR_GSO_UDP_TUNNEL_IPV6 bit together with VIRTIO_NET_HDR_GSO_UDP, as the
latter is deprecated in favor of UDP_L4 and no new feature will support it.

The driver MUST NOT set the VIRTIO_NET_HDR_GSO_UDP_TUNNEL_IPV4 bit and the
VIRTIO_NET_HDR_GSO_UDP_TUNNEL_IPV6 bit together.

The driver MUST NOT set the VIRTIO_NET_HDR_F_UDP_TUNNEL_CSUM bit \field{flags}
without setting either the VIRTIO_NET_HDR_GSO_UDP_TUNNEL_IPV4 bit or
the VIRTIO_NET_HDR_GSO_UDP_TUNNEL_IPV6 bit in \field{gso_type}.

If the VIRTIO_NET_F_CSUM feature has been negotiated, the
driver MAY set the VIRTIO_NET_HDR_F_NEEDS_CSUM bit in
\field{flags}, if so:
\begin{enumerate}
\item the driver MUST validate the packet checksum at
	offset \field{csum_offset} from \field{csum_start} as well as all
	preceding offsets;
\begin{note}
If \field{gso_type} differs from VIRTIO_NET_HDR_GSO_NONE and the
VIRTIO_NET_HDR_GSO_UDP_TUNNEL_IPV4 bit or the VIRTIO_NET_HDR_GSO_UDP_TUNNEL_IPV6
bit are not set in \field{gso_type}, \field{csum_offset}
points to the only transport header present in the packet, and there are no
additional preceding checksums validated by VIRTIO_NET_HDR_F_NEEDS_CSUM.
\end{note}
\item the driver MUST set the packet checksum stored in the
	buffer to the TCP/UDP pseudo header;
\item the driver MUST set \field{csum_start} and
	\field{csum_offset} such that calculating a ones'
	complement checksum from \field{csum_start} up until the end of
	the packet and storing the result at offset \field{csum_offset}
	from  \field{csum_start} will result in a fully checksummed
	packet;
\end{enumerate}

If none of the VIRTIO_NET_F_HOST_TSO4, TSO6, USO or UFO options have
been negotiated, the driver MUST set \field{gso_type} to
VIRTIO_NET_HDR_GSO_NONE.

If \field{gso_type} differs from VIRTIO_NET_HDR_GSO_NONE, then
the driver MUST also set the VIRTIO_NET_HDR_F_NEEDS_CSUM bit in
\field{flags} and MUST set \field{gso_size} to indicate the
desired MSS.

If one of the VIRTIO_NET_F_HOST_TSO4, TSO6, USO or UFO options have
been negotiated:
\begin{itemize}
\item If the VIRTIO_NET_F_GUEST_HDRLEN feature has been negotiated,
	and \field{gso_type} differs from VIRTIO_NET_HDR_GSO_NONE,
	the driver MUST set \field{hdr_len} to a value equal to the length
	of the headers, including the transport header. If \field{gso_type}
	has the VIRTIO_NET_HDR_GSO_UDP_TUNNEL_IPV4 bit or the
	VIRTIO_NET_HDR_GSO_UDP_TUNNEL_IPV6 bit set, \field{hdr_len} includes
	the inner transport header.

\item If the VIRTIO_NET_F_GUEST_HDRLEN feature has not been negotiated,
	or \field{gso_type} is VIRTIO_NET_HDR_GSO_NONE,
	the driver SHOULD set \field{hdr_len} to a value
	not less than the length of the headers, including the transport
	header.
\end{itemize}

If the VIRTIO_NET_F_HOST_UDP_TUNNEL_GSO option has been negotiated, the
driver MAY set the VIRTIO_NET_HDR_GSO_UDP_TUNNEL_IPV4 bit or the
VIRTIO_NET_HDR_GSO_UDP_TUNNEL_IPV6 bit in \field{gso_type}, if so:
\begin{itemize}
\item the driver MUST set \field{outer_th_offset} to the outer UDP header
  offset and \field{inner_nh_offset} to the inner network header offset.
  The \field{csum_start} and \field{csum_offset} fields point respectively
  to the inner transport header and inner transport checksum field.
\end{itemize}

If the VIRTIO_NET_F_HOST_UDP_TUNNEL_GSO_CSUM feature has been negotiated,
and the VIRTIO_NET_HDR_GSO_UDP_TUNNEL_IPV4 bit or
VIRTIO_NET_HDR_GSO_UDP_TUNNEL_IPV6 bit in \field{gso_type} are set,
the driver MAY set the VIRTIO_NET_HDR_F_UDP_TUNNEL_CSUM bit in
\field{flags}, if so the driver MUST set the packet outer UDP header checksum
to the outer UDP pseudo header checksum.

\begin{note}
calculating a ones' complement checksum from \field{outer_th_offset}
up until the end of the packet and storing the result at offset 6
from \field{outer_th_offset} will result in a fully checksummed outer UDP packet;
\end{note}

If the VIRTIO_NET_HDR_GSO_UDP_TUNNEL_IPV4 bit or the
VIRTIO_NET_HDR_GSO_UDP_TUNNEL_IPV6 bit in \field{gso_type} are set
and the VIRTIO_NET_F_HOST_UDP_TUNNEL_GSO_CSUM feature has not
been negotiated, the
outer UDP header MUST NOT require checksum validation. That is, the
outer UDP checksum value MUST be 0 or the validated complete checksum
for such header.

\begin{note}
The valid complete checksum of the outer UDP header of individual segments
can be computed by the driver prior to segmentation only if the GSO packet
size is a multiple of \field{gso_size}, because then all segments
have the same size and thus all data included in the outer UDP
checksum is the same for every segment. These pre-computed segment
length and checksum fields are different from those of the GSO
packet.
In this scenario the outer UDP header of the GSO packet must carry the
segmented UDP packet length.
\end{note}

If the VIRTIO_NET_F_HOST_UDP_TUNNEL_GSO option has not
been negotiated, the driver MUST NOT set either the VIRTIO_NET_HDR_F_GSO_UDP_TUNNEL_IPV4
bit or the VIRTIO_NET_HDR_F_GSO_UDP_TUNNEL_IPV6 in \field{gso_type}.

If the VIRTIO_NET_F_HOST_UDP_TUNNEL_GSO_CSUM option has not been negotiated,
the driver MUST NOT set the VIRTIO_NET_HDR_F_UDP_TUNNEL_CSUM bit
in \field{flags}.

The driver SHOULD accept the VIRTIO_NET_F_GUEST_HDRLEN feature if it has
been offered, and if it's able to provide the exact header length.

The driver MUST NOT set the VIRTIO_NET_HDR_F_DATA_VALID and
VIRTIO_NET_HDR_F_RSC_INFO bits in \field{flags}.

The driver MUST NOT set the VIRTIO_NET_HDR_F_DATA_VALID bit in \field{flags}
together with the VIRTIO_NET_HDR_F_GSO_UDP_TUNNEL_IPV4 bit or the
VIRTIO_NET_HDR_F_GSO_UDP_TUNNEL_IPV6 bit in \field{gso_type}.

\devicenormative{\paragraph}{Packet Transmission}{Device Types / Network Device / Device Operation / Packet Transmission}
The device MUST ignore \field{flag} bits that it does not recognize.

If VIRTIO_NET_HDR_F_NEEDS_CSUM bit in \field{flags} is not set, the
device MUST NOT use the \field{csum_start} and \field{csum_offset}.

If one of the VIRTIO_NET_F_HOST_TSO4, TSO6, USO or UFO options have
been negotiated:
\begin{itemize}
\item If the VIRTIO_NET_F_GUEST_HDRLEN feature has been negotiated,
	and \field{gso_type} differs from VIRTIO_NET_HDR_GSO_NONE,
	the device MAY use \field{hdr_len} as the transport header size.

	\begin{note}
	Caution should be taken by the implementation so as to prevent
	a malicious driver from attacking the device by setting an incorrect hdr_len.
	\end{note}

\item If the VIRTIO_NET_F_GUEST_HDRLEN feature has not been negotiated,
	or \field{gso_type} is VIRTIO_NET_HDR_GSO_NONE,
	the device MAY use \field{hdr_len} only as a hint about the
	transport header size.
	The device MUST NOT rely on \field{hdr_len} to be correct.

	\begin{note}
	This is due to various bugs in implementations.
	\end{note}
\end{itemize}

If both the VIRTIO_NET_HDR_GSO_UDP_TUNNEL_IPV4 bit and
the VIRTIO_NET_HDR_GSO_UDP_TUNNEL_IPV6 bit in in \field{gso_type} are set,
the device MUST NOT accept the packet.

If the VIRTIO_NET_HDR_GSO_UDP_TUNNEL_IPV4 bit and the VIRTIO_NET_HDR_GSO_UDP_TUNNEL_IPV6
bit in \field{gso_type} are not set, the device MUST NOT use the
\field{outer_th_offset} and \field{inner_nh_offset}.

If either the VIRTIO_NET_HDR_GSO_UDP_TUNNEL_IPV4 bit or
the VIRTIO_NET_HDR_GSO_UDP_TUNNEL_IPV6 bit in \field{gso_type} are set, and any of
the following is true:
\begin{itemize}
\item the VIRTIO_NET_HDR_F_NEEDS_CSUM is not set in \field{flags}
\item the VIRTIO_NET_HDR_F_DATA_VALID is set in \field{flags}
\item the \field{gso_type} excluding the VIRTIO_NET_HDR_GSO_UDP_TUNNEL_IPV4
bit and the VIRTIO_NET_HDR_GSO_UDP_TUNNEL_IPV6 bit is VIRTIO_NET_HDR_GSO_NONE
\end{itemize}
the device MUST NOT accept the packet.

If the VIRTIO_NET_HDR_F_UDP_TUNNEL_CSUM bit in \field{flags} is set,
and both the bits VIRTIO_NET_HDR_GSO_UDP_TUNNEL_IPV4 and
VIRTIO_NET_HDR_GSO_UDP_TUNNEL_IPV6 in \field{gso_type} are not set,
the device MOST NOT accept the packet.

If VIRTIO_NET_HDR_F_NEEDS_CSUM is not set, the device MUST NOT
rely on the packet checksum being correct.
\paragraph{Packet Transmission Interrupt}\label{sec:Device Types / Network Device / Device Operation / Packet Transmission / Packet Transmission Interrupt}

Often a driver will suppress transmission virtqueue interrupts
and check for used packets in the transmit path of following
packets.

The normal behavior in this interrupt handler is to retrieve
used buffers from the virtqueue and free the corresponding
headers and packets.

\subsubsection{Setting Up Receive Buffers}\label{sec:Device Types / Network Device / Device Operation / Setting Up Receive Buffers}

It is generally a good idea to keep the receive virtqueue as
fully populated as possible: if it runs out, network performance
will suffer.

If the VIRTIO_NET_F_GUEST_TSO4, VIRTIO_NET_F_GUEST_TSO6,
VIRTIO_NET_F_GUEST_UFO, VIRTIO_NET_F_GUEST_USO4 or VIRTIO_NET_F_GUEST_USO6
features are used, the maximum incoming packet
will be 65589 bytes long (14 bytes of Ethernet header, plus 40 bytes of
the IPv6 header, plus 65535 bytes of maximum IPv6 payload including any
extension header), otherwise 1514 bytes.
When VIRTIO_NET_F_HASH_REPORT is not negotiated, the required receive buffer
size is either 65601 or 1526 bytes accounting for 20 bytes of
\field{struct virtio_net_hdr} followed by receive packet.
When VIRTIO_NET_F_HASH_REPORT is negotiated, the required receive buffer
size is either 65609 or 1534 bytes accounting for 12 bytes of
\field{struct virtio_net_hdr} followed by receive packet.

\drivernormative{\paragraph}{Setting Up Receive Buffers}{Device Types / Network Device / Device Operation / Setting Up Receive Buffers}

\begin{itemize}
\item If VIRTIO_NET_F_MRG_RXBUF is not negotiated:
  \begin{itemize}
    \item If VIRTIO_NET_F_GUEST_TSO4, VIRTIO_NET_F_GUEST_TSO6, VIRTIO_NET_F_GUEST_UFO,
	VIRTIO_NET_F_GUEST_USO4 or VIRTIO_NET_F_GUEST_USO6 are negotiated, the driver SHOULD populate
      the receive queue(s) with buffers of at least 65609 bytes if
      VIRTIO_NET_F_HASH_REPORT is negotiated, and of at least 65601 bytes if not.
    \item Otherwise, the driver SHOULD populate the receive queue(s)
      with buffers of at least 1534 bytes if VIRTIO_NET_F_HASH_REPORT
      is negotiated, and of at least 1526 bytes if not.
  \end{itemize}
\item If VIRTIO_NET_F_MRG_RXBUF is negotiated, each buffer MUST be at
least size of \field{struct virtio_net_hdr},
i.e. 20 bytes if VIRTIO_NET_F_HASH_REPORT is negotiated, and 12 bytes if not.
\end{itemize}

\begin{note}
Obviously each buffer can be split across multiple descriptor elements.
\end{note}

When calculating the size of \field{struct virtio_net_hdr}, the driver
MUST consider all the fields inclusive up to \field{padding_reserved},
i.e. 20 bytes if VIRTIO_NET_F_HASH_REPORT is negotiated, and 12 bytes if not.

If VIRTIO_NET_F_MQ is negotiated, each of receiveq1\ldots receiveqN
that will be used SHOULD be populated with receive buffers.

\devicenormative{\paragraph}{Setting Up Receive Buffers}{Device Types / Network Device / Device Operation / Setting Up Receive Buffers}

The device MUST set \field{num_buffers} to the number of descriptors used to
hold the incoming packet.

The device MUST use only a single descriptor if VIRTIO_NET_F_MRG_RXBUF
was not negotiated.
\begin{note}
{This means that \field{num_buffers} will always be 1
if VIRTIO_NET_F_MRG_RXBUF is not negotiated.}
\end{note}

\subsubsection{Processing of Incoming Packets}\label{sec:Device Types / Network Device / Device Operation / Processing of Incoming Packets}
\label{sec:Device Types / Network Device / Device Operation / Processing of Packets}%old label for latexdiff

When a packet is copied into a buffer in the receiveq, the
optimal path is to disable further used buffer notifications for the
receiveq and process packets until no more are found, then re-enable
them.

Processing incoming packets involves:

\begin{enumerate}
\item \field{num_buffers} indicates how many descriptors
  this packet is spread over (including this one): this will
  always be 1 if VIRTIO_NET_F_MRG_RXBUF was not negotiated.
  This allows receipt of large packets without having to allocate large
  buffers: a packet that does not fit in a single buffer can flow
  over to the next buffer, and so on. In this case, there will be
  at least \field{num_buffers} used buffers in the virtqueue, and the device
  chains them together to form a single packet in a way similar to
  how it would store it in a single buffer spread over multiple
  descriptors.
  The other buffers will not begin with a \field{struct virtio_net_hdr}.

\item If
  \field{num_buffers} is one, then the entire packet will be
  contained within this buffer, immediately following the struct
  virtio_net_hdr.
\item If the VIRTIO_NET_F_GUEST_CSUM feature was negotiated, the
  VIRTIO_NET_HDR_F_DATA_VALID bit in \field{flags} can be
  set: if so, device has validated the packet checksum.
  If the VIRTIO_NET_F_GUEST_UDP_TUNNEL_GSO_CSUM feature has been negotiated,
  and the VIRTIO_NET_HDR_F_UDP_TUNNEL_CSUM bit is set in \field{flags},
  both the outer UDP checksum and the inner transport checksum
  have been validated, otherwise only one level of checksums (the outer one
  in case of tunnels) has been validated.
\end{enumerate}

Additionally, VIRTIO_NET_F_GUEST_CSUM, TSO4, TSO6, UDP, UDP_TUNNEL
and ECN features enable receive checksum, large receive offload and ECN
support which are the input equivalents of the transmit checksum,
transmit segmentation offloading and ECN features, as described
in \ref{sec:Device Types / Network Device / Device Operation /
Packet Transmission}:
\begin{enumerate}
\item If the VIRTIO_NET_F_GUEST_TSO4, TSO6, UFO, USO4 or USO6 options were
  negotiated, then \field{gso_type} MAY be something other than
  VIRTIO_NET_HDR_GSO_NONE, and \field{gso_size} field indicates the
  desired MSS (see Packet Transmission point 2).
\item If the VIRTIO_NET_F_RSC_EXT option was negotiated (this
  implies one of VIRTIO_NET_F_GUEST_TSO4, TSO6), the
  device processes also duplicated ACK segments, reports
  number of coalesced TCP segments in \field{csum_start} field and
  number of duplicated ACK segments in \field{csum_offset} field
  and sets bit VIRTIO_NET_HDR_F_RSC_INFO in \field{flags}.
\item If the VIRTIO_NET_F_GUEST_CSUM feature was negotiated, the
  VIRTIO_NET_HDR_F_NEEDS_CSUM bit in \field{flags} can be
  set: if so, the packet checksum at offset \field{csum_offset}
  from \field{csum_start} and any preceding checksums
  have been validated.  The checksum on the packet is incomplete and
  if bit VIRTIO_NET_HDR_F_RSC_INFO is not set in \field{flags},
  then \field{csum_start} and \field{csum_offset} indicate how to calculate it
  (see Packet Transmission point 1).
\begin{note}
If \field{gso_type} differs from VIRTIO_NET_HDR_GSO_NONE and the
VIRTIO_NET_HDR_GSO_UDP_TUNNEL_IPV4 bit or the VIRTIO_NET_HDR_GSO_UDP_TUNNEL_IPV6
bit are not set, \field{csum_offset}
points to the only transport header present in the packet, and there are no
additional preceding checksums validated by VIRTIO_NET_HDR_F_NEEDS_CSUM.
\end{note}
\item If the VIRTIO_NET_F_GUEST_UDP_TUNNEL_GSO option was negotiated and
  \field{gso_type} is not VIRTIO_NET_HDR_GSO_NONE, the
  VIRTIO_NET_HDR_GSO_UDP_TUNNEL_IPV4 bit or the VIRTIO_NET_HDR_GSO_UDP_TUNNEL_IPV6
  bit MAY be set. In such case the \field{outer_th_offset} and
  \field{inner_nh_offset} fields indicate the corresponding
  headers information.
\item If the VIRTIO_NET_F_GUEST_UDP_TUNNEL_GSO_CSUM feature was
negotiated, and
  the VIRTIO_NET_HDR_GSO_UDP_TUNNEL_IPV4 bit or the VIRTIO_NET_HDR_GSO_UDP_TUNNEL_IPV6
  are set in \field{gso_type}, the VIRTIO_NET_HDR_F_UDP_TUNNEL_CSUM bit in the
  \field{flags} can be set: if so, the outer UDP checksum has been validated
  and the UDP header checksum at offset 6 from from \field{outer_th_offset}
  is set to the outer UDP pseudo header checksum.

\begin{note}
If the VIRTIO_NET_HDR_GSO_UDP_TUNNEL_IPV4 bit or VIRTIO_NET_HDR_GSO_UDP_TUNNEL_IPV6
bit are set in \field{gso_type}, the \field{csum_start} field refers to
the inner transport header offset (see Packet Transmission point 1).
If the VIRTIO_NET_HDR_F_UDP_TUNNEL_CSUM bit in \field{flags} is set both
the inner and the outer header checksums have been validated by
VIRTIO_NET_HDR_F_NEEDS_CSUM, otherwise only the inner transport header
checksum has been validated.
\end{note}
\end{enumerate}

If applicable, the device calculates per-packet hash for incoming packets as
defined in \ref{sec:Device Types / Network Device / Device Operation / Processing of Incoming Packets / Hash calculation for incoming packets}.

If applicable, the device reports hash information for incoming packets as
defined in \ref{sec:Device Types / Network Device / Device Operation / Processing of Incoming Packets / Hash reporting for incoming packets}.

\devicenormative{\paragraph}{Processing of Incoming Packets}{Device Types / Network Device / Device Operation / Processing of Incoming Packets}
\label{devicenormative:Device Types / Network Device / Device Operation / Processing of Packets}%old label for latexdiff

If VIRTIO_NET_F_MRG_RXBUF has not been negotiated, the device MUST set
\field{num_buffers} to 1.

If VIRTIO_NET_F_MRG_RXBUF has been negotiated, the device MUST set
\field{num_buffers} to indicate the number of buffers
the packet (including the header) is spread over.

If a receive packet is spread over multiple buffers, the device
MUST use all buffers but the last (i.e. the first \field{num_buffers} -
1 buffers) completely up to the full length of each buffer
supplied by the driver.

The device MUST use all buffers used by a single receive
packet together, such that at least \field{num_buffers} are
observed by driver as used.

If VIRTIO_NET_F_GUEST_CSUM is not negotiated, the device MUST set
\field{flags} to zero and SHOULD supply a fully checksummed
packet to the driver.

If VIRTIO_NET_F_GUEST_TSO4 is not negotiated, the device MUST NOT set
\field{gso_type} to VIRTIO_NET_HDR_GSO_TCPV4.

If VIRTIO_NET_F_GUEST_UDP is not negotiated, the device MUST NOT set
\field{gso_type} to VIRTIO_NET_HDR_GSO_UDP.

If VIRTIO_NET_F_GUEST_TSO6 is not negotiated, the device MUST NOT set
\field{gso_type} to VIRTIO_NET_HDR_GSO_TCPV6.

If none of VIRTIO_NET_F_GUEST_USO4 or VIRTIO_NET_F_GUEST_USO6 have been negotiated,
the device MUST NOT set \field{gso_type} to VIRTIO_NET_HDR_GSO_UDP_L4.

If VIRTIO_NET_F_GUEST_UDP_TUNNEL_GSO is not negotiated, the device MUST NOT set
either the VIRTIO_NET_HDR_GSO_UDP_TUNNEL_IPV4 bit or the
VIRTIO_NET_HDR_GSO_UDP_TUNNEL_IPV6 bit in \field{gso_type}.

If VIRTIO_NET_F_GUEST_UDP_TUNNEL_GSO_CSUM is not negotiated the device MUST NOT set
the VIRTIO_NET_HDR_F_UDP_TUNNEL_CSUM bit in \field{flags}.

The device SHOULD NOT send to the driver TCP packets requiring segmentation offload
which have the Explicit Congestion Notification bit set, unless the
VIRTIO_NET_F_GUEST_ECN feature is negotiated, in which case the
device MUST set the VIRTIO_NET_HDR_GSO_ECN bit in
\field{gso_type}.

If the VIRTIO_NET_F_GUEST_CSUM feature has been negotiated, the
device MAY set the VIRTIO_NET_HDR_F_NEEDS_CSUM bit in
\field{flags}, if so:
\begin{enumerate}
\item the device MUST validate the packet checksum at
	offset \field{csum_offset} from \field{csum_start} as well as all
	preceding offsets;
\item the device MUST set the packet checksum stored in the
	receive buffer to the TCP/UDP pseudo header;
\item the device MUST set \field{csum_start} and
	\field{csum_offset} such that calculating a ones'
	complement checksum from \field{csum_start} up until the
	end of the packet and storing the result at offset
	\field{csum_offset} from  \field{csum_start} will result in a
	fully checksummed packet;
\end{enumerate}

The device MUST NOT send to the driver GSO packets encapsulated in UDP
tunnel and requiring segmentation offload, unless the
VIRTIO_NET_F_GUEST_UDP_TUNNEL_GSO is negotiated, in which case the device MUST set
the VIRTIO_NET_HDR_GSO_UDP_TUNNEL_IPV4 bit or the VIRTIO_NET_HDR_GSO_UDP_TUNNEL_IPV6
bit in \field{gso_type} according to the inner network header protocol type,
MUST set the \field{outer_th_offset} and \field{inner_nh_offset} fields
to the corresponding header information, and the outer UDP header MUST NOT
require checksum offload.

If the VIRTIO_NET_F_GUEST_UDP_TUNNEL_GSO_CSUM feature has not been negotiated,
the device MUST NOT send the driver GSO packets encapsulated in UDP
tunnel and requiring segmentation and outer checksum offload.

If none of the VIRTIO_NET_F_GUEST_TSO4, TSO6, UFO, USO4 or USO6 options have
been negotiated, the device MUST set \field{gso_type} to
VIRTIO_NET_HDR_GSO_NONE.

If \field{gso_type} differs from VIRTIO_NET_HDR_GSO_NONE, then
the device MUST also set the VIRTIO_NET_HDR_F_NEEDS_CSUM bit in
\field{flags} MUST set \field{gso_size} to indicate the desired MSS.
If VIRTIO_NET_F_RSC_EXT was negotiated, the device MUST also
set VIRTIO_NET_HDR_F_RSC_INFO bit in \field{flags},
set \field{csum_start} to number of coalesced TCP segments and
set \field{csum_offset} to number of received duplicated ACK segments.

If VIRTIO_NET_F_RSC_EXT was not negotiated, the device MUST
not set VIRTIO_NET_HDR_F_RSC_INFO bit in \field{flags}.

If one of the VIRTIO_NET_F_GUEST_TSO4, TSO6, UFO, USO4 or USO6 options have
been negotiated, the device SHOULD set \field{hdr_len} to a value
not less than the length of the headers, including the transport
header. If \field{gso_type} has the VIRTIO_NET_HDR_GSO_UDP_TUNNEL_IPV4 bit
or the VIRTIO_NET_HDR_GSO_UDP_TUNNEL_IPV6 bit set, the referenced transport
header is the inner one.

If the VIRTIO_NET_F_GUEST_CSUM feature has been negotiated, the
device MAY set the VIRTIO_NET_HDR_F_DATA_VALID bit in
\field{flags}, if so, the device MUST validate the packet
checksum. If the VIRTIO_NET_F_GUEST_UDP_TUNNEL_GSO_CSUM feature has
been negotiated, and the VIRTIO_NET_HDR_F_UDP_TUNNEL_CSUM bit set in
\field{flags}, both the outer UDP checksum and the inner transport
checksum have been validated.
Otherwise level of checksum is validated: in case of multiple
encapsulated protocols the outermost one.

If either the VIRTIO_NET_HDR_GSO_UDP_TUNNEL_IPV4 bit or the
VIRTIO_NET_HDR_GSO_UDP_TUNNEL_IPV6 bit in \field{gso_type} are set,
the device MUST NOT set the VIRTIO_NET_HDR_F_DATA_VALID bit in
\field{flags}.

If the VIRTIO_NET_F_GUEST_UDP_TUNNEL_GSO_CSUM feature has been negotiated
and either the VIRTIO_NET_HDR_GSO_UDP_TUNNEL_IPV4 bit is set or the
VIRTIO_NET_HDR_GSO_UDP_TUNNEL_IPV6 bit is set in \field{gso_type}, the
device MAY set the VIRTIO_NET_HDR_F_UDP_TUNNEL_CSUM bit in
\field{flags}, if so the device MUST set the packet outer UDP checksum
stored in the receive buffer to the outer UDP pseudo header.

Otherwise, the VIRTIO_NET_F_GUEST_UDP_TUNNEL_GSO_CSUM feature has been
negotiated, either the VIRTIO_NET_HDR_GSO_UDP_TUNNEL_IPV4 bit is set or the
VIRTIO_NET_HDR_GSO_UDP_TUNNEL_IPV6 bit is set in \field{gso_type},
and the bit VIRTIO_NET_HDR_F_UDP_TUNNEL_CSUM is not set in
\field{flags}, the device MUST either provide a zero outer UDP header
checksum or a fully checksummed outer UDP header.

\drivernormative{\paragraph}{Processing of Incoming
Packets}{Device Types / Network Device / Device Operation /
Processing of Incoming Packets}

The driver MUST ignore \field{flag} bits that it does not recognize.

If VIRTIO_NET_HDR_F_NEEDS_CSUM bit in \field{flags} is not set or
if VIRTIO_NET_HDR_F_RSC_INFO bit \field{flags} is set, the
driver MUST NOT use the \field{csum_start} and \field{csum_offset}.

If one of the VIRTIO_NET_F_GUEST_TSO4, TSO6, UFO, USO4 or USO6 options have
been negotiated, the driver MAY use \field{hdr_len} only as a hint about the
transport header size.
The driver MUST NOT rely on \field{hdr_len} to be correct.
\begin{note}
This is due to various bugs in implementations.
\end{note}

If neither VIRTIO_NET_HDR_F_NEEDS_CSUM nor
VIRTIO_NET_HDR_F_DATA_VALID is set, the driver MUST NOT
rely on the packet checksum being correct.

If both the VIRTIO_NET_HDR_GSO_UDP_TUNNEL_IPV4 bit and
the VIRTIO_NET_HDR_GSO_UDP_TUNNEL_IPV6 bit in in \field{gso_type} are set,
the driver MUST NOT accept the packet.

If the VIRTIO_NET_HDR_GSO_UDP_TUNNEL_IPV4 bit or the VIRTIO_NET_HDR_GSO_UDP_TUNNEL_IPV6
bit in \field{gso_type} are not set, the driver MUST NOT use the
\field{outer_th_offset} and \field{inner_nh_offset}.

If either the VIRTIO_NET_HDR_GSO_UDP_TUNNEL_IPV4 bit or
the VIRTIO_NET_HDR_GSO_UDP_TUNNEL_IPV6 bit in \field{gso_type} are set, and any of
the following is true:
\begin{itemize}
\item the VIRTIO_NET_HDR_F_NEEDS_CSUM bit is not set in \field{flags}
\item the VIRTIO_NET_HDR_F_DATA_VALID bit is set in \field{flags}
\item the \field{gso_type} excluding the VIRTIO_NET_HDR_GSO_UDP_TUNNEL_IPV4
bit and the VIRTIO_NET_HDR_GSO_UDP_TUNNEL_IPV6 bit is VIRTIO_NET_HDR_GSO_NONE
\end{itemize}
the driver MUST NOT accept the packet.

If the VIRTIO_NET_HDR_F_UDP_TUNNEL_CSUM bit and the VIRTIO_NET_HDR_F_NEEDS_CSUM
bit in \field{flags} are set,
and both the bits VIRTIO_NET_HDR_GSO_UDP_TUNNEL_IPV4 and
VIRTIO_NET_HDR_GSO_UDP_TUNNEL_IPV6 in \field{gso_type} are not set,
the driver MOST NOT accept the packet.

\paragraph{Hash calculation for incoming packets}
\label{sec:Device Types / Network Device / Device Operation / Processing of Incoming Packets / Hash calculation for incoming packets}

A device attempts to calculate a per-packet hash in the following cases:
\begin{itemize}
\item The feature VIRTIO_NET_F_RSS was negotiated. The device uses the hash to determine the receive virtqueue to place incoming packets.
\item The feature VIRTIO_NET_F_HASH_REPORT was negotiated. The device reports the hash value and the hash type with the packet.
\end{itemize}

If the feature VIRTIO_NET_F_RSS was negotiated:
\begin{itemize}
\item The device uses \field{hash_types} of the virtio_net_rss_config structure as 'Enabled hash types' bitmask.
\item If additionally the feature VIRTIO_NET_F_HASH_TUNNEL was negotiated, the device uses \field{enabled_tunnel_types} of the
      virtnet_hash_tunnel structure as 'Encapsulation types enabled for inner header hash' bitmask.
\item The device uses a key as defined in \field{hash_key_data} and \field{hash_key_length} of the virtio_net_rss_config structure (see
\ref{sec:Device Types / Network Device / Device Operation / Control Virtqueue / Receive-side scaling (RSS) / Setting RSS parameters}).
\end{itemize}

If the feature VIRTIO_NET_F_RSS was not negotiated:
\begin{itemize}
\item The device uses \field{hash_types} of the virtio_net_hash_config structure as 'Enabled hash types' bitmask.
\item If additionally the feature VIRTIO_NET_F_HASH_TUNNEL was negotiated, the device uses \field{enabled_tunnel_types} of the
      virtnet_hash_tunnel structure as 'Encapsulation types enabled for inner header hash' bitmask.
\item The device uses a key as defined in \field{hash_key_data} and \field{hash_key_length} of the virtio_net_hash_config structure (see
\ref{sec:Device Types / Network Device / Device Operation / Control Virtqueue / Automatic receive steering in multiqueue mode / Hash calculation}).
\end{itemize}

Note that if the device offers VIRTIO_NET_F_HASH_REPORT, even if it supports only one pair of virtqueues, it MUST support
at least one of commands of VIRTIO_NET_CTRL_MQ class to configure reported hash parameters:
\begin{itemize}
\item If the device offers VIRTIO_NET_F_RSS, it MUST support VIRTIO_NET_CTRL_MQ_RSS_CONFIG command per
 \ref{sec:Device Types / Network Device / Device Operation / Control Virtqueue / Receive-side scaling (RSS) / Setting RSS parameters}.
\item Otherwise the device MUST support VIRTIO_NET_CTRL_MQ_HASH_CONFIG command per
 \ref{sec:Device Types / Network Device / Device Operation / Control Virtqueue / Automatic receive steering in multiqueue mode / Hash calculation}.
\end{itemize}

The per-packet hash calculation can depend on the IP packet type. See
\hyperref[intro:IP]{[IP]}, \hyperref[intro:UDP]{[UDP]} and \hyperref[intro:TCP]{[TCP]}.

\subparagraph{Supported/enabled hash types}
\label{sec:Device Types / Network Device / Device Operation / Processing of Incoming Packets / Hash calculation for incoming packets / Supported/enabled hash types}
Hash types applicable for IPv4 packets:
\begin{lstlisting}
#define VIRTIO_NET_HASH_TYPE_IPv4              (1 << 0)
#define VIRTIO_NET_HASH_TYPE_TCPv4             (1 << 1)
#define VIRTIO_NET_HASH_TYPE_UDPv4             (1 << 2)
\end{lstlisting}
Hash types applicable for IPv6 packets without extension headers
\begin{lstlisting}
#define VIRTIO_NET_HASH_TYPE_IPv6              (1 << 3)
#define VIRTIO_NET_HASH_TYPE_TCPv6             (1 << 4)
#define VIRTIO_NET_HASH_TYPE_UDPv6             (1 << 5)
\end{lstlisting}
Hash types applicable for IPv6 packets with extension headers
\begin{lstlisting}
#define VIRTIO_NET_HASH_TYPE_IP_EX             (1 << 6)
#define VIRTIO_NET_HASH_TYPE_TCP_EX            (1 << 7)
#define VIRTIO_NET_HASH_TYPE_UDP_EX            (1 << 8)
\end{lstlisting}

\subparagraph{IPv4 packets}
\label{sec:Device Types / Network Device / Device Operation / Processing of Incoming Packets / Hash calculation for incoming packets / IPv4 packets}
The device calculates the hash on IPv4 packets according to 'Enabled hash types' bitmask as follows:
\begin{itemize}
\item If VIRTIO_NET_HASH_TYPE_TCPv4 is set and the packet has
a TCP header, the hash is calculated over the following fields:
\begin{itemize}
\item Source IP address
\item Destination IP address
\item Source TCP port
\item Destination TCP port
\end{itemize}
\item Else if VIRTIO_NET_HASH_TYPE_UDPv4 is set and the
packet has a UDP header, the hash is calculated over the following fields:
\begin{itemize}
\item Source IP address
\item Destination IP address
\item Source UDP port
\item Destination UDP port
\end{itemize}
\item Else if VIRTIO_NET_HASH_TYPE_IPv4 is set, the hash is
calculated over the following fields:
\begin{itemize}
\item Source IP address
\item Destination IP address
\end{itemize}
\item Else the device does not calculate the hash
\end{itemize}

\subparagraph{IPv6 packets without extension header}
\label{sec:Device Types / Network Device / Device Operation / Processing of Incoming Packets / Hash calculation for incoming packets / IPv6 packets without extension header}
The device calculates the hash on IPv6 packets without extension
headers according to 'Enabled hash types' bitmask as follows:
\begin{itemize}
\item If VIRTIO_NET_HASH_TYPE_TCPv6 is set and the packet has
a TCPv6 header, the hash is calculated over the following fields:
\begin{itemize}
\item Source IPv6 address
\item Destination IPv6 address
\item Source TCP port
\item Destination TCP port
\end{itemize}
\item Else if VIRTIO_NET_HASH_TYPE_UDPv6 is set and the
packet has a UDPv6 header, the hash is calculated over the following fields:
\begin{itemize}
\item Source IPv6 address
\item Destination IPv6 address
\item Source UDP port
\item Destination UDP port
\end{itemize}
\item Else if VIRTIO_NET_HASH_TYPE_IPv6 is set, the hash is
calculated over the following fields:
\begin{itemize}
\item Source IPv6 address
\item Destination IPv6 address
\end{itemize}
\item Else the device does not calculate the hash
\end{itemize}

\subparagraph{IPv6 packets with extension header}
\label{sec:Device Types / Network Device / Device Operation / Processing of Incoming Packets / Hash calculation for incoming packets / IPv6 packets with extension header}
The device calculates the hash on IPv6 packets with extension
headers according to 'Enabled hash types' bitmask as follows:
\begin{itemize}
\item If VIRTIO_NET_HASH_TYPE_TCP_EX is set and the packet
has a TCPv6 header, the hash is calculated over the following fields:
\begin{itemize}
\item Home address from the home address option in the IPv6 destination options header. If the extension header is not present, use the Source IPv6 address.
\item IPv6 address that is contained in the Routing-Header-Type-2 from the associated extension header. If the extension header is not present, use the Destination IPv6 address.
\item Source TCP port
\item Destination TCP port
\end{itemize}
\item Else if VIRTIO_NET_HASH_TYPE_UDP_EX is set and the
packet has a UDPv6 header, the hash is calculated over the following fields:
\begin{itemize}
\item Home address from the home address option in the IPv6 destination options header. If the extension header is not present, use the Source IPv6 address.
\item IPv6 address that is contained in the Routing-Header-Type-2 from the associated extension header. If the extension header is not present, use the Destination IPv6 address.
\item Source UDP port
\item Destination UDP port
\end{itemize}
\item Else if VIRTIO_NET_HASH_TYPE_IP_EX is set, the hash is
calculated over the following fields:
\begin{itemize}
\item Home address from the home address option in the IPv6 destination options header. If the extension header is not present, use the Source IPv6 address.
\item IPv6 address that is contained in the Routing-Header-Type-2 from the associated extension header. If the extension header is not present, use the Destination IPv6 address.
\end{itemize}
\item Else skip IPv6 extension headers and calculate the hash as
defined for an IPv6 packet without extension headers
(see \ref{sec:Device Types / Network Device / Device Operation / Processing of Incoming Packets / Hash calculation for incoming packets / IPv6 packets without extension header}).
\end{itemize}

\paragraph{Inner Header Hash}
\label{sec:Device Types / Network Device / Device Operation / Processing of Incoming Packets / Inner Header Hash}

If VIRTIO_NET_F_HASH_TUNNEL has been negotiated, the driver can send the command
VIRTIO_NET_CTRL_HASH_TUNNEL_SET to configure the calculation of the inner header hash.

\begin{lstlisting}
struct virtnet_hash_tunnel {
    le32 enabled_tunnel_types;
};

#define VIRTIO_NET_CTRL_HASH_TUNNEL 7
 #define VIRTIO_NET_CTRL_HASH_TUNNEL_SET 0
\end{lstlisting}

Field \field{enabled_tunnel_types} contains the bitmask of encapsulation types enabled for inner header hash.
See \ref{sec:Device Types / Network Device / Device Operation / Processing of Incoming Packets /
Hash calculation for incoming packets / Encapsulation types supported/enabled for inner header hash}.

The class VIRTIO_NET_CTRL_HASH_TUNNEL has one command:
VIRTIO_NET_CTRL_HASH_TUNNEL_SET sets \field{enabled_tunnel_types} for the device using the
virtnet_hash_tunnel structure, which is read-only for the device.

Inner header hash is disabled by VIRTIO_NET_CTRL_HASH_TUNNEL_SET with \field{enabled_tunnel_types} set to 0.

Initially (before the driver sends any VIRTIO_NET_CTRL_HASH_TUNNEL_SET command) all
encapsulation types are disabled for inner header hash.

\subparagraph{Encapsulated packet}
\label{sec:Device Types / Network Device / Device Operation / Processing of Incoming Packets / Hash calculation for incoming packets / Encapsulated packet}

Multiple tunneling protocols allow encapsulating an inner, payload packet in an outer, encapsulated packet.
The encapsulated packet thus contains an outer header and an inner header, and the device calculates the
hash over either the inner header or the outer header.

If VIRTIO_NET_F_HASH_TUNNEL is negotiated and a received encapsulated packet's outer header matches one of the
encapsulation types enabled in \field{enabled_tunnel_types}, then the device uses the inner header for hash
calculations (only a single level of encapsulation is currently supported).

If VIRTIO_NET_F_HASH_TUNNEL is negotiated and a received packet's (outer) header does not match any encapsulation
types enabled in \field{enabled_tunnel_types}, then the device uses the outer header for hash calculations.

\subparagraph{Encapsulation types supported/enabled for inner header hash}
\label{sec:Device Types / Network Device / Device Operation / Processing of Incoming Packets /
Hash calculation for incoming packets / Encapsulation types supported/enabled for inner header hash}

Encapsulation types applicable for inner header hash:
\begin{lstlisting}[escapechar=|]
#define VIRTIO_NET_HASH_TUNNEL_TYPE_GRE_2784    (1 << 0) /* |\hyperref[intro:rfc2784]{[RFC2784]}| */
#define VIRTIO_NET_HASH_TUNNEL_TYPE_GRE_2890    (1 << 1) /* |\hyperref[intro:rfc2890]{[RFC2890]}| */
#define VIRTIO_NET_HASH_TUNNEL_TYPE_GRE_7676    (1 << 2) /* |\hyperref[intro:rfc7676]{[RFC7676]}| */
#define VIRTIO_NET_HASH_TUNNEL_TYPE_GRE_UDP     (1 << 3) /* |\hyperref[intro:rfc8086]{[GRE-in-UDP]}| */
#define VIRTIO_NET_HASH_TUNNEL_TYPE_VXLAN       (1 << 4) /* |\hyperref[intro:vxlan]{[VXLAN]}| */
#define VIRTIO_NET_HASH_TUNNEL_TYPE_VXLAN_GPE   (1 << 5) /* |\hyperref[intro:vxlan-gpe]{[VXLAN-GPE]}| */
#define VIRTIO_NET_HASH_TUNNEL_TYPE_GENEVE      (1 << 6) /* |\hyperref[intro:geneve]{[GENEVE]}| */
#define VIRTIO_NET_HASH_TUNNEL_TYPE_IPIP        (1 << 7) /* |\hyperref[intro:ipip]{[IPIP]}| */
#define VIRTIO_NET_HASH_TUNNEL_TYPE_NVGRE       (1 << 8) /* |\hyperref[intro:nvgre]{[NVGRE]}| */
\end{lstlisting}

\subparagraph{Advice}
Example uses of the inner header hash:
\begin{itemize}
\item Legacy tunneling protocols, lacking the outer header entropy, can use RSS with the inner header hash to
      distribute flows with identical outer but different inner headers across various queues, improving performance.
\item Identify an inner flow distributed across multiple outer tunnels.
\end{itemize}

As using the inner header hash completely discards the outer header entropy, care must be taken
if the inner header is controlled by an adversary, as the adversary can then intentionally create
configurations with insufficient entropy.

Besides disabling the inner header hash, mitigations would depend on how the hash is used. When the hash
use is limited to the RSS queue selection, the inner header hash may have quality of service (QoS) limitations.

\devicenormative{\subparagraph}{Inner Header Hash}{Device Types / Network Device / Device Operation / Control Virtqueue / Inner Header Hash}

If the (outer) header of the received packet does not match any encapsulation types enabled
in \field{enabled_tunnel_types}, the device MUST calculate the hash on the outer header.

If the device receives any bits in \field{enabled_tunnel_types} which are not set in \field{supported_tunnel_types},
it SHOULD respond to the VIRTIO_NET_CTRL_HASH_TUNNEL_SET command with VIRTIO_NET_ERR.

If the driver sets \field{enabled_tunnel_types} to 0 through VIRTIO_NET_CTRL_HASH_TUNNEL_SET or upon the device reset,
the device MUST disable the inner header hash for all encapsulation types.

\drivernormative{\subparagraph}{Inner Header Hash}{Device Types / Network Device / Device Operation / Control Virtqueue / Inner Header Hash}

The driver MUST have negotiated the VIRTIO_NET_F_HASH_TUNNEL feature when issuing the VIRTIO_NET_CTRL_HASH_TUNNEL_SET command.

The driver MUST NOT set any bits in \field{enabled_tunnel_types} which are not set in \field{supported_tunnel_types}.

The driver MUST ignore bits in \field{supported_tunnel_types} which are not documented in this specification.

\paragraph{Hash reporting for incoming packets}
\label{sec:Device Types / Network Device / Device Operation / Processing of Incoming Packets / Hash reporting for incoming packets}

If VIRTIO_NET_F_HASH_REPORT was negotiated and
 the device has calculated the hash for the packet, the device fills \field{hash_report} with the report type of calculated hash
and \field{hash_value} with the value of calculated hash.

If VIRTIO_NET_F_HASH_REPORT was negotiated but due to any reason the
hash was not calculated, the device sets \field{hash_report} to VIRTIO_NET_HASH_REPORT_NONE.

Possible values that the device can report in \field{hash_report} are defined below.
They correspond to supported hash types defined in
\ref{sec:Device Types / Network Device / Device Operation / Processing of Incoming Packets / Hash calculation for incoming packets / Supported/enabled hash types}
as follows:

VIRTIO_NET_HASH_TYPE_XXX = 1 << (VIRTIO_NET_HASH_REPORT_XXX - 1)

\begin{lstlisting}
#define VIRTIO_NET_HASH_REPORT_NONE            0
#define VIRTIO_NET_HASH_REPORT_IPv4            1
#define VIRTIO_NET_HASH_REPORT_TCPv4           2
#define VIRTIO_NET_HASH_REPORT_UDPv4           3
#define VIRTIO_NET_HASH_REPORT_IPv6            4
#define VIRTIO_NET_HASH_REPORT_TCPv6           5
#define VIRTIO_NET_HASH_REPORT_UDPv6           6
#define VIRTIO_NET_HASH_REPORT_IPv6_EX         7
#define VIRTIO_NET_HASH_REPORT_TCPv6_EX        8
#define VIRTIO_NET_HASH_REPORT_UDPv6_EX        9
\end{lstlisting}

\subsubsection{Control Virtqueue}\label{sec:Device Types / Network Device / Device Operation / Control Virtqueue}

The driver uses the control virtqueue (if VIRTIO_NET_F_CTRL_VQ is
negotiated) to send commands to manipulate various features of
the device which would not easily map into the configuration
space.

All commands are of the following form:

\begin{lstlisting}
struct virtio_net_ctrl {
        u8 class;
        u8 command;
        u8 command-specific-data[];
        u8 ack;
        u8 command-specific-result[];
};

/* ack values */
#define VIRTIO_NET_OK     0
#define VIRTIO_NET_ERR    1
\end{lstlisting}

The \field{class}, \field{command} and command-specific-data are set by the
driver, and the device sets the \field{ack} byte and optionally
\field{command-specific-result}. There is little the driver can
do except issue a diagnostic if \field{ack} is not VIRTIO_NET_OK.

The command VIRTIO_NET_CTRL_STATS_QUERY and VIRTIO_NET_CTRL_STATS_GET contain
\field{command-specific-result}.

\paragraph{Packet Receive Filtering}\label{sec:Device Types / Network Device / Device Operation / Control Virtqueue / Packet Receive Filtering}
\label{sec:Device Types / Network Device / Device Operation / Control Virtqueue / Setting Promiscuous Mode}%old label for latexdiff

If the VIRTIO_NET_F_CTRL_RX and VIRTIO_NET_F_CTRL_RX_EXTRA
features are negotiated, the driver can send control commands for
promiscuous mode, multicast, unicast and broadcast receiving.

\begin{note}
In general, these commands are best-effort: unwanted
packets could still arrive.
\end{note}

\begin{lstlisting}
#define VIRTIO_NET_CTRL_RX    0
 #define VIRTIO_NET_CTRL_RX_PROMISC      0
 #define VIRTIO_NET_CTRL_RX_ALLMULTI     1
 #define VIRTIO_NET_CTRL_RX_ALLUNI       2
 #define VIRTIO_NET_CTRL_RX_NOMULTI      3
 #define VIRTIO_NET_CTRL_RX_NOUNI        4
 #define VIRTIO_NET_CTRL_RX_NOBCAST      5
\end{lstlisting}


\devicenormative{\subparagraph}{Packet Receive Filtering}{Device Types / Network Device / Device Operation / Control Virtqueue / Packet Receive Filtering}

If the VIRTIO_NET_F_CTRL_RX feature has been negotiated,
the device MUST support the following VIRTIO_NET_CTRL_RX class
commands:
\begin{itemize}
\item VIRTIO_NET_CTRL_RX_PROMISC turns promiscuous mode on and
off. The command-specific-data is one byte containing 0 (off) or
1 (on). If promiscuous mode is on, the device SHOULD receive all
incoming packets.
This SHOULD take effect even if one of the other modes set by
a VIRTIO_NET_CTRL_RX class command is on.
\item VIRTIO_NET_CTRL_RX_ALLMULTI turns all-multicast receive on and
off. The command-specific-data is one byte containing 0 (off) or
1 (on). When all-multicast receive is on the device SHOULD allow
all incoming multicast packets.
\end{itemize}

If the VIRTIO_NET_F_CTRL_RX_EXTRA feature has been negotiated,
the device MUST support the following VIRTIO_NET_CTRL_RX class
commands:
\begin{itemize}
\item VIRTIO_NET_CTRL_RX_ALLUNI turns all-unicast receive on and
off. The command-specific-data is one byte containing 0 (off) or
1 (on). When all-unicast receive is on the device SHOULD allow
all incoming unicast packets.
\item VIRTIO_NET_CTRL_RX_NOMULTI suppresses multicast receive.
The command-specific-data is one byte containing 0 (multicast
receive allowed) or 1 (multicast receive suppressed).
When multicast receive is suppressed, the device SHOULD NOT
send multicast packets to the driver.
This SHOULD take effect even if VIRTIO_NET_CTRL_RX_ALLMULTI is on.
This filter SHOULD NOT apply to broadcast packets.
\item VIRTIO_NET_CTRL_RX_NOUNI suppresses unicast receive.
The command-specific-data is one byte containing 0 (unicast
receive allowed) or 1 (unicast receive suppressed).
When unicast receive is suppressed, the device SHOULD NOT
send unicast packets to the driver.
This SHOULD take effect even if VIRTIO_NET_CTRL_RX_ALLUNI is on.
\item VIRTIO_NET_CTRL_RX_NOBCAST suppresses broadcast receive.
The command-specific-data is one byte containing 0 (broadcast
receive allowed) or 1 (broadcast receive suppressed).
When broadcast receive is suppressed, the device SHOULD NOT
send broadcast packets to the driver.
This SHOULD take effect even if VIRTIO_NET_CTRL_RX_ALLMULTI is on.
\end{itemize}

\drivernormative{\subparagraph}{Packet Receive Filtering}{Device Types / Network Device / Device Operation / Control Virtqueue / Packet Receive Filtering}

If the VIRTIO_NET_F_CTRL_RX feature has not been negotiated,
the driver MUST NOT issue commands VIRTIO_NET_CTRL_RX_PROMISC or
VIRTIO_NET_CTRL_RX_ALLMULTI.

If the VIRTIO_NET_F_CTRL_RX_EXTRA feature has not been negotiated,
the driver MUST NOT issue commands
 VIRTIO_NET_CTRL_RX_ALLUNI,
 VIRTIO_NET_CTRL_RX_NOMULTI,
 VIRTIO_NET_CTRL_RX_NOUNI or
 VIRTIO_NET_CTRL_RX_NOBCAST.

\paragraph{Setting MAC Address Filtering}\label{sec:Device Types / Network Device / Device Operation / Control Virtqueue / Setting MAC Address Filtering}

If the VIRTIO_NET_F_CTRL_RX feature is negotiated, the driver can
send control commands for MAC address filtering.

\begin{lstlisting}
struct virtio_net_ctrl_mac {
        le32 entries;
        u8 macs[entries][6];
};

#define VIRTIO_NET_CTRL_MAC    1
 #define VIRTIO_NET_CTRL_MAC_TABLE_SET        0
 #define VIRTIO_NET_CTRL_MAC_ADDR_SET         1
\end{lstlisting}

The device can filter incoming packets by any number of destination
MAC addresses\footnote{Since there are no guarantees, it can use a hash filter or
silently switch to allmulti or promiscuous mode if it is given too
many addresses.
}. This table is set using the class
VIRTIO_NET_CTRL_MAC and the command VIRTIO_NET_CTRL_MAC_TABLE_SET. The
command-specific-data is two variable length tables of 6-byte MAC
addresses (as described in struct virtio_net_ctrl_mac). The first table contains unicast addresses, and the second
contains multicast addresses.

The VIRTIO_NET_CTRL_MAC_ADDR_SET command is used to set the
default MAC address which rx filtering
accepts (and if VIRTIO_NET_F_MAC has been negotiated,
this will be reflected in \field{mac} in config space).

The command-specific-data for VIRTIO_NET_CTRL_MAC_ADDR_SET is
the 6-byte MAC address.

\devicenormative{\subparagraph}{Setting MAC Address Filtering}{Device Types / Network Device / Device Operation / Control Virtqueue / Setting MAC Address Filtering}

The device MUST have an empty MAC filtering table on reset.

The device MUST update the MAC filtering table before it consumes
the VIRTIO_NET_CTRL_MAC_TABLE_SET command.

The device MUST update \field{mac} in config space before it consumes
the VIRTIO_NET_CTRL_MAC_ADDR_SET command, if VIRTIO_NET_F_MAC has
been negotiated.

The device SHOULD drop incoming packets which have a destination MAC which
matches neither the \field{mac} (or that set with VIRTIO_NET_CTRL_MAC_ADDR_SET)
nor the MAC filtering table.

\drivernormative{\subparagraph}{Setting MAC Address Filtering}{Device Types / Network Device / Device Operation / Control Virtqueue / Setting MAC Address Filtering}

If VIRTIO_NET_F_CTRL_RX has not been negotiated,
the driver MUST NOT issue VIRTIO_NET_CTRL_MAC class commands.

If VIRTIO_NET_F_CTRL_RX has been negotiated,
the driver SHOULD issue VIRTIO_NET_CTRL_MAC_ADDR_SET
to set the default mac if it is different from \field{mac}.

The driver MUST follow the VIRTIO_NET_CTRL_MAC_TABLE_SET command
by a le32 number, followed by that number of non-multicast
MAC addresses, followed by another le32 number, followed by
that number of multicast addresses.  Either number MAY be 0.

\subparagraph{Legacy Interface: Setting MAC Address Filtering}\label{sec:Device Types / Network Device / Device Operation / Control Virtqueue / Setting MAC Address Filtering / Legacy Interface: Setting MAC Address Filtering}
When using the legacy interface, transitional devices and drivers
MUST format \field{entries} in struct virtio_net_ctrl_mac
according to the native endian of the guest rather than
(necessarily when not using the legacy interface) little-endian.

Legacy drivers that didn't negotiate VIRTIO_NET_F_CTRL_MAC_ADDR
changed \field{mac} in config space when NIC is accepting
incoming packets. These drivers always wrote the mac value from
first to last byte, therefore after detecting such drivers,
a transitional device MAY defer MAC update, or MAY defer
processing incoming packets until driver writes the last byte
of \field{mac} in the config space.

\paragraph{VLAN Filtering}\label{sec:Device Types / Network Device / Device Operation / Control Virtqueue / VLAN Filtering}

If the driver negotiates the VIRTIO_NET_F_CTRL_VLAN feature, it
can control a VLAN filter table in the device. The VLAN filter
table applies only to VLAN tagged packets.

When VIRTIO_NET_F_CTRL_VLAN is negotiated, the device starts with
an empty VLAN filter table.

\begin{note}
Similar to the MAC address based filtering, the VLAN filtering
is also best-effort: unwanted packets could still arrive.
\end{note}

\begin{lstlisting}
#define VIRTIO_NET_CTRL_VLAN       2
 #define VIRTIO_NET_CTRL_VLAN_ADD             0
 #define VIRTIO_NET_CTRL_VLAN_DEL             1
\end{lstlisting}

Both the VIRTIO_NET_CTRL_VLAN_ADD and VIRTIO_NET_CTRL_VLAN_DEL
command take a little-endian 16-bit VLAN id as the command-specific-data.

VIRTIO_NET_CTRL_VLAN_ADD command adds the specified VLAN to the
VLAN filter table.

VIRTIO_NET_CTRL_VLAN_DEL command removes the specified VLAN from
the VLAN filter table.

\devicenormative{\subparagraph}{VLAN Filtering}{Device Types / Network Device / Device Operation / Control Virtqueue / VLAN Filtering}

When VIRTIO_NET_F_CTRL_VLAN is not negotiated, the device MUST
accept all VLAN tagged packets.

When VIRTIO_NET_F_CTRL_VLAN is negotiated, the device MUST
accept all VLAN tagged packets whose VLAN tag is present in
the VLAN filter table and SHOULD drop all VLAN tagged packets
whose VLAN tag is absent in the VLAN filter table.

\subparagraph{Legacy Interface: VLAN Filtering}\label{sec:Device Types / Network Device / Device Operation / Control Virtqueue / VLAN Filtering / Legacy Interface: VLAN Filtering}
When using the legacy interface, transitional devices and drivers
MUST format the VLAN id
according to the native endian of the guest rather than
(necessarily when not using the legacy interface) little-endian.

\paragraph{Gratuitous Packet Sending}\label{sec:Device Types / Network Device / Device Operation / Control Virtqueue / Gratuitous Packet Sending}

If the driver negotiates the VIRTIO_NET_F_GUEST_ANNOUNCE (depends
on VIRTIO_NET_F_CTRL_VQ), the device can ask the driver to send gratuitous
packets; this is usually done after the guest has been physically
migrated, and needs to announce its presence on the new network
links. (As hypervisor does not have the knowledge of guest
network configuration (eg. tagged vlan) it is simplest to prod
the guest in this way).

\begin{lstlisting}
#define VIRTIO_NET_CTRL_ANNOUNCE       3
 #define VIRTIO_NET_CTRL_ANNOUNCE_ACK             0
\end{lstlisting}

The driver checks VIRTIO_NET_S_ANNOUNCE bit in the device configuration \field{status} field
when it notices the changes of device configuration. The
command VIRTIO_NET_CTRL_ANNOUNCE_ACK is used to indicate that
driver has received the notification and device clears the
VIRTIO_NET_S_ANNOUNCE bit in \field{status}.

Processing this notification involves:

\begin{enumerate}
\item Sending the gratuitous packets (eg. ARP) or marking there are pending
  gratuitous packets to be sent and letting deferred routine to
  send them.

\item Sending VIRTIO_NET_CTRL_ANNOUNCE_ACK command through control
  vq.
\end{enumerate}

\drivernormative{\subparagraph}{Gratuitous Packet Sending}{Device Types / Network Device / Device Operation / Control Virtqueue / Gratuitous Packet Sending}

If the driver negotiates VIRTIO_NET_F_GUEST_ANNOUNCE, it SHOULD notify
network peers of its new location after it sees the VIRTIO_NET_S_ANNOUNCE bit
in \field{status}.  The driver MUST send a command on the command queue
with class VIRTIO_NET_CTRL_ANNOUNCE and command VIRTIO_NET_CTRL_ANNOUNCE_ACK.

\devicenormative{\subparagraph}{Gratuitous Packet Sending}{Device Types / Network Device / Device Operation / Control Virtqueue / Gratuitous Packet Sending}

If VIRTIO_NET_F_GUEST_ANNOUNCE is negotiated, the device MUST clear the
VIRTIO_NET_S_ANNOUNCE bit in \field{status} upon receipt of a command buffer
with class VIRTIO_NET_CTRL_ANNOUNCE and command VIRTIO_NET_CTRL_ANNOUNCE_ACK
before marking the buffer as used.

\paragraph{Device operation in multiqueue mode}\label{sec:Device Types / Network Device / Device Operation / Control Virtqueue / Device operation in multiqueue mode}

This specification defines the following modes that a device MAY implement for operation with multiple transmit/receive virtqueues:
\begin{itemize}
\item Automatic receive steering as defined in \ref{sec:Device Types / Network Device / Device Operation / Control Virtqueue / Automatic receive steering in multiqueue mode}.
 If a device supports this mode, it offers the VIRTIO_NET_F_MQ feature bit.
\item Receive-side scaling as defined in \ref{devicenormative:Device Types / Network Device / Device Operation / Control Virtqueue / Receive-side scaling (RSS) / RSS processing}.
 If a device supports this mode, it offers the VIRTIO_NET_F_RSS feature bit.
\end{itemize}

A device MAY support one of these features or both. The driver MAY negotiate any set of these features that the device supports.

Multiqueue is disabled by default.

The driver enables multiqueue by sending a command using \field{class} VIRTIO_NET_CTRL_MQ. The \field{command} selects the mode of multiqueue operation, as follows:
\begin{lstlisting}
#define VIRTIO_NET_CTRL_MQ    4
 #define VIRTIO_NET_CTRL_MQ_VQ_PAIRS_SET        0 (for automatic receive steering)
 #define VIRTIO_NET_CTRL_MQ_RSS_CONFIG          1 (for configurable receive steering)
 #define VIRTIO_NET_CTRL_MQ_HASH_CONFIG         2 (for configurable hash calculation)
\end{lstlisting}

If more than one multiqueue mode is negotiated, the resulting device configuration is defined by the last command sent by the driver.

\paragraph{Automatic receive steering in multiqueue mode}\label{sec:Device Types / Network Device / Device Operation / Control Virtqueue / Automatic receive steering in multiqueue mode}

If the driver negotiates the VIRTIO_NET_F_MQ feature bit (depends on VIRTIO_NET_F_CTRL_VQ), it MAY transmit outgoing packets on one
of the multiple transmitq1\ldots transmitqN and ask the device to
queue incoming packets into one of the multiple receiveq1\ldots receiveqN
depending on the packet flow.

The driver enables multiqueue by
sending the VIRTIO_NET_CTRL_MQ_VQ_PAIRS_SET command, specifying
the number of the transmit and receive queues to be used up to
\field{max_virtqueue_pairs}; subsequently,
transmitq1\ldots transmitqn and receiveq1\ldots receiveqn where
n=\field{virtqueue_pairs} MAY be used.
\begin{lstlisting}
struct virtio_net_ctrl_mq_pairs_set {
       le16 virtqueue_pairs;
};
#define VIRTIO_NET_CTRL_MQ_VQ_PAIRS_MIN        1
#define VIRTIO_NET_CTRL_MQ_VQ_PAIRS_MAX        0x8000

\end{lstlisting}

When multiqueue is enabled by VIRTIO_NET_CTRL_MQ_VQ_PAIRS_SET command, the device MUST use automatic receive steering
based on packet flow. Programming of the receive steering
classificator is implicit. After the driver transmitted a packet of a
flow on transmitqX, the device SHOULD cause incoming packets for that flow to
be steered to receiveqX. For uni-directional protocols, or where
no packets have been transmitted yet, the device MAY steer a packet
to a random queue out of the specified receiveq1\ldots receiveqn.

Multiqueue is disabled by VIRTIO_NET_CTRL_MQ_VQ_PAIRS_SET with \field{virtqueue_pairs} to 1 (this is
the default) and waiting for the device to use the command buffer.

\drivernormative{\subparagraph}{Automatic receive steering in multiqueue mode}{Device Types / Network Device / Device Operation / Control Virtqueue / Automatic receive steering in multiqueue mode}

The driver MUST configure the virtqueues before enabling them with the
VIRTIO_NET_CTRL_MQ_VQ_PAIRS_SET command.

The driver MUST NOT request a \field{virtqueue_pairs} of 0 or
greater than \field{max_virtqueue_pairs} in the device configuration space.

The driver MUST queue packets only on any transmitq1 before the
VIRTIO_NET_CTRL_MQ_VQ_PAIRS_SET command.

The driver MUST NOT queue packets on transmit queues greater than
\field{virtqueue_pairs} once it has placed the VIRTIO_NET_CTRL_MQ_VQ_PAIRS_SET command in the available ring.

\devicenormative{\subparagraph}{Automatic receive steering in multiqueue mode}{Device Types / Network Device / Device Operation / Control Virtqueue / Automatic receive steering in multiqueue mode}

After initialization of reset, the device MUST queue packets only on receiveq1.

The device MUST NOT queue packets on receive queues greater than
\field{virtqueue_pairs} once it has placed the
VIRTIO_NET_CTRL_MQ_VQ_PAIRS_SET command in a used buffer.

If the destination receive queue is being reset (See \ref{sec:Basic Facilities of a Virtio Device / Virtqueues / Virtqueue Reset}),
the device SHOULD re-select another random queue. If all receive queues are
being reset, the device MUST drop the packet.

\subparagraph{Legacy Interface: Automatic receive steering in multiqueue mode}\label{sec:Device Types / Network Device / Device Operation / Control Virtqueue / Automatic receive steering in multiqueue mode / Legacy Interface: Automatic receive steering in multiqueue mode}
When using the legacy interface, transitional devices and drivers
MUST format \field{virtqueue_pairs}
according to the native endian of the guest rather than
(necessarily when not using the legacy interface) little-endian.

\subparagraph{Hash calculation}\label{sec:Device Types / Network Device / Device Operation / Control Virtqueue / Automatic receive steering in multiqueue mode / Hash calculation}
If VIRTIO_NET_F_HASH_REPORT was negotiated and the device uses automatic receive steering,
the device MUST support a command to configure hash calculation parameters.

The driver provides parameters for hash calculation as follows:

\field{class} VIRTIO_NET_CTRL_MQ, \field{command} VIRTIO_NET_CTRL_MQ_HASH_CONFIG.

The \field{command-specific-data} has following format:
\begin{lstlisting}
struct virtio_net_hash_config {
    le32 hash_types;
    le16 reserved[4];
    u8 hash_key_length;
    u8 hash_key_data[hash_key_length];
};
\end{lstlisting}
Field \field{hash_types} contains a bitmask of allowed hash types as
defined in
\ref{sec:Device Types / Network Device / Device Operation / Processing of Incoming Packets / Hash calculation for incoming packets / Supported/enabled hash types}.
Initially the device has all hash types disabled and reports only VIRTIO_NET_HASH_REPORT_NONE.

Field \field{reserved} MUST contain zeroes. It is defined to make the structure to match the layout of virtio_net_rss_config structure,
defined in \ref{sec:Device Types / Network Device / Device Operation / Control Virtqueue / Receive-side scaling (RSS)}.

Fields \field{hash_key_length} and \field{hash_key_data} define the key to be used in hash calculation.

\paragraph{Receive-side scaling (RSS)}\label{sec:Device Types / Network Device / Device Operation / Control Virtqueue / Receive-side scaling (RSS)}
A device offers the feature VIRTIO_NET_F_RSS if it supports RSS receive steering with Toeplitz hash calculation and configurable parameters.

A driver queries RSS capabilities of the device by reading device configuration as defined in \ref{sec:Device Types / Network Device / Device configuration layout}

\subparagraph{Setting RSS parameters}\label{sec:Device Types / Network Device / Device Operation / Control Virtqueue / Receive-side scaling (RSS) / Setting RSS parameters}

Driver sends a VIRTIO_NET_CTRL_MQ_RSS_CONFIG command using the following format for \field{command-specific-data}:
\begin{lstlisting}
struct rss_rq_id {
   le16 vq_index_1_16: 15; /* Bits 1 to 16 of the virtqueue index */
   le16 reserved: 1; /* Set to zero */
};

struct virtio_net_rss_config {
    le32 hash_types;
    le16 indirection_table_mask;
    struct rss_rq_id unclassified_queue;
    struct rss_rq_id indirection_table[indirection_table_length];
    le16 max_tx_vq;
    u8 hash_key_length;
    u8 hash_key_data[hash_key_length];
};
\end{lstlisting}
Field \field{hash_types} contains a bitmask of allowed hash types as
defined in
\ref{sec:Device Types / Network Device / Device Operation / Processing of Incoming Packets / Hash calculation for incoming packets / Supported/enabled hash types}.

Field \field{indirection_table_mask} is a mask to be applied to
the calculated hash to produce an index in the
\field{indirection_table} array.
Number of entries in \field{indirection_table} is (\field{indirection_table_mask} + 1).

\field{rss_rq_id} is a receive virtqueue id. \field{vq_index_1_16}
consists of bits 1 to 16 of a virtqueue index. For example, a
\field{vq_index_1_16} value of 3 corresponds to virtqueue index 6,
which maps to receiveq4.

Field \field{unclassified_queue} specifies the receive virtqueue id in which to
place unclassified packets.

Field \field{indirection_table} is an array of receive virtqueues ids.

A driver sets \field{max_tx_vq} to inform a device how many transmit virtqueues it may use (transmitq1\ldots transmitq \field{max_tx_vq}).

Fields \field{hash_key_length} and \field{hash_key_data} define the key to be used in hash calculation.

\drivernormative{\subparagraph}{Setting RSS parameters}{Device Types / Network Device / Device Operation / Control Virtqueue / Receive-side scaling (RSS) }

A driver MUST NOT send the VIRTIO_NET_CTRL_MQ_RSS_CONFIG command if the feature VIRTIO_NET_F_RSS has not been negotiated.

A driver MUST fill the \field{indirection_table} array only with
enabled receive virtqueues ids.

The number of entries in \field{indirection_table} (\field{indirection_table_mask} + 1) MUST be a power of two.

A driver MUST use \field{indirection_table_mask} values that are less than \field{rss_max_indirection_table_length} reported by a device.

A driver MUST NOT set any VIRTIO_NET_HASH_TYPE_ flags that are not supported by a device.

\devicenormative{\subparagraph}{RSS processing}{Device Types / Network Device / Device Operation / Control Virtqueue / Receive-side scaling (RSS) / RSS processing}
The device MUST determine the destination queue for a network packet as follows:
\begin{itemize}
\item Calculate the hash of the packet as defined in \ref{sec:Device Types / Network Device / Device Operation / Processing of Incoming Packets / Hash calculation for incoming packets}.
\item If the device did not calculate the hash for the specific packet, the device directs the packet to the receiveq specified by \field{unclassified_queue} of virtio_net_rss_config structure.
\item Apply \field{indirection_table_mask} to the calculated hash
and use the result as the index in the indirection table to get
the destination receive virtqueue id.
\item If the destination receive queue is being reset (See \ref{sec:Basic Facilities of a Virtio Device / Virtqueues / Virtqueue Reset}), the device MUST drop the packet.
\end{itemize}

\paragraph{RSS Context}\label{sec:Device Types / Network Device / Device Operation / Control Virtqueue / RSS Context}

An RSS context consists of configurable parameters specified by \ref{sec:Device Types / Network Device
/ Device Operation / Control Virtqueue / Receive-side scaling (RSS)}.

The RSS configuration supported by VIRTIO_NET_F_RSS is considered the default RSS configuration.

The device offers the feature VIRTIO_NET_F_RSS_CONTEXT if it supports one or multiple RSS contexts
(excluding the default RSS configuration) and configurable parameters.

\subparagraph{Querying RSS Context Capability}\label{sec:Device Types / Network Device / Device Operation / Control Virtqueue / RSS Context / Querying RSS Context Capability}

\begin{lstlisting}
#define VIRTNET_RSS_CTX_CTRL 9
 #define VIRTNET_RSS_CTX_CTRL_CAP_GET  0
 #define VIRTNET_RSS_CTX_CTRL_ADD      1
 #define VIRTNET_RSS_CTX_CTRL_MOD      2
 #define VIRTNET_RSS_CTX_CTRL_DEL      3

struct virtnet_rss_ctx_cap {
    le16 max_rss_contexts;
}
\end{lstlisting}

Field \field{max_rss_contexts} specifies the maximum number of RSS contexts \ref{sec:Device Types / Network Device /
Device Operation / Control Virtqueue / RSS Context} supported by the device.

The driver queries the RSS context capability of the device by sending the command VIRTNET_RSS_CTX_CTRL_CAP_GET
with the structure virtnet_rss_ctx_cap.

For the command VIRTNET_RSS_CTX_CTRL_CAP_GET, the structure virtnet_rss_ctx_cap is write-only for the device.

\subparagraph{Setting RSS Context Parameters}\label{sec:Device Types / Network Device / Device Operation / Control Virtqueue / RSS Context / Setting RSS Context Parameters}

\begin{lstlisting}
struct virtnet_rss_ctx_add_modify {
    le16 rss_ctx_id;
    u8 reserved[6];
    struct virtio_net_rss_config rss;
};

struct virtnet_rss_ctx_del {
    le16 rss_ctx_id;
};
\end{lstlisting}

RSS context parameters:
\begin{itemize}
\item  \field{rss_ctx_id}: ID of the specific RSS context.
\item  \field{rss}: RSS context parameters of the specific RSS context whose id is \field{rss_ctx_id}.
\end{itemize}

\field{reserved} is reserved and it is ignored by the device.

If the feature VIRTIO_NET_F_RSS_CONTEXT has been negotiated, the driver can send the following
VIRTNET_RSS_CTX_CTRL class commands:
\begin{enumerate}
\item VIRTNET_RSS_CTX_CTRL_ADD: use the structure virtnet_rss_ctx_add_modify to
       add an RSS context configured as \field{rss} and id as \field{rss_ctx_id} for the device.
\item VIRTNET_RSS_CTX_CTRL_MOD: use the structure virtnet_rss_ctx_add_modify to
       configure parameters of the RSS context whose id is \field{rss_ctx_id} as \field{rss} for the device.
\item VIRTNET_RSS_CTX_CTRL_DEL: use the structure virtnet_rss_ctx_del to delete
       the RSS context whose id is \field{rss_ctx_id} for the device.
\end{enumerate}

For commands VIRTNET_RSS_CTX_CTRL_ADD and VIRTNET_RSS_CTX_CTRL_MOD, the structure virtnet_rss_ctx_add_modify is read-only for the device.
For the command VIRTNET_RSS_CTX_CTRL_DEL, the structure virtnet_rss_ctx_del is read-only for the device.

\devicenormative{\subparagraph}{RSS Context}{Device Types / Network Device / Device Operation / Control Virtqueue / RSS Context}

The device MUST set \field{max_rss_contexts} to at least 1 if it offers VIRTIO_NET_F_RSS_CONTEXT.

Upon reset, the device MUST clear all previously configured RSS contexts.

\drivernormative{\subparagraph}{RSS Context}{Device Types / Network Device / Device Operation / Control Virtqueue / RSS Context}

The driver MUST have negotiated the VIRTIO_NET_F_RSS_CONTEXT feature when issuing the VIRTNET_RSS_CTX_CTRL class commands.

The driver MUST set \field{rss_ctx_id} to between 1 and \field{max_rss_contexts} inclusive.

The driver MUST NOT send the command VIRTIO_NET_CTRL_MQ_VQ_PAIRS_SET when the device has successfully configured at least one RSS context.

\paragraph{Offloads State Configuration}\label{sec:Device Types / Network Device / Device Operation / Control Virtqueue / Offloads State Configuration}

If the VIRTIO_NET_F_CTRL_GUEST_OFFLOADS feature is negotiated, the driver can
send control commands for dynamic offloads state configuration.

\subparagraph{Setting Offloads State}\label{sec:Device Types / Network Device / Device Operation / Control Virtqueue / Offloads State Configuration / Setting Offloads State}

To configure the offloads, the following layout structure and
definitions are used:

\begin{lstlisting}
le64 offloads;

#define VIRTIO_NET_F_GUEST_CSUM       1
#define VIRTIO_NET_F_GUEST_TSO4       7
#define VIRTIO_NET_F_GUEST_TSO6       8
#define VIRTIO_NET_F_GUEST_ECN        9
#define VIRTIO_NET_F_GUEST_UFO        10
#define VIRTIO_NET_F_GUEST_UDP_TUNNEL_GSO  46
#define VIRTIO_NET_F_GUEST_UDP_TUNNEL_GSO_CSUM 47
#define VIRTIO_NET_F_GUEST_USO4       54
#define VIRTIO_NET_F_GUEST_USO6       55

#define VIRTIO_NET_CTRL_GUEST_OFFLOADS       5
 #define VIRTIO_NET_CTRL_GUEST_OFFLOADS_SET   0
\end{lstlisting}

The class VIRTIO_NET_CTRL_GUEST_OFFLOADS has one command:
VIRTIO_NET_CTRL_GUEST_OFFLOADS_SET applies the new offloads configuration.

le64 value passed as command data is a bitmask, bits set define
offloads to be enabled, bits cleared - offloads to be disabled.

There is a corresponding device feature for each offload. Upon feature
negotiation corresponding offload gets enabled to preserve backward
compatibility.

\drivernormative{\subparagraph}{Setting Offloads State}{Device Types / Network Device / Device Operation / Control Virtqueue / Offloads State Configuration / Setting Offloads State}

A driver MUST NOT enable an offload for which the appropriate feature
has not been negotiated.

\subparagraph{Legacy Interface: Setting Offloads State}\label{sec:Device Types / Network Device / Device Operation / Control Virtqueue / Offloads State Configuration / Setting Offloads State / Legacy Interface: Setting Offloads State}
When using the legacy interface, transitional devices and drivers
MUST format \field{offloads}
according to the native endian of the guest rather than
(necessarily when not using the legacy interface) little-endian.


\paragraph{Notifications Coalescing}\label{sec:Device Types / Network Device / Device Operation / Control Virtqueue / Notifications Coalescing}

If the VIRTIO_NET_F_NOTF_COAL feature is negotiated, the driver can
send commands VIRTIO_NET_CTRL_NOTF_COAL_TX_SET and VIRTIO_NET_CTRL_NOTF_COAL_RX_SET
for notification coalescing.

If the VIRTIO_NET_F_VQ_NOTF_COAL feature is negotiated, the driver can
send commands VIRTIO_NET_CTRL_NOTF_COAL_VQ_SET and VIRTIO_NET_CTRL_NOTF_COAL_VQ_GET
for virtqueue notification coalescing.

\begin{lstlisting}
struct virtio_net_ctrl_coal {
    le32 max_packets;
    le32 max_usecs;
};

struct virtio_net_ctrl_coal_vq {
    le16 vq_index;
    le16 reserved;
    struct virtio_net_ctrl_coal coal;
};

#define VIRTIO_NET_CTRL_NOTF_COAL 6
 #define VIRTIO_NET_CTRL_NOTF_COAL_TX_SET  0
 #define VIRTIO_NET_CTRL_NOTF_COAL_RX_SET 1
 #define VIRTIO_NET_CTRL_NOTF_COAL_VQ_SET 2
 #define VIRTIO_NET_CTRL_NOTF_COAL_VQ_GET 3
\end{lstlisting}

Coalescing parameters:
\begin{itemize}
\item \field{vq_index}: The virtqueue index of an enabled transmit or receive virtqueue.
\item \field{max_usecs} for RX: Maximum number of microseconds to delay a RX notification.
\item \field{max_usecs} for TX: Maximum number of microseconds to delay a TX notification.
\item \field{max_packets} for RX: Maximum number of packets to receive before a RX notification.
\item \field{max_packets} for TX: Maximum number of packets to send before a TX notification.
\end{itemize}

\field{reserved} is reserved and it is ignored by the device.

Read/Write attributes for coalescing parameters:
\begin{itemize}
\item For commands VIRTIO_NET_CTRL_NOTF_COAL_TX_SET and VIRTIO_NET_CTRL_NOTF_COAL_RX_SET, the structure virtio_net_ctrl_coal is write-only for the driver.
\item For the command VIRTIO_NET_CTRL_NOTF_COAL_VQ_SET, the structure virtio_net_ctrl_coal_vq is write-only for the driver.
\item For the command VIRTIO_NET_CTRL_NOTF_COAL_VQ_GET, \field{vq_index} and \field{reserved} are write-only
      for the driver, and the structure virtio_net_ctrl_coal is read-only for the driver.
\end{itemize}

The class VIRTIO_NET_CTRL_NOTF_COAL has the following commands:
\begin{enumerate}
\item VIRTIO_NET_CTRL_NOTF_COAL_TX_SET: use the structure virtio_net_ctrl_coal to set the \field{max_usecs} and \field{max_packets} parameters for all transmit virtqueues.
\item VIRTIO_NET_CTRL_NOTF_COAL_RX_SET: use the structure virtio_net_ctrl_coal to set the \field{max_usecs} and \field{max_packets} parameters for all receive virtqueues.
\item VIRTIO_NET_CTRL_NOTF_COAL_VQ_SET: use the structure virtio_net_ctrl_coal_vq to set the \field{max_usecs} and \field{max_packets} parameters
                                        for an enabled transmit/receive virtqueue whose index is \field{vq_index}.
\item VIRTIO_NET_CTRL_NOTF_COAL_VQ_GET: use the structure virtio_net_ctrl_coal_vq to get the \field{max_usecs} and \field{max_packets} parameters
                                        for an enabled transmit/receive virtqueue whose index is \field{vq_index}.
\end{enumerate}

The device may generate notifications more or less frequently than specified by set commands of the VIRTIO_NET_CTRL_NOTF_COAL class.

If coalescing parameters are being set, the device applies the last coalescing parameters set for a
virtqueue, regardless of the command used to set the parameters. Use the following command sequence
with two pairs of virtqueues as an example:
Each of the following commands sets \field{max_usecs} and \field{max_packets} parameters for virtqueues.
\begin{itemize}
\item Command1: VIRTIO_NET_CTRL_NOTF_COAL_RX_SET sets coalescing parameters for virtqueues having index 0 and index 2. Virtqueues having index 1 and index 3 retain their previous parameters.
\item Command2: VIRTIO_NET_CTRL_NOTF_COAL_VQ_SET with \field{vq_index} = 0 sets coalescing parameters for virtqueue having index 0. Virtqueue having index 2 retains the parameters from command1.
\item Command3: VIRTIO_NET_CTRL_NOTF_COAL_VQ_GET with \field{vq_index} = 0, the device responds with coalescing parameters of vq_index 0 set by command2.
\item Command4: VIRTIO_NET_CTRL_NOTF_COAL_VQ_SET with \field{vq_index} = 1 sets coalescing parameters for virtqueue having index 1. Virtqueue having index 3 retains its previous parameters.
\item Command5: VIRTIO_NET_CTRL_NOTF_COAL_TX_SET sets coalescing parameters for virtqueues having index 1 and index 3, and overrides the parameters set by command4.
\item Command6: VIRTIO_NET_CTRL_NOTF_COAL_VQ_GET with \field{vq_index} = 1, the device responds with coalescing parameters of index 1 set by command5.
\end{itemize}

\subparagraph{Operation}\label{sec:Device Types / Network Device / Device Operation / Control Virtqueue / Notifications Coalescing / Operation}

The device sends a used buffer notification once the notification conditions are met and if the notifications are not suppressed as explained in \ref{sec:Basic Facilities of a Virtio Device / Virtqueues / Used Buffer Notification Suppression}.

When the device has non-zero \field{max_usecs} and non-zero \field{max_packets}, it starts counting microseconds and packets upon receiving/sending a packet.
The device counts packets and microseconds for each receive virtqueue and transmit virtqueue separately.
In this case, the notification conditions are met when \field{max_usecs} microseconds elapse, or upon sending/receiving \field{max_packets} packets, whichever happens first.
Afterwards, the device waits for the next packet and starts counting packets and microseconds again.

When the device has \field{max_usecs} = 0 or \field{max_packets} = 0, the notification conditions are met after every packet received/sent.

\subparagraph{RX Example}\label{sec:Device Types / Network Device / Device Operation / Control Virtqueue / Notifications Coalescing / RX Example}

If, for example:
\begin{itemize}
\item \field{max_usecs} = 10.
\item \field{max_packets} = 15.
\end{itemize}
then each receive virtqueue of a device will operate as follows:
\begin{itemize}
\item The device will count packets received on each virtqueue until it accumulates 15, or until 10 microseconds elapsed since the first one was received.
\item If the notifications are not suppressed by the driver, the device will send an used buffer notification, otherwise, the device will not send an used buffer notification as long as the notifications are suppressed.
\end{itemize}

\subparagraph{TX Example}\label{sec:Device Types / Network Device / Device Operation / Control Virtqueue / Notifications Coalescing / TX Example}

If, for example:
\begin{itemize}
\item \field{max_usecs} = 10.
\item \field{max_packets} = 15.
\end{itemize}
then each transmit virtqueue of a device will operate as follows:
\begin{itemize}
\item The device will count packets sent on each virtqueue until it accumulates 15, or until 10 microseconds elapsed since the first one was sent.
\item If the notifications are not suppressed by the driver, the device will send an used buffer notification, otherwise, the device will not send an used buffer notification as long as the notifications are suppressed.
\end{itemize}

\subparagraph{Notifications When Coalescing Parameters Change}\label{sec:Device Types / Network Device / Device Operation / Control Virtqueue / Notifications Coalescing / Notifications When Coalescing Parameters Change}

When the coalescing parameters of a device change, the device needs to check if the new notification conditions are met and send a used buffer notification if so.

For example, \field{max_packets} = 15 for a device with a single transmit virtqueue: if the device sends 10 packets and afterwards receives a
VIRTIO_NET_CTRL_NOTF_COAL_TX_SET command with \field{max_packets} = 8, then the notification condition is immediately considered to be met;
the device needs to immediately send a used buffer notification, if the notifications are not suppressed by the driver.

\drivernormative{\subparagraph}{Notifications Coalescing}{Device Types / Network Device / Device Operation / Control Virtqueue / Notifications Coalescing}

The driver MUST set \field{vq_index} to the virtqueue index of an enabled transmit or receive virtqueue.

The driver MUST have negotiated the VIRTIO_NET_F_NOTF_COAL feature when issuing commands VIRTIO_NET_CTRL_NOTF_COAL_TX_SET and VIRTIO_NET_CTRL_NOTF_COAL_RX_SET.

The driver MUST have negotiated the VIRTIO_NET_F_VQ_NOTF_COAL feature when issuing commands VIRTIO_NET_CTRL_NOTF_COAL_VQ_SET and VIRTIO_NET_CTRL_NOTF_COAL_VQ_GET.

The driver MUST ignore the values of coalescing parameters received from the VIRTIO_NET_CTRL_NOTF_COAL_VQ_GET command if the device responds with VIRTIO_NET_ERR.

\devicenormative{\subparagraph}{Notifications Coalescing}{Device Types / Network Device / Device Operation / Control Virtqueue / Notifications Coalescing}

The device MUST ignore \field{reserved}.

The device SHOULD respond to VIRTIO_NET_CTRL_NOTF_COAL_TX_SET and VIRTIO_NET_CTRL_NOTF_COAL_RX_SET commands with VIRTIO_NET_ERR if it was not able to change the parameters.

The device MUST respond to the VIRTIO_NET_CTRL_NOTF_COAL_VQ_SET command with VIRTIO_NET_ERR if it was not able to change the parameters.

The device MUST respond to VIRTIO_NET_CTRL_NOTF_COAL_VQ_SET and VIRTIO_NET_CTRL_NOTF_COAL_VQ_GET commands with
VIRTIO_NET_ERR if the designated virtqueue is not an enabled transmit or receive virtqueue.

Upon disabling and re-enabling a transmit virtqueue, the device MUST set the coalescing parameters of the virtqueue
to those configured through the VIRTIO_NET_CTRL_NOTF_COAL_TX_SET command, or, if the driver did not set any TX coalescing parameters, to 0.

Upon disabling and re-enabling a receive virtqueue, the device MUST set the coalescing parameters of the virtqueue
to those configured through the VIRTIO_NET_CTRL_NOTF_COAL_RX_SET command, or, if the driver did not set any RX coalescing parameters, to 0.

The behavior of the device in response to set commands of the VIRTIO_NET_CTRL_NOTF_COAL class is best-effort:
the device MAY generate notifications more or less frequently than specified.

A device SHOULD NOT send used buffer notifications to the driver if the notifications are suppressed, even if the notification conditions are met.

Upon reset, a device MUST initialize all coalescing parameters to 0.

\paragraph{Device Statistics}\label{sec:Device Types / Network Device / Device Operation / Control Virtqueue / Device Statistics}

If the VIRTIO_NET_F_DEVICE_STATS feature is negotiated, the driver can obtain
device statistics from the device by using the following command.

Different types of virtqueues have different statistics. The statistics of the
receiveq are different from those of the transmitq.

The statistics of a certain type of virtqueue are also divided into multiple types
because different types require different features. This enables the expansion
of new statistics.

In one command, the driver can obtain the statistics of one or multiple virtqueues.
Additionally, the driver can obtain multiple type statistics of each virtqueue.

\subparagraph{Query Statistic Capabilities}\label{sec:Device Types / Network Device / Device Operation / Control Virtqueue / Device Statistics / Query Statistic Capabilities}

\begin{lstlisting}
#define VIRTIO_NET_CTRL_STATS         8
#define VIRTIO_NET_CTRL_STATS_QUERY   0
#define VIRTIO_NET_CTRL_STATS_GET     1

struct virtio_net_stats_capabilities {

#define VIRTIO_NET_STATS_TYPE_CVQ       (1 << 32)

#define VIRTIO_NET_STATS_TYPE_RX_BASIC  (1 << 0)
#define VIRTIO_NET_STATS_TYPE_RX_CSUM   (1 << 1)
#define VIRTIO_NET_STATS_TYPE_RX_GSO    (1 << 2)
#define VIRTIO_NET_STATS_TYPE_RX_SPEED  (1 << 3)

#define VIRTIO_NET_STATS_TYPE_TX_BASIC  (1 << 16)
#define VIRTIO_NET_STATS_TYPE_TX_CSUM   (1 << 17)
#define VIRTIO_NET_STATS_TYPE_TX_GSO    (1 << 18)
#define VIRTIO_NET_STATS_TYPE_TX_SPEED  (1 << 19)

    le64 supported_stats_types[1];
}
\end{lstlisting}

To obtain device statistic capability, use the VIRTIO_NET_CTRL_STATS_QUERY
command. When the command completes successfully, \field{command-specific-result}
is in the format of \field{struct virtio_net_stats_capabilities}.

\subparagraph{Get Statistics}\label{sec:Device Types / Network Device / Device Operation / Control Virtqueue / Device Statistics / Get Statistics}

\begin{lstlisting}
struct virtio_net_ctrl_queue_stats {
       struct {
           le16 vq_index;
           le16 reserved[3];
           le64 types_bitmap[1];
       } stats[];
};

struct virtio_net_stats_reply_hdr {
#define VIRTIO_NET_STATS_TYPE_REPLY_CVQ       32

#define VIRTIO_NET_STATS_TYPE_REPLY_RX_BASIC  0
#define VIRTIO_NET_STATS_TYPE_REPLY_RX_CSUM   1
#define VIRTIO_NET_STATS_TYPE_REPLY_RX_GSO    2
#define VIRTIO_NET_STATS_TYPE_REPLY_RX_SPEED  3

#define VIRTIO_NET_STATS_TYPE_REPLY_TX_BASIC  16
#define VIRTIO_NET_STATS_TYPE_REPLY_TX_CSUM   17
#define VIRTIO_NET_STATS_TYPE_REPLY_TX_GSO    18
#define VIRTIO_NET_STATS_TYPE_REPLY_TX_SPEED  19
    u8 type;
    u8 reserved;
    le16 vq_index;
    le16 reserved1;
    le16 size;
}
\end{lstlisting}

To obtain device statistics, use the VIRTIO_NET_CTRL_STATS_GET command with the
\field{command-specific-data} which is in the format of
\field{struct virtio_net_ctrl_queue_stats}. When the command completes
successfully, \field{command-specific-result} contains multiple statistic
results, each statistic result has the \field{struct virtio_net_stats_reply_hdr}
as the header.

The fields of the \field{struct virtio_net_ctrl_queue_stats}:
\begin{description}
    \item [vq_index]
        The index of the virtqueue to obtain the statistics.

    \item [types_bitmap]
        This is a bitmask of the types of statistics to be obtained. Therefore, a
        \field{stats} inside \field{struct virtio_net_ctrl_queue_stats} may
        indicate multiple statistic replies for the virtqueue.
\end{description}

The fields of the \field{struct virtio_net_stats_reply_hdr}:
\begin{description}
    \item [type]
        The type of the reply statistic.

    \item [vq_index]
        The virtqueue index of the reply statistic.

    \item [size]
        The number of bytes for the statistics entry including size of \field{struct virtio_net_stats_reply_hdr}.

\end{description}

\subparagraph{Controlq Statistics}\label{sec:Device Types / Network Device / Device Operation / Control Virtqueue / Device Statistics / Controlq Statistics}

The structure corresponding to the controlq statistics is
\field{struct virtio_net_stats_cvq}. The corresponding type is
VIRTIO_NET_STATS_TYPE_CVQ. This is for the controlq.

\begin{lstlisting}
struct virtio_net_stats_cvq {
    struct virtio_net_stats_reply_hdr hdr;

    le64 command_num;
    le64 ok_num;
};
\end{lstlisting}

\begin{description}
    \item [command_num]
        The number of commands received by the device including the current command.

    \item [ok_num]
        The number of commands completed successfully by the device including the current command.
\end{description}


\subparagraph{Receiveq Basic Statistics}\label{sec:Device Types / Network Device / Device Operation / Control Virtqueue / Device Statistics / Receiveq Basic Statistics}

The structure corresponding to the receiveq basic statistics is
\field{struct virtio_net_stats_rx_basic}. The corresponding type is
VIRTIO_NET_STATS_TYPE_RX_BASIC. This is for the receiveq.

Receiveq basic statistics do not require any feature. As long as the device supports
VIRTIO_NET_F_DEVICE_STATS, the following are the receiveq basic statistics.

\begin{lstlisting}
struct virtio_net_stats_rx_basic {
    struct virtio_net_stats_reply_hdr hdr;

    le64 rx_notifications;

    le64 rx_packets;
    le64 rx_bytes;

    le64 rx_interrupts;

    le64 rx_drops;
    le64 rx_drop_overruns;
};
\end{lstlisting}

The packets described below were all presented on the specified virtqueue.
\begin{description}
    \item [rx_notifications]
        The number of driver notifications received by the device for this
        receiveq.

    \item [rx_packets]
        This is the number of packets passed to the driver by the device.

    \item [rx_bytes]
        This is the bytes of packets passed to the driver by the device.

    \item [rx_interrupts]
        The number of interrupts generated by the device for this receiveq.

    \item [rx_drops]
        This is the number of packets dropped by the device. The count includes
        all types of packets dropped by the device.

    \item [rx_drop_overruns]
        This is the number of packets dropped by the device when no more
        descriptors were available.

\end{description}

\subparagraph{Transmitq Basic Statistics}\label{sec:Device Types / Network Device / Device Operation / Control Virtqueue / Device Statistics / Transmitq Basic Statistics}

The structure corresponding to the transmitq basic statistics is
\field{struct virtio_net_stats_tx_basic}. The corresponding type is
VIRTIO_NET_STATS_TYPE_TX_BASIC. This is for the transmitq.

Transmitq basic statistics do not require any feature. As long as the device supports
VIRTIO_NET_F_DEVICE_STATS, the following are the transmitq basic statistics.

\begin{lstlisting}
struct virtio_net_stats_tx_basic {
    struct virtio_net_stats_reply_hdr hdr;

    le64 tx_notifications;

    le64 tx_packets;
    le64 tx_bytes;

    le64 tx_interrupts;

    le64 tx_drops;
    le64 tx_drop_malformed;
};
\end{lstlisting}

The packets described below are all for a specific virtqueue.
\begin{description}
    \item [tx_notifications]
        The number of driver notifications received by the device for this
        transmitq.

    \item [tx_packets]
        This is the number of packets sent by the device (not the packets
        got from the driver).

    \item [tx_bytes]
        This is the number of bytes sent by the device for all the sent packets
        (not the bytes sent got from the driver).

    \item [tx_interrupts]
        The number of interrupts generated by the device for this transmitq.

    \item [tx_drops]
        The number of packets dropped by the device. The count includes all
        types of packets dropped by the device.

    \item [tx_drop_malformed]
        The number of packets dropped by the device, when the descriptors are
        malformed. For example, the buffer is too short.
\end{description}

\subparagraph{Receiveq CSUM Statistics}\label{sec:Device Types / Network Device / Device Operation / Control Virtqueue / Device Statistics / Receiveq CSUM Statistics}

The structure corresponding to the receiveq checksum statistics is
\field{struct virtio_net_stats_rx_csum}. The corresponding type is
VIRTIO_NET_STATS_TYPE_RX_CSUM. This is for the receiveq.

Only after the VIRTIO_NET_F_GUEST_CSUM is negotiated, the receiveq checksum
statistics can be obtained.

\begin{lstlisting}
struct virtio_net_stats_rx_csum {
    struct virtio_net_stats_reply_hdr hdr;

    le64 rx_csum_valid;
    le64 rx_needs_csum;
    le64 rx_csum_none;
    le64 rx_csum_bad;
};
\end{lstlisting}

The packets described below were all presented on the specified virtqueue.
\begin{description}
    \item [rx_csum_valid]
        The number of packets with VIRTIO_NET_HDR_F_DATA_VALID.

    \item [rx_needs_csum]
        The number of packets with VIRTIO_NET_HDR_F_NEEDS_CSUM.

    \item [rx_csum_none]
        The number of packets without hardware checksum. The packet here refers
        to the non-TCP/UDP packet that the device cannot recognize.

    \item [rx_csum_bad]
        The number of packets with checksum mismatch.

\end{description}

\subparagraph{Transmitq CSUM Statistics}\label{sec:Device Types / Network Device / Device Operation / Control Virtqueue / Device Statistics / Transmitq CSUM Statistics}

The structure corresponding to the transmitq checksum statistics is
\field{struct virtio_net_stats_tx_csum}. The corresponding type is
VIRTIO_NET_STATS_TYPE_TX_CSUM. This is for the transmitq.

Only after the VIRTIO_NET_F_CSUM is negotiated, the transmitq checksum
statistics can be obtained.

The following are the transmitq checksum statistics:

\begin{lstlisting}
struct virtio_net_stats_tx_csum {
    struct virtio_net_stats_reply_hdr hdr;

    le64 tx_csum_none;
    le64 tx_needs_csum;
};
\end{lstlisting}

The packets described below are all for a specific virtqueue.
\begin{description}
    \item [tx_csum_none]
        The number of packets which do not require hardware checksum.

    \item [tx_needs_csum]
        The number of packets which require checksum calculation by the device.

\end{description}

\subparagraph{Receiveq GSO Statistics}\label{sec:Device Types / Network Device / Device Operation / Control Virtqueue / Device Statistics / Receiveq GSO Statistics}

The structure corresponding to the receivq GSO statistics is
\field{struct virtio_net_stats_rx_gso}. The corresponding type is
VIRTIO_NET_STATS_TYPE_RX_GSO. This is for the receiveq.

If one or more of the VIRTIO_NET_F_GUEST_TSO4, VIRTIO_NET_F_GUEST_TSO6
have been negotiated, the receiveq GSO statistics can be obtained.

GSO packets refer to packets passed by the device to the driver where
\field{gso_type} is not VIRTIO_NET_HDR_GSO_NONE.

\begin{lstlisting}
struct virtio_net_stats_rx_gso {
    struct virtio_net_stats_reply_hdr hdr;

    le64 rx_gso_packets;
    le64 rx_gso_bytes;
    le64 rx_gso_packets_coalesced;
    le64 rx_gso_bytes_coalesced;
};
\end{lstlisting}

The packets described below were all presented on the specified virtqueue.
\begin{description}
    \item [rx_gso_packets]
        The number of the GSO packets received by the device.

    \item [rx_gso_bytes]
        The bytes of the GSO packets received by the device.
        This includes the header size of the GSO packet.

    \item [rx_gso_packets_coalesced]
        The number of the GSO packets coalesced by the device.

    \item [rx_gso_bytes_coalesced]
        The bytes of the GSO packets coalesced by the device.
        This includes the header size of the GSO packet.
\end{description}

\subparagraph{Transmitq GSO Statistics}\label{sec:Device Types / Network Device / Device Operation / Control Virtqueue / Device Statistics / Transmitq GSO Statistics}

The structure corresponding to the transmitq GSO statistics is
\field{struct virtio_net_stats_tx_gso}. The corresponding type is
VIRTIO_NET_STATS_TYPE_TX_GSO. This is for the transmitq.

If one or more of the VIRTIO_NET_F_HOST_TSO4, VIRTIO_NET_F_HOST_TSO6,
VIRTIO_NET_F_HOST_USO options have been negotiated, the transmitq GSO statistics
can be obtained.

GSO packets refer to packets passed by the driver to the device where
\field{gso_type} is not VIRTIO_NET_HDR_GSO_NONE.
See more \ref{sec:Device Types / Network Device / Device Operation / Packet
Transmission}.

\begin{lstlisting}
struct virtio_net_stats_tx_gso {
    struct virtio_net_stats_reply_hdr hdr;

    le64 tx_gso_packets;
    le64 tx_gso_bytes;
    le64 tx_gso_segments;
    le64 tx_gso_segments_bytes;
    le64 tx_gso_packets_noseg;
    le64 tx_gso_bytes_noseg;
};
\end{lstlisting}

The packets described below are all for a specific virtqueue.
\begin{description}
    \item [tx_gso_packets]
        The number of the GSO packets sent by the device.

    \item [tx_gso_bytes]
        The bytes of the GSO packets sent by the device.

    \item [tx_gso_segments]
        The number of segments prepared from GSO packets.

    \item [tx_gso_segments_bytes]
        The bytes of segments prepared from GSO packets.

    \item [tx_gso_packets_noseg]
        The number of the GSO packets without segmentation.

    \item [tx_gso_bytes_noseg]
        The bytes of the GSO packets without segmentation.

\end{description}

\subparagraph{Receiveq Speed Statistics}\label{sec:Device Types / Network Device / Device Operation / Control Virtqueue / Device Statistics / Receiveq Speed Statistics}

The structure corresponding to the receiveq speed statistics is
\field{struct virtio_net_stats_rx_speed}. The corresponding type is
VIRTIO_NET_STATS_TYPE_RX_SPEED. This is for the receiveq.

The device has the allowance for the speed. If VIRTIO_NET_F_SPEED_DUPLEX has
been negotiated, the driver can get this by \field{speed}. When the received
packets bitrate exceeds the \field{speed}, some packets may be dropped by the
device.

\begin{lstlisting}
struct virtio_net_stats_rx_speed {
    struct virtio_net_stats_reply_hdr hdr;

    le64 rx_packets_allowance_exceeded;
    le64 rx_bytes_allowance_exceeded;
};
\end{lstlisting}

The packets described below were all presented on the specified virtqueue.
\begin{description}
    \item [rx_packets_allowance_exceeded]
        The number of the packets dropped by the device due to the received
        packets bitrate exceeding the \field{speed}.

    \item [rx_bytes_allowance_exceeded]
        The bytes of the packets dropped by the device due to the received
        packets bitrate exceeding the \field{speed}.

\end{description}

\subparagraph{Transmitq Speed Statistics}\label{sec:Device Types / Network Device / Device Operation / Control Virtqueue / Device Statistics / Transmitq Speed Statistics}

The structure corresponding to the transmitq speed statistics is
\field{struct virtio_net_stats_tx_speed}. The corresponding type is
VIRTIO_NET_STATS_TYPE_TX_SPEED. This is for the transmitq.

The device has the allowance for the speed. If VIRTIO_NET_F_SPEED_DUPLEX has
been negotiated, the driver can get this by \field{speed}. When the transmit
packets bitrate exceeds the \field{speed}, some packets may be dropped by the
device.

\begin{lstlisting}
struct virtio_net_stats_tx_speed {
    struct virtio_net_stats_reply_hdr hdr;

    le64 tx_packets_allowance_exceeded;
    le64 tx_bytes_allowance_exceeded;
};
\end{lstlisting}

The packets described below were all presented on the specified virtqueue.
\begin{description}
    \item [tx_packets_allowance_exceeded]
        The number of the packets dropped by the device due to the transmit packets
        bitrate exceeding the \field{speed}.

    \item [tx_bytes_allowance_exceeded]
        The bytes of the packets dropped by the device due to the transmit packets
        bitrate exceeding the \field{speed}.

\end{description}

\devicenormative{\subparagraph}{Device Statistics}{Device Types / Network Device / Device Operation / Control Virtqueue / Device Statistics}

When the VIRTIO_NET_F_DEVICE_STATS feature is negotiated, the device MUST reply
to the command VIRTIO_NET_CTRL_STATS_QUERY with the
\field{struct virtio_net_stats_capabilities}. \field{supported_stats_types}
includes all the statistic types supported by the device.

If \field{struct virtio_net_ctrl_queue_stats} is incorrect (such as the
following), the device MUST set \field{ack} to VIRTIO_NET_ERR. Even if there is
only one error, the device MUST fail the entire command.
\begin{itemize}
    \item \field{vq_index} exceeds the queue range.
    \item \field{types_bitmap} contains unknown types.
    \item One or more of the bits present in \field{types_bitmap} is not valid
        for the specified virtqueue.
    \item The feature corresponding to the specified \field{types_bitmap} was
        not negotiated.
\end{itemize}

The device MUST set the actual size of the bytes occupied by the reply to the
\field{size} of the \field{hdr}. And the device MUST set the \field{type} and
the \field{vq_index} of the statistic header.

The \field{command-specific-result} buffer allocated by the driver may be
smaller or bigger than all the statistics specified by
\field{struct virtio_net_ctrl_queue_stats}. The device MUST fill up only upto
the valid bytes.

The statistics counter replied by the device MUST wrap around to zero by the
device on the overflow.

\drivernormative{\subparagraph}{Device Statistics}{Device Types / Network Device / Device Operation / Control Virtqueue / Device Statistics}

The types contained in the \field{types_bitmap} MUST be queried from the device
via command VIRTIO_NET_CTRL_STATS_QUERY.

\field{types_bitmap} in \field{struct virtio_net_ctrl_queue_stats} MUST be valid to the
vq specified by \field{vq_index}.

The \field{command-specific-result} buffer allocated by the driver MUST have
enough capacity to store all the statistics reply headers defined in
\field{struct virtio_net_ctrl_queue_stats}. If the
\field{command-specific-result} buffer is fully utilized by the device but some
replies are missed, it is possible that some statistics may exceed the capacity
of the driver's records. In such cases, the driver should allocate additional
space for the \field{command-specific-result} buffer.

\subsubsection{Flow filter}\label{sec:Device Types / Network Device / Device Operation / Flow filter}

A network device can support one or more flow filter rules. Each flow filter rule
is applied by matching a packet and then taking an action, such as directing the packet
to a specific receiveq or dropping the packet. An example of a match is
matching on specific source and destination IP addresses.

A flow filter rule is a device resource object that consists of a key,
a processing priority, and an action to either direct a packet to a
receive queue or drop the packet.

Each rule uses a classifier. The key is matched against the packet using
a classifier, defining which fields in the packet are matched.
A classifier resource object consists of one or more field selectors, each with
a type that specifies the header fields to be matched against, and a mask.
The mask can match whole fields or parts of a field in a header. Each
rule resource object depends on the classifier resource object.

When a packet is received, relevant fields are extracted
(in the same way) from both the packet and the key according to the
classifier. The resulting field contents are then compared -
if they are identical the rule action is taken, if they are not, the rule is ignored.

Multiple flow filter rules are part of a group. The rule resource object
depends on the group. Each rule within a
group has a rule priority, and each group also has a group priority. For a
packet, a group with the highest priority is selected first. Within a group,
rules are applied from highest to lowest priority, until one of the rules
matches the packet and an action is taken. If all the rules within a group
are ignored, the group with the next highest priority is selected, and so on.

The device and the driver indicates flow filter resource limits using the capability
\ref{par:Device Types / Network Device / Device Operation / Flow filter / Device and driver capabilities / VIRTIO-NET-FF-RESOURCE-CAP} specifying the limits on the number of flow filter rule,
group and classifier resource objects. The capability \ref{par:Device Types / Network Device / Device Operation / Flow filter / Device and driver capabilities / VIRTIO-NET-FF-SELECTOR-CAP} specifies which selectors the device supports.
The driver indicates the selectors it is using by setting the flow
filter selector capability, prior to adding any resource objects.

The capability \ref{par:Device Types / Network Device / Device Operation / Flow filter / Device and driver capabilities / VIRTIO-NET-FF-ACTION-CAP} specifies which actions the device supports.

The driver controls the flow filter rule, classifier and group resource objects using
administration commands described in
\ref{sec:Basic Facilities of a Virtio Device / Device groups / Group administration commands / Device resource objects}.

\paragraph{Packet processing order}\label{sec:sec:Device Types / Network Device / Device Operation / Flow filter / Packet processing order}

Note that flow filter rules are applied after MAC/VLAN filtering. Flow filter
rules take precedence over steering: if a flow filter rule results in an action,
the steering configuration does not apply. The steering configuration only applies
to packets for which no flow filter rule action was performed. For example,
incoming packets can be processed in the following order:

\begin{itemize}
\item apply steering configuration received using control virtqueue commands
      VIRTIO_NET_CTRL_RX, VIRTIO_NET_CTRL_MAC and VIRTIO_NET_CTRL_VLAN.
\item apply flow filter rules if any.
\item if no filter rule applied, apply steering configuration received using command
      VIRTIO_NET_CTRL_MQ_RSS_CONFIG or as per automatic receive steering.
\end{itemize}

Some incoming packet processing examples:
\begin{itemize}
\item If the packet is dropped by the flow filter rule, RSS
      steering is ignored for the packet.
\item If the packet is directed to a specific receiveq using flow filter rule,
      the RSS steering is ignored for the packet.
\item If a packet is dropped due to the VIRTIO_NET_CTRL_MAC configuration,
      both flow filter rules and the RSS steering are ignored for the packet.
\item If a packet does not match any flow filter rules,
      the RSS steering is used to select the receiveq for the packet (if enabled).
\item If there are two flow filter groups configured as group_A and group_B
      with respective group priorities as 4, and 5; flow filter rules of
      group_B are applied first having highest group priority, if there is a match,
      the flow filter rules of group_A are ignored; if there is no match for
      the flow filter rules in group_B, the flow filter rules of next level group_A are applied.
\end{itemize}

\paragraph{Device and driver capabilities}
\label{par:Device Types / Network Device / Device Operation / Flow filter / Device and driver capabilities}

\subparagraph{VIRTIO_NET_FF_RESOURCE_CAP}
\label{par:Device Types / Network Device / Device Operation / Flow filter / Device and driver capabilities / VIRTIO-NET-FF-RESOURCE-CAP}

The capability VIRTIO_NET_FF_RESOURCE_CAP indicates the flow filter resource limits.
\field{cap_specific_data} is in the format
\field{struct virtio_net_ff_cap_data}.

\begin{lstlisting}
struct virtio_net_ff_cap_data {
        le32 groups_limit;
        le32 selectors_limit;
        le32 rules_limit;
        le32 rules_per_group_limit;
        u8 last_rule_priority;
        u8 selectors_per_classifier_limit;
};
\end{lstlisting}

\field{groups_limit}, and \field{selectors_limit} represent the maximum
number of flow filter groups and selectors, respectively, that the driver can create.
 \field{rules_limit} is the maximum number of
flow fiilter rules that the driver can create across all the groups.
\field{rules_per_group_limit} is the maximum number of flow filter rules that the driver
can create for each flow filter group.

\field{last_rule_priority} is the highest priority that can be assigned to a
flow filter rule.

\field{selectors_per_classifier_limit} is the maximum number of selectors
that a classifier can have.

\subparagraph{VIRTIO_NET_FF_SELECTOR_CAP}
\label{par:Device Types / Network Device / Device Operation / Flow filter / Device and driver capabilities / VIRTIO-NET-FF-SELECTOR-CAP}

The capability VIRTIO_NET_FF_SELECTOR_CAP lists the supported selectors and the
supported packet header fields for each selector.
\field{cap_specific_data} is in the format \field{struct virtio_net_ff_cap_mask_data}.

\begin{lstlisting}[label={lst:Device Types / Network Device / Device Operation / Flow filter / Device and driver capabilities / VIRTIO-NET-FF-SELECTOR-CAP / virtio-net-ff-selector}]
struct virtio_net_ff_selector {
        u8 type;
        u8 flags;
        u8 reserved[2];
        u8 length;
        u8 reserved1[3];
        u8 mask[];
};

struct virtio_net_ff_cap_mask_data {
        u8 count;
        u8 reserved[7];
        struct virtio_net_ff_selector selectors[];
};

#define VIRTIO_NET_FF_MASK_F_PARTIAL_MASK (1 << 0)
\end{lstlisting}

\field{count} indicates number of valid entries in the \field{selectors} array.
\field{selectors[]} is an array of supported selectors. Within each array entry:
\field{type} specifies the type of the packet header, as defined in table
\ref{table:Device Types / Network Device / Device Operation / Flow filter / Device and driver capabilities / VIRTIO-NET-FF-SELECTOR-CAP / flow filter selector types}. \field{mask} specifies which fields of the
packet header can be matched in a flow filter rule.

Each \field{type} is also listed in table
\ref{table:Device Types / Network Device / Device Operation / Flow filter / Device and driver capabilities / VIRTIO-NET-FF-SELECTOR-CAP / flow filter selector types}. \field{mask} is a byte array
in network byte order. For example, when \field{type} is VIRTIO_NET_FF_MASK_TYPE_IPV6,
the \field{mask} is in the format \hyperref[intro:IPv6-Header-Format]{IPv6 Header Format}.

If partial masking is not set, then all bits in each field have to be either all 0s
to ignore this field or all 1s to match on this field. If partial masking is set,
then any combination of bits can bit set to match on these bits.
For example, when a selector \field{type} is VIRTIO_NET_FF_MASK_TYPE_ETH, if
\field{mask[0-12]} are zero and \field{mask[13-14]} are 0xff (all 1s), it
indicates that matching is only supported for \field{EtherType} of
\field{Ethernet MAC frame}, matching is not supported for
\field{Destination Address} and \field{Source Address}.

The entries in the array \field{selectors} are ordered by
\field{type}, with each \field{type} value only appearing once.

\field{length} is the length of a dynamic array \field{mask} in bytes.
\field{reserved} and \field{reserved1} are reserved and set to zero.

\begin{table}[H]
\caption{Flow filter selector types}
\label{table:Device Types / Network Device / Device Operation / Flow filter / Device and driver capabilities / VIRTIO-NET-FF-SELECTOR-CAP / flow filter selector types}
\begin{tabularx}{\textwidth}{ |l|X|X| }
\hline
Type & Name & Description \\
\hline \hline
0x0 & - & Reserved \\
\hline
0x1 & VIRTIO_NET_FF_MASK_TYPE_ETH & 14 bytes of frame header starting from destination address described in \hyperref[intro:IEEE 802.3-2022]{IEEE 802.3-2022} \\
\hline
0x2 & VIRTIO_NET_FF_MASK_TYPE_IPV4 & 20 bytes of \hyperref[intro:Internet-Header-Format]{IPv4: Internet Header Format} \\
\hline
0x3 & VIRTIO_NET_FF_MASK_TYPE_IPV6 & 40 bytes of \hyperref[intro:IPv6-Header-Format]{IPv6 Header Format} \\
\hline
0x4 & VIRTIO_NET_FF_MASK_TYPE_TCP & 20 bytes of \hyperref[intro:TCP-Header-Format]{TCP Header Format} \\
\hline
0x5 & VIRTIO_NET_FF_MASK_TYPE_UDP & 8 bytes of UDP header described in \hyperref[intro:UDP]{UDP} \\
\hline
0x6 - 0xFF & & Reserved for future \\
\hline
\end{tabularx}
\end{table}

When VIRTIO_NET_FF_MASK_F_PARTIAL_MASK (bit 0) is set, it indicates that
partial masking is supported for all the fields of the selector identified by \field{type}.

For the selector \field{type} VIRTIO_NET_FF_MASK_TYPE_IPV4, if a partial mask is unsupported,
then matching on an individual bit of \field{Flags} in the
\field{IPv4: Internet Header Format} is unsupported. \field{Flags} has to match as a whole
if it is supported.

For the selector \field{type} VIRTIO_NET_FF_MASK_TYPE_IPV4, \field{mask} includes fields
up to the \field{Destination Address}; that is, \field{Options} and
\field{Padding} are excluded.

For the selector \field{type} VIRTIO_NET_FF_MASK_TYPE_IPV6, the \field{Next Header} field
of the \field{mask} corresponds to the \field{Next Header} in the packet
when \field{IPv6 Extension Headers} are not present. When the packet includes
one or more \field{IPv6 Extension Headers}, the \field{Next Header} field of
the \field{mask} corresponds to the \field{Next Header} of the last
\field{IPv6 Extension Header} in the packet.

For the selector \field{type} VIRTIO_NET_FF_MASK_TYPE_TCP, \field{Control bits}
are treated as individual fields for matching; that is, matching individual
\field{Control bits} does not depend on the partial mask support.

\subparagraph{VIRTIO_NET_FF_ACTION_CAP}
\label{par:Device Types / Network Device / Device Operation / Flow filter / Device and driver capabilities / VIRTIO-NET-FF-ACTION-CAP}

The capability VIRTIO_NET_FF_ACTION_CAP lists the supported actions in a rule.
\field{cap_specific_data} is in the format \field{struct virtio_net_ff_cap_actions}.

\begin{lstlisting}
struct virtio_net_ff_actions {
        u8 count;
        u8 reserved[7];
        u8 actions[];
};
\end{lstlisting}

\field{actions} is an array listing all possible actions.
The entries in the array are ordered from the smallest to the largest,
with each supported value appearing exactly once. Each entry can have the
following values:

\begin{table}[H]
\caption{Flow filter rule actions}
\label{table:Device Types / Network Device / Device Operation / Flow filter / Device and driver capabilities / VIRTIO-NET-FF-ACTION-CAP / flow filter rule actions}
\begin{tabularx}{\textwidth}{ |l|X|X| }
\hline
Action & Name & Description \\
\hline \hline
0x0 & - & reserved \\
\hline
0x1 & VIRTIO_NET_FF_ACTION_DROP & Matching packet will be dropped by the device \\
\hline
0x2 & VIRTIO_NET_FF_ACTION_DIRECT_RX_VQ & Matching packet will be directed to a receive queue \\
\hline
0x3 - 0xFF & & Reserved for future \\
\hline
\end{tabularx}
\end{table}

\paragraph{Resource objects}
\label{par:Device Types / Network Device / Device Operation / Flow filter / Resource objects}

\subparagraph{VIRTIO_NET_RESOURCE_OBJ_FF_GROUP}\label{par:Device Types / Network Device / Device Operation / Flow filter / Resource objects / VIRTIO-NET-RESOURCE-OBJ-FF-GROUP}

A flow filter group contains between 0 and \field{rules_limit} rules, as specified by the
capability VIRTIO_NET_FF_RESOURCE_CAP. For the flow filter group object both
\field{resource_obj_specific_data} and
\field{resource_obj_specific_result} are in the format
\field{struct virtio_net_resource_obj_ff_group}.

\begin{lstlisting}
struct virtio_net_resource_obj_ff_group {
        le16 group_priority;
};
\end{lstlisting}

\field{group_priority} specifies the priority for the group. Each group has a
distinct priority. For each incoming packet, the device tries to apply rules
from groups from higher \field{group_priority} value to lower, until either a
rule matches the packet or all groups have been tried.

\subparagraph{VIRTIO_NET_RESOURCE_OBJ_FF_CLASSIFIER}\label{par:Device Types / Network Device / Device Operation / Flow filter / Resource objects / VIRTIO-NET-RESOURCE-OBJ-FF-CLASSIFIER}

A classifier is used to match a flow filter key against a packet. The
classifier defines the desired packet fields to match, and is represented by
the VIRTIO_NET_RESOURCE_OBJ_FF_CLASSIFIER device resource object.

For the flow filter classifier object both \field{resource_obj_specific_data} and
\field{resource_obj_specific_result} are in the format
\field{struct virtio_net_resource_obj_ff_classifier}.

\begin{lstlisting}
struct virtio_net_resource_obj_ff_classifier {
        u8 count;
        u8 reserved[7];
        struct virtio_net_ff_selector selectors[];
};
\end{lstlisting}

A classifier is an array of \field{selectors}. The number of selectors in the
array is indicated by \field{count}. The selector has a type that specifies
the header fields to be matched against, and a mask.
See \ref{lst:Device Types / Network Device / Device Operation / Flow filter / Device and driver capabilities / VIRTIO-NET-FF-SELECTOR-CAP / virtio-net-ff-selector}
for details about selectors.

The first selector is always VIRTIO_NET_FF_MASK_TYPE_ETH. When there are multiple
selectors, a second selector can be either VIRTIO_NET_FF_MASK_TYPE_IPV4
or VIRTIO_NET_FF_MASK_TYPE_IPV6. If the third selector exists, the third
selector can be either VIRTIO_NET_FF_MASK_TYPE_UDP or VIRTIO_NET_FF_MASK_TYPE_TCP.
For example, to match a Ethernet IPv6 UDP packet,
\field{selectors[0].type} is set to VIRTIO_NET_FF_MASK_TYPE_ETH, \field{selectors[1].type}
is set to VIRTIO_NET_FF_MASK_TYPE_IPV6 and \field{selectors[2].type} is
set to VIRTIO_NET_FF_MASK_TYPE_UDP; accordingly, \field{selectors[0].mask[0-13]} is
for Ethernet header fields, \field{selectors[1].mask[0-39]} is set for IPV6 header
and \field{selectors[2].mask[0-7]} is set for UDP header.

When there are multiple selectors, the type of the (N+1)\textsuperscript{th} selector
affects the mask of the (N)\textsuperscript{th} selector. If
\field{count} is 2 or more, all the mask bits within \field{selectors[0]}
corresponding to \field{EtherType} of an Ethernet header are set.

If \field{count} is more than 2:
\begin{itemize}
\item if \field{selector[1].type} is, VIRTIO_NET_FF_MASK_TYPE_IPV4, then, all the mask bits within
\field{selector[1]} for \field{Protocol} is set.
\item if \field{selector[1].type} is, VIRTIO_NET_FF_MASK_TYPE_IPV6, then, all the mask bits within
\field{selector[1]} for \field{Next Header} is set.
\end{itemize}

If for a given packet header field, a subset of bits of a field is to be matched,
and if the partial mask is supported, the flow filter
mask object can specify a mask which has fewer bits set than the packet header
field size. For example, a partial mask for the Ethernet header source mac
address can be of 1-bit for multicast detection instead of 48-bits.

\subparagraph{VIRTIO_NET_RESOURCE_OBJ_FF_RULE}\label{par:Device Types / Network Device / Device Operation / Flow filter / Resource objects / VIRTIO-NET-RESOURCE-OBJ-FF-RULE}

Each flow filter rule resource object comprises a key, a priority, and an action.
For the flow filter rule object,
\field{resource_obj_specific_data} and
\field{resource_obj_specific_result} are in the format
\field{struct virtio_net_resource_obj_ff_rule}.

\begin{lstlisting}
struct virtio_net_resource_obj_ff_rule {
        le32 group_id;
        le32 classifier_id;
        u8 rule_priority;
        u8 key_length; /* length of key in bytes */
        u8 action;
        u8 reserved;
        le16 vq_index;
        u8 reserved1[2];
        u8 keys[][];
};
\end{lstlisting}

\field{group_id} is the resource object ID of the flow filter group to which
this rule belongs. \field{classifier_id} is the resource object ID of the
classifier used to match a packet against the \field{key}.

\field{rule_priority} denotes the priority of the rule within the group
specified by the \field{group_id}.
Rules within the group are applied from the highest to the lowest priority
until a rule matches the packet and an
action is taken. Rules with the same priority can be applied in any order.

\field{reserved} and \field{reserved1} are reserved and set to 0.

\field{keys[][]} is an array of keys to match against packets, using
the classifier specified by \field{classifier_id}. Each entry (key) comprises
a byte array, and they are located one immediately after another.
The size (number of entries) of the array is exactly the same as that of
\field{selectors} in the classifier, or in other words, \field{count}
in the classifier.

\field{key_length} specifies the total length of \field{keys} in bytes.
In other words, it equals the sum total of \field{length} of all
selectors in \field{selectors} in the classifier specified by
\field{classifier_id}.

For example, if a classifier object's \field{selectors[0].type} is
VIRTIO_NET_FF_MASK_TYPE_ETH and \field{selectors[1].type} is
VIRTIO_NET_FF_MASK_TYPE_IPV6,
then selectors[0].length is 14 and selectors[1].length is 40.
Accordingly, the \field{key_length} is set to 54.
This setting indicates that the \field{key} array's length is 54 bytes
comprising a first byte array of 14 bytes for the
Ethernet MAC header in bytes 0-13, immediately followed by 40 bytes for the
IPv6 header in bytes 14-53.

When there are multiple selectors in the classifier object, the key bytes
for (N)\textsuperscript{th} selector are set so that
(N+1)\textsuperscript{th} selector can be matched.

If \field{count} is 2 or more, key bytes of \field{EtherType}
are set according to \hyperref[intro:IEEE 802 Ethertypes]{IEEE 802 Ethertypes}
for VIRTIO_NET_FF_MASK_TYPE_IPV4 or VIRTIO_NET_FF_MASK_TYPE_IPV6 respectively.

If \field{count} is more than 2, when \field{selector[1].type} is
VIRTIO_NET_FF_MASK_TYPE_IPV4 or VIRTIO_NET_FF_MASK_TYPE_IPV6, key
bytes of \field{Protocol} or \field{Next Header} is set as per
\field{Protocol Numbers} defined \hyperref[intro:IANA Protocol Numbers]{IANA Protocol Numbers}
respectively.

\field{action} is the action to take when a packet matches the
\field{key} using the \field{classifier_id}. Supported actions are described in
\ref{table:Device Types / Network Device / Device Operation / Flow filter / Device and driver capabilities / VIRTIO-NET-FF-ACTION-CAP / flow filter rule actions}.

\field{vq_index} specifies a receive virtqueue. When the \field{action} is set
to VIRTIO_NET_FF_ACTION_DIRECT_RX_VQ, and the packet matches the \field{key},
the matching packet is directed to this virtqueue.

Note that at most one action is ever taken for a given packet. If a rule is
applied and an action is taken, the action of other rules is not taken.

\devicenormative{\paragraph}{Flow filter}{Device Types / Network Device / Device Operation / Flow filter}

When the device supports flow filter operations,
\begin{itemize}
\item the device MUST set VIRTIO_NET_FF_RESOURCE_CAP, VIRTIO_NET_FF_SELECTOR_CAP
and VIRTIO_NET_FF_ACTION_CAP capability in the \field{supported_caps} in the
command VIRTIO_ADMIN_CMD_CAP_SUPPORT_QUERY.
\item the device MUST support the administration commands
VIRTIO_ADMIN_CMD_RESOURCE_OBJ_CREATE,
VIRTIO_ADMIN_CMD_RESOURCE_OBJ_MODIFY, VIRTIO_ADMIN_CMD_RESOURCE_OBJ_QUERY,
VIRTIO_ADMIN_CMD_RESOURCE_OBJ_DESTROY for the resource types
VIRTIO_NET_RESOURCE_OBJ_FF_GROUP, VIRTIO_NET_RESOURCE_OBJ_FF_CLASSIFIER and
VIRTIO_NET_RESOURCE_OBJ_FF_RULE.
\end{itemize}

When any of the VIRTIO_NET_FF_RESOURCE_CAP, VIRTIO_NET_FF_SELECTOR_CAP, or
VIRTIO_NET_FF_ACTION_CAP capability is disabled, the device SHOULD set
\field{status} to VIRTIO_ADMIN_STATUS_Q_INVALID_OPCODE for the commands
VIRTIO_ADMIN_CMD_RESOURCE_OBJ_CREATE,
VIRTIO_ADMIN_CMD_RESOURCE_OBJ_MODIFY, VIRTIO_ADMIN_CMD_RESOURCE_OBJ_QUERY,
and VIRTIO_ADMIN_CMD_RESOURCE_OBJ_DESTROY. These commands apply to the resource
\field{type} of VIRTIO_NET_RESOURCE_OBJ_FF_GROUP, VIRTIO_NET_RESOURCE_OBJ_FF_CLASSIFIER, and
VIRTIO_NET_RESOURCE_OBJ_FF_RULE.

The device SHOULD set \field{status} to VIRTIO_ADMIN_STATUS_EINVAL for the
command VIRTIO_ADMIN_CMD_RESOURCE_OBJ_CREATE when the resource \field{type}
is VIRTIO_NET_RESOURCE_OBJ_FF_GROUP, if a flow filter group already exists
with the supplied \field{group_priority}.

The device SHOULD set \field{status} to VIRTIO_ADMIN_STATUS_ENOSPC for the
command VIRTIO_ADMIN_CMD_RESOURCE_OBJ_CREATE when the resource \field{type}
is VIRTIO_NET_RESOURCE_OBJ_FF_GROUP, if the number of flow filter group
objects in the device exceeds the lower of the configured driver
capabilities \field{groups_limit} and \field{rules_per_group_limit}.

The device SHOULD set \field{status} to VIRTIO_ADMIN_STATUS_ENOSPC for the
command VIRTIO_ADMIN_CMD_RESOURCE_OBJ_CREATE when the resource \field{type} is
VIRTIO_NET_RESOURCE_OBJ_FF_CLASSIFIER, if the number of flow filter selector
objects in the device exceeds the configured driver capability
\field{selectors_limit}.

The device SHOULD set \field{status} to VIRTIO_ADMIN_STATUS_EBUSY for the
command VIRTIO_ADMIN_CMD_RESOURCE_OBJ_DESTROY for a flow filter group when
the flow filter group has one or more flow filter rules depending on it.

The device SHOULD set \field{status} to VIRTIO_ADMIN_STATUS_EBUSY for the
command VIRTIO_ADMIN_CMD_RESOURCE_OBJ_DESTROY for a flow filter classifier when
the flow filter classifier has one or more flow filter rules depending on it.

The device SHOULD fail the command VIRTIO_ADMIN_CMD_RESOURCE_OBJ_CREATE for the
flow filter rule resource object if,
\begin{itemize}
\item \field{vq_index} is not a valid receive virtqueue index for
the VIRTIO_NET_FF_ACTION_DIRECT_RX_VQ action,
\item \field{priority} is greater than or equal to
      \field{last_rule_priority},
\item \field{id} is greater than or equal to \field{rules_limit} or
      greater than or equal to \field{rules_per_group_limit}, whichever is lower,
\item the length of \field{keys} and the length of all the mask bytes of
      \field{selectors[].mask} as referred by \field{classifier_id} differs,
\item the supplied \field{action} is not supported in the capability VIRTIO_NET_FF_ACTION_CAP.
\end{itemize}

When the flow filter directs a packet to the virtqueue identified by
\field{vq_index} and if the receive virtqueue is reset, the device
MUST drop such packets.

Upon applying a flow filter rule to a packet, the device MUST STOP any further
application of rules and cease applying any other steering configurations.

For multiple flow filter groups, the device MUST apply the rules from
the group with the highest priority. If any rule from this group is applied,
the device MUST ignore the remaining groups. If none of the rules from the
highest priority group match, the device MUST apply the rules from
the group with the next highest priority, until either a rule matches or
all groups have been attempted.

The device MUST apply the rules within the group from the highest to the
lowest priority until a rule matches the packet, and the device MUST take
the action. If an action is taken, the device MUST not take any other
action for this packet.

The device MAY apply the rules with the same \field{rule_priority} in any
order within the group.

The device MUST process incoming packets in the following order:
\begin{itemize}
\item apply the steering configuration received using control virtqueue
      commands VIRTIO_NET_CTRL_RX, VIRTIO_NET_CTRL_MAC, and
      VIRTIO_NET_CTRL_VLAN.
\item apply flow filter rules if any.
\item if no filter rule is applied, apply the steering configuration
      received using the command VIRTIO_NET_CTRL_MQ_RSS_CONFIG
      or according to automatic receive steering.
\end{itemize}

When processing an incoming packet, if the packet is dropped at any stage, the device
MUST skip further processing.

When the device drops the packet due to the configuration done using the control
virtqueue commands VIRTIO_NET_CTRL_RX or VIRTIO_NET_CTRL_MAC or VIRTIO_NET_CTRL_VLAN,
the device MUST skip flow filter rules for this packet.

When the device performs flow filter match operations and if the operation
result did not have any match in all the groups, the receive packet processing
continues to next level, i.e. to apply configuration done using
VIRTIO_NET_CTRL_MQ_RSS_CONFIG command.

The device MUST support the creation of flow filter classifier objects
using the command VIRTIO_ADMIN_CMD_RESOURCE_OBJ_CREATE with \field{flags}
set to VIRTIO_NET_FF_MASK_F_PARTIAL_MASK;
this support is required even if all the bits of the masks are set for
a field in \field{selectors}, provided that partial masking is supported
for the selectors.

\drivernormative{\paragraph}{Flow filter}{Device Types / Network Device / Device Operation / Flow filter}

The driver MUST enable VIRTIO_NET_FF_RESOURCE_CAP, VIRTIO_NET_FF_SELECTOR_CAP,
and VIRTIO_NET_FF_ACTION_CAP capabilities to use flow filter.

The driver SHOULD NOT remove a flow filter group using the command
VIRTIO_ADMIN_CMD_RESOURCE_OBJ_DESTROY when one or more flow filter rules
depend on that group. The driver SHOULD only destroy the group after
all the associated rules have been destroyed.

The driver SHOULD NOT remove a flow filter classifier using the command
VIRTIO_ADMIN_CMD_RESOURCE_OBJ_DESTROY when one or more flow filter rules
depend on the classifier. The driver SHOULD only destroy the classifier
after all the associated rules have been destroyed.

The driver SHOULD NOT add multiple flow filter rules with the same
\field{rule_priority} within a flow filter group, as these rules MAY match
the same packet. The driver SHOULD assign different \field{rule_priority}
values to different flow filter rules if multiple rules may match a single
packet.

For the command VIRTIO_ADMIN_CMD_RESOURCE_OBJ_CREATE, when creating a resource
of \field{type} VIRTIO_NET_RESOURCE_OBJ_FF_CLASSIFIER, the driver MUST set:
\begin{itemize}
\item \field{selectors[0].type} to VIRTIO_NET_FF_MASK_TYPE_ETH.
\item \field{selectors[1].type} to VIRTIO_NET_FF_MASK_TYPE_IPV4 or
      VIRTIO_NET_FF_MASK_TYPE_IPV6 when \field{count} is more than 1,
\item \field{selectors[2].type} VIRTIO_NET_FF_MASK_TYPE_UDP or
      VIRTIO_NET_FF_MASK_TYPE_TCP when \field{count} is more than 2.
\end{itemize}

For the command VIRTIO_ADMIN_CMD_RESOURCE_OBJ_CREATE, when creating a resource
of \field{type} VIRTIO_NET_RESOURCE_OBJ_FF_CLASSIFIER, the driver MUST set:
\begin{itemize}
\item \field{selectors[0].mask} bytes to all 1s for the \field{EtherType}
       when \field{count} is 2 or more.
\item \field{selectors[1].mask} bytes to all 1s for \field{Protocol} or \field{Next Header}
       when \field{selector[1].type} is VIRTIO_NET_FF_MASK_TYPE_IPV4 or VIRTIO_NET_FF_MASK_TYPE_IPV6,
       and when \field{count} is more than 2.
\end{itemize}

For the command VIRTIO_ADMIN_CMD_RESOURCE_OBJ_CREATE, the resource \field{type}
VIRTIO_NET_RESOURCE_OBJ_FF_RULE, if the corresponding classifier object's
\field{count} is 2 or more, the driver MUST SET the \field{keys} bytes of
\field{EtherType} in accordance with
\hyperref[intro:IEEE 802 Ethertypes]{IEEE 802 Ethertypes}
for either VIRTIO_NET_FF_MASK_TYPE_IPV4 or VIRTIO_NET_FF_MASK_TYPE_IPV6.

For the command VIRTIO_ADMIN_CMD_RESOURCE_OBJ_CREATE, when creating a resource of
\field{type} VIRTIO_NET_RESOURCE_OBJ_FF_RULE, if the corresponding classifier
object's \field{count} is more than 2, and the \field{selector[1].type} is either
VIRTIO_NET_FF_MASK_TYPE_IPV4 or VIRTIO_NET_FF_MASK_TYPE_IPV6, the driver MUST
set the \field{keys} bytes for the \field{Protocol} or \field{Next Header}
according to \hyperref[intro:IANA Protocol Numbers]{IANA Protocol Numbers} respectively.

The driver SHOULD set all the bits for a field in the mask of a selector in both the
capability and the classifier object, unless the VIRTIO_NET_FF_MASK_F_PARTIAL_MASK
is enabled.

\subsubsection{Legacy Interface: Framing Requirements}\label{sec:Device
Types / Network Device / Legacy Interface: Framing Requirements}

When using legacy interfaces, transitional drivers which have not
negotiated VIRTIO_F_ANY_LAYOUT MUST use a single descriptor for the
\field{struct virtio_net_hdr} on both transmit and receive, with the
network data in the following descriptors.

Additionally, when using the control virtqueue (see \ref{sec:Device
Types / Network Device / Device Operation / Control Virtqueue})
, transitional drivers which have not
negotiated VIRTIO_F_ANY_LAYOUT MUST:
\begin{itemize}
\item for all commands, use a single 2-byte descriptor including the first two
fields: \field{class} and \field{command}
\item for all commands except VIRTIO_NET_CTRL_MAC_TABLE_SET
use a single descriptor including command-specific-data
with no padding.
\item for the VIRTIO_NET_CTRL_MAC_TABLE_SET command use exactly
two descriptors including command-specific-data with no padding:
the first of these descriptors MUST include the
virtio_net_ctrl_mac table structure for the unicast addresses with no padding,
the second of these descriptors MUST include the
virtio_net_ctrl_mac table structure for the multicast addresses
with no padding.
\item for all commands, use a single 1-byte descriptor for the
\field{ack} field
\end{itemize}

See \ref{sec:Basic
Facilities of a Virtio Device / Virtqueues / Message Framing}.


\chapter{Reserved Feature Bits}\label{sec:Reserved Feature Bits}

Currently these device-independent feature bits are defined:

\begin{description}
  \item[VIRTIO_F_INDIRECT_DESC (28)] Negotiating this feature indicates
  that the driver can use descriptors with the VIRTQ_DESC_F_INDIRECT
  flag set, as described in \ref{sec:Basic Facilities of a Virtio
Device / Virtqueues / The Virtqueue Descriptor Table / Indirect
Descriptors}~\nameref{sec:Basic Facilities of a Virtio Device /
Virtqueues / The Virtqueue Descriptor Table / Indirect
Descriptors} and \ref{sec:Packed Virtqueues / Indirect Flag: Scatter-Gather Support}~\nameref{sec:Packed Virtqueues / Indirect Flag: Scatter-Gather Support}.
  \item[VIRTIO_F_EVENT_IDX(29)] This feature enables the \field{used_event}
  and the \field{avail_event} fields as described in
\ref{sec:Basic Facilities of a Virtio Device / Virtqueues / Used Buffer Notification Suppression}, \ref{sec:Basic Facilities of a Virtio Device / Virtqueues / The Virtqueue Used Ring} and \ref{sec:Packed Virtqueues / Driver and Device Event Suppression}.


  \item[VIRTIO_F_VERSION_1(32)] This indicates compliance with this
    specification, giving a simple way to detect legacy devices or drivers.

  \item[VIRTIO_F_ACCESS_PLATFORM(33)] This feature indicates that
  the device can be used on a platform where device access to data
  in memory is limited and/or translated. E.g. this is the case if the device can be located
  behind an IOMMU that translates bus addresses from the device into physical
  addresses in memory, if the device can be limited to only access
  certain memory addresses or if special commands such as
  a cache flush can be needed to synchronise data in memory with
  the device. Whether accesses are actually limited or translated
  is described by platform-specific means.
  If this feature bit is set to 0, then the device
  has same access to memory addresses supplied to it as the
  driver has.
  In particular, the device will always use physical addresses
  matching addresses used by the driver (typically meaning
  physical addresses used by the CPU)
  and not translated further, and can access any address supplied to it by
  the driver. When clear, this overrides any platform-specific description of
  whether device access is limited or translated in any way, e.g.
  whether an IOMMU may be present.
  \item[VIRTIO_F_RING_PACKED(34)] This feature indicates
  support for the packed virtqueue layout as described in
  \ref{sec:Basic Facilities of a Virtio Device / Packed Virtqueues}~\nameref{sec:Basic Facilities of a Virtio Device / Packed Virtqueues}.
  \item[VIRTIO_F_IN_ORDER(35)] This feature indicates
  that all buffers are used by the device in the same
  order in which they have been made available.
  \item[VIRTIO_F_ORDER_PLATFORM(36)] This feature indicates
  that memory accesses by the driver and the device are ordered
  in a way described by the platform.

  If this feature bit is negotiated, the ordering in effect for any
  memory accesses by the driver that need to be ordered in a specific way
  with respect to accesses by the device is the one suitable for devices
  described by the platform. This implies that the driver needs to use
  memory barriers suitable for devices described by the platform; e.g.
  for the PCI transport in the case of hardware PCI devices.

  If this feature bit is not negotiated, then the device
  and driver are assumed to be implemented in software, that is
  they can be assumed to run on identical CPUs
  in an SMP configuration.
  Thus a weaker form of memory barriers is sufficient
  to yield better performance.
  \item[VIRTIO_F_SR_IOV(37)] This feature indicates that
  the device supports Single Root I/O Virtualization.
  Currently only PCI devices support this feature.
  \item[VIRTIO_F_NOTIFICATION_DATA(38)] This feature indicates
  that the driver passes extra data (besides identifying the virtqueue)
  in its device notifications.
  See \ref{sec:Basic Facilities of a Virtio Device / Driver notifications}~\nameref{sec:Basic Facilities of a Virtio Device / Driver notifications}.

  \item[VIRTIO_F_NOTIF_CONFIG_DATA(39)] This feature indicates that the driver
  uses the data provided by the device as a virtqueue identifier in available
  buffer notifications.
  As mentioned in section \ref{sec:Basic Facilities of a Virtio Device / Driver notifications}, when the
  driver is required to send an available buffer notification to the device, it
  sends the virtqueue index to be notified. The method of delivering
  notifications is transport specific.
  With the PCI transport, the device can optionally provide a per-virtqueue value
  for the driver to use in driver notifications, instead of the virtqueue index.
  Some devices may benefit from this flexibility by providing, for example,
  an internal virtqueue identifier, or an internal offset related to the
  virtqueue index.

  This feature indicates the availability of such value. The definition of the
  data to be provided in driver notification and the delivery method is
  transport specific.
  For more details about driver notifications over PCI see \ref{sec:Virtio Transport Options / Virtio Over PCI Bus / PCI-specific Initialization And Device Operation / Available Buffer Notifications}.

  \item[VIRTIO_F_RING_RESET(40)] This feature indicates
  that the driver can reset a queue individually.
  See \ref{sec:Basic Facilities of a Virtio Device / Virtqueues / Virtqueue Reset}.

  \item[VIRTIO_F_ADMIN_VQ(41)] This feature indicates that the device exposes one or more
  administration virtqueues.
  At the moment this feature is only supported for devices using
  \ref{sec:Virtio Transport Options / Virtio Over PCI
	  Bus}~\nameref{sec:Virtio Transport Options / Virtio Over PCI Bus}
	  as the transport and is reserved for future use for
	  devices using other transports (see
	  \ref{drivernormative:Basic Facilities of a Virtio Device / Feature Bits}
	and
	\ref{devicenormative:Basic Facilities of a Virtio Device / Feature Bits} for
	handling features reserved for future use.

\end{description}

\drivernormative{\section}{Reserved Feature Bits}{Reserved Feature Bits}

A driver MUST accept VIRTIO_F_VERSION_1 if it is offered.  A driver
MAY fail to operate further if VIRTIO_F_VERSION_1 is not offered.

A driver SHOULD accept VIRTIO_F_ACCESS_PLATFORM if it is offered, and it MUST
then either disable the IOMMU or configure the IOMMU to translate bus addresses
passed to the device into physical addresses in memory.  If
VIRTIO_F_ACCESS_PLATFORM is not offered, then a driver MUST pass only physical
addresses to the device.

A driver SHOULD accept VIRTIO_F_RING_PACKED if it is offered.

A driver SHOULD accept VIRTIO_F_ORDER_PLATFORM if it is offered.
If VIRTIO_F_ORDER_PLATFORM has been negotiated, a driver MUST use
the barriers suitable for hardware devices.

If VIRTIO_F_SR_IOV has been negotiated, a driver MAY enable
virtual functions through the device's PCI SR-IOV capability
structure.  A driver MUST NOT negotiate VIRTIO_F_SR_IOV if
the device does not have a PCI SR-IOV capability structure
or is not a PCI device.  A driver MUST negotiate
VIRTIO_F_SR_IOV and complete the feature negotiation
(including checking the FEATURES_OK \field{device status}
bit) before enabling virtual functions through the device's
PCI SR-IOV capability structure.  After once successfully
negotiating VIRTIO_F_SR_IOV, the driver MAY enable virtual
functions through the device's PCI SR-IOV capability
structure even if the device or the system has been fully
or partially reset, and even without re-negotiating
VIRTIO_F_SR_IOV after the reset.

A driver SHOULD accept VIRTIO_F_NOTIF_CONFIG_DATA if it is offered.

\devicenormative{\section}{Reserved Feature Bits}{Reserved Feature Bits}

A device MUST offer VIRTIO_F_VERSION_1.  A device MAY fail to operate further
if VIRTIO_F_VERSION_1 is not accepted.

A device SHOULD offer VIRTIO_F_ACCESS_PLATFORM if its access to
memory is through bus addresses distinct from and translated
by the platform to physical addresses used by the driver, and/or
if it can only access certain memory addresses with said access
specified and/or granted by the platform.
A device MAY fail to operate further if VIRTIO_F_ACCESS_PLATFORM is not
accepted.

If VIRTIO_F_IN_ORDER has been negotiated, a device MUST use
buffers in the same order in which they have been available.

A device MAY fail to operate further if
VIRTIO_F_ORDER_PLATFORM is offered but not accepted.
A device MAY operate in a slower emulation mode if
VIRTIO_F_ORDER_PLATFORM is offered but not accepted.

It is RECOMMENDED that an add-in card based PCI device
offers both VIRTIO_F_ACCESS_PLATFORM and
VIRTIO_F_ORDER_PLATFORM for maximum portability.

A device SHOULD offer VIRTIO_F_SR_IOV if it is a PCI device
and presents a PCI SR-IOV capability structure, otherwise
it MUST NOT offer VIRTIO_F_SR_IOV.

\section{Legacy Interface: Reserved Feature Bits}\label{sec:Reserved Feature Bits / Legacy Interface: Reserved Feature Bits}

Transitional devices MAY offer the following:
\begin{description}
\item[VIRTIO_F_NOTIFY_ON_EMPTY (24)] If this feature
  has been negotiated by driver, the device MUST issue
  a used buffer notification if the device runs
  out of available descriptors on a virtqueue, even though
  notifications are suppressed using the VIRTQ_AVAIL_F_NO_INTERRUPT
  flag or the \field{used_event} field.
\begin{note}
  An example of a driver using this feature is the legacy
  networking driver: it doesn't need to know every time a packet
  is transmitted, but it does need to free the transmitted
  packets a finite time after they are transmitted. It can avoid
  using a timer if the device notifies it when all the packets
  are transmitted.
\end{note}
\end{description}

Transitional devices MUST offer, and if offered by the device
transitional drivers MUST accept the following:
\begin{description}
\item[VIRTIO_F_ANY_LAYOUT (27)] This feature indicates that the device
  accepts arbitrary descriptor layouts, as described in Section
  \ref{sec:Basic Facilities of a Virtio Device / Virtqueues / Message Framing / Legacy Interface: Message Framing}~\nameref{sec:Basic Facilities of a Virtio Device / Virtqueues / Message Framing / Legacy Interface: Message Framing}.

\item[UNUSED (30)] Bit 30 is used by qemu's implementation to check
  for experimental early versions of virtio which did not perform
  correct feature negotiation, and SHOULD NOT be negotiated.
\end{description}

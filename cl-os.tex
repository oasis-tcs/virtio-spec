3abace8 & 27 Sep 2023 & Cornelia Huck & { remove enumerate usage that makes the tool unhappy


Signed-off-by: Cornelia Huck <cohuck@redhat.com>

 } \\
\hline
db68dc0 & 28 Sep 2023 & Cornelia Huck & { Revert "remove enumerate usage that makes the tool unhappy"


This reverts commit 3abace87db23ddceaf9688a405dd3fd540023977.

We can fix it properly instead.

Signed-off-by: Cornelia Huck <cohuck@redhat.com>

 } \\
\hline
c7bef01 & 28 Sep 2023 & Michael S. Tsirkin & { html: add missing enumitem package


makediffhtml.sh currently fails with:

! Missing number, treated as zero.
<to be read again>
                   \textbackslash c@*
l.25850 \textbackslash begin\{enumerate\}[label=\textbackslash alph*
                                      .]
?
! Emergency stop.
<to be read again>
                   \textbackslash c@*
l.25850 \textbackslash begin\{enumerate\}[label=\textbackslash alph*
                                      .]

Some web searches turned up suggestions to use enumitem and in fact,
virtio.tex already does this - but virtio-html.tex doesn't.

Adding \textbackslash usepackage\{enumitem\} in virtio-html.tex too fixes the issue.

Signed-off-by: Michael S. Tsirkin <mst@redhat.com>
Signed-off-by: Cornelia Huck <cohuck@redhat.com>

 } \\
\hline
3fdaa17 & 30 Oct 2023 & jeshwank & { virtio-tee: Reserve device ID 46 for TEE device


In a virtual environment, an application running in guest VM may want
to delegate security sensitive tasks to a Trusted Application (TA)
running within a Trusted Execution Environment (TEE). A TEE is a trusted
OS running in some secure environment, for example, TrustZone on ARM
CPUs, or a separate secure co-processor etc.

A virtual TEE device emulates a TEE within a guest VM. Such a virtual
TEE device supports multiple operations such as:

VIRTIO_TEE_CMD_OPEN_DEVICE – Open a communication channel with virtio
                             TEE device.
VIRTIO_TEE_CMD_CLOSE_DEVICE – Close communication channel with virtio
                              TEE device.
VIRTIO_TEE_CMD_GET_VERSION – Get version of virtio TEE.
VIRTIO_TEE_CMD_OPEN_SESSION – Open a session to communicate with
                              trusted application running in TEE.
VIRTIO_TEE_CMD_CLOSE_SESSION – Close a session to end communication
                               with trusted application running in TEE.
VIRTIO_TEE_CMD_INVOKE_FUNC – Invoke a command or function in trusted
                             application running in TEE.
VIRTIO_TEE_CMD_CANCEL_REQ – Cancel an ongoing command within TEE.
VIRTIO_TEE_CMD_REGISTER_MEM - Register shared memory with TEE.
VIRTIO_TEE_CMD_UNREGISTER_MEM - Unregister shared memory from TEE.

We would like to reserve device ID 46 for Virtio-TEE device.

Fixes: \url{https://github.com/oasis-tcs/virtio-spec/issues/175}

Signed-off-by: Jeshwanth Kumar <jeshwanthkumar.nk@amd.com>
Reviewed-by: Rijo Thomas <Rijo-john.Thomas@amd.com>
Reviewed-by: Parav Pandit <parav@nvidia.com>
Acked-by: Sumit Garg <sumit.garg@linaro.org>
Signed-off-by: Cornelia Huck <cohuck@redhat.com>

 } \\
\hline
42f3899 & 30 Oct 2023 & Xuan Zhuo & { virtio-net: support device stats


This patch allows the driver to obtain some statistics from the device.

In the device implementation, we can count a lot of such information,
which can be used for debugging and judging the running status of the
device. We hope to directly display it to the user through ethtool.

To get stats atomically, try to get stats for all/multiple queue pairs
in one command.

Fixes: \url{https://github.com/oasis-tcs/virtio-spec/issues/180}

Signed-off-by: Xuan Zhuo <xuanzhuo@linux.alibaba.com>
Suggested-by: Michael S. Tsirkin <mst@redhat.com>
Reviewed-by: Parav Pandit <parav@nvidia.com>
Signed-off-by: Cornelia Huck <cohuck@redhat.com>

 } \\
\hline
925b42e & 30 Oct 2023 & Parav Pandit & { conformance: Add missing virtqueue reset conformance references


Add the missing references to the virtqueue reset related conformance
requirements.

Signed-off-by: Parav Pandit <parav@nvidia.com>
Reviewed-by: Xuan Zhuo <xuanzhuo@linux.alibaba.com>
[CH: pushed as an editorial change]
Signed-off-by: Cornelia Huck <cohuck@redhat.com>

 } \\
\hline
68d5cc4 & 30 Oct 2023 & Parav Pandit & { packed-ring: Change host,guest to device,driver


Rest of the packed ring description already uses the device
and driver terminology. Change the introductory line as well from
host and guest to device and driver respectively.

Signed-off-by: Parav Pandit <parav@nvidia.com>
Acked-by: Michael S. Tsirkin <mst@redhat.com>
[CH: pushed as an editorial update]
Signed-off-by: Cornelia Huck <cohuck@redhat.com>

 } \\
\hline
c8249d7 & 30 Oct 2023 & Cornelia Huck & { editorial: allow for longer device id table


Move to "longtable" to allow the table to span multiple pages (it
became too long to fit on one page with the latest addition.)

Signed-off-by: Cornelia Huck <cohuck@redhat.com>

 } \\
\hline
eb16e33 & 03 Nov 2023 & Cornelia Huck & { editorial: various fixes for 1.3-csd01



- Set approval date to 06 October 2023. (applies to front page subtitle,
citation format, PDF page footers)

- Set filenames and URIs to show csd01 instead of wd01 (also PDF footers)

- Set "Previous stage" to "N/A" (don't list a different numbered Version)

- In "Related work", change (3x) text - "Latest version" to "Latest stage"

- In "Notices", set the copyright year to 2023.


- In the first line of Appendix D. Revision History, replace "the previous
version" with "Version 1.2"


- In Section 1.3, apply the current IETF-recommended text:
"The key words "MUST", "MUST NOT", "REQUIRED", "SHALL", "SHALL NOT",
"SHOULD", "SHOULD NOT", "RECOMMENDED", "NOT RECOMMENDED", "MAY", and
"OPTIONAL" in this document are to be interpreted as described in BCP 14
[[RFC2119](\#link)] and [[RFC8174](\#link)] when, and only when, they appear
in all capitals, as shown here."

- Also, add the RFC 8174 reference:
[RFC8174]
Leiba, B., "Ambiguity of Uppercase vs Lowercase in RFC 2119 Key Words", BCP
14, RFC 8174, DOI 10.17487/RFC8174, May 2017,
\url{https://www.rfc-editor.org/info/rfc8174.}

Reported-by: Paul Knight <paul.knight@oasis-open.org>
Signed-off-by: Cornelia Huck <cohuck@redhat.com>

 } \\
\hline
43a948e & 03 Nov 2023 & Cornelia Huck & { editorial: update "Computer Language Definitions" URL


Split out from the other fixes for 1.3-csd01 so that we can fixup the
diff.


- In "Status", (fourth paragraph) change the hyperlink under (Computer
Language Definitions) to be "
\url{https://www.oasis-open.org/policies-guidelines/tc-process-2017-05-26/\#wpComponentsCompLang}
"

Reported-by: Paul Knight <paul.knight@oasis-open.org>
Signed-off-by: Cornelia Huck <cohuck@redhat.com>

 } \\
\hline
5fc35a7 & 03 Nov 2023 & Cornelia Huck & { makediff: update list of cherry-picks


We don't need to apply the old commits anymore, but we have to apply
the URL update to work around a not-yet-debugged latexdiff problem.

Signed-off-by: Cornelia Huck <cohuck@redhat.com>

 } \\
\hline
4cb03b1 & 17 Nov 2023 & Parav Pandit & { description: Avoid splitting the word virtqueue


Don't split the word virtqueue.

Signed-off-by: Parav Pandit <parav@nvidia.com>
[CH: applied as editorial]
Signed-off-by: Cornelia Huck <cohuck@redhat.com>

 } \\
\hline
878a5e7 & 13 Feb 2024 & Shujun Xue & { Define the DEVICE ID of Virtio Cpu balloon device as 47.


The Virtio CPU balloon device is a primitive device for managing guest
CPU capacity: the device asks for certain CPU cores to be online,
offline or throttled, and the driver performs the requested operation.
This allows the guest to adapt to changes in allowance of underlying
CPU capacity.

Fixes: \url{https://github.com/oasis-tcs/virtio-spec/issues/187}

Signed-off-by: Shujun Xue <shujunxue@google.com>
Signed-off-by: Cornelia Huck <cohuck@redhat.com>

 } \\
\hline
07bb9f7 & 13 Feb 2024 & Harald Mommer & { virtio-can: Device specification.


virtio-can is a virtual CAN device. It provides a way to give access to
a CAN controller from a driver guest. The device is aimed to be used by
driver guests running a HLOS as well as by driver guests running a
typical RTOS as used in controller environments.

Fixes: \url{https://github.com/oasis-tcs/virtio-spec/issues/186}

Signed-off-by: Harald Mommer <Harald.Mommer@opensynergy.com>
Signed-off-by: Mikhail Golubev-Ciuchea <Mikhail.Golubev-Ciuchea@opensynergy.com>
Signed-off-by: Cornelia Huck <cohuck@redhat.com>

 } \\
\hline
193daee & 13 Feb 2024 & Heng Qi & { virtio-net: support the RSS context


Commit 84a1d9c48200 ("net: ethtool: extend RXNFC API to support RSS spreading of
filter matches") adds support for RSS context as a destination for receive flow filters
(see WIP work: \url{https://lists.oasis-open.org/archives/virtio-comment/202308/msg00194.html).}

An RSS context consists of configurable parameters specified by receive-side scaling.

Some use cases:
1. When users want some data flows to be steered to specific multiple rxqs, they can set
   receive flow filter rules for these data flows to an RSS context with desired rxqs.
2. Traffic isolation. Used when users want the traffic of certain applications to occupy
   several queues without being distubed.

How to set/configure an RSS context:
Assuming no RSS context has been created before.
1. ethtool -X eth0 context new start 5 equal 8

This command creates an RSS context with an id=1 for eth0, and fills in the indirection
table with rxq indexes 5-8 circularly. The hash key and hash types reuse the default
RSS configuration.

Then, we can use 'ethtool -x eth0 context 1' to query the above configuration.

2. ethtool -X eth0 context new start 6 equal 7 \textbackslash 
   hkey 8f:bf:dd:11:23:58:d2:8a:00:31:d0:32:a3:b5:1f:\textbackslash 
   1f:e4:d1:fe:47:7f:64:42:fd:d0:61:16:b8:b0:f9:71:e8:2d:36:7f:18:dd:4d:c8:f3

This command creates an RSS context with an id=2 for eth0, and fills in the indirection
table with rxq indexes 6-7 circularly. The hash key is 8f:bf:dd:11:23:58:d2:8a:00:31:d0\textbackslash 
:32:a3:b5:1f:1f:e4:d1:fe:47:7f:64:42:fd:d0:61:16:b8:b0:f9:71:e8:2d:36:7f:18:dd:4d:c8:f3.
Hash types reuse the default RSS configuration.

3. ethtool -N eth0 rx-flow-hash tcp4 sdfn context 1

This command specifies the hash types for the RSS context whose id=1 on eth0.
Now this RSS context only has the hash key to reuse the default RSS configuration.

4. ethtool -N eth0 flow-type udp4 src-ip 1.1.1.1 context 1

This command configures a receive flow filter rule for eth0, and the data flow matching
this rule will continue to select the final rxq according to the RSS context configuration
with id=1.

Fixes: \url{https://github.com/oasis-tcs/virtio-spec/issues/178}

Signed-off-by: Heng Qi <hengqi@linux.alibaba.com>
Signed-off-by: Parav Pandit <parav@nvidia.com>
Acked-by: Satananda Burla <sburla@marvell.com>
Acked-by: Xuan Zhuo <xuanzhuo@linux.alibaba.com>
Signed-off-by: Cornelia Huck <cohuck@redhat.com>

 } \\
\hline
bb11bf9 & 13 Feb 2024 & Haixu Cui & { content: Rename SPI master to SPI controller


SPI master is an outdated term and should use SPI controller.

Signed-off-by: Haixu Cui <quic_haixcui@quicinc.com>
Reviewed-by: Viresh Kumar <viresh.kumar@linaro.org>
Signed-off-by: Cornelia Huck <cohuck@redhat.com>

 } \\
\hline
a402acc & 13 Feb 2024 & Haixu Cui & { virtio-spi: add the device specification


The Virtio SPI (Serial Peripheral Interface) device is a virtual
SPI controller that allows the driver to operate and use the SPI
controller under the control of the host.

This patch adds the specification for virtio-spi.

Fixes: \url{https://github.com/oasis-tcs/virtio-spec/issues/189}

Signed-off-by: Haixu Cui <quic_haixcui@quicinc.com>
Reviewed-by: Viresh Kumar <viresh.kumar@linaro.org>
Signed-off-by: Cornelia Huck <cohuck@redhat.com>

 } \\
\hline
37c6a40 & 16 Feb 2024 & Pape, Andreas (ADITG/ESS3) & { sound: add sampling rates 12000Hz and 24000Hz


24kHz is used for 'super wideband' voice transmission 12kHz is added 'for completeness'

Fixes: \url{https://github.com/oasis-tcs/virtio-spec/issues/184}

Signed-off-by: Andreas Pape <apape@de.adit-jv.com>
Reviewed-by: Anton Yakovlev <anton.yakovlev@opensynergy.com>
Signed-off-by: Cornelia Huck <cohuck@redhat.com>

 } \\
\hline
71fe8e9 & 21 May 2024 & Michael S. Tsirkin & { README.md: update mailing list info


As approved by the TC, we are moving to a less formal
way of discussing the specification, on the mailing lists
provided by the Linux Foundation:

\url{https://groups.oasis-open.org/higherlogic/ws/public/ballot?id=3820}

Update README.md, CONTRIBUTING.md and newdevice.tex accordingly.

Use this opportunity to explain when and how to use each
mailing list.

Oh yes, and device numbers are reserved through virtio-comment
not through virtio-dev. Correct that.

Message-Id: <8f5db33c96d685fcebca3579b05d09b64dd720d9.1715766697.git.mst@redhat.com>
Signed-off-by: Michael S. Tsirkin <mst@redhat.com>

 } \\
\hline
6678bbf & 11 Jul 2024 & Parav Pandit & { Add link for the feature bits section


Device common feature bits overview in the basic facilities and
their actual description are apart by 24 chapters.

Help reader to directly reach out to feature bits definitions from
the basic chapter.

MST: merged as a trivial editorial change

Signed-off-by: Parav Pandit <parav@nvidia.com>
Message-Id: <20240612073522.2571082-1-parav@nvidia.com>
Signed-off-by: Michael S. Tsirkin <mst@redhat.com>

 } \\
\hline
550b760 & 11 Jul 2024 & Michael S. Tsirkin & { can: drop a broken conformance link


CAN device has a single device conformance section, drop
a link to a non-existent section, fixing a Latex error.

Fixes: 07bb9f7 ("virtio-can: Device specification.")
Message-ID: <87a6c3c4463e9f304cdd5dd5fac2194a95a8bcd2.1720724591.git.mst@redhat.com>
Signed-off-by: Michael S. Tsirkin <mst@redhat.com>

 } \\
\hline
d7c486b & 11 Jul 2024 & Michael S. Tsirkin & { introduction: fix label


label to rfc8174 is incorrect, leading to undefined reference
and multiple defined reference latex errors.

Fix it up.

Message-Id: <52b6fc16793e3b24284aefafda94322c739b9f7f.1720725332.git.mst@redhat.com>
Fixes: eb16e33 ("editorial: various fixes for 1.3-csd01")
Signed-off-by: Michael S. Tsirkin <mst@redhat.com>

 } \\
\hline
7e454b6 & 11 Jul 2024 & Michael S. Tsirkin & { makediff: cherry pick label change


label changes tend to trip up makediff - this one
causes an undefined reference from deleted section.
To fix, cherry-pick it.

Message-Id: <216774e3fa811e0a6e994a3815f0804ddfc0bb03.1720731399.git.mst@redhat.com>
Signed-off-by: Michael S. Tsirkin <mst@redhat.com>

 } \\
\hline
b495841 & 11 Jul 2024 & Michael S. Tsirkin & { transport-mmio: fix up makediff from 1.2


we fixed a typo in label name: in the diff old links become
undefined. Add the old label back for now, we can drop it
down the road.

Message-Id: <fc98549666e0fe74e890a7a62ca71248e709712b.1720731329.git.mst@redhat.com>
Fixes: ca97719 ("transport-mmio: Fix spellings and white spaces")
Signed-off-by: Michael S. Tsirkin <mst@redhat.com>

 } \\
\hline
75086cb & 11 Jul 2024 & Michael S. Tsirkin & { Merge tag 'v1.3-wd02' into virtio-1.4


Merge fixes from master branch (1.3) up to tag v1.3-wd02
commit b495841 ("transport-mmio: fix up makediff from 1.2")

Signed-off-by: Michael S. Tsirkin <mst@redhat.com>

 } \\
\hline
61b65cb & 11 Jul 2024 & Cornelia Huck & { editorial: allow for longer device id table: makediff 1.3


Move to "longtable" to allow the table to span multiple pages (it
became too long to fit on one page with the latest addition.)

Signed-off-by: Cornelia Huck <cohuck@redhat.com>
(cherry picked from commit c8249d73d2fdbbfd38e8bf45c8492057bc2485e9)

 } \\
\hline
6fa0087 & 11 Jul 2024 & Michael S. Tsirkin & { Merge tag 'v1.3-wd02-makediff' into virtio-1.4


this is just so makediff can cherry pick a workaround
for latexdiff.

Signed-off-by: Michael S. Tsirkin <mst@redhat.com>

 } \\
\hline
79b04be & 11 Jul 2024 & Michael S. Tsirkin & { makediff: update base version to 1.3


might cause minor conflicts if we merge 1.3 branch
again, but seems cleaner than seeing full diff here.

Signed-off-by: Michael S. Tsirkin <mst@redhat.com>

 } \\
\hline
18e47c0 & 11 Jul 2024 & Parav Pandit & { virtio-blk: Fix data type of num_queues field


Correct the endianness of num_queues field to be little endian.

Suggested-by: Max Gurtovoy <mgurtovoy@nvidia.com>
Reviewed-by: Max Gurtovoy <mgurtovoy@nvidia.com>
Signed-off-by: Parav Pandit <parav@nvidia.com>
Reviewed-by: Stefano Garzarella <sgarzare@redhat.com>
Reviewed-by: Stefan Hajnoczi <stefanha@redhat.com>
Message-Id: <20240521122119.2004071-1-parav@nvidia.com>
Signed-off-by: Michael S. Tsirkin <mst@redhat.com>

 } \\
\hline
0080681 & 11 Jul 2024 & Parav Pandit & { virtio-net: Fix receive buffer size calculation text


Receive buffer size calculation is based on the following
negotiated features.

The text has wrong calculation for IPv6 and also it has missed
VIRTIO_NET_F_HASH_REPORT.

The problem of igorance of VIRTIO_NET_F_HASH_REPORT is reported
in [1], however fix for ipv6 payload length must also be
considered.

Since for the both the fixes touching same requirements, a
new issue is created as [2].

This patch brings following fixes.

1. Fix annotating struct virtio_net_hdr as field
2. Fix receive buffer calculation for guest GSO cases to consider
   ipv6 payload length
3. small grammar corrections for article
4. reword the requirement to consider the virtio_ndr_hdr which is
   depends on the negotiated feature, hence first clarify the
   struct virtio_net_hdr size

[1] \url{https://github.com/oasis-tcs/virtio-spec/issues/170}
[2] \url{https://github.com/oasis-tcs/virtio-spec/issues/183}

Fixes: \url{https://github.com/oasis-tcs/virtio-spec/issues/170}
Fixes: \url{https://github.com/oasis-tcs/virtio-spec/issues/183}
Reviewed-by: Xuan Zhuo <xuanzhuo@linux.alibaba.com>
Signed-off-by: Parav Pandit <parav@nvidia.com>
Message-Id: <20240606102014.2103986-2-parav@nvidia.com>
Signed-off-by: Michael S. Tsirkin <mst@redhat.com>

 } \\
\hline
bf0fdf8 & 11 Jul 2024 & Parav Pandit & { virtio-net: Clarify the size of the struct virtio_net_hdr for tx


The feature VIRTIO_NET_F_HASH_REPORT only applies to the receive side.
However, when VIRTIO_NET_F_HASH_REPORT feature was introduced, it was
not clarified that the size of the struct virtio_net_hdr on the packet
transmission also uses higher size when VIRTIO_NET_F_HASH_REPORT is
negotiated.

Explicitly clarify this.

Fixes: \url{https://github.com/oasis-tcs/virtio-spec/issues/183}
Reviewed-by: Xuan Zhuo <xuanzhuo@linux.alibaba.com>
Signed-off-by: Parav Pandit <parav@nvidia.com>
Message-Id: <20240606102014.2103986-3-parav@nvidia.com>
Signed-off-by: Michael S. Tsirkin <mst@redhat.com>

 } \\
\hline
6abd42f & 11 Jul 2024 & Parav Pandit & { virtio-net: Annotate virtio_net_hdr as field


At several places struct virtio_net_hdr missed out the field
annotation. Add it.

Reviewed-by: Xuan Zhuo <xuanzhuo@linux.alibaba.com>
Signed-off-by: Parav Pandit <parav@nvidia.com>
Message-Id: <20240606102014.2103986-4-parav@nvidia.com>
Signed-off-by: Michael S. Tsirkin <mst@redhat.com>

 } \\
\hline
fdd39cf & 11 Jul 2024 & Parav Pandit & { admin: Introduce self group


Define self group to control the self device itself.

Subsequent patches introduces the concept of device capabilities
and device resources which utilizes the self group to access
capabilities and uses device resources to refer to the device itself.

Fixes: \url{https://github.com/oasis-tcs/virtio-spec/issues/179}
Signed-off-by: Parav Pandit <parav@nvidia.com>
Signed-off-by: Michael S. Tsirkin <mst@redhat.com>
Acked-by: Satananda Burla <sburla@marvell.com>
Message-Id: <20240604132903.2093195-2-parav@nvidia.com>
Signed-off-by: Michael S. Tsirkin <mst@redhat.com>

 } \\
\hline
bfe175d & 11 Jul 2024 & Parav Pandit & { admin: Use already defined names for the legacy commands


Instead of description, use the existing name for defining the legacy
commands. While at it, prefer the shorter label names which
are already unique and refer them as hyperreference in the table
for quick naviation.

This is editorial change to align to subsequent patches.

Fixes: \url{https://github.com/oasis-tcs/virtio-spec/issues/179}
Signed-off-by: Parav Pandit <parav@nvidia.com>
Acked-by: Satananda Burla <sburla@marvell.com>
Message-Id: <20240604132903.2093195-3-parav@nvidia.com>
Signed-off-by: Michael S. Tsirkin <mst@redhat.com>

 } \\
\hline
88a24d2 & 11 Jul 2024 & Parav Pandit & { admin: Add theory of operation for capability admin commands


Device capability indicates the supported functionality and
resources of the device to the driver.

Driver capability indicates the supported functionality
and resources which driver will be using. Driver capability is
subset of the device capability.

Add theory of operation describing it.

Fixes: \url{https://github.com/oasis-tcs/virtio-spec/issues/179}
Signed-off-by: Parav Pandit <parav@nvidia.com>
Signed-off-by: Michael S. Tsirkin <mst@redhat.com>
Acked-by: Satananda Burla <sburla@marvell.com>
Message-Id: <20240604132903.2093195-4-parav@nvidia.com>
Signed-off-by: Michael S. Tsirkin <mst@redhat.com>

 } \\
\hline
72a91e6 & 11 Jul 2024 & Parav Pandit & { admin: Prepare table for multipage listing


If the table spans across two pages, it is not readable.
Make use of xltabular package that supports table spanning
across multiple pages.

Signed-off-by: Parav Pandit <parav@nvidia.com>
Acked-by: Satananda Burla <sburla@marvell.com>
Message-Id: <20240604132903.2093195-5-parav@nvidia.com>
Signed-off-by: Michael S. Tsirkin <mst@redhat.com>

 } \\
\hline
d60a390 & 11 Jul 2024 & Parav Pandit & { admin: Add capability admin commands


Add three capabilities related administration commands.
First to get the device capability.
Second to set the driver capability.
Third for the driver to discover which capabilities can be accessed.

Even though the current series restricts device capability reading
to the self group type, same structure and command format etc will
be reusable in future to read the capability from the owner device
and also to set the device capability via owner device using
new DEV_CAP_SET command.

Resource objects are introduced in subsequent patch which utilizes
the capability, however some description around resource object
limit is covered in this patch to keep things simple.

Fixes: \url{https://github.com/oasis-tcs/virtio-spec/issues/179}
Signed-off-by: Parav Pandit <parav@nvidia.com>
Signed-off-by: Michael S. Tsirkin <mst@redhat.com>
Acked-by: Satananda Burla <sburla@marvell.com>
Message-Id: <20240604132903.2093195-6-parav@nvidia.com>
Signed-off-by: Michael S. Tsirkin <mst@redhat.com>

 } \\
\hline
738f0a2 & 11 Jul 2024 & Parav Pandit & { admin: Add theory of operation for device resource objects


The driver controls the device by means of device resource objects.
These operations include create, query, modify, and destroy the device
resource objects.

Fixes: \url{https://github.com/oasis-tcs/virtio-spec/issues/179}
Signed-off-by: Parav Pandit <parav@nvidia.com>
Signed-off-by: Michael S. Tsirkin <mst@redhat.com>
Acked-by: Satananda Burla <sburla@marvell.com>
Message-Id: <20240604132903.2093195-7-parav@nvidia.com>
Signed-off-by: Michael S. Tsirkin <mst@redhat.com>

 } \\
\hline
5040f36 & 11 Jul 2024 & Parav Pandit & { admin: Add device resource objects admin commands


Add generic administration commands create, modify, query and destroy
device resource object.

Each resource object defines resource specific attributes for the commands.
Once the resource object is created by the driver, it can be query/modify or
destroyed by the driver.

Each resource object is identified using a unique resource id per
resource type.

Fixes: \url{https://github.com/oasis-tcs/virtio-spec/issues/179}
Signed-off-by: Parav Pandit <parav@nvidia.com>
Signed-off-by: Michael S. Tsirkin <mst@redhat.com>
Acked-by: Satananda Burla <sburla@marvell.com>
Message-Id: <20240604132903.2093195-8-parav@nvidia.com>
Signed-off-by: Michael S. Tsirkin <mst@redhat.com>

 } \\
\hline
7b3bdd6 & 11 Jul 2024 & Parav Pandit & { virtio-net: Add theory of operation for flow filter


Currently packet allow/drop interface has following limitations.

1. Driver can either select which MAC and VLANs to consider
for allowing/dropping packets, here, the driver has a
limitation that driver needs to supply full mac
table or full vlan table for each type. Driver cannot add or
delete an individual entry.

2. Driver cannot select mac+vlan combination for which
to allow/drop packet.

3. Driver cannot not set other commonly used packet match fields
such as IP header fields, TCP, UDP, SCP header fields.

4. Driver cannot steer specific packets based on the match
fields to specific receiveq.

5. Driver do not have multiple or dedicated virtqueues to
   perform flow filter requests in accelerated manner in
   the device.

Flow filter as a generic framework overcome above limitations.

As starting point it is useful to support at least two use cases.
a. ARFS
b. ethtool ntuple steering

In future it can be further extended for usecases such as
switching device, connection tracking or may be more.

The flow filter has following properties.

1. It is an extendible object that driver can create, destroy.
2. It belongs to a flow filter group.
3. Each flow filter rule is identified using a unique id,
   has priority, match key, destination(rq) and action(allow/drop).
4. Flow filter key also refers to the mask to learn which fields
   of the packets to match.

This patch adds theory of operation for flow filter functionality.

Fixes: \url{https://github.com/oasis-tcs/virtio-spec/issues/179}
Signed-off-by: Parav Pandit <parav@nvidia.com>
Signed-off-by: Heng Qi <hengqi@linux.alibaba.com>
Signed-off-by: Michael S. Tsirkin <mst@redhat.com>
Acked-by: Satananda Burla <sburla@marvell.com>
Message-Id: <20240604132903.2093195-9-parav@nvidia.com>
Signed-off-by: Michael S. Tsirkin <mst@redhat.com>

 } \\
\hline
899bb0c & 11 Jul 2024 & Parav Pandit & { virtio-net: Add flow filter capability


Add first flow filter capability that indicates device resource
limits such as number of flow filter rules, groups and selectors.

add second capability that indicates supported selectors defining
which packet headers and their fields supported.

[Parav Pandit: resolve conflict: editorial: in spec links]
Fixes: \url{https://github.com/oasis-tcs/virtio-spec/issues/179}
Signed-off-by: Parav Pandit <parav@nvidia.com>
Signed-off-by: Heng Qi <hengqi@linux.alibaba.com>
Signed-off-by: Michael S. Tsirkin <mst@redhat.com>
Acked-by: Satananda Burla <sburla@marvell.com>
Message-Id: <20240604132903.2093195-10-parav@nvidia.com>
Signed-off-by: Michael S. Tsirkin <mst@redhat.com>

 } \\
\hline
224e98b & 11 Jul 2024 & Parav Pandit & { virtio-net: Add flow filter group, classifier and rule resource objects


Flow filter rules depend on the flow filter group resource object.
The device can have one or more flow filter groups. Each flow filter
group has its own order. The group order defines the packet
processing order in the flow filter domain.

Define flow filter classifier object which consists of one
or multiple types of packet header fields to consider
for matching. A mask can match on partial field or whole
field if the device supports partial masking for a given
mask type.

Define flow filter rule resource which consists of
match key, reference to group and mask objects and
an action.

Currently it covers the most common filter types and value
of Ethernet header, IPV4 and IPV6 headers and TCP and UDP headers.

Fixes: \url{https://github.com/oasis-tcs/virtio-spec/issues/179}
Signed-off-by: Parav Pandit <parav@nvidia.com>
Signed-off-by: Heng Qi <hengqi@linux.alibaba.com>
Signed-off-by: Michael S. Tsirkin <mst@redhat.com>
Acked-by: Satananda Burla <sburla@marvell.com>
Message-Id: <20240604132903.2093195-11-parav@nvidia.com>
Signed-off-by: Michael S. Tsirkin <mst@redhat.com>

 } \\
\hline
3b7b737 & 11 Jul 2024 & Parav Pandit & { virtio-net: Add flow filter device and driver requirements


Add device and driver flow filter requirements.

Fixes: \url{https://github.com/oasis-tcs/virtio-spec/issues/179}
Signed-off-by: Parav Pandit <parav@nvidia.com>
Signed-off-by: Heng Qi <hengqi@linux.alibaba.com>
Acked-by: Satananda Burla <sburla@marvell.com>
Message-Id: <20240604132903.2093195-12-parav@nvidia.com>
Signed-off-by: Michael S. Tsirkin <mst@redhat.com>

 } \\
\hline
9b3129f & 12 Jul 2024 & Martin Kröning & { virtio_pci_cap64: specify offset_hi, length_hi endianness


While the capability introduction says "This virtio structure capability uses little-endian format,"
it might be preferrable to be explicit about the endianness of offset_hi and length_hi.

Signed-off-by: Martin Kröning <martin.kroening@eonerc.rwth-aachen.de>
Reviewed-by: Parav Pandit <parav@nvidia.com>
Fixes: \url{https://github.com/oasis-tcs/virtio-spec/issues/196}
Message-Id: <DF89BB0F-6BA4-4DAB-AEC3-03AAF858EC8E@eonerc.rwth-aachen.de>
Signed-off-by: Michael S. Tsirkin <mst@redhat.com>

 } \\
\hline
67141d1 & 12 Jul 2024 & Parav Pandit & { admin: Add theory of operation for device parts


A device can get and set the state of the device which is
organized as multiple device parts. This can be useful
in cases of VM snapshot, VM migration, debug or some other use
cases.

Add the theory of operations on how to get/set device parts
and how it affects when the device is stopped or resumed.

Fixes: \url{https://github.com/oasis-tcs/virtio-spec/issues/176}
Signed-off-by: Michael S. Tsirkin <mst@redhat.com>
Signed-off-by: Parav Pandit <parav@nvidia.com>
Message-Id: <20240601145042.2074739-2-parav@nvidia.com>

 } \\
\hline
a1281ad & 12 Jul 2024 & Parav Pandit & { admin: Extend resource objects for sr-iov group type


occasionally the group owner device accesses the device parts
of the member device. Extend the usage of resource objects
for member device parts access.

Fixes: \url{https://github.com/oasis-tcs/virtio-spec/issues/176}
Signed-off-by: Parav Pandit <parav@nvidia.com>
Message-Id: <20240601145042.2074739-3-parav@nvidia.com>

 } \\
\hline
aa4f6f0 & 12 Jul 2024 & Parav Pandit & { admin: Add admin commands for device parts


Add administration commands to handle device parts such as
get/set device parts, get metadata of the device parts,

Fixes: \url{https://github.com/oasis-tcs/virtio-spec/issues/176}
Signed-off-by: Parav Pandit <parav@nvidia.com>
Message-Id: <20240601145042.2074739-4-parav@nvidia.com>

 } \\
\hline
617aa2d & 12 Jul 2024 & Parav Pandit & { admin: Define common device parts


Define common device parts that represents the state of the device.
The driver can get and set these device parts using administration
commands.

[Parav Pandit: resolve conflict: editorial: empty blank line at end of file newdevice.tex]
Fixes: \url{https://github.com/oasis-tcs/virtio-spec/issues/176}
Signed-off-by: Parav Pandit <parav@nvidia.com>
Message-Id: <20240601145042.2074739-5-parav@nvidia.com>

 } \\
\hline
52d320c & 12 Jul 2024 & Parav Pandit & { admin: Add requirements of device parts commands


Add device and driver side requirements for the device parts
related commands.

Fixes: \url{https://github.com/oasis-tcs/virtio-spec/issues/176}
Signed-off-by: Parav Pandit <parav@nvidia.com>
Message-Id: <20240601145042.2074739-6-parav@nvidia.com>

 } \\
\hline
24e93e1 & 15 Jul 2024 & Parav Pandit & { editorial: replace hyperref with ref


Replace hyperreference with the name reference.
Fix the broken reference link for the DEVICE_STATUS part.

Fixes: aa4f6f069ab3 ("admin: Add admin commands for device parts")
Fixes: 617aa2d62a88 ("admin: Define common device parts")
Suggested-by: Michael S. Tsirkin <mst@redhat.com>
Signed-off-by: Parav Pandit <parav@nvidia.com>
Message-Id: <20240714144023.3291803-1-parav@vr-arch-host06.mtvr.labs.mlnx>
Signed-off-by: Michael S. Tsirkin <mst@redhat.com>

 } \\
\hline
7379354 & 15 Jul 2024 & Michael S. Tsirkin & { makediff: look in subjects only


We currently use rev-list with -F to find commits.
This helps if commit subject has any special characters
but means that pattern can apply anywhere in the line.
In particular, a common Fixes: (hash) "subject" pattern
we started using implies that fixups are picked up
instead of the commit.

Message-Id: <1130700f4ae71a93d9887af35cf105731219f057.1721062136.git.mst@redhat.com>
Signed-off-by: Michael S. Tsirkin <mst@redhat.com>

 } \\
\hline
98ad5fc & 15 Jul 2024 & Michael S. Tsirkin & { makediff: cherry pick table env change


latexdiff can not cope with environment name changes.
cherry pick to avoid confusing it.

Fixes: 72a91e6 ("admin: Prepare table for multipage listing")
Message-Id: <c210d1bf4c64145ad3aae7155ef6050433f23346.1721062136.git.mst@redhat.com>
Signed-off-by: Michael S. Tsirkin <mst@redhat.com>

 } \\
\hline
849b421 & 15 Jul 2024 & Michael S. Tsirkin & { admin: switch to tabularx


xltabular seems to confuse htlatex. Switch to an older
tabularx which seems to be sufficient for our purposes.

Message-Id: <29ebb999f6d3808edbf32cd6e95b69fefc1a6e34.1721062136.git.mst@redhat.com>
Signed-off-by: Michael S. Tsirkin <mst@redhat.com>

 } \\
\hline
8e81624 & 15 Jul 2024 & Michael S. Tsirkin & { admin-cmds-device-parts: switch to tabularx


xltabular seems to confuse htlatex. Switch to an older
tabularx which seems to be sufficient for our purposes.

Message-Id: <bd3cf4e620485271d777f9cfc51d1ae9ee1b8169.1721062136.git.mst@redhat.com>
Signed-off-by: Michael S. Tsirkin <mst@redhat.com>

 } \\
\hline
886fc33 & 15 Jul 2024 & Michael S. Tsirkin & { device-parts: switch to tabularx


xltabular seems to confuse htlatex. Switch to an older
tabularx which seems to be sufficient for our purposes.

Message-Id: <00f72154c624fbbe8c3da9dbdb4e4369c36b7f10.1721062136.git.mst@redhat.com>
Signed-off-by: Michael S. Tsirkin <mst@redhat.com>

 } \\
\hline
a5fb7b5 & 15 Jul 2024 & Michael S. Tsirkin & { admin-cmds-resource-objects: switch to tabularx


xltabular seems to confuse htlatex. Switch to an older
tabularx which seems to be sufficient for our purposes.

Message-Id: <3f7be88a42af0a384621ef07b65c7a31755ef42e.1721062136.git.mst@redhat.com>
Signed-off-by: Michael S. Tsirkin <mst@redhat.com>

 } \\
\hline
459f1aa & 15 Jul 2024 & Michael S. Tsirkin & { virtio: replace xltabular with ltablex


now that we only use tabularx, switch to ltablex to
make tabularx tables paginate properly.

Message-Id: <15b11ce5ac9bbe1aaa7818403b999ff39c74a479.1721062136.git.mst@redhat.com>
Signed-off-by: Michael S. Tsirkin <mst@redhat.com>

 } \\
\hline
e48de84 & 15 Jul 2024 & Michael S. Tsirkin & { makediff: cherry pick switch to tabularx


Message-Id: <88cdb7020f5b2771a6d201b1f65005da97e54f16.1721062136.git.mst@redhat.com>
Signed-off-by: Michael S. Tsirkin <mst@redhat.com>

 } \\
\hline
6c23405 & 15 Jul 2024 & Michael S. Tsirkin & { admin: get rid of _ in labels


it's not the 1st time we find out underscores in labels confuse
latexdiff machinery when generating html.
Errors look like this:

! Missing \textbackslash endcsname inserted.
<to be read again>
                   \textbackslash unhbox
....

?
! Emergency stop.
<to be read again>
                   \textbackslash unhbox

To fix, convert underscores in labels to dashes.

Message-Id: <3e773fcfd4246e66e6f39d3be95d2c0335e34532.1721062136.git.mst@redhat.com>
Signed-off-by: Michael S. Tsirkin <mst@redhat.com>

 } \\
\hline
2a5a957 & 23 Oct 2024 & Viresh Kumar & { virtio-transport: Add a section to define mandatory transport requirements


The current Virtio documentation lacks a set of generic requirements
applicable to all transports. Defining these generic requirements could
be beneficial when integrating support for a new transport.

This section outlines the essential requirements that any transport
method must adhere to.

Signed-off-by: Viresh Kumar <viresh.kumar@linaro.org>
Fixes: \url{https://github.com/oasis-tcs/virtio-spec/issues/201}
Reviewed-by: Stefano Garzarella <sgarzare@redhat.com>
Link: \url{https://lore.kernel.org/r/4c24ede14e2545b2e9c69e7d9b79dc15744fd965.1723647318.git.viresh.kumar@linaro.org}

 } \\
\hline
1d38844 & 23 Oct 2024 & Parav Pandit & { device-parts: editorial: Add missing struct keyword


'struct' keyword was missing in the definition.
Add it.

Branch: virtio-1.4
Fixes: aa4f6f06 ("admin: Add admin commands for device parts")
Signed-off-by: Parav Pandit <parav@nvidia.com>
Link: \url{https://lore.kernel.org/r/20240916030835.68178-2-parav@nvidia.com}

 } \\
\hline
9081da1 & 23 Oct 2024 & Parav Pandit & { device-parts: editorial: Fix metadata type name


The description is for the type VIRTIO_ADMIN_CMD_DEV_PARTS_METADATA_TYPE_LIST;
however, it referred to a non-existent definition. Correct it.

Branch: virtio-1.4
Fixes: aa4f6f069ab3 ("admin: Add admin commands for device parts")
Fixes: \url{https://github.com/oasis-tcs/virtio-spec/issues/205}
Reviewed-by: Matias Ezequiel Vara Larsen <mvaralar@redhat.com>
Signed-off-by: Parav Pandit <parav@nvidia.com>
Link: \url{https://lore.kernel.org/r/20240916030835.68178-3-parav@nvidia.com}

 } \\
\hline
b820c61 & 23 Oct 2024 & Parav Pandit & { crypto: editorial: Fix spelling errors


Fix spelling errors.

Branch: virtio-1.4
Fixes: a385dd3366a2 ("virtio-crypto: Add virtio crypto device specification")
Fixes: \url{https://github.com/oasis-tcs/virtio-spec/issues/205}
Reviewed-by: Matias Ezequiel Vara Larsen <mvaralar@redhat.com>
Signed-off-by: Parav Pandit <parav@nvidia.com>
Link: \url{https://lore.kernel.org/r/20240916030835.68178-4-parav@nvidia.com}

 } \\
\hline
ec1845b & 23 Oct 2024 & Parav Pandit & { gpu: editorial: Fix spelling errors


Fix spelling errors.

Branch: virtio-1.4
Fixes: fed64230bf31 ("Add virtio gpu device specification.")
Fixes: \url{https://github.com/oasis-tcs/virtio-spec/issues/205}
Reviewed-by: Matias Ezequiel Vara Larsen <mvaralar@redhat.com>
Signed-off-by: Parav Pandit <parav@nvidia.com>
Link: \url{https://lore.kernel.org/r/20240916030835.68178-5-parav@nvidia.com}

 } \\
\hline
9d60d84 & 23 Oct 2024 & Parav Pandit & { common: editorial: Fix spelling errors


Fix spelling errors.

Branch: virtio-1.4
Fixes: 5f1a8ac61c15 ("admin: introduce virtio admin virtqueues")
Fixes: 68f66ff7a3d9 ("content: define what an exported object is")
Fixes: ef16b644cc25 ("content.tex: spec text converted to latex")
Fixes: \url{https://github.com/oasis-tcs/virtio-spec/issues/205}
Reviewed-by: Matias Ezequiel Vara Larsen <mvaralar@redhat.com>
Signed-off-by: Parav Pandit <parav@nvidia.com>
Link: \url{https://lore.kernel.org/r/20240916030835.68178-6-parav@nvidia.com}

 } \\
\hline
e98070a & 03 Nov 2024 & Albert Esteve & { content: Reserve ID for media device


Reserve device ID 49 for media device.
Compatible with video encoder, video decoder,
and capture devices.

Signed-off-by: Albert Esteve <aesteve@redhat.com>
Fixes: \url{https://github.com/oasis-tcs/virtio-spec/issues/202}
Link: \url{https://lore.kernel.org/r/20240906070833.1949727-1-aesteve@redhat.com}

 } \\
\hline
417fb28 & 03 Nov 2024 & Parav Pandit & { virtio-gpu: Fix spelling error


Fix spelling error.

Fixes: fed64230bf31 ("Add virtio gpu device specification.")
Fixes: \url{https://github.com/oasis-tcs/virtio-spec/issues/192}
Signed-off-by: Parav Pandit <parav@nvidia.com>
Reviewed-by: Matias Ezequiel Vara Larsen <mvaralar@redhat.com>

 } \\
\hline
d17c70a & 03 Nov 2024 & Parav Pandit & { Revert "virtio-gpu: Fix spelling error"


This reverts commit 417fb28849c6ecac3b18a35e8a154c03f8dba66f.
Reverted due to missing Link in commit log.

Signed-off-by: Parav Pandit <parav@nvidia.com>

 } \\
\hline
4a5df40 & 03 Nov 2024 & Parav Pandit & { virtio-gpu: Fix spelling error


Fix spelling error.

Fixes: fed64230bf31 ("Add virtio gpu device specification.")
Fixes: \url{https://github.com/oasis-tcs/virtio-spec/issues/192}
Signed-off-by: Parav Pandit <parav@nvidia.com>
Reviewed-by: Matias Ezequiel Vara Larsen <mvaralar@redhat.com>
Link: \url{https://lore.kernel.org/r/20241027044457.495730-1-parav@nvidia.com}

 } \\
\hline
a1849e3 & 25 Nov 2024 & Juraj Marcin & { virtio-mem: introduce VIRTIO_MEM_F_PERSISTENT_SUSPEND


Before, the behavior while suspending to a deep sleep state and waking
up was not specified. For example, in x86 QEMU VM all devices receive a
reset request during wake-up. This would lead to unplugging of all the
plugged memory blocks. Due to this, suspending is disallowed in the
Linux Kernel, when plugged memory is present and
VIRTIO_MEM_F_PERSISTENT_SUSPEND feature flag is not advertised by the
virtio-mem device.

This new flag should signal to the guest driver, that the device can
correctly suspend to a deep sleep state and then wake up without
disrupting the plugged memory blocks. This feature flag is already
supported in the Linux Kernel [1] and also supported in QEMU [2].

[1]: \url{https://lore.kernel.org/all/20240318120645.105664-1-david@redhat.com/}
[2]: \url{https://lore.kernel.org/all/20240904103722.946194-1-jmarcin@redhat.com/}

Reviewed-by: David Hildenbrand <david@redhat.com>
Reviewed-by: Matias Ezequiel Vara Larsen <mvaralar@redhat.com>
Signed-off-by: Juraj Marcin <jmarcin@redhat.com>
Fixes: \url{https://github.com/oasis-tcs/virtio-spec/issues/207}
Link: \url{https://lore.kernel.org/r/20241009090202.10544-2-jmarcin@redhat.com}

 } \\
\hline
c8c01f2 & 25 Nov 2024 & Parav Pandit & { device-parts: editorial: Replace duplicated part type


Device part type VIRTIO_DEV_PART_VQ_CFG was duplicated in
the description; it is supposed to be VIRTIO_DEV_PART_VQ_NOTIFY_CFG.
Fix it.

Fixes: \url{https://github.com/oasis-tcs/virtio-spec/issues/209}
Fixes: 617aa2d62a88 ("Signed-off-by: Parav Pandit <parav@nvidia.com>")
Reviewed-by: Matias Ezequiel Vara Larsen <mvaralar@redhat.com>
Signed-off-by: Parav Pandit <parav@nvidia.com>
Link: \url{https://lore.kernel.org/r/20241020114141.451478-2-parav@nvidia.com}

 } \\
\hline
f21f831 & 25 Nov 2024 & Parav Pandit & { device-parts: Add device type specific raw selector


The subsequent patch defines the device-type-specific parts. For
these parts, the raw selector format is defined to ensure that
each device type can specify its format accurately.

Fixes: \url{https://github.com/oasis-tcs/virtio-spec/issues/209}
Signed-off-by: Parav Pandit <parav@nvidia.com>
Reviewed-by: Matias Ezequiel Vara Larsen <mvaralar@redhat.com>
Link: \url{https://lore.kernel.org/r/20241020114141.451478-3-parav@nvidia.com}
Note: Merged with tab replaced with white spaces.

 } \\
\hline
b2990c8 & 25 Nov 2024 & Parav Pandit & { virtio-net: Define cvq configuration related device parts


virtio net driver sends the control virtqueue commands for
device configuration. Such driver configuration is currently
not captured in the device parts.

This series adds several of such device parts which represents
the network device specific configuration.

It is done by utilizing the existing device parts structure.
A new generic selector format is added to enable device type
specific device parts.

This series also reuses the existing control virtqueue command
structures, fields, and values to define the network device parts.

Fixes: \url{https://github.com/oasis-tcs/virtio-spec/issues/209}
Reviewed-by: Matias Ezequiel Vara Larsen <mvaralar@redhat.com>
Signed-off-by: Parav Pandit <parav@nvidia.com>
Link: \url{https://lore.kernel.org/r/20241020114141.451478-4-parav@nvidia.com}

 } \\
\hline
8e443f4 & 20 Jan 2025 & Paolo Abeni & { virtio-net: clarify NEEDS_CSUM semantic for GSO packats.


The current wording is a bit unclear hinting to possible additional
nested headers. For GSO packets virtio net (currently) supports
offload for the checksum of single transport header, explicitly state
that in both the driver and device sections.

Signed-off-by: Paolo Abeni <pabeni@redhat.com>
Reviewed-by: Jason Wang <jasowang@redhat.com>
Reviewed-by: Parav Pandit <parav@nvidia.com>
Link: \url{https://lore.kernel.org/r/81034d9f4b12f7bd7aa6f7f5266cb6d551d0823c.1732699986.git.pabeni@redhat.com}

 } \\
\hline
95c085b & 20 Jan 2025 & Paolo Abeni & { virtio-net: clarify DATA_VALID semantic for encap protos.


DATA_VALID allows offloading a single checksum level, leaving
unspecified which header checksum is offloaded when one or more
encapsulated protocols are present.

In such a case, the only option usable from the guest OS is
offloading the outermost checksum. That also matches the existing
implementation.

Explicitly state such the constraint, to remove any ambiguity and
make later changes more straightforward.

Signed-off-by: Paolo Abeni <pabeni@redhat.com>
Reviewed-by: Jason Wang <jasowang@redhat.com>
Reviewed-by: Parav Pandit <parav@nvidia.com>
Link: \url{https://lore.kernel.org/r/ec6fdbfba2c4fc1969d83799836ebb694a01fb30.1732699986.git.pabeni@redhat.com}

 } \\
\hline
8cd457d & 20 Jan 2025 & Paolo Abeni & { virtio-net: define UDP tunnel segmentation offload feature


The VIRTIO_NET_HDR_GSO_UDP_TUNNEL_IPV\{4,6\} are two gso_type flags
allowing respectively GSO over UDPv4 tunnel and GSO over UDPv6 tunnel.
They can be negotiated on both the host and guest sides.

One constraint addressed here is that the virtio side (either device or
driver) receiving a UDP tunneled GSO packet must be able to reconstruct
completely the inner and outer headers offset - to allow for later GSO.

To accommodate such need, new fields are introduced in the virtio_net
header: outer_th_offset and inner_nh_offset.
They map directly to the corresponding header information. The inner
transport header is implied by the (inner) checksum offload.

Those fields are required because a virtio device H/W implementation
performing segmentation for UDP tunneled packet will need to touch
the outer transport protocol (for the UDP length filed), the
inner network protocol (for the total length field, in the IPv4
case).

Note that segmentation will additionally need to touch
the outer network protocol and the inner transport protocol. The first
is implied/easily found with trivial parsing, the latter is identified
by the existing csum_start field.

Note that there is no concept of UDP tunnel type negotiation (e.g.
vxlan, geneve, vxlan-gpe, etc.), as a virtio device H/W implementation
can perform segmentation for every possible UDP-tunnel given the
specified new fields.
In the reverse direction, if a virtio device H/W implementation receives
some traffic for an unknown or unsupported UDP tunnel, it will simply
not aggregate the wire packet in a GSO one.

Signed-off-by: Paolo Abeni <pabeni@redhat.com>
Reviewed-by: Jason Wang <jasowang@redhat.com>
Link: \url{https://lore.kernel.org/r/3e1c45071c8cf4efc7ca0ecc7b52a73c6bf983fe.1732699986.git.pabeni@redhat.com}

 } \\
\hline
3fea589 & 20 Jan 2025 & Paolo Abeni & { virtio-net: define UDP tunnel checksum offload feature


This complements the previous change, additionally
introducing the UDP tunnel checksum offload feature.

Differently from the plain checksum offload feature, this
depends on UDP tunnel segmentation being available, as outer checksum
computation for non GSO packets is cheap and H/W implementation of
such a feature is complex.

UDP tunnel checksum offload does not introduce additional fields,
instead it leverages the outer transport offset introduced by the
UDP tunnel segmentation feature to locate the outer checksum
inside the packet.

When UDP tunnel checksum offload is negotiated:

- the driver requests the outer UDP checksum offload setting the
VIRTIO_NET_HDR_F_UDP_TUNNEL_CSUM bit in the flag field. Such bit
is not allocated inside the gso_type field to prevent space
exhaustion there.

- in the device -> driver direction, the VIRTIO_NET_HDR_F_DATA_VALID
bit semantic is extended, covering both the inner and the outer
checksum validation.

Signed-off-by: Paolo Abeni <pabeni@redhat.com>
Reviewed-by: Jason Wang <jasowang@redhat.com>
Link: \url{https://lore.kernel.org/r/e87cf0c61f5ea3783cea0bcc4ea74f6e73f453d7.1732699986.git.pabeni@redhat.com}

 } \\
\hline
14d3f88 & 03 Feb 2025 & Parav Pandit & { virtio-net: Fix to avoid using reserved feature bits


Listed patches in the fixes tag, incorrectly used the reserved feature bits.
Fix them to use the well defined device specific range.

Fixes: \url{https://github.com/oasis-tcs/virtio-spec/issues/212}
Fixes: \url{https://github.com/oasis-tcs/virtio-spec/issues/213}
Fixes: 8cd457d8aa82 ("virtio-net: define UDP tunnel segmentation offload feature")
Fixes: 3fea589bd7c6 ("virtio-net: define UDP tunnel checksum offload feature")
Signed-off-by: Parav Pandit <parav@nvidia.com>
Reviewed-by: Cornelia Huck <cohuck@redhat.com>
Acked-by: Michael S. Tsirkin <mst@redhat.com>
Link: \url{https://lore.kernel.org/r/20250126062058.13695-1-parav@nvidia.com}

 } \\
\hline
124fcd0 & 24 Feb 2025 & Steffen Trumtrar & { virtio-net: Fix receive buffer size typo


The commit 00806815385340dd411cc67df3f6837935bb5e26 introduced a slight
typo in the struct virtio_net_hdr size calculation depending on
VIRTIO_NET_F_HASH_REPORT negotiation.

Without VIRTIO_NET_F_HASH_REPORT the struct is smaller than with the
feature. This mix up only occurs in one instance; sizes are correct in
all other occurences.

Fix this typo.

Fixes: 008068153853 ("virtio-net: Fix receive buffer size calculation text")
Signed-off-by: Steffen Trumtrar <s.trumtrar@pengutronix.de>
Reviewed-by: Parav Pandit <parav@nvidia.com>
Acked-by: Michael S. Tsirkin <mst@redhat.com>
Fixes: \url{https://github.com/oasis-tcs/virtio-spec/issues/216}
Link: \url{https://lore.kernel.org/r/20250207-v1-4-topic-virtio-net-receive-buffer-fix-v1-1-efcef167d6bc@pengutronix.de}

 } \\
\hline
a9b942e & 10 Mar 2025 & Parav Pandit & { virtio-net: Rename selectors_limit to classifiers_limit


A classifier object consists of one or more selectors. The number of
selectors per classifier object is already annotated in the field
selectors_per_classifier_limit.

The field selectors_limit is intended to reflect the limit of
classifier objects. Hence, rename it to classifiers_limit.

Fixes: \url{https://github.com/oasis-tcs/virtio-spec/issues/215}
Fixes: 3b7b7371dbed ("virtio-net: Add flow filter device and driver requirements")
Fixes: 899bb0ca24d8 ("virtio-net: Add flow filter capability")
Signed-off-by: Parav Pandit <parav@nvidia.com>
Reviewed-by: Matias Ezequiel Vara Larsen <mvaralar@redhat.com>
Link: \url{https://lore.kernel.org/r/20250215061158.21053-1-parav@nvidia.com}

 } \\
\hline
4c73a70 & 10 Apr 2025 & Parav Pandit & { conformance: Add missing virtqueue reset conformance references


Add the missing references to the virtqueue reset related conformance
requirements.

(cherry picked from commit 925b42e3f72fdd113a5e4cc219b739c2c74dba23)

Signed-off-by: Parav Pandit <parav@nvidia.com>
Reviewed-by: Xuan Zhuo <xuanzhuo@linux.alibaba.com>
[CH: pushed as an editorial change]
Signed-off-by: Cornelia Huck <cohuck@redhat.com>
Signed-off-by: Matias Ezequiel Vara Larsen <mvaralar@redhat.com>
Link: \url{https://lore.kernel.org/r/20250401133543.801184-2-mvaralar@redhat.com}

 } \\
\hline
82b4313 & 10 Apr 2025 & Parav Pandit & { packed-ring: Change host,guest to device,driver


Rest of the packed ring description already uses the device
and driver terminology. Change the introductory line as well from
host and guest to device and driver respectively.

(cherry picked from commit 68d5cc4fe4081a49235885005647b095b7965c0b)

Signed-off-by: Parav Pandit <parav@nvidia.com>
Acked-by: Michael S. Tsirkin <mst@redhat.com>
[CH: pushed as an editorial update]
Signed-off-by: Cornelia Huck <cohuck@redhat.com>
Signed-off-by: Matias Ezequiel Vara Larsen <mvaralar@redhat.com>
Link: \url{https://lore.kernel.org/r/20250401133543.801184-3-mvaralar@redhat.com}

 } \\
\hline
32a7417 & 10 Apr 2025 & Parav Pandit & { description: Avoid splitting the word virtqueue


Don't split the word virtqueue.

(cherry picked from commit 4cb03b12dc951f0152cd2cd9c79b24492e174e43)

Signed-off-by: Parav Pandit <parav@nvidia.com>
[CH: applied as editorial]
Signed-off-by: Cornelia Huck <cohuck@redhat.com>
Signed-off-by: Matias Ezequiel Vara Larsen <mvaralar@redhat.com>
Link: \url{https://lore.kernel.org/r/20250401133543.801184-4-mvaralar@redhat.com}

 } \\
\hline
36e9060 & 10 Apr 2025 & Haixu Cui & { content: Rename SPI master to SPI controller


SPI master is an outdated term and should use SPI controller.

(cherry picked from commit bb11bf9b25cc86c6ff02bf9f243da55b0d383a32)

Signed-off-by: Haixu Cui <quic_haixcui@quicinc.com>
Reviewed-by: Viresh Kumar <viresh.kumar@linaro.org>
Signed-off-by: Cornelia Huck <cohuck@redhat.com>
Signed-off-by: Matias Ezequiel Vara Larsen <mvaralar@redhat.com>
Link: \url{https://lore.kernel.org/r/20250401133543.801184-5-mvaralar@redhat.com}

 } \\
\hline
8297084 & 10 Apr 2025 & Parav Pandit & { virtio-blk: Fix data type of num_queues field


Correct the endianness of num_queues field to be little endian.

(cherry picked from commit 18e47c01207b2983a7dfd1e2c7de1bdc408d391a)

Suggested-by: Max Gurtovoy <mgurtovoy@nvidia.com>
Reviewed-by: Max Gurtovoy <mgurtovoy@nvidia.com>
Signed-off-by: Parav Pandit <parav@nvidia.com>
Reviewed-by: Stefano Garzarella <sgarzare@redhat.com>
Reviewed-by: Stefan Hajnoczi <stefanha@redhat.com>
Message-Id: <20240521122119.2004071-1-parav@nvidia.com>
Signed-off-by: Michael S. Tsirkin <mst@redhat.com>
Signed-off-by: Matias Ezequiel Vara Larsen <mvaralar@redhat.com>
Link: \url{https://lore.kernel.org/r/20250401133543.801184-6-mvaralar@redhat.com}

 } \\
\hline
a22cd3e & 10 Apr 2025 & Parav Pandit & { virtio-net: Fix receive buffer size calculation text


Receive buffer size calculation is based on the following
negotiated features.

The text has wrong calculation for IPv6 and also it has missed
VIRTIO_NET_F_HASH_REPORT.

The problem of igorance of VIRTIO_NET_F_HASH_REPORT is reported
in [1], however fix for ipv6 payload length must also be
considered.

Since for the both the fixes touching same requirements, a
new issue is created as [2].

This patch brings following fixes.

1. Fix annotating struct virtio_net_hdr as field
2. Fix receive buffer calculation for guest GSO cases to consider
   ipv6 payload length
3. small grammar corrections for article
4. reword the requirement to consider the virtio_ndr_hdr which is
   depends on the negotiated feature, hence first clarify the
   struct virtio_net_hdr size

[1] \url{https://github.com/oasis-tcs/virtio-spec/issues/170}
[2] \url{https://github.com/oasis-tcs/virtio-spec/issues/183}

(cherry picked from commit 00806815385340dd411cc67df3f6837935bb5e26)

Fixes: \url{https://github.com/oasis-tcs/virtio-spec/issues/170}
Fixes: \url{https://github.com/oasis-tcs/virtio-spec/issues/183}
Reviewed-by: Xuan Zhuo <xuanzhuo@linux.alibaba.com>
Signed-off-by: Parav Pandit <parav@nvidia.com>
Message-Id: <20240606102014.2103986-2-parav@nvidia.com>
Signed-off-by: Michael S. Tsirkin <mst@redhat.com>
Signed-off-by: Matias Ezequiel Vara Larsen <mvaralar@redhat.com>
Link: \url{https://lore.kernel.org/r/20250401133543.801184-7-mvaralar@redhat.com}

 } \\
\hline
25d8101 & 10 Apr 2025 & Parav Pandit & { virtio-net: Clarify the size of the struct virtio_net_hdr for tx


The feature VIRTIO_NET_F_HASH_REPORT only applies to the receive side.
However, when VIRTIO_NET_F_HASH_REPORT feature was introduced, it was
not clarified that the size of the struct virtio_net_hdr on the packet
transmission also uses higher size when VIRTIO_NET_F_HASH_REPORT is
negotiated.

Explicitly clarify this.

(cherry picked from commit bf0fdf8ba828b694a22c44d45cb3fd34cf813e99)

Fixes: \url{https://github.com/oasis-tcs/virtio-spec/issues/183}
Reviewed-by: Xuan Zhuo <xuanzhuo@linux.alibaba.com>
Signed-off-by: Parav Pandit <parav@nvidia.com>
Message-Id: <20240606102014.2103986-3-parav@nvidia.com>
Signed-off-by: Michael S. Tsirkin <mst@redhat.com>
Signed-off-by: Matias Ezequiel Vara Larsen <mvaralar@redhat.com>
Link: \url{https://lore.kernel.org/r/20250401133543.801184-8-mvaralar@redhat.com}

 } \\
\hline
fb926ab & 10 Apr 2025 & Parav Pandit & { virtio-net: Annotate virtio_net_hdr as field


At several places struct virtio_net_hdr missed out the field
annotation. Add it.

(cherry picked from commit 6abd42fd2398718ff689dc51fa93d38ede97be8f)

Reviewed-by: Xuan Zhuo <xuanzhuo@linux.alibaba.com>
Signed-off-by: Parav Pandit <parav@nvidia.com>
Message-Id: <20240606102014.2103986-4-parav@nvidia.com>
Signed-off-by: Michael S. Tsirkin <mst@redhat.com>
Signed-off-by: Matias Ezequiel Vara Larsen <mvaralar@redhat.com>
Link: \url{https://lore.kernel.org/r/20250401133543.801184-9-mvaralar@redhat.com}

 } \\
\hline
b16f382 & 10 Apr 2025 & Martin Kröning & { virtio_pci_cap64: specify offset_hi, length_hi endianness


While the capability introduction says "This virtio structure capability uses little-endian format,"
it might be preferrable to be explicit about the endianness of offset_hi and length_hi.

(cherry picked from commit 9b3129fe72360a78e76b6dd890d3abc5a45fa915)

Signed-off-by: Martin Kröning <martin.kroening@eonerc.rwth-aachen.de>
Reviewed-by: Parav Pandit <parav@nvidia.com>
Fixes: \url{https://github.com/oasis-tcs/virtio-spec/issues/196}
Message-Id: <DF89BB0F-6BA4-4DAB-AEC3-03AAF858EC8E@eonerc.rwth-aachen.de>
Signed-off-by: Michael S. Tsirkin <mst@redhat.com>
Signed-off-by: Matias Ezequiel Vara Larsen <mvaralar@redhat.com>
Link: \url{https://lore.kernel.org/r/20250401133543.801184-10-mvaralar@redhat.com}

 } \\
\hline
376348f & 10 Apr 2025 & Parav Pandit & { gpu: editorial: Fix spelling errors


Fix spelling errors.

(cherry picked from commit ec1845bd5a0261a65e29137f949ba03bf2fb44e2)

Branch: virtio-1.4
Fixes: fed64230bf31 ("Add virtio gpu device specification.")
Fixes: \url{https://github.com/oasis-tcs/virtio-spec/issues/205}
Reviewed-by: Matias Ezequiel Vara Larsen <mvaralar@redhat.com>
Signed-off-by: Parav Pandit <parav@nvidia.com>
Link: \url{https://lore.kernel.org/r/20240916030835.68178-5-parav@nvidia.com}
Signed-off-by: Matias Ezequiel Vara Larsen <mvaralar@redhat.com>
Link: \url{https://lore.kernel.org/r/20250401133543.801184-11-mvaralar@redhat.com}

 } \\
\hline
12289b0 & 10 Apr 2025 & Paolo Abeni & { virtio-net: clarify NEEDS_CSUM semantic for GSO packats.


The current wording is a bit unclear hinting to possible additional
nested headers. For GSO packets virtio net (currently) supports
offload for the checksum of single transport header, explicitly state
that in both the driver and device sections.

(cherry picked from commit 8e443f483843c3d42a4128778f1c7548a02c48bf)

Signed-off-by: Paolo Abeni <pabeni@redhat.com>
Reviewed-by: Jason Wang <jasowang@redhat.com>
Reviewed-by: Parav Pandit <parav@nvidia.com>
Link: \url{https://lore.kernel.org/r/81034d9f4b12f7bd7aa6f7f5266cb6d551d0823c.1732699986.git.pabeni@redhat.com}
Signed-off-by: Matias Ezequiel Vara Larsen <mvaralar@redhat.com>
Link: \url{https://lore.kernel.org/r/20250401133543.801184-12-mvaralar@redhat.com}

 } \\
\hline
d3c2f14 & 10 Apr 2025 & Paolo Abeni & { virtio-net: clarify DATA_VALID semantic for encap protos.


DATA_VALID allows offloading a single checksum level, leaving
unspecified which header checksum is offloaded when one or more
encapsulated protocols are present.

In such a case, the only option usable from the guest OS is
offloading the outermost checksum. That also matches the existing
implementation.

Explicitly state such the constraint, to remove any ambiguity and
make later changes more straightforward.

(cherry picked from commit 95c085b2f16de4bcdcae9c42d7828d9b8efc7836)

Signed-off-by: Paolo Abeni <pabeni@redhat.com>
Reviewed-by: Jason Wang <jasowang@redhat.com>
Reviewed-by: Parav Pandit <parav@nvidia.com>
Link: \url{https://lore.kernel.org/r/ec6fdbfba2c4fc1969d83799836ebb694a01fb30.1732699986.git.pabeni@redhat.com}
Signed-off-by: Matias Ezequiel Vara Larsen <mvaralar@redhat.com>
Link: \url{https://lore.kernel.org/r/20250401133543.801184-13-mvaralar@redhat.com}

 } \\
\hline
16718cb & 10 Apr 2025 & Steffen Trumtrar & { virtio-net: Fix receive buffer size typo


The commit 00806815385340dd411cc67df3f6837935bb5e26 introduced a slight
typo in the struct virtio_net_hdr size calculation depending on
VIRTIO_NET_F_HASH_REPORT negotiation.

Without VIRTIO_NET_F_HASH_REPORT the struct is smaller than with the
feature. This mix up only occurs in one instance; sizes are correct in
all other occurences.

Fix this typo.

(cherry picked from commit 124fcd0e97f209aab19639e7369116d99ede22a2)

Fixes: 008068153853 ("virtio-net: Fix receive buffer size calculation text")
Signed-off-by: Steffen Trumtrar <s.trumtrar@pengutronix.de>
Reviewed-by: Parav Pandit <parav@nvidia.com>
Acked-by: Michael S. Tsirkin <mst@redhat.com>
Fixes: \url{https://github.com/oasis-tcs/virtio-spec/issues/216}
Link: \url{https://lore.kernel.org/r/20250207-v1-4-topic-virtio-net-receive-buffer-fix-v1-1-efcef167d6bc@pengutronix.de}
Signed-off-by: Matias Ezequiel Vara Larsen <mvaralar@redhat.com>
Link: \url{https://lore.kernel.org/r/20250401133543.801184-14-mvaralar@redhat.com}

 } \\
\hline
8d76f64 & 20 May 2025 & Kommula Shiva Shankar & { virtio-net: Introduce a new field to indicate outer network header offset


This patch introduces a new field in the virtio_net_hdr called outer_nh_offset, along with a new net device feature, VIRTIO_NET_F_OUT_NET_HEADER.

Currently, drivers lack a dedicated field to signal the start of the network header to the device when performing checksum offload
and segmentation offload. This requires the device to read the packet in data path, which significantly affects performance.
Additionally, some hardware implementations require knowledge of the outer L3 offset (aka L2 length) for inline IPsec hardware acceleration.

To address this limitation, we propose to introduce a new field in the virtio_net_hdr called
outer_nh_offset.

The outer_nh_offset represents the start byte offset of the outer network header from the beginning of the packet.

This issue was briefly discussed on the mailing list in a different thread, which can be found here.
\url{https://lore.kernel.org/all/DM4PR18MB4269FAAC3CFC7E57E25DFBD2DF8B2@DM4PR18MB4269.namprd18.prod.outlook.com/}

v4->v5
 - Added padding bytes to virtio_net_hdr to ensure 64b alignment
 - Addressed pending review comments
v4:\url{https://lore.kernel.org/virtio-comment/20250304075955.208450-1-kshankar@marvell.com/}

v3 -> v4
 - Removed the union of new flag with existing flags. Added as a separate field
   in the virtio net header
 - Renamed out_nh_offset to outer_nh_offset to maintain consistency with other fields
 - Spellchecks in commit message description
v3:\url{https://lore.kernel.org/all/20250217172509.107212-1-kshankar@marvell.com}

v2 -> v3:
 - Rebase to virtio-1.4
 - Addressed pending review comments related to wording.
v2:\url{https://lore.kernel.org/all/20250128142152.3662988-1-kshankar@marvell.com/}

v1 -> v2:
 - explicitly state that the out_nh_offset can be set only when a valid network header is present.
 - updated out_nh_offset usage in the RX direction.
 - minor word cleanup.
v1: \url{https://lore.kernel.org/virtio-comment/20250114171636.3175670-1-kshankar@marvell.com/}

Signed-off-by: Kommula Shiva Shankar <kshankar@marvell.com>
Reviewed-by: Parav Pandit <parav@nvidia.com>
Fixes: \url{https://github.com/oasis-tcs/virtio-spec/issues/222}
Link: \url{https://lore.kernel.org/r/20250401195655.486230-1-kshankar@marvell.com}

 } \\
\hline
fa12149 & 20 May 2025 & Aiswarya Cyriac & { Reserve device ID 49 for Virtio USB controller device


Virtio USB controller device is a dual role device, which
can function as a USB host controller or a USB device
controller or support both roles. Additionally, device provides
support for switching of roles between USB host and device.

Signed-off-by: Aiswarya Cyriac <quic_acyriac@quicinc.com>
Reviewed-by: Matias Ezequiel Vara Larsen <mvaralar@redhat.com>
Fixes: \url{https://github.com/oasis-tcs/virtio-spec/issues/211}
Link: \url{https://lore.kernel.org/r/20241129111325.952-1-quic_acyriac@quicinc.com}

 } \\
\hline
c7553a7 & 04 Jun 2025 & Srujana Challa & { virtio-crypto: Add IPsec service operation and Capabilities


This commit introduces the IPsec service operation to the Crypto
device, enabling offloading of IPsec processing.

Capabilities:

1. IPsec Resource Capability (VIRTIO_CRYPTO_IPSEC_RESOURCE_CAP):
   Indicates the device's IPsec resource limits, such as the number of
   outbound and inbound Security Associations (SAs).
2. IPsec SA Capability (VIRTIO_CRYPTO_IPSEC_SA_CAP): Specifies the
   supported IPsec modes, along with the supported cryptographic
   algorithms, authentication algorithms, IPsec options and
   anti-replay window size.

Signed-off-by: Srujana Challa <schalla@marvell.com>
Reviewed-by: Parav Pandit <parav@nvidia.com>
Fixes: \url{https://github.com/oasis-tcs/virtio-spec/issues/226}
Link: \url{https://lore.kernel.org/r/20250429131953.1949757-2-schalla@marvell.com}

 } \\
\hline
9d2cf8d & 04 Jun 2025 & Srujana Challa & { virtio-crypto: Add resource objects for IPsec outbound and inbound SAs


This commit introduces resource objects to enable the driver/device to
create IPsec Security Associations (SAs) for both inbound and outbound
directions.

The IPsec SA objects include essential parameters required for packet
outbound and inbound processing, such as SPI, tunnel headers, IPsec mode,
IPsec options and cipher/authentication specific data.

Signed-off-by: Srujana Challa <schalla@marvell.com>
Reviewed-by: Parav Pandit <parav@nvidia.com>
Fixes: \url{https://github.com/oasis-tcs/virtio-spec/issues/226}
Link: \url{https://lore.kernel.org/r/20250429131953.1949757-3-schalla@marvell.com}

 } \\
\hline
4e0baa8 & 04 Jun 2025 & Srujana Challa & { virtio-crypto: Add new IPsec opcodes to data request


Adds new IPsec opcodes, VIRTIO_CRYPTO_IPSEC_OUTBOUND and
VIRTIO_CRYPTO_IPSEC_INBOUND and defines opcode specific
data structures for IPsec data processing.

Reviewed-by: Matias Ezequiel Vara Larsen <mvaralar@redhat.com>
Reviewed-by: Parav Pandit <parav@nvidia.com>
Signed-off-by: Srujana Challa <schalla@marvell.com>
Fixes: \url{https://github.com/oasis-tcs/virtio-spec/issues/226}
Link: \url{https://lore.kernel.org/r/20250429131953.1949757-4-schalla@marvell.com}

 } \\
\hline
dac8e79 & 04 Jun 2025 & Srujana Challa & { virtio-crypto: Add device and driver requirements for IPsec operation


Add device and driver requirements for IPsec Operation.

Signed-off-by: Srujana Challa <schalla@marvell.com>
Reviewed-by: Parav Pandit <parav@nvidia.com>
Fixes: \url{https://github.com/oasis-tcs/virtio-spec/issues/226}
Link: \url{https://lore.kernel.org/r/20250429131953.1949757-5-schalla@marvell.com}

 } \\
\hline
45809a3 & 04 Jun 2025 & Srujana Challa & { virtio-net: Add IPsec operation, capabilities and resource objects


This commit introduces the IPsec Operation to the Net device
along with the capabilities and resource objects. This enables
the offloading of IPsec processing, both before transmission
and after reception, thereby providing inline offload
capabilities.

Capbilities:

1. IPsec Resource Capability (VIRTIO_CRYPTO_IPSEC_RESOURCE_CAP):
   Indicates the device's IPsec resource limits, such as the number of
   encryption and decryption Security Associations (SAs).
2. IPsec SA Capability (VIRTIO_CRYPTO_IPSEC_SA_CAP): Specifies the
   supported IPsec modes, along with the supported cryptographic
   algorithms, authentication algorithms, IPsec options and
   anti-replay window size.

Resource objects:
1. VIRTIO_NET_RESOURCE_OBJ_IPSEC_OUTB_SA
2. VIRTIO_NET_RESOURCE_OBJ_IPSEC_INB_SA

These IPsec SA resource objects encompass parameters necessary
for packet encryption and decryption. These include the SPI,
tunnel headers, IPsec mode, IPsec options, and metadata specific
to cipher and authentication.

This patch refers the Virtio-crypto IPsec service operation
capabilities and resource objects data structures and crypto algorithm
definitions to avoid duplication, however the admin command type vaule
differs between Virtio-crypto and Virtio-net.

Signed-off-by: Srujana Challa <schalla@marvell.com>
Reviewed-by: Parav Pandit <parav@nvidia.com>
Fixes: \url{https://github.com/oasis-tcs/virtio-spec/issues/227}
Link: \url{https://lore.kernel.org/r/20250520121924.2169258-2-schalla@marvell.com}

 } \\
\hline
a9604fa & 04 Jun 2025 & Srujana Challa & { virtio-net: Add new flow filter selector and action for IPsec


This update introduces a new flow filter selector to match
the ESP header and adds a new flow filter action for IPsec
processing.

Signed-off-by: Srujana Challa <schalla@marvell.com>
Reviewed-by: Parav Pandit <parav@nvidia.com>
Fixes: \url{https://github.com/oasis-tcs/virtio-spec/issues/227}
Link: \url{https://lore.kernel.org/r/20250520121924.2169258-3-schalla@marvell.com}

 } \\
\hline
fd15f89 & 04 Jun 2025 & Srujana Challa & { virtio-net: extend virtio_net_hdr for IPsec support


Add IPsec resource object identifiers to the virtio_net_hdr for
identifying encryption/decryption operations on tx and rx side
respectively, along with flags.

Signed-off-by: Srujana Challa <schalla@marvell.com>
Reviewed-by: Parav Pandit <parav@nvidia.com>
Fixes: \url{https://github.com/oasis-tcs/virtio-spec/issues/227}
Link: \url{https://lore.kernel.org/r/20250520121924.2169258-4-schalla@marvell.com}

 } \\
\hline
a0b809a & 04 Jun 2025 & Srujana Challa & { virtio-net: Add IPsec operation device and driver requirements


Add device and driver requirements for IPsec Operation.

Signed-off-by: Srujana Challa <schalla@marvell.com>
Reviewed-by: Parav Pandit <parav@nvidia.com>
Fixes: \url{https://github.com/oasis-tcs/virtio-spec/issues/227}
Link: \url{https://lore.kernel.org/r/20250520121924.2169258-5-schalla@marvell.com}

 } \\
\hline
c5e5810 & 09 Jul 2025 & Albert Esteve & { virtio-media: Add virtio media device specification


Virtio-media is an encapsulation of the V4L2 UAPI into
virtio, able to virtualize any video device supported
by V4L2.

Note that virtio-media does not require the use of a
V4L2 device driver on the host or guest side -
V4L2 is only used as a host-guest protocol,
and both sides are free to convert it from/to any
model that they wish to use.

Reviewed-by: Matias Ezequiel Vara Larsen <mvaralar@redhat.com>
Reviewed-by: Alexandre Courbot <gnurou@gmail.com>
Signed-off-by: Albert Esteve <aesteve@redhat.com>

 } \\
\hline
340d076 & 15 Jul 2025 & Zhu Lingshan & { virtio: re-order device status bits


This commit re-arranges the device status bits,
to list them in ascending order.

Signed-off-by: Zhu Lingshan <lingshan.zhu@amd.com>
Reviewed-by: Parav Pandit <parav@nvidia.com>
Fixes: \url{https://github.com/oasis-tcs/virtio-spec/issues/229}
Reviewed-by: Jason Wang <jasowang@redhat.com>
Link: \url{https://lore.kernel.org/r/20250704101739.354522-2-lingshan.zhu@amd.com}

 } \\
\hline
b1d6ef6 & 15 Jul 2025 & Zhu Lingshan & { virtio: document feature bit 42


This commit documents feture bit 42
VIRTIO_NET_F_GUEST_RSC6

Signed-off-by: Zhu Lingshan <lingshan.zhu@amd.com>
Reviewed-by: Parav Pandit <parav@nvidia.com>
Fixes: 94384142 ("content: Declare virtio-net legacy feature bits 41-42")
Fixes: \url{https://github.com/oasis-tcs/virtio-spec/issues/229}
Reviewed-by: Jason Wang <jasowang@redhat.com>
Link: \url{https://lore.kernel.org/r/20250704101739.354522-3-lingshan.zhu@amd.com}

 } \\
\hline
f219fef & 15 Jul 2025 & Zhu Lingshan & { virtio: introduce SUSPEND and RESUME feature


This commit allows the driver to suspend the
device through a new device status bit SUSPEND
and resume the device running by re-setting
DRIVER_OK bit in device status.

Signed-off-by: Zhu Lingshan <lingshan.zhu@amd.com>
Signed-off-by: Jason Wang <jasowang@redhat.com>
Reviewed-by: Parav Pandit <parav@nvidia.com>
Fixes: \url{https://github.com/oasis-tcs/virtio-spec/issues/229}
Link: \url{https://lore.kernel.org/r/20250704101739.354522-4-lingshan.zhu@amd.com}

 } \\
\hline
6275e11 & 15 Jul 2025 & Parav Pandit & { virtio-net: Fix ipsec broken conformance links


Fix the broken conformance links for ipsec device and driver
requirements.

Fixes: a0b809a7ddbd ("virtio-net: Add IPsec operation device and driver requirements")
Signed-off-by: Parav Pandit <parav@nvidia.com>
Reviewed-by: Matias Ezequiel Vara Larsen <mvaralar@redhat.com>
Link: \url{https://lore.kernel.org/r/20250709140049.507870-1-parav@nvidia.com}

 } \\
\hline
65645f2 & 22 Sep 2025 & Peter Hilber & { virtio-rtc: Add initial device specification


The virtio-rtc device provides information about current time through
one or more clocks. As such, it is a Real-Time Clock (RTC) device.

The normative statements for this device follow in the next patch.

For this device, there is a Linux kernel driver patch series which is
being upstreamed, and a proprietary device implementation.

Miscellaneous

-------------

The spec does not specify how a driver should interpret clock readings,
esp. also not how to perform clock synchronization.

The device uses the former "Timer/Clock" device id which is already part
of the specification. This device id was registered a long time ago and
should be unused according to the author's information. The name "RTC"
was determined to be the best for a device which focuses on current
time.

Signed-off-by: Peter Hilber <quic_philber@quicinc.com>
Link: \url{https://lore.kernel.org/r/20250710090648.1711-3-quic_philber@quicinc.com}

 } \\
\hline
f111987 & 22 Sep 2025 & Peter Hilber & { virtio-rtc: Add initial normative statements


Add the normative statements for the initial device specification.

Signed-off-by: Peter Hilber <quic_philber@quicinc.com>
Link: \url{https://lore.kernel.org/r/20250710090648.1711-4-quic_philber@quicinc.com}

 } \\
\hline
0a6a441 & 22 Sep 2025 & Peter Hilber & { virtio-rtc: Add alarm feature


Add the VIRTIO_RTC_F_ALARM feature (without normative statements).

The intended use case is: A driver needs to react when an alarm time has
been reached, but at alarm time, the driver may be in a sleep state or
powered off. The alarm feature can resume and notify the driver in this
case. Alarms may be retained across device resets.

Peculiarities

-------------

Unlike usual alarm clocks, a virtio-rtc alarm-capable clock may step
autonomously at any time: An alarm may change back from "expired" to
"not expired" before the driver has started processing an alarm
notification.

To address the above, and the device resets, define "alarm expiration"
in such a way that the driver always has a chance to react to an alarm,
and make the device always responsible for notifying the driver about an
alarm expiration.

The VIRTIO_RTC_REQ_SET_ALARM_ENABLED request is there so that the Linux
ioctls RTC_AIE_ON and RTC_AIE_OFF only need to emit one request.

Signed-off-by: Peter Hilber <quic_philber@quicinc.com>
Link: \url{https://lore.kernel.org/r/20250710090648.1711-5-quic_philber@quicinc.com}

 } \\
\hline
f978e07 & 22 Sep 2025 & Peter Hilber & { virtio-rtc: Add normative statements for alarm feature


Add the normative statements for the alarm feature added previously.

Signed-off-by: Peter Hilber <quic_philber@quicinc.com>
Link: \url{https://lore.kernel.org/r/20250710090648.1711-6-quic_philber@quicinc.com}

 } \\
\hline
928d969 & 23 Sep 2025 & Matias Ezequiel Vara Larsen & { git-publish: add profile


Add git-publish profile and document how to use it.

Signed-off-by: Matias Ezequiel Vara Larsen <mvaralar@redhat.com>
Reviewed-by: Stefano Garzarella <sgarzare@redhat.com>
Reviewed-by: Albert Esteve <aesteve@redhat.com>
Acked-by: Cornelia Huck <cohuck@redhat.com>
Link: \url{https://lore.kernel.org/r/20250305164546.1484029-1-mvaralar@redhat.com}

 } \\
\hline
d9742cb & 26 Sep 2025 & Matias Ezequiel Vara Larsen & { README.md: add example email for TC vote request


Signed-off-by: Matias Ezequiel Vara Larsen <mvaralar@redhat.com>
Link: \url{https://lore.kernel.org/r/20250320114326.2075821-1-mvaralar@redhat.com}

 } \\
\hline
63aaa4b & 13 Oct 2025 & Parav Pandit & { Merge branch 'virtio-1.4'


Resolved conflict in net device description.

Signed-off-by: Parav Pandit <parav@nvidia.com>

 } \\
\hline
9f903b2 & 28 Oct 2025 & Parav Pandit & { edit: add changelog for 1.4


Add the changelog for 1.4.

Signed-off-by: Parav Pandit <parav@nvidia.com>
Message-Id: <20251027174756.56284-3-parav@nvidia.com>

 } \\
\hline
b4899d2 & 28 Oct 2025 & Parav Pandit & { edit: Remove annotation of special character


PDF generation parsing complains when special character
annotation is listed in it. Remove such annotation in changelog.

Signed-off-by: Parav Pandit <parav@nvidia.com>
Message-Id: <20251027174756.56284-4-parav@nvidia.com>

 } \\
\hline
5e4a575 & 28 Oct 2025 & Michael S. Tsirkin & { drop generated files


Generated files:
    virtio-v1.3-csd01.aux
    virtio-v1.3-csd01.log
    virtio-v1.3-csd01.out
    virtio-v1.3-csd01.pdf
    virtio-v1.3-csd01.toc

have no business being in tree. Drop them.

Fixes: 63aaa4b ("Merge branch 'virtio-1.4'")
Cc: "Parav Pandit" <parav@nvidia.com>
Signed-off-by: Michael S. Tsirkin <mst@redhat.com>

 } \\
\hline
21e81fa & 06 Nov 2025 & Michael S. Tsirkin & { virtio-html: add missing makeatother


A workaround for F21 added \textbackslash makeatletter but not \textbackslash makeatother

I see no specific issues around this but theoretically can interfere
with some packages.

Fix this up.

Fixes: 86e51b4 ("html: work around bug in html generation")
Signed-off-by: Michael S. Tsirkin <mst@redhat.com>

 } \\
\hline
563b6c8 & 07 Nov 2025 & Michael S. Tsirkin & { editorial: suppress noitem errors in diff


latexdiff tends to leave empty itemize/enumerate lists around
instead of deleting lists.

For example:

\textbackslash begin\{itemize\}\%DIFAUXCMD
\%DIFDELCMD <   \textbackslash item rx_usecs: Maximum number of usecs to delay a RX notification.
\%DIFDELCMD <
\%DIFDELCMD <   \textbackslash item rx_max_packets: Maximum number of packets to receive before a RX notification.
\textbackslash end\{itemize\}\%DIFAUXCMD

this makes latex stop with an error:
    Something's wrong--perhaps a missing \textbackslash item
to fix, suppress the error.

Signed-off-by: Michael S. Tsirkin <mst@redhat.com>

 } \\
\hline
f9abfd5 & 07 Nov 2025 & Sergio Lopez & { virtio-gpu: support blob alignment information


There's an increasing number of machines supporting multiple page sizes
and, on these machines, the host and a guest can be running with
different pages sizes. In addition to this, there might be physical
devices that require to operate their memory at a specific granularity.

In these cases, if they are to use Shared Memory Regions, the device
and the driver must operate with the same granularity, as otherwise
the former might not be able to fulfill the requests sent by the
latter.

For the GPU device, this has an impact on blob creation and mapping. To
address the problem, allow the device to require certain alignment
constrains for blob resources by extending the device configuration
with the field "blob_alignment" and introducing the
VIRTIO_GPU_F_BLOB_ALIGNMENT feature.

Signed-off-by: Sergio Lopez <slp@redhat.com>

 } \\
\hline
b2741ed & 07 Nov 2025 & Parav Pandit & { virtio-gpu: Fix missing driver conformance file


Cited patch in the fixes tag missed to include the driver-conformance
file. Add it.

Fixes: f52fb20cbe38 ("virtio-gpu: support blob alignment information")
Signed-off-by: Parav Pandit <parav@nvidia.com>
Reviewed-by: Matias Ezequiel Vara Larsen <mvaralar@redhat.com>
Link: \url{https://lore.kernel.org/r/20250614181302.277794-1-parav@nvidia.com}

 } \\
\hline
8494bb9 & 07 Nov 2025 & Parav Pandit & { virtio-crypto: editorial: fix broken label link


Fix broken link to resource objects.

Fixes: c7553a71b7eb ("virtio-crypto: Add IPsec service operation and Capabilities")
Reported-by: Michael S. Tsirkin <mst@redhat.com>
Signed-off-by: Parav Pandit <parav@nvidia.com>

 } \\
\hline
e8557d8 & 07 Nov 2025 & Parav Pandit & { changelog: Add changelog for 1.2 and 1.3


Add the changelog from version 1.2 to v1.3 for bookkeeping purpose.

Signed-off-by: Parav Pandit <parav@nvidia.com>

 } \\
\hline
521d5cc & 07 Nov 2025 & Parav Pandit & { gitlog: Add executable permision to script


gitlog.pl is the script to generate the changelog.
Make it executable.

Signed-off-by: Parav Pandit <parav@nvidia.com>

 } \\
\hline
4e2a6b9 & 07 Nov 2025 & Parav Pandit & { acknowledgements: update for 1.4


Move some names to the section for previous versions, add names of new
contributors, etc.

Signed-off-by: Parav Pandit <parav@nvidia.com>

 } \\
\hline
0bad6c8 & 07 Nov 2025 & Parav Pandit & { changelog: Update changelog for additonal 1.4 patches


Update the change log further for missing ipsec, gpu blob
and other editorial patches.

Signed-off-by: Parav Pandit <parav@nvidia.com>

 } \\
\hline
14b4403 & 10 Nov 2025 & Parav Pandit & { virtio-net: editorial: fix broken label link


Fix broken link to crypto outbound SA resource objects.

Fixes: fd15f89a870f ("virtio-net: extend virtio_net_hdr for IPsec support")
Reported-by: Michael S. Tsirkin <mst@redhat.com>
Signed-off-by: Parav Pandit <parav@nvidia.com>

 } \\
\hline
9e065a9 & 11 Nov 2025 & Parav Pandit & { editorial: Add a new chair and a new editor


Add Matias as the chair an Parav as the enditor.

Signed-off-by: Parav Pandit <parav@nvidia.com>

 } \\
\hline
0d71ab8 & 11 Nov 2025 & Parav Pandit & { editorial: Update the new oasis open link


OASIS updated the link to the virtio tc community lately.
Update the link to it.

Signed-off-by: Parav Pandit <parav@nvidia.com>

 } \\
\hline
7672c71 & 13 Nov 2025 & Parav Pandit & { REVISION: update to 1.4


Update the revision to 1.4 working draft and date.
Since 1.3 was never released, keep the last version of 1.2 for diff generation.

Signed-off-by: Parav Pandit <parav@nvidia.com>

 } \\
\hline
f6c548f & 16 Nov 2025 & Parav Pandit & { editorial: Remove hyperlink with hash tag


The new hyperlink contains the hash letter.
The new replaced link breaks the diff generation.

This is because diff operation generates DIFadd, DIFdel
commands inside the href.

This causes deep recursive macro expansion resulting into
below error.
Avoid this small change in the diff generation.

Remove the other patches which already exists now in the
master branch.

! TeX capacity exceeded, sorry [input stack size=10000].

\textbackslash @setfontsize \#1\#2\#3->\textbackslash @nomath \#1 \textbackslash ifx \textbackslash protect

\textbackslash @typeset@protect \textbackslash let \textbackslash @curr... l.577 ...delines/tc-process\#wpComponentsCompLang\}.


Signed-off-by: Parav Pandit <parav@nvidia.com>

 } \\
\hline
92a4614 & 16 Nov 2025 & Parav Pandit & { editorial: Avoid logtable keyword as diff


Commit X started using longtable.
Diff generation is sneaking logtable between DIFaddbegin
and DIFaddend. This makes LaTeX unhappy breaking the
diff generation.

Using SAFEENV or PICTUREENV or appending texcmd or appending
text cmd is not helpful either to resolve to generate the
desired result. It deletes the whole table and adds as
new table showing big delta, which is undesired.

Until diff infrastructure improved, workaround to remove those
weird diffs with simpler perl script.

Signed-off-by: Parav Pandit <parav@nvidia.com>

 } \\
\hline
267ced1 & 16 Nov 2025 & Parav Pandit & { editorial: Update diff version to 1.2


Update the previous version to 1.2 so that 1.4 spec
diff can be with previously release version 1.2.

Signed-off-by: Parav Pandit <parav@nvidia.com>

 } \\
\hline
fc3c135 & 16 Nov 2025 & Parav Pandit & { editorial: Advance the date with fixes for diff generation


Advance the date for editorial fixes addition for
diff generation.

Signed-off-by: Parav Pandit <parav@nvidia.com>

 } \\
\hline
7cdd452 & 18 Nov 2025 & Parav Pandit & { editorial: Update revision to csd01


Update the revision to v1.4-csd01 and its new date.

Signed-off-by: Parav Pandit <parav@nvidia.com>

 } \\
\hline
75e8068 & 18 Nov 2025 & Parav Pandit & { editorial: Prepare public review draft


Prepare public review draft version 1.4-csprd01.

Signed-off-by: Parav Pandit <parav@nvidia.com>

 } \\
\hline
26c96dd & 19 Nov 2025 & Parav Pandit & { editorial: Restore back the previous stage links


Restore back the link to the previous version 1.2.

Signed-off-by: Parav Pandit <parav@nvidia.com>

 } \\
\hline
defdc1b & 19 Nov 2025 & Parav Pandit & { Revert "editorial: Prepare public review draft"


This reverts commit 75e806852781ae98035df6f97c7af748eff1e9c1.
Few issues were found in the draft.
Hence, revert to restart the process.

Signed-off-by: Parav Pandit <parav@nvidia.com>

 } \\
\hline
5833db4 & 19 Nov 2025 & Parav Pandit & { Revert "editorial: Update revision to csd01"


This reverts commit 7cdd4521c657e893489b7722fb693d936743cb36.
Few issues were found in the draft.
Hence, revert to restart the process.

Signed-off-by: Parav Pandit <parav@nvidia.com>

 } \\
\hline
2132a93 & 03 Dec 2025 & Michael S. Tsirkin & { net: pad virtio_net_ff_cap_data


struct virtio_net_ff_cap_data has 4 byte fields but the size is not a
multiple of 4.  drivers can easily get it wrong since compilers tend to
add padding to align such structures.

Since we are always allowed to pad or truncate admin commands, let's do
just that here.

Fixes: 899bb0ca24d8 ("virtio-net: Add flow filter capability")
Fixes: \url{https://github.com/oasis-tcs/virtio-spec/issues/236}
Cc: Parav Pandit <parav@nvidia.com>
Signed-off-by: Michael S. Tsirkin <mst@redhat.com>
Reviewed-by: Parav Pandit <parav@nvidia.com>

 } \\
\hline

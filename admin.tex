\section{Device groups}\label{sec:Basic Facilities of a Virtio Device / Device groups}

It is occasionally useful to have a device control a group of
other devices. Terminology used in such cases:

\begin{description}
\item[Device group]
        or just group, includes zero or more devices.
\item[Owner device]
        or owner, the device controlling the group.
\item[Member device]
        a device within a group. The owner device itself is not
	a member of the group.
\item[Member identifier]
        each member has this identifier, unique within the group
	and used to address it through the owner device.
\item[Group type identifier]
	specifies what kind of member devices there are in a
	group, how the member identifier is interpreted
	and what kind of control the owner has.
	A given owner can control multiple groups
	of different types but only a single group of a given type,
	thus the type and the owner together identify the group.
	\footnote{Even though some group types only support
			specific transports, group type identifiers
			are global rather than transport-specific -
			a flood of new group types is not expected.}
\end{description}

\begin{note}
Each device only has a single driver, thus for the purposes of
this section, "the driver" is usually unambiguous and refers to
the driver of the owner device.  When there's ambiguity, "owner
driver" refers to the driver of the owner device, while "member
driver" refers to the driver of a member device.
\end{note}

The following group types, and their identifiers, are currently specified:
\begin{description}
\item[SR-IOV group type (0x1)]
This device group has a PCI Single Root I/O Virtualization
(SR-IOV) physical function (PF) device as the owner and includes
all its SR-IOV virtual functions (VFs) as members (see
\hyperref[intro:PCIe]{[PCIe]}).

The PF device itself is not a member of the group.

The group type identifier for this group is 0x1.

A member identifier for this group can have a value from 0x1 to
\field{NumVFs} as specified in the
SR-IOV Extended Capability of the owner device
and equals the SR-IOV VF number of the member device;
the group only exists when the \field{VF Enable} bit
in the SR-IOV Control Register within the
SR-IOV Extended Capability of the owner device is set
(see \hyperref[intro:PCIe]{[PCIe]}).

Both owner and member devices for this group type use the Virtio
PCI transport (see \ref{sec:Virtio Transport Options / Virtio Over PCI Bus}).
\end{description}

\subsection{Group administration commands}\label{sec:Basic Facilities of a Virtio Device / Device groups / Group administration commands}

The driver sends group administration commands to the owner device of
a group to control member devices of the group.
This mechanism can
be used, for example, to configure a member device before it is
initialized by its driver.
\footnote{The term "administration" is intended to be interpreted
widely to include any kind of control. See specific commands
for detail.}

All the group administration commands are of the following form:

\begin{lstlisting}
struct virtio_admin_cmd {
        /* Device-readable part */
        le16 opcode;
        /*
         * 1       - SR-IOV
         * 2-65535 - reserved
         */
        le16 group_type;
        /* unused, reserved for future extensions */
        u8 reserved1[12];
        le64 group_member_id;
        le64 command_specific_data[];

        /* Device-writable part */
        le16 status;
        le16 status_qualifier;
        /* unused, reserved for future extensions */
        u8 reserved2[4];
        u8 command_specific_result[];
};
\end{lstlisting}

For all commands, \field{opcode}, \field{group_type} and if
necessary \field{group_member_id} and \field{command_specific_data} are
set by the driver, and the owner device sets \field{status} and if
needed \field{status_qualifier} and
\field{command_specific_result}.

Generally, any unused device-readable fields are set to zero by the driver
and ignored by the device.  Any unused device-writeable fields are set to zero
by the device and ignored by the driver.

\field{opcode} specifies the command. The valid
values for \field{opcode} can be found in the following table:

\begin{tabular}{|l|l|}
\hline
opcode & Name & Command Description \\
\hline \hline
0x0000 - 0x7FFF & - & Commands using \field{struct virtio_admin_cmd}    \\
\hline
0x8000 - 0xFFFF & - & Reserved for future commands (possibly using a different structure)    \\
\hline
\end{tabular}

The \field{group_type} specifies the group type identifier.
The \field{group_member_id} specifies the member identifier within the group.
See section \ref{sec:Introduction / Terminology / Device group}
for the definition of the group type identifier and group member
identifier.

The \field{status} describes the command result and possibly
failure reason at an abstract level, this is appropriate for
forwarding to applications. The \field{status_qualifier} describes
failures at a low virtio specific level, as appropriate for debugging.
The following table describes possible \field{status} values;
to simplify common implementations, they are intentionally
matching common \hyperref[intro:errno]{Linux error names and numbers}:

\begin{tabular}{|l|l|l|}
\hline
Status (decimal) & Name & Description \\
\hline \hline
00   & VIRTIO_ADMIN_STATUS_OK    & successful completion  \\
\hline
11   & VIRTIO_ADMIN_STATUS_EAGAIN    & try again \\
\hline
12   & VIRTIO_ADMIN_STATUS_ENOMEM    & insufficient resources \\
\hline
22   & VIRTIO_ADMIN_STATUS_EINVAL    & invalid command \\
\hline
other   & -    & group administration command error  \\
\hline
\end{tabular}

When \field{status} is VIRTIO_ADMIN_STATUS_OK, \field{status_qualifier}
is reserved and set to zero by the device.

The following table describes possible \field{status_qualifier} values:
\begin{tabular}{|l|l|l|}
\hline
Status & Name & Description \\
\hline \hline
0x00   & VIRTIO_ADMIN_STATUS_Q_OK               & used with VIRTIO_ADMIN_STATUS_OK  \\
\hline
0x01   & VIRTIO_ADMIN_STATUS_Q_INVALID_COMMAND  & command error: no additional information  \\
\hline
0x02   & VIRTIO_ADMIN_STATUS_Q_INVALID_OPCODE   & unsupported or invalid \field{opcode}  \\
\hline
0x03   & VIRTIO_ADMIN_STATUS_Q_INVALID_FIELD    & unsupported or invalid field within \field{command_specific_data}  \\
\hline
0x04   & VIRTIO_ADMIN_STATUS_Q_INVALID_GROUP    & unsupported or invalid \field{group_type} \\
\hline
0x05   & VIRTIO_ADMIN_STATUS_Q_INVALID_MEMBER   & unsupported or invalid \field{group_member_id} \\
\hline
0x06   & VIRTIO_ADMIN_STATUS_Q_NORESOURCE       & out of internal resources: ok to retry \\
\hline
0x07   & VIRTIO_ADMIN_STATUS_Q_TRYAGAIN         & command blocks for too long: should retry \\
\hline
0x08-0xFFFF   & -    & reserved for future use \\
\hline
\end{tabular}

Each command uses a different \field{command_specific_data} and
\field{command_specific_result} structures and the length of
\field{command_specific_data} and \field{command_specific_result}
depends on these structures and is described separately or is
implicit in the structure description.
